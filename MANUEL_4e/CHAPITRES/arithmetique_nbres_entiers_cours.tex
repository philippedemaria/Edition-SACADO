\chapter{Arithmétique}
{https://sacado.xyz/qcm/parcours_show_course/0/117129}
{
 \begin{CpsCol}
\textbf{Les savoir-faire du parcours}
 \begin{itemize}
 \item Utiliser et écrire en ligne la division euclidienne de a par b, deux nombres entiers.
 \item Utiliser la définition de multiple ou un diviseur
 \item Déterminer des multiples ou des diviseurs d’un nombre donné
 \item Déterminer si un entier est un multiple ou un diviseur d’un autre entier
 \item Utiliser les critères de divisibilité par 2, 3, 5, 9, 10
 \item Modéliser et résoudre des problèmes mettant en jeu la divisibilité
 \item Savoir tester si un nombre est premier ou ne l'est pas en utilisant la calculatrice ou un logiciel
 \item Donner la liste des nombres premiers inférieurs à un entier donné (crible d'Eratosthène)
 \item Décomposer un nombre entier en produit de facteurs premiers (à la main ou à l'aide d'un logiciel)
 \item Simplifier une fraction pour la rendre irréductible
 \end{itemize}
 
 \end{CpsCol}
 
 \begin{His}
 zzzz
 \end{His}

\begin{ExoDec}{Chercher.}{1234}{1}{0}{0}{0}

 On souhaite ranger 142 bonbons dans des boites de 12 bonbons. Les boites doivent être complétées entièrement avant
d'utiliser d'une nouvelle boite.
\begin{enumerate}
\item Combien de bonbons ne sont-ils pas rangés dans une boite complète?
\item Combien de boites sont-elles entièrement remplies?
\item Quel est le nombre minimal de boites nécessaires pour ranger tous les 142 bonbons? Combien de bonbons faudrait-il
rajouter pour remplir la boite incomplète?
\end{enumerate}

\end{ExoDec}
 
 
\begin{ExoDec}{Chercher.}{1234}{1}{0}{0}{0}

Trois bateaux partent de Marseille, l'un tous les 7 jours, le second tous les 12 jours, le troisième tous les 14 jours. Ils partent
tous les trois le 1er mars. À quelle prochaine date partiront-ils encore tous les trois ensemble du port de Marseille ?



\end{ExoDec} 
 
 
}


\begin{pageCours}




\section{La division Euclidienne}



\begin{DefT}{Division euclidienne}

Écrire la division euclidienne\index{Division euclidienne} d'un nombre entier naturel $a$ par un entier naturel $b$, tous deux non nuls, c'est déterminer les nombres entiers $q$ et $r$ tels que $a = b \times q + r$ avec $0 \leq r <  b$.

$q$ est appelé le quotient de la division euclidienne de $a$ par $b$.

$r$ est appelé le reste de la division euclidienne de $a$ par $b$.
\end{DefT}


\begin{ExQr}{1234}
L'égalité de la division euclidienne de $254$ par $7$ s'écrit : $254 = 36 \times 7 +2$. $36$ est le quotient et $2$ est le reste.
\end{ExQr}


\begin{Att}

$22 = 3 \times 5 +7$ mais cette égalité n'est pas l'écriture de la division euclidienne de $22$ par $5$ ou par $3$. En effet, $7 > 3$ et $7 > 5$.

L'égalité de la division euclidienne de $22$ par $3$ s’écrit : $22 = 3 \times 7 +1$

L'égalité de la division euclidienne de $22$ par $5$ s’écrit : $22 = 4 \times 5 +2$.
\end{Att}

\section{Multiples et diviseurs}

 
  

\begin{DefT}{Multiples et diviseurs}\index{Multiple}\index{Diviseur}
 Soit $a$ et $b$ deux nombres entiers positifs. 
 
 On dit que $a$ est un \textbf{multiple} de $b$ s'il existe un nombre entier $k$ tel que $a = k \times b$.
 
 On dit alors que $b$ est un \textbf{diviseur} de $a$. 
\end{DefT}

\begin{Rq}

Le reste de la division euclidienne de $a$ par $b$ est alors égal à $0$. 
\end{Rq}


\begin{minipage}{0.6\linewidth}

\begin{ExQr}{1234}
Soit $a=56$ et $b=7$, on a $56 = 7 \times 8+0 = 7 \times 8$. 

Donc $56$ est un multiple de $7$.

\end{ExQr}
\end{minipage}
\begin{minipage}{0.4\linewidth}
\begin{OuQr}{1234}

 Calculateur de multiples et de diviseurs
\end{OuQr}
\end{minipage}
 
 
 
 
 
\section{Critère de divisibilité}



\begin{minipage}{0.6\linewidth}

\begin{ThT}{Critères de divisibilité}\index{Critères de divisibilité}
 
Un nombre entier est divisible  :
 
\begin{itemize}
 \item par 2, si son chiffre des unités est pair ;
 \item par 3, si la somme de ses chiffres est un multiple de $3$ ;
 \item par 5, si  son chiffre des unités est $0$ ou $5$ ;
 \item par 9, si la somme de ses chiffres est un multiple de $9$ ;
 \item par 10, si  son chiffre des unités est $0$.
\end{itemize}
 
\end{ThT}
\end{minipage}
\begin{minipage}{0.4\linewidth}
\begin{MeQr}{1234}

Utiliser les critères de divisibilité
\end{MeQr}
\end{minipage}






\end{pageCours} 
\begin{pageAD} 
 

\Sf{Utiliser et écrire en ligne la division euclidienne de $a$ par $b$, deux nombres entiers.}
 
  
\begin{ExoCad}{Communiquer.}{1234}{2}{0}{0}{0}{0}

\begin{enumerate}

\item Effectue les divisions euclidiennes suivantes :


\begin{minipage}{0.3\linewidth}
\begin{equation*}
\renewcommand{\arraystretch}{1.2}
\renewcommand{\arraycolsep}{2pt}
  \begin{array}{rrrr|rrr}
 & 2  & 9 & 7 & & 9 \\
\cline{5-7}
 & &  &  &  & \\
    & &  &  &   &   &  \\
    &  &  &  &   &   &  \\
    &&  &  &   &   &  \\
  \end{array}
\end{equation*}
\end{minipage}
\begin{minipage}{0.3\linewidth}
\begin{equation*}
\renewcommand{\arraystretch}{1.2}
\renewcommand{\arraycolsep}{2pt}
  \begin{array}{rrrr|rrr}
 & 3  & 5 & 7 &  1 & 2 \\
\cline{5-7}
 & &  &  &  & \\
    & &  &  &   &   &  \\
    &  &  &  &   &   &  \\
    &&  &  &   &   &  \\
  \end{array}
\end{equation*}

\end{minipage}
\begin{minipage}{0.3\linewidth}
\begin{equation*}
\renewcommand{\arraystretch}{1.2}
\renewcommand{\arraycolsep}{2pt}
  \begin{array}{rrrr|rrr}
1 & 6  & 8 & 3 & 9 & 5 \\
\cline{5-7}
 & &  &  &  & \\
    & &  &  &   &   &  \\
    &  &  &  &   &   &  \\
    &&  &  &   &   &  \\
  \end{array}
\end{equation*}
\end{minipage}


\item Traduis chacune de ces divisions par une égalité. 

\begin{enumerate}
\item $279 = $ \point{1}
\item $357 = $ \point{1}
\item $\np{1683} = $ \point{1}
\end{enumerate}
\end{enumerate}
\end{ExoCad}



\begin{ExoCad}{Calculer.}{1234}{0}{0}{0}{0}{0}

 Dans la division de $85$ par $6$, le reste est égal à $1$ et le quotient est égal à $14$. Écrire ce résultat par une égalité.
 
 \point{1}
 
\end{ExoCad}

\begin{ExoCad}{Calculer.}{1234}{0}{0}{0}{0}{0}

 Cédric attend le bus qui peut contenir 53 personnes. Il passe un bus toutes les 17 minutes. Il y a 164 personnes devant
Cédric. Dans combien de temps Cédric pourra monter dans un bus sachant qu'il vient d'en voir un partir ?\point{5}
\end{ExoCad}


\Sf{Utiliser la distributivité}

\begin{ExoCad}{Représenter. Calculer.}{1234}{0}{0}{0}{0}{0}

 
 
\end{ExoCad}

\begin{ExoCad}{Calculer.}{1234}{0}{0}{0}{0}{0}

 
\end{ExoCad}

 
\end{pageAD}


%%%%%%%%%%%%%%%%%%%%%%%%%%%%%%%%%%%%%%%%%%%%%%%%%%%%%%%%%%%%%%%%%%%
%%%%  Niveau 1
%%%%%%%%%%%%%%%%%%%%%%%%%%%%%%%%%%%%%%%%%%%%%%%%%%%%%%%%%%%%%%%%%%%
\begin{pageParcoursu} 

 %%%%%%%%%%%%%%%%%%%%%%%%%%%
\begin{ExoCu}{Représenter.}{1234}{2}{0}{0}{0}{0}


\end{ExoCu}
%%%%%%%%%%%%%%%%%%%%%%%%%%%
\begin{ExoCu}{Représenter.}{1234}{2}{0}{0}{0}{0}


\end{ExoCu}
%%%%%%%%%%%%%%%%%%%%%%%%%%%
\begin{ExoCu}{Représenter.}{1234}{2}{0}{0}{0}{0}

\end{ExoCu}


%%%%%%%%%%%%%%%%%%%%%%%%%%%
\begin{ExoCu}{Raisonner.}{1234}{2}{0}{0}{0}{0}

\end{ExoCu}

%%%%%%%%%%%%%%%%%%%%%%%%%%%
\begin{ExoCu}{Représenter.}{1234}{2}{0}{0}{0}{0}
  
 

\begin{enumerate}

\item Effectue les divisions euclidiennes suivantes :


\begin{minipage}{0.3\linewidth}
\opidiv{279}{9}  
\begin{equation*}
\renewcommand{\arraystretch}{1.2}
\renewcommand{\arraycolsep}{2pt}
  \begin{array}{rrrr|rrr}
 & 2  & 9 & 7 & & 9 \\
\cline{5-7}
 & &  &  &  & \\
    & &  &  &   &   &  \\
    &  &  &  &   &   &  \\
    &&  &  &   &   &  \\
  \end{array}
\end{equation*}
\end{minipage}
\begin{minipage}{0.3\linewidth}
\item \opidiv{357}{5}
\begin{equation*}
\renewcommand{\arraystretch}{1.2}
\renewcommand{\arraycolsep}{2pt}
  \begin{array}{rrrr|rrr}
 & 3  & 5 & 7 &  1 & 2 \\
\cline{5-7}
 & &  &  &  & \\
    & &  &  &   &   &  \\
    &  &  &  &   &   &  \\
    &&  &  &   &   &  \\
  \end{array}
\end{equation*}

\end{minipage}
\begin{minipage}{0.3\linewidth}
\begin{equation*}
\renewcommand{\arraystretch}{1.2}
\renewcommand{\arraycolsep}{2pt}
  \begin{array}{rrrr|rrr}
1 & 6  & 8 & 3 & 9 & 5 \\
\cline{5-7}
 & &  &  &  & \\
    & &  &  &   &   &  \\
    &  &  &  &   &   &  \\
    &&  &  &   &   &  \\
  \end{array}
\end{equation*}
\end{minipage}


\item Traduis chacune de ces divisions par une égalité.
\end{enumerate}
\end{ExoCu}

 


\end{pageParcoursu}

  
%%%%%%%%%%%%%%%%%%%%%%%%%%%%%%%%%%%%%%%%%%%%%%%%%%%%%%%%%%%%%%%%%%%
%%%%  Niveau 2
%%%%%%%%%%%%%%%%%%%%%%%%%%%%%%%%%%%%%%%%%%%%%%%%%%%%%%%%%%%%%%%%%%%



\begin{pageParcoursd} 
 
%%%%%%%%%%%%%%%%%%%%%%%%%%%%%%%%%%%%%%%%%%%%%%%%%%%%%%%%%%%%%%%%%%%
\begin{ExoCd}{Représenter.}{1234}{2}{0}{0}{0}{0}


 
\end{ExoCd}

 
%%%%%%%%%%%%%%%%%%%%%%%%%%%%%%%%%%%%%%%%%%%%%%%%%%%%%%%%%%%%%%%%%%%
\begin{ExoCd}{Chercher.communiquer.}{1234}{2}{0}{0}{0}{0}



\end{ExoCd}


%%%%%%%%%%%%%%%%%%%%%%%%%%%%%%%%%%%%%%%%%%%%%%%%%%%%%%%%%%%%%%%%%%%
\begin{ExoCd}{Représenter. Raisonner.}{1234}{2}{0}{0}{0}{0}


\end{ExoCd}

 %%%%%%%%%%%%%%%%%%%%%%%%%%%%%%%%%%%%%%%%%%%%%%%%%%%%%%%%%%%%%%%%%%%
\begin{ExoCd}{Représenter. Raisonner.}{1234}{2}{0}{0}{0}{0}


\end{ExoCd}
 
%%%%%%%%%%%%%%%%%%%%%%%%%%%%%%%%%%%%%%%%%%%%%%%%%%%%%%%%%%%%%%%%%%%
\begin{ExoCd}{Représenter. Raisonner.}{1234}{2}{0}{0}{0}{0}


\end{ExoCd}
 
\end{pageParcoursd}

%%%%%%%%%%%%%%%%%%%%%%%%%%%%%%%%%%%%%%%%%%%%%%%%%%%%%%%%%%%%%%%%%%%
%%%%  Niveau 3
%%%%%%%%%%%%%%%%%%%%%%%%%%%%%%%%%%%%%%%%%%%%%%%%%%%%%%%%%%%%%%%%%%%
\begin{pageParcourst}

%%%%%%%%%%%%%%%%%%%%%%%%%%%%%%%%%%%%%%%%%%%%%%%%%%%%%%%%%%%%%%%%%%%
\begin{ExoCt}{Représenter.}{1234}{2}{0}{0}{0}{0}

 
Montrer que la somme de trois entiers consécutifs est toujours un multiple de 3. \point{6}
 

\end{ExoCt}

%%%%%%%%%%%%%%%%%%%%%%%%%%%%%%%%%%%%%%%%%%%%%%%%%%%%%%%%%%%%%%%%%%%
\begin{ExoCt}{Représenter. Raisonner.}{1234}{2}{0}{0}{0}{0}
 
 


\end{ExoCt}


%%%%%%%%%%%%%%%%%%%%%%%%%%%%%%%%%%%%%%%%%%%%%%%%%%%%%%%%%%%%%%%%%%%
\begin{ExoCt}{Raisonner.}{1234}{2}{0}{0}{0}{0}
 
\end{ExoCt}

%%%%%%%%%%%%%%%%%%%%%%%%%%%%%%%%%%%%%%%%%%%%%%%%%%%%%%%%%%%%%%%%%%%
\begin{ExoCt}{Représenter.}{1234}{2}{0}{0}{0}{0}

 

\end{ExoCt}

%%%%%%%%%%%%%%%%%%%%%%%%%%%%%%%%%%%%%%%%%%%%%%%%%%%%%%%%%%%%%%%%%%%
\begin{ExoCt}{Représenter.}{1234}{2}{0}{0}{0}{0}

 

\end{ExoCt} 
 
\end{pageParcourst}

%%%%%%%%%%%%%%%%%%%%%%%%%%%%%%%%%%%%%%%%%%%%%%%%%%%%%%%%%%%%%%%%%%%
%%%%  Brouillon
%%%%%%%%%%%%%%%%%%%%%%%%%%%%%%%%%%%%%%%%%%%%%%%%%%%%%%%%%%%%%%%%%%%


\begin{pageBrouillon} 
 
\ligne{32}



\end{pageBrouillon}

%%%%%%%%%%%%%%%%%%%%%%%%%%%%%%%%%%%%%%%%%%%%%%%%%%%%%%%%%%%%%%%%%%%
%%%%  Auto
%%%%%%%%%%%%%%%%%%%%%%%%%%%%%%%%%%%%%%%%%%%%%%%%%%%%%%%%%%%%%%%%%%%


%%%%%%%%%%%%%%%%%%%%%%%%%%%%%%%%%%%%%%%%%%%%%%%%%%%%%%%%%%%%%%%%%%%
\begin{pageAuto} 


\begin{ExoAuto}{Raisonner.}{1234}{2}{0}{0}{0}{0}

 
%%%%%%%%%%%%%%%%%%%%%%%%%%%%%%%%%%%%%%%%%%%%%%%%%%%%%%%%%%%%%%%%%%%
\end{ExoAuto}

\begin{ExoAuto}{Raisonner.}{1234}{2}{0}{0}{0}{0}
  

\end{ExoAuto}

%%%%%%%%%%%%%%%%%%%%%%%%%%%%%%%%%%%%%%%%%%%%%%%%%%%%%%%%%%%%%%%%%%%
\begin{ExoAuto}{Raisonner.}{1234}{2}{0}{0}{0}{0}

 
 

\end{ExoAuto}

 
%%%%%%%%%%%%%%%%%%%%%%%%%%%%%%%%%%%%%%%%%%%%%%%%%%%%%%%%%%%%%%%%%%%
\begin{ExoAuto}{Raisonner.}{1234}{2}{0}{0}{0}{0}

 
 

\end{ExoAuto}


\end{pageAuto}
