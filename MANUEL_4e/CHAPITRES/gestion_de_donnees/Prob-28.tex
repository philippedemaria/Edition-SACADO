
On a diffusé un documentaire sur les tremblements de terre et la fréquence à laquelle ils se 
produisent. Ce reportage comprenait un débat sur la prévisibilité des tremblements de terre. 
Un géologue a affirmé : « Au cours des vingt prochaines années, la probabilité qu’un 
tremblement de terre se produise à Zedville est de deux sur trois. »
Parmi les propositions suivantes, laquelle exprime le mieux ce que veut dire ce géologue ?

\begin{description}
\item[A.] Puisque  $\frac{2}{3} \times 20 \approx 13,3$, il y aura donc un tremblement de terre à Zedville dans 13 à 14  
ans à partir de maintenant.
\item[B.] $\frac{2}{3}$  est supérieur à  $\frac{1}{2}$, on peut donc être certain qu’il y aura un tremblement de terre  
à Zedville au cours des 20 prochaines années. 
\item[C.] La probabilité d’avoir un tremblement de terre à Zedville dans les vingt prochaines  
années est plus forte que la probabilité de ne pas en avoir.
\item[D.] On ne peut pas dire ce qui se passera, car personne ne peut être certain du moment 
où un tremblement de terre se produit
\end{description}

