\documentclass[a4paper,dvipsnames,french,10pt]{book}

% Option pour les PAP (commenter la ligne au-dessus)
%\documentclass[a4paper,dvipsnames,french,14pt]{extbook} 

\input{MISC/preambule_sacado.tex}

\newcommand{\titreChap}{Nombres relatifs}

\input{MISC/environnements_sacado.tex}
\input{MISC/macro_sacado.tex}

\title{Mathématiques 4e  : le livre sacado}
\author{L'équipe SACADO}

\begin{document}

%\maketitle



%\chapter{Titre de la séquence}{XX} %XX = numero de la séquence
%{URL du parcours}
%{
%\begin{CpsCol}
%	\textbf{Les savoir-faire du parcours}
% 	\begin{itemize}
% 		\item SF1
% 		\item SF2
% 	\end{itemize}
% \end{CpsCol}

%\begin{His}
%	% Un paragraphe parlant de la vie d'un ou une mathématicien ou mathématicienne
%\end{His}
%
%\begin{ExoDec}{Compétence.}{1234}{1}{0}{0}{0}
%% Un exercice de découverte/d'accroche
%\end{ExoDec}
%}

%%%%%%%%%%%%%%%%%%%%%%%%%%%%%%%%%%%%%%%%%%%%%%%%%%%%%%%%%%%%%%%%%%%
%%%% Parcours niveau 1
%%%%%%%%%%%%%%%%%%%%%%%%%%%%%%%%%%%%%%%%%%%%%%%%%%%%%%%%%%%%%%%%%%%

\begin{pageParcoursu} % Début du parcours niveau 1

%Premier exo du parcours 1
\begin{ExoCuN}{Raisonner.}{0}{0}{0}{0}{0}
Compléter les phrases suivantes :

\mini{.47\linewidth}{
\begin{enumerate}
\item L'opposé du nombre $7,5$ est \textcolor{red}{$-7,5$}
\item L'opposé du nombre $-12$ est \textcolor{red}{$12$}
\item L'opposé du nombre $1$ est \textcolor{red}{$-1$}
\end{enumerate}
}{.47\linewidth}{
\begin{enumerate}
\setcounter{enumi}{3}
\item L'opposé du nombre $-60$ est \textcolor{red}{$60$}
\item L'opposé du nombre $-2,67$ est \textcolor{red}{$2,67$}
\item L'opposé du nombre $8$ est \textcolor{red}{$-8$}
\end{enumerate}
}

\end{ExoCuN}

%Deuxième exo du parcours 1
\begin{ExoCuN}{Raisonner.}{0}{0}{0}{0}{0}
\begin{enumerate}
 \item Ranger les nombres suivants dans l'ordre croissant : $36 \hspace{.5cm} -30 \hspace{.5cm} -7 \hspace{.5cm} 1$ \vspace{.2cm}

\begin{center}
$\textcolor{red}{-30}<\textcolor{red}{-7}<\textcolor{red}{1}<\textcolor{red}{36}$
\end{center}


 \item Ranger les nombres suivants dans l'ordre décroissant : $-21,9 \hspace{.5cm} -21,86 \hspace{.5cm} -21,53 \hspace{.5cm} -21,67$ \vspace{.2cm}
 
\begin{center}
$\textcolor{red}{-21,53}>\textcolor{red}{-21,67}>\textcolor{red}{-21,86}>\textcolor{red}{-21,9}$
\end{center}

\end{enumerate}
\end{ExoCuN}

%Troisième exo du parcours 1
\begin{ExoCuN}{Calculer.}{0}{0}{0}{0}{0}
Effectuer les opérations suivantes :\vspace{.2cm}

\mini{.47\linewidth}{
\begin{enumerate}
\item $A=(+6,4)+(+25,5)=\textcolor{red}{31,9}$ \vspace{.2cm}
\item $B=(+16,4)+(-23,8)=\textcolor{red}{-7,4}$ \vspace{.2cm}
\item $C=(-11,9)+(+19,7)=\textcolor{red}{7,8}$ \vspace{.2cm}
\end{enumerate}
}{.47\linewidth}{
\begin{enumerate}
\setcounter{enumi}{3}
\item $D=(+3)-(+26)=\textcolor{red}{-23}$ \vspace{.2cm}
\item $E=(-5)-(-28)=\textcolor{red}{23}$ \vspace{.2cm}
\item $F=(-17)-(-16)=\textcolor{red}{-1}$ \vspace{.2cm}
\end{enumerate}
}

\end{ExoCuN}

%Quatrième exo du parcours 1
\begin{ExoCuN}{Calculer.}{0}{0}{0}{0}{0}
Effectuer les opérations suivantes :\vspace{.2cm}
\begin{enumerate}
\item $A=(-7)+(+9)+(+2)+(-7)=\textcolor{red}{-3}$\vspace{.2cm}
\item $B=(-10)-(+5)+(+4)+(-12)=\textcolor{red}{-23}$
\end{enumerate}
\end{ExoCuN}



\mini{.47\linewidth}{
%Cinquième exo du parcours 1
\begin{ExoCuN}{Raisonner.}{0}{0}{0}{0}{0}
Déterminer le signe des opérations suivantes :\vspace{.2cm}
\begin{enumerate}
\item $A=(+4)\times(-7)$ A est un nombre \textcolor{red}{négatif}\vspace{.2cm}
\item $B=(+6)\div(-6)$ B est un nombre \textcolor{red}{négatif}\vspace{.2cm}
\item $C=(-4)\times(-6)$ C est un nombre \textcolor{red}{positif}
\end{enumerate}
\end{ExoCuN}
}{.47\linewidth}{
%Sixième exo du parcours 1
\begin{ExoCuN}{Raisonner. Calculer.}{0}{0}{0}{0}{0}
Effectuer les opérations suivantes :\vspace{.2cm}
\begin{enumerate}
\item $A=(-2)\times(-4,6)=\textcolor{red}{+9,2}$ \vspace{.2cm}
\item $B=(-5)\times(+4)=\textcolor{red}{-20}$ \vspace{.2cm}
\item $C=(+6)\div(-2)=\textcolor{red}{-3}$ \vspace{.2cm}
\end{enumerate}
\end{ExoCuN}
}

%Septième exo du parcours 1
\begin{ExoCuN}{Raisonner.}{0}{0}{0}{0}{0}
Déterminer le signe des opérations suivantes :\vspace{.2cm}

\mini{.47\linewidth}{
 $A=\dfrac{(-2)\times(+9)}{(-4)\times(-3)}$ \vspace{.2cm}

A est un nombre \textcolor{red}{négatif}
}{.47\linewidth}{
$B=\dfrac{(-4)\times(+8)}{(+7)}$ \vspace{.2cm}

B est un nombre \textcolor{red}{négatif}
}
\end{ExoCuN}

%Huitième exo du parcours 1
\begin{ExoCuN}{Raisonner. Calculer.}{0}{0}{0}{0}{0}
Effectuer l'enchainement d'opérations suivant :

$A=(-1)-(+9)\times(-8)$\vspace{.2cm}

$A=\textcolor{red}{(-1)-(-72)}$\vspace{.2cm}

$A=\textcolor{red}{(+71)}$
\end{ExoCuN}




\end{pageParcoursu} % Fin du parcours niveau 1

%%%%%%%%%%%%%%%%%%%%%%%%%%%%%%%%%%%%%%%%%%%%%%%%%%%%%%%%%%%%%%%%%%%
%%%% Parcours niveau 2
%%%%%%%%%%%%%%%%%%%%%%%%%%%%%%%%%%%%%%%%%%%%%%%%%%%%%%%%%%%%%%%%%%%

\begin{pageParcoursd} % Début du parcours niveau 2

%Premier exo du parcours 2
\begin{ExoCdN}{Raisonner.}{0}{0}{0}{0}{0}
Retrouver le nombre manquant :\vspace{.2cm}

\mini{.47\linewidth}{
\begin{enumerate}
\item $(+12,9)+\textcolor{red}{(-12,9)}=0$ \vspace{.2cm}
\item $(-169)+\textcolor{red}{(+169)}=0$ \vspace{.2cm}
\item $\textcolor{red}{(+5,4)}+(-5,4)=0$ \vspace{.2cm}
\end{enumerate}
}{.47\linewidth}{
\begin{enumerate}
\setcounter{enumi}{3}
\item $(+7,1)+\textcolor{red}{(-7,1)}=0$ \vspace{.2cm}
\item $\textcolor{red}{(-21)}+(+21)=0$ \vspace{.2cm}
\item $(-30)+\textcolor{red}{(+30)}=0$ \vspace{.2cm}
\end{enumerate}
}

\end{ExoCdN}

\mini{.47\linewidth}{
%Deuxième exo du parcours 2
\begin{ExoCdN}{Raisonner. Calculer.}{0}{0}{0}{0}{0}
Compléter les opérations suivantes :\vspace{.2cm}

\begin{enumerate}
\item $A=(+49)+\textcolor{red}{(+67)}=116$ \vspace{.2cm}
\item $B=(+71)+\textcolor{red}{(-84)}=-13$ \vspace{.2cm}
\item $C=(+57)+\textcolor{red}{(-73)}=-16$ \vspace{.2cm}
\end{enumerate}
\end{ExoCdN}
}{.47\linewidth}{
%Troisième exo du parcours 2
\begin{ExoCdN}{Calculer.}{0}{0}{0}{0}{0}
Effectuer les opérations suivantes :\vspace{.2cm}

\begin{enumerate}
\item $A=(-8,2)-(-26,9)=\textcolor{red}{18,7}$ \vspace{.2cm}
\item $B=(-18)-(+19)=\textcolor{red}{-37}$ \vspace{.2cm}
\item $C=(+10,5)-(+19,7)=\textcolor{red}{-9,2}$ \vspace{.2cm}
\end{enumerate}
\end{ExoCdN}
}




%Quatrième exo du parcours 2
\begin{ExoCdN}{Calculer.}{0}{0}{0}{0}{0}
Effectuer les opérations suivantes :\vspace{.2cm}
\begin{enumerate}
\item $A=(-2)-(+5)-(-2)+(-4)=\textcolor{red}{-9} $\vspace{.2cm}
\item $B=-10-5+4-(-12)=\textcolor{red}{1} $
\end{enumerate}
\end{ExoCdN}

%Cinquième exo du parcours 1
\begin{ExoCdN}{Raisonner.}{0}{0}{0}{0}{0}
Déterminer le signe des opérations suivantes :\vspace{.2cm}

\mini{.47\linewidth}{
\begin{enumerate}
\item $A=(+4)\times(-7)$ A est un nombre \textcolor{red}{négatif} \vspace{.2cm}
\item $B=(+6)\div(-6)$ B est un nombre \textcolor{red}{négatif} \vspace{.2cm}
\item $C=(-4)\times(-6)$ C est un nombre \textcolor{red}{positif}
\end{enumerate}
}{.47\linewidth}{
\begin{enumerate}
\setcounter{enumi}{3}
\item $D=\dfrac{-11}{-2}$ D est un nombre \textcolor{red}{positif}\vspace{.2cm}
\item $E=-\dfrac{16}{-7}$ E est un nombre \textcolor{red}{positif}\vspace{.2cm}
\item $F=-3,4\times5$ F est un nombre \textcolor{red}{négatif}
\end{enumerate}
}

\end{ExoCdN}


%Sixième exo du parcours 1
\begin{ExoCdN}{Raisonner. Calculer.}{0}{0}{0}{0}{0}
Effectuer les opérations suivantes :\vspace{.2cm}

\mini{.47\linewidth}{
\begin{enumerate}
\item $A=(-2)\times(-4,6)=\textcolor{red}{9,2}$ \vspace{.2cm}
\item $B=(-5)\times4=\textcolor{red}{-20}$ \vspace{.2cm}
\item $C=6\div(-2)=\textcolor{red}{-3}$ \vspace{.2cm}
\end{enumerate}
}{.47\linewidth}{
\begin{enumerate}
\setcounter{enumi}{3}
\item $D=(-50)\times 10=\textcolor{red}{-500}$ \vspace{.2cm}
\item $E=(-25)\div (-5)=\textcolor{red}{5}$ \vspace{.2cm}
\item $F=3\times(-3,7)=\textcolor{red}{-11,1}$ \vspace{.2cm}
\end{enumerate}
}

\end{ExoCdN}

%Septième exo du parcours 1
\begin{ExoCdN}{Raisonner.}{0}{0}{0}{0}{0}
Déterminer le signe de l'opération suivante :\vspace{.2cm}

\mini{.47\linewidth}{
 $A=-\dfrac{(+10)\times(+7)}{(+1)\times(+7)}$ \vspace{.2cm}

A est un nombre \textcolor{red}{négatif}
}{.47\linewidth}{
$B=-\dfrac{(-6)\times(+8)}{(+7)\times(+6)}$ \vspace{.2cm}

B est un nombre \textcolor{red}{positif}
}
\end{ExoCdN}

%Huitième exo du parcours 2
\begin{ExoCdN}{Raisonner. Calculer.}{0}{0}{0}{0}{0}
Effectuer l'enchainement d'opérations suivant :

$A=(-7)\times(-1)+8\times(-2)$\vspace{.2cm}

$A=\textcolor{red}{(+7)+(-16)}$\vspace{.2cm}

$A=\textcolor{red}{-9}$
\end{ExoCdN}

\end{pageParcoursd} % Fin du parcours niveau 2

%%%%%%%%%%%%%%%%%%%%%%%%%%%%%%%%%%%%%%%%%%%%%%%%%%%%%%%%%%%%%%%%%%%
%%%% Parcours niveau 3
%%%%%%%%%%%%%%%%%%%%%%%%%%%%%%%%%%%%%%%%%%%%%%%%%%%%%%%%%%%%%%%%%%%

\begin{pageParcourst} % Début du parcours niveau 3

%Premier exo du parcours 3
\begin{ExoCtN}{Raisonner.}{0}{0}{0}{0}{0}
Retrouver le nombre manquant :\vspace{.2cm}

\mini{.47\linewidth}{
\begin{enumerate}
\item $12,9+\textcolor{red}{(-12,9)}=0$ \vspace{.2cm}
\item $-169+\textcolor{red}{169}=0$ \vspace{.2cm}
\item $\textcolor{red}{5,4}-5,4=0$ \vspace{.2cm}
\end{enumerate}
}{.47\linewidth}{
\begin{enumerate}
\setcounter{enumi}{3}
\item $\dfrac{7}{5}+\textcolor{red}{(-\dfrac{7}{5})}=0$ \vspace{.2cm}
\item $\textcolor{red}{-21}+21=0$ \vspace{.2cm}
\item $-30+\textcolor{red}{30}=0$ \vspace{.2cm}
\end{enumerate}
}
\end{ExoCtN}

\mini{.47\linewidth}{
%Deuxième exo du parcours 3
\begin{ExoCtN}{Calculer.}{0}{0}{0}{0}{0}
Compléter les opérations suivantes :\vspace{.2cm}

\begin{enumerate}
\item $A=-27,2+\textcolor{red}{67,3}=40,1$ \vspace{.2cm}
\item $B=-11,3+\textcolor{red}{25,8}=14,5$ \vspace{.2cm}
\item $C=1,6+\textcolor{red}{(-6,8)}=-5,2$ \vspace{.2cm}
\end{enumerate}
\end{ExoCtN}
}{.47\linewidth}{
%Troisième exo du parcours 3
\begin{ExoCtN}{Calculer.}{0}{0}{0}{0}{0}
Effectuer les opérations suivantes :\vspace{.2cm}

\begin{enumerate}
\item $A=-8,2-(-26,9)=\textcolor{red}{18,7}$ \vspace{.2cm}
\item $B=-18-19=\textcolor{red}{-37}$ \vspace{.2cm}
\item $C=10,5-19,7=\textcolor{red}{-9,2}$ \vspace{.2cm}
\end{enumerate}
\end{ExoCtN}
}

\mini{.47\linewidth}{
%Quatrième exo du parcours 3
\begin{ExoCtN}{Calculer.}{0}{0}{0}{0}{0}
Effectuer les opérations suivantes :\vspace{.2cm}
\begin{enumerate}
\item $A=-20-6-6-9=\textcolor{red}{-41}$\vspace{.2cm}
\item $B=12-(-15+10)+5=\textcolor{red}{22}$
\end{enumerate}
\end{ExoCtN}
}{.47\linewidth}{
%Cinquième exo du parcours 3
\begin{ExoCtN}{Raisonner. Calculer.}{0}{0}{0}{0}{0}
Effectuer les opérations suivantes :\vspace{.2cm}
\begin{enumerate}
\item $A=-2\times(-4,6)=\textcolor{red}{9,2}$ \vspace{.2cm}
\item $B=-5\times4=\textcolor{red}{-20}$ \vspace{.2cm}
\item $C=26,6\div(-2)=\textcolor{red}{-13,3}$ \vspace{.2cm}
\end{enumerate}
\end{ExoCtN}
}


%Cinquième exo du parcours 1
\begin{ExoCtN}{Raisonner.}{0}{0}{0}{0}{0}
Déterminer le signe de l'opération suivante :\vspace{.2cm}

\mini{.47\linewidth}{
 $A=-\dfrac{10\times(-7)}{-1\times(-7)}$ \vspace{.2cm}

A est un nombre \textcolor{red}{positif}
}{.47\linewidth}{
$B=-\dfrac{-6\times8\times(-6)}{-7\times6}$ \vspace{.2cm}

B est un nombre \textcolor{red}{positif}
}
\end{ExoCtN}

%Cinquième exo du parcours 3
\begin{ExoCtN}{Raisonner. Calculer.}{0}{0}{0}{0}{0}
Effectuer l'enchainement d'opérations suivant :

$A=(-6)\times(-10)+\dfrac{-15}{5}$\vspace{.2cm}

$A=\textcolor{red}{60+(-3)}$\vspace{.2cm}

$A=\textcolor{red}{57}$
\end{ExoCtN}

%Sixième exo du parcours 3
\begin{ExoCtN}{Raisonner. Calculer.}{0}{0}{0}{0}{0}
Voici un programme de calcul :

\mini{.47\linewidth}{
\begin{tcolorbox}[colback=white]
\begin{itemize}
\item Choisir un nombre.
\item Ajouter $-4$.
\item Retirer $-2,5$.
\item Prendre l'opposé du résultat.
\end{itemize}
\end{tcolorbox}
}{.47\linewidth}{
Appliquer ce programme à chacun de ces nombres :\vspace{.2cm}
\begin{enumerate}
\item Pour $-2,5$ le résultat est : 

1. \textcolor{red}{$-2,5+(-4)=-6,5$}

2. \textcolor{red}{$-6,5-(-2,5)=-4$}

3. \textcolor{red}{L'opposé de $-4$ est $+4$} \vspace{.2cm}
\item Pour $0$ le résultat est : \textcolor{red}{$1,5$}\vspace{.2cm}
\item Pour $-1,5$ le résultat est : \textcolor{red}{$3$}\vspace{.2cm}
\item Pour $-1$ le résultat est : \textcolor{red}{$2,5$}
\end{enumerate}
}

\end{ExoCtN}

\end{pageParcourst} % Fin du parcours niveau 3

\end{document}
