
Pour chaque affirmation, dire en justifiant, si elle est vraie ou fausse.

\bigskip

\begin{tabularx}{\linewidth}{l X}
\textbf{Affirmation 1 :} &~\\
&\begin{tabular}{|l|}\hline
\textbf{Programme de calcul A}\\
Choisir un nombre\\
Ajouter 3\\
Multiplier le résultat par 2\\
Soustraire le double du nombre de départ\\ \hline
\end{tabular}\\
&Le résultat du programme de calcul A est toujours égal à 6.\\
\textbf{Affirmation 2 :} &Le résultat du calcul $\dfrac{7}{5} - \dfrac{4}{5} \times \dfrac{1}{3}$ est égal à $\dfrac{1}{5}$.\\
\textbf{Affirmation 3 :} &La solution de l'équation $4x - 5 = x + 1$ est une solution de l'équation $x^2 - 2x = 0$.\\
\textbf{Affirmation 4 :} &Pour tous les nombres entiers $n$ compris entre $2$ et $9$, $2^n - 1$ est un nombre premier.\\
\end{tabularx}