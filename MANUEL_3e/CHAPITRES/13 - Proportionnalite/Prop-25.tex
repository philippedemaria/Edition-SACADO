
\textit{En application de la réglementation, au moins 2\% des places de stationnement matérialisées, situées sur les voies, publiques ou privées, ouvertes à la circulation publique, et 2\% des places des parcs de stationnement des ERP doivent être réservées aux titulaires de la carte européenne de stationnement.}
\begin{tiny}
\hfill{Source :http://www.developpement-durable.gouv.fr/IMG/pdf/Stationnement-reserve-handicapes.pdf}
\end{tiny}

\begin{enumerate}
\item Un parking de 500 places propose 12 places de stationnement pour les titulaires de la carte européenne de stationnement.

Le gérant de parking respecte-t-il la loi ?

\item Dans le parking de centre ville, il y 35 places réservées aux titulaires de la carte européenne de stationnement. Quel nombre maximal de places peut contenir ce parking sans enfreindre la loi ?
\end{enumerate}
