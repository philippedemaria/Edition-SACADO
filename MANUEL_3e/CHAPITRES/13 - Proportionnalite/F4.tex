\begin{titre}[Les nombres rationnels]

{\LARGE {\color{bleu3}Simplifier une fraction}}
\end{titre}



\begin{CpsCol}
\textbf{Utiliser des nombres pour calculer et résoudre des problèmes}
\begin{description}
\item[$\square$] Calculer avec des fractions (calcul exact, approché, mental, à la main ou instrumenté)
\item[$\square$] Repérer et placer des nombres sur une droite graduée
\end{description}
\end{CpsCol}




\begin{reg}\textbf{(Multiplication)}\\
Le dénominateur du produit est le produit des dénominateurs et le numérateur du produit est le produit des numérateurs.
\end{reg}

   
\begin{Ex}   
$\frac{3}{4}\times \frac{-5}{21} = \frac{3\times{-5}}{4 \times 21}=\frac{-15}{84} = \frac{5}{28}$ 
\end{Ex}
\begin{Rq}   
\begin{itemize}
\item Dans la pratique, on essaie de \og simplifier \fg{} avant d'effectuer les multiplications.
\item Il est plus simple de multiplier que de diviser car on ne doit pas recherche un dénominateur commun.
\end{itemize}
\end{Rq}


\begin{Def}
\textbf{L'inverse} d'un nombre rationnel $x$ non nul est le nombre $y$,noté $x^{-1}$ ou $\frac{1}{x}$ tel que $xy=1$
\end{Def}

   
\begin{Ex}   
L'inverse de $2$ est $\frac{1}{2}$ ou encore $0,5$.\\
L'inverse de $\frac{-3}{4}$ est $\frac{4}{-3}$ ou encore $\frac{-4}{3}$
\end{Ex}


\begin{reg}\textbf{(Division)}\\
Pour diviser par un nombre, on multiplie par son inverse.
\end{reg}

   
\begin{Ex}   
$\frac{3}{4}\div \frac{-5}{21} =\frac{3}{4}\times \frac{-21}{5}=\frac{-63}{20} $ 
\\$\frac{3}{\frac{4}{5}}=3\times \frac{5}{4}=\frac{15}{4}$ 
\end{Ex}



\begin{autoeval}
\begin{tabular}{p{12cm}p{0.5cm}p{0.5cm}p{0.5cm}p{1cm}}
\textbf{Compétences visées} &  M I & MF & MF  & TBM \vcomp \\ 
Pratiquer le calcul exact, approché, mental, à la main ou instrumenté & $\square$ & $\square$  & $\square$ & $\square$ \vcomp \\ 
Repérer et placer des nombres sur une droite graduée & $\square$ & $\square$ & $\square$ & $\square$ \vcomp \\ 
\end{tabular}
{\footnotesize MI : maitrise insuffisante ; MF = Maitrise fragile ; MS = Maitrise satisfaisante ; TBM = Très bonne maitrise}
 
\end{autoeval}