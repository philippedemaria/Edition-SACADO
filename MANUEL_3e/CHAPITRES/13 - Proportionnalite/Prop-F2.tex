\begin{titre}[La proportionnalité]

\Titre{Les échelles}{1}
\end{titre}


\begin{CpsCol}
\textbf{Utiliser des nombres pour calculer et résoudre des problèmes}
\begin{description}
\item[$\square$] Calculer une échelle
\item[$\square$] Utiliser une échelle
\end{description}
\end{CpsCol}


\begin{DefT}{Échelle}
Il n'est pas possible de représenter le monde réel sur une feuille, sur un écran, un GPS $\cdots$. Lorsqu'on souhaite le dessiner, on utilise une \textbf{échelle}\index{Échelle} pour respecter les distances réelles.
\end{DefT}

\begin{Rqs}\index{Échelle!Formule}
\begin{description}
\item L'échelle d'une représentation est le coefficient égal à $\frac{\text{Distance sur la carte}}{\text{Distance réelle}}$. 
\item Google Maps utilise une longueur en guise d'échelle.
\end{description}
\end{Rqs}

