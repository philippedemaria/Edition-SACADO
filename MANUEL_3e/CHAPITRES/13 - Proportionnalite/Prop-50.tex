
Un ouvrier doit peindre une surface totale d'environ 168 m$^2$.

 
\begin{itemize}
\item[$\bullet~~$] Un pot de 10~L de peinture permet de couvrir une surface de 40~m$^2 $ ;
\item[$\bullet~~$] Le coût d'un pot de 10~L de peinture est de 400~\euro{} ;
\item[$\bullet~~$] Un ouvrier peint une surface de 42 m$^2$ à l'heure.
\end{itemize}


 
\medskip

Compléter cette facture à l'aide des informations fournies ci-dessus.

\medskip

\begin{tabularx}{\linewidth}{|*{4}{>{\centering \arraybackslash}X|}}\hline 
\textbf{Quantité}& \textbf{Désignation} &\textbf{Prix unitaire} &\textbf{Prix total}\\ \hline 
5& pots d'antirouille &500,00~\euro&\np{2500,00}~\euro\\ \hline 
\dotfill&pots de peinture&400,00~\euro &\dotfill\\ \hline
\dotfill&\small heures (main d'{\oe}uvre)&35,00~\euro &\dotfill\\ \hline
\multicolumn{3}{|l|}{Total HT (co\^ut hors taxe)}&\dotfill\\ \hline  
\multicolumn{3}{|l|}{Montant de la TVA à 19,6\,\%}&\dotfill\\ \hline 
\multicolumn{3}{|l|}{TOTAL TTC (coût toutes taxes comprises)}&\dotfill\\ \hline
\end{tabularx}
