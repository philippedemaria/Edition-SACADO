\begin{titre}[La proportionnalité]

\Titre{Les pourcentages}{1,5}
\end{titre}

\begin{CpsCol}
\textbf{Utiliser des nombres pour calculer et résoudre des problèmes}
\begin{description}
\item[$\square$] Calculer un pourcentage
\item[$\square$] Savoir appliquer un pourcentage
\item[$\square$] Résoudre des problèmes de pourcentage
\item[$\square$] Établir le lien entre coefficient multiplicateur et pourcentage
\end{description}
\end{CpsCol}



\begin{DefT}{Pourcentage}
Pour comparer deux proportions, on les ramène à un base référence commune égale à 100. On parle alors de  pourcentage\index{Pourcentage}.
\end{DefT}

\begin{Ex}
Dans la Cinquième A, il y a 5 filles sur 20 élèves et dans la Cinquième B, on compte 6 filles pour 25 élèves.
\begin{description}
\item[En Cinquième A] il y a $\frac{5}{20}=\frac{5 {\color{red}\times 5} }{20 {\color{red}\times 5}} = \frac{25}{100} = 25\%$ de filles dans la classe.
\item[En Cinquième B] il y a $\frac{6}{25}=\frac{6 {\color{red}\times 4} }{25 {\color{red}\times 4}} = \frac{24}{100} = 24\%$ de filles dans la classe.
\end{description}
La cinquième A comporte \textbf{en proportion} plus de fille que la cinquième B.
\end{Ex}



\begin{DefT}{Coefficient multiplicateur}\index{Coefficient multiplicateur}
Lorsque une valeur évolue selon un pourcentage de $t \%$, le coefficient multiplicateur est égal à $(1+t\%)$.
\end{DefT}


\begin{Ex}
\textit{La Taxe sur la Valeur Ajoutée en France est de 20,6\%. Un produit alimentaire coute 15,20 € Hors Taxe. Quel est sa valeur TTC ?}\\
Soit P le prix du produit alimentaire.\\
$P = 15,20 \times (1+20,6\%)= 15,20 \times (1+0,206) = 15,20 \times 1,206 \approx 18,33$.\\
Le prix TTC est donc 18,33 €.
\end{Ex}

\begin{Ex}
\textit{Le jour des soldes, une paire de ski à 428 € est soldée à 25\%. Quel est le prix après la solde ? }\\
Soit P le prix de la paire de ski.\\
$P = 428 \times (1-25\%)= 428 \times (1-0,25) = 428 \times 0,75 =321 $.\\
Le prix soldé est donc 321 €.
\end{Ex}