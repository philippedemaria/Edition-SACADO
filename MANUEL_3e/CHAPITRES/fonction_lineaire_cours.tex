\chapter{Arithmétique}
{https://sacado.xyz/qcm/parcours_show_course/0/117129}
{
 \begin{CpsCol}
\textbf{Les savoir-faire du parcours}
 \begin{itemize}
 \item Utiliser les notations et le vocabulaire fonctionnels.
 \item Passer d'un mode de représentation d'une fonction à un autre.
 \item Déterminer, à partir de tous les modes de représentation, l'image d'un nombre.
 \item Déterminer un antécédent à partir d'une représentation graphique ou d'un tableau de valeurs d'une fonction.
 \item Déterminer de manière algébrique l'antécédent par une fonction, dans des cas se ramenant à la résolution d'une équation du premier degré.
 \end{itemize}
 
 \end{CpsCol}
 
 \begin{His}
 zzzz
 \end{His}

\begin{ExoDec}{Chercher.}{1234}{1}{0}{0}{0}

 On souhaite ranger 142 bonbons dans des boites de 12 bonbons. Les boites doivent être complétées entièrement avant
d'utiliser d'une nouvelle boite.
\begin{enumerate}
\item Combien de bonbons ne sont-ils pas rangés dans une boite complète?
\item Combien de boites sont-elles entièrement remplies?
\item Quel est le nombre minimal de boites nécessaires pour ranger tous les 142 bonbons? Combien de bonbons faudrait-il
rajouter pour remplir la boite incomplète?
\end{enumerate}

\end{ExoDec}
 
 
\begin{ExoDec}{Chercher.}{1234}{1}{0}{0}{0}

Trois bateaux partent de Marseille, l'un tous les 7 jours, le second tous les 12 jours, le troisième tous les 14 jours. Ils partent
tous les trois le 1er mars. À quelle prochaine date partiront-ils encore tous les trois ensemble du port de Marseille ?



\end{ExoDec} 
 
 
}


\begin{pageCours}




\section{Notion de fonction}


\begin{DefT}{En fonction de\index{En fonction de}}
Lorsqu'une relation associe deux quantités, on dit que l'on peut exprimer une quantité \textbf{en fonction de} l'autre. \\ En général, on peut alors établir une formule qui lie ces deux quantités.
\end{DefT}

\begin{Ex}
1 pain au chocolat coûtent 1,45 \euro{} donc $n$ pains au chocolats coûtent $1,45n$. On peut dire que le prix $p$ de $n$ pains au chocolat s'obtient par la formule $p=1,45n$. Le prix est \textbf{en fonction du} nombre de pains au chocolats.
\end{Ex}




\begin{ExQr}{1234}
L'égalité de la division euclidienne de $254$ par $7$ s'écrit : $254 = 36 \times 7 +2$. $36$ est le quotient et $2$ est le reste.
\end{ExQr}


\begin{Att}

$22 = 3 \times 5 +7$ mais cette égalité n'est pas l'écriture de la division euclidienne de $22$ par $5$ ou par $3$. En effet, $7 > 3$ et $7 > 5$.

L'égalité de la division euclidienne de $22$ par $3$ s’écrit : $22 = 3 \times 7 +1$

L'égalité de la division euclidienne de $22$ par $5$ s’écrit : $22 = 4 \times 5 +2$.
\end{Att}

\section{Multiples et diviseurs}

 
  

\begin{DefT}{Multiples et diviseurs}\index{Multiple}\index{Diviseur}
 Soit $a$ et $b$ deux nombres entiers positifs. 
 
 On dit que $a$ est un \textbf{multiple} de $b$ s'il existe un nombre entier $k$ tel que $a = k \times b$.
 
 On dit alors que $b$ est un \textbf{diviseur} de $a$. 
\end{DefT}

\begin{Rq}

Le reste de la division euclidienne de $a$ par $b$ est alors égal à $0$. 
\end{Rq}


\begin{minipage}{0.6\linewidth}

\begin{ExQr}{1234}
Soit $a=56$ et $b=7$, on a $56 = 7 \times 8+0 = 7 \times 8$. 

Donc $56$ est un multiple de $7$.

\end{ExQr}
\end{minipage}
\begin{minipage}{0.4\linewidth}
\begin{OuQr}{1234}

 Calculateur de multiples et de diviseurs
\end{OuQr}
\end{minipage}
 
 
 
 



 

 
\begin{DefT}{Variable}
Une variable \index{Variable} mathématique désigne une valeur arbitraire, pas totalement précisée, ou même inconnue - appartenant à un ensemble. Les valeurs de la variable varie durant l'exercice. Comme ces valeurs ne sont pas fixes, on note la variable par une lettre.
\end{DefT}

\begin{Ex}
Un carré a pour coté $x$. Son périmètre est $\mathscr P = 4x$. On préfère écrire $\mathscr{P}(x) = 4x$. $x$ est la variable car pour chaque valeur de $x$ positive, on obtient une valeur du périmètre. Le périmètre est en fonction de $x$.
\end{Ex}

\begin{DefT}{Fonction}
Une \textbf{fonction} \index{Fonction} est un procédé qui à un nombre donné associe un unique nombre. La fonction est explicitée par une expression littérale en fonction de la variable.
\end{DefT}

\begin{Nt}
On écrit $f : x \longmapsto f(x)$ et on lit $f$ est la fonction qui à $x$ associe le nombre $f(x)$.
\end{Nt}

 
\section{Image, antécédent}


\begin{DefT}{Image, antécédent}
L'\textbf{image} \index{Image} d'un nombre par $f$ est l'unique nombre obtenu après le procédé calculatoire de la fonction.\\ Lorsque $f : a \longmapsto b$, on dit que $b$ est l'image de $a$ par $f$. On note alors $f(a)=b$. \\
$a$ est appelé un \index{Antécédent} \textbf{antécédent} de $b$ par $f$.
\end{DefT}

\begin{Ex}
Un triangle équilatéral de coté 4 cm a un périmètre égal à 12 cm. \\
On peut alors dire que 12 est l'image de 4 par la fonction $f$, où $f : 4 \mapsto 12$. $4$ est un antécédent de 12 par $f$.\\Plus généralement, $f : x \mapsto 3x$ ou $f(x)=3x$. $f(x)$ est l'image de $x$ par la fonction $f$.\\
La fonction $f$ est la fonction qui a une longueur du coté d'un triangle équilatéral associe son périmètre.
\end{Ex}

\begin{Ety}
\textbf{Antécédent} est composé de anté - cédent : qui vient avant le procédé. \\Un antécédent vient donc avant la flèche qui symbolise  le procédé. Antécédent $\mapsto$  image.
\end{Ety}
 





\end{pageCours} 
\begin{pageAD} 
 

\Sf{Utiliser et écrire en ligne la division euclidienne de $a$ par $b$, deux nombres entiers.}
 


\begin{ExoCad}{Calculer.}{1234}{0}{0}{0}{0}{0}
 
\begin{enumerate}
\item Exprimer le périmètre $\mathscr P$ d'un triangle équilatéral en fonction de la longueur d'un coté $c$.
\item La largeur d'un rectangle est 3 cm. Exprimer le périmètre $\mathscr P$ de ce rectangle en fonction de sa longueur $L$.
\item 12 œufs coutent 1,50 \euro{}. Exprime le prix $p$ de $n$ œufs achetés en fonction de $n$. 
\end{enumerate}
\end{ExoCad}

\begin{ExoCad}{Calculer.}{1234}{0}{0}{0}{0}{0}
 
\textbf{La fourmi est-elle la plus forte du monde ?}

[...] Des études précédentes affirment que l’insecte est capable de porter jusqu’à 1.000 fois son propre poids, soit l’équivalent d’un oisillon tombé du nid. Toutefois, ce chiffre, déjà sensationnel, pourrait être encore plus élevé, selon les récentes conclusions de travaux menés par des ingénieurs en mécanique et aérospatiale de l'Ohio State University. Ces derniers affirment que la fourmi serait capable de porter jusqu'à 5.000 fois son poids !
[...]

 http://www.maxisciences.com/fourmi/les-fourmis-des-insectes-encore-plus-forts-qu-039-on-ne-pense_art32040.html

\begin{enumerate}
\item Quelle serait la charge $C$ que porterait un humain de 15 kg ?

 
\item Quelle serait la charge $C$ que porterait un humain de 45 kg ?

 
\item Quelle serait la charge $C$ que porterait un humain de $p$ kg ?
 
\end{enumerate}
\end{ExoCad}

\begin{ExoCad}{Calculer.}{1234}{0}{0}{0}{0}{0}
 
Le débit moyen de téléchargement est de 0,85 Mo/s.
\begin{enumerate}
\item Complète ce tableau

 


\begin{tabular}{|c|>{\centering\arraybackslash}p{2cm}|>{\centering\arraybackslash}p{2cm}|>{\centering\arraybackslash}p{2cm}|>{\centering\arraybackslash}p{2cm}|}
\hline 
$d$ en Mo & 1 & 2 & 10 & 12 \\
\hline 
temps $t$ en $s$ &  &  &  &    \\ 
\hline 
\end{tabular} 





 
\item Déterminer une formule donnant le débit moyen $d$ en fonction du temps $t$.

\item Représentation graphique dans un repère.
\begin{enumerate}
 
\item Recopier et complète les axes du repère par les données qu'ils représentent.
\item Tracer dans le repère ci-dessous les points dont l'abscisse est le débit moyen de téléchargement et l'ordonnée le temps correspondant.

\definecolor{cqcqcq}{rgb}{0.7529411764705882,0.7529411764705882,0.7529411764705882}
\begin{tikzpicture}[line cap=round,line join=round,>=triangle 45,x=0.883779264214047cm,y=0.497340425531907cm]
\draw [color=cqcqcq,, xstep=0.883779264214047cm,ystep=0.994680851063814cm] (-0.666982024597919,-1.5294117647059702) grid (16.30558183538316,20.58823529411791);
\draw[->,color=black] (-0.666982024597919,0.) -- (16.30558183538316,0.);
\foreach \x in {,1.,2.,3.,4.,5.,6.,7.,8.,9.,10.,11.,12.,13.,14.,15.,16.}
\draw[shift={(\x,0)},color=black] (0pt,2pt) -- (0pt,-2pt) node[below] {\footnotesize $\x$};
\draw[->,color=black] (0.,-1.5294117647059702) -- (0.,20.58823529411791);
\foreach \y in {,2.,4.,6.,8.,10.,12.,14.,16.,18.,20.}
\draw[shift={(0,\y)},color=black] (2pt,0pt) -- (-2pt,0pt) node[left] {\footnotesize $\y$};
\draw[color=black] (0pt,-10pt) node[right] {\footnotesize $0$};
\clip(-0.666982024597919,-1.5294117647059702) rectangle (16.30558183538316,20.58823529411791);
\end{tikzpicture}

\item Quel est la forme de cette représentation graphique ?


\item Quel est le temps nécessaire pour télécharger 9 Mo ?


\end{enumerate}


\end{enumerate}

\end{ExoCad}

\begin{ExoCad}{Calculer.}{1234}{0}{0}{0}{0}{0}
 
Pour réaliser un bon sirop d'orgeat, il faut 1 volume de sirop pour 7 volumes d'eau.
\begin{enumerate}
\item Quelle est le volume d'eau pour 2 cl de sirop pour avoir un bon sirop d'orgeat ?
\item Quelle est le volume de sirop pour 28 cl d'eau ?
\item Exprime le volume d'eau $V$ en fonction du volume $v$ de sirop.
\end{enumerate}
\end{ExoCad}

\begin{ExoCad}{Calculer.}{1234}{0}{0}{0}{0}{0}
 
Un rectangle a pour dimension $x$ cm de longueur sur $x-5$ cm de largeur. 
\begin{enumerate}
\item Quelle est la valeur la plus petite pour $x$ ?
\item Déterminer l'aire de ce rectangle en fonction de $x$. 
\end{enumerate}
\end{ExoCad}

\begin{ExoCad}{Calculer.}{1234}{0}{0}{0}{0}{0}
 
Le débit d'une connexion internet varie en fonction de la distance du modem par rapport au central téléphonique le plus proche.
 
On a représenté ci-dessous la fonction qui, à la distance du modem au central téléphonique (en kilomètres), associe son débit théorique (en mégabits par seconde).

\begin{center}
\definecolor{ffqqqq}{rgb}{1.,0.,0.}
\definecolor{ccqqqq}{rgb}{0.8,0.,0.}
\definecolor{cqcqcq}{rgb}{0.7529411764705882,0.7529411764705882,0.7529411764705882}
\begin{tikzpicture}[line cap=round,line join=round,>=triangle 45,x=0.9640102827763489cm,y=0.28851540616246374cm]
\draw [color=cqcqcq,, xstep=0.9640102827763489cm,ystep=1.4425770308123187cm] (-0.86,-6.699029126213654) grid (14.7,27.961165048543776);
\draw[->,color=black] (-0.86,0.) -- (14.7,0.);
\foreach \x in {,1.,2.,3.,4.,5.,6.,7.,8.,9.,10.,11.,12.,13.,14.}
\draw[shift={(\x,0)},color=black] (0pt,2pt) -- (0pt,-2pt) node[below] {\footnotesize $\x$};
\draw[->,color=black] (0.,-6.699029126213654) -- (0.,27.961165048543776);
\foreach \y in {-5.,5.,10.,15.,20.,25.}
\draw[shift={(0,\y)},color=black] (2pt,0pt) -- (-2pt,0pt) node[left] {\footnotesize $\y$};
\draw[color=black] (0pt,-10pt) node[right] {\footnotesize $0$};
\clip(-0.86,-6.699029126213654) rectangle (14.7,27.961165048543776);
\draw (11.22,-2.427184466019461) node[anchor=north west] {distance (en km)};
\draw (0.1,26.89320388349523) node[anchor=north west] {débit (en Mbit/s)};
\draw[line width=1.2pt,color=ccqqqq] (3.980000084400006,9.611064014960892) -- (3.980000084400006,9.611064014960892);
\draw[line width=1.2pt,color=ccqqqq] (3.980000084400006,9.611064014960892) -- (4.005000084179178,9.295416750131462);
\draw[line width=1.2pt,color=ccqqqq] (4.005000084179178,9.295416750131462) -- (4.03000008395835,9.164431077570768);
\draw[line width=1.2pt,color=ccqqqq] (4.03000008395835,9.164431077570768) -- (4.0550000837375215,9.063922185272446);
\draw[line width=1.2pt,color=ccqqqq] (4.0550000837375215,9.063922185272446) -- (4.080000083516693,8.979189252407128);
\draw[line width=1.2pt,color=ccqqqq] (4.080000083516693,8.979189252407128) -- (4.105000083295865,8.904538031760932);
\draw[line width=1.2pt,color=ccqqqq] (4.105000083295865,8.904538031760932) -- (4.1300000830750365,8.837048164803448);
\draw[line width=1.2pt,color=ccqqqq] (4.1300000830750365,8.837048164803448) -- (4.155000082854208,8.77498482395023);
\draw[line width=1.2pt,color=ccqqqq] (4.155000082854208,8.77498482395023) -- (4.18000008263338,8.717217672769973);
\draw[line width=1.2pt,color=ccqqqq] (4.18000008263338,8.717217672769973) -- (4.2050000824125515,8.662961576752313);
\draw[line width=1.2pt,color=ccqqqq] (4.2050000824125515,8.662961576752313) -- (4.230000082191723,8.611644884160292);
\draw[line width=1.2pt,color=ccqqqq] (4.230000082191723,8.611644884160292) -- (4.255000081970895,8.562836044061221);
\draw[line width=1.2pt,color=ccqqqq] (4.255000081970895,8.562836044061221) -- (4.2800000817500665,8.516199784278882);
\draw[line width=1.2pt,color=ccqqqq] (4.2800000817500665,8.516199784278882) -- (4.305000081529238,8.471469480432848);
\draw[line width=1.2pt,color=ccqqqq] (4.305000081529238,8.471469480432848) -- (4.33000008130841,8.428428954487512);
\draw[line width=1.2pt,color=ccqqqq] (4.33000008130841,8.428428954487512) -- (4.3550000810875815,8.386900044736674);
\draw[line width=1.2pt,color=ccqqqq] (4.3550000810875815,8.386900044736674) -- (4.380000080866753,8.346733856614815);
\draw[line width=1.2pt,color=ccqqqq] (4.380000080866753,8.346733856614815) -- (4.405000080645925,8.307804443797998);
\draw[line width=1.2pt,color=ccqqqq] (4.405000080645925,8.307804443797998) -- (4.4300000804250965,8.270004142153198);
\draw[line width=1.2pt,color=ccqqqq] (4.4300000804250965,8.270004142153198) -- (4.455000080204268,8.23324005696212);
\draw[line width=1.2pt,color=ccqqqq] (4.455000080204268,8.23324005696212) -- (4.48000007998344,8.197431373056968);
\draw[line width=1.2pt,color=ccqqqq] (4.48000007998344,8.197431373056968) -- (4.5050000797626115,8.16250726384192);
\draw[line width=1.2pt,color=ccqqqq] (4.5050000797626115,8.16250726384192) -- (4.530000079541783,8.128405243870475);
\draw[line width=1.2pt,color=ccqqqq] (4.530000079541783,8.128405243870475) -- (4.555000079320955,8.095069855128038);
\draw[line width=1.2pt,color=ccqqqq] (4.555000079320955,8.095069855128038) -- (4.5800000791001265,8.062451607942936);
\draw[line width=1.2pt,color=ccqqqq] (4.5800000791001265,8.062451607942936) -- (4.605000078879298,8.03050611868424);
\draw[line width=1.2pt,color=ccqqqq] (4.605000078879298,8.03050611868424) -- (4.63000007865847,7.999193401320192);
\draw[line width=1.2pt,color=ccqqqq] (4.63000007865847,7.999193401320192) -- (4.6550000784376415,7.968477280556971);
\draw[line width=1.2pt,color=ccqqqq] (4.6550000784376415,7.968477280556971) -- (4.680000078216813,7.938324901988602);
\draw[line width=1.2pt,color=ccqqqq] (4.680000078216813,7.938324901988602) -- (4.705000077995985,7.908706320349446);
\draw[line width=1.2pt,color=ccqqqq] (4.705000077995985,7.908706320349446) -- (4.7300000777751565,7.879594151167834);
\draw[line width=1.2pt,color=ccqqqq] (4.7300000777751565,7.879594151167834) -- (4.755000077554328,7.8509632742820274);
\draw[line width=1.2pt,color=ccqqqq] (4.755000077554328,7.8509632742820274) -- (4.7800000773335,7.822790580082413);
\draw[line width=1.2pt,color=ccqqqq] (4.7800000773335,7.822790580082413) -- (4.8050000771126715,7.795054751186973);
\draw[line width=1.2pt,color=ccqqqq] (4.8050000771126715,7.795054751186973) -- (4.830000076891843,7.767736073684234);
\draw[line width=1.2pt,color=ccqqqq] (4.830000076891843,7.767736073684234) -- (4.855000076671015,7.740816273191991);
\draw[line width=1.2pt,color=ccqqqq] (4.855000076671015,7.740816273191991) -- (4.8800000764501865,7.714278371857126);
\draw[line width=1.2pt,color=ccqqqq] (4.8800000764501865,7.714278371857126) -- (4.905000076229358,7.688106563117079);
\draw[line width=1.2pt,color=ccqqqq] (4.905000076229358,7.688106563117079) -- (4.93000007600853,7.662286101598826);
\draw[line width=1.2pt,color=ccqqqq] (4.93000007600853,7.662286101598826) -- (4.9550000757877015,7.636803205977386);
\draw[line width=1.2pt,color=ccqqqq] (4.9550000757877015,7.636803205977386) -- (4.980000075566873,7.611644972976852);
\draw[line width=1.2pt,color=ccqqqq] (4.980000075566873,7.611644972976852) -- (5.005000075346045,7.586799300990548);
\draw[line width=1.2pt,color=ccqqqq] (5.005000075346045,7.586799300990548) -- (5.0300000751252165,7.562254822037102);
\draw[line width=1.2pt,color=ccqqqq] (5.0300000751252165,7.562254822037102) -- (5.055000074904388,7.538000840966746);
\draw[line width=1.2pt,color=ccqqqq] (5.055000074904388,7.538000840966746) -- (5.08000007468356,7.514027280995442);
\draw[line width=1.2pt,color=ccqqqq] (5.08000007468356,7.514027280995442) -- (5.1050000744627315,7.490324634779946);
\draw[line width=1.2pt,color=ccqqqq] (5.1050000744627315,7.490324634779946) -- (5.130000074241903,7.466883920360071);
\draw[line width=1.2pt,color=ccqqqq] (5.130000074241903,7.466883920360071) -- (5.155000074021075,7.443696641389093);
\draw[line width=1.2pt,color=ccqqqq] (5.155000074021075,7.443696641389093) -- (5.1800000738002465,7.42075475115296);
\draw[line width=1.2pt,color=ccqqqq] (5.1800000738002465,7.42075475115296) -- (5.205000073579418,7.3980506199462726);
\draw[line width=1.2pt,color=ccqqqq] (5.205000073579418,7.3980506199462726) -- (5.23000007335859,7.375577005430018);
\draw[line width=1.2pt,color=ccqqqq] (5.23000007335859,7.375577005430018) -- (5.2550000731377615,7.3533270256445915);
\draw[line width=1.2pt,color=ccqqqq] (5.2550000731377615,7.3533270256445915) -- (5.280000072916933,7.331294134393068);
\draw[line width=1.2pt,color=ccqqqq] (5.280000072916933,7.331294134393068) -- (5.305000072696105,7.309472098745116);
\draw[line width=1.2pt,color=ccqqqq] (5.305000072696105,7.309472098745116) -- (5.3300000724752765,7.287854978442488);
\draw[line width=1.2pt,color=ccqqqq] (5.3300000724752765,7.287854978442488) -- (5.355000072254448,7.26643710701321);
\draw[line width=1.2pt,color=ccqqqq] (5.355000072254448,7.26643710701321) -- (5.38000007203362,7.245213074424357);
\draw[line width=1.2pt,color=ccqqqq] (5.38000007203362,7.245213074424357) -- (5.4050000718127915,7.224177711122924);
\draw[line width=1.2pt,color=ccqqqq] (5.4050000718127915,7.224177711122924) -- (5.430000071591963,7.203326073331376);
\draw[line width=1.2pt,color=ccqqqq] (5.430000071591963,7.203326073331376) -- (5.455000071371135,7.182653429479438);
\draw[line width=1.2pt,color=ccqqqq] (5.455000071371135,7.182653429479438) -- (5.4800000711503065,7.162155247666564);
\draw[line width=1.2pt,color=ccqqqq] (5.4800000711503065,7.162155247666564) -- (5.505000070929478,7.141827184061017);
\draw[line width=1.2pt,color=ccqqqq] (5.505000070929478,7.141827184061017) -- (5.53000007070865,7.121665072151424);
\draw[line width=1.2pt,color=ccqqqq] (5.53000007070865,7.121665072151424) -- (5.5550000704878215,7.101664912775457);
\draw[line width=1.2pt,color=ccqqqq] (5.5550000704878215,7.101664912775457) -- (5.580000070266993,7.081822864858085);
\draw[line width=1.2pt,color=ccqqqq] (5.580000070266993,7.081822864858085) -- (5.605000070046165,7.062135236798599);
\draw[line width=1.2pt,color=ccqqqq] (5.605000070046165,7.062135236798599) -- (5.6300000698253365,7.042598478451748);
\draw[line width=1.2pt,color=ccqqqq] (5.6300000698253365,7.042598478451748) -- (5.655000069604508,7.023209173653632);
\draw[line width=1.2pt,color=ccqqqq] (5.655000069604508,7.023209173653632) -- (5.68000006938368,7.003964033247812);
\draw[line width=1.2pt,color=ccqqqq] (5.68000006938368,7.003964033247812) -- (5.7050000691628515,6.9848598885712905);
\draw[line width=1.2pt,color=ccqqqq] (5.7050000691628515,6.9848598885712905) -- (5.730000068942023,6.965893685363862);
\draw[line width=1.2pt,color=ccqqqq] (5.730000068942023,6.965893685363862) -- (5.755000068721195,6.947062478067669);
\draw[line width=1.2pt,color=ccqqqq] (5.755000068721195,6.947062478067669) -- (5.7800000685003665,6.928363424486817);
\draw[line width=1.2pt,color=ccqqqq] (5.7800000685003665,6.928363424486817) -- (5.805000068279538,6.909793780779684);
\draw[line width=1.2pt,color=ccqqqq] (5.805000068279538,6.909793780779684) -- (5.83000006805871,6.891350896758877);
\draw[line width=1.2pt,color=ccqqqq] (5.83000006805871,6.891350896758877) -- (5.8550000678378815,6.873032211476109);
\draw[line width=1.2pt,color=ccqqqq] (5.8550000678378815,6.873032211476109) -- (5.880000067617053,6.854835249071113);
\draw[line width=1.2pt,color=ccqqqq] (5.880000067617053,6.854835249071113) -- (5.905000067396225,6.836757614865575);
\draw[line width=1.2pt,color=ccqqqq] (5.905000067396225,6.836757614865575) -- (5.9300000671753965,6.818796991684635);
\draw[line width=1.2pt,color=ccqqqq] (5.9300000671753965,6.818796991684635) -- (5.955000066954568,6.800951136389952);
\draw[line width=1.2pt,color=ccqqqq] (5.955000066954568,6.800951136389952) -- (5.98000006673374,6.783217876609654);
\draw[line width=1.2pt,color=ccqqqq] (5.98000006673374,6.783217876609654) -- (6.0050000665129115,6.765595107651673);
\draw[line width=1.2pt,color=ccqqqq] (6.0050000665129115,6.765595107651673) -- (6.030000066292083,6.748080789588054);
\draw[line width=1.2pt,color=ccqqqq] (6.030000066292083,6.748080789588054) -- (6.055000066071255,6.730672944498802);
\draw[line width=1.2pt,color=ccqqqq] (6.055000066071255,6.730672944498802) -- (6.0800000658504265,6.713369653864725);
\draw[line width=1.2pt,color=ccqqqq] (6.0800000658504265,6.713369653864725) -- (6.105000065629598,6.696169056099541);
\draw[line width=1.2pt,color=ccqqqq] (6.105000065629598,6.696169056099541) -- (6.13000006540877,6.6790693442122775);
\draw[line width=1.2pt,color=ccqqqq] (6.13000006540877,6.6790693442122775) -- (6.1550000651879415,6.662068763591635);
\draw[line width=1.2pt,color=ccqqqq] (6.1550000651879415,6.662068763591635) -- (6.180000064967113,6.64516560990464);
\draw[line width=1.2pt,color=ccqqqq] (6.180000064967113,6.64516560990464) -- (6.205000064746285,6.628358227102457);
\draw[line width=1.2pt,color=ccqqqq] (6.205000064746285,6.628358227102457) -- (6.2300000645254565,6.611645005526753);
\draw[line width=1.2pt,color=ccqqqq] (6.2300000645254565,6.611645005526753) -- (6.255000064304628,6.595024380110499);
\draw[line width=1.2pt,color=ccqqqq] (6.255000064304628,6.595024380110499) -- (6.2800000640838,6.578494828667497);
\draw[line width=1.2pt,color=ccqqqq] (6.2800000640838,6.578494828667497) -- (6.3050000638629715,6.562054870265356);
\draw[line width=1.2pt,color=ccqqqq] (6.3050000638629715,6.562054870265356) -- (6.330000063642143,6.5457030636769895);
\draw[line width=1.2pt,color=ccqqqq] (6.330000063642143,6.5457030636769895) -- (6.355000063421315,6.529438005906052);
\draw[line width=1.2pt,color=ccqqqq] (6.355000063421315,6.529438005906052) -- (6.3800000632004865,6.513258330782052);
\draw[line width=1.2pt,color=ccqqqq] (6.3800000632004865,6.513258330782052) -- (6.405000062979658,6.497162707621137);
\draw[line width=1.2pt,color=ccqqqq] (6.405000062979658,6.497162707621137) -- (6.43000006275883,6.4811498399488725);
\draw[line width=1.2pt,color=ccqqqq] (6.43000006275883,6.4811498399488725) -- (6.4550000625380015,6.465218464281507);
\draw[line width=1.2pt,color=ccqqqq] (6.4550000625380015,6.465218464281507) -- (6.480000062317173,6.449367348962503);
\draw[line width=1.2pt,color=ccqqqq] (6.480000062317173,6.449367348962503) -- (6.505000062096345,6.4335952930513125);
\draw[line width=1.2pt,color=ccqqqq] (6.505000062096345,6.4335952930513125) -- (6.5300000618755165,6.417901125261509);
\draw[line width=1.2pt,color=ccqqqq] (6.5300000618755165,6.417901125261509) -- (6.555000061654688,6.4022837029457);
\draw[line width=1.2pt,color=ccqqqq] (6.555000061654688,6.4022837029457) -- (6.58000006143386,6.386741911124642);
\draw[line width=1.2pt,color=ccqqqq] (6.58000006143386,6.386741911124642) -- (6.6050000612130315,6.37127466155829);
\draw[line width=1.2pt,color=ccqqqq] (6.6050000612130315,6.37127466155829) -- (6.630000060992203,6.355880891856569);
\draw[line width=1.2pt,color=ccqqqq] (6.630000060992203,6.355880891856569) -- (6.655000060771375,6.340559564627797);
\draw[line width=1.2pt,color=ccqqqq] (6.655000060771375,6.340559564627797) -- (6.6800000605505465,6.325309666662839);
\draw[line width=1.2pt,color=ccqqqq] (6.6800000605505465,6.325309666662839) -- (6.705000060329718,6.310130208153198);
\draw[line width=1.2pt,color=ccqqqq] (6.705000060329718,6.310130208153198) -- (6.73000006010889,6.295020221941299);
\draw[line width=1.2pt,color=ccqqqq] (6.73000006010889,6.295020221941299) -- (6.7550000598880615,6.279978762801371);
\draw[line width=1.2pt,color=ccqqqq] (6.7550000598880615,6.279978762801371) -- (6.780000059667233,6.26500490674943);
\draw[line width=1.2pt,color=ccqqqq] (6.780000059667233,6.26500490674943) -- (6.805000059446405,6.250097750380925);
\draw[line width=1.2pt,color=ccqqqq] (6.805000059446405,6.250097750380925) -- (6.8300000592255765,6.235256410234693);
\draw[line width=1.2pt,color=ccqqqq] (6.8300000592255765,6.235256410234693) -- (6.855000059004748,6.220480022181987);
\draw[line width=1.2pt,color=ccqqqq] (6.855000059004748,6.220480022181987) -- (6.88000005878392,6.205767740839337);
\draw[line width=1.2pt,color=ccqqqq] (6.88000005878392,6.205767740839337) -- (6.9050000585630915,6.191118739004158);
\draw[line width=1.2pt,color=ccqqqq] (6.9050000585630915,6.191118739004158) -- (6.930000058342263,6.176532207112012);
\draw[line width=1.2pt,color=ccqqqq] (6.930000058342263,6.176532207112012) -- (6.955000058121435,6.162007352714531);
\draw[line width=1.2pt,color=ccqqqq] (6.955000058121435,6.162007352714531) -- (6.9800000579006065,6.147543399977038);
\draw[line width=1.2pt,color=ccqqqq] (6.9800000579006065,6.147543399977038) -- (7.005000057679778,6.133139589194966);
\draw[line width=1.2pt,color=ccqqqq] (7.005000057679778,6.133139589194966) -- (7.03000005745895,6.118795176328228);
\draw[line width=1.2pt,color=ccqqqq] (7.03000005745895,6.118795176328228) -- (7.0550000572381215,6.104509432552736);
\draw[line width=1.2pt,color=ccqqqq] (7.0550000572381215,6.104509432552736) -- (7.080000057017293,6.090281643828276);
\draw[line width=1.2pt,color=ccqqqq] (7.080000057017293,6.090281643828276) -- (7.105000056796465,6.076111110482054);
\draw[line width=1.2pt,color=ccqqqq] (7.105000056796465,6.076111110482054) -- (7.1300000565756365,6.061997146807195);
\draw[line width=1.2pt,color=ccqqqq] (7.1300000565756365,6.061997146807195) -- (7.155000056354808,6.04793908067555);
\draw[line width=1.2pt,color=ccqqqq] (7.155000056354808,6.04793908067555) -- (7.18000005613398,6.033936253164211);
\draw[line width=1.2pt,color=ccqqqq] (7.18000005613398,6.033936253164211) -- (7.2050000559131515,6.019988018195112);
\draw[line width=1.2pt,color=ccqqqq] (7.2050000559131515,6.019988018195112) -- (7.230000055692323,6.0060937421871925);
\draw[line width=1.2pt,color=ccqqqq] (7.230000055692323,6.0060937421871925) -- (7.255000055471495,5.992252803720571);
\draw[line width=1.2pt,color=ccqqqq] (7.255000055471495,5.992252803720571) -- (7.2800000552506665,5.978464593212242);
\draw[line width=1.2pt,color=ccqqqq] (7.2800000552506665,5.978464593212242) -- (7.305000055029838,5.964728512602799);
\draw[line width=1.2pt,color=ccqqqq] (7.305000055029838,5.964728512602799) -- (7.33000005480901,5.951043975053737);
\draw[line width=1.2pt,color=ccqqqq] (7.33000005480901,5.951043975053737) -- (7.3550000545881815,5.937410404654915);
\draw[line width=1.2pt,color=ccqqqq] (7.3550000545881815,5.937410404654915) -- (7.380000054367353,5.923827236141732);
\draw[line width=1.2pt,color=ccqqqq] (7.380000054367353,5.923827236141732) -- (7.405000054146525,5.910293914621666);
\draw[line width=1.2pt,color=ccqqqq] (7.405000054146525,5.910293914621666) -- (7.4300000539256965,5.8968098953097625);
\draw[line width=1.2pt,color=ccqqqq] (7.4300000539256965,5.8968098953097625) -- (7.455000053704868,5.883374643272739);
\draw[line width=1.2pt,color=ccqqqq] (7.455000053704868,5.883374643272739) -- (7.48000005348404,5.869987633181361);
\draw[line width=1.2pt,color=ccqqqq] (7.48000005348404,5.869987633181361) -- (7.5050000532632115,5.856648349070761);
\draw[line width=1.2pt,color=ccqqqq] (7.5050000532632115,5.856648349070761) -- (7.530000053042383,5.843356284108396);
\draw[line width=1.2pt,color=ccqqqq] (7.530000053042383,5.843356284108396) -- (7.555000052821555,5.830110940369342);
\draw[line width=1.2pt,color=ccqqqq] (7.555000052821555,5.830110940369342) -- (7.5800000526007265,5.816911828618651);
\draw[line width=1.2pt,color=ccqqqq] (7.5800000526007265,5.816911828618651) -- (7.605000052379898,5.8037584681005);
\draw[line width=1.2pt,color=ccqqqq] (7.605000052379898,5.8037584681005) -- (7.63000005215907,5.790650386333859);
\draw[line width=1.2pt,color=ccqqqq] (7.63000005215907,5.790650386333859) -- (7.6550000519382415,5.777587118914469);
\draw[line width=1.2pt,color=ccqqqq] (7.6550000519382415,5.777587118914469) -- (7.680000051717413,5.764568209322852);
\draw[line width=1.2pt,color=ccqqqq] (7.680000051717413,5.764568209322852) -- (7.705000051496585,5.751593208738158);
\draw[line width=1.2pt,color=ccqqqq] (7.705000051496585,5.751593208738158) -- (7.7300000512757565,5.73866167585762);
\draw[line width=1.2pt,color=ccqqqq] (7.7300000512757565,5.73866167585762) -- (7.755000051054928,5.725773176721428);
\draw[line width=1.2pt,color=ccqqqq] (7.755000051054928,5.725773176721428) -- (7.7800000508341,5.712927284542794);
\draw[line width=1.2pt,color=ccqqqq] (7.7800000508341,5.712927284542794) -- (7.8050000506132715,5.700123579543063);
\draw[line width=1.2pt,color=ccqqqq] (7.8050000506132715,5.700123579543063) -- (7.830000050392443,5.687361648791641);
\draw[line width=1.2pt,color=ccqqqq] (7.830000050392443,5.687361648791641) -- (7.855000050171615,5.674641086050612);
\draw[line width=1.2pt,color=ccqqqq] (7.855000050171615,5.674641086050612) -- (7.8800000499507865,5.6619614916238605);
\draw[line width=1.2pt,color=ccqqqq] (7.8800000499507865,5.6619614916238605) -- (7.905000049729958,5.649322472210515);
\draw[line width=1.2pt,color=ccqqqq] (7.905000049729958,5.649322472210515) -- (7.93000004950913,5.636723640762619);
\draw[line width=1.2pt,color=ccqqqq] (7.93000004950913,5.636723640762619) -- (7.9550000492883015,5.62416461634682);
\draw[line width=1.2pt,color=ccqqqq] (7.9550000492883015,5.62416461634682) -- (7.980000049067473,5.611645024009987);
\draw[line width=1.2pt,color=ccqqqq] (7.980000049067473,5.611645024009987) -- (8.005000048846645,5.599164494648593);
\draw[line width=1.2pt,color=ccqqqq] (8.005000048846645,5.599164494648593) -- (8.030000048625817,5.58672266488174);
\draw[line width=1.2pt,color=ccqqqq] (8.030000048625817,5.58672266488174) -- (8.05500004840499,5.574319176927717);
\draw[line width=1.2pt,color=ccqqqq] (8.05500004840499,5.574319176927717) -- (8.080000048184163,5.561953678483946);
\draw[line width=1.2pt,color=ccqqqq] (8.080000048184163,5.561953678483946) -- (8.105000047963335,5.54962582261023);
\draw[line width=1.2pt,color=ccqqqq] (8.105000047963335,5.54962582261023) -- (8.130000047742508,5.537335267615178);
\draw[line width=1.2pt,color=ccqqqq] (8.130000047742508,5.537335267615178) -- (8.15500004752168,5.525081676945686);
\draw[line width=1.2pt,color=ccqqqq] (8.15500004752168,5.525081676945686) -- (8.180000047300853,5.512864719079432);
\draw[line width=1.2pt,color=ccqqqq] (8.180000047300853,5.512864719079432) -- (8.205000047080025,5.500684067420198);
\draw[line width=1.2pt,color=ccqqqq] (8.205000047080025,5.500684067420198) -- (8.230000046859198,5.488539400196015);
\draw[line width=1.2pt,color=ccqqqq] (8.230000046859198,5.488539400196015) -- (8.25500004663837,5.476430400359971);
\draw[line width=1.2pt,color=ccqqqq] (8.25500004663837,5.476430400359971) -- (8.280000046417543,5.464356755493653);
\draw[line width=1.2pt,color=ccqqqq] (8.280000046417543,5.464356755493653) -- (8.305000046196716,5.452318157713086);
\draw[line width=1.2pt,color=ccqqqq] (8.305000046196716,5.452318157713086) -- (8.330000045975888,5.440314303577131);
\draw[line width=1.2pt,color=ccqqqq] (8.330000045975888,5.440314303577131) -- (8.35500004575506,5.4283448939982435);
\draw[line width=1.2pt,color=ccqqqq] (8.35500004575506,5.4283448939982435) -- (8.380000045534233,5.416409634155524);
\draw[line width=1.2pt,color=ccqqqq] (8.380000045534233,5.416409634155524) -- (8.405000045313406,5.40450823340999);
\draw[line width=1.2pt,color=ccqqqq] (8.405000045313406,5.40450823340999) -- (8.430000045092578,5.392640405221995);
\draw[line width=1.2pt,color=ccqqqq] (8.430000045092578,5.392640405221995) -- (8.45500004487175,5.380805867070738);
\draw[line width=1.2pt,color=ccqqqq] (8.45500004487175,5.380805867070738) -- (8.480000044650923,5.369004340375792);
\draw[line width=1.2pt,color=ccqqqq] (8.480000044650923,5.369004340375792) -- (8.505000044430096,5.357235550420585);
\draw[line width=1.2pt,color=ccqqqq] (8.505000044430096,5.357235550420585) -- (8.530000044209268,5.345499226277791);
\draw[line width=1.2pt,color=ccqqqq] (8.530000044209268,5.345499226277791) -- (8.555000043988441,5.333795100736559);
\draw[line width=1.2pt,color=ccqqqq] (8.555000043988441,5.333795100736559) -- (8.580000043767614,5.322122910231526);
\draw[line width=1.2pt,color=ccqqqq] (8.580000043767614,5.322122910231526) -- (8.605000043546786,5.310482394773566);
\draw[line width=1.2pt,color=ccqqqq] (8.605000043546786,5.310482394773566) -- (8.630000043325959,5.298873297882227);
\draw[line width=1.2pt,color=ccqqqq] (8.630000043325959,5.298873297882227) -- (8.655000043105131,5.287295366519787);
\draw[line width=1.2pt,color=ccqqqq] (8.655000043105131,5.287295366519787) -- (8.680000042884304,5.27574835102691);
\draw[line width=1.2pt,color=ccqqqq] (8.680000042884304,5.27574835102691) -- (8.705000042663476,5.264232005059829);
\draw[line width=1.2pt,color=ccqqqq] (8.705000042663476,5.264232005059829) -- (8.730000042442649,5.2527460855290276);
\draw[line width=1.2pt,color=ccqqqq] (8.730000042442649,5.2527460855290276) -- (8.755000042221821,5.241290352539373);
\draw[line width=1.2pt,color=ccqqqq] (8.755000042221821,5.241290352539373) -- (8.780000042000994,5.229864569331652);
\draw[line width=1.2pt,color=ccqqqq] (8.780000042000994,5.229864569331652) -- (8.805000041780167,5.218468502225483);
\draw[line width=1.2pt,color=ccqqqq] (8.805000041780167,5.218468502225483) -- (8.83000004155934,5.207101920563552);
\draw[line width=1.2pt,color=ccqqqq] (8.83000004155934,5.207101920563552) -- (8.855000041338512,5.195764596657142);
\draw[line width=1.2pt,color=ccqqqq] (8.855000041338512,5.195764596657142) -- (8.880000041117684,5.184456305732917);
\draw[line width=1.2pt,color=ccqqqq] (8.880000041117684,5.184456305732917) -- (8.905000040896857,5.17317682588093);
\draw[line width=1.2pt,color=ccqqqq] (8.905000040896857,5.17317682588093) -- (8.93000004067603,5.161925938003811);
\draw[line width=1.2pt,color=ccqqqq] (8.93000004067603,5.161925938003811) -- (8.955000040455202,5.150703425767115);
\draw[line width=1.2pt,color=ccqqqq] (8.955000040455202,5.150703425767115) -- (8.980000040234374,5.1395090755507855);
\draw[line width=1.2pt,color=ccqqqq] (8.980000040234374,5.1395090755507855) -- (9.005000040013547,5.128342676401712);
\draw[line width=1.2pt,color=ccqqqq] (9.005000040013547,5.128342676401712) -- (9.03000003979272,5.11720401998735);
\draw[line width=1.2pt,color=ccqqqq] (9.03000003979272,5.11720401998735) -- (9.055000039571892,5.106092900550365);
\draw[line width=1.2pt,color=ccqqqq] (9.055000039571892,5.106092900550365) -- (9.080000039351065,5.095009114864294);
\draw[line width=1.2pt,color=ccqqqq] (9.080000039351065,5.095009114864294) -- (9.105000039130237,5.083952462190169);
\draw[line width=1.2pt,color=ccqqqq] (9.105000039130237,5.083952462190169) -- (9.13000003890941,5.072922744234103);
\draw[line width=1.2pt,color=ccqqqq] (9.13000003890941,5.072922744234103) -- (9.155000038688582,5.061919765105797);
\draw[line width=1.2pt,color=ccqqqq] (9.155000038688582,5.061919765105797) -- (9.180000038467755,5.050943331277946);
\draw[line width=1.2pt,color=ccqqqq] (9.180000038467755,5.050943331277946) -- (9.205000038246927,5.039993251546526);
\draw[line width=1.2pt,color=ccqqqq] (9.205000038246927,5.039993251546526) -- (9.2300000380261,5.029069336991934);
\draw[line width=1.2pt,color=ccqqqq] (9.2300000380261,5.029069336991934) -- (9.255000037805273,5.018171400940955);
\draw[line width=1.2pt,color=ccqqqq] (9.255000037805273,5.018171400940955) -- (9.280000037584445,5.0072992589295495);
\draw[line width=1.2pt,color=ccqqqq] (9.280000037584445,5.0072992589295495) -- (9.305000037363618,4.9964527286664175);
\draw[line width=1.2pt,color=ccqqqq] (9.305000037363618,4.9964527286664175) -- (9.33000003714279,4.985631629997345);
\draw[line width=1.2pt,color=ccqqqq] (9.33000003714279,4.985631629997345) -- (9.355000036921963,4.974835784870281);
\draw[line width=1.2pt,color=ccqqqq] (9.355000036921963,4.974835784870281) -- (9.380000036701135,4.96406501730117);
\draw[line width=1.2pt,color=ccqqqq] (9.380000036701135,4.96406501730117) -- (9.405000036480308,4.953319153340466);
\draw[line width=1.2pt,color=ccqqqq] (9.405000036480308,4.953319153340466) -- (9.43000003625948,4.942598021040368);
\draw[line width=1.2pt,color=ccqqqq] (9.43000003625948,4.942598021040368) -- (9.455000036038653,4.9319014504227106);
\draw[line width=1.2pt,color=ccqqqq] (9.455000036038653,4.9319014504227106) -- (9.480000035817826,4.921229273447523);
\draw[line width=1.2pt,color=ccqqqq] (9.480000035817826,4.921229273447523) -- (9.505000035596998,4.910581323982229);
\draw[line width=1.2pt,color=ccqqqq] (9.505000035596998,4.910581323982229) -- (9.53000003537617,4.899957437771477);
\draw[line width=1.2pt,color=ccqqqq] (9.53000003537617,4.899957437771477) -- (9.555000035155343,4.8893574524075705);
\draw[line width=1.2pt,color=ccqqqq] (9.555000035155343,4.8893574524075705) -- (9.580000034934516,4.878781207301504);
\draw[line width=1.2pt,color=ccqqqq] (9.580000034934516,4.878781207301504) -- (9.605000034713688,4.868228543654579);
\draw[line width=1.2pt,color=ccqqqq] (9.605000034713688,4.868228543654579) -- (9.630000034492861,4.857699304430583);
\draw[line width=1.2pt,color=ccqqqq] (9.630000034492861,4.857699304430583) -- (9.655000034272033,4.8471933343285265);
\draw[line width=1.2pt,color=ccqqqq] (9.655000034272033,4.8471933343285265) -- (9.680000034051206,4.836710479755915);
\draw[line width=1.2pt,color=ccqqqq] (9.680000034051206,4.836710479755915) -- (9.705000033830379,4.826250588802551);
\draw[line width=1.2pt,color=ccqqqq] (9.705000033830379,4.826250588802551) -- (9.730000033609551,4.815813511214853);
\draw[line width=1.2pt,color=ccqqqq] (9.730000033609551,4.815813511214853) -- (9.755000033388724,4.8053990983706685);
\draw[line width=1.2pt,color=ccqqqq] (9.755000033388724,4.8053990983706685) -- (9.780000033167896,4.7950072032545865);
\draw[line width=1.2pt,color=ccqqqq] (9.780000033167896,4.7950072032545865) -- (9.805000032947069,4.784637680433719);
\draw[line width=1.2pt,color=ccqqqq] (9.805000032947069,4.784637680433719) -- (9.830000032726241,4.774290386033959);
\draw[line width=1.2pt,color=ccqqqq] (9.830000032726241,4.774290386033959) -- (9.855000032505414,4.763965177716686);
\draw[line width=1.2pt,color=ccqqqq] (9.855000032505414,4.763965177716686) -- (9.880000032284586,4.753661914655923);
\draw[line width=1.2pt,color=ccqqqq] (9.880000032284586,4.753661914655923) -- (9.905000032063759,4.7433804575159275);
\draw[line width=1.2pt,color=ccqqqq] (9.905000032063759,4.7433804575159275) -- (9.930000031842932,4.7331206684292075);
\draw[line width=1.2pt,color=ccqqqq] (9.930000031842932,4.7331206684292075) -- (9.955000031622104,4.722882410974948);
\draw[line width=1.2pt,color=ccqqqq] (9.955000031622104,4.722882410974948) -- (9.980000031401277,4.71266555015785);
\draw[line width=1.2pt,color=ccqqqq] (9.980000031401277,4.71266555015785) -- (10.00500003118045,4.702469952387365);
\draw[line width=1.2pt,color=ccqqqq] (10.00500003118045,4.702469952387365) -- (10.030000030959622,4.6922954854573105);
\draw[line width=1.2pt,color=ccqqqq] (10.030000030959622,4.6922954854573105) -- (10.055000030738794,4.682142018525873);
\draw[line width=1.2pt,color=ccqqqq] (10.055000030738794,4.682142018525873) -- (10.080000030517967,4.6720094220959725);
\draw[line width=1.2pt,color=ccqqqq] (10.080000030517967,4.6720094220959725) -- (10.10500003029714,4.661897567995998);
\draw[line width=1.2pt,color=ccqqqq] (10.10500003029714,4.661897567995998) -- (10.130000030076312,4.651806329360887);
\draw[line width=1.2pt,color=ccqqqq] (10.130000030076312,4.651806329360887) -- (10.155000029855485,4.641735580613555);
\draw[line width=1.2pt,color=ccqqqq] (10.155000029855485,4.641735580613555) -- (10.180000029634657,4.631685197446666);
\draw[line width=1.2pt,color=ccqqqq] (10.180000029634657,4.631685197446666) -- (10.20500002941383,4.621655056804731);
\draw[line width=1.2pt,color=ccqqqq] (10.20500002941383,4.621655056804731) -- (10.230000029193002,4.611645036866523);
\draw[line width=1.2pt,color=ccqqqq] (10.230000029193002,4.611645036866523) -- (10.255000028972175,4.601655017027823);
\draw[line width=1.2pt,color=ccqqqq] (10.255000028972175,4.601655017027823) -- (10.280000028751347,4.59168487788446);
\draw[line width=1.2pt,color=ccqqqq] (10.280000028751347,4.59168487788446) -- (10.30500002853052,4.581734501215663);
\draw[line width=1.2pt,color=ccqqqq] (10.30500002853052,4.581734501215663) -- (10.330000028309692,4.571803769967704);
\draw[line width=1.2pt,color=ccqqqq] (10.330000028309692,4.571803769967704) -- (10.355000028088865,4.561892568237837);
\draw[line width=1.2pt,color=ccqqqq] (10.355000028088865,4.561892568237837) -- (10.380000027868038,4.552000781258508);
\draw[line width=1.2pt,color=ccqqqq] (10.380000027868038,4.552000781258508) -- (10.40500002764721,4.542128295381857);
\draw[line width=1.2pt,color=ccqqqq] (10.40500002764721,4.542128295381857) -- (10.430000027426383,4.532274998064479);
\draw[line width=1.2pt,color=ccqqqq] (10.430000027426383,4.532274998064479) -- (10.455000027205555,4.5224407778524585);
\draw[line width=1.2pt,color=ccqqqq] (10.455000027205555,4.5224407778524585) -- (10.480000026984728,4.512625524366658);
\draw[line width=1.2pt,color=ccqqqq] (10.480000026984728,4.512625524366658) -- (10.5050000267639,4.502829128288268);
\draw[line width=1.2pt,color=ccqqqq] (10.5050000267639,4.502829128288268) -- (10.530000026543073,4.4930514813445965);
\draw[line width=1.2pt,color=ccqqqq] (10.530000026543073,4.4930514813445965) -- (10.555000026322245,4.48329247629511);
\draw[line width=1.2pt,color=ccqqqq] (10.555000026322245,4.48329247629511) -- (10.580000026101418,4.47355200691771);
\draw[line width=1.2pt,color=ccqqqq] (10.580000026101418,4.47355200691771) -- (10.60500002588059,4.463829967995244);
\draw[line width=1.2pt,color=ccqqqq] (10.60500002588059,4.463829967995244) -- (10.630000025659763,4.454126255302244);
\draw[line width=1.2pt,color=ccqqqq] (10.630000025659763,4.454126255302244) -- (10.655000025438936,4.444440765591893);
\draw[line width=1.2pt,color=ccqqqq] (10.655000025438936,4.444440765591893) -- (10.680000025218108,4.434773396583205);
\draw[line width=1.2pt,color=ccqqqq] (10.680000025218108,4.434773396583205) -- (10.70500002499728,4.425124046948423);
\draw[line width=1.2pt,color=ccqqqq] (10.70500002499728,4.425124046948423) -- (10.730000024776453,4.4154926163006305);
\draw[line width=1.2pt,color=ccqqqq] (10.730000024776453,4.4154926163006305) -- (10.755000024555626,4.405879005181566);
\draw[line width=1.2pt,color=ccqqqq] (10.755000024555626,4.405879005181566) -- (10.780000024334798,4.396283115049636);
\draw[line width=1.2pt,color=ccqqqq] (10.780000024334798,4.396283115049636) -- (10.805000024113971,4.3867048482681374);
\draw[line width=1.2pt,color=ccqqqq] (10.805000024113971,4.3867048482681374) -- (10.830000023893144,4.377144108093663);
\draw[line width=1.2pt,color=ccqqqq] (10.830000023893144,4.377144108093663) -- (10.855000023672316,4.367600798664699);
\draw[line width=1.2pt,color=ccqqqq] (10.855000023672316,4.367600798664699) -- (10.880000023451489,4.358074824990416);
\draw[line width=1.2pt,color=ccqqqq] (10.880000023451489,4.358074824990416) -- (10.905000023230661,4.348566092939634);
\draw[line width=1.2pt,color=ccqqqq] (10.905000023230661,4.348566092939634) -- (10.930000023009834,4.339074509229969);
\draw[line width=1.2pt,color=ccqqqq] (10.930000023009834,4.339074509229969) -- (10.955000022789006,4.329599981417164);
\draw[line width=1.2pt,color=ccqqqq] (10.955000022789006,4.329599981417164) -- (10.980000022568179,4.320142417884573);
\draw[line width=1.2pt,color=ccqqqq] (10.980000022568179,4.320142417884573) -- (11.005000022347351,4.310701727832833);
\draw[line width=1.2pt,color=ccqqqq] (11.005000022347351,4.310701727832833) -- (11.030000022126524,4.301277821269691);
\draw[line width=1.2pt,color=ccqqqq] (11.030000022126524,4.301277821269691) -- (11.055000021905697,4.291870608999996);
\draw[line width=1.2pt,color=ccqqqq] (11.055000021905697,4.291870608999996) -- (11.080000021684869,4.282480002615847);
\draw[line width=1.2pt,color=ccqqqq] (11.080000021684869,4.282480002615847) -- (11.105000021464042,4.2731059144869015);
\draw[line width=1.2pt,color=ccqqqq] (11.105000021464042,4.2731059144869015) -- (11.130000021243214,4.263748257750837);
\draw[line width=1.2pt,color=ccqqqq] (11.130000021243214,4.263748257750837) -- (11.155000021022387,4.254406946303956);
\draw[line width=1.2pt,color=ccqqqq] (11.155000021022387,4.254406946303956) -- (11.18000002080156,4.245081894791945);
\draw[line width=1.2pt,color=ccqqqq] (11.18000002080156,4.245081894791945) -- (11.205000020580732,4.235773018600774);
\draw[line width=1.2pt,color=ccqqqq] (11.205000020580732,4.235773018600774) -- (11.230000020359904,4.226480233847742);
\draw[line width=1.2pt,color=ccqqqq] (11.230000020359904,4.226480233847742) -- (11.255000020139077,4.2172034573726505);
\draw[line width=1.2pt,color=ccqqqq] (11.255000020139077,4.2172034573726505) -- (11.28000001991825,4.207942606729131);
\draw[line width=1.2pt,color=ccqqqq] (11.28000001991825,4.207942606729131) -- (11.305000019697422,4.198697600176089);
\draw[line width=1.2pt,color=ccqqqq] (11.305000019697422,4.198697600176089) -- (11.330000019476595,4.18946835666929);
\draw[line width=1.2pt,color=ccqqqq] (11.330000019476595,4.18946835666929) -- (11.355000019255767,4.180254795853068);
\draw[line width=1.2pt,color=ccqqqq] (11.355000019255767,4.180254795853068) -- (11.38000001903494,4.171056838052164);
\draw[line width=1.2pt,color=ccqqqq] (11.38000001903494,4.171056838052164) -- (11.405000018814112,4.1618744042636875);
\draw[line width=1.2pt,color=ccqqqq] (11.405000018814112,4.1618744042636875) -- (11.430000018593285,4.152707416149199);
\draw[line width=1.2pt,color=ccqqqq] (11.430000018593285,4.152707416149199) -- (11.455000018372457,4.143555796026916);
\draw[line width=1.2pt,color=ccqqqq] (11.455000018372457,4.143555796026916) -- (11.48000001815163,4.134419466864025);
\draw[line width=1.2pt,color=ccqqqq] (11.48000001815163,4.134419466864025) -- (11.505000017930803,4.125298352269125);
\draw[line width=1.2pt,color=ccqqqq] (11.505000017930803,4.125298352269125) -- (11.530000017709975,4.116192376484769);
\draw[line width=1.2pt,color=ccqqqq] (11.530000017709975,4.116192376484769) -- (11.555000017489148,4.10710146438013);
\draw[line width=1.2pt,color=ccqqqq] (11.555000017489148,4.10710146438013) -- (11.58000001726832,4.098025541443757);
\draw[line width=1.2pt,color=ccqqqq] (11.58000001726832,4.098025541443757) -- (11.605000017047493,4.088964533776464);
\draw[line width=1.2pt,color=ccqqqq] (11.605000017047493,4.088964533776464) -- (11.630000016826665,4.079918368084298);
\draw[line width=1.2pt,color=ccqqqq] (11.630000016826665,4.079918368084298) -- (11.655000016605838,4.070886971671629);
\draw[line width=1.2pt,color=ccqqqq] (11.655000016605838,4.070886971671629) -- (11.68000001638501,4.061870272434333);
\draw[line width=1.2pt,color=ccqqqq] (11.68000001638501,4.061870272434333) -- (11.705000016164183,4.052868198853072);
\draw[line width=1.2pt,color=ccqqqq] (11.705000016164183,4.052868198853072) -- (11.730000015943356,4.04388067998668);
\draw[line width=1.2pt,color=ccqqqq] (11.730000015943356,4.04388067998668) -- (11.755000015722528,4.034907645465638);
\draw[line width=1.2pt,color=ccqqqq] (11.755000015722528,4.034907645465638) -- (11.7800000155017,4.025949025485648);
\draw[line width=1.2pt,color=ccqqqq] (11.7800000155017,4.025949025485648) -- (11.805000015280873,4.017004750801295);
\draw[line width=1.2pt,color=ccqqqq] (11.805000015280873,4.017004750801295) -- (11.830000015060046,4.0080747527198035);
\draw[line width=1.2pt,color=ccqqqq] (11.830000015060046,4.0080747527198035) -- (11.855000014839218,3.9991589630948816);
\draw[line width=1.2pt,color=ccqqqq] (11.855000014839218,3.9991589630948816) -- (11.88000001461839,3.9902573143206554);
\draw[line width=1.2pt,color=ccqqqq] (11.88000001461839,3.9902573143206554) -- (11.905000014397563,3.9813697393256877);
\draw[line width=1.2pt,color=ccqqqq] (11.905000014397563,3.9813697393256877) -- (11.930000014176736,3.97249617156708);
\draw[line width=1.2pt,color=ccqqqq] (11.930000014176736,3.97249617156708) -- (11.955000013955908,3.963636545024662);
\draw[line width=1.2pt,color=ccqqqq] (11.955000013955908,3.963636545024662) -- (11.980000013735081,3.95479079419526);
\draw[line width=1.2pt,color=ccqqqq] (11.980000013735081,3.95479079419526) -- (12.005000013514254,3.9459588540870447);
\draw[line width=1.2pt,color=ccqqqq] (12.005000013514254,3.9459588540870447) -- (12.030000013293426,3.937140660213964);
\draw[line width=1.2pt,color=ccqqqq] (12.030000013293426,3.937140660213964) -- (12.055000013072599,3.9283361485902475);
\draw[line width=1.2pt,color=ccqqqq] (12.055000013072599,3.9283361485902475) -- (12.080000012851771,3.919545255724989);
\draw[line width=1.2pt,color=ccqqqq] (12.080000012851771,3.919545255724989) -- (12.105000012630944,3.9107679186168056);
\draw[line width=1.2pt,color=ccqqqq] (12.105000012630944,3.9107679186168056) -- (12.130000012410116,3.9020040747485707);
\draw[line width=1.2pt,color=ccqqqq] (12.130000012410116,3.9020040747485707) -- (12.155000012189289,3.893253662082217);
\draw[line width=1.2pt,color=ccqqqq] (12.155000012189289,3.893253662082217) -- (12.180000011968461,3.884516619053615);
\draw[line width=1.2pt,color=ccqqqq] (12.180000011968461,3.884516619053615) -- (12.205000011747634,3.8757928845675167);
\draw[line width=1.2pt,color=ccqqqq] (12.205000011747634,3.8757928845675167) -- (12.230000011526807,3.8670823979925766);
\draw[line width=1.2pt,color=ccqqqq] (12.230000011526807,3.8670823979925766) -- (12.25500001130598,3.8583850991564272);
\draw[line width=1.2pt,color=ccqqqq] (12.25500001130598,3.8583850991564272) -- (12.280000011085152,3.8497009283408383);
\draw[line width=1.2pt,color=ccqqqq] (12.280000011085152,3.8497009283408383) -- (12.305000010864324,3.841029826276925);
\draw[line width=1.2pt,color=ccqqqq] (12.305000010864324,3.841029826276925) -- (12.330000010643497,3.832371734140434);
\draw[line width=1.2pt,color=ccqqqq] (12.330000010643497,3.832371734140434) -- (12.35500001042267,3.823726593547085);
\draw[line width=1.2pt,color=ccqqqq] (12.35500001042267,3.823726593547085) -- (12.380000010201842,3.815094346547979);
\draw[line width=1.2pt,color=ccqqqq] (12.380000010201842,3.815094346547979) -- (12.405000009981014,3.8064749356250633);
\draw[line width=1.2pt,color=ccqqqq] (12.405000009981014,3.8064749356250633) -- (12.430000009760187,3.7978683036866645);
\draw[line width=1.2pt,color=ccqqqq] (12.430000009760187,3.7978683036866645) -- (12.45500000953936,3.7892743940630735);
\draw[line width=1.2pt,color=ccqqqq] (12.45500000953936,3.7892743940630735) -- (12.480000009318532,3.7806931505021932);
\draw[line width=1.2pt,color=ccqqqq] (12.480000009318532,3.7806931505021932) -- (12.505000009097705,3.772124517165243);
\draw[line width=1.2pt,color=ccqqqq] (12.505000009097705,3.772124517165243) -- (12.530000008876877,3.763568438622518);
\draw[line width=1.2pt,color=ccqqqq] (12.530000008876877,3.763568438622518) -- (12.55500000865605,3.7550248598492075);
\draw[line width=1.2pt,color=ccqqqq] (12.55500000865605,3.7550248598492075) -- (12.580000008435222,3.7464937262212654);
\draw[line width=1.2pt,color=ccqqqq] (12.580000008435222,3.7464937262212654) -- (12.605000008214395,3.737974983511336);
\draw[line width=1.2pt,color=ccqqqq] (12.605000008214395,3.737974983511336) -- (12.630000007993567,3.7294685778847327);
\draw[line width=1.2pt,color=ccqqqq] (12.630000007993567,3.7294685778847327) -- (12.65500000777274,3.720974455895468);
\draw[line width=1.2pt,color=ccqqqq] (12.65500000777274,3.720974455895468) -- (12.680000007551913,3.7124925644823357);
\draw[line width=1.2pt,color=ccqqqq] (12.680000007551913,3.7124925644823357) -- (12.705000007331085,3.7040228509650452);
\draw[line width=1.2pt,color=ccqqqq] (12.705000007331085,3.7040228509650452) -- (12.730000007110258,3.695565263040402);
\draw[line width=1.2pt,color=ccqqqq] (12.730000007110258,3.695565263040402) -- (12.75500000688943,3.6871197487785423);
\draw[line width=1.2pt,color=ccqqqq] (12.75500000688943,3.6871197487785423) -- (12.780000006668603,3.6786862566192085);
\draw[line width=1.2pt,color=ccqqqq] (12.780000006668603,3.6786862566192085) -- (12.805000006447775,3.6702647353680815);
\draw[line width=1.2pt,color=ccqqqq] (12.805000006447775,3.6702647353680815) -- (12.830000006226948,3.661855134193152);
\draw[line width=1.2pt,color=ccqqqq] (12.830000006226948,3.661855134193152) -- (12.85500000600612,3.653457402621142);
\draw[line width=1.2pt,color=ccqqqq] (12.85500000600612,3.653457402621142) -- (12.880000005785293,3.6450714905339705);
\draw[line width=1.2pt,color=ccqqqq] (12.880000005785293,3.6450714905339705) -- (12.905000005564466,3.6366973481652627);
\draw[line width=1.2pt,color=ccqqqq] (12.905000005564466,3.6366973481652627) -- (12.930000005343638,3.6283349260969064);
\draw[line width=1.2pt,color=ccqqqq] (12.930000005343638,3.6283349260969064) -- (12.95500000512281,3.619984175255647);
\draw[line width=1.2pt,color=ccqqqq] (12.95500000512281,3.619984175255647) -- (12.980000004901983,3.6116450469097297);
\draw[line width=1.2pt,color=ccqqqq] (12.980000004901983,3.6116450469097297) -- (13.005000004681156,3.603317492665581);
\draw[line width=1.2pt,color=ccqqqq] (13.005000004681156,3.603317492665581) -- (13.030000004460328,3.5950014644645316);
\draw[line width=1.2pt,color=ccqqqq] (13.030000004460328,3.5950014644645316) -- (13.055000004239501,3.5866969145795817);
\draw[line width=1.2pt,color=ccqqqq] (13.055000004239501,3.5866969145795817) -- (13.080000004018673,3.5784037956122043);
\draw[line width=1.2pt,color=ccqqqq] (13.080000004018673,3.5784037956122043) -- (13.105000003797846,3.57012206048919);
\draw[line width=1.2pt,color=ccqqqq] (13.105000003797846,3.57012206048919) -- (13.130000003577019,3.561851662459529);
\draw[line width=1.2pt,color=ccqqqq] (13.130000003577019,3.561851662459529) -- (13.155000003356191,3.5535925550913303);
\draw[line width=1.2pt,color=ccqqqq] (13.155000003356191,3.5535925550913303) -- (13.180000003135364,3.5453446922687846);
\draw[line width=1.2pt,color=ccqqqq] (13.180000003135364,3.5453446922687846) -- (13.205000002914536,3.5371080281891567);
\draw[line width=1.2pt,color=ccqqqq] (13.205000002914536,3.5371080281891567) -- (13.230000002693709,3.5288825173598184);
\draw[line width=1.2pt,color=ccqqqq] (13.230000002693709,3.5288825173598184) -- (13.255000002472881,3.520668114595318);
\draw[line width=1.2pt,color=ccqqqq] (13.255000002472881,3.520668114595318) -- (13.280000002252054,3.512464775014484);
\draw[line width=1.2pt,color=ccqqqq] (13.280000002252054,3.512464775014484) -- (13.305000002031226,3.5042724540375643);
\draw[line width=1.2pt,color=ccqqqq] (13.305000002031226,3.5042724540375643) -- (13.330000001810399,3.496091107383398);
\draw[line width=1.2pt,color=ccqqqq] (13.330000001810399,3.496091107383398) -- (13.355000001589572,3.487920691066627);
\draw[line width=1.2pt,color=ccqqqq] (13.355000001589572,3.487920691066627) -- (13.380000001368744,3.479761161394932);
\draw[line width=1.2pt,color=ccqqqq] (13.380000001368744,3.479761161394932) -- (13.405000001147917,3.471612474966311);
\draw[line width=1.2pt,color=ccqqqq] (13.405000001147917,3.471612474966311) -- (13.43000000092709,3.463474588666383);
\draw[line width=1.2pt,color=ccqqqq] (13.43000000092709,3.463474588666383) -- (13.455000000706262,3.4553474596657265);
\draw[line width=1.2pt,color=ccqqqq] (13.455000000706262,3.4553474596657265) -- (13.480000000485434,3.4472310454172517);
\draw[line width=1.2pt,color=ccqqqq] (13.480000000485434,3.4472310454172517) -- (13.505000000264607,3.4391253036536016);
\draw[line width=1.2pt,color=ccqqqq] (13.505000000264607,3.4391253036536016) -- (13.53000000004378,3.43103019238458);
\draw[line width=1.2pt,color=ccqqqq] (13.53000000004378,3.43103019238458) -- (13.554999999822952,3.42294566989462);
\draw[line width=1.2pt,color=ccqqqq] (13.554999999822952,3.42294566989462) -- (13.579999999602125,3.4148716947402704);
\draw[line width=1.2pt,color=ccqqqq] (13.579999999602125,3.4148716947402704) -- (13.604999999381297,3.4068082257477217);
\draw[line width=1.2pt,color=ccqqqq] (13.604999999381297,3.4068082257477217) -- (13.62999999916047,3.398755222010352);
\draw[line width=1.2pt,color=ccqqqq] (13.62999999916047,3.398755222010352) -- (13.654999998939642,3.3907126428863075);
\draw[line width=1.2pt,color=ccqqqq] (13.654999998939642,3.3907126428863075) -- (13.679999998718815,3.382680447996113);
\draw[line width=1.2pt,color=ccqqqq] (13.679999998718815,3.382680447996113) -- (13.704999998497987,3.374658597220301);
\draw[line width=1.2pt,color=ccqqqq] (13.704999998497987,3.374658597220301) -- (13.72999999827716,3.3666470506970763);
\draw[line width=1.2pt,color=ccqqqq] (13.72999999827716,3.3666470506970763) -- (13.754999998056332,3.358645768820007);
\draw[line width=1.2pt,color=ccqqqq] (13.754999998056332,3.358645768820007) -- (13.779999997835505,3.3506547122357357);
\draw[line width=1.2pt,color=ccqqqq] (13.779999997835505,3.3506547122357357) -- (13.804999997614678,3.3426738418417257);
\draw[line width=1.2pt,color=ccqqqq] (13.804999997614678,3.3426738418417257) -- (13.82999999739385,3.3347031187840264);
\draw[line width=1.2pt,color=ccqqqq] (13.82999999739385,3.3347031187840264) -- (13.854999997173023,3.3267425044550656);
\draw[line width=1.2pt,color=ccqqqq] (13.854999997173023,3.3267425044550656) -- (13.879999996952195,3.3187919604914704);
\draw[line width=1.2pt,color=ccqqqq] (13.879999996952195,3.3187919604914704) -- (13.904999996731368,3.310851448771909);
\draw[line width=1.2pt,color=ccqqqq] (13.904999996731368,3.310851448771909) -- (13.92999999651054,3.3029209314149552);
\draw[line width=1.2pt,color=ccqqqq] (13.92999999651054,3.3029209314149552) -- (13.954999996289713,3.295000370776987);
\draw[line width=1.2pt,color=ffqqqq] (0.06000009320000301,23.54369038834948) -- (0.06000009320000301,23.54369038834948);
\draw[line width=1.2pt,color=ffqqqq] (0.06000009320000301,23.54369038834948) -- (0.07000009263434179,23.54360038668206);
\draw[line width=1.2pt,color=ffqqqq] (0.07000009263434179,23.54360038668206) -- (0.08000009206868056,23.543330385035006);
\draw[line width=1.2pt,color=ffqqqq] (0.08000009206868056,23.543330385035006) -- (0.09000009150301932,23.542880383408313);
\draw[line width=1.2pt,color=ffqqqq] (0.09000009150301932,23.542880383408313) -- (0.1000000909373581,23.54225038180199);
\draw[line width=1.2pt,color=ffqqqq] (0.1000000909373581,23.54225038180199) -- (0.11000009037169686,23.541440380216027);
\draw[line width=1.2pt,color=ffqqqq] (0.11000009037169686,23.541440380216027) -- (0.12000008980603563,23.540450378650426);
\draw[line width=1.2pt,color=ffqqqq] (0.12000008980603563,23.540450378650426) -- (0.1300000892403744,23.539280377105193);
\draw[line width=1.2pt,color=ffqqqq] (0.1300000892403744,23.539280377105193) -- (0.14000008867471317,23.53793037558032);
\draw[line width=1.2pt,color=ffqqqq] (0.14000008867471317,23.53793037558032) -- (0.15000008810905194,23.53640037407581);
\draw[line width=1.2pt,color=ffqqqq] (0.15000008810905194,23.53640037407581) -- (0.1600000875433907,23.53469037259167);
\draw[line width=1.2pt,color=ffqqqq] (0.1600000875433907,23.53469037259167) -- (0.17000008697772948,23.532800371127887);
\draw[line width=1.2pt,color=ffqqqq] (0.17000008697772948,23.532800371127887) -- (0.18000008641206824,23.530730369684473);
\draw[line width=1.2pt,color=ffqqqq] (0.18000008641206824,23.530730369684473) -- (0.190000085846407,23.52848036826142);
\draw[line width=1.2pt,color=ffqqqq] (0.190000085846407,23.52848036826142) -- (0.20000008528074578,23.52605036685873);
\draw[line width=1.2pt,color=ffqqqq] (0.20000008528074578,23.52605036685873) -- (0.21000008471508455,23.523440365476407);
\draw[line width=1.2pt,color=ffqqqq] (0.21000008471508455,23.523440365476407) -- (0.22000008414942332,23.520650364114445);
\draw[line width=1.2pt,color=ffqqqq] (0.22000008414942332,23.520650364114445) -- (0.2300000835837621,23.517680362772847);
\draw[line width=1.2pt,color=ffqqqq] (0.2300000835837621,23.517680362772847) -- (0.24000008301810086,23.514530361451616);
\draw[line width=1.2pt,color=ffqqqq] (0.24000008301810086,23.514530361451616) -- (0.25000008245243965,23.511200360150745);
\draw[line width=1.2pt,color=ffqqqq] (0.25000008245243965,23.511200360150745) -- (0.2600000818867784,23.50769035887024);
\draw[line width=1.2pt,color=ffqqqq] (0.2600000818867784,23.50769035887024) -- (0.2700000813211172,23.504000357610096);
\draw[line width=1.2pt,color=ffqqqq] (0.2700000813211172,23.504000357610096) -- (0.28000008075545596,23.50013035637032);
\draw[line width=1.2pt,color=ffqqqq] (0.28000008075545596,23.50013035637032) -- (0.29000008018979473,23.496080355150905);
\draw[line width=1.2pt,color=ffqqqq] (0.29000008018979473,23.496080355150905) -- (0.3000000796241335,23.491850353951854);
\draw[line width=1.2pt,color=ffqqqq] (0.3000000796241335,23.491850353951854) -- (0.31000007905847227,23.487440352773167);
\draw[line width=1.2pt,color=ffqqqq] (0.31000007905847227,23.487440352773167) -- (0.32000007849281104,23.482850351614843);
\draw[line width=1.2pt,color=ffqqqq] (0.32000007849281104,23.482850351614843) -- (0.3300000779271498,23.478080350476887);
\draw[line width=1.2pt,color=ffqqqq] (0.3300000779271498,23.478080350476887) -- (0.3400000773614886,23.47313034935929);
\draw[line width=1.2pt,color=ffqqqq] (0.3400000773614886,23.47313034935929) -- (0.35000007679582734,23.46800034826206);
\draw[line width=1.2pt,color=ffqqqq] (0.35000007679582734,23.46800034826206) -- (0.3600000762301661,23.46269034718519);
\draw[line width=1.2pt,color=ffqqqq] (0.3600000762301661,23.46269034718519) -- (0.3700000756645049,23.457200346128687);
\draw[line width=1.2pt,color=ffqqqq] (0.3700000756645049,23.457200346128687) -- (0.38000007509884365,23.451530345092547);
\draw[line width=1.2pt,color=ffqqqq] (0.38000007509884365,23.451530345092547) -- (0.3900000745331824,23.44568034407677);
\draw[line width=1.2pt,color=ffqqqq] (0.3900000745331824,23.44568034407677) -- (0.4000000739675212,23.439650343081357);
\draw[line width=1.2pt,color=ffqqqq] (0.4000000739675212,23.439650343081357) -- (0.41000007340185995,23.43344034210631);
\draw[line width=1.2pt,color=ffqqqq] (0.41000007340185995,23.43344034210631) -- (0.4200000728361987,23.427050341151624);
\draw[line width=1.2pt,color=ffqqqq] (0.4200000728361987,23.427050341151624) -- (0.4300000722705375,23.420480340217303);
\draw[line width=1.2pt,color=ffqqqq] (0.4300000722705375,23.420480340217303) -- (0.44000007170487626,23.413730339303346);
\draw[line width=1.2pt,color=ffqqqq] (0.44000007170487626,23.413730339303346) -- (0.45000007113921503,23.406800338409752);
\draw[line width=1.2pt,color=ffqqqq] (0.45000007113921503,23.406800338409752) -- (0.4600000705735538,23.399690337536523);
\draw[line width=1.2pt,color=ffqqqq] (0.4600000705735538,23.399690337536523) -- (0.47000007000789257,23.392400336683657);
\draw[line width=1.2pt,color=ffqqqq] (0.47000007000789257,23.392400336683657) -- (0.48000006944223134,23.384930335851156);
\draw[line width=1.2pt,color=ffqqqq] (0.48000006944223134,23.384930335851156) -- (0.4900000688765701,23.377280335039018);
\draw[line width=1.2pt,color=ffqqqq] (0.4900000688765701,23.377280335039018) -- (0.5000000683109089,23.36945033424724);
\draw[line width=1.2pt,color=ffqqqq] (0.5000000683109089,23.36945033424724) -- (0.5100000677452478,23.36144033347583);
\draw[line width=1.2pt,color=ffqqqq] (0.5100000677452478,23.36144033347583) -- (0.5200000671795866,23.353250332724784);
\draw[line width=1.2pt,color=ffqqqq] (0.5200000671795866,23.353250332724784) -- (0.5300000666139254,23.3448803319941);
\draw[line width=1.2pt,color=ffqqqq] (0.5300000666139254,23.3448803319941) -- (0.5400000660482642,23.336330331283783);
\draw[line width=1.2pt,color=ffqqqq] (0.5400000660482642,23.336330331283783) -- (0.550000065482603,23.327600330593825);
\draw[line width=1.2pt,color=ffqqqq] (0.550000065482603,23.327600330593825) -- (0.5600000649169419,23.318690329924234);
\draw[line width=1.2pt,color=ffqqqq] (0.5600000649169419,23.318690329924234) -- (0.5700000643512807,23.309600329275007);
\draw[line width=1.2pt,color=ffqqqq] (0.5700000643512807,23.309600329275007) -- (0.5800000637856195,23.300330328646144);
\draw[line width=1.2pt,color=ffqqqq] (0.5800000637856195,23.300330328646144) -- (0.5900000632199583,23.29088032803764);
\draw[line width=1.2pt,color=ffqqqq] (0.5900000632199583,23.29088032803764) -- (0.6000000626542972,23.281250327449506);
\draw[line width=1.2pt,color=ffqqqq] (0.6000000626542972,23.281250327449506) -- (0.610000062088636,23.27144032688173);
\draw[line width=1.2pt,color=ffqqqq] (0.610000062088636,23.27144032688173) -- (0.6200000615229748,23.261450326334323);
\draw[line width=1.2pt,color=ffqqqq] (0.6200000615229748,23.261450326334323) -- (0.6300000609573136,23.25128032580728);
\draw[line width=1.2pt,color=ffqqqq] (0.6300000609573136,23.25128032580728) -- (0.6400000603916525,23.240930325300596);
\draw[line width=1.2pt,color=ffqqqq] (0.6400000603916525,23.240930325300596) -- (0.6500000598259913,23.23040032481428);
\draw[line width=1.2pt,color=ffqqqq] (0.6500000598259913,23.23040032481428) -- (0.6600000592603301,23.219690324348328);
\draw[line width=1.2pt,color=ffqqqq] (0.6600000592603301,23.219690324348328) -- (0.6700000586946689,23.208800323902736);
\draw[line width=1.2pt,color=ffqqqq] (0.6700000586946689,23.208800323902736) -- (0.6800000581290078,23.19773032347751);
\draw[line width=1.2pt,color=ffqqqq] (0.6800000581290078,23.19773032347751) -- (0.6900000575633466,23.186480323072647);
\draw[line width=1.2pt,color=ffqqqq] (0.6900000575633466,23.186480323072647) -- (0.7000000569976854,23.17505032268815);
\draw[line width=1.2pt,color=ffqqqq] (0.7000000569976854,23.17505032268815) -- (0.7100000564320242,23.163440322324014);
\draw[line width=1.2pt,color=ffqqqq] (0.7100000564320242,23.163440322324014) -- (0.7200000558663631,23.151650321980245);
\draw[line width=1.2pt,color=ffqqqq] (0.7200000558663631,23.151650321980245) -- (0.7300000553007019,23.139680321656837);
\draw[line width=1.2pt,color=ffqqqq] (0.7300000553007019,23.139680321656837) -- (0.7400000547350407,23.12753032135379);
\draw[line width=1.2pt,color=ffqqqq] (0.7400000547350407,23.12753032135379) -- (0.7500000541693795,23.115200321071114);
\draw[line width=1.2pt,color=ffqqqq] (0.7500000541693795,23.115200321071114) -- (0.7600000536037184,23.102690320808797);
\draw[line width=1.2pt,color=ffqqqq] (0.7600000536037184,23.102690320808797) -- (0.7700000530380572,23.090000320566848);
\draw[line width=1.2pt,color=ffqqqq] (0.7700000530380572,23.090000320566848) -- (0.780000052472396,23.07713032034526);
\draw[line width=1.2pt,color=ffqqqq] (0.780000052472396,23.07713032034526) -- (0.7900000519067348,23.064080320144033);
\draw[line width=1.2pt,color=ffqqqq] (0.7900000519067348,23.064080320144033) -- (0.8000000513410737,23.050850319963175);
\draw[line width=1.2pt,color=ffqqqq] (0.8000000513410737,23.050850319963175) -- (0.8100000507754125,23.037440319802677);
\draw[line width=1.2pt,color=ffqqqq] (0.8100000507754125,23.037440319802677) -- (0.8200000502097513,23.023850319662543);
\draw[line width=1.2pt,color=ffqqqq] (0.8200000502097513,23.023850319662543) -- (0.8300000496440901,23.010080319542773);
\draw[line width=1.2pt,color=ffqqqq] (0.8300000496440901,23.010080319542773) -- (0.840000049078429,22.99613031944337);
\draw[line width=1.2pt,color=ffqqqq] (0.840000049078429,22.99613031944337) -- (0.8500000485127678,22.982000319364328);
\draw[line width=1.2pt,color=ffqqqq] (0.8500000485127678,22.982000319364328) -- (0.8600000479471066,22.96769031930565);
\draw[line width=1.2pt,color=ffqqqq] (0.8600000479471066,22.96769031930565) -- (0.8700000473814454,22.953200319267335);
\draw[line width=1.2pt,color=ffqqqq] (0.8700000473814454,22.953200319267335) -- (0.8800000468157843,22.938530319249388);
\draw[line width=1.2pt,color=ffqqqq] (0.8800000468157843,22.938530319249388) -- (0.8900000462501231,22.9236803192518);
\draw[line width=1.2pt,color=ffqqqq] (0.8900000462501231,22.9236803192518) -- (0.9000000456844619,22.908650319274578);
\draw[line width=1.2pt,color=ffqqqq] (0.9000000456844619,22.908650319274578) -- (0.9100000451188007,22.89344031931772);
\draw[line width=1.2pt,color=ffqqqq] (0.9100000451188007,22.89344031931772) -- (0.9200000445531396,22.878050319381224);
\draw[line width=1.2pt,color=ffqqqq] (0.9200000445531396,22.878050319381224) -- (0.9300000439874784,22.862480319465092);
\draw[line width=1.2pt,color=ffqqqq] (0.9300000439874784,22.862480319465092) -- (0.9400000434218172,22.846730319569325);
\draw[line width=1.2pt,color=ffqqqq] (0.9400000434218172,22.846730319569325) -- (0.950000042856156,22.83080031969392);
\draw[line width=1.2pt,color=ffqqqq] (0.950000042856156,22.83080031969392) -- (0.9600000422904948,22.81469031983888);
\draw[line width=1.2pt,color=ffqqqq] (0.9600000422904948,22.81469031983888) -- (0.9700000417248337,22.798400320004205);
\draw[line width=1.2pt,color=ffqqqq] (0.9700000417248337,22.798400320004205) -- (0.9800000411591725,22.781930320189893);
\draw[line width=1.2pt,color=ffqqqq] (0.9800000411591725,22.781930320189893) -- (0.9900000405935113,22.765280320395945);
\draw[line width=1.2pt,color=ffqqqq] (0.9900000405935113,22.765280320395945) -- (1.0000000400278501,22.74845032062236);
\draw[line width=1.2pt,color=ffqqqq] (1.0000000400278501,22.74845032062236) -- (1.0100000394621889,22.73144032086914);
\draw[line width=1.2pt,color=ffqqqq] (1.0100000394621889,22.73144032086914) -- (1.0200000388965276,22.714250321136284);
\draw[line width=1.2pt,color=ffqqqq] (1.0200000388965276,22.714250321136284) -- (1.0300000383308663,22.69688032142379);
\draw[line width=1.2pt,color=ffqqqq] (1.0300000383308663,22.69688032142379) -- (1.040000037765205,22.679330321731662);
\draw[line width=1.2pt,color=ffqqqq] (1.040000037765205,22.679330321731662) -- (1.0500000371995437,22.661600322059897);
\draw[line width=1.2pt,color=ffqqqq] (1.0500000371995437,22.661600322059897) -- (1.0600000366338824,22.643690322408496);
\draw[line width=1.2pt,color=ffqqqq] (1.0600000366338824,22.643690322408496) -- (1.0700000360682211,22.62560032277746);
\draw[line width=1.2pt,color=ffqqqq] (1.0700000360682211,22.62560032277746) -- (1.0800000355025599,22.607330323166785);
\draw[line width=1.2pt,color=ffqqqq] (1.0800000355025599,22.607330323166785) -- (1.0900000349368986,22.588880323576475);
\draw[line width=1.2pt,color=ffqqqq] (1.0900000349368986,22.588880323576475) -- (1.1000000343712373,22.57025032400653);
\draw[line width=1.2pt,color=ffqqqq] (1.1000000343712373,22.57025032400653) -- (1.110000033805576,22.551440324456948);
\draw[line width=1.2pt,color=ffqqqq] (1.110000033805576,22.551440324456948) -- (1.1200000332399147,22.532450324927726);
\draw[line width=1.2pt,color=ffqqqq] (1.1200000332399147,22.532450324927726) -- (1.1300000326742534,22.513280325418872);
\draw[line width=1.2pt,color=ffqqqq] (1.1300000326742534,22.513280325418872) -- (1.1400000321085921,22.493930325930382);
\draw[line width=1.2pt,color=ffqqqq] (1.1400000321085921,22.493930325930382) -- (1.1500000315429308,22.474400326462256);
\draw[line width=1.2pt,color=ffqqqq] (1.1500000315429308,22.474400326462256) -- (1.1600000309772696,22.45469032701449);
\draw[line width=1.2pt,color=ffqqqq] (1.1600000309772696,22.45469032701449) -- (1.1700000304116083,22.43480032758709);
\draw[line width=1.2pt,color=ffqqqq] (1.1700000304116083,22.43480032758709) -- (1.180000029845947,22.414730328180056);
\draw[line width=1.2pt,color=ffqqqq] (1.180000029845947,22.414730328180056) -- (1.1900000292802857,22.394480328793385);
\draw[line width=1.2pt,color=ffqqqq] (1.1900000292802857,22.394480328793385) -- (1.2000000287146244,22.374050329427075);
\draw[line width=1.2pt,color=ffqqqq] (1.2000000287146244,22.374050329427075) -- (1.2100000281489631,22.35344033008113);
\draw[line width=1.2pt,color=ffqqqq] (1.2100000281489631,22.35344033008113) -- (1.2200000275833018,22.332650330755552);
\draw[line width=1.2pt,color=ffqqqq] (1.2200000275833018,22.332650330755552) -- (1.2300000270176406,22.311680331450333);
\draw[line width=1.2pt,color=ffqqqq] (1.2300000270176406,22.311680331450333) -- (1.2400000264519793,22.29053033216548);
\draw[line width=1.2pt,color=ffqqqq] (1.2400000264519793,22.29053033216548) -- (1.250000025886318,22.269200332900994);
\draw[line width=1.2pt,color=ffqqqq] (1.250000025886318,22.269200332900994) -- (1.2600000253206567,22.247690333656866);
\draw[line width=1.2pt,color=ffqqqq] (1.2600000253206567,22.247690333656866) -- (1.2700000247549954,22.226000334433106);
\draw[line width=1.2pt,color=ffqqqq] (1.2700000247549954,22.226000334433106) -- (1.2800000241893341,22.204130335229706);
\draw[line width=1.2pt,color=ffqqqq] (1.2800000241893341,22.204130335229706) -- (1.2900000236236728,22.182080336046674);
\draw[line width=1.2pt,color=ffqqqq] (1.2900000236236728,22.182080336046674) -- (1.3000000230580115,22.159850336884002);
\draw[line width=1.2pt,color=ffqqqq] (1.3000000230580115,22.159850336884002) -- (1.3100000224923503,22.137440337741698);
\draw[line width=1.2pt,color=ffqqqq] (1.3100000224923503,22.137440337741698) -- (1.320000021926689,22.114850338619753);
\draw[line width=1.2pt,color=ffqqqq] (1.320000021926689,22.114850338619753) -- (1.3300000213610277,22.092080339518176);
\draw[line width=1.2pt,color=ffqqqq] (1.3300000213610277,22.092080339518176) -- (1.3400000207953664,22.06913034043696);
\draw[line width=1.2pt,color=ffqqqq] (1.3400000207953664,22.06913034043696) -- (1.3500000202297051,22.04600034137611);
\draw[line width=1.2pt,color=ffqqqq] (1.3500000202297051,22.04600034137611) -- (1.3600000196640438,22.022690342335622);
\draw[line width=1.2pt,color=ffqqqq] (1.3600000196640438,22.022690342335622) -- (1.3700000190983825,21.9992003433155);
\draw[line width=1.2pt,color=ffqqqq] (1.3700000190983825,21.9992003433155) -- (1.3800000185327213,21.97553034431574);
\draw[line width=1.2pt,color=ffqqqq] (1.3800000185327213,21.97553034431574) -- (1.39000001796706,21.951680345336342);
\draw[line width=1.2pt,color=ffqqqq] (1.39000001796706,21.951680345336342) -- (1.4000000174013987,21.927650346377312);
\draw[line width=1.2pt,color=ffqqqq] (1.4000000174013987,21.927650346377312) -- (1.4100000168357374,21.903440347438643);
\draw[line width=1.2pt,color=ffqqqq] (1.4100000168357374,21.903440347438643) -- (1.420000016270076,21.879050348520337);
\draw[line width=1.2pt,color=ffqqqq] (1.420000016270076,21.879050348520337) -- (1.4300000157044148,21.8544803496224);
\draw[line width=1.2pt,color=ffqqqq] (1.4300000157044148,21.8544803496224) -- (1.4400000151387535,21.82973035074482);
\draw[line width=1.2pt,color=ffqqqq] (1.4400000151387535,21.82973035074482) -- (1.4500000145730922,21.804800351887607);
\draw[line width=1.2pt,color=ffqqqq] (1.4500000145730922,21.804800351887607) -- (1.460000014007431,21.77969035305076);
\draw[line width=1.2pt,color=ffqqqq] (1.460000014007431,21.77969035305076) -- (1.4700000134417697,21.754400354234274);
\draw[line width=1.2pt,color=ffqqqq] (1.4700000134417697,21.754400354234274) -- (1.4800000128761084,21.728930355438152);
\draw[line width=1.2pt,color=ffqqqq] (1.4800000128761084,21.728930355438152) -- (1.490000012310447,21.703280356662393);
\draw[line width=1.2pt,color=ffqqqq] (1.490000012310447,21.703280356662393) -- (1.5000000117447858,21.677450357907002);
\draw[line width=1.2pt,color=ffqqqq] (1.5000000117447858,21.677450357907002) -- (1.5100000111791245,21.65144035917197);
\draw[line width=1.2pt,color=ffqqqq] (1.5100000111791245,21.65144035917197) -- (1.5200000106134632,21.625250360457304);
\draw[line width=1.2pt,color=ffqqqq] (1.5200000106134632,21.625250360457304) -- (1.530000010047802,21.598880361763);
\draw[line width=1.2pt,color=ffqqqq] (1.530000010047802,21.598880361763) -- (1.5400000094821407,21.572330363089062);
\draw[line width=1.2pt,color=ffqqqq] (1.5400000094821407,21.572330363089062) -- (1.5500000089164794,21.545600364435487);
\draw[line width=1.2pt,color=ffqqqq] (1.5500000089164794,21.545600364435487) -- (1.560000008350818,21.51869036580228);
\draw[line width=1.2pt,color=ffqqqq] (1.560000008350818,21.51869036580228) -- (1.5700000077851568,21.49160036718943);
\draw[line width=1.2pt,color=ffqqqq] (1.5700000077851568,21.49160036718943) -- (1.5800000072194955,21.464330368596947);
\draw[line width=1.2pt,color=ffqqqq] (1.5800000072194955,21.464330368596947) -- (1.5900000066538342,21.436880370024827);
\draw[line width=1.2pt,color=ffqqqq] (1.5900000066538342,21.436880370024827) -- (1.600000006088173,21.40925037147307);
\draw[line width=1.2pt,color=ffqqqq] (1.600000006088173,21.40925037147307) -- (1.6100000055225117,21.38144037294168);
\draw[line width=1.2pt,color=ffqqqq] (1.6100000055225117,21.38144037294168) -- (1.6200000049568504,21.35345037443065);
\draw[line width=1.2pt,color=ffqqqq] (1.6200000049568504,21.35345037443065) -- (1.630000004391189,21.325280375939986);
\draw[line width=1.2pt,color=ffqqqq] (1.630000004391189,21.325280375939986) -- (1.6400000038255278,21.296930377469685);
\draw[line width=1.2pt,color=ffqqqq] (1.6400000038255278,21.296930377469685) -- (1.6500000032598665,21.26840037901975);
\draw[line width=1.2pt,color=ffqqqq] (1.6500000032598665,21.26840037901975) -- (1.6600000026942052,21.239690380590176);
\draw[line width=1.2pt,color=ffqqqq] (1.6600000026942052,21.239690380590176) -- (1.670000002128544,21.210800382180967);
\draw[line width=1.2pt,color=ffqqqq] (1.670000002128544,21.210800382180967) -- (1.6800000015628826,21.18173038379212);
\draw[line width=1.2pt,color=ffqqqq] (1.6800000015628826,21.18173038379212) -- (1.6900000009972214,21.152480385423637);
\draw[line width=1.2pt,color=ffqqqq] (1.6900000009972214,21.152480385423637) -- (1.70000000043156,21.12305038707552);
\draw[line width=1.2pt,color=ffqqqq] (1.70000000043156,21.12305038707552) -- (1.7099999998658988,21.093440388747766);
\draw[line width=1.2pt,color=ffqqqq] (1.7099999998658988,21.093440388747766) -- (1.7199999993002375,21.063650390440376);
\draw[line width=1.2pt,color=ffqqqq] (1.7199999993002375,21.063650390440376) -- (1.7299999987345762,21.03368039215335);
\draw[line width=1.2pt,color=ffqqqq] (1.7299999987345762,21.03368039215335) -- (1.739999998168915,21.003530393886688);
\draw[line width=1.2pt,color=ffqqqq] (1.739999998168915,21.003530393886688) -- (1.7499999976032536,20.97320039564039);
\draw[line width=1.2pt,color=ffqqqq] (1.7499999976032536,20.97320039564039) -- (1.7599999970375924,20.942690397414452);
\draw[line width=1.2pt,color=ffqqqq] (1.7599999970375924,20.942690397414452) -- (1.769999996471931,20.912000399208882);
\draw[line width=1.2pt,color=ffqqqq] (1.769999996471931,20.912000399208882) -- (1.7799999959062698,20.881130401023675);
\draw[line width=1.2pt,color=ffqqqq] (1.7799999959062698,20.881130401023675) -- (1.7899999953406085,20.85008040285883);
\draw[line width=1.2pt,color=ffqqqq] (1.7899999953406085,20.85008040285883) -- (1.7999999947749472,20.81885040471435);
\draw[line width=1.2pt,color=ffqqqq] (1.7999999947749472,20.81885040471435) -- (1.809999994209286,20.787440406590235);
\draw[line width=1.2pt,color=ffqqqq] (1.809999994209286,20.787440406590235) -- (1.8199999936436246,20.755850408486484);
\draw[line width=1.2pt,color=ffqqqq] (1.8199999936436246,20.755850408486484) -- (1.8299999930779633,20.724080410403094);
\draw[line width=1.2pt,color=ffqqqq] (1.8299999930779633,20.724080410403094) -- (1.839999992512302,20.69213041234007);
\draw[line width=1.2pt,color=ffqqqq] (1.839999992512302,20.69213041234007) -- (1.8499999919466408,20.660000414297407);
\draw[line width=1.2pt,color=ffqqqq] (1.8499999919466408,20.660000414297407) -- (1.8599999913809795,20.62769041627511);
\draw[line width=1.2pt,color=ffqqqq] (1.8599999913809795,20.62769041627511) -- (1.8699999908153182,20.59520041827318);
\draw[line width=1.2pt,color=ffqqqq] (1.8699999908153182,20.59520041827318) -- (1.879999990249657,20.56253042029161);
\draw[line width=1.2pt,color=ffqqqq] (1.879999990249657,20.56253042029161) -- (1.8899999896839956,20.529680422330404);
\draw[line width=1.2pt,color=ffqqqq] (1.8899999896839956,20.529680422330404) -- (1.8999999891183343,20.496650424389564);
\draw[line width=1.2pt,color=ffqqqq] (1.8999999891183343,20.496650424389564) -- (1.909999988552673,20.463440426469084);
\draw[line width=1.2pt,color=ffqqqq] (1.909999988552673,20.463440426469084) -- (1.9199999879870118,20.43005042856897);
\draw[line width=1.2pt,color=ffqqqq] (1.9199999879870118,20.43005042856897) -- (1.9299999874213505,20.39648043068922);
\draw[line width=1.2pt,color=ffqqqq] (1.9299999874213505,20.39648043068922) -- (1.9399999868556892,20.362730432829835);
\draw[line width=1.2pt,color=ffqqqq] (1.9399999868556892,20.362730432829835) -- (1.949999986290028,20.32880043499081);
\draw[line width=1.2pt,color=ffqqqq] (1.949999986290028,20.32880043499081) -- (1.9599999857243666,20.29469043717215);
\draw[line width=1.2pt,color=ffqqqq] (1.9599999857243666,20.29469043717215) -- (1.9699999851587053,20.260400439373857);
\draw[line width=1.2pt,color=ffqqqq] (1.9699999851587053,20.260400439373857) -- (1.979999984593044,20.225930441595924);
\draw[line width=1.2pt,color=ffqqqq] (1.979999984593044,20.225930441595924) -- (1.9899999840273828,20.19128044383836);
\draw[line width=1.2pt,color=ffqqqq] (1.9899999840273828,20.19128044383836) -- (1.9999999834617215,20.156450446101154);
\draw[line width=1.2pt,color=ffqqqq] (1.9999999834617215,20.156450446101154) -- (2.00999998289606,20.121440448384313);
\draw[line width=1.2pt,color=ffqqqq] (2.00999998289606,20.121440448384313) -- (2.019999982330399,20.08625045068784);
\draw[line width=1.2pt,color=ffqqqq] (2.019999982330399,20.08625045068784) -- (2.029999981764738,20.050880453011725);
\draw[line width=1.2pt,color=ffqqqq] (2.029999981764738,20.050880453011725) -- (2.039999981199077,20.015330455355976);
\draw[line width=1.2pt,color=ffqqqq] (2.039999981199077,20.015330455355976) -- (2.049999980633416,19.97960045772059);
\draw[line width=1.2pt,color=ffqqqq] (2.049999980633416,19.97960045772059) -- (2.059999980067755,19.943690460105568);
\draw[line width=1.2pt,color=ffqqqq] (2.059999980067755,19.943690460105568) -- (2.069999979502094,19.90760046251091);
\draw[line width=1.2pt,color=ffqqqq] (2.069999979502094,19.90760046251091) -- (2.0799999789364327,19.871330464936616);
\draw[line width=1.2pt,color=ffqqqq] (2.0799999789364327,19.871330464936616) -- (2.0899999783707717,19.834880467382686);
\draw[line width=1.2pt,color=ffqqqq] (2.0899999783707717,19.834880467382686) -- (2.0999999778051106,19.79825046984912);
\draw[line width=1.2pt,color=ffqqqq] (2.0999999778051106,19.79825046984912) -- (2.1099999772394495,19.761440472335916);
\draw[line width=1.2pt,color=ffqqqq] (2.1099999772394495,19.761440472335916) -- (2.1199999766737885,19.724450474843078);
\draw[line width=1.2pt,color=ffqqqq] (2.1199999766737885,19.724450474843078) -- (2.1299999761081274,19.687280477370603);
\draw[line width=1.2pt,color=ffqqqq] (2.1299999761081274,19.687280477370603) -- (2.1399999755424663,19.649930479918492);
\draw[line width=1.2pt,color=ffqqqq] (2.1399999755424663,19.649930479918492) -- (2.1499999749768053,19.612400482486745);
\draw[line width=1.2pt,color=ffqqqq] (2.1499999749768053,19.612400482486745) -- (2.159999974411144,19.574690485075358);
\draw[line width=1.2pt,color=ffqqqq] (2.159999974411144,19.574690485075358) -- (2.169999973845483,19.53680048768434);
\draw[line width=1.2pt,color=ffqqqq] (2.169999973845483,19.53680048768434) -- (2.179999973279822,19.498730490313683);
\draw[line width=1.2pt,color=ffqqqq] (2.179999973279822,19.498730490313683) -- (2.189999972714161,19.46048049296339);
\draw[line width=1.2pt,color=ffqqqq] (2.189999972714161,19.46048049296339) -- (2.1999999721485,19.422050495633464);
\draw[line width=1.2pt,color=ffqqqq] (2.1999999721485,19.422050495633464) -- (2.209999971582839,19.3834404983239);
\draw[line width=1.2pt,color=ffqqqq] (2.209999971582839,19.3834404983239) -- (2.219999971017178,19.3446505010347);
\draw[line width=1.2pt,color=ffqqqq] (2.219999971017178,19.3446505010347) -- (2.2299999704515168,19.30568050376586);
\draw[line width=1.2pt,color=ffqqqq] (2.2299999704515168,19.30568050376586) -- (2.2399999698858557,19.266530506517388);
\draw[line width=1.2pt,color=ffqqqq] (2.2399999698858557,19.266530506517388) -- (2.2499999693201946,19.227200509289275);
\draw[line width=1.2pt,color=ffqqqq] (2.2499999693201946,19.227200509289275) -- (2.2599999687545336,19.18769051208153);
\draw[line width=1.2pt,color=ffqqqq] (2.2599999687545336,19.18769051208153) -- (2.2699999681888725,19.14800051489415);
\draw[line width=1.2pt,color=ffqqqq] (2.2699999681888725,19.14800051489415) -- (2.2799999676232114,19.108130517727133);
\draw[line width=1.2pt,color=ffqqqq] (2.2799999676232114,19.108130517727133) -- (2.2899999670575504,19.068080520580477);
\draw[line width=1.2pt,color=ffqqqq] (2.2899999670575504,19.068080520580477) -- (2.2999999664918893,19.027850523454187);
\draw[line width=1.2pt,color=ffqqqq] (2.2999999664918893,19.027850523454187) -- (2.3099999659262282,18.98744052634826);
\draw[line width=1.2pt,color=ffqqqq] (2.3099999659262282,18.98744052634826) -- (2.319999965360567,18.946850529262697);
\draw[line width=1.2pt,color=ffqqqq] (2.319999965360567,18.946850529262697) -- (2.329999964794906,18.9060805321975);
\draw[line width=1.2pt,color=ffqqqq] (2.329999964794906,18.9060805321975) -- (2.339999964229245,18.865130535152662);
\draw[line width=1.2pt,color=ffqqqq] (2.339999964229245,18.865130535152662) -- (2.349999963663584,18.824000538128193);
\draw[line width=1.2pt,color=ffqqqq] (2.349999963663584,18.824000538128193) -- (2.359999963097923,18.782690541124083);
\draw[line width=1.2pt,color=ffqqqq] (2.359999963097923,18.782690541124083) -- (2.369999962532262,18.74120054414034);
\draw[line width=1.2pt,color=ffqqqq] (2.369999962532262,18.74120054414034) -- (2.379999961966601,18.69953054717696);
\draw[line width=1.2pt,color=ffqqqq] (2.379999961966601,18.69953054717696) -- (2.3899999614009397,18.65768055023394);
\draw[line width=1.2pt,color=ffqqqq] (2.3899999614009397,18.65768055023394) -- (2.3999999608352787,18.61565055331129);
\draw[line width=1.2pt,color=ffqqqq] (2.3999999608352787,18.61565055331129) -- (2.4099999602696176,18.573440556409004);
\draw[line width=1.2pt,color=ffqqqq] (2.4099999602696176,18.573440556409004) -- (2.4199999597039565,18.531050559527078);
\draw[line width=1.2pt,color=ffqqqq] (2.4199999597039565,18.531050559527078) -- (2.4299999591382955,18.488480562665515);
\draw[line width=1.2pt,color=ffqqqq] (2.4299999591382955,18.488480562665515) -- (2.4399999585726344,18.44573056582432);
\draw[line width=1.2pt,color=ffqqqq] (2.4399999585726344,18.44573056582432) -- (2.4499999580069733,18.402800569003485);
\draw[line width=1.2pt,color=ffqqqq] (2.4499999580069733,18.402800569003485) -- (2.4599999574413123,18.359690572203014);
\draw[line width=1.2pt,color=ffqqqq] (2.4599999574413123,18.359690572203014) -- (2.469999956875651,18.31640057542291);
\draw[line width=1.2pt,color=ffqqqq] (2.469999956875651,18.31640057542291) -- (2.47999995630999,18.272930578663168);
\draw[line width=1.2pt,color=ffqqqq] (2.47999995630999,18.272930578663168) -- (2.489999955744329,18.22928058192379);
\draw[line width=1.2pt,color=ffqqqq] (2.489999955744329,18.22928058192379) -- (2.499999955178668,18.185450585204773);
\draw[line width=1.2pt,color=ffqqqq] (2.499999955178668,18.185450585204773) -- (2.509999954613007,18.14144058850612);
\draw[line width=1.2pt,color=ffqqqq] (2.509999954613007,18.14144058850612) -- (2.519999954047346,18.097250591827837);
\draw[line width=1.2pt,color=ffqqqq] (2.519999954047346,18.097250591827837) -- (2.529999953481685,18.052880595169913);
\draw[line width=1.2pt,color=ffqqqq] (2.529999953481685,18.052880595169913) -- (2.5399999529160238,18.008330598532353);
\draw[line width=1.2pt,color=ffqqqq] (2.5399999529160238,18.008330598532353) -- (2.5499999523503627,17.963600601915157);
\draw[line width=1.2pt,color=ffqqqq] (2.5499999523503627,17.963600601915157) -- (2.5599999517847016,17.918690605318325);
\draw[line width=1.2pt,color=ffqqqq] (2.5599999517847016,17.918690605318325) -- (2.5699999512190406,17.87360060874186);
\draw[line width=1.2pt,color=ffqqqq] (2.5699999512190406,17.87360060874186) -- (2.5799999506533795,17.828330612185752);
\draw[line width=1.2pt,color=ffqqqq] (2.5799999506533795,17.828330612185752) -- (2.5899999500877184,17.782880615650015);
\draw[line width=1.2pt,color=ffqqqq] (2.5899999500877184,17.782880615650015) -- (2.5999999495220574,17.737250619134638);
\draw[line width=1.2pt,color=ffqqqq] (2.5999999495220574,17.737250619134638) -- (2.6099999489563963,17.691440622639625);
\draw[line width=1.2pt,color=ffqqqq] (2.6099999489563963,17.691440622639625) -- (2.6199999483907352,17.645450626164976);
\draw[line width=1.2pt,color=ffqqqq] (2.6199999483907352,17.645450626164976) -- (2.629999947825074,17.59928062971069);
\draw[line width=1.2pt,color=ffqqqq] (2.629999947825074,17.59928062971069) -- (2.639999947259413,17.55293063327677);
\draw[line width=1.2pt,color=ffqqqq] (2.639999947259413,17.55293063327677) -- (2.649999946693752,17.506400636863212);
\draw[line width=1.2pt,color=ffqqqq] (2.649999946693752,17.506400636863212) -- (2.659999946128091,17.45969064047002);
\draw[line width=1.2pt,color=ffqqqq] (2.659999946128091,17.45969064047002) -- (2.66999994556243,17.41280064409719);
\draw[line width=1.2pt,color=ffqqqq] (2.66999994556243,17.41280064409719) -- (2.679999944996769,17.36573064774472);
\draw[line width=1.2pt,color=ffqqqq] (2.679999944996769,17.36573064774472) -- (2.689999944431108,17.318480651412617);
\draw[line width=1.2pt,color=ffqqqq] (2.689999944431108,17.318480651412617) -- (2.6999999438654467,17.27105065510088);
\draw[line width=1.2pt,color=ffqqqq] (2.6999999438654467,17.27105065510088) -- (2.7099999432997857,17.223440658809505);
\draw[line width=1.2pt,color=ffqqqq] (2.7099999432997857,17.223440658809505) -- (2.7199999427341246,17.175650662538494);
\draw[line width=1.2pt,color=ffqqqq] (2.7199999427341246,17.175650662538494) -- (2.7299999421684635,17.127680666287848);
\draw[line width=1.2pt,color=ffqqqq] (2.7299999421684635,17.127680666287848) -- (2.7399999416028025,17.079530670057565);
\draw[line width=1.2pt,color=ffqqqq] (2.7399999416028025,17.079530670057565) -- (2.7499999410371414,17.031200673847643);
\draw[line width=1.2pt,color=ffqqqq] (2.7499999410371414,17.031200673847643) -- (2.7599999404714803,16.982690677658088);
\draw[line width=1.2pt,color=ffqqqq] (2.7599999404714803,16.982690677658088) -- (2.7699999399058193,16.934000681488897);
\draw[line width=1.2pt,color=ffqqqq] (2.7699999399058193,16.934000681488897) -- (2.779999939340158,16.885130685340066);
\draw[line width=1.2pt,color=ffqqqq] (2.779999939340158,16.885130685340066) -- (2.789999938774497,16.836080689211602);
\draw[line width=1.2pt,color=ffqqqq] (2.789999938774497,16.836080689211602) -- (2.799999938208836,16.786850693103503);
\draw[line width=1.2pt,color=ffqqqq] (2.799999938208836,16.786850693103503) -- (2.809999937643175,16.737440697015767);
\draw[line width=1.2pt,color=ffqqqq] (2.809999937643175,16.737440697015767) -- (2.819999937077514,16.687850700948392);
\draw[line width=1.2pt,color=ffqqqq] (2.819999937077514,16.687850700948392) -- (2.829999936511853,16.638080704901384);
\draw[line width=1.2pt,color=ffqqqq] (2.829999936511853,16.638080704901384) -- (2.839999935946192,16.58813070887474);
\draw[line width=1.2pt,color=ffqqqq] (2.839999935946192,16.58813070887474) -- (2.8499999353805308,16.538000712868456);
\draw[line width=1.2pt,color=ffqqqq] (2.8499999353805308,16.538000712868456) -- (2.8599999348148697,16.48769071688254);
\draw[line width=1.2pt,color=ffqqqq] (2.8599999348148697,16.48769071688254) -- (2.8699999342492086,16.437200720916984);
\draw[line width=1.2pt,color=ffqqqq] (2.8699999342492086,16.437200720916984) -- (2.8799999336835476,16.386530724971795);
\draw[line width=1.2pt,color=ffqqqq] (2.8799999336835476,16.386530724971795) -- (2.8899999331178865,16.335680729046967);
\draw[line width=1.2pt,color=ffqqqq] (2.8899999331178865,16.335680729046967) -- (2.8999999325522254,16.284650733142506);
\draw[line width=1.2pt,color=ffqqqq] (2.8999999325522254,16.284650733142506) -- (2.9099999319865644,16.233440737258405);
\draw[line width=1.2pt,color=ffqqqq] (2.9099999319865644,16.233440737258405) -- (2.9199999314209033,16.182050741394672);
\draw[line width=1.2pt,color=ffqqqq] (2.9199999314209033,16.182050741394672) -- (2.9299999308552422,16.1304807455513);
\draw[line width=1.2pt,color=ffqqqq] (2.9299999308552422,16.1304807455513) -- (2.939999930289581,16.078730749728294);
\draw[line width=1.2pt,color=ffqqqq] (2.939999930289581,16.078730749728294) -- (2.94999992972392,16.02680075392565);
\draw[line width=1.2pt,color=ffqqqq] (2.94999992972392,16.02680075392565) -- (2.959999929158259,15.97469075814337);
\draw[line width=1.2pt,color=ffqqqq] (2.959999929158259,15.97469075814337) -- (2.969999928592598,15.922400762381454);
\draw[line width=1.2pt,color=ffqqqq] (2.969999928592598,15.922400762381454) -- (2.979999928026937,15.8699307666399);
\draw[line width=1.2pt,color=ffqqqq] (2.979999928026937,15.8699307666399) -- (2.989999927461276,15.817280770918712);
\draw[line width=1.2pt,color=ffqqqq] (2.989999927461276,15.817280770918712) -- (2.999999926895615,15.764450775217888);
\draw[line width=1.2pt,color=ffqqqq] (2.999999926895615,15.764450775217888) -- (3.0099999263299537,15.711440779537426);
\draw[line width=1.2pt,color=ffqqqq] (3.0099999263299537,15.711440779537426) -- (3.0199999257642927,15.65825078387733);
\draw[line width=1.2pt,color=ffqqqq] (3.0199999257642927,15.65825078387733) -- (3.0299999251986316,15.604880788237597);
\draw[line width=1.2pt,color=ffqqqq] (3.0299999251986316,15.604880788237597) -- (3.0399999246329705,15.551330792618227);
\draw[line width=1.2pt,color=ffqqqq] (3.0399999246329705,15.551330792618227) -- (3.0499999240673095,15.49760079701922);
\draw[line width=1.2pt,color=ffqqqq] (3.0499999240673095,15.49760079701922) -- (3.0599999235016484,15.44369080144058);
\draw[line width=1.2pt,color=ffqqqq] (3.0599999235016484,15.44369080144058) -- (3.0699999229359873,15.389600805882301);
\draw[line width=1.2pt,color=ffqqqq] (3.0699999229359873,15.389600805882301) -- (3.0799999223703263,15.335330810344386);
\draw[line width=1.2pt,color=ffqqqq] (3.0799999223703263,15.335330810344386) -- (3.089999921804665,15.280880814826835);
\draw[line width=1.2pt,color=ffqqqq] (3.089999921804665,15.280880814826835) -- (3.099999921239004,15.22625081932965);
\draw[line width=1.2pt,color=ffqqqq] (3.099999921239004,15.22625081932965) -- (3.109999920673343,15.171440823852826);
\draw[line width=1.2pt,color=ffqqqq] (3.109999920673343,15.171440823852826) -- (3.119999920107682,15.116450828396367);
\draw[line width=1.2pt,color=ffqqqq] (3.119999920107682,15.116450828396367) -- (3.129999919542021,15.061280832960273);
\draw[line width=1.2pt,color=ffqqqq] (3.129999919542021,15.061280832960273) -- (3.13999991897636,15.005930837544541);
\draw[line width=1.2pt,color=ffqqqq] (3.13999991897636,15.005930837544541) -- (3.149999918410699,14.950400842149174);
\draw[line width=1.2pt,color=ffqqqq] (3.149999918410699,14.950400842149174) -- (3.1599999178450378,14.89469084677417);
\draw[line width=1.2pt,color=ffqqqq] (3.1599999178450378,14.89469084677417) -- (3.1699999172793767,14.838800851419528);
\draw[line width=1.2pt,color=ffqqqq] (3.1699999172793767,14.838800851419528) -- (3.1799999167137156,14.782730856085253);
\draw[line width=1.2pt,color=ffqqqq] (3.1799999167137156,14.782730856085253) -- (3.1899999161480546,14.726480860771339);
\draw[line width=1.2pt,color=ffqqqq] (3.1899999161480546,14.726480860771339) -- (3.1999999155823935,14.670050865477792);
\draw[line width=1.2pt,color=ffqqqq] (3.1999999155823935,14.670050865477792) -- (3.2099999150167324,14.613440870204606);
\draw[line width=1.2pt,color=ffqqqq] (3.2099999150167324,14.613440870204606) -- (3.2199999144510714,14.556650874951785);
\draw[line width=1.2pt,color=ffqqqq] (3.2199999144510714,14.556650874951785) -- (3.2299999138854103,14.499680879719328);
\draw[line width=1.2pt,color=ffqqqq] (3.2299999138854103,14.499680879719328) -- (3.2399999133197492,14.442530884507235);
\draw[line width=1.2pt,color=ffqqqq] (3.2399999133197492,14.442530884507235) -- (3.249999912754088,14.385200889315504);
\draw[line width=1.2pt,color=ffqqqq] (3.249999912754088,14.385200889315504) -- (3.259999912188427,14.327690894144139);
\draw[line width=1.2pt,color=ffqqqq] (3.259999912188427,14.327690894144139) -- (3.269999911622766,14.270000898993136);
\draw[line width=1.2pt,color=ffqqqq] (3.269999911622766,14.270000898993136) -- (3.279999911057105,14.212130903862498);
\draw[line width=1.2pt,color=ffqqqq] (3.279999911057105,14.212130903862498) -- (3.289999910491444,14.154080908752224);
\draw[line width=1.2pt,color=ffqqqq] (3.289999910491444,14.154080908752224) -- (3.299999909925783,14.095850913662314);
\draw[line width=1.2pt,color=ffqqqq] (3.299999909925783,14.095850913662314) -- (3.309999909360122,14.037440918592765);
\draw[line width=1.2pt,color=ffqqqq] (3.309999909360122,14.037440918592765) -- (3.3199999087944607,13.978850923543582);
\draw[line width=1.2pt,color=ffqqqq] (3.3199999087944607,13.978850923543582) -- (3.3299999082287997,13.920080928514762);
\draw[line width=1.2pt,color=ffqqqq] (3.3299999082287997,13.920080928514762) -- (3.3399999076631386,13.861130933506308);
\draw[line width=1.2pt,color=ffqqqq] (3.3399999076631386,13.861130933506308) -- (3.3499999070974775,13.802000938518216);
\draw[line width=1.2pt,color=ffqqqq] (3.3499999070974775,13.802000938518216) -- (3.3599999065318165,13.742690943550489);
\draw[line width=1.2pt,color=ffqqqq] (3.3599999065318165,13.742690943550489) -- (3.3699999059661554,13.683200948603123);
\draw[line width=1.2pt,color=ffqqqq] (3.3699999059661554,13.683200948603123) -- (3.3799999054004943,13.623530953676124);
\draw[line width=1.2pt,color=ffqqqq] (3.3799999054004943,13.623530953676124) -- (3.3899999048348333,13.563680958769487);
\draw[line width=1.2pt,color=ffqqqq] (3.3899999048348333,13.563680958769487) -- (3.399999904269172,13.503650963883214);
\draw[line width=1.2pt,color=ffqqqq] (3.399999904269172,13.503650963883214) -- (3.409999903703511,13.443440969017304);
\draw[line width=1.2pt,color=ffqqqq] (3.409999903703511,13.443440969017304) -- (3.41999990313785,13.383050974171761);
\draw[line width=1.2pt,color=ffqqqq] (3.41999990313785,13.383050974171761) -- (3.429999902572189,13.32248097934658);
\draw[line width=1.2pt,color=ffqqqq] (3.429999902572189,13.32248097934658) -- (3.439999902006528,13.261730984541762);
\draw[line width=1.2pt,color=ffqqqq] (3.439999902006528,13.261730984541762) -- (3.449999901440867,13.200800989757308);
\draw[line width=1.2pt,color=ffqqqq] (3.449999901440867,13.200800989757308) -- (3.459999900875206,13.139690994993217);
\draw[line width=1.2pt,color=ffqqqq] (3.459999900875206,13.139690994993217) -- (3.4699999003095447,13.07840100024949);
\draw[line width=1.2pt,color=ffqqqq] (3.4699999003095447,13.07840100024949) -- (3.4799998997438837,13.016931005526128);
\draw[line width=1.2pt,color=ffqqqq] (3.4799998997438837,13.016931005526128) -- (3.4899998991782226,12.95528101082313);
\draw[line width=1.2pt,color=ffqqqq] (3.4899998991782226,12.95528101082313) -- (3.4999998986125616,12.893451016140496);
\draw[line width=1.2pt,color=ffqqqq] (3.4999998986125616,12.893451016140496) -- (3.5099998980469005,12.831441021478224);
\draw[line width=1.2pt,color=ffqqqq] (3.5099998980469005,12.831441021478224) -- (3.5199998974812394,12.769251026836317);
\draw[line width=1.2pt,color=ffqqqq] (3.5199998974812394,12.769251026836317) -- (3.5299998969155784,12.706881032214774);
\draw[line width=1.2pt,color=ffqqqq] (3.5299998969155784,12.706881032214774) -- (3.5399998963499173,12.644331037613595);
\draw[line width=1.2pt,color=ffqqqq] (3.5399998963499173,12.644331037613595) -- (3.5499998957842562,12.581601043032778);
\draw[line width=1.2pt,color=ffqqqq] (3.5499998957842562,12.581601043032778) -- (3.559999895218595,12.518691048472327);
\draw[line width=1.2pt,color=ffqqqq] (3.559999895218595,12.518691048472327) -- (3.569999894652934,12.455601053932238);
\draw[line width=1.2pt,color=ffqqqq] (3.569999894652934,12.455601053932238) -- (3.579999894087273,12.392331059412513);
\draw[line width=1.2pt,color=ffqqqq] (3.579999894087273,12.392331059412513) -- (3.589999893521612,12.328881064913153);
\draw[line width=1.2pt,color=ffqqqq] (3.589999893521612,12.328881064913153) -- (3.599999892955951,12.265251070434157);
\draw[line width=1.2pt,color=ffqqqq] (3.599999892955951,12.265251070434157) -- (3.60999989239029,12.201441075975524);
\draw[line width=1.2pt,color=ffqqqq] (3.60999989239029,12.201441075975524) -- (3.619999891824629,12.137451081537254);
\draw[line width=1.2pt,color=ffqqqq] (3.619999891824629,12.137451081537254) -- (3.6299998912589677,12.073281087119348);
\draw[line width=1.2pt,color=ffqqqq] (3.6299998912589677,12.073281087119348) -- (3.6399998906933067,12.008931092721808);
\draw[line width=1.2pt,color=ffqqqq] (3.6399998906933067,12.008931092721808) -- (3.6499998901276456,11.944401098344628);
\draw[line width=1.2pt,color=ffqqqq] (3.6499998901276456,11.944401098344628) -- (3.6599998895619845,11.879691103987815);
\draw[line width=1.2pt,color=ffqqqq] (3.6599998895619845,11.879691103987815) -- (3.6699998889963235,11.814801109651365);
\draw[line width=1.2pt,color=ffqqqq] (3.6699998889963235,11.814801109651365) -- (3.6799998884306624,11.749731115335278);
\draw[line width=1.2pt,color=ffqqqq] (3.6799998884306624,11.749731115335278) -- (3.6899998878650013,11.684481121039555);
\draw[line width=1.2pt,color=ffqqqq] (3.6899998878650013,11.684481121039555) -- (3.6999998872993403,11.619051126764196);
\draw[line width=1.2pt,color=ffqqqq] (3.6999998872993403,11.619051126764196) -- (3.709999886733679,11.553441132509201);
\draw[line width=1.2pt,color=ffqqqq] (3.709999886733679,11.553441132509201) -- (3.719999886168018,11.48765113827457);
\draw[line width=1.2pt,color=ffqqqq] (3.719999886168018,11.48765113827457) -- (3.729999885602357,11.421681144060303);
\draw[line width=1.2pt,color=ffqqqq] (3.729999885602357,11.421681144060303) -- (3.739999885036696,11.3555311498664);
\draw[line width=1.2pt,color=ffqqqq] (3.739999885036696,11.3555311498664) -- (3.749999884471035,11.28920115569286);
\draw[line width=1.2pt,color=ffqqqq] (3.749999884471035,11.28920115569286) -- (3.759999883905374,11.222691161539684);
\draw[line width=1.2pt,color=ffqqqq] (3.759999883905374,11.222691161539684) -- (3.769999883339713,11.15600116740687);
\draw[line width=1.2pt,color=ffqqqq] (3.769999883339713,11.15600116740687) -- (3.7799998827740517,11.089131173294422);
\draw[line width=1.2pt,color=ffqqqq] (3.7799998827740517,11.089131173294422) -- (3.7899998822083907,11.022081179202338);
\draw[line width=1.2pt,color=ffqqqq] (3.7899998822083907,11.022081179202338) -- (3.7999998816427296,10.954851185130618);
\draw[line width=1.2pt,color=ffqqqq] (3.7999998816427296,10.954851185130618) -- (3.8099998810770686,10.887441191079262);
\draw[line width=1.2pt,color=ffqqqq] (3.8099998810770686,10.887441191079262) -- (3.8199998805114075,10.819851197048267);
\draw[line width=1.2pt,color=ffqqqq] (3.8199998805114075,10.819851197048267) -- (3.8299998799457464,10.752081203037639);
\draw[line width=1.2pt,color=ffqqqq] (3.8299998799457464,10.752081203037639) -- (3.8399998793800854,10.684131209047372);
\draw[line width=1.2pt,color=ffqqqq] (3.8399998793800854,10.684131209047372) -- (3.8499998788144243,10.61600121507747);
\draw[line width=1.2pt,color=ffqqqq] (3.8499998788144243,10.61600121507747) -- (3.8599998782487632,10.547691221127932);
\draw[line width=1.2pt,color=ffqqqq] (3.8599998782487632,10.547691221127932) -- (3.869999877683102,10.479201227198757);
\draw[line width=1.2pt,color=ffqqqq] (3.869999877683102,10.479201227198757) -- (3.879999877117441,10.410531233289948);
\draw[line width=1.2pt,color=ffqqqq] (3.879999877117441,10.410531233289948) -- (3.88999987655178,10.3416812394015);
\draw[line width=1.2pt,color=ffqqqq] (3.88999987655178,10.3416812394015) -- (3.899999875986119,10.272651245533417);
\draw[line width=1.2pt,color=ffqqqq] (3.899999875986119,10.272651245533417) -- (3.909999875420458,10.2034412516857);
\draw[line width=1.2pt,color=ffqqqq] (3.909999875420458,10.2034412516857) -- (3.919999874854797,10.134051257858342);
\draw[line width=1.2pt,color=ffqqqq] (3.919999874854797,10.134051257858342) -- (3.9299998742891358,10.06448126405135);
\draw[line width=1.2pt,color=ffqqqq] (3.9299998742891358,10.06448126405135) -- (3.9399998737234747,9.994731270264724);
\draw[line width=1.2pt,color=ffqqqq] (3.9399998737234747,9.994731270264724) -- (3.9499998731578136,9.92480127649846);
\draw[line width=1.2pt,color=ffqqqq] (3.9499998731578136,9.92480127649846) -- (3.9599998725921526,9.85469128275256);
\draw[line width=1.2pt,color=ffqqqq] (3.9599998725921526,9.85469128275256) -- (3.9699998720264915,9.784401289027024);
\draw[line width=1.2pt,color=ffqqqq] (3.9699998720264915,9.784401289027024) -- (3.9799998714608305,9.713931295321851);
\draw[line width=1.2pt,color=ffqqqq] (3.9799998714608305,9.713931295321851) -- (3.9899998708951694,9.643281301637042);
\draw[line width=1.2pt,color=ffqqqq] (3.9899998708951694,9.643281301637042) -- (3.9999998703295083,9.572451307972598);
\draw[line width=1.2pt,color=ffqqqq] (3.9999998703295083,9.572451307972598) -- (4.009999869763847,9.501441314328517);
\draw[line width=1.2pt,color=ffqqqq] (4.009999869763847,9.501441314328517) -- (4.019999869198186,9.430251320704802);
\draw[line width=1.2pt,color=ffqqqq] (4.019999869198186,9.430251320704802) -- (4.029999868632524,9.358881327101452);
\draw[line width=1.2pt,color=ffqqqq] (4.029999868632524,9.358881327101452) -- (4.039999868066863,9.287331333518466);
\draw[line width=1.2pt,color=ffqqqq] (4.039999868066863,9.287331333518466) -- (4.049999867501201,9.215601339955843);
\end{tikzpicture}
\end{center} 

\begin{enumerate}
\item Marie habite à 2,5~km d'un central téléphonique. Quel débit de connexion obtient-elle ? \point{2}
 
\item Paul obtient un débit de $20$ Mbits/s. 
À quelle distance du central téléphonique habite-t-il ? \point{2} 
\item Pour pouvoir recevoir la télévision par internet, le débit doit être au moins de $15$ Mbits/s. 
À quelle distance maximum du central doit-on habiter pour pouvoir recevoir la télévision par internet ?  \point{2}

 
\end{enumerate}

\end{ExoCad}


\begin{ExoCad}{Calculer.}{1234}{0}{0}{0}{0}{0}
 
\begin{minipage}{8cm}

On donne la courbe $\mathcal{C}_f$ d'une fonction $f$.

\begin{enumerate}
\item Comment se nomme la variable ? \point{1}
\item Quelle est son unité ? \point{1}
\item Que représente ce graphique ?  \point{2}
\item Quelle est la taille des pousses  le 9 ème jour ?\point{2}
\item Au bout de combien de jours les pousses dépassent 30 mm ?\point{2}
\item Déterminer un antécédent de $50$ par $f$ ?\point{2}
\item Déterminer une valeur approximative de l'image de $4$ par $f$ ?\point{1}

\end{enumerate}
\end{minipage}
\begin{minipage}{8cm}
\begin{center}
\definecolor{ccqqqq}{rgb}{0.8,0.,0.}
\definecolor{cqcqcq}{rgb}{0.7529411764705882,0.7529411764705882,0.7529411764705882}
\begin{tikzpicture}[line cap=round,line join=round,>=triangle 45,x=0.6837606837606837cm,y=0.1399215686274509cm]
\draw [color=cqcqcq,, xstep=0.6837606837606837cm,ystep=1.3992156862745089cm] (-0.88,-2.578475336322897) grid (10.82,54.59641255605382);
\draw[->,color=black] (-0.88,0.) -- (10.82,0.);
\foreach \x in {,1.,2.,3.,4.,5.,6.,7.,8.,9.,10.}
\draw[shift={(\x,0)},color=black] (0pt,2pt) -- (0pt,-2pt) node[below] {\footnotesize $\x$};
\draw[->,color=black] (0.,-2.578475336322897) -- (0.,54.59641255605382);
\foreach \y in {,10.,20.,30.,40.,50.}
\draw[shift={(0,\y)},color=black] (2pt,0pt) -- (-2pt,0pt) node[left] {\footnotesize $\y$};
\draw[color=black] (0pt,-10pt) node[right] {\footnotesize $0$};
\clip(-0.88,-2.578475336322897) rectangle (10.82,54.59641255605382);
\draw[line width=1.2pt,color=ccqqqq] (5.599999991374502E-8,0.0) -- (0.0,0.0);
\draw[line width=1.2pt,color=ccqqqq] (0.0,0.0) -- (0.02499999966110996,0.0);
\draw[line width=1.2pt,color=ccqqqq] (0.02499999966110996,0.0) -- (0.04999999932221992,0.0012499999661109962);
\draw[line width=1.2pt,color=ccqqqq] (0.04999999932221992,0.0012499999661109962) -- (0.07499999898332987,0.002812499923749741);
\draw[line width=1.2pt,color=ccqqqq] (0.07499999898332987,0.002812499923749741) -- (0.09999999864443984,0.004999999864443985);
\draw[line width=1.2pt,color=ccqqqq] (0.09999999864443984,0.004999999864443985) -- (0.12499999830554981,0.007812499788193728);
\draw[line width=1.2pt,color=ccqqqq] (0.12499999830554981,0.007812499788193728) -- (0.14999999796665978,0.011249999694998968);
\draw[line width=1.2pt,color=ccqqqq] (0.14999999796665978,0.011249999694998968) -- (0.17499999762776974,0.015312499584859708);
\draw[line width=1.2pt,color=ccqqqq] (0.17499999762776974,0.015312499584859708) -- (0.1999999972888797,0.019999999457775947);
\draw[line width=1.2pt,color=ccqqqq] (0.1999999972888797,0.019999999457775947) -- (0.22499999694998968,0.025312499313747683);
\draw[line width=1.2pt,color=ccqqqq] (0.22499999694998968,0.025312499313747683) -- (0.24999999661109965,0.031249999152774918);
\draw[line width=1.2pt,color=ccqqqq] (0.24999999661109965,0.031249999152774918) -- (0.2749999962722096,0.03781249897485765);
\draw[line width=1.2pt,color=ccqqqq] (0.2749999962722096,0.03781249897485765) -- (0.29999999593331955,0.04499999877999587);
\draw[line width=1.2pt,color=ccqqqq] (0.29999999593331955,0.04499999877999587) -- (0.3249999955944295,0.05281249856818959);
\draw[line width=1.2pt,color=ccqqqq] (0.3249999955944295,0.05281249856818959) -- (0.34999999525553943,0.06124999833943881);
\draw[line width=1.2pt,color=ccqqqq] (0.34999999525553943,0.06124999833943881) -- (0.37499999491664937,0.07031249809374353);
\draw[line width=1.2pt,color=ccqqqq] (0.37499999491664937,0.07031249809374353) -- (0.3999999945777593,0.07999999783110374);
\draw[line width=1.2pt,color=ccqqqq] (0.3999999945777593,0.07999999783110374) -- (0.42499999423886925,0.09031249755151945);
\draw[line width=1.2pt,color=ccqqqq] (0.42499999423886925,0.09031249755151945) -- (0.4499999938999792,0.10124999725499065);
\draw[line width=1.2pt,color=ccqqqq] (0.4499999938999792,0.10124999725499065) -- (0.47499999356108913,0.11281249694151736);
\draw[line width=1.2pt,color=ccqqqq] (0.47499999356108913,0.11281249694151736) -- (0.49999999322219907,0.12499999661109956);
\draw[line width=1.2pt,color=ccqqqq] (0.49999999322219907,0.12499999661109956) -- (0.524999992883309,0.13781249626373726);
\draw[line width=1.2pt,color=ccqqqq] (0.524999992883309,0.13781249626373726) -- (0.549999992544419,0.15124999589943047);
\draw[line width=1.2pt,color=ccqqqq] (0.549999992544419,0.15124999589943047) -- (0.574999992205529,0.1653124955181792);
\draw[line width=1.2pt,color=ccqqqq] (0.574999992205529,0.1653124955181792) -- (0.599999991866639,0.17999999511998344);
\draw[line width=1.2pt,color=ccqqqq] (0.599999991866639,0.17999999511998344) -- (0.624999991527749,0.19531249470484316);
\draw[line width=1.2pt,color=ccqqqq] (0.624999991527749,0.19531249470484316) -- (0.649999991188859,0.21124999427275837);
\draw[line width=1.2pt,color=ccqqqq] (0.649999991188859,0.21124999427275837) -- (0.674999990849969,0.2278124938237291);
\draw[line width=1.2pt,color=ccqqqq] (0.674999990849969,0.2278124938237291) -- (0.699999990511079,0.24499999335775532);
\draw[line width=1.2pt,color=ccqqqq] (0.699999990511079,0.24499999335775532) -- (0.724999990172189,0.26281249287483704);
\draw[line width=1.2pt,color=ccqqqq] (0.724999990172189,0.26281249287483704) -- (0.749999989833299,0.2812499923749743);
\draw[line width=1.2pt,color=ccqqqq] (0.749999989833299,0.2812499923749743) -- (0.774999989494409,0.300312491858167);
\draw[line width=1.2pt,color=ccqqqq] (0.774999989494409,0.300312491858167) -- (0.799999989155519,0.3199999913244152);
\draw[line width=1.2pt,color=ccqqqq] (0.799999989155519,0.3199999913244152) -- (0.824999988816629,0.34031249077371895);
\draw[line width=1.2pt,color=ccqqqq] (0.824999988816629,0.34031249077371895) -- (0.8499999884777389,0.3612499902060782);
\draw[line width=1.2pt,color=ccqqqq] (0.8499999884777389,0.3612499902060782) -- (0.8749999881388489,0.3828124896214929);
\draw[line width=1.2pt,color=ccqqqq] (0.8749999881388489,0.3828124896214929) -- (0.8999999877999589,0.4049999890199631);
\draw[line width=1.2pt,color=ccqqqq] (0.8999999877999589,0.4049999890199631) -- (0.9249999874610689,0.42781248840148883);
\draw[line width=1.2pt,color=ccqqqq] (0.9249999874610689,0.42781248840148883) -- (0.9499999871221789,0.45124998776607006);
\draw[line width=1.2pt,color=ccqqqq] (0.9499999871221789,0.45124998776607006) -- (0.9749999867832889,0.47531248711370677);
\draw[line width=1.2pt,color=ccqqqq] (0.9749999867832889,0.47531248711370677) -- (0.9999999864443989,0.499999986444399);
\draw[line width=1.2pt,color=ccqqqq] (0.9999999864443989,0.499999986444399) -- (1.0249999861055088,0.5253124857581466);
\draw[line width=1.2pt,color=ccqqqq] (1.0249999861055088,0.5253124857581466) -- (1.0499999857666187,0.5512499850549497);
\draw[line width=1.2pt,color=ccqqqq] (1.0499999857666187,0.5512499850549497) -- (1.0749999854277286,0.5778124843348084);
\draw[line width=1.2pt,color=ccqqqq] (1.0749999854277286,0.5778124843348084) -- (1.0999999850888385,0.6049999835977224);
\draw[line width=1.2pt,color=ccqqqq] (1.0999999850888385,0.6049999835977224) -- (1.1249999847499483,0.632812482843692);
\draw[line width=1.2pt,color=ccqqqq] (1.1249999847499483,0.632812482843692) -- (1.1499999844110582,0.6612499820727171);
\draw[line width=1.2pt,color=ccqqqq] (1.1499999844110582,0.6612499820727171) -- (1.174999984072168,0.6903124812847976);
\draw[line width=1.2pt,color=ccqqqq] (1.174999984072168,0.6903124812847976) -- (1.199999983733278,0.7199999804799337);
\draw[line width=1.2pt,color=ccqqqq] (1.199999983733278,0.7199999804799337) -- (1.2249999833943879,0.7503124796581253);
\draw[line width=1.2pt,color=ccqqqq] (1.2249999833943879,0.7503124796581253) -- (1.2499999830554978,0.7812499788193723);
\draw[line width=1.2pt,color=ccqqqq] (1.2499999830554978,0.7812499788193723) -- (1.2749999827166076,0.8128124779636748);
\draw[line width=1.2pt,color=ccqqqq] (1.2749999827166076,0.8128124779636748) -- (1.2999999823777175,0.8449999770910329);
\draw[line width=1.2pt,color=ccqqqq] (1.2999999823777175,0.8449999770910329) -- (1.3249999820388274,0.8778124762014464);
\draw[line width=1.2pt,color=ccqqqq] (1.3249999820388274,0.8778124762014464) -- (1.3499999816999373,0.9112499752949155);
\draw[line width=1.2pt,color=ccqqqq] (1.3499999816999373,0.9112499752949155) -- (1.3749999813610472,0.9453124743714401);
\draw[line width=1.2pt,color=ccqqqq] (1.3749999813610472,0.9453124743714401) -- (1.399999981022157,0.9799999734310201);
\draw[line width=1.2pt,color=ccqqqq] (1.399999981022157,0.9799999734310201) -- (1.424999980683267,1.0153124724736555);
\draw[line width=1.2pt,color=ccqqqq] (1.424999980683267,1.0153124724736555) -- (1.4499999803443768,1.0512499714993466);
\draw[line width=1.2pt,color=ccqqqq] (1.4499999803443768,1.0512499714993466) -- (1.4749999800054867,1.0878124705080932);
\draw[line width=1.2pt,color=ccqqqq] (1.4749999800054867,1.0878124705080932) -- (1.4999999796665966,1.1249999694998951);
\draw[line width=1.2pt,color=ccqqqq] (1.4999999796665966,1.1249999694998951) -- (1.5249999793277065,1.1628124684747525);
\draw[line width=1.2pt,color=ccqqqq] (1.5249999793277065,1.1628124684747525) -- (1.5499999789888164,1.2012499674326655);
\draw[line width=1.2pt,color=ccqqqq] (1.5499999789888164,1.2012499674326655) -- (1.5749999786499262,1.240312466373634);
\draw[line width=1.2pt,color=ccqqqq] (1.5749999786499262,1.240312466373634) -- (1.5999999783110361,1.2799999652976581);
\draw[line width=1.2pt,color=ccqqqq] (1.5999999783110361,1.2799999652976581) -- (1.624999977972146,1.3203124642047375);
\draw[line width=1.2pt,color=ccqqqq] (1.624999977972146,1.3203124642047375) -- (1.649999977633256,1.3612499630948725);
\draw[line width=1.2pt,color=ccqqqq] (1.649999977633256,1.3612499630948725) -- (1.6749999772943658,1.4028124619680629);
\draw[line width=1.2pt,color=ccqqqq] (1.6749999772943658,1.4028124619680629) -- (1.6999999769554757,1.444999960824309);
\draw[line width=1.2pt,color=ccqqqq] (1.6999999769554757,1.444999960824309) -- (1.7249999766165856,1.4878124596636104);
\draw[line width=1.2pt,color=ccqqqq] (1.7249999766165856,1.4878124596636104) -- (1.7499999762776954,1.5312499584859673);
\draw[line width=1.2pt,color=ccqqqq] (1.7499999762776954,1.5312499584859673) -- (1.7749999759388053,1.5753124572913797);
\draw[line width=1.2pt,color=ccqqqq] (1.7749999759388053,1.5753124572913797) -- (1.7999999755999152,1.6199999560798477);
\draw[line width=1.2pt,color=ccqqqq] (1.7999999755999152,1.6199999560798477) -- (1.824999975261025,1.6653124548513711);
\draw[line width=1.2pt,color=ccqqqq] (1.824999975261025,1.6653124548513711) -- (1.849999974922135,1.71124995360595);
\draw[line width=1.2pt,color=ccqqqq] (1.849999974922135,1.71124995360595) -- (1.8749999745832449,1.7578124523435845);
\draw[line width=1.2pt,color=ccqqqq] (1.8749999745832449,1.7578124523435845) -- (1.8999999742443547,1.8049999510642742);
\draw[line width=1.2pt,color=ccqqqq] (1.8999999742443547,1.8049999510642742) -- (1.9249999739054646,1.8528124497680198);
\draw[line width=1.2pt,color=ccqqqq] (1.9249999739054646,1.8528124497680198) -- (1.9499999735665745,1.9012499484548206);
\draw[line width=1.2pt,color=ccqqqq] (1.9499999735665745,1.9012499484548206) -- (1.9749999732276844,1.950312447124677);
\draw[line width=1.2pt,color=ccqqqq] (1.9749999732276844,1.950312447124677) -- (1.9999999728887943,1.999999945777589);
\draw[line width=1.2pt,color=ccqqqq] (1.9999999728887943,1.999999945777589) -- (2.0249999725499044,2.0503124444135565);
\draw[line width=1.2pt,color=ccqqqq] (2.0249999725499044,2.0503124444135565) -- (2.0499999722110145,2.10124994303258);
\draw[line width=1.2pt,color=ccqqqq] (2.0499999722110145,2.10124994303258) -- (2.0749999718721246,2.152812441634659);
\draw[line width=1.2pt,color=ccqqqq] (2.0749999718721246,2.152812441634659) -- (2.0999999715332347,2.204999940219793);
\draw[line width=1.2pt,color=ccqqqq] (2.0999999715332347,2.204999940219793) -- (2.124999971194345,2.257812438787983);
\draw[line width=1.2pt,color=ccqqqq] (2.124999971194345,2.257812438787983) -- (2.149999970855455,2.3112499373392286);
\draw[line width=1.2pt,color=ccqqqq] (2.149999970855455,2.3112499373392286) -- (2.174999970516565,2.365312435873529);
\draw[line width=1.2pt,color=ccqqqq] (2.174999970516565,2.365312435873529) -- (2.199999970177675,2.419999934390886);
\draw[line width=1.2pt,color=ccqqqq] (2.199999970177675,2.419999934390886) -- (2.2249999698387852,2.4753124328912977);
\draw[line width=1.2pt,color=ccqqqq] (2.2249999698387852,2.4753124328912977) -- (2.2499999694998953,2.531249931374765);
\draw[line width=1.2pt,color=ccqqqq] (2.2499999694998953,2.531249931374765) -- (2.2749999691610054,2.587812429841288);
\draw[line width=1.2pt,color=ccqqqq] (2.2749999691610054,2.587812429841288) -- (2.2999999688221155,2.6449999282908663);
\draw[line width=1.2pt,color=ccqqqq] (2.2999999688221155,2.6449999282908663) -- (2.3249999684832257,2.7028124267235003);
\draw[line width=1.2pt,color=ccqqqq] (2.3249999684832257,2.7028124267235003) -- (2.3499999681443358,2.7612499251391895);
\draw[line width=1.2pt,color=ccqqqq] (2.3499999681443358,2.7612499251391895) -- (2.374999967805446,2.8203124235379344);
\draw[line width=1.2pt,color=ccqqqq] (2.374999967805446,2.8203124235379344) -- (2.399999967466556,2.879999921919735);
\draw[line width=1.2pt,color=ccqqqq] (2.399999967466556,2.879999921919735) -- (2.424999967127666,2.9403124202845907);
\draw[line width=1.2pt,color=ccqqqq] (2.424999967127666,2.9403124202845907) -- (2.449999966788776,3.001249918632502);
\draw[line width=1.2pt,color=ccqqqq] (2.449999966788776,3.001249918632502) -- (2.4749999664498863,3.062812416963469);
\draw[line width=1.2pt,color=ccqqqq] (2.4749999664498863,3.062812416963469) -- (2.4999999661109964,3.1249999152774914);
\draw[line width=1.2pt,color=ccqqqq] (2.4999999661109964,3.1249999152774914) -- (2.5249999657721065,3.1878124135745693);
\draw[line width=1.2pt,color=ccqqqq] (2.5249999657721065,3.1878124135745693) -- (2.5499999654332166,3.251249911854703);
\draw[line width=1.2pt,color=ccqqqq] (2.5499999654332166,3.251249911854703) -- (2.5749999650943267,3.3153124101178917);
\draw[line width=1.2pt,color=ccqqqq] (2.5749999650943267,3.3153124101178917) -- (2.599999964755437,3.3799999083641366);
\draw[line width=1.2pt,color=ccqqqq] (2.599999964755437,3.3799999083641366) -- (2.624999964416547,3.445312406593436);
\draw[line width=1.2pt,color=ccqqqq] (2.624999964416547,3.445312406593436) -- (2.649999964077657,3.511249904805792);
\draw[line width=1.2pt,color=ccqqqq] (2.649999964077657,3.511249904805792) -- (2.674999963738767,3.577812403001203);
\draw[line width=1.2pt,color=ccqqqq] (2.674999963738767,3.577812403001203) -- (2.6999999633998772,3.6449999011796694);
\draw[line width=1.2pt,color=ccqqqq] (2.6999999633998772,3.6449999011796694) -- (2.7249999630609874,3.7128123993411912);
\draw[line width=1.2pt,color=ccqqqq] (2.7249999630609874,3.7128123993411912) -- (2.7499999627220975,3.7812498974857687);
\draw[line width=1.2pt,color=ccqqqq] (2.7499999627220975,3.7812498974857687) -- (2.7749999623832076,3.8503123956134018);
\draw[line width=1.2pt,color=ccqqqq] (2.7749999623832076,3.8503123956134018) -- (2.7999999620443177,3.91999989372409);
\draw[line width=1.2pt,color=ccqqqq] (2.7999999620443177,3.91999989372409) -- (2.8249999617054278,3.990312391817834);
\draw[line width=1.2pt,color=ccqqqq] (2.8249999617054278,3.990312391817834) -- (2.849999961366538,4.061249889894634);
\draw[line width=1.2pt,color=ccqqqq] (2.849999961366538,4.061249889894634) -- (2.874999961027648,4.1328123879544885);
\draw[line width=1.2pt,color=ccqqqq] (2.874999961027648,4.1328123879544885) -- (2.899999960688758,4.204999885997399);
\draw[line width=1.2pt,color=ccqqqq] (2.899999960688758,4.204999885997399) -- (2.924999960349868,4.277812384023365);
\draw[line width=1.2pt,color=ccqqqq] (2.924999960349868,4.277812384023365) -- (2.9499999600109783,4.351249882032387);
\draw[line width=1.2pt,color=ccqqqq] (2.9499999600109783,4.351249882032387) -- (2.9749999596720884,4.425312380024464);
\draw[line width=1.2pt,color=ccqqqq] (2.9749999596720884,4.425312380024464) -- (2.9999999593331985,4.499999877999596);
\draw[line width=1.2pt,color=ccqqqq] (2.9999999593331985,4.499999877999596) -- (3.0249999589943086,4.575312375957784);
\draw[line width=1.2pt,color=ccqqqq] (3.0249999589943086,4.575312375957784) -- (3.0499999586554187,4.651249873899028);
\draw[line width=1.2pt,color=ccqqqq] (3.0499999586554187,4.651249873899028) -- (3.074999958316529,4.727812371823327);
\draw[line width=1.2pt,color=ccqqqq] (3.074999958316529,4.727812371823327) -- (3.099999957977639,4.804999869730682);
\draw[line width=1.2pt,color=ccqqqq] (3.099999957977639,4.804999869730682) -- (3.124999957638749,4.882812367621091);
\draw[line width=1.2pt,color=ccqqqq] (3.124999957638749,4.882812367621091) -- (3.149999957299859,4.961249865494557);
\draw[line width=1.2pt,color=ccqqqq] (3.149999957299859,4.961249865494557) -- (3.1749999569609693,5.040312363351078);
\draw[line width=1.2pt,color=ccqqqq] (3.1749999569609693,5.040312363351078) -- (3.1999999566220794,5.119999861190655);
\draw[line width=1.2pt,color=ccqqqq] (3.1999999566220794,5.119999861190655) -- (3.2249999562831895,5.200312359013287);
\draw[line width=1.2pt,color=ccqqqq] (3.2249999562831895,5.200312359013287) -- (3.2499999559442996,5.281249856818975);
\draw[line width=1.2pt,color=ccqqqq] (3.2499999559442996,5.281249856818975) -- (3.2749999556054097,5.362812354607717);
\draw[line width=1.2pt,color=ccqqqq] (3.2749999556054097,5.362812354607717) -- (3.29999995526652,5.4449998523795164);
\draw[line width=1.2pt,color=ccqqqq] (3.29999995526652,5.4449998523795164) -- (3.32499995492763,5.52781235013437);
\draw[line width=1.2pt,color=ccqqqq] (3.32499995492763,5.52781235013437) -- (3.34999995458874,5.61124984787228);
\draw[line width=1.2pt,color=ccqqqq] (3.34999995458874,5.61124984787228) -- (3.37499995424985,5.695312345593245);
\draw[line width=1.2pt,color=ccqqqq] (3.37499995424985,5.695312345593245) -- (3.39999995391096,5.779999843297266);
\draw[line width=1.2pt,color=ccqqqq] (3.39999995391096,5.779999843297266) -- (3.4249999535720703,5.8653123409843415);
\draw[line width=1.2pt,color=ccqqqq] (3.4249999535720703,5.8653123409843415) -- (3.4499999532331804,5.951249838654474);
\draw[line width=1.2pt,color=ccqqqq] (3.4499999532331804,5.951249838654474) -- (3.4749999528942905,6.037812336307661);
\draw[line width=1.2pt,color=ccqqqq] (3.4749999528942905,6.037812336307661) -- (3.4999999525554006,6.124999833943903);
\draw[line width=1.2pt,color=ccqqqq] (3.4999999525554006,6.124999833943903) -- (3.5249999522165107,6.212812331563201);
\draw[line width=1.2pt,color=ccqqqq] (3.5249999522165107,6.212812331563201) -- (3.549999951877621,6.301249829165555);
\draw[line width=1.2pt,color=ccqqqq] (3.549999951877621,6.301249829165555) -- (3.574999951538731,6.390312326750965);
\draw[line width=1.2pt,color=ccqqqq] (3.574999951538731,6.390312326750965) -- (3.599999951199841,6.479999824319429);
\draw[line width=1.2pt,color=ccqqqq] (3.599999951199841,6.479999824319429) -- (3.624999950860951,6.570312321870949);
\draw[line width=1.2pt,color=ccqqqq] (3.624999950860951,6.570312321870949) -- (3.6499999505220613,6.6612498194055245);
\draw[line width=1.2pt,color=ccqqqq] (3.6499999505220613,6.6612498194055245) -- (3.6749999501831714,6.752812316923156);
\draw[line width=1.2pt,color=ccqqqq] (3.6749999501831714,6.752812316923156) -- (3.6999999498442815,6.844999814423843);
\draw[line width=1.2pt,color=ccqqqq] (3.6999999498442815,6.844999814423843) -- (3.7249999495053916,6.937812311907585);
\draw[line width=1.2pt,color=ccqqqq] (3.7249999495053916,6.937812311907585) -- (3.7499999491665017,7.0312498093743825);
\draw[line width=1.2pt,color=ccqqqq] (3.7499999491665017,7.0312498093743825) -- (3.774999948827612,7.125312306824235);
\draw[line width=1.2pt,color=ccqqqq] (3.774999948827612,7.125312306824235) -- (3.799999948488722,7.219999804257145);
\draw[line width=1.2pt,color=ccqqqq] (3.799999948488722,7.219999804257145) -- (3.824999948149832,7.315312301673109);
\draw[line width=1.2pt,color=ccqqqq] (3.824999948149832,7.315312301673109) -- (3.849999947810942,7.411249799072128);
\draw[line width=1.2pt,color=ccqqqq] (3.849999947810942,7.411249799072128) -- (3.8749999474720522,7.507812296454204);
\draw[line width=1.2pt,color=ccqqqq] (3.8749999474720522,7.507812296454204) -- (3.8999999471331623,7.604999793819334);
\draw[line width=1.2pt,color=ccqqqq] (3.8999999471331623,7.604999793819334) -- (3.9249999467942724,7.702812291167521);
\draw[line width=1.2pt,color=ccqqqq] (3.9249999467942724,7.702812291167521) -- (3.9499999464553825,7.801249788498763);
\draw[line width=1.2pt,color=ccqqqq] (3.9499999464553825,7.801249788498763) -- (3.9749999461164927,7.90031228581306);
\draw[line width=1.2pt,color=ccqqqq] (3.9749999461164927,7.90031228581306) -- (3.9999999457776028,7.999999783110413);
\draw[line width=1.2pt,color=ccqqqq] (3.9999999457776028,7.999999783110413) -- (4.024999945438712,8.100312280390819);
\draw[line width=1.2pt,color=ccqqqq] (4.024999945438712,8.100312280390819) -- (4.049999945099822,8.201249777654281);
\draw[line width=1.2pt,color=ccqqqq] (4.049999945099822,8.201249777654281) -- (4.074999944760932,8.302812274900798);
\draw[line width=1.2pt,color=ccqqqq] (4.074999944760932,8.302812274900798) -- (4.099999944422041,8.404999772130372);
\draw[line width=1.2pt,color=ccqqqq] (4.099999944422041,8.404999772130372) -- (4.124999944083151,8.507812269342999);
\draw[line width=1.2pt,color=ccqqqq] (4.124999944083151,8.507812269342999) -- (4.149999943744261,8.611249766538684);
\draw[line width=1.2pt,color=ccqqqq] (4.149999943744261,8.611249766538684) -- (4.17499994340537,8.715312263717424);
\draw[line width=1.2pt,color=ccqqqq] (4.17499994340537,8.715312263717424) -- (4.19999994306648,8.819999760879218);
\draw[line width=1.2pt,color=ccqqqq] (4.19999994306648,8.819999760879218) -- (4.22499994272759,8.925312258024068);
\draw[line width=1.2pt,color=ccqqqq] (4.22499994272759,8.925312258024068) -- (4.249999942388699,9.031249755151974);
\draw[line width=1.2pt,color=ccqqqq] (4.249999942388699,9.031249755151974) -- (4.274999942049809,9.137812252262936);
\draw[line width=1.2pt,color=ccqqqq] (4.274999942049809,9.137812252262936) -- (4.299999941710919,9.244999749356952);
\draw[line width=1.2pt,color=ccqqqq] (4.299999941710919,9.244999749356952) -- (4.324999941372028,9.352812246434024);
\draw[line width=1.2pt,color=ccqqqq] (4.324999941372028,9.352812246434024) -- (4.349999941033138,9.461249743494152);
\draw[line width=1.2pt,color=ccqqqq] (4.349999941033138,9.461249743494152) -- (4.374999940694248,9.570312240537335);
\draw[line width=1.2pt,color=ccqqqq] (4.374999940694248,9.570312240537335) -- (4.399999940355357,9.679999737563573);
\draw[line width=1.2pt,color=ccqqqq] (4.399999940355357,9.679999737563573) -- (4.424999940016467,9.790312234572868);
\draw[line width=1.2pt,color=ccqqqq] (4.424999940016467,9.790312234572868) -- (4.449999939677577,9.901249731565217);
\draw[line width=1.2pt,color=ccqqqq] (4.449999939677577,9.901249731565217) -- (4.474999939338686,10.012812228540623);
\draw[line width=1.2pt,color=ccqqqq] (4.474999939338686,10.012812228540623) -- (4.499999938999796,10.124999725499084);
\draw[line width=1.2pt,color=ccqqqq] (4.499999938999796,10.124999725499084) -- (4.524999938660906,10.2378122224406);
\draw[line width=1.2pt,color=ccqqqq] (4.524999938660906,10.2378122224406) -- (4.549999938322015,10.351249719365171);
\draw[line width=1.2pt,color=ccqqqq] (4.549999938322015,10.351249719365171) -- (4.574999937983125,10.4653122162728);
\draw[line width=1.2pt,color=ccqqqq] (4.574999937983125,10.4653122162728) -- (4.599999937644235,10.579999713163481);
\draw[line width=1.2pt,color=ccqqqq] (4.599999937644235,10.579999713163481) -- (4.624999937305344,10.695312210037219);
\draw[line width=1.2pt,color=ccqqqq] (4.624999937305344,10.695312210037219) -- (4.649999936966454,10.811249706894014);
\draw[line width=1.2pt,color=ccqqqq] (4.649999936966454,10.811249706894014) -- (4.674999936627564,10.927812203733861);
\draw[line width=1.2pt,color=ccqqqq] (4.674999936627564,10.927812203733861) -- (4.699999936288673,11.044999700556767);
\draw[line width=1.2pt,color=ccqqqq] (4.699999936288673,11.044999700556767) -- (4.724999935949783,11.162812197362726);
\draw[line width=1.2pt,color=ccqqqq] (4.724999935949783,11.162812197362726) -- (4.749999935610893,11.281249694151741);
\draw[line width=1.2pt,color=ccqqqq] (4.749999935610893,11.281249694151741) -- (4.774999935272002,11.400312190923813);
\draw[line width=1.2pt,color=ccqqqq] (4.774999935272002,11.400312190923813) -- (4.799999934933112,11.51999968767894);
\draw[line width=1.2pt,color=ccqqqq] (4.799999934933112,11.51999968767894) -- (4.824999934594222,11.640312184417121);
\draw[line width=1.2pt,color=ccqqqq] (4.824999934594222,11.640312184417121) -- (4.849999934255331,11.76124968113836);
\draw[line width=1.2pt,color=ccqqqq] (4.849999934255331,11.76124968113836) -- (4.874999933916441,11.882812177842652);
\draw[line width=1.2pt,color=ccqqqq] (4.874999933916441,11.882812177842652) -- (4.899999933577551,12.00499967453);
\draw[line width=1.2pt,color=ccqqqq] (4.899999933577551,12.00499967453) -- (4.92499993323866,12.127812171200404);
\draw[line width=1.2pt,color=ccqqqq] (4.92499993323866,12.127812171200404) -- (4.94999993289977,12.251249667853862);
\draw[line width=1.2pt,color=ccqqqq] (4.94999993289977,12.251249667853862) -- (4.97499993256088,12.375312164490378);
\draw[line width=1.2pt,color=ccqqqq] (4.97499993256088,12.375312164490378) -- (4.999999932221989,12.499999661109948);
\draw[line width=1.2pt,color=ccqqqq] (4.999999932221989,12.499999661109948) -- (5.024999931883099,12.625312157712575);
\draw[line width=1.2pt,color=ccqqqq] (5.024999931883099,12.625312157712575) -- (5.049999931544209,12.751249654298256);
\draw[line width=1.2pt,color=ccqqqq] (5.049999931544209,12.751249654298256) -- (5.074999931205318,12.877812150866992);
\draw[line width=1.2pt,color=ccqqqq] (5.074999931205318,12.877812150866992) -- (5.099999930866428,13.004999647418785);
\draw[line width=1.2pt,color=ccqqqq] (5.099999930866428,13.004999647418785) -- (5.1249999305275376,13.132812143953632);
\draw[line width=1.2pt,color=ccqqqq] (5.1249999305275376,13.132812143953632) -- (5.149999930188647,13.261249640471535);
\draw[line width=1.2pt,color=ccqqqq] (5.149999930188647,13.261249640471535) -- (5.174999929849757,13.390312136972494);
\draw[line width=1.2pt,color=ccqqqq] (5.174999929849757,13.390312136972494) -- (5.1999999295108665,13.519999633456509);
\draw[line width=1.2pt,color=ccqqqq] (5.1999999295108665,13.519999633456509) -- (5.224999929171976,13.650312129923579);
\draw[line width=1.2pt,color=ccqqqq] (5.224999929171976,13.650312129923579) -- (5.249999928833086,13.781249626373704);
\draw[line width=1.2pt,color=ccqqqq] (5.249999928833086,13.781249626373704) -- (5.2749999284941955,13.912812122806884);
\draw[line width=1.2pt,color=ccqqqq] (5.2749999284941955,13.912812122806884) -- (5.299999928155305,14.04499961922312);
\draw[line width=1.2pt,color=ccqqqq] (5.299999928155305,14.04499961922312) -- (5.324999927816415,14.177812115622412);
\draw[line width=1.2pt,color=ccqqqq] (5.324999927816415,14.177812115622412) -- (5.3499999274775245,14.311249612004758);
\draw[line width=1.2pt,color=ccqqqq] (5.3499999274775245,14.311249612004758) -- (5.374999927138634,14.445312108370162);
\draw[line width=1.2pt,color=ccqqqq] (5.374999927138634,14.445312108370162) -- (5.399999926799744,14.579999604718619);
\draw[line width=1.2pt,color=ccqqqq] (5.399999926799744,14.579999604718619) -- (5.4249999264608535,14.715312101050133);
\draw[line width=1.2pt,color=ccqqqq] (5.4249999264608535,14.715312101050133) -- (5.449999926121963,14.851249597364703);
\draw[line width=1.2pt,color=ccqqqq] (5.449999926121963,14.851249597364703) -- (5.474999925783073,14.987812093662326);
\draw[line width=1.2pt,color=ccqqqq] (5.474999925783073,14.987812093662326) -- (5.4999999254441825,15.124999589943007);
\draw[line width=1.2pt,color=ccqqqq] (5.4999999254441825,15.124999589943007) -- (5.524999925105292,15.262812086206742);
\draw[line width=1.2pt,color=ccqqqq] (5.524999925105292,15.262812086206742) -- (5.549999924766402,15.401249582453532);
\draw[line width=1.2pt,color=ccqqqq] (5.549999924766402,15.401249582453532) -- (5.5749999244275115,15.54031207868338);
\draw[line width=1.2pt,color=ccqqqq] (5.5749999244275115,15.54031207868338) -- (5.599999924088621,15.67999957489628);
\draw[line width=1.2pt,color=ccqqqq] (5.599999924088621,15.67999957489628) -- (5.624999923749731,15.82031207109224);
\draw[line width=1.2pt,color=ccqqqq] (5.624999923749731,15.82031207109224) -- (5.6499999234108405,15.96124956727125);
\draw[line width=1.2pt,color=ccqqqq] (5.6499999234108405,15.96124956727125) -- (5.67499992307195,16.10281206343332);
\draw[line width=1.2pt,color=ccqqqq] (5.67499992307195,16.10281206343332) -- (5.69999992273306,16.244999559578442);
\draw[line width=1.2pt,color=ccqqqq] (5.69999992273306,16.244999559578442) -- (5.724999922394169,16.387812055706622);
\draw[line width=1.2pt,color=ccqqqq] (5.724999922394169,16.387812055706622) -- (5.749999922055279,16.531249551817858);
\draw[line width=1.2pt,color=ccqqqq] (5.749999922055279,16.531249551817858) -- (5.774999921716389,16.675312047912147);
\draw[line width=1.2pt,color=ccqqqq] (5.774999921716389,16.675312047912147) -- (5.799999921377498,16.819999543989493);
\draw[line width=1.2pt,color=ccqqqq] (5.799999921377498,16.819999543989493) -- (5.824999921038608,16.965312040049895);
\draw[line width=1.2pt,color=ccqqqq] (5.824999921038608,16.965312040049895) -- (5.849999920699718,17.111249536093354);
\draw[line width=1.2pt,color=ccqqqq] (5.849999920699718,17.111249536093354) -- (5.874999920360827,17.257812032119865);
\draw[line width=1.2pt,color=ccqqqq] (5.874999920360827,17.257812032119865) -- (5.899999920021937,17.404999528129434);
\draw[line width=1.2pt,color=ccqqqq] (5.899999920021937,17.404999528129434) -- (5.924999919683047,17.552812024122055);
\draw[line width=1.2pt,color=ccqqqq] (5.924999919683047,17.552812024122055) -- (5.949999919344156,17.701249520097733);
\draw[line width=1.2pt,color=ccqqqq] (5.949999919344156,17.701249520097733) -- (5.974999919005266,17.850312016056467);
\draw[line width=1.2pt,color=ccqqqq] (5.974999919005266,17.850312016056467) -- (5.999999918666376,17.999999511998258);
\draw[line width=1.2pt,color=ccqqqq] (5.999999918666376,17.999999511998258) -- (6.024999918327485,18.1503120079231);
\draw[line width=1.2pt,color=ccqqqq] (6.024999918327485,18.1503120079231) -- (6.049999917988595,18.301249503831002);
\draw[line width=1.2pt,color=ccqqqq] (6.049999917988595,18.301249503831002) -- (6.074999917649705,18.45281199972196);
\draw[line width=1.2pt,color=ccqqqq] (6.074999917649705,18.45281199972196) -- (6.099999917310814,18.604999495595973);
\draw[line width=1.2pt,color=ccqqqq] (6.099999917310814,18.604999495595973) -- (6.124999916971924,18.75781199145304);
\draw[line width=1.2pt,color=ccqqqq] (6.124999916971924,18.75781199145304) -- (6.149999916633034,18.911249487293162);
\draw[line width=1.2pt,color=ccqqqq] (6.149999916633034,18.911249487293162) -- (6.174999916294143,19.065311983116338);
\draw[line width=1.2pt,color=ccqqqq] (6.174999916294143,19.065311983116338) -- (6.199999915955253,19.21999947892257);
\draw[line width=1.2pt,color=ccqqqq] (6.199999915955253,19.21999947892257) -- (6.224999915616363,19.37531197471186);
\draw[line width=1.2pt,color=ccqqqq] (6.224999915616363,19.37531197471186) -- (6.249999915277472,19.531249470484205);
\draw[line width=1.2pt,color=ccqqqq] (6.249999915277472,19.531249470484205) -- (6.274999914938582,19.687811966239607);
\draw[line width=1.2pt,color=ccqqqq] (6.274999914938582,19.687811966239607) -- (6.299999914599692,19.844999461978063);
\draw[line width=1.2pt,color=ccqqqq] (6.299999914599692,19.844999461978063) -- (6.324999914260801,20.00281195769957);
\draw[line width=1.2pt,color=ccqqqq] (6.324999914260801,20.00281195769957) -- (6.349999913921911,20.16124945340414);
\draw[line width=1.2pt,color=ccqqqq] (6.349999913921911,20.16124945340414) -- (6.374999913583021,20.32031194909176);
\draw[line width=1.2pt,color=ccqqqq] (6.374999913583021,20.32031194909176) -- (6.39999991324413,20.479999444762438);
\draw[line width=1.2pt,color=ccqqqq] (6.39999991324413,20.479999444762438) -- (6.42499991290524,20.640311940416172);
\draw[line width=1.2pt,color=ccqqqq] (6.42499991290524,20.640311940416172) -- (6.44999991256635,20.80124943605296);
\draw[line width=1.2pt,color=ccqqqq] (6.44999991256635,20.80124943605296) -- (6.474999912227459,20.962811931672803);
\draw[line width=1.2pt,color=ccqqqq] (6.474999912227459,20.962811931672803) -- (6.499999911888569,21.124999427275704);
\draw[line width=1.2pt,color=ccqqqq] (6.499999911888569,21.124999427275704) -- (6.524999911549679,21.287811922861657);
\draw[line width=1.2pt,color=ccqqqq] (6.524999911549679,21.287811922861657) -- (6.549999911210788,21.451249418430667);
\draw[line width=1.2pt,color=ccqqqq] (6.549999911210788,21.451249418430667) -- (6.574999910871898,21.615311913982733);
\draw[line width=1.2pt,color=ccqqqq] (6.574999910871898,21.615311913982733) -- (6.599999910533008,21.779999409517853);
\draw[line width=1.2pt,color=ccqqqq] (6.599999910533008,21.779999409517853) -- (6.624999910194117,21.945311905036032);
\draw[line width=1.2pt,color=ccqqqq] (6.624999910194117,21.945311905036032) -- (6.649999909855227,22.111249400537265);
\draw[line width=1.2pt,color=ccqqqq] (6.649999909855227,22.111249400537265) -- (6.674999909516337,22.27781189602155);
\draw[line width=1.2pt,color=ccqqqq] (6.674999909516337,22.27781189602155) -- (6.699999909177446,22.444999391488896);
\draw[line width=1.2pt,color=ccqqqq] (6.699999909177446,22.444999391488896) -- (6.724999908838556,22.612811886939294);
\draw[line width=1.2pt,color=ccqqqq] (6.724999908838556,22.612811886939294) -- (6.749999908499666,22.781249382372746);
\draw[line width=1.2pt,color=ccqqqq] (6.749999908499666,22.781249382372746) -- (6.774999908160775,22.950311877789257);
\draw[line width=1.2pt,color=ccqqqq] (6.774999908160775,22.950311877789257) -- (6.799999907821885,23.119999373188822);
\draw[line width=1.2pt,color=ccqqqq] (6.799999907821885,23.119999373188822) -- (6.824999907482995,23.290311868571443);
\draw[line width=1.2pt,color=ccqqqq] (6.824999907482995,23.290311868571443) -- (6.849999907144104,23.461249363937117);
\draw[line width=1.2pt,color=ccqqqq] (6.849999907144104,23.461249363937117) -- (6.874999906805214,23.63281185928585);
\draw[line width=1.2pt,color=ccqqqq] (6.874999906805214,23.63281185928585) -- (6.8999999064663236,23.804999354617635);
\draw[line width=1.2pt,color=ccqqqq] (6.8999999064663236,23.804999354617635) -- (6.924999906127433,23.97781184993248);
\draw[line width=1.2pt,color=ccqqqq] (6.924999906127433,23.97781184993248) -- (6.949999905788543,24.151249345230376);
\draw[line width=1.2pt,color=ccqqqq] (6.949999905788543,24.151249345230376) -- (6.9749999054496525,24.325311840511333);
\draw[line width=1.2pt,color=ccqqqq] (6.9749999054496525,24.325311840511333) -- (6.999999905110762,24.49999933577534);
\draw[line width=1.2pt,color=ccqqqq] (6.999999905110762,24.49999933577534) -- (7.024999904771872,24.675311831022405);
\draw[line width=1.2pt,color=ccqqqq] (7.024999904771872,24.675311831022405) -- (7.0499999044329815,24.851249326252525);
\draw[line width=1.2pt,color=ccqqqq] (7.0499999044329815,24.851249326252525) -- (7.074999904094091,25.0278118214657);
\draw[line width=1.2pt,color=ccqqqq] (7.074999904094091,25.0278118214657) -- (7.099999903755201,25.20499931666193);
\draw[line width=1.2pt,color=ccqqqq] (7.099999903755201,25.20499931666193) -- (7.1249999034163105,25.38281181184122);
\draw[line width=1.2pt,color=ccqqqq] (7.1249999034163105,25.38281181184122) -- (7.14999990307742,25.56124930700356);
\draw[line width=1.2pt,color=ccqqqq] (7.14999990307742,25.56124930700356) -- (7.17499990273853,25.740311802148955);
\draw[line width=1.2pt,color=ccqqqq] (7.17499990273853,25.740311802148955) -- (7.1999999023996395,25.91999929727741);
\draw[line width=1.2pt,color=ccqqqq] (7.1999999023996395,25.91999929727741) -- (7.224999902060749,26.10031179238892);
\draw[line width=1.2pt,color=ccqqqq] (7.224999902060749,26.10031179238892) -- (7.249999901721859,26.28124928748348);
\draw[line width=1.2pt,color=ccqqqq] (7.249999901721859,26.28124928748348) -- (7.2749999013829685,26.4628117825611);
\draw[line width=1.2pt,color=ccqqqq] (7.2749999013829685,26.4628117825611) -- (7.299999901044078,26.644999277621775);
\draw[line width=1.2pt,color=ccqqqq] (7.299999901044078,26.644999277621775) -- (7.324999900705188,26.827811772665505);
\draw[line width=1.2pt,color=ccqqqq] (7.324999900705188,26.827811772665505) -- (7.3499999003662975,27.011249267692293);
\draw[line width=1.2pt,color=ccqqqq] (7.3499999003662975,27.011249267692293) -- (7.374999900027407,27.195311762702133);
\draw[line width=1.2pt,color=ccqqqq] (7.374999900027407,27.195311762702133) -- (7.399999899688517,27.37999925769503);
\draw[line width=1.2pt,color=ccqqqq] (7.399999899688517,27.37999925769503) -- (7.4249998993496265,27.565311752670983);
\draw[line width=1.2pt,color=ccqqqq] (7.4249998993496265,27.565311752670983) -- (7.449999899010736,27.75124924762999);
\draw[line width=1.2pt,color=ccqqqq] (7.449999899010736,27.75124924762999) -- (7.474999898671846,27.937811742572052);
\draw[line width=1.2pt,color=ccqqqq] (7.474999898671846,27.937811742572052) -- (7.499999898332955,28.12499923749717);
\draw[line width=1.2pt,color=ccqqqq] (7.499999898332955,28.12499923749717) -- (7.524999897994065,28.312811732405343);
\draw[line width=1.2pt,color=ccqqqq] (7.524999897994065,28.312811732405343) -- (7.549999897655175,28.501249227296576);
\draw[line width=1.2pt,color=ccqqqq] (7.549999897655175,28.501249227296576) -- (7.574999897316284,28.69031172217086);
\draw[line width=1.2pt,color=ccqqqq] (7.574999897316284,28.69031172217086) -- (7.599999896977394,28.8799992170282);
\draw[line width=1.2pt,color=ccqqqq] (7.599999896977394,28.8799992170282) -- (7.624999896638504,29.070311711868598);
\draw[line width=1.2pt,color=ccqqqq] (7.624999896638504,29.070311711868598) -- (7.649999896299613,29.26124920669205);
\draw[line width=1.2pt,color=ccqqqq] (7.649999896299613,29.26124920669205) -- (7.674999895960723,29.452811701498554);
\draw[line width=1.2pt,color=ccqqqq] (7.674999895960723,29.452811701498554) -- (7.699999895621833,29.64499919628812);
\draw[line width=1.2pt,color=ccqqqq] (7.699999895621833,29.64499919628812) -- (7.724999895282942,29.837811691060736);
\draw[line width=1.2pt,color=ccqqqq] (7.724999895282942,29.837811691060736) -- (7.749999894944052,30.03124918581641);
\draw[line width=1.2pt,color=ccqqqq] (7.749999894944052,30.03124918581641) -- (7.774999894605162,30.225311680555137);
\draw[line width=1.2pt,color=ccqqqq] (7.774999894605162,30.225311680555137) -- (7.799999894266271,30.41999917527692);
\draw[line width=1.2pt,color=ccqqqq] (7.799999894266271,30.41999917527692) -- (7.824999893927381,30.61531166998176);
\draw[line width=1.2pt,color=ccqqqq] (7.824999893927381,30.61531166998176) -- (7.849999893588491,30.811249164669658);
\draw[line width=1.2pt,color=ccqqqq] (7.849999893588491,30.811249164669658) -- (7.8749998932496,31.007811659340607);
\draw[line width=1.2pt,color=ccqqqq] (7.8749998932496,31.007811659340607) -- (7.89999989291071,31.204999153994613);
\draw[line width=1.2pt,color=ccqqqq] (7.89999989291071,31.204999153994613) -- (7.92499989257182,31.402811648631676);
\draw[line width=1.2pt,color=ccqqqq] (7.92499989257182,31.402811648631676) -- (7.949999892232929,31.601249143251795);
\draw[line width=1.2pt,color=ccqqqq] (7.949999892232929,31.601249143251795) -- (7.974999891894039,31.800311637854968);
\draw[line width=1.2pt,color=ccqqqq] (7.974999891894039,31.800311637854968) -- (7.999999891555149,31.999999132441197);
\draw[line width=1.2pt,color=ccqqqq] (7.999999891555149,31.999999132441197) -- (8.02499989121626,32.200311627010485);
\draw[line width=1.2pt,color=ccqqqq] (8.02499989121626,32.200311627010485) -- (8.04999989087737,32.401249121562834);
\draw[line width=1.2pt,color=ccqqqq] (8.04999989087737,32.401249121562834) -- (8.07499989053848,32.602811616098236);
\draw[line width=1.2pt,color=ccqqqq] (8.07499989053848,32.602811616098236) -- (8.09999989019959,32.80499911061669);
\draw[line width=1.2pt,color=ccqqqq] (8.09999989019959,32.80499911061669) -- (8.124999889860701,33.007811605118206);
\draw[line width=1.2pt,color=ccqqqq] (8.124999889860701,33.007811605118206) -- (8.149999889521812,33.211249099602774);
\draw[line width=1.2pt,color=ccqqqq] (8.149999889521812,33.211249099602774) -- (8.174999889182923,33.415311594070396);
\draw[line width=1.2pt,color=ccqqqq] (8.174999889182923,33.415311594070396) -- (8.199999888844033,33.61999908852108);
\draw[line width=1.2pt,color=ccqqqq] (8.199999888844033,33.61999908852108) -- (8.224999888505144,33.82531158295481);
\draw[line width=1.2pt,color=ccqqqq] (8.224999888505144,33.82531158295481) -- (8.249999888166254,34.031249077371605);
\draw[line width=1.2pt,color=ccqqqq] (8.249999888166254,34.031249077371605) -- (8.274999887827365,34.23781157177145);
\draw[line width=1.2pt,color=ccqqqq] (8.274999887827365,34.23781157177145) -- (8.299999887488475,34.44499906615435);
\draw[line width=1.2pt,color=ccqqqq] (8.299999887488475,34.44499906615435) -- (8.324999887149586,34.652811560520306);
\draw[line width=1.2pt,color=ccqqqq] (8.324999887149586,34.652811560520306) -- (8.349999886810696,34.86124905486932);
\draw[line width=1.2pt,color=ccqqqq] (8.349999886810696,34.86124905486932) -- (8.374999886471807,35.07031154920139);
\draw[line width=1.2pt,color=ccqqqq] (8.374999886471807,35.07031154920139) -- (8.399999886132917,35.27999904351651);
\draw[line width=1.2pt,color=ccqqqq] (8.399999886132917,35.27999904351651) -- (8.424999885794028,35.49031153781469);
\draw[line width=1.2pt,color=ccqqqq] (8.424999885794028,35.49031153781469) -- (8.449999885455139,35.70124903209593);
\draw[line width=1.2pt,color=ccqqqq] (8.449999885455139,35.70124903209593) -- (8.47499988511625,35.912811526360215);
\draw[line width=1.2pt,color=ccqqqq] (8.47499988511625,35.912811526360215) -- (8.49999988477736,36.12499902060756);
\draw[line width=1.2pt,color=ccqqqq] (8.49999988477736,36.12499902060756) -- (8.52499988443847,36.337811514837966);
\draw[line width=1.2pt,color=ccqqqq] (8.52499988443847,36.337811514837966) -- (8.54999988409958,36.551249009051425);
\draw[line width=1.2pt,color=ccqqqq] (8.54999988409958,36.551249009051425) -- (8.574999883760691,36.765311503247936);
\draw[line width=1.2pt,color=ccqqqq] (8.574999883760691,36.765311503247936) -- (8.599999883421802,36.9799989974275);
\draw[line width=1.2pt,color=ccqqqq] (8.599999883421802,36.9799989974275) -- (8.624999883082912,37.195311491590125);
\draw[line width=1.2pt,color=ccqqqq] (8.624999883082912,37.195311491590125) -- (8.649999882744023,37.4112489857358);
\draw[line width=1.2pt,color=ccqqqq] (8.649999882744023,37.4112489857358) -- (8.674999882405134,37.62781147986454);
\draw[line width=1.2pt,color=ccqqqq] (8.674999882405134,37.62781147986454) -- (8.699999882066244,37.84499897397633);
\draw[line width=1.2pt,color=ccqqqq] (8.699999882066244,37.84499897397633) -- (8.724999881727355,38.062811468071175);
\draw[line width=1.2pt,color=ccqqqq] (8.724999881727355,38.062811468071175) -- (8.749999881388465,38.28124896214908);
\draw[line width=1.2pt,color=ccqqqq] (8.749999881388465,38.28124896214908) -- (8.774999881049576,38.500311456210035);
\draw[line width=1.2pt,color=ccqqqq] (8.774999881049576,38.500311456210035) -- (8.799999880710686,38.719998950254045);
\draw[line width=1.2pt,color=ccqqqq] (8.799999880710686,38.719998950254045) -- (8.824999880371797,38.940311444281114);
\draw[line width=1.2pt,color=ccqqqq] (8.824999880371797,38.940311444281114) -- (8.849999880032907,39.16124893829124);
\draw[line width=1.2pt,color=ccqqqq] (8.849999880032907,39.16124893829124) -- (8.874999879694018,39.38281143228441);
\draw[line width=1.2pt,color=ccqqqq] (8.874999879694018,39.38281143228441) -- (8.899999879355128,39.60499892626065);
\draw[line width=1.2pt,color=ccqqqq] (8.899999879355128,39.60499892626065) -- (8.924999879016239,39.82781142021994);
\draw[line width=1.2pt,color=ccqqqq] (8.924999879016239,39.82781142021994) -- (8.94999987867735,40.05124891416229);
\draw[line width=1.2pt,color=ccqqqq] (8.94999987867735,40.05124891416229) -- (8.97499987833846,40.27531140808769);
\draw[line width=1.2pt,color=ccqqqq] (8.97499987833846,40.27531140808769) -- (8.99999987799957,40.49999890199614);
\draw[line width=1.2pt,color=ccqqqq] (8.99999987799957,40.49999890199614) -- (9.024999877660681,40.72531139588766);
\draw[line width=1.2pt,color=ccqqqq] (9.024999877660681,40.72531139588766) -- (9.049999877321792,40.951248889762226);
\draw[line width=1.2pt,color=ccqqqq] (9.049999877321792,40.951248889762226) -- (9.074999876982902,41.17781138361985);
\draw[line width=1.2pt,color=ccqqqq] (9.074999876982902,41.17781138361985) -- (9.099999876644013,41.40499887746053);
\draw[line width=1.2pt,color=ccqqqq] (9.099999876644013,41.40499887746053) -- (9.124999876305123,41.63281137128426);
\draw[line width=1.2pt,color=ccqqqq] (9.124999876305123,41.63281137128426) -- (9.149999875966234,41.86124886509105);
\draw[line width=1.2pt,color=ccqqqq] (9.149999875966234,41.86124886509105) -- (9.174999875627345,42.090311358880896);
\draw[line width=1.2pt,color=ccqqqq] (9.174999875627345,42.090311358880896) -- (9.199999875288455,42.3199988526538);
\draw[line width=1.2pt,color=ccqqqq] (9.199999875288455,42.3199988526538) -- (9.224999874949566,42.55031134640975);
\draw[line width=1.2pt,color=ccqqqq] (9.224999874949566,42.55031134640975) -- (9.249999874610676,42.78124884014876);
\draw[line width=1.2pt,color=ccqqqq] (9.249999874610676,42.78124884014876) -- (9.274999874271787,43.01281133387083);
\draw[line width=1.2pt,color=ccqqqq] (9.274999874271787,43.01281133387083) -- (9.299999873932897,43.244998827575955);
\draw[line width=1.2pt,color=ccqqqq] (9.299999873932897,43.244998827575955) -- (9.324999873594008,43.477811321264134);
\draw[line width=1.2pt,color=ccqqqq] (9.324999873594008,43.477811321264134) -- (9.349999873255118,43.71124881493537);
\draw[line width=1.2pt,color=ccqqqq] (9.349999873255118,43.71124881493537) -- (9.374999872916229,43.94531130858965);
\draw[line width=1.2pt,color=ccqqqq] (9.374999872916229,43.94531130858965) -- (9.39999987257734,44.179998802227);
\draw[line width=1.2pt,color=ccqqqq] (9.39999987257734,44.179998802227) -- (9.42499987223845,44.4153112958474);
\draw[line width=1.2pt,color=ccqqqq] (9.42499987223845,44.4153112958474) -- (9.44999987189956,44.651248789450854);
\draw[line width=1.2pt,color=ccqqqq] (9.44999987189956,44.651248789450854) -- (9.474999871560671,44.887811283037365);
\draw[line width=1.2pt,color=ccqqqq] (9.474999871560671,44.887811283037365) -- (9.499999871221782,45.12499877660694);
\draw[line width=1.2pt,color=ccqqqq] (9.499999871221782,45.12499877660694) -- (9.524999870882892,45.362811270159554);
\draw[line width=1.2pt,color=ccqqqq] (9.524999870882892,45.362811270159554) -- (9.549999870544003,45.60124876369523);
\draw[line width=1.2pt,color=ccqqqq] (9.549999870544003,45.60124876369523) -- (9.574999870205113,45.84031125721397);
\draw[line width=1.2pt,color=ccqqqq] (9.574999870205113,45.84031125721397) -- (9.599999869866224,46.07999875071576);
\draw[line width=1.2pt,color=ccqqqq] (9.599999869866224,46.07999875071576) -- (9.624999869527334,46.3203112442006);
\draw[line width=1.2pt,color=ccqqqq] (9.624999869527334,46.3203112442006) -- (9.649999869188445,46.5612487376685);
\draw[line width=1.2pt,color=ccqqqq] (9.649999869188445,46.5612487376685) -- (9.674999868849556,46.802811231119456);
\draw[line width=1.2pt,color=ccqqqq] (9.674999868849556,46.802811231119456) -- (9.699999868510666,47.04499872455347);
\draw[line width=1.2pt,color=ccqqqq] (9.699999868510666,47.04499872455347) -- (9.724999868171777,47.287811217970535);
\draw[line width=1.2pt,color=ccqqqq] (9.724999868171777,47.287811217970535) -- (9.749999867832887,47.53124871137066);
\draw[line width=1.2pt,color=ccqqqq] (9.749999867832887,47.53124871137066) -- (9.774999867493998,47.775311204753834);
\draw[line width=1.2pt,color=ccqqqq] (9.774999867493998,47.775311204753834) -- (9.799999867155108,48.01999869812007);
\draw[line width=1.2pt,color=ccqqqq] (9.799999867155108,48.01999869812007) -- (9.824999866816219,48.26531119146936);
\draw[line width=1.2pt,color=ccqqqq] (9.824999866816219,48.26531119146936) -- (9.84999986647733,48.5112486848017);
\draw[line width=1.2pt,color=ccqqqq] (9.84999986647733,48.5112486848017) -- (9.87499986613844,48.7578111781171);
\draw[line width=1.2pt,color=ccqqqq] (9.87499986613844,48.7578111781171) -- (9.89999986579955,49.00499867141556);
\draw[line width=1.2pt,color=ccqqqq] (9.89999986579955,49.00499867141556) -- (9.924999865460661,49.25281116469707);
\draw[line width=1.2pt,color=ccqqqq] (9.924999865460661,49.25281116469707) -- (9.949999865121772,49.50124865796164);
\draw[line width=1.2pt,color=ccqqqq] (9.949999865121772,49.50124865796164) -- (9.974999864782882,49.75031115120926);
\draw[line width=1.2pt,color=ccqqqq] (9.974999864782882,49.75031115120926) -- (9.999999864443993,49.999998644439934);
\draw (0.12,52.13004484304933) node[anchor=north west] {Tailles des pousses (mm)};
\draw (7.56,2.9147982062780033) node[anchor=north west] {Temps (jour)};
\end{tikzpicture}
\end{center}
\end{minipage}

\end{ExoCad}

\begin{ExoCad}{Calculer.}{1234}{0}{0}{0}{0}{0}
 
Ce graphique représente le niveau sonore en fonction de la distance à laquelle se trouve une personne de la source émettrice du son. Le niveau sonore est exprimé en (dB) et la distance en mètre.

\begin{minipage}{7cm}
\begin{enumerate}
\item Complète le graphique en notant sur chaque axe la légende.
\item Comment se nomme la variable ? Quelle est son unité ?
\item Quel est le niveau sonore pour une personne situés à 300 mètres de la source sonore ?
\item Quel est le niveau sonore pour une personne situés à 600 mètres de la source sonore ?
\item Une personne perçoit un son à un niveau de  40 dB. A Quelle distance de la source se trouve-t-il ?

\end{enumerate}
\end{minipage}
\begin{minipage}{9cm}
\begin{center}
\definecolor{qqqqff}{rgb}{0.,0.,1.}
\definecolor{cqcqcq}{rgb}{0.7529411764705882,0.7529411764705882,0.7529411764705882}
\begin{tikzpicture}[line cap=round,line join=round,>=triangle 45,x=0.007174400979366021cm,y=0.07955814622432517cm]
\draw [color=cqcqcq,, xstep=0.717440097936602cm,ystep=1.5911629244865033cm] (-57.6948075297355,-11.76791192653425) grid (1336.1497836438775,113.92631895465824);
\draw[->,color=black] (-57.6948075297355,0.) -- (1336.1497836438775,0.);
\foreach \x in {,100.,200.,300.,400.,500.,600.,700.,800.,900.,1000.,1100.,1200.,1300.}
\draw[shift={(\x,0)},color=black] (0pt,2pt) -- (0pt,-2pt) node[below] {\footnotesize $\x$};
\draw[->,color=black] (0.,-11.76791192653425) -- (0.,113.92631895465824);
\foreach \y in {,20.,40.,60.,80.,100.}
\draw[shift={(0,\y)},color=black] (2pt,0pt) -- (-2pt,0pt) node[left] {\footnotesize $\y$};
\draw[color=black] (0pt,-10pt) node[right] {\footnotesize $0$};
\clip(-57.6948075297355,-11.76791192653425) rectangle (1336.1497836438775,113.92631895465824);
\draw [color=qqqqff] (59.68157909541087,107.80845815955595)-- (155.86500702435026,65.8176863386266);
\draw [color=qqqqff] (155.86500702435026,65.8176863386266)-- (351.49231806626085,31.335198220777333);
\draw [color=qqqqff] (351.49231806626085,31.335198220777333)-- (600.,20.);
\draw [color=qqqqff] (600.,20.)-- (1324.7381904997662,13.537784998661582);
\end{tikzpicture}
\end{center}
\end{minipage}

\end{ExoCad}

\begin{ExoCad}{Calculer.}{1234}{0}{0}{0}{0}{0}
 
\begin{minipage}{0.49\linewidth}

On propose la figure suivante.

\definecolor{qqqqff}{rgb}{0.,0.,1.}
\definecolor{ffffff}{rgb}{1.,1.,1.}
\definecolor{zzttqq}{rgb}{0.6,0.2,0.}
\begin{tikzpicture}[line cap=round,line join=round,>=triangle 45,x=1.0cm,y=1.0cm]
\clip(3.4041911842591923,2.866575798081113) rectangle (12.252666255261191,9.209615907843705);
\fill[color=zzttqq,fill=zzttqq,fill opacity=0.25] (5.,8.) -- (5.,4.) -- (10.,4.) -- (10.,8.) -- cycle;
\draw [color=zzttqq] (5.,8.)-- (5.,4.);
\draw [color=zzttqq] (5.,4.)-- (10.,4.);
\draw [shift={(10.,5.5)},color=zzttqq,fill=zzttqq,fill opacity=0.25]  plot[domain=-1.5707963267948966:1.5707963267948966,variable=\t]({1.*1.5*cos(\t r)+0.*1.5*sin(\t r)},{0.*1.5*cos(\t r)+1.*1.5*sin(\t r)});
\draw [shift={(7.498580889309367,8.)},color=zzttqq]  plot[domain=3.141592653589793:6.283185307179586,variable=\t]({1.*1.498580889309367*cos(\t r)+0.*1.498580889309367*sin(\t r)},{0.*1.498580889309367*cos(\t r)+1.*1.498580889309367*sin(\t r)});
\draw [shift={(7.5,8.)},color=ffffff,fill=ffffff,fill opacity=1.0]  plot[domain=3.141592653589793:6.283185307179586,variable=\t]({1.*1.5*cos(\t r)+0.*1.5*sin(\t r)},{0.*1.5*cos(\t r)+1.*1.5*sin(\t r)});
\draw (5.0005655527389345,8.376347904148352)-- (5.958390173826779,8.376347904148352);
\draw (4.3012015436906665,7.9852221064954305)-- (4.3164051091047595,4.0059422521135355);
\draw (11.,8.)-- (11.,7.);
\draw (8.983899691231244,8.34233696522201)-- (9.972131443147275,8.32533149575884);
\draw (5.0157691181530275,3.5127836376815917)-- (9.987335008561367,3.495778168218421);
\draw (5.350247557263068,9.02255574374883) node[anchor=north west] {$x$};
\draw (9.34878526116947,8.971539335359319) node[anchor=north west] {$x$};
\draw (11.158009545446511,7.866183820253237) node[anchor=north west] {$x$};
\draw (3.784280319611512,6.59077361051545) node[anchor=north west] {$5$};
\draw (7.2050825377823875,3.5808055155342737) node[anchor=north west] {$6$};
\begin{scriptsize}
\draw [fill=qqqqff,shift={(5.0005655527389345,8.376347904148352)},rotate=90] (0,0) ++(0 pt,2.25pt) -- ++(1.9485571585149868pt,-3.375pt)--++(-3.8971143170299736pt,0 pt) -- ++(1.9485571585149868pt,3.375pt);
\draw [fill=qqqqff,shift={(5.958390173826779,8.376347904148352)},rotate=270] (0,0) ++(0 pt,2.25pt) -- ++(1.9485571585149868pt,-3.375pt)--++(-3.8971143170299736pt,0 pt) -- ++(1.9485571585149868pt,3.375pt);
\draw [fill=qqqqff,shift={(4.3012015436906665,7.9852221064954305)}] (0,0) ++(0 pt,2.25pt) -- ++(1.9485571585149868pt,-3.375pt)--++(-3.8971143170299736pt,0 pt) -- ++(1.9485571585149868pt,3.375pt);
\draw [fill=qqqqff,shift={(4.3164051091047595,4.0059422521135355)},rotate=180] (0,0) ++(0 pt,2.25pt) -- ++(1.9485571585149868pt,-3.375pt)--++(-3.8971143170299736pt,0 pt) -- ++(1.9485571585149868pt,3.375pt);
\draw [fill=qqqqff,shift={(11.,8.)}] (0,0) ++(0 pt,2.25pt) -- ++(1.9485571585149868pt,-3.375pt)--++(-3.8971143170299736pt,0 pt) -- ++(1.9485571585149868pt,3.375pt);
\draw [fill=qqqqff,shift={(11.,7.)},rotate=180] (0,0) ++(0 pt,2.25pt) -- ++(1.9485571585149868pt,-3.375pt)--++(-3.8971143170299736pt,0 pt) -- ++(1.9485571585149868pt,3.375pt);
\draw [fill=qqqqff,shift={(8.983899691231244,8.34233696522201)},rotate=90] (0,0) ++(0 pt,2.25pt) -- ++(1.9485571585149868pt,-3.375pt)--++(-3.8971143170299736pt,0 pt) -- ++(1.9485571585149868pt,3.375pt);
\draw [fill=qqqqff,shift={(9.972131443147275,8.32533149575884)},rotate=270] (0,0) ++(0 pt,2.25pt) -- ++(1.9485571585149868pt,-3.375pt)--++(-3.8971143170299736pt,0 pt) -- ++(1.9485571585149868pt,3.375pt);
\draw [fill=qqqqff,shift={(5.0157691181530275,3.5127836376815917)},rotate=90] (0,0) ++(0 pt,2.25pt) -- ++(1.9485571585149868pt,-3.375pt)--++(-3.8971143170299736pt,0 pt) -- ++(1.9485571585149868pt,3.375pt);
\draw [fill=qqqqff,shift={(9.987335008561367,3.495778168218421)},rotate=270] (0,0) ++(0 pt,2.25pt) -- ++(1.9485571585149868pt,-3.375pt)--++(-3.8971143170299736pt,0 pt) -- ++(1.9485571585149868pt,3.375pt);
\end{scriptsize}
\end{tikzpicture}
\end{minipage}
\begin{minipage}{0.49\linewidth}
\begin{enumerate}
\item Quelle est la valeur la plus petite pour $x$ ? Et la plus grande valeur pour $x$ ? \point{3}
\item Détermine le périmètre $\mathscr{P}$ de cette surface en fonction de $x$.\point{5}
\item Calcule $\mathscr{P}(2)$.\point{2}
\end{enumerate}
\end{minipage}
\end{ExoCad}

 
\begin{ExoCad}{Calculer.}{1234}{0}{0}{0}{0}{0}
 
Aux États-Unis, la température se mesure en de-
gré Fahrenheit (en $\deg$F). En France, elle se mesure
en degré Celsius (en $\deg$C). Pour faire les conversions
d’une unité à l’autre, on a utilisé un tableur.
Voici une copie de l’écran obtenu ci-contre.

 

\begin{enumerate}
\item Quelle température en \°F correspond à une température de 20 \°C ?
\item Quelle température en \°C correspond à une température de 41 \°F ?
\item Pour convertir la température de $\deg$C en $\deg$F, il faut multiplier
la température en
$\deg$C par 1,8 puis ajouter 32.
On a écrit une formule en B3 puis on l’a recopiée vers le bas.
Quelle formule a-t-on pu saisir dans la cellule B3?
\end{enumerate}
\end{ExoCad}

\begin{ExoCad}{Calculer.}{1234}{0}{0}{0}{0}{0}
 
Monsieur Philibert voyage un mois à travers Costa Rica. A l'arrivée à l'aéroport, il reçoit un sms sur son smartphone. 

\begin{center}
\begin{description}
\item  0,85 \euro{} par sms
\item  1,95 \euro{} par appel
\end{description}
\end{center}

Pour bénéficier de ces tarifs, il paie un abonnement mensuel de 9,90 \euro{} à son opérateur français.

\begin{enumerate}[leftmargin=*]
\item Quel est le prix que M. Philibert va payer pour 10 appels passés durant son voyage ? \point{2}
\item 
\begin{enumerate}
\item Exprime $f(n)$ qui détermine le coût de $n$ appels dans le mois durant lequel il visite au Costa Rica. \point{1}
\item Calcule alors l'image de 10 par $f$. \point{1}
\end{enumerate}
\end{enumerate}
\end{ExoCad}

\begin{ExoCad}{Calculer.}{1234}{0}{0}{0}{0}{0}
 
A la séance de 18 heures, le prix des places de cinémas est de 5 \euro{} pour les moins de 12 ans (enfant) et de 8 \euro{} sinon (adultes).

\begin{enumerate}[leftmargin=*]
\item Pierre et Marie, deux élèves de CP, vont au cinéma à 18h00. Quel est le montant total des places de cinémas ? \point{2}
\item Sasha, Tristan et Colin, sont trois frères et  vont au cinéma à 18h00. Les deux cadets ont respectivement 8 et 11 ans et l'ainé a 16 ans. Quel est le montant total des places de cinéma ? \point{2}
\item Monsieur et Madame Cinefil vont au cinéma à la séance de 18h00 avec leur 3 enfants, Marie 17 ans, Audrey 14 ans et Tom 10 ans. Quel est le montant total des places de cinéma ? \point{3}
\item Exprimer le prix total des places $\mathcal{P}$ en fonction du nombres $n$ d'enfants de moins de 12 ans et du nombre de personnes $m$ dont l'âge dépassent 12 ans. \point{2}
\item Un groupe d'amis va au cinéma. Le prix total des places est $68$ \euro{}. Quel est le nombre $e$ d'enfants de moins de 12 ans ?  et le nombre de personnes $a$ dont l'âge dépassent 12 ans. \point{7}
\end{enumerate}
\end{ExoCad}




\begin{ExoCad}{Calculer.}{1234}{0}{0}{0}{0}{0}

 
\begin{enumerate}
\item Exprimer le périmètre $\mathscr P$ d'un triangle équilatéral en fonction de la longueur d'un coté $c$.
\item La largeur d'un rectangle est 3 cm. Exprimer le périmètre $\mathscr P$ de ce rectangle en fonction de sa longueur $L$.
\item 12 œufs coutent 1,50 \euro{}. Exprime le prix $p$ de $n$ œufs achetés en fonction de $n$. 
\end{enumerate}

 
\end{ExoCad}

\begin{ExoCad}{Calculer.}{1234}{0}{0}{0}{0}{0}



\textbf{La fourmi est-elle la plus forte du monde ?}

[...] Des études précédentes affirment que l’insecte est capable de porter jusqu’à 1.000 fois son propre poids, soit l’équivalent d’un oisillon tombé du nid. Toutefois, ce chiffre, déjà sensationnel, pourrait être encore plus élevé, selon les récentes conclusions de travaux menés par des ingénieurs en mécanique et aérospatiale de l'Ohio State University. Ces derniers affirment que la fourmi serait capable de porter jusqu'à 5.000 fois son poids !
[...]

 http://www.maxisciences.com/fourmi/les-fourmis-des-insectes-encore-plus-forts-qu-039-on-ne-pense_art32040.html

\begin{enumerate}
\item Quelle serait la charge $C$ que porterait un humain de 15 kg ?

 
\item Quelle serait la charge $C$ que porterait un humain de 45 kg ?

 
\item Quelle serait la charge $C$ que porterait un humain de $p$ kg ?
 
\end{enumerate}

\end{ExoCad}


\Sf{Utiliser la distributivité}

\begin{ExoCad}{Représenter. Calculer.}{1234}{0}{0}{0}{0}{0}

 
 
Le débit moyen de téléchargement est de 0,85 Mo/s.
\begin{enumerate}
\item Complète ce tableau

 


\begin{tabular}{|c|>{\centering\arraybackslash}p{2cm}|>{\centering\arraybackslash}p{2cm}|>{\centering\arraybackslash}p{2cm}|>{\centering\arraybackslash}p{2cm}|}
\hline 
$d$ en Mo & 1 & 2 & 10 & 12 \\
\hline 
temps $t$ en $s$ &  &  &  &    \\ 
\hline 
\end{tabular} 





 
\item Déterminer une formule donnant le débit moyen $d$ en fonction du temps $t$.

\item Représentation graphique dans un repère.
\begin{enumerate}
 
\item Recopier et complète les axes du repère par les données qu'ils représentent.
\item Tracer dans le repère ci-dessous les points dont l'abscisse est le débit moyen de téléchargement et l'ordonnée le temps correspondant.

\definecolor{cqcqcq}{rgb}{0.7529411764705882,0.7529411764705882,0.7529411764705882}
\begin{tikzpicture}[line cap=round,line join=round,>=triangle 45,x=0.883779264214047cm,y=0.497340425531907cm]
\draw [color=cqcqcq,, xstep=0.883779264214047cm,ystep=0.994680851063814cm] (-0.666982024597919,-1.5294117647059702) grid (16.30558183538316,20.58823529411791);
\draw[->,color=black] (-0.666982024597919,0.) -- (16.30558183538316,0.);
\foreach \x in {,1.,2.,3.,4.,5.,6.,7.,8.,9.,10.,11.,12.,13.,14.,15.,16.}
\draw[shift={(\x,0)},color=black] (0pt,2pt) -- (0pt,-2pt) node[below] {\footnotesize $\x$};
\draw[->,color=black] (0.,-1.5294117647059702) -- (0.,20.58823529411791);
\foreach \y in {,2.,4.,6.,8.,10.,12.,14.,16.,18.,20.}
\draw[shift={(0,\y)},color=black] (2pt,0pt) -- (-2pt,0pt) node[left] {\footnotesize $\y$};
\draw[color=black] (0pt,-10pt) node[right] {\footnotesize $0$};
\clip(-0.666982024597919,-1.5294117647059702) rectangle (16.30558183538316,20.58823529411791);
\end{tikzpicture}

\item Quel est la forme de cette représentation graphique ?


\item Quel est le temps nécessaire pour télécharger 9 Mo ?


\end{enumerate}


\end{enumerate}



 
\end{ExoCad}

\begin{ExoCad}{Calculer.}{1234}{0}{0}{0}{0}{0}

  
 
Pour réaliser un bon sirop d'orgeat, il faut 1 volume de sirop pour 7 volumes d'eau.
\begin{enumerate}
\item Quelle est le volume d'eau pour 2 cl de sirop pour avoir un bon sirop d'orgeat ?
\item Quelle est le volume de sirop pour 28 cl d'eau ?
\item Exprime le volume d'eau $V$ en fonction du volume $v$ de sirop.
\end{enumerate}

\end{ExoCad}

 
\end{pageAD}


%%%%%%%%%%%%%%%%%%%%%%%%%%%%%%%%%%%%%%%%%%%%%%%%%%%%%%%%%%%%%%%%%%%
%%%%  Niveau 1
%%%%%%%%%%%%%%%%%%%%%%%%%%%%%%%%%%%%%%%%%%%%%%%%%%%%%%%%%%%%%%%%%%%
\begin{pageParcoursu} 

 %%%%%%%%%%%%%%%%%%%%%%%%%%%
\begin{ExoCu}{Représenter.}{1234}{2}{0}{0}{0}{0}


 
Un rectangle a pour dimension $x$ cm de longueur sur $x-5$ cm de largeur. 
\begin{enumerate}
\item Quelle est la valeur la plus petite pour $x$ ?
\item Déterminer l'aire de ce rectangle en fonction de $x$. 
\end{enumerate}
 

\end{ExoCu}
%%%%%%%%%%%%%%%%%%%%%%%%%%%
\begin{ExoCu}{Représenter.}{1234}{2}{0}{0}{0}{0}

 
 
 
Le débit d'une connexion internet varie en fonction de la distance du modem par rapport au central téléphonique le plus proche.
 
On a représenté ci-dessous la fonction qui, à la distance du modem au central téléphonique (en kilomètres), associe son débit théorique (en mégabits par seconde).

\begin{center}
\definecolor{ffqqqq}{rgb}{1.,0.,0.}
\definecolor{ccqqqq}{rgb}{0.8,0.,0.}
\definecolor{cqcqcq}{rgb}{0.7529411764705882,0.7529411764705882,0.7529411764705882}
\begin{tikzpicture}[line cap=round,line join=round,>=triangle 45,x=0.9640102827763489cm,y=0.28851540616246374cm]
\draw [color=cqcqcq,, xstep=0.9640102827763489cm,ystep=1.4425770308123187cm] (-0.86,-6.699029126213654) grid (14.7,27.961165048543776);
\draw[->,color=black] (-0.86,0.) -- (14.7,0.);
\foreach \x in {,1.,2.,3.,4.,5.,6.,7.,8.,9.,10.,11.,12.,13.,14.}
\draw[shift={(\x,0)},color=black] (0pt,2pt) -- (0pt,-2pt) node[below] {\footnotesize $\x$};
\draw[->,color=black] (0.,-6.699029126213654) -- (0.,27.961165048543776);
\foreach \y in {-5.,5.,10.,15.,20.,25.}
\draw[shift={(0,\y)},color=black] (2pt,0pt) -- (-2pt,0pt) node[left] {\footnotesize $\y$};
\draw[color=black] (0pt,-10pt) node[right] {\footnotesize $0$};
\clip(-0.86,-6.699029126213654) rectangle (14.7,27.961165048543776);
\draw (11.22,-2.427184466019461) node[anchor=north west] {distance (en km)};
\draw (0.1,26.89320388349523) node[anchor=north west] {débit (en Mbit/s)};
\draw[line width=1.2pt,color=ccqqqq] (3.980000084400006,9.611064014960892) -- (3.980000084400006,9.611064014960892);
\draw[line width=1.2pt,color=ccqqqq] (3.980000084400006,9.611064014960892) -- (4.005000084179178,9.295416750131462);
\draw[line width=1.2pt,color=ccqqqq] (4.005000084179178,9.295416750131462) -- (4.03000008395835,9.164431077570768);
\draw[line width=1.2pt,color=ccqqqq] (4.03000008395835,9.164431077570768) -- (4.0550000837375215,9.063922185272446);
\draw[line width=1.2pt,color=ccqqqq] (4.0550000837375215,9.063922185272446) -- (4.080000083516693,8.979189252407128);
\draw[line width=1.2pt,color=ccqqqq] (4.080000083516693,8.979189252407128) -- (4.105000083295865,8.904538031760932);
\draw[line width=1.2pt,color=ccqqqq] (4.105000083295865,8.904538031760932) -- (4.1300000830750365,8.837048164803448);
\draw[line width=1.2pt,color=ccqqqq] (4.1300000830750365,8.837048164803448) -- (4.155000082854208,8.77498482395023);
\draw[line width=1.2pt,color=ccqqqq] (4.155000082854208,8.77498482395023) -- (4.18000008263338,8.717217672769973);
\draw[line width=1.2pt,color=ccqqqq] (4.18000008263338,8.717217672769973) -- (4.2050000824125515,8.662961576752313);
\draw[line width=1.2pt,color=ccqqqq] (4.2050000824125515,8.662961576752313) -- (4.230000082191723,8.611644884160292);
\draw[line width=1.2pt,color=ccqqqq] (4.230000082191723,8.611644884160292) -- (4.255000081970895,8.562836044061221);
\draw[line width=1.2pt,color=ccqqqq] (4.255000081970895,8.562836044061221) -- (4.2800000817500665,8.516199784278882);
\draw[line width=1.2pt,color=ccqqqq] (4.2800000817500665,8.516199784278882) -- (4.305000081529238,8.471469480432848);
\draw[line width=1.2pt,color=ccqqqq] (4.305000081529238,8.471469480432848) -- (4.33000008130841,8.428428954487512);
\draw[line width=1.2pt,color=ccqqqq] (4.33000008130841,8.428428954487512) -- (4.3550000810875815,8.386900044736674);
\draw[line width=1.2pt,color=ccqqqq] (4.3550000810875815,8.386900044736674) -- (4.380000080866753,8.346733856614815);
\draw[line width=1.2pt,color=ccqqqq] (4.380000080866753,8.346733856614815) -- (4.405000080645925,8.307804443797998);
\draw[line width=1.2pt,color=ccqqqq] (4.405000080645925,8.307804443797998) -- (4.4300000804250965,8.270004142153198);
\draw[line width=1.2pt,color=ccqqqq] (4.4300000804250965,8.270004142153198) -- (4.455000080204268,8.23324005696212);
\draw[line width=1.2pt,color=ccqqqq] (4.455000080204268,8.23324005696212) -- (4.48000007998344,8.197431373056968);
\draw[line width=1.2pt,color=ccqqqq] (4.48000007998344,8.197431373056968) -- (4.5050000797626115,8.16250726384192);
\draw[line width=1.2pt,color=ccqqqq] (4.5050000797626115,8.16250726384192) -- (4.530000079541783,8.128405243870475);
\draw[line width=1.2pt,color=ccqqqq] (4.530000079541783,8.128405243870475) -- (4.555000079320955,8.095069855128038);
\draw[line width=1.2pt,color=ccqqqq] (4.555000079320955,8.095069855128038) -- (4.5800000791001265,8.062451607942936);
\draw[line width=1.2pt,color=ccqqqq] (4.5800000791001265,8.062451607942936) -- (4.605000078879298,8.03050611868424);
\draw[line width=1.2pt,color=ccqqqq] (4.605000078879298,8.03050611868424) -- (4.63000007865847,7.999193401320192);
\draw[line width=1.2pt,color=ccqqqq] (4.63000007865847,7.999193401320192) -- (4.6550000784376415,7.968477280556971);
\draw[line width=1.2pt,color=ccqqqq] (4.6550000784376415,7.968477280556971) -- (4.680000078216813,7.938324901988602);
\draw[line width=1.2pt,color=ccqqqq] (4.680000078216813,7.938324901988602) -- (4.705000077995985,7.908706320349446);
\draw[line width=1.2pt,color=ccqqqq] (4.705000077995985,7.908706320349446) -- (4.7300000777751565,7.879594151167834);
\draw[line width=1.2pt,color=ccqqqq] (4.7300000777751565,7.879594151167834) -- (4.755000077554328,7.8509632742820274);
\draw[line width=1.2pt,color=ccqqqq] (4.755000077554328,7.8509632742820274) -- (4.7800000773335,7.822790580082413);
\draw[line width=1.2pt,color=ccqqqq] (4.7800000773335,7.822790580082413) -- (4.8050000771126715,7.795054751186973);
\draw[line width=1.2pt,color=ccqqqq] (4.8050000771126715,7.795054751186973) -- (4.830000076891843,7.767736073684234);
\draw[line width=1.2pt,color=ccqqqq] (4.830000076891843,7.767736073684234) -- (4.855000076671015,7.740816273191991);
\draw[line width=1.2pt,color=ccqqqq] (4.855000076671015,7.740816273191991) -- (4.8800000764501865,7.714278371857126);
\draw[line width=1.2pt,color=ccqqqq] (4.8800000764501865,7.714278371857126) -- (4.905000076229358,7.688106563117079);
\draw[line width=1.2pt,color=ccqqqq] (4.905000076229358,7.688106563117079) -- (4.93000007600853,7.662286101598826);
\draw[line width=1.2pt,color=ccqqqq] (4.93000007600853,7.662286101598826) -- (4.9550000757877015,7.636803205977386);
\draw[line width=1.2pt,color=ccqqqq] (4.9550000757877015,7.636803205977386) -- (4.980000075566873,7.611644972976852);
\draw[line width=1.2pt,color=ccqqqq] (4.980000075566873,7.611644972976852) -- (5.005000075346045,7.586799300990548);
\draw[line width=1.2pt,color=ccqqqq] (5.005000075346045,7.586799300990548) -- (5.0300000751252165,7.562254822037102);
\draw[line width=1.2pt,color=ccqqqq] (5.0300000751252165,7.562254822037102) -- (5.055000074904388,7.538000840966746);
\draw[line width=1.2pt,color=ccqqqq] (5.055000074904388,7.538000840966746) -- (5.08000007468356,7.514027280995442);
\draw[line width=1.2pt,color=ccqqqq] (5.08000007468356,7.514027280995442) -- (5.1050000744627315,7.490324634779946);
\draw[line width=1.2pt,color=ccqqqq] (5.1050000744627315,7.490324634779946) -- (5.130000074241903,7.466883920360071);
\draw[line width=1.2pt,color=ccqqqq] (5.130000074241903,7.466883920360071) -- (5.155000074021075,7.443696641389093);
\draw[line width=1.2pt,color=ccqqqq] (5.155000074021075,7.443696641389093) -- (5.1800000738002465,7.42075475115296);
\draw[line width=1.2pt,color=ccqqqq] (5.1800000738002465,7.42075475115296) -- (5.205000073579418,7.3980506199462726);
\draw[line width=1.2pt,color=ccqqqq] (5.205000073579418,7.3980506199462726) -- (5.23000007335859,7.375577005430018);
\draw[line width=1.2pt,color=ccqqqq] (5.23000007335859,7.375577005430018) -- (5.2550000731377615,7.3533270256445915);
\draw[line width=1.2pt,color=ccqqqq] (5.2550000731377615,7.3533270256445915) -- (5.280000072916933,7.331294134393068);
\draw[line width=1.2pt,color=ccqqqq] (5.280000072916933,7.331294134393068) -- (5.305000072696105,7.309472098745116);
\draw[line width=1.2pt,color=ccqqqq] (5.305000072696105,7.309472098745116) -- (5.3300000724752765,7.287854978442488);
\draw[line width=1.2pt,color=ccqqqq] (5.3300000724752765,7.287854978442488) -- (5.355000072254448,7.26643710701321);
\draw[line width=1.2pt,color=ccqqqq] (5.355000072254448,7.26643710701321) -- (5.38000007203362,7.245213074424357);
\draw[line width=1.2pt,color=ccqqqq] (5.38000007203362,7.245213074424357) -- (5.4050000718127915,7.224177711122924);
\draw[line width=1.2pt,color=ccqqqq] (5.4050000718127915,7.224177711122924) -- (5.430000071591963,7.203326073331376);
\draw[line width=1.2pt,color=ccqqqq] (5.430000071591963,7.203326073331376) -- (5.455000071371135,7.182653429479438);
\draw[line width=1.2pt,color=ccqqqq] (5.455000071371135,7.182653429479438) -- (5.4800000711503065,7.162155247666564);
\draw[line width=1.2pt,color=ccqqqq] (5.4800000711503065,7.162155247666564) -- (5.505000070929478,7.141827184061017);
\draw[line width=1.2pt,color=ccqqqq] (5.505000070929478,7.141827184061017) -- (5.53000007070865,7.121665072151424);
\draw[line width=1.2pt,color=ccqqqq] (5.53000007070865,7.121665072151424) -- (5.5550000704878215,7.101664912775457);
\draw[line width=1.2pt,color=ccqqqq] (5.5550000704878215,7.101664912775457) -- (5.580000070266993,7.081822864858085);
\draw[line width=1.2pt,color=ccqqqq] (5.580000070266993,7.081822864858085) -- (5.605000070046165,7.062135236798599);
\draw[line width=1.2pt,color=ccqqqq] (5.605000070046165,7.062135236798599) -- (5.6300000698253365,7.042598478451748);
\draw[line width=1.2pt,color=ccqqqq] (5.6300000698253365,7.042598478451748) -- (5.655000069604508,7.023209173653632);
\draw[line width=1.2pt,color=ccqqqq] (5.655000069604508,7.023209173653632) -- (5.68000006938368,7.003964033247812);
\draw[line width=1.2pt,color=ccqqqq] (5.68000006938368,7.003964033247812) -- (5.7050000691628515,6.9848598885712905);
\draw[line width=1.2pt,color=ccqqqq] (5.7050000691628515,6.9848598885712905) -- (5.730000068942023,6.965893685363862);
\draw[line width=1.2pt,color=ccqqqq] (5.730000068942023,6.965893685363862) -- (5.755000068721195,6.947062478067669);
\draw[line width=1.2pt,color=ccqqqq] (5.755000068721195,6.947062478067669) -- (5.7800000685003665,6.928363424486817);
\draw[line width=1.2pt,color=ccqqqq] (5.7800000685003665,6.928363424486817) -- (5.805000068279538,6.909793780779684);
\draw[line width=1.2pt,color=ccqqqq] (5.805000068279538,6.909793780779684) -- (5.83000006805871,6.891350896758877);
\draw[line width=1.2pt,color=ccqqqq] (5.83000006805871,6.891350896758877) -- (5.8550000678378815,6.873032211476109);
\draw[line width=1.2pt,color=ccqqqq] (5.8550000678378815,6.873032211476109) -- (5.880000067617053,6.854835249071113);
\draw[line width=1.2pt,color=ccqqqq] (5.880000067617053,6.854835249071113) -- (5.905000067396225,6.836757614865575);
\draw[line width=1.2pt,color=ccqqqq] (5.905000067396225,6.836757614865575) -- (5.9300000671753965,6.818796991684635);
\draw[line width=1.2pt,color=ccqqqq] (5.9300000671753965,6.818796991684635) -- (5.955000066954568,6.800951136389952);
\draw[line width=1.2pt,color=ccqqqq] (5.955000066954568,6.800951136389952) -- (5.98000006673374,6.783217876609654);
\draw[line width=1.2pt,color=ccqqqq] (5.98000006673374,6.783217876609654) -- (6.0050000665129115,6.765595107651673);
\draw[line width=1.2pt,color=ccqqqq] (6.0050000665129115,6.765595107651673) -- (6.030000066292083,6.748080789588054);
\draw[line width=1.2pt,color=ccqqqq] (6.030000066292083,6.748080789588054) -- (6.055000066071255,6.730672944498802);
\draw[line width=1.2pt,color=ccqqqq] (6.055000066071255,6.730672944498802) -- (6.0800000658504265,6.713369653864725);
\draw[line width=1.2pt,color=ccqqqq] (6.0800000658504265,6.713369653864725) -- (6.105000065629598,6.696169056099541);
\draw[line width=1.2pt,color=ccqqqq] (6.105000065629598,6.696169056099541) -- (6.13000006540877,6.6790693442122775);
\draw[line width=1.2pt,color=ccqqqq] (6.13000006540877,6.6790693442122775) -- (6.1550000651879415,6.662068763591635);
\draw[line width=1.2pt,color=ccqqqq] (6.1550000651879415,6.662068763591635) -- (6.180000064967113,6.64516560990464);
\draw[line width=1.2pt,color=ccqqqq] (6.180000064967113,6.64516560990464) -- (6.205000064746285,6.628358227102457);
\draw[line width=1.2pt,color=ccqqqq] (6.205000064746285,6.628358227102457) -- (6.2300000645254565,6.611645005526753);
\draw[line width=1.2pt,color=ccqqqq] (6.2300000645254565,6.611645005526753) -- (6.255000064304628,6.595024380110499);
\draw[line width=1.2pt,color=ccqqqq] (6.255000064304628,6.595024380110499) -- (6.2800000640838,6.578494828667497);
\draw[line width=1.2pt,color=ccqqqq] (6.2800000640838,6.578494828667497) -- (6.3050000638629715,6.562054870265356);
\draw[line width=1.2pt,color=ccqqqq] (6.3050000638629715,6.562054870265356) -- (6.330000063642143,6.5457030636769895);
\draw[line width=1.2pt,color=ccqqqq] (6.330000063642143,6.5457030636769895) -- (6.355000063421315,6.529438005906052);
\draw[line width=1.2pt,color=ccqqqq] (6.355000063421315,6.529438005906052) -- (6.3800000632004865,6.513258330782052);
\draw[line width=1.2pt,color=ccqqqq] (6.3800000632004865,6.513258330782052) -- (6.405000062979658,6.497162707621137);
\draw[line width=1.2pt,color=ccqqqq] (6.405000062979658,6.497162707621137) -- (6.43000006275883,6.4811498399488725);
\draw[line width=1.2pt,color=ccqqqq] (6.43000006275883,6.4811498399488725) -- (6.4550000625380015,6.465218464281507);
\draw[line width=1.2pt,color=ccqqqq] (6.4550000625380015,6.465218464281507) -- (6.480000062317173,6.449367348962503);
\draw[line width=1.2pt,color=ccqqqq] (6.480000062317173,6.449367348962503) -- (6.505000062096345,6.4335952930513125);
\draw[line width=1.2pt,color=ccqqqq] (6.505000062096345,6.4335952930513125) -- (6.5300000618755165,6.417901125261509);
\draw[line width=1.2pt,color=ccqqqq] (6.5300000618755165,6.417901125261509) -- (6.555000061654688,6.4022837029457);
\draw[line width=1.2pt,color=ccqqqq] (6.555000061654688,6.4022837029457) -- (6.58000006143386,6.386741911124642);
\draw[line width=1.2pt,color=ccqqqq] (6.58000006143386,6.386741911124642) -- (6.6050000612130315,6.37127466155829);
\draw[line width=1.2pt,color=ccqqqq] (6.6050000612130315,6.37127466155829) -- (6.630000060992203,6.355880891856569);
\draw[line width=1.2pt,color=ccqqqq] (6.630000060992203,6.355880891856569) -- (6.655000060771375,6.340559564627797);
\draw[line width=1.2pt,color=ccqqqq] (6.655000060771375,6.340559564627797) -- (6.6800000605505465,6.325309666662839);
\draw[line width=1.2pt,color=ccqqqq] (6.6800000605505465,6.325309666662839) -- (6.705000060329718,6.310130208153198);
\draw[line width=1.2pt,color=ccqqqq] (6.705000060329718,6.310130208153198) -- (6.73000006010889,6.295020221941299);
\draw[line width=1.2pt,color=ccqqqq] (6.73000006010889,6.295020221941299) -- (6.7550000598880615,6.279978762801371);
\draw[line width=1.2pt,color=ccqqqq] (6.7550000598880615,6.279978762801371) -- (6.780000059667233,6.26500490674943);
\draw[line width=1.2pt,color=ccqqqq] (6.780000059667233,6.26500490674943) -- (6.805000059446405,6.250097750380925);
\draw[line width=1.2pt,color=ccqqqq] (6.805000059446405,6.250097750380925) -- (6.8300000592255765,6.235256410234693);
\draw[line width=1.2pt,color=ccqqqq] (6.8300000592255765,6.235256410234693) -- (6.855000059004748,6.220480022181987);
\draw[line width=1.2pt,color=ccqqqq] (6.855000059004748,6.220480022181987) -- (6.88000005878392,6.205767740839337);
\draw[line width=1.2pt,color=ccqqqq] (6.88000005878392,6.205767740839337) -- (6.9050000585630915,6.191118739004158);
\draw[line width=1.2pt,color=ccqqqq] (6.9050000585630915,6.191118739004158) -- (6.930000058342263,6.176532207112012);
\draw[line width=1.2pt,color=ccqqqq] (6.930000058342263,6.176532207112012) -- (6.955000058121435,6.162007352714531);
\draw[line width=1.2pt,color=ccqqqq] (6.955000058121435,6.162007352714531) -- (6.9800000579006065,6.147543399977038);
\draw[line width=1.2pt,color=ccqqqq] (6.9800000579006065,6.147543399977038) -- (7.005000057679778,6.133139589194966);
\draw[line width=1.2pt,color=ccqqqq] (7.005000057679778,6.133139589194966) -- (7.03000005745895,6.118795176328228);
\draw[line width=1.2pt,color=ccqqqq] (7.03000005745895,6.118795176328228) -- (7.0550000572381215,6.104509432552736);
\draw[line width=1.2pt,color=ccqqqq] (7.0550000572381215,6.104509432552736) -- (7.080000057017293,6.090281643828276);
\draw[line width=1.2pt,color=ccqqqq] (7.080000057017293,6.090281643828276) -- (7.105000056796465,6.076111110482054);
\draw[line width=1.2pt,color=ccqqqq] (7.105000056796465,6.076111110482054) -- (7.1300000565756365,6.061997146807195);
\draw[line width=1.2pt,color=ccqqqq] (7.1300000565756365,6.061997146807195) -- (7.155000056354808,6.04793908067555);
\draw[line width=1.2pt,color=ccqqqq] (7.155000056354808,6.04793908067555) -- (7.18000005613398,6.033936253164211);
\draw[line width=1.2pt,color=ccqqqq] (7.18000005613398,6.033936253164211) -- (7.2050000559131515,6.019988018195112);
\draw[line width=1.2pt,color=ccqqqq] (7.2050000559131515,6.019988018195112) -- (7.230000055692323,6.0060937421871925);
\draw[line width=1.2pt,color=ccqqqq] (7.230000055692323,6.0060937421871925) -- (7.255000055471495,5.992252803720571);
\draw[line width=1.2pt,color=ccqqqq] (7.255000055471495,5.992252803720571) -- (7.2800000552506665,5.978464593212242);
\draw[line width=1.2pt,color=ccqqqq] (7.2800000552506665,5.978464593212242) -- (7.305000055029838,5.964728512602799);
\draw[line width=1.2pt,color=ccqqqq] (7.305000055029838,5.964728512602799) -- (7.33000005480901,5.951043975053737);
\draw[line width=1.2pt,color=ccqqqq] (7.33000005480901,5.951043975053737) -- (7.3550000545881815,5.937410404654915);
\draw[line width=1.2pt,color=ccqqqq] (7.3550000545881815,5.937410404654915) -- (7.380000054367353,5.923827236141732);
\draw[line width=1.2pt,color=ccqqqq] (7.380000054367353,5.923827236141732) -- (7.405000054146525,5.910293914621666);
\draw[line width=1.2pt,color=ccqqqq] (7.405000054146525,5.910293914621666) -- (7.4300000539256965,5.8968098953097625);
\draw[line width=1.2pt,color=ccqqqq] (7.4300000539256965,5.8968098953097625) -- (7.455000053704868,5.883374643272739);
\draw[line width=1.2pt,color=ccqqqq] (7.455000053704868,5.883374643272739) -- (7.48000005348404,5.869987633181361);
\draw[line width=1.2pt,color=ccqqqq] (7.48000005348404,5.869987633181361) -- (7.5050000532632115,5.856648349070761);
\draw[line width=1.2pt,color=ccqqqq] (7.5050000532632115,5.856648349070761) -- (7.530000053042383,5.843356284108396);
\draw[line width=1.2pt,color=ccqqqq] (7.530000053042383,5.843356284108396) -- (7.555000052821555,5.830110940369342);
\draw[line width=1.2pt,color=ccqqqq] (7.555000052821555,5.830110940369342) -- (7.5800000526007265,5.816911828618651);
\draw[line width=1.2pt,color=ccqqqq] (7.5800000526007265,5.816911828618651) -- (7.605000052379898,5.8037584681005);
\draw[line width=1.2pt,color=ccqqqq] (7.605000052379898,5.8037584681005) -- (7.63000005215907,5.790650386333859);
\draw[line width=1.2pt,color=ccqqqq] (7.63000005215907,5.790650386333859) -- (7.6550000519382415,5.777587118914469);
\draw[line width=1.2pt,color=ccqqqq] (7.6550000519382415,5.777587118914469) -- (7.680000051717413,5.764568209322852);
\draw[line width=1.2pt,color=ccqqqq] (7.680000051717413,5.764568209322852) -- (7.705000051496585,5.751593208738158);
\draw[line width=1.2pt,color=ccqqqq] (7.705000051496585,5.751593208738158) -- (7.7300000512757565,5.73866167585762);
\draw[line width=1.2pt,color=ccqqqq] (7.7300000512757565,5.73866167585762) -- (7.755000051054928,5.725773176721428);
\draw[line width=1.2pt,color=ccqqqq] (7.755000051054928,5.725773176721428) -- (7.7800000508341,5.712927284542794);
\draw[line width=1.2pt,color=ccqqqq] (7.7800000508341,5.712927284542794) -- (7.8050000506132715,5.700123579543063);
\draw[line width=1.2pt,color=ccqqqq] (7.8050000506132715,5.700123579543063) -- (7.830000050392443,5.687361648791641);
\draw[line width=1.2pt,color=ccqqqq] (7.830000050392443,5.687361648791641) -- (7.855000050171615,5.674641086050612);
\draw[line width=1.2pt,color=ccqqqq] (7.855000050171615,5.674641086050612) -- (7.8800000499507865,5.6619614916238605);
\draw[line width=1.2pt,color=ccqqqq] (7.8800000499507865,5.6619614916238605) -- (7.905000049729958,5.649322472210515);
\draw[line width=1.2pt,color=ccqqqq] (7.905000049729958,5.649322472210515) -- (7.93000004950913,5.636723640762619);
\draw[line width=1.2pt,color=ccqqqq] (7.93000004950913,5.636723640762619) -- (7.9550000492883015,5.62416461634682);
\draw[line width=1.2pt,color=ccqqqq] (7.9550000492883015,5.62416461634682) -- (7.980000049067473,5.611645024009987);
\draw[line width=1.2pt,color=ccqqqq] (7.980000049067473,5.611645024009987) -- (8.005000048846645,5.599164494648593);
\draw[line width=1.2pt,color=ccqqqq] (8.005000048846645,5.599164494648593) -- (8.030000048625817,5.58672266488174);
\draw[line width=1.2pt,color=ccqqqq] (8.030000048625817,5.58672266488174) -- (8.05500004840499,5.574319176927717);
\draw[line width=1.2pt,color=ccqqqq] (8.05500004840499,5.574319176927717) -- (8.080000048184163,5.561953678483946);
\draw[line width=1.2pt,color=ccqqqq] (8.080000048184163,5.561953678483946) -- (8.105000047963335,5.54962582261023);
\draw[line width=1.2pt,color=ccqqqq] (8.105000047963335,5.54962582261023) -- (8.130000047742508,5.537335267615178);
\draw[line width=1.2pt,color=ccqqqq] (8.130000047742508,5.537335267615178) -- (8.15500004752168,5.525081676945686);
\draw[line width=1.2pt,color=ccqqqq] (8.15500004752168,5.525081676945686) -- (8.180000047300853,5.512864719079432);
\draw[line width=1.2pt,color=ccqqqq] (8.180000047300853,5.512864719079432) -- (8.205000047080025,5.500684067420198);
\draw[line width=1.2pt,color=ccqqqq] (8.205000047080025,5.500684067420198) -- (8.230000046859198,5.488539400196015);
\draw[line width=1.2pt,color=ccqqqq] (8.230000046859198,5.488539400196015) -- (8.25500004663837,5.476430400359971);
\draw[line width=1.2pt,color=ccqqqq] (8.25500004663837,5.476430400359971) -- (8.280000046417543,5.464356755493653);
\draw[line width=1.2pt,color=ccqqqq] (8.280000046417543,5.464356755493653) -- (8.305000046196716,5.452318157713086);
\draw[line width=1.2pt,color=ccqqqq] (8.305000046196716,5.452318157713086) -- (8.330000045975888,5.440314303577131);
\draw[line width=1.2pt,color=ccqqqq] (8.330000045975888,5.440314303577131) -- (8.35500004575506,5.4283448939982435);
\draw[line width=1.2pt,color=ccqqqq] (8.35500004575506,5.4283448939982435) -- (8.380000045534233,5.416409634155524);
\draw[line width=1.2pt,color=ccqqqq] (8.380000045534233,5.416409634155524) -- (8.405000045313406,5.40450823340999);
\draw[line width=1.2pt,color=ccqqqq] (8.405000045313406,5.40450823340999) -- (8.430000045092578,5.392640405221995);
\draw[line width=1.2pt,color=ccqqqq] (8.430000045092578,5.392640405221995) -- (8.45500004487175,5.380805867070738);
\draw[line width=1.2pt,color=ccqqqq] (8.45500004487175,5.380805867070738) -- (8.480000044650923,5.369004340375792);
\draw[line width=1.2pt,color=ccqqqq] (8.480000044650923,5.369004340375792) -- (8.505000044430096,5.357235550420585);
\draw[line width=1.2pt,color=ccqqqq] (8.505000044430096,5.357235550420585) -- (8.530000044209268,5.345499226277791);
\draw[line width=1.2pt,color=ccqqqq] (8.530000044209268,5.345499226277791) -- (8.555000043988441,5.333795100736559);
\draw[line width=1.2pt,color=ccqqqq] (8.555000043988441,5.333795100736559) -- (8.580000043767614,5.322122910231526);
\draw[line width=1.2pt,color=ccqqqq] (8.580000043767614,5.322122910231526) -- (8.605000043546786,5.310482394773566);
\draw[line width=1.2pt,color=ccqqqq] (8.605000043546786,5.310482394773566) -- (8.630000043325959,5.298873297882227);
\draw[line width=1.2pt,color=ccqqqq] (8.630000043325959,5.298873297882227) -- (8.655000043105131,5.287295366519787);
\draw[line width=1.2pt,color=ccqqqq] (8.655000043105131,5.287295366519787) -- (8.680000042884304,5.27574835102691);
\draw[line width=1.2pt,color=ccqqqq] (8.680000042884304,5.27574835102691) -- (8.705000042663476,5.264232005059829);
\draw[line width=1.2pt,color=ccqqqq] (8.705000042663476,5.264232005059829) -- (8.730000042442649,5.2527460855290276);
\draw[line width=1.2pt,color=ccqqqq] (8.730000042442649,5.2527460855290276) -- (8.755000042221821,5.241290352539373);
\draw[line width=1.2pt,color=ccqqqq] (8.755000042221821,5.241290352539373) -- (8.780000042000994,5.229864569331652);
\draw[line width=1.2pt,color=ccqqqq] (8.780000042000994,5.229864569331652) -- (8.805000041780167,5.218468502225483);
\draw[line width=1.2pt,color=ccqqqq] (8.805000041780167,5.218468502225483) -- (8.83000004155934,5.207101920563552);
\draw[line width=1.2pt,color=ccqqqq] (8.83000004155934,5.207101920563552) -- (8.855000041338512,5.195764596657142);
\draw[line width=1.2pt,color=ccqqqq] (8.855000041338512,5.195764596657142) -- (8.880000041117684,5.184456305732917);
\draw[line width=1.2pt,color=ccqqqq] (8.880000041117684,5.184456305732917) -- (8.905000040896857,5.17317682588093);
\draw[line width=1.2pt,color=ccqqqq] (8.905000040896857,5.17317682588093) -- (8.93000004067603,5.161925938003811);
\draw[line width=1.2pt,color=ccqqqq] (8.93000004067603,5.161925938003811) -- (8.955000040455202,5.150703425767115);
\draw[line width=1.2pt,color=ccqqqq] (8.955000040455202,5.150703425767115) -- (8.980000040234374,5.1395090755507855);
\draw[line width=1.2pt,color=ccqqqq] (8.980000040234374,5.1395090755507855) -- (9.005000040013547,5.128342676401712);
\draw[line width=1.2pt,color=ccqqqq] (9.005000040013547,5.128342676401712) -- (9.03000003979272,5.11720401998735);
\draw[line width=1.2pt,color=ccqqqq] (9.03000003979272,5.11720401998735) -- (9.055000039571892,5.106092900550365);
\draw[line width=1.2pt,color=ccqqqq] (9.055000039571892,5.106092900550365) -- (9.080000039351065,5.095009114864294);
\draw[line width=1.2pt,color=ccqqqq] (9.080000039351065,5.095009114864294) -- (9.105000039130237,5.083952462190169);
\draw[line width=1.2pt,color=ccqqqq] (9.105000039130237,5.083952462190169) -- (9.13000003890941,5.072922744234103);
\draw[line width=1.2pt,color=ccqqqq] (9.13000003890941,5.072922744234103) -- (9.155000038688582,5.061919765105797);
\draw[line width=1.2pt,color=ccqqqq] (9.155000038688582,5.061919765105797) -- (9.180000038467755,5.050943331277946);
\draw[line width=1.2pt,color=ccqqqq] (9.180000038467755,5.050943331277946) -- (9.205000038246927,5.039993251546526);
\draw[line width=1.2pt,color=ccqqqq] (9.205000038246927,5.039993251546526) -- (9.2300000380261,5.029069336991934);
\draw[line width=1.2pt,color=ccqqqq] (9.2300000380261,5.029069336991934) -- (9.255000037805273,5.018171400940955);
\draw[line width=1.2pt,color=ccqqqq] (9.255000037805273,5.018171400940955) -- (9.280000037584445,5.0072992589295495);
\draw[line width=1.2pt,color=ccqqqq] (9.280000037584445,5.0072992589295495) -- (9.305000037363618,4.9964527286664175);
\draw[line width=1.2pt,color=ccqqqq] (9.305000037363618,4.9964527286664175) -- (9.33000003714279,4.985631629997345);
\draw[line width=1.2pt,color=ccqqqq] (9.33000003714279,4.985631629997345) -- (9.355000036921963,4.974835784870281);
\draw[line width=1.2pt,color=ccqqqq] (9.355000036921963,4.974835784870281) -- (9.380000036701135,4.96406501730117);
\draw[line width=1.2pt,color=ccqqqq] (9.380000036701135,4.96406501730117) -- (9.405000036480308,4.953319153340466);
\draw[line width=1.2pt,color=ccqqqq] (9.405000036480308,4.953319153340466) -- (9.43000003625948,4.942598021040368);
\draw[line width=1.2pt,color=ccqqqq] (9.43000003625948,4.942598021040368) -- (9.455000036038653,4.9319014504227106);
\draw[line width=1.2pt,color=ccqqqq] (9.455000036038653,4.9319014504227106) -- (9.480000035817826,4.921229273447523);
\draw[line width=1.2pt,color=ccqqqq] (9.480000035817826,4.921229273447523) -- (9.505000035596998,4.910581323982229);
\draw[line width=1.2pt,color=ccqqqq] (9.505000035596998,4.910581323982229) -- (9.53000003537617,4.899957437771477);
\draw[line width=1.2pt,color=ccqqqq] (9.53000003537617,4.899957437771477) -- (9.555000035155343,4.8893574524075705);
\draw[line width=1.2pt,color=ccqqqq] (9.555000035155343,4.8893574524075705) -- (9.580000034934516,4.878781207301504);
\draw[line width=1.2pt,color=ccqqqq] (9.580000034934516,4.878781207301504) -- (9.605000034713688,4.868228543654579);
\draw[line width=1.2pt,color=ccqqqq] (9.605000034713688,4.868228543654579) -- (9.630000034492861,4.857699304430583);
\draw[line width=1.2pt,color=ccqqqq] (9.630000034492861,4.857699304430583) -- (9.655000034272033,4.8471933343285265);
\draw[line width=1.2pt,color=ccqqqq] (9.655000034272033,4.8471933343285265) -- (9.680000034051206,4.836710479755915);
\draw[line width=1.2pt,color=ccqqqq] (9.680000034051206,4.836710479755915) -- (9.705000033830379,4.826250588802551);
\draw[line width=1.2pt,color=ccqqqq] (9.705000033830379,4.826250588802551) -- (9.730000033609551,4.815813511214853);
\draw[line width=1.2pt,color=ccqqqq] (9.730000033609551,4.815813511214853) -- (9.755000033388724,4.8053990983706685);
\draw[line width=1.2pt,color=ccqqqq] (9.755000033388724,4.8053990983706685) -- (9.780000033167896,4.7950072032545865);
\draw[line width=1.2pt,color=ccqqqq] (9.780000033167896,4.7950072032545865) -- (9.805000032947069,4.784637680433719);
\draw[line width=1.2pt,color=ccqqqq] (9.805000032947069,4.784637680433719) -- (9.830000032726241,4.774290386033959);
\draw[line width=1.2pt,color=ccqqqq] (9.830000032726241,4.774290386033959) -- (9.855000032505414,4.763965177716686);
\draw[line width=1.2pt,color=ccqqqq] (9.855000032505414,4.763965177716686) -- (9.880000032284586,4.753661914655923);
\draw[line width=1.2pt,color=ccqqqq] (9.880000032284586,4.753661914655923) -- (9.905000032063759,4.7433804575159275);
\draw[line width=1.2pt,color=ccqqqq] (9.905000032063759,4.7433804575159275) -- (9.930000031842932,4.7331206684292075);
\draw[line width=1.2pt,color=ccqqqq] (9.930000031842932,4.7331206684292075) -- (9.955000031622104,4.722882410974948);
\draw[line width=1.2pt,color=ccqqqq] (9.955000031622104,4.722882410974948) -- (9.980000031401277,4.71266555015785);
\draw[line width=1.2pt,color=ccqqqq] (9.980000031401277,4.71266555015785) -- (10.00500003118045,4.702469952387365);
\draw[line width=1.2pt,color=ccqqqq] (10.00500003118045,4.702469952387365) -- (10.030000030959622,4.6922954854573105);
\draw[line width=1.2pt,color=ccqqqq] (10.030000030959622,4.6922954854573105) -- (10.055000030738794,4.682142018525873);
\draw[line width=1.2pt,color=ccqqqq] (10.055000030738794,4.682142018525873) -- (10.080000030517967,4.6720094220959725);
\draw[line width=1.2pt,color=ccqqqq] (10.080000030517967,4.6720094220959725) -- (10.10500003029714,4.661897567995998);
\draw[line width=1.2pt,color=ccqqqq] (10.10500003029714,4.661897567995998) -- (10.130000030076312,4.651806329360887);
\draw[line width=1.2pt,color=ccqqqq] (10.130000030076312,4.651806329360887) -- (10.155000029855485,4.641735580613555);
\draw[line width=1.2pt,color=ccqqqq] (10.155000029855485,4.641735580613555) -- (10.180000029634657,4.631685197446666);
\draw[line width=1.2pt,color=ccqqqq] (10.180000029634657,4.631685197446666) -- (10.20500002941383,4.621655056804731);
\draw[line width=1.2pt,color=ccqqqq] (10.20500002941383,4.621655056804731) -- (10.230000029193002,4.611645036866523);
\draw[line width=1.2pt,color=ccqqqq] (10.230000029193002,4.611645036866523) -- (10.255000028972175,4.601655017027823);
\draw[line width=1.2pt,color=ccqqqq] (10.255000028972175,4.601655017027823) -- (10.280000028751347,4.59168487788446);
\draw[line width=1.2pt,color=ccqqqq] (10.280000028751347,4.59168487788446) -- (10.30500002853052,4.581734501215663);
\draw[line width=1.2pt,color=ccqqqq] (10.30500002853052,4.581734501215663) -- (10.330000028309692,4.571803769967704);
\draw[line width=1.2pt,color=ccqqqq] (10.330000028309692,4.571803769967704) -- (10.355000028088865,4.561892568237837);
\draw[line width=1.2pt,color=ccqqqq] (10.355000028088865,4.561892568237837) -- (10.380000027868038,4.552000781258508);
\draw[line width=1.2pt,color=ccqqqq] (10.380000027868038,4.552000781258508) -- (10.40500002764721,4.542128295381857);
\draw[line width=1.2pt,color=ccqqqq] (10.40500002764721,4.542128295381857) -- (10.430000027426383,4.532274998064479);
\draw[line width=1.2pt,color=ccqqqq] (10.430000027426383,4.532274998064479) -- (10.455000027205555,4.5224407778524585);
\draw[line width=1.2pt,color=ccqqqq] (10.455000027205555,4.5224407778524585) -- (10.480000026984728,4.512625524366658);
\draw[line width=1.2pt,color=ccqqqq] (10.480000026984728,4.512625524366658) -- (10.5050000267639,4.502829128288268);
\draw[line width=1.2pt,color=ccqqqq] (10.5050000267639,4.502829128288268) -- (10.530000026543073,4.4930514813445965);
\draw[line width=1.2pt,color=ccqqqq] (10.530000026543073,4.4930514813445965) -- (10.555000026322245,4.48329247629511);
\draw[line width=1.2pt,color=ccqqqq] (10.555000026322245,4.48329247629511) -- (10.580000026101418,4.47355200691771);
\draw[line width=1.2pt,color=ccqqqq] (10.580000026101418,4.47355200691771) -- (10.60500002588059,4.463829967995244);
\draw[line width=1.2pt,color=ccqqqq] (10.60500002588059,4.463829967995244) -- (10.630000025659763,4.454126255302244);
\draw[line width=1.2pt,color=ccqqqq] (10.630000025659763,4.454126255302244) -- (10.655000025438936,4.444440765591893);
\draw[line width=1.2pt,color=ccqqqq] (10.655000025438936,4.444440765591893) -- (10.680000025218108,4.434773396583205);
\draw[line width=1.2pt,color=ccqqqq] (10.680000025218108,4.434773396583205) -- (10.70500002499728,4.425124046948423);
\draw[line width=1.2pt,color=ccqqqq] (10.70500002499728,4.425124046948423) -- (10.730000024776453,4.4154926163006305);
\draw[line width=1.2pt,color=ccqqqq] (10.730000024776453,4.4154926163006305) -- (10.755000024555626,4.405879005181566);
\draw[line width=1.2pt,color=ccqqqq] (10.755000024555626,4.405879005181566) -- (10.780000024334798,4.396283115049636);
\draw[line width=1.2pt,color=ccqqqq] (10.780000024334798,4.396283115049636) -- (10.805000024113971,4.3867048482681374);
\draw[line width=1.2pt,color=ccqqqq] (10.805000024113971,4.3867048482681374) -- (10.830000023893144,4.377144108093663);
\draw[line width=1.2pt,color=ccqqqq] (10.830000023893144,4.377144108093663) -- (10.855000023672316,4.367600798664699);
\draw[line width=1.2pt,color=ccqqqq] (10.855000023672316,4.367600798664699) -- (10.880000023451489,4.358074824990416);
\draw[line width=1.2pt,color=ccqqqq] (10.880000023451489,4.358074824990416) -- (10.905000023230661,4.348566092939634);
\draw[line width=1.2pt,color=ccqqqq] (10.905000023230661,4.348566092939634) -- (10.930000023009834,4.339074509229969);
\draw[line width=1.2pt,color=ccqqqq] (10.930000023009834,4.339074509229969) -- (10.955000022789006,4.329599981417164);
\draw[line width=1.2pt,color=ccqqqq] (10.955000022789006,4.329599981417164) -- (10.980000022568179,4.320142417884573);
\draw[line width=1.2pt,color=ccqqqq] (10.980000022568179,4.320142417884573) -- (11.005000022347351,4.310701727832833);
\draw[line width=1.2pt,color=ccqqqq] (11.005000022347351,4.310701727832833) -- (11.030000022126524,4.301277821269691);
\draw[line width=1.2pt,color=ccqqqq] (11.030000022126524,4.301277821269691) -- (11.055000021905697,4.291870608999996);
\draw[line width=1.2pt,color=ccqqqq] (11.055000021905697,4.291870608999996) -- (11.080000021684869,4.282480002615847);
\draw[line width=1.2pt,color=ccqqqq] (11.080000021684869,4.282480002615847) -- (11.105000021464042,4.2731059144869015);
\draw[line width=1.2pt,color=ccqqqq] (11.105000021464042,4.2731059144869015) -- (11.130000021243214,4.263748257750837);
\draw[line width=1.2pt,color=ccqqqq] (11.130000021243214,4.263748257750837) -- (11.155000021022387,4.254406946303956);
\draw[line width=1.2pt,color=ccqqqq] (11.155000021022387,4.254406946303956) -- (11.18000002080156,4.245081894791945);
\draw[line width=1.2pt,color=ccqqqq] (11.18000002080156,4.245081894791945) -- (11.205000020580732,4.235773018600774);
\draw[line width=1.2pt,color=ccqqqq] (11.205000020580732,4.235773018600774) -- (11.230000020359904,4.226480233847742);
\draw[line width=1.2pt,color=ccqqqq] (11.230000020359904,4.226480233847742) -- (11.255000020139077,4.2172034573726505);
\draw[line width=1.2pt,color=ccqqqq] (11.255000020139077,4.2172034573726505) -- (11.28000001991825,4.207942606729131);
\draw[line width=1.2pt,color=ccqqqq] (11.28000001991825,4.207942606729131) -- (11.305000019697422,4.198697600176089);
\draw[line width=1.2pt,color=ccqqqq] (11.305000019697422,4.198697600176089) -- (11.330000019476595,4.18946835666929);
\draw[line width=1.2pt,color=ccqqqq] (11.330000019476595,4.18946835666929) -- (11.355000019255767,4.180254795853068);
\draw[line width=1.2pt,color=ccqqqq] (11.355000019255767,4.180254795853068) -- (11.38000001903494,4.171056838052164);
\draw[line width=1.2pt,color=ccqqqq] (11.38000001903494,4.171056838052164) -- (11.405000018814112,4.1618744042636875);
\draw[line width=1.2pt,color=ccqqqq] (11.405000018814112,4.1618744042636875) -- (11.430000018593285,4.152707416149199);
\draw[line width=1.2pt,color=ccqqqq] (11.430000018593285,4.152707416149199) -- (11.455000018372457,4.143555796026916);
\draw[line width=1.2pt,color=ccqqqq] (11.455000018372457,4.143555796026916) -- (11.48000001815163,4.134419466864025);
\draw[line width=1.2pt,color=ccqqqq] (11.48000001815163,4.134419466864025) -- (11.505000017930803,4.125298352269125);
\draw[line width=1.2pt,color=ccqqqq] (11.505000017930803,4.125298352269125) -- (11.530000017709975,4.116192376484769);
\draw[line width=1.2pt,color=ccqqqq] (11.530000017709975,4.116192376484769) -- (11.555000017489148,4.10710146438013);
\draw[line width=1.2pt,color=ccqqqq] (11.555000017489148,4.10710146438013) -- (11.58000001726832,4.098025541443757);
\draw[line width=1.2pt,color=ccqqqq] (11.58000001726832,4.098025541443757) -- (11.605000017047493,4.088964533776464);
\draw[line width=1.2pt,color=ccqqqq] (11.605000017047493,4.088964533776464) -- (11.630000016826665,4.079918368084298);
\draw[line width=1.2pt,color=ccqqqq] (11.630000016826665,4.079918368084298) -- (11.655000016605838,4.070886971671629);
\draw[line width=1.2pt,color=ccqqqq] (11.655000016605838,4.070886971671629) -- (11.68000001638501,4.061870272434333);
\draw[line width=1.2pt,color=ccqqqq] (11.68000001638501,4.061870272434333) -- (11.705000016164183,4.052868198853072);
\draw[line width=1.2pt,color=ccqqqq] (11.705000016164183,4.052868198853072) -- (11.730000015943356,4.04388067998668);
\draw[line width=1.2pt,color=ccqqqq] (11.730000015943356,4.04388067998668) -- (11.755000015722528,4.034907645465638);
\draw[line width=1.2pt,color=ccqqqq] (11.755000015722528,4.034907645465638) -- (11.7800000155017,4.025949025485648);
\draw[line width=1.2pt,color=ccqqqq] (11.7800000155017,4.025949025485648) -- (11.805000015280873,4.017004750801295);
\draw[line width=1.2pt,color=ccqqqq] (11.805000015280873,4.017004750801295) -- (11.830000015060046,4.0080747527198035);
\draw[line width=1.2pt,color=ccqqqq] (11.830000015060046,4.0080747527198035) -- (11.855000014839218,3.9991589630948816);
\draw[line width=1.2pt,color=ccqqqq] (11.855000014839218,3.9991589630948816) -- (11.88000001461839,3.9902573143206554);
\draw[line width=1.2pt,color=ccqqqq] (11.88000001461839,3.9902573143206554) -- (11.905000014397563,3.9813697393256877);
\draw[line width=1.2pt,color=ccqqqq] (11.905000014397563,3.9813697393256877) -- (11.930000014176736,3.97249617156708);
\draw[line width=1.2pt,color=ccqqqq] (11.930000014176736,3.97249617156708) -- (11.955000013955908,3.963636545024662);
\draw[line width=1.2pt,color=ccqqqq] (11.955000013955908,3.963636545024662) -- (11.980000013735081,3.95479079419526);
\draw[line width=1.2pt,color=ccqqqq] (11.980000013735081,3.95479079419526) -- (12.005000013514254,3.9459588540870447);
\draw[line width=1.2pt,color=ccqqqq] (12.005000013514254,3.9459588540870447) -- (12.030000013293426,3.937140660213964);
\draw[line width=1.2pt,color=ccqqqq] (12.030000013293426,3.937140660213964) -- (12.055000013072599,3.9283361485902475);
\draw[line width=1.2pt,color=ccqqqq] (12.055000013072599,3.9283361485902475) -- (12.080000012851771,3.919545255724989);
\draw[line width=1.2pt,color=ccqqqq] (12.080000012851771,3.919545255724989) -- (12.105000012630944,3.9107679186168056);
\draw[line width=1.2pt,color=ccqqqq] (12.105000012630944,3.9107679186168056) -- (12.130000012410116,3.9020040747485707);
\draw[line width=1.2pt,color=ccqqqq] (12.130000012410116,3.9020040747485707) -- (12.155000012189289,3.893253662082217);
\draw[line width=1.2pt,color=ccqqqq] (12.155000012189289,3.893253662082217) -- (12.180000011968461,3.884516619053615);
\draw[line width=1.2pt,color=ccqqqq] (12.180000011968461,3.884516619053615) -- (12.205000011747634,3.8757928845675167);
\draw[line width=1.2pt,color=ccqqqq] (12.205000011747634,3.8757928845675167) -- (12.230000011526807,3.8670823979925766);
\draw[line width=1.2pt,color=ccqqqq] (12.230000011526807,3.8670823979925766) -- (12.25500001130598,3.8583850991564272);
\draw[line width=1.2pt,color=ccqqqq] (12.25500001130598,3.8583850991564272) -- (12.280000011085152,3.8497009283408383);
\draw[line width=1.2pt,color=ccqqqq] (12.280000011085152,3.8497009283408383) -- (12.305000010864324,3.841029826276925);
\draw[line width=1.2pt,color=ccqqqq] (12.305000010864324,3.841029826276925) -- (12.330000010643497,3.832371734140434);
\draw[line width=1.2pt,color=ccqqqq] (12.330000010643497,3.832371734140434) -- (12.35500001042267,3.823726593547085);
\draw[line width=1.2pt,color=ccqqqq] (12.35500001042267,3.823726593547085) -- (12.380000010201842,3.815094346547979);
\draw[line width=1.2pt,color=ccqqqq] (12.380000010201842,3.815094346547979) -- (12.405000009981014,3.8064749356250633);
\draw[line width=1.2pt,color=ccqqqq] (12.405000009981014,3.8064749356250633) -- (12.430000009760187,3.7978683036866645);
\draw[line width=1.2pt,color=ccqqqq] (12.430000009760187,3.7978683036866645) -- (12.45500000953936,3.7892743940630735);
\draw[line width=1.2pt,color=ccqqqq] (12.45500000953936,3.7892743940630735) -- (12.480000009318532,3.7806931505021932);
\draw[line width=1.2pt,color=ccqqqq] (12.480000009318532,3.7806931505021932) -- (12.505000009097705,3.772124517165243);
\draw[line width=1.2pt,color=ccqqqq] (12.505000009097705,3.772124517165243) -- (12.530000008876877,3.763568438622518);
\draw[line width=1.2pt,color=ccqqqq] (12.530000008876877,3.763568438622518) -- (12.55500000865605,3.7550248598492075);
\draw[line width=1.2pt,color=ccqqqq] (12.55500000865605,3.7550248598492075) -- (12.580000008435222,3.7464937262212654);
\draw[line width=1.2pt,color=ccqqqq] (12.580000008435222,3.7464937262212654) -- (12.605000008214395,3.737974983511336);
\draw[line width=1.2pt,color=ccqqqq] (12.605000008214395,3.737974983511336) -- (12.630000007993567,3.7294685778847327);
\draw[line width=1.2pt,color=ccqqqq] (12.630000007993567,3.7294685778847327) -- (12.65500000777274,3.720974455895468);
\draw[line width=1.2pt,color=ccqqqq] (12.65500000777274,3.720974455895468) -- (12.680000007551913,3.7124925644823357);
\draw[line width=1.2pt,color=ccqqqq] (12.680000007551913,3.7124925644823357) -- (12.705000007331085,3.7040228509650452);
\draw[line width=1.2pt,color=ccqqqq] (12.705000007331085,3.7040228509650452) -- (12.730000007110258,3.695565263040402);
\draw[line width=1.2pt,color=ccqqqq] (12.730000007110258,3.695565263040402) -- (12.75500000688943,3.6871197487785423);
\draw[line width=1.2pt,color=ccqqqq] (12.75500000688943,3.6871197487785423) -- (12.780000006668603,3.6786862566192085);
\draw[line width=1.2pt,color=ccqqqq] (12.780000006668603,3.6786862566192085) -- (12.805000006447775,3.6702647353680815);
\draw[line width=1.2pt,color=ccqqqq] (12.805000006447775,3.6702647353680815) -- (12.830000006226948,3.661855134193152);
\draw[line width=1.2pt,color=ccqqqq] (12.830000006226948,3.661855134193152) -- (12.85500000600612,3.653457402621142);
\draw[line width=1.2pt,color=ccqqqq] (12.85500000600612,3.653457402621142) -- (12.880000005785293,3.6450714905339705);
\draw[line width=1.2pt,color=ccqqqq] (12.880000005785293,3.6450714905339705) -- (12.905000005564466,3.6366973481652627);
\draw[line width=1.2pt,color=ccqqqq] (12.905000005564466,3.6366973481652627) -- (12.930000005343638,3.6283349260969064);
\draw[line width=1.2pt,color=ccqqqq] (12.930000005343638,3.6283349260969064) -- (12.95500000512281,3.619984175255647);
\draw[line width=1.2pt,color=ccqqqq] (12.95500000512281,3.619984175255647) -- (12.980000004901983,3.6116450469097297);
\draw[line width=1.2pt,color=ccqqqq] (12.980000004901983,3.6116450469097297) -- (13.005000004681156,3.603317492665581);
\draw[line width=1.2pt,color=ccqqqq] (13.005000004681156,3.603317492665581) -- (13.030000004460328,3.5950014644645316);
\draw[line width=1.2pt,color=ccqqqq] (13.030000004460328,3.5950014644645316) -- (13.055000004239501,3.5866969145795817);
\draw[line width=1.2pt,color=ccqqqq] (13.055000004239501,3.5866969145795817) -- (13.080000004018673,3.5784037956122043);
\draw[line width=1.2pt,color=ccqqqq] (13.080000004018673,3.5784037956122043) -- (13.105000003797846,3.57012206048919);
\draw[line width=1.2pt,color=ccqqqq] (13.105000003797846,3.57012206048919) -- (13.130000003577019,3.561851662459529);
\draw[line width=1.2pt,color=ccqqqq] (13.130000003577019,3.561851662459529) -- (13.155000003356191,3.5535925550913303);
\draw[line width=1.2pt,color=ccqqqq] (13.155000003356191,3.5535925550913303) -- (13.180000003135364,3.5453446922687846);
\draw[line width=1.2pt,color=ccqqqq] (13.180000003135364,3.5453446922687846) -- (13.205000002914536,3.5371080281891567);
\draw[line width=1.2pt,color=ccqqqq] (13.205000002914536,3.5371080281891567) -- (13.230000002693709,3.5288825173598184);
\draw[line width=1.2pt,color=ccqqqq] (13.230000002693709,3.5288825173598184) -- (13.255000002472881,3.520668114595318);
\draw[line width=1.2pt,color=ccqqqq] (13.255000002472881,3.520668114595318) -- (13.280000002252054,3.512464775014484);
\draw[line width=1.2pt,color=ccqqqq] (13.280000002252054,3.512464775014484) -- (13.305000002031226,3.5042724540375643);
\draw[line width=1.2pt,color=ccqqqq] (13.305000002031226,3.5042724540375643) -- (13.330000001810399,3.496091107383398);
\draw[line width=1.2pt,color=ccqqqq] (13.330000001810399,3.496091107383398) -- (13.355000001589572,3.487920691066627);
\draw[line width=1.2pt,color=ccqqqq] (13.355000001589572,3.487920691066627) -- (13.380000001368744,3.479761161394932);
\draw[line width=1.2pt,color=ccqqqq] (13.380000001368744,3.479761161394932) -- (13.405000001147917,3.471612474966311);
\draw[line width=1.2pt,color=ccqqqq] (13.405000001147917,3.471612474966311) -- (13.43000000092709,3.463474588666383);
\draw[line width=1.2pt,color=ccqqqq] (13.43000000092709,3.463474588666383) -- (13.455000000706262,3.4553474596657265);
\draw[line width=1.2pt,color=ccqqqq] (13.455000000706262,3.4553474596657265) -- (13.480000000485434,3.4472310454172517);
\draw[line width=1.2pt,color=ccqqqq] (13.480000000485434,3.4472310454172517) -- (13.505000000264607,3.4391253036536016);
\draw[line width=1.2pt,color=ccqqqq] (13.505000000264607,3.4391253036536016) -- (13.53000000004378,3.43103019238458);
\draw[line width=1.2pt,color=ccqqqq] (13.53000000004378,3.43103019238458) -- (13.554999999822952,3.42294566989462);
\draw[line width=1.2pt,color=ccqqqq] (13.554999999822952,3.42294566989462) -- (13.579999999602125,3.4148716947402704);
\draw[line width=1.2pt,color=ccqqqq] (13.579999999602125,3.4148716947402704) -- (13.604999999381297,3.4068082257477217);
\draw[line width=1.2pt,color=ccqqqq] (13.604999999381297,3.4068082257477217) -- (13.62999999916047,3.398755222010352);
\draw[line width=1.2pt,color=ccqqqq] (13.62999999916047,3.398755222010352) -- (13.654999998939642,3.3907126428863075);
\draw[line width=1.2pt,color=ccqqqq] (13.654999998939642,3.3907126428863075) -- (13.679999998718815,3.382680447996113);
\draw[line width=1.2pt,color=ccqqqq] (13.679999998718815,3.382680447996113) -- (13.704999998497987,3.374658597220301);
\draw[line width=1.2pt,color=ccqqqq] (13.704999998497987,3.374658597220301) -- (13.72999999827716,3.3666470506970763);
\draw[line width=1.2pt,color=ccqqqq] (13.72999999827716,3.3666470506970763) -- (13.754999998056332,3.358645768820007);
\draw[line width=1.2pt,color=ccqqqq] (13.754999998056332,3.358645768820007) -- (13.779999997835505,3.3506547122357357);
\draw[line width=1.2pt,color=ccqqqq] (13.779999997835505,3.3506547122357357) -- (13.804999997614678,3.3426738418417257);
\draw[line width=1.2pt,color=ccqqqq] (13.804999997614678,3.3426738418417257) -- (13.82999999739385,3.3347031187840264);
\draw[line width=1.2pt,color=ccqqqq] (13.82999999739385,3.3347031187840264) -- (13.854999997173023,3.3267425044550656);
\draw[line width=1.2pt,color=ccqqqq] (13.854999997173023,3.3267425044550656) -- (13.879999996952195,3.3187919604914704);
\draw[line width=1.2pt,color=ccqqqq] (13.879999996952195,3.3187919604914704) -- (13.904999996731368,3.310851448771909);
\draw[line width=1.2pt,color=ccqqqq] (13.904999996731368,3.310851448771909) -- (13.92999999651054,3.3029209314149552);
\draw[line width=1.2pt,color=ccqqqq] (13.92999999651054,3.3029209314149552) -- (13.954999996289713,3.295000370776987);
\draw[line width=1.2pt,color=ffqqqq] (0.06000009320000301,23.54369038834948) -- (0.06000009320000301,23.54369038834948);
\draw[line width=1.2pt,color=ffqqqq] (0.06000009320000301,23.54369038834948) -- (0.07000009263434179,23.54360038668206);
\draw[line width=1.2pt,color=ffqqqq] (0.07000009263434179,23.54360038668206) -- (0.08000009206868056,23.543330385035006);
\draw[line width=1.2pt,color=ffqqqq] (0.08000009206868056,23.543330385035006) -- (0.09000009150301932,23.542880383408313);
\draw[line width=1.2pt,color=ffqqqq] (0.09000009150301932,23.542880383408313) -- (0.1000000909373581,23.54225038180199);
\draw[line width=1.2pt,color=ffqqqq] (0.1000000909373581,23.54225038180199) -- (0.11000009037169686,23.541440380216027);
\draw[line width=1.2pt,color=ffqqqq] (0.11000009037169686,23.541440380216027) -- (0.12000008980603563,23.540450378650426);
\draw[line width=1.2pt,color=ffqqqq] (0.12000008980603563,23.540450378650426) -- (0.1300000892403744,23.539280377105193);
\draw[line width=1.2pt,color=ffqqqq] (0.1300000892403744,23.539280377105193) -- (0.14000008867471317,23.53793037558032);
\draw[line width=1.2pt,color=ffqqqq] (0.14000008867471317,23.53793037558032) -- (0.15000008810905194,23.53640037407581);
\draw[line width=1.2pt,color=ffqqqq] (0.15000008810905194,23.53640037407581) -- (0.1600000875433907,23.53469037259167);
\draw[line width=1.2pt,color=ffqqqq] (0.1600000875433907,23.53469037259167) -- (0.17000008697772948,23.532800371127887);
\draw[line width=1.2pt,color=ffqqqq] (0.17000008697772948,23.532800371127887) -- (0.18000008641206824,23.530730369684473);
\draw[line width=1.2pt,color=ffqqqq] (0.18000008641206824,23.530730369684473) -- (0.190000085846407,23.52848036826142);
\draw[line width=1.2pt,color=ffqqqq] (0.190000085846407,23.52848036826142) -- (0.20000008528074578,23.52605036685873);
\draw[line width=1.2pt,color=ffqqqq] (0.20000008528074578,23.52605036685873) -- (0.21000008471508455,23.523440365476407);
\draw[line width=1.2pt,color=ffqqqq] (0.21000008471508455,23.523440365476407) -- (0.22000008414942332,23.520650364114445);
\draw[line width=1.2pt,color=ffqqqq] (0.22000008414942332,23.520650364114445) -- (0.2300000835837621,23.517680362772847);
\draw[line width=1.2pt,color=ffqqqq] (0.2300000835837621,23.517680362772847) -- (0.24000008301810086,23.514530361451616);
\draw[line width=1.2pt,color=ffqqqq] (0.24000008301810086,23.514530361451616) -- (0.25000008245243965,23.511200360150745);
\draw[line width=1.2pt,color=ffqqqq] (0.25000008245243965,23.511200360150745) -- (0.2600000818867784,23.50769035887024);
\draw[line width=1.2pt,color=ffqqqq] (0.2600000818867784,23.50769035887024) -- (0.2700000813211172,23.504000357610096);
\draw[line width=1.2pt,color=ffqqqq] (0.2700000813211172,23.504000357610096) -- (0.28000008075545596,23.50013035637032);
\draw[line width=1.2pt,color=ffqqqq] (0.28000008075545596,23.50013035637032) -- (0.29000008018979473,23.496080355150905);
\draw[line width=1.2pt,color=ffqqqq] (0.29000008018979473,23.496080355150905) -- (0.3000000796241335,23.491850353951854);
\draw[line width=1.2pt,color=ffqqqq] (0.3000000796241335,23.491850353951854) -- (0.31000007905847227,23.487440352773167);
\draw[line width=1.2pt,color=ffqqqq] (0.31000007905847227,23.487440352773167) -- (0.32000007849281104,23.482850351614843);
\draw[line width=1.2pt,color=ffqqqq] (0.32000007849281104,23.482850351614843) -- (0.3300000779271498,23.478080350476887);
\draw[line width=1.2pt,color=ffqqqq] (0.3300000779271498,23.478080350476887) -- (0.3400000773614886,23.47313034935929);
\draw[line width=1.2pt,color=ffqqqq] (0.3400000773614886,23.47313034935929) -- (0.35000007679582734,23.46800034826206);
\draw[line width=1.2pt,color=ffqqqq] (0.35000007679582734,23.46800034826206) -- (0.3600000762301661,23.46269034718519);
\draw[line width=1.2pt,color=ffqqqq] (0.3600000762301661,23.46269034718519) -- (0.3700000756645049,23.457200346128687);
\draw[line width=1.2pt,color=ffqqqq] (0.3700000756645049,23.457200346128687) -- (0.38000007509884365,23.451530345092547);
\draw[line width=1.2pt,color=ffqqqq] (0.38000007509884365,23.451530345092547) -- (0.3900000745331824,23.44568034407677);
\draw[line width=1.2pt,color=ffqqqq] (0.3900000745331824,23.44568034407677) -- (0.4000000739675212,23.439650343081357);
\draw[line width=1.2pt,color=ffqqqq] (0.4000000739675212,23.439650343081357) -- (0.41000007340185995,23.43344034210631);
\draw[line width=1.2pt,color=ffqqqq] (0.41000007340185995,23.43344034210631) -- (0.4200000728361987,23.427050341151624);
\draw[line width=1.2pt,color=ffqqqq] (0.4200000728361987,23.427050341151624) -- (0.4300000722705375,23.420480340217303);
\draw[line width=1.2pt,color=ffqqqq] (0.4300000722705375,23.420480340217303) -- (0.44000007170487626,23.413730339303346);
\draw[line width=1.2pt,color=ffqqqq] (0.44000007170487626,23.413730339303346) -- (0.45000007113921503,23.406800338409752);
\draw[line width=1.2pt,color=ffqqqq] (0.45000007113921503,23.406800338409752) -- (0.4600000705735538,23.399690337536523);
\draw[line width=1.2pt,color=ffqqqq] (0.4600000705735538,23.399690337536523) -- (0.47000007000789257,23.392400336683657);
\draw[line width=1.2pt,color=ffqqqq] (0.47000007000789257,23.392400336683657) -- (0.48000006944223134,23.384930335851156);
\draw[line width=1.2pt,color=ffqqqq] (0.48000006944223134,23.384930335851156) -- (0.4900000688765701,23.377280335039018);
\draw[line width=1.2pt,color=ffqqqq] (0.4900000688765701,23.377280335039018) -- (0.5000000683109089,23.36945033424724);
\draw[line width=1.2pt,color=ffqqqq] (0.5000000683109089,23.36945033424724) -- (0.5100000677452478,23.36144033347583);
\draw[line width=1.2pt,color=ffqqqq] (0.5100000677452478,23.36144033347583) -- (0.5200000671795866,23.353250332724784);
\draw[line width=1.2pt,color=ffqqqq] (0.5200000671795866,23.353250332724784) -- (0.5300000666139254,23.3448803319941);
\draw[line width=1.2pt,color=ffqqqq] (0.5300000666139254,23.3448803319941) -- (0.5400000660482642,23.336330331283783);
\draw[line width=1.2pt,color=ffqqqq] (0.5400000660482642,23.336330331283783) -- (0.550000065482603,23.327600330593825);
\draw[line width=1.2pt,color=ffqqqq] (0.550000065482603,23.327600330593825) -- (0.5600000649169419,23.318690329924234);
\draw[line width=1.2pt,color=ffqqqq] (0.5600000649169419,23.318690329924234) -- (0.5700000643512807,23.309600329275007);
\draw[line width=1.2pt,color=ffqqqq] (0.5700000643512807,23.309600329275007) -- (0.5800000637856195,23.300330328646144);
\draw[line width=1.2pt,color=ffqqqq] (0.5800000637856195,23.300330328646144) -- (0.5900000632199583,23.29088032803764);
\draw[line width=1.2pt,color=ffqqqq] (0.5900000632199583,23.29088032803764) -- (0.6000000626542972,23.281250327449506);
\draw[line width=1.2pt,color=ffqqqq] (0.6000000626542972,23.281250327449506) -- (0.610000062088636,23.27144032688173);
\draw[line width=1.2pt,color=ffqqqq] (0.610000062088636,23.27144032688173) -- (0.6200000615229748,23.261450326334323);
\draw[line width=1.2pt,color=ffqqqq] (0.6200000615229748,23.261450326334323) -- (0.6300000609573136,23.25128032580728);
\draw[line width=1.2pt,color=ffqqqq] (0.6300000609573136,23.25128032580728) -- (0.6400000603916525,23.240930325300596);
\draw[line width=1.2pt,color=ffqqqq] (0.6400000603916525,23.240930325300596) -- (0.6500000598259913,23.23040032481428);
\draw[line width=1.2pt,color=ffqqqq] (0.6500000598259913,23.23040032481428) -- (0.6600000592603301,23.219690324348328);
\draw[line width=1.2pt,color=ffqqqq] (0.6600000592603301,23.219690324348328) -- (0.6700000586946689,23.208800323902736);
\draw[line width=1.2pt,color=ffqqqq] (0.6700000586946689,23.208800323902736) -- (0.6800000581290078,23.19773032347751);
\draw[line width=1.2pt,color=ffqqqq] (0.6800000581290078,23.19773032347751) -- (0.6900000575633466,23.186480323072647);
\draw[line width=1.2pt,color=ffqqqq] (0.6900000575633466,23.186480323072647) -- (0.7000000569976854,23.17505032268815);
\draw[line width=1.2pt,color=ffqqqq] (0.7000000569976854,23.17505032268815) -- (0.7100000564320242,23.163440322324014);
\draw[line width=1.2pt,color=ffqqqq] (0.7100000564320242,23.163440322324014) -- (0.7200000558663631,23.151650321980245);
\draw[line width=1.2pt,color=ffqqqq] (0.7200000558663631,23.151650321980245) -- (0.7300000553007019,23.139680321656837);
\draw[line width=1.2pt,color=ffqqqq] (0.7300000553007019,23.139680321656837) -- (0.7400000547350407,23.12753032135379);
\draw[line width=1.2pt,color=ffqqqq] (0.7400000547350407,23.12753032135379) -- (0.7500000541693795,23.115200321071114);
\draw[line width=1.2pt,color=ffqqqq] (0.7500000541693795,23.115200321071114) -- (0.7600000536037184,23.102690320808797);
\draw[line width=1.2pt,color=ffqqqq] (0.7600000536037184,23.102690320808797) -- (0.7700000530380572,23.090000320566848);
\draw[line width=1.2pt,color=ffqqqq] (0.7700000530380572,23.090000320566848) -- (0.780000052472396,23.07713032034526);
\draw[line width=1.2pt,color=ffqqqq] (0.780000052472396,23.07713032034526) -- (0.7900000519067348,23.064080320144033);
\draw[line width=1.2pt,color=ffqqqq] (0.7900000519067348,23.064080320144033) -- (0.8000000513410737,23.050850319963175);
\draw[line width=1.2pt,color=ffqqqq] (0.8000000513410737,23.050850319963175) -- (0.8100000507754125,23.037440319802677);
\draw[line width=1.2pt,color=ffqqqq] (0.8100000507754125,23.037440319802677) -- (0.8200000502097513,23.023850319662543);
\draw[line width=1.2pt,color=ffqqqq] (0.8200000502097513,23.023850319662543) -- (0.8300000496440901,23.010080319542773);
\draw[line width=1.2pt,color=ffqqqq] (0.8300000496440901,23.010080319542773) -- (0.840000049078429,22.99613031944337);
\draw[line width=1.2pt,color=ffqqqq] (0.840000049078429,22.99613031944337) -- (0.8500000485127678,22.982000319364328);
\draw[line width=1.2pt,color=ffqqqq] (0.8500000485127678,22.982000319364328) -- (0.8600000479471066,22.96769031930565);
\draw[line width=1.2pt,color=ffqqqq] (0.8600000479471066,22.96769031930565) -- (0.8700000473814454,22.953200319267335);
\draw[line width=1.2pt,color=ffqqqq] (0.8700000473814454,22.953200319267335) -- (0.8800000468157843,22.938530319249388);
\draw[line width=1.2pt,color=ffqqqq] (0.8800000468157843,22.938530319249388) -- (0.8900000462501231,22.9236803192518);
\draw[line width=1.2pt,color=ffqqqq] (0.8900000462501231,22.9236803192518) -- (0.9000000456844619,22.908650319274578);
\draw[line width=1.2pt,color=ffqqqq] (0.9000000456844619,22.908650319274578) -- (0.9100000451188007,22.89344031931772);
\draw[line width=1.2pt,color=ffqqqq] (0.9100000451188007,22.89344031931772) -- (0.9200000445531396,22.878050319381224);
\draw[line width=1.2pt,color=ffqqqq] (0.9200000445531396,22.878050319381224) -- (0.9300000439874784,22.862480319465092);
\draw[line width=1.2pt,color=ffqqqq] (0.9300000439874784,22.862480319465092) -- (0.9400000434218172,22.846730319569325);
\draw[line width=1.2pt,color=ffqqqq] (0.9400000434218172,22.846730319569325) -- (0.950000042856156,22.83080031969392);
\draw[line width=1.2pt,color=ffqqqq] (0.950000042856156,22.83080031969392) -- (0.9600000422904948,22.81469031983888);
\draw[line width=1.2pt,color=ffqqqq] (0.9600000422904948,22.81469031983888) -- (0.9700000417248337,22.798400320004205);
\draw[line width=1.2pt,color=ffqqqq] (0.9700000417248337,22.798400320004205) -- (0.9800000411591725,22.781930320189893);
\draw[line width=1.2pt,color=ffqqqq] (0.9800000411591725,22.781930320189893) -- (0.9900000405935113,22.765280320395945);
\draw[line width=1.2pt,color=ffqqqq] (0.9900000405935113,22.765280320395945) -- (1.0000000400278501,22.74845032062236);
\draw[line width=1.2pt,color=ffqqqq] (1.0000000400278501,22.74845032062236) -- (1.0100000394621889,22.73144032086914);
\draw[line width=1.2pt,color=ffqqqq] (1.0100000394621889,22.73144032086914) -- (1.0200000388965276,22.714250321136284);
\draw[line width=1.2pt,color=ffqqqq] (1.0200000388965276,22.714250321136284) -- (1.0300000383308663,22.69688032142379);
\draw[line width=1.2pt,color=ffqqqq] (1.0300000383308663,22.69688032142379) -- (1.040000037765205,22.679330321731662);
\draw[line width=1.2pt,color=ffqqqq] (1.040000037765205,22.679330321731662) -- (1.0500000371995437,22.661600322059897);
\draw[line width=1.2pt,color=ffqqqq] (1.0500000371995437,22.661600322059897) -- (1.0600000366338824,22.643690322408496);
\draw[line width=1.2pt,color=ffqqqq] (1.0600000366338824,22.643690322408496) -- (1.0700000360682211,22.62560032277746);
\draw[line width=1.2pt,color=ffqqqq] (1.0700000360682211,22.62560032277746) -- (1.0800000355025599,22.607330323166785);
\draw[line width=1.2pt,color=ffqqqq] (1.0800000355025599,22.607330323166785) -- (1.0900000349368986,22.588880323576475);
\draw[line width=1.2pt,color=ffqqqq] (1.0900000349368986,22.588880323576475) -- (1.1000000343712373,22.57025032400653);
\draw[line width=1.2pt,color=ffqqqq] (1.1000000343712373,22.57025032400653) -- (1.110000033805576,22.551440324456948);
\draw[line width=1.2pt,color=ffqqqq] (1.110000033805576,22.551440324456948) -- (1.1200000332399147,22.532450324927726);
\draw[line width=1.2pt,color=ffqqqq] (1.1200000332399147,22.532450324927726) -- (1.1300000326742534,22.513280325418872);
\draw[line width=1.2pt,color=ffqqqq] (1.1300000326742534,22.513280325418872) -- (1.1400000321085921,22.493930325930382);
\draw[line width=1.2pt,color=ffqqqq] (1.1400000321085921,22.493930325930382) -- (1.1500000315429308,22.474400326462256);
\draw[line width=1.2pt,color=ffqqqq] (1.1500000315429308,22.474400326462256) -- (1.1600000309772696,22.45469032701449);
\draw[line width=1.2pt,color=ffqqqq] (1.1600000309772696,22.45469032701449) -- (1.1700000304116083,22.43480032758709);
\draw[line width=1.2pt,color=ffqqqq] (1.1700000304116083,22.43480032758709) -- (1.180000029845947,22.414730328180056);
\draw[line width=1.2pt,color=ffqqqq] (1.180000029845947,22.414730328180056) -- (1.1900000292802857,22.394480328793385);
\draw[line width=1.2pt,color=ffqqqq] (1.1900000292802857,22.394480328793385) -- (1.2000000287146244,22.374050329427075);
\draw[line width=1.2pt,color=ffqqqq] (1.2000000287146244,22.374050329427075) -- (1.2100000281489631,22.35344033008113);
\draw[line width=1.2pt,color=ffqqqq] (1.2100000281489631,22.35344033008113) -- (1.2200000275833018,22.332650330755552);
\draw[line width=1.2pt,color=ffqqqq] (1.2200000275833018,22.332650330755552) -- (1.2300000270176406,22.311680331450333);
\draw[line width=1.2pt,color=ffqqqq] (1.2300000270176406,22.311680331450333) -- (1.2400000264519793,22.29053033216548);
\draw[line width=1.2pt,color=ffqqqq] (1.2400000264519793,22.29053033216548) -- (1.250000025886318,22.269200332900994);
\draw[line width=1.2pt,color=ffqqqq] (1.250000025886318,22.269200332900994) -- (1.2600000253206567,22.247690333656866);
\draw[line width=1.2pt,color=ffqqqq] (1.2600000253206567,22.247690333656866) -- (1.2700000247549954,22.226000334433106);
\draw[line width=1.2pt,color=ffqqqq] (1.2700000247549954,22.226000334433106) -- (1.2800000241893341,22.204130335229706);
\draw[line width=1.2pt,color=ffqqqq] (1.2800000241893341,22.204130335229706) -- (1.2900000236236728,22.182080336046674);
\draw[line width=1.2pt,color=ffqqqq] (1.2900000236236728,22.182080336046674) -- (1.3000000230580115,22.159850336884002);
\draw[line width=1.2pt,color=ffqqqq] (1.3000000230580115,22.159850336884002) -- (1.3100000224923503,22.137440337741698);
\draw[line width=1.2pt,color=ffqqqq] (1.3100000224923503,22.137440337741698) -- (1.320000021926689,22.114850338619753);
\draw[line width=1.2pt,color=ffqqqq] (1.320000021926689,22.114850338619753) -- (1.3300000213610277,22.092080339518176);
\draw[line width=1.2pt,color=ffqqqq] (1.3300000213610277,22.092080339518176) -- (1.3400000207953664,22.06913034043696);
\draw[line width=1.2pt,color=ffqqqq] (1.3400000207953664,22.06913034043696) -- (1.3500000202297051,22.04600034137611);
\draw[line width=1.2pt,color=ffqqqq] (1.3500000202297051,22.04600034137611) -- (1.3600000196640438,22.022690342335622);
\draw[line width=1.2pt,color=ffqqqq] (1.3600000196640438,22.022690342335622) -- (1.3700000190983825,21.9992003433155);
\draw[line width=1.2pt,color=ffqqqq] (1.3700000190983825,21.9992003433155) -- (1.3800000185327213,21.97553034431574);
\draw[line width=1.2pt,color=ffqqqq] (1.3800000185327213,21.97553034431574) -- (1.39000001796706,21.951680345336342);
\draw[line width=1.2pt,color=ffqqqq] (1.39000001796706,21.951680345336342) -- (1.4000000174013987,21.927650346377312);
\draw[line width=1.2pt,color=ffqqqq] (1.4000000174013987,21.927650346377312) -- (1.4100000168357374,21.903440347438643);
\draw[line width=1.2pt,color=ffqqqq] (1.4100000168357374,21.903440347438643) -- (1.420000016270076,21.879050348520337);
\draw[line width=1.2pt,color=ffqqqq] (1.420000016270076,21.879050348520337) -- (1.4300000157044148,21.8544803496224);
\draw[line width=1.2pt,color=ffqqqq] (1.4300000157044148,21.8544803496224) -- (1.4400000151387535,21.82973035074482);
\draw[line width=1.2pt,color=ffqqqq] (1.4400000151387535,21.82973035074482) -- (1.4500000145730922,21.804800351887607);
\draw[line width=1.2pt,color=ffqqqq] (1.4500000145730922,21.804800351887607) -- (1.460000014007431,21.77969035305076);
\draw[line width=1.2pt,color=ffqqqq] (1.460000014007431,21.77969035305076) -- (1.4700000134417697,21.754400354234274);
\draw[line width=1.2pt,color=ffqqqq] (1.4700000134417697,21.754400354234274) -- (1.4800000128761084,21.728930355438152);
\draw[line width=1.2pt,color=ffqqqq] (1.4800000128761084,21.728930355438152) -- (1.490000012310447,21.703280356662393);
\draw[line width=1.2pt,color=ffqqqq] (1.490000012310447,21.703280356662393) -- (1.5000000117447858,21.677450357907002);
\draw[line width=1.2pt,color=ffqqqq] (1.5000000117447858,21.677450357907002) -- (1.5100000111791245,21.65144035917197);
\draw[line width=1.2pt,color=ffqqqq] (1.5100000111791245,21.65144035917197) -- (1.5200000106134632,21.625250360457304);
\draw[line width=1.2pt,color=ffqqqq] (1.5200000106134632,21.625250360457304) -- (1.530000010047802,21.598880361763);
\draw[line width=1.2pt,color=ffqqqq] (1.530000010047802,21.598880361763) -- (1.5400000094821407,21.572330363089062);
\draw[line width=1.2pt,color=ffqqqq] (1.5400000094821407,21.572330363089062) -- (1.5500000089164794,21.545600364435487);
\draw[line width=1.2pt,color=ffqqqq] (1.5500000089164794,21.545600364435487) -- (1.560000008350818,21.51869036580228);
\draw[line width=1.2pt,color=ffqqqq] (1.560000008350818,21.51869036580228) -- (1.5700000077851568,21.49160036718943);
\draw[line width=1.2pt,color=ffqqqq] (1.5700000077851568,21.49160036718943) -- (1.5800000072194955,21.464330368596947);
\draw[line width=1.2pt,color=ffqqqq] (1.5800000072194955,21.464330368596947) -- (1.5900000066538342,21.436880370024827);
\draw[line width=1.2pt,color=ffqqqq] (1.5900000066538342,21.436880370024827) -- (1.600000006088173,21.40925037147307);
\draw[line width=1.2pt,color=ffqqqq] (1.600000006088173,21.40925037147307) -- (1.6100000055225117,21.38144037294168);
\draw[line width=1.2pt,color=ffqqqq] (1.6100000055225117,21.38144037294168) -- (1.6200000049568504,21.35345037443065);
\draw[line width=1.2pt,color=ffqqqq] (1.6200000049568504,21.35345037443065) -- (1.630000004391189,21.325280375939986);
\draw[line width=1.2pt,color=ffqqqq] (1.630000004391189,21.325280375939986) -- (1.6400000038255278,21.296930377469685);
\draw[line width=1.2pt,color=ffqqqq] (1.6400000038255278,21.296930377469685) -- (1.6500000032598665,21.26840037901975);
\draw[line width=1.2pt,color=ffqqqq] (1.6500000032598665,21.26840037901975) -- (1.6600000026942052,21.239690380590176);
\draw[line width=1.2pt,color=ffqqqq] (1.6600000026942052,21.239690380590176) -- (1.670000002128544,21.210800382180967);
\draw[line width=1.2pt,color=ffqqqq] (1.670000002128544,21.210800382180967) -- (1.6800000015628826,21.18173038379212);
\draw[line width=1.2pt,color=ffqqqq] (1.6800000015628826,21.18173038379212) -- (1.6900000009972214,21.152480385423637);
\draw[line width=1.2pt,color=ffqqqq] (1.6900000009972214,21.152480385423637) -- (1.70000000043156,21.12305038707552);
\draw[line width=1.2pt,color=ffqqqq] (1.70000000043156,21.12305038707552) -- (1.7099999998658988,21.093440388747766);
\draw[line width=1.2pt,color=ffqqqq] (1.7099999998658988,21.093440388747766) -- (1.7199999993002375,21.063650390440376);
\draw[line width=1.2pt,color=ffqqqq] (1.7199999993002375,21.063650390440376) -- (1.7299999987345762,21.03368039215335);
\draw[line width=1.2pt,color=ffqqqq] (1.7299999987345762,21.03368039215335) -- (1.739999998168915,21.003530393886688);
\draw[line width=1.2pt,color=ffqqqq] (1.739999998168915,21.003530393886688) -- (1.7499999976032536,20.97320039564039);
\draw[line width=1.2pt,color=ffqqqq] (1.7499999976032536,20.97320039564039) -- (1.7599999970375924,20.942690397414452);
\draw[line width=1.2pt,color=ffqqqq] (1.7599999970375924,20.942690397414452) -- (1.769999996471931,20.912000399208882);
\draw[line width=1.2pt,color=ffqqqq] (1.769999996471931,20.912000399208882) -- (1.7799999959062698,20.881130401023675);
\draw[line width=1.2pt,color=ffqqqq] (1.7799999959062698,20.881130401023675) -- (1.7899999953406085,20.85008040285883);
\draw[line width=1.2pt,color=ffqqqq] (1.7899999953406085,20.85008040285883) -- (1.7999999947749472,20.81885040471435);
\draw[line width=1.2pt,color=ffqqqq] (1.7999999947749472,20.81885040471435) -- (1.809999994209286,20.787440406590235);
\draw[line width=1.2pt,color=ffqqqq] (1.809999994209286,20.787440406590235) -- (1.8199999936436246,20.755850408486484);
\draw[line width=1.2pt,color=ffqqqq] (1.8199999936436246,20.755850408486484) -- (1.8299999930779633,20.724080410403094);
\draw[line width=1.2pt,color=ffqqqq] (1.8299999930779633,20.724080410403094) -- (1.839999992512302,20.69213041234007);
\draw[line width=1.2pt,color=ffqqqq] (1.839999992512302,20.69213041234007) -- (1.8499999919466408,20.660000414297407);
\draw[line width=1.2pt,color=ffqqqq] (1.8499999919466408,20.660000414297407) -- (1.8599999913809795,20.62769041627511);
\draw[line width=1.2pt,color=ffqqqq] (1.8599999913809795,20.62769041627511) -- (1.8699999908153182,20.59520041827318);
\draw[line width=1.2pt,color=ffqqqq] (1.8699999908153182,20.59520041827318) -- (1.879999990249657,20.56253042029161);
\draw[line width=1.2pt,color=ffqqqq] (1.879999990249657,20.56253042029161) -- (1.8899999896839956,20.529680422330404);
\draw[line width=1.2pt,color=ffqqqq] (1.8899999896839956,20.529680422330404) -- (1.8999999891183343,20.496650424389564);
\draw[line width=1.2pt,color=ffqqqq] (1.8999999891183343,20.496650424389564) -- (1.909999988552673,20.463440426469084);
\draw[line width=1.2pt,color=ffqqqq] (1.909999988552673,20.463440426469084) -- (1.9199999879870118,20.43005042856897);
\draw[line width=1.2pt,color=ffqqqq] (1.9199999879870118,20.43005042856897) -- (1.9299999874213505,20.39648043068922);
\draw[line width=1.2pt,color=ffqqqq] (1.9299999874213505,20.39648043068922) -- (1.9399999868556892,20.362730432829835);
\draw[line width=1.2pt,color=ffqqqq] (1.9399999868556892,20.362730432829835) -- (1.949999986290028,20.32880043499081);
\draw[line width=1.2pt,color=ffqqqq] (1.949999986290028,20.32880043499081) -- (1.9599999857243666,20.29469043717215);
\draw[line width=1.2pt,color=ffqqqq] (1.9599999857243666,20.29469043717215) -- (1.9699999851587053,20.260400439373857);
\draw[line width=1.2pt,color=ffqqqq] (1.9699999851587053,20.260400439373857) -- (1.979999984593044,20.225930441595924);
\draw[line width=1.2pt,color=ffqqqq] (1.979999984593044,20.225930441595924) -- (1.9899999840273828,20.19128044383836);
\draw[line width=1.2pt,color=ffqqqq] (1.9899999840273828,20.19128044383836) -- (1.9999999834617215,20.156450446101154);
\draw[line width=1.2pt,color=ffqqqq] (1.9999999834617215,20.156450446101154) -- (2.00999998289606,20.121440448384313);
\draw[line width=1.2pt,color=ffqqqq] (2.00999998289606,20.121440448384313) -- (2.019999982330399,20.08625045068784);
\draw[line width=1.2pt,color=ffqqqq] (2.019999982330399,20.08625045068784) -- (2.029999981764738,20.050880453011725);
\draw[line width=1.2pt,color=ffqqqq] (2.029999981764738,20.050880453011725) -- (2.039999981199077,20.015330455355976);
\draw[line width=1.2pt,color=ffqqqq] (2.039999981199077,20.015330455355976) -- (2.049999980633416,19.97960045772059);
\draw[line width=1.2pt,color=ffqqqq] (2.049999980633416,19.97960045772059) -- (2.059999980067755,19.943690460105568);
\draw[line width=1.2pt,color=ffqqqq] (2.059999980067755,19.943690460105568) -- (2.069999979502094,19.90760046251091);
\draw[line width=1.2pt,color=ffqqqq] (2.069999979502094,19.90760046251091) -- (2.0799999789364327,19.871330464936616);
\draw[line width=1.2pt,color=ffqqqq] (2.0799999789364327,19.871330464936616) -- (2.0899999783707717,19.834880467382686);
\draw[line width=1.2pt,color=ffqqqq] (2.0899999783707717,19.834880467382686) -- (2.0999999778051106,19.79825046984912);
\draw[line width=1.2pt,color=ffqqqq] (2.0999999778051106,19.79825046984912) -- (2.1099999772394495,19.761440472335916);
\draw[line width=1.2pt,color=ffqqqq] (2.1099999772394495,19.761440472335916) -- (2.1199999766737885,19.724450474843078);
\draw[line width=1.2pt,color=ffqqqq] (2.1199999766737885,19.724450474843078) -- (2.1299999761081274,19.687280477370603);
\draw[line width=1.2pt,color=ffqqqq] (2.1299999761081274,19.687280477370603) -- (2.1399999755424663,19.649930479918492);
\draw[line width=1.2pt,color=ffqqqq] (2.1399999755424663,19.649930479918492) -- (2.1499999749768053,19.612400482486745);
\draw[line width=1.2pt,color=ffqqqq] (2.1499999749768053,19.612400482486745) -- (2.159999974411144,19.574690485075358);
\draw[line width=1.2pt,color=ffqqqq] (2.159999974411144,19.574690485075358) -- (2.169999973845483,19.53680048768434);
\draw[line width=1.2pt,color=ffqqqq] (2.169999973845483,19.53680048768434) -- (2.179999973279822,19.498730490313683);
\draw[line width=1.2pt,color=ffqqqq] (2.179999973279822,19.498730490313683) -- (2.189999972714161,19.46048049296339);
\draw[line width=1.2pt,color=ffqqqq] (2.189999972714161,19.46048049296339) -- (2.1999999721485,19.422050495633464);
\draw[line width=1.2pt,color=ffqqqq] (2.1999999721485,19.422050495633464) -- (2.209999971582839,19.3834404983239);
\draw[line width=1.2pt,color=ffqqqq] (2.209999971582839,19.3834404983239) -- (2.219999971017178,19.3446505010347);
\draw[line width=1.2pt,color=ffqqqq] (2.219999971017178,19.3446505010347) -- (2.2299999704515168,19.30568050376586);
\draw[line width=1.2pt,color=ffqqqq] (2.2299999704515168,19.30568050376586) -- (2.2399999698858557,19.266530506517388);
\draw[line width=1.2pt,color=ffqqqq] (2.2399999698858557,19.266530506517388) -- (2.2499999693201946,19.227200509289275);
\draw[line width=1.2pt,color=ffqqqq] (2.2499999693201946,19.227200509289275) -- (2.2599999687545336,19.18769051208153);
\draw[line width=1.2pt,color=ffqqqq] (2.2599999687545336,19.18769051208153) -- (2.2699999681888725,19.14800051489415);
\draw[line width=1.2pt,color=ffqqqq] (2.2699999681888725,19.14800051489415) -- (2.2799999676232114,19.108130517727133);
\draw[line width=1.2pt,color=ffqqqq] (2.2799999676232114,19.108130517727133) -- (2.2899999670575504,19.068080520580477);
\draw[line width=1.2pt,color=ffqqqq] (2.2899999670575504,19.068080520580477) -- (2.2999999664918893,19.027850523454187);
\draw[line width=1.2pt,color=ffqqqq] (2.2999999664918893,19.027850523454187) -- (2.3099999659262282,18.98744052634826);
\draw[line width=1.2pt,color=ffqqqq] (2.3099999659262282,18.98744052634826) -- (2.319999965360567,18.946850529262697);
\draw[line width=1.2pt,color=ffqqqq] (2.319999965360567,18.946850529262697) -- (2.329999964794906,18.9060805321975);
\draw[line width=1.2pt,color=ffqqqq] (2.329999964794906,18.9060805321975) -- (2.339999964229245,18.865130535152662);
\draw[line width=1.2pt,color=ffqqqq] (2.339999964229245,18.865130535152662) -- (2.349999963663584,18.824000538128193);
\draw[line width=1.2pt,color=ffqqqq] (2.349999963663584,18.824000538128193) -- (2.359999963097923,18.782690541124083);
\draw[line width=1.2pt,color=ffqqqq] (2.359999963097923,18.782690541124083) -- (2.369999962532262,18.74120054414034);
\draw[line width=1.2pt,color=ffqqqq] (2.369999962532262,18.74120054414034) -- (2.379999961966601,18.69953054717696);
\draw[line width=1.2pt,color=ffqqqq] (2.379999961966601,18.69953054717696) -- (2.3899999614009397,18.65768055023394);
\draw[line width=1.2pt,color=ffqqqq] (2.3899999614009397,18.65768055023394) -- (2.3999999608352787,18.61565055331129);
\draw[line width=1.2pt,color=ffqqqq] (2.3999999608352787,18.61565055331129) -- (2.4099999602696176,18.573440556409004);
\draw[line width=1.2pt,color=ffqqqq] (2.4099999602696176,18.573440556409004) -- (2.4199999597039565,18.531050559527078);
\draw[line width=1.2pt,color=ffqqqq] (2.4199999597039565,18.531050559527078) -- (2.4299999591382955,18.488480562665515);
\draw[line width=1.2pt,color=ffqqqq] (2.4299999591382955,18.488480562665515) -- (2.4399999585726344,18.44573056582432);
\draw[line width=1.2pt,color=ffqqqq] (2.4399999585726344,18.44573056582432) -- (2.4499999580069733,18.402800569003485);
\draw[line width=1.2pt,color=ffqqqq] (2.4499999580069733,18.402800569003485) -- (2.4599999574413123,18.359690572203014);
\draw[line width=1.2pt,color=ffqqqq] (2.4599999574413123,18.359690572203014) -- (2.469999956875651,18.31640057542291);
\draw[line width=1.2pt,color=ffqqqq] (2.469999956875651,18.31640057542291) -- (2.47999995630999,18.272930578663168);
\draw[line width=1.2pt,color=ffqqqq] (2.47999995630999,18.272930578663168) -- (2.489999955744329,18.22928058192379);
\draw[line width=1.2pt,color=ffqqqq] (2.489999955744329,18.22928058192379) -- (2.499999955178668,18.185450585204773);
\draw[line width=1.2pt,color=ffqqqq] (2.499999955178668,18.185450585204773) -- (2.509999954613007,18.14144058850612);
\draw[line width=1.2pt,color=ffqqqq] (2.509999954613007,18.14144058850612) -- (2.519999954047346,18.097250591827837);
\draw[line width=1.2pt,color=ffqqqq] (2.519999954047346,18.097250591827837) -- (2.529999953481685,18.052880595169913);
\draw[line width=1.2pt,color=ffqqqq] (2.529999953481685,18.052880595169913) -- (2.5399999529160238,18.008330598532353);
\draw[line width=1.2pt,color=ffqqqq] (2.5399999529160238,18.008330598532353) -- (2.5499999523503627,17.963600601915157);
\draw[line width=1.2pt,color=ffqqqq] (2.5499999523503627,17.963600601915157) -- (2.5599999517847016,17.918690605318325);
\draw[line width=1.2pt,color=ffqqqq] (2.5599999517847016,17.918690605318325) -- (2.5699999512190406,17.87360060874186);
\draw[line width=1.2pt,color=ffqqqq] (2.5699999512190406,17.87360060874186) -- (2.5799999506533795,17.828330612185752);
\draw[line width=1.2pt,color=ffqqqq] (2.5799999506533795,17.828330612185752) -- (2.5899999500877184,17.782880615650015);
\draw[line width=1.2pt,color=ffqqqq] (2.5899999500877184,17.782880615650015) -- (2.5999999495220574,17.737250619134638);
\draw[line width=1.2pt,color=ffqqqq] (2.5999999495220574,17.737250619134638) -- (2.6099999489563963,17.691440622639625);
\draw[line width=1.2pt,color=ffqqqq] (2.6099999489563963,17.691440622639625) -- (2.6199999483907352,17.645450626164976);
\draw[line width=1.2pt,color=ffqqqq] (2.6199999483907352,17.645450626164976) -- (2.629999947825074,17.59928062971069);
\draw[line width=1.2pt,color=ffqqqq] (2.629999947825074,17.59928062971069) -- (2.639999947259413,17.55293063327677);
\draw[line width=1.2pt,color=ffqqqq] (2.639999947259413,17.55293063327677) -- (2.649999946693752,17.506400636863212);
\draw[line width=1.2pt,color=ffqqqq] (2.649999946693752,17.506400636863212) -- (2.659999946128091,17.45969064047002);
\draw[line width=1.2pt,color=ffqqqq] (2.659999946128091,17.45969064047002) -- (2.66999994556243,17.41280064409719);
\draw[line width=1.2pt,color=ffqqqq] (2.66999994556243,17.41280064409719) -- (2.679999944996769,17.36573064774472);
\draw[line width=1.2pt,color=ffqqqq] (2.679999944996769,17.36573064774472) -- (2.689999944431108,17.318480651412617);
\draw[line width=1.2pt,color=ffqqqq] (2.689999944431108,17.318480651412617) -- (2.6999999438654467,17.27105065510088);
\draw[line width=1.2pt,color=ffqqqq] (2.6999999438654467,17.27105065510088) -- (2.7099999432997857,17.223440658809505);
\draw[line width=1.2pt,color=ffqqqq] (2.7099999432997857,17.223440658809505) -- (2.7199999427341246,17.175650662538494);
\draw[line width=1.2pt,color=ffqqqq] (2.7199999427341246,17.175650662538494) -- (2.7299999421684635,17.127680666287848);
\draw[line width=1.2pt,color=ffqqqq] (2.7299999421684635,17.127680666287848) -- (2.7399999416028025,17.079530670057565);
\draw[line width=1.2pt,color=ffqqqq] (2.7399999416028025,17.079530670057565) -- (2.7499999410371414,17.031200673847643);
\draw[line width=1.2pt,color=ffqqqq] (2.7499999410371414,17.031200673847643) -- (2.7599999404714803,16.982690677658088);
\draw[line width=1.2pt,color=ffqqqq] (2.7599999404714803,16.982690677658088) -- (2.7699999399058193,16.934000681488897);
\draw[line width=1.2pt,color=ffqqqq] (2.7699999399058193,16.934000681488897) -- (2.779999939340158,16.885130685340066);
\draw[line width=1.2pt,color=ffqqqq] (2.779999939340158,16.885130685340066) -- (2.789999938774497,16.836080689211602);
\draw[line width=1.2pt,color=ffqqqq] (2.789999938774497,16.836080689211602) -- (2.799999938208836,16.786850693103503);
\draw[line width=1.2pt,color=ffqqqq] (2.799999938208836,16.786850693103503) -- (2.809999937643175,16.737440697015767);
\draw[line width=1.2pt,color=ffqqqq] (2.809999937643175,16.737440697015767) -- (2.819999937077514,16.687850700948392);
\draw[line width=1.2pt,color=ffqqqq] (2.819999937077514,16.687850700948392) -- (2.829999936511853,16.638080704901384);
\draw[line width=1.2pt,color=ffqqqq] (2.829999936511853,16.638080704901384) -- (2.839999935946192,16.58813070887474);
\draw[line width=1.2pt,color=ffqqqq] (2.839999935946192,16.58813070887474) -- (2.8499999353805308,16.538000712868456);
\draw[line width=1.2pt,color=ffqqqq] (2.8499999353805308,16.538000712868456) -- (2.8599999348148697,16.48769071688254);
\draw[line width=1.2pt,color=ffqqqq] (2.8599999348148697,16.48769071688254) -- (2.8699999342492086,16.437200720916984);
\draw[line width=1.2pt,color=ffqqqq] (2.8699999342492086,16.437200720916984) -- (2.8799999336835476,16.386530724971795);
\draw[line width=1.2pt,color=ffqqqq] (2.8799999336835476,16.386530724971795) -- (2.8899999331178865,16.335680729046967);
\draw[line width=1.2pt,color=ffqqqq] (2.8899999331178865,16.335680729046967) -- (2.8999999325522254,16.284650733142506);
\draw[line width=1.2pt,color=ffqqqq] (2.8999999325522254,16.284650733142506) -- (2.9099999319865644,16.233440737258405);
\draw[line width=1.2pt,color=ffqqqq] (2.9099999319865644,16.233440737258405) -- (2.9199999314209033,16.182050741394672);
\draw[line width=1.2pt,color=ffqqqq] (2.9199999314209033,16.182050741394672) -- (2.9299999308552422,16.1304807455513);
\draw[line width=1.2pt,color=ffqqqq] (2.9299999308552422,16.1304807455513) -- (2.939999930289581,16.078730749728294);
\draw[line width=1.2pt,color=ffqqqq] (2.939999930289581,16.078730749728294) -- (2.94999992972392,16.02680075392565);
\draw[line width=1.2pt,color=ffqqqq] (2.94999992972392,16.02680075392565) -- (2.959999929158259,15.97469075814337);
\draw[line width=1.2pt,color=ffqqqq] (2.959999929158259,15.97469075814337) -- (2.969999928592598,15.922400762381454);
\draw[line width=1.2pt,color=ffqqqq] (2.969999928592598,15.922400762381454) -- (2.979999928026937,15.8699307666399);
\draw[line width=1.2pt,color=ffqqqq] (2.979999928026937,15.8699307666399) -- (2.989999927461276,15.817280770918712);
\draw[line width=1.2pt,color=ffqqqq] (2.989999927461276,15.817280770918712) -- (2.999999926895615,15.764450775217888);
\draw[line width=1.2pt,color=ffqqqq] (2.999999926895615,15.764450775217888) -- (3.0099999263299537,15.711440779537426);
\draw[line width=1.2pt,color=ffqqqq] (3.0099999263299537,15.711440779537426) -- (3.0199999257642927,15.65825078387733);
\draw[line width=1.2pt,color=ffqqqq] (3.0199999257642927,15.65825078387733) -- (3.0299999251986316,15.604880788237597);
\draw[line width=1.2pt,color=ffqqqq] (3.0299999251986316,15.604880788237597) -- (3.0399999246329705,15.551330792618227);
\draw[line width=1.2pt,color=ffqqqq] (3.0399999246329705,15.551330792618227) -- (3.0499999240673095,15.49760079701922);
\draw[line width=1.2pt,color=ffqqqq] (3.0499999240673095,15.49760079701922) -- (3.0599999235016484,15.44369080144058);
\draw[line width=1.2pt,color=ffqqqq] (3.0599999235016484,15.44369080144058) -- (3.0699999229359873,15.389600805882301);
\draw[line width=1.2pt,color=ffqqqq] (3.0699999229359873,15.389600805882301) -- (3.0799999223703263,15.335330810344386);
\draw[line width=1.2pt,color=ffqqqq] (3.0799999223703263,15.335330810344386) -- (3.089999921804665,15.280880814826835);
\draw[line width=1.2pt,color=ffqqqq] (3.089999921804665,15.280880814826835) -- (3.099999921239004,15.22625081932965);
\draw[line width=1.2pt,color=ffqqqq] (3.099999921239004,15.22625081932965) -- (3.109999920673343,15.171440823852826);
\draw[line width=1.2pt,color=ffqqqq] (3.109999920673343,15.171440823852826) -- (3.119999920107682,15.116450828396367);
\draw[line width=1.2pt,color=ffqqqq] (3.119999920107682,15.116450828396367) -- (3.129999919542021,15.061280832960273);
\draw[line width=1.2pt,color=ffqqqq] (3.129999919542021,15.061280832960273) -- (3.13999991897636,15.005930837544541);
\draw[line width=1.2pt,color=ffqqqq] (3.13999991897636,15.005930837544541) -- (3.149999918410699,14.950400842149174);
\draw[line width=1.2pt,color=ffqqqq] (3.149999918410699,14.950400842149174) -- (3.1599999178450378,14.89469084677417);
\draw[line width=1.2pt,color=ffqqqq] (3.1599999178450378,14.89469084677417) -- (3.1699999172793767,14.838800851419528);
\draw[line width=1.2pt,color=ffqqqq] (3.1699999172793767,14.838800851419528) -- (3.1799999167137156,14.782730856085253);
\draw[line width=1.2pt,color=ffqqqq] (3.1799999167137156,14.782730856085253) -- (3.1899999161480546,14.726480860771339);
\draw[line width=1.2pt,color=ffqqqq] (3.1899999161480546,14.726480860771339) -- (3.1999999155823935,14.670050865477792);
\draw[line width=1.2pt,color=ffqqqq] (3.1999999155823935,14.670050865477792) -- (3.2099999150167324,14.613440870204606);
\draw[line width=1.2pt,color=ffqqqq] (3.2099999150167324,14.613440870204606) -- (3.2199999144510714,14.556650874951785);
\draw[line width=1.2pt,color=ffqqqq] (3.2199999144510714,14.556650874951785) -- (3.2299999138854103,14.499680879719328);
\draw[line width=1.2pt,color=ffqqqq] (3.2299999138854103,14.499680879719328) -- (3.2399999133197492,14.442530884507235);
\draw[line width=1.2pt,color=ffqqqq] (3.2399999133197492,14.442530884507235) -- (3.249999912754088,14.385200889315504);
\draw[line width=1.2pt,color=ffqqqq] (3.249999912754088,14.385200889315504) -- (3.259999912188427,14.327690894144139);
\draw[line width=1.2pt,color=ffqqqq] (3.259999912188427,14.327690894144139) -- (3.269999911622766,14.270000898993136);
\draw[line width=1.2pt,color=ffqqqq] (3.269999911622766,14.270000898993136) -- (3.279999911057105,14.212130903862498);
\draw[line width=1.2pt,color=ffqqqq] (3.279999911057105,14.212130903862498) -- (3.289999910491444,14.154080908752224);
\draw[line width=1.2pt,color=ffqqqq] (3.289999910491444,14.154080908752224) -- (3.299999909925783,14.095850913662314);
\draw[line width=1.2pt,color=ffqqqq] (3.299999909925783,14.095850913662314) -- (3.309999909360122,14.037440918592765);
\draw[line width=1.2pt,color=ffqqqq] (3.309999909360122,14.037440918592765) -- (3.3199999087944607,13.978850923543582);
\draw[line width=1.2pt,color=ffqqqq] (3.3199999087944607,13.978850923543582) -- (3.3299999082287997,13.920080928514762);
\draw[line width=1.2pt,color=ffqqqq] (3.3299999082287997,13.920080928514762) -- (3.3399999076631386,13.861130933506308);
\draw[line width=1.2pt,color=ffqqqq] (3.3399999076631386,13.861130933506308) -- (3.3499999070974775,13.802000938518216);
\draw[line width=1.2pt,color=ffqqqq] (3.3499999070974775,13.802000938518216) -- (3.3599999065318165,13.742690943550489);
\draw[line width=1.2pt,color=ffqqqq] (3.3599999065318165,13.742690943550489) -- (3.3699999059661554,13.683200948603123);
\draw[line width=1.2pt,color=ffqqqq] (3.3699999059661554,13.683200948603123) -- (3.3799999054004943,13.623530953676124);
\draw[line width=1.2pt,color=ffqqqq] (3.3799999054004943,13.623530953676124) -- (3.3899999048348333,13.563680958769487);
\draw[line width=1.2pt,color=ffqqqq] (3.3899999048348333,13.563680958769487) -- (3.399999904269172,13.503650963883214);
\draw[line width=1.2pt,color=ffqqqq] (3.399999904269172,13.503650963883214) -- (3.409999903703511,13.443440969017304);
\draw[line width=1.2pt,color=ffqqqq] (3.409999903703511,13.443440969017304) -- (3.41999990313785,13.383050974171761);
\draw[line width=1.2pt,color=ffqqqq] (3.41999990313785,13.383050974171761) -- (3.429999902572189,13.32248097934658);
\draw[line width=1.2pt,color=ffqqqq] (3.429999902572189,13.32248097934658) -- (3.439999902006528,13.261730984541762);
\draw[line width=1.2pt,color=ffqqqq] (3.439999902006528,13.261730984541762) -- (3.449999901440867,13.200800989757308);
\draw[line width=1.2pt,color=ffqqqq] (3.449999901440867,13.200800989757308) -- (3.459999900875206,13.139690994993217);
\draw[line width=1.2pt,color=ffqqqq] (3.459999900875206,13.139690994993217) -- (3.4699999003095447,13.07840100024949);
\draw[line width=1.2pt,color=ffqqqq] (3.4699999003095447,13.07840100024949) -- (3.4799998997438837,13.016931005526128);
\draw[line width=1.2pt,color=ffqqqq] (3.4799998997438837,13.016931005526128) -- (3.4899998991782226,12.95528101082313);
\draw[line width=1.2pt,color=ffqqqq] (3.4899998991782226,12.95528101082313) -- (3.4999998986125616,12.893451016140496);
\draw[line width=1.2pt,color=ffqqqq] (3.4999998986125616,12.893451016140496) -- (3.5099998980469005,12.831441021478224);
\draw[line width=1.2pt,color=ffqqqq] (3.5099998980469005,12.831441021478224) -- (3.5199998974812394,12.769251026836317);
\draw[line width=1.2pt,color=ffqqqq] (3.5199998974812394,12.769251026836317) -- (3.5299998969155784,12.706881032214774);
\draw[line width=1.2pt,color=ffqqqq] (3.5299998969155784,12.706881032214774) -- (3.5399998963499173,12.644331037613595);
\draw[line width=1.2pt,color=ffqqqq] (3.5399998963499173,12.644331037613595) -- (3.5499998957842562,12.581601043032778);
\draw[line width=1.2pt,color=ffqqqq] (3.5499998957842562,12.581601043032778) -- (3.559999895218595,12.518691048472327);
\draw[line width=1.2pt,color=ffqqqq] (3.559999895218595,12.518691048472327) -- (3.569999894652934,12.455601053932238);
\draw[line width=1.2pt,color=ffqqqq] (3.569999894652934,12.455601053932238) -- (3.579999894087273,12.392331059412513);
\draw[line width=1.2pt,color=ffqqqq] (3.579999894087273,12.392331059412513) -- (3.589999893521612,12.328881064913153);
\draw[line width=1.2pt,color=ffqqqq] (3.589999893521612,12.328881064913153) -- (3.599999892955951,12.265251070434157);
\draw[line width=1.2pt,color=ffqqqq] (3.599999892955951,12.265251070434157) -- (3.60999989239029,12.201441075975524);
\draw[line width=1.2pt,color=ffqqqq] (3.60999989239029,12.201441075975524) -- (3.619999891824629,12.137451081537254);
\draw[line width=1.2pt,color=ffqqqq] (3.619999891824629,12.137451081537254) -- (3.6299998912589677,12.073281087119348);
\draw[line width=1.2pt,color=ffqqqq] (3.6299998912589677,12.073281087119348) -- (3.6399998906933067,12.008931092721808);
\draw[line width=1.2pt,color=ffqqqq] (3.6399998906933067,12.008931092721808) -- (3.6499998901276456,11.944401098344628);
\draw[line width=1.2pt,color=ffqqqq] (3.6499998901276456,11.944401098344628) -- (3.6599998895619845,11.879691103987815);
\draw[line width=1.2pt,color=ffqqqq] (3.6599998895619845,11.879691103987815) -- (3.6699998889963235,11.814801109651365);
\draw[line width=1.2pt,color=ffqqqq] (3.6699998889963235,11.814801109651365) -- (3.6799998884306624,11.749731115335278);
\draw[line width=1.2pt,color=ffqqqq] (3.6799998884306624,11.749731115335278) -- (3.6899998878650013,11.684481121039555);
\draw[line width=1.2pt,color=ffqqqq] (3.6899998878650013,11.684481121039555) -- (3.6999998872993403,11.619051126764196);
\draw[line width=1.2pt,color=ffqqqq] (3.6999998872993403,11.619051126764196) -- (3.709999886733679,11.553441132509201);
\draw[line width=1.2pt,color=ffqqqq] (3.709999886733679,11.553441132509201) -- (3.719999886168018,11.48765113827457);
\draw[line width=1.2pt,color=ffqqqq] (3.719999886168018,11.48765113827457) -- (3.729999885602357,11.421681144060303);
\draw[line width=1.2pt,color=ffqqqq] (3.729999885602357,11.421681144060303) -- (3.739999885036696,11.3555311498664);
\draw[line width=1.2pt,color=ffqqqq] (3.739999885036696,11.3555311498664) -- (3.749999884471035,11.28920115569286);
\draw[line width=1.2pt,color=ffqqqq] (3.749999884471035,11.28920115569286) -- (3.759999883905374,11.222691161539684);
\draw[line width=1.2pt,color=ffqqqq] (3.759999883905374,11.222691161539684) -- (3.769999883339713,11.15600116740687);
\draw[line width=1.2pt,color=ffqqqq] (3.769999883339713,11.15600116740687) -- (3.7799998827740517,11.089131173294422);
\draw[line width=1.2pt,color=ffqqqq] (3.7799998827740517,11.089131173294422) -- (3.7899998822083907,11.022081179202338);
\draw[line width=1.2pt,color=ffqqqq] (3.7899998822083907,11.022081179202338) -- (3.7999998816427296,10.954851185130618);
\draw[line width=1.2pt,color=ffqqqq] (3.7999998816427296,10.954851185130618) -- (3.8099998810770686,10.887441191079262);
\draw[line width=1.2pt,color=ffqqqq] (3.8099998810770686,10.887441191079262) -- (3.8199998805114075,10.819851197048267);
\draw[line width=1.2pt,color=ffqqqq] (3.8199998805114075,10.819851197048267) -- (3.8299998799457464,10.752081203037639);
\draw[line width=1.2pt,color=ffqqqq] (3.8299998799457464,10.752081203037639) -- (3.8399998793800854,10.684131209047372);
\draw[line width=1.2pt,color=ffqqqq] (3.8399998793800854,10.684131209047372) -- (3.8499998788144243,10.61600121507747);
\draw[line width=1.2pt,color=ffqqqq] (3.8499998788144243,10.61600121507747) -- (3.8599998782487632,10.547691221127932);
\draw[line width=1.2pt,color=ffqqqq] (3.8599998782487632,10.547691221127932) -- (3.869999877683102,10.479201227198757);
\draw[line width=1.2pt,color=ffqqqq] (3.869999877683102,10.479201227198757) -- (3.879999877117441,10.410531233289948);
\draw[line width=1.2pt,color=ffqqqq] (3.879999877117441,10.410531233289948) -- (3.88999987655178,10.3416812394015);
\draw[line width=1.2pt,color=ffqqqq] (3.88999987655178,10.3416812394015) -- (3.899999875986119,10.272651245533417);
\draw[line width=1.2pt,color=ffqqqq] (3.899999875986119,10.272651245533417) -- (3.909999875420458,10.2034412516857);
\draw[line width=1.2pt,color=ffqqqq] (3.909999875420458,10.2034412516857) -- (3.919999874854797,10.134051257858342);
\draw[line width=1.2pt,color=ffqqqq] (3.919999874854797,10.134051257858342) -- (3.9299998742891358,10.06448126405135);
\draw[line width=1.2pt,color=ffqqqq] (3.9299998742891358,10.06448126405135) -- (3.9399998737234747,9.994731270264724);
\draw[line width=1.2pt,color=ffqqqq] (3.9399998737234747,9.994731270264724) -- (3.9499998731578136,9.92480127649846);
\draw[line width=1.2pt,color=ffqqqq] (3.9499998731578136,9.92480127649846) -- (3.9599998725921526,9.85469128275256);
\draw[line width=1.2pt,color=ffqqqq] (3.9599998725921526,9.85469128275256) -- (3.9699998720264915,9.784401289027024);
\draw[line width=1.2pt,color=ffqqqq] (3.9699998720264915,9.784401289027024) -- (3.9799998714608305,9.713931295321851);
\draw[line width=1.2pt,color=ffqqqq] (3.9799998714608305,9.713931295321851) -- (3.9899998708951694,9.643281301637042);
\draw[line width=1.2pt,color=ffqqqq] (3.9899998708951694,9.643281301637042) -- (3.9999998703295083,9.572451307972598);
\draw[line width=1.2pt,color=ffqqqq] (3.9999998703295083,9.572451307972598) -- (4.009999869763847,9.501441314328517);
\draw[line width=1.2pt,color=ffqqqq] (4.009999869763847,9.501441314328517) -- (4.019999869198186,9.430251320704802);
\draw[line width=1.2pt,color=ffqqqq] (4.019999869198186,9.430251320704802) -- (4.029999868632524,9.358881327101452);
\draw[line width=1.2pt,color=ffqqqq] (4.029999868632524,9.358881327101452) -- (4.039999868066863,9.287331333518466);
\draw[line width=1.2pt,color=ffqqqq] (4.039999868066863,9.287331333518466) -- (4.049999867501201,9.215601339955843);
\end{tikzpicture}
\end{center} 

\begin{enumerate}
\item Marie habite à 2,5~km d'un central téléphonique. Quel débit de connexion obtient-elle ? \point{2}
 
\item Paul obtient un débit de $20$ Mbits/s. 
À quelle distance du central téléphonique habite-t-il ? \point{2} 
\item Pour pouvoir recevoir la télévision par internet, le débit doit être au moins de $15$ Mbits/s. 
À quelle distance maximum du central doit-on habiter pour pouvoir recevoir la télévision par internet ?  \point{2}

 
\end{enumerate}



\end{ExoCu}
%%%%%%%%%%%%%%%%%%%%%%%%%%%
\begin{ExoCu}{Représenter.}{1234}{2}{0}{0}{0}{0}

\begin{minipage}{8cm}

On donne la courbe $\mathcal{C}_f$ d'une fonction $f$.

\begin{enumerate}
\item Comment se nomme la variable ? \point{1}
\item Quelle est son unité ? \point{1}
\item Que représente ce graphique ?  \point{2}
\item Quelle est la taille des pousses  le 9 ème jour ?\point{2}
\item Au bout de combien de jours les pousses dépassent 30 mm ?\point{2}
\item Déterminer un antécédent de $50$ par $f$ ?\point{2}
\item Déterminer une valeur approximative de l'image de $4$ par $f$ ?\point{1}

\end{enumerate}
\end{minipage}
\begin{minipage}{8cm}
\begin{center}
\definecolor{ccqqqq}{rgb}{0.8,0.,0.}
\definecolor{cqcqcq}{rgb}{0.7529411764705882,0.7529411764705882,0.7529411764705882}
\begin{tikzpicture}[line cap=round,line join=round,>=triangle 45,x=0.6837606837606837cm,y=0.1399215686274509cm]
\draw [color=cqcqcq,, xstep=0.6837606837606837cm,ystep=1.3992156862745089cm] (-0.88,-2.578475336322897) grid (10.82,54.59641255605382);
\draw[->,color=black] (-0.88,0.) -- (10.82,0.);
\foreach \x in {,1.,2.,3.,4.,5.,6.,7.,8.,9.,10.}
\draw[shift={(\x,0)},color=black] (0pt,2pt) -- (0pt,-2pt) node[below] {\footnotesize $\x$};
\draw[->,color=black] (0.,-2.578475336322897) -- (0.,54.59641255605382);
\foreach \y in {,10.,20.,30.,40.,50.}
\draw[shift={(0,\y)},color=black] (2pt,0pt) -- (-2pt,0pt) node[left] {\footnotesize $\y$};
\draw[color=black] (0pt,-10pt) node[right] {\footnotesize $0$};
\clip(-0.88,-2.578475336322897) rectangle (10.82,54.59641255605382);
\draw[line width=1.2pt,color=ccqqqq] (5.599999991374502E-8,0.0) -- (0.0,0.0);
\draw[line width=1.2pt,color=ccqqqq] (0.0,0.0) -- (0.02499999966110996,0.0);
\draw[line width=1.2pt,color=ccqqqq] (0.02499999966110996,0.0) -- (0.04999999932221992,0.0012499999661109962);
\draw[line width=1.2pt,color=ccqqqq] (0.04999999932221992,0.0012499999661109962) -- (0.07499999898332987,0.002812499923749741);
\draw[line width=1.2pt,color=ccqqqq] (0.07499999898332987,0.002812499923749741) -- (0.09999999864443984,0.004999999864443985);
\draw[line width=1.2pt,color=ccqqqq] (0.09999999864443984,0.004999999864443985) -- (0.12499999830554981,0.007812499788193728);
\draw[line width=1.2pt,color=ccqqqq] (0.12499999830554981,0.007812499788193728) -- (0.14999999796665978,0.011249999694998968);
\draw[line width=1.2pt,color=ccqqqq] (0.14999999796665978,0.011249999694998968) -- (0.17499999762776974,0.015312499584859708);
\draw[line width=1.2pt,color=ccqqqq] (0.17499999762776974,0.015312499584859708) -- (0.1999999972888797,0.019999999457775947);
\draw[line width=1.2pt,color=ccqqqq] (0.1999999972888797,0.019999999457775947) -- (0.22499999694998968,0.025312499313747683);
\draw[line width=1.2pt,color=ccqqqq] (0.22499999694998968,0.025312499313747683) -- (0.24999999661109965,0.031249999152774918);
\draw[line width=1.2pt,color=ccqqqq] (0.24999999661109965,0.031249999152774918) -- (0.2749999962722096,0.03781249897485765);
\draw[line width=1.2pt,color=ccqqqq] (0.2749999962722096,0.03781249897485765) -- (0.29999999593331955,0.04499999877999587);
\draw[line width=1.2pt,color=ccqqqq] (0.29999999593331955,0.04499999877999587) -- (0.3249999955944295,0.05281249856818959);
\draw[line width=1.2pt,color=ccqqqq] (0.3249999955944295,0.05281249856818959) -- (0.34999999525553943,0.06124999833943881);
\draw[line width=1.2pt,color=ccqqqq] (0.34999999525553943,0.06124999833943881) -- (0.37499999491664937,0.07031249809374353);
\draw[line width=1.2pt,color=ccqqqq] (0.37499999491664937,0.07031249809374353) -- (0.3999999945777593,0.07999999783110374);
\draw[line width=1.2pt,color=ccqqqq] (0.3999999945777593,0.07999999783110374) -- (0.42499999423886925,0.09031249755151945);
\draw[line width=1.2pt,color=ccqqqq] (0.42499999423886925,0.09031249755151945) -- (0.4499999938999792,0.10124999725499065);
\draw[line width=1.2pt,color=ccqqqq] (0.4499999938999792,0.10124999725499065) -- (0.47499999356108913,0.11281249694151736);
\draw[line width=1.2pt,color=ccqqqq] (0.47499999356108913,0.11281249694151736) -- (0.49999999322219907,0.12499999661109956);
\draw[line width=1.2pt,color=ccqqqq] (0.49999999322219907,0.12499999661109956) -- (0.524999992883309,0.13781249626373726);
\draw[line width=1.2pt,color=ccqqqq] (0.524999992883309,0.13781249626373726) -- (0.549999992544419,0.15124999589943047);
\draw[line width=1.2pt,color=ccqqqq] (0.549999992544419,0.15124999589943047) -- (0.574999992205529,0.1653124955181792);
\draw[line width=1.2pt,color=ccqqqq] (0.574999992205529,0.1653124955181792) -- (0.599999991866639,0.17999999511998344);
\draw[line width=1.2pt,color=ccqqqq] (0.599999991866639,0.17999999511998344) -- (0.624999991527749,0.19531249470484316);
\draw[line width=1.2pt,color=ccqqqq] (0.624999991527749,0.19531249470484316) -- (0.649999991188859,0.21124999427275837);
\draw[line width=1.2pt,color=ccqqqq] (0.649999991188859,0.21124999427275837) -- (0.674999990849969,0.2278124938237291);
\draw[line width=1.2pt,color=ccqqqq] (0.674999990849969,0.2278124938237291) -- (0.699999990511079,0.24499999335775532);
\draw[line width=1.2pt,color=ccqqqq] (0.699999990511079,0.24499999335775532) -- (0.724999990172189,0.26281249287483704);
\draw[line width=1.2pt,color=ccqqqq] (0.724999990172189,0.26281249287483704) -- (0.749999989833299,0.2812499923749743);
\draw[line width=1.2pt,color=ccqqqq] (0.749999989833299,0.2812499923749743) -- (0.774999989494409,0.300312491858167);
\draw[line width=1.2pt,color=ccqqqq] (0.774999989494409,0.300312491858167) -- (0.799999989155519,0.3199999913244152);
\draw[line width=1.2pt,color=ccqqqq] (0.799999989155519,0.3199999913244152) -- (0.824999988816629,0.34031249077371895);
\draw[line width=1.2pt,color=ccqqqq] (0.824999988816629,0.34031249077371895) -- (0.8499999884777389,0.3612499902060782);
\draw[line width=1.2pt,color=ccqqqq] (0.8499999884777389,0.3612499902060782) -- (0.8749999881388489,0.3828124896214929);
\draw[line width=1.2pt,color=ccqqqq] (0.8749999881388489,0.3828124896214929) -- (0.8999999877999589,0.4049999890199631);
\draw[line width=1.2pt,color=ccqqqq] (0.8999999877999589,0.4049999890199631) -- (0.9249999874610689,0.42781248840148883);
\draw[line width=1.2pt,color=ccqqqq] (0.9249999874610689,0.42781248840148883) -- (0.9499999871221789,0.45124998776607006);
\draw[line width=1.2pt,color=ccqqqq] (0.9499999871221789,0.45124998776607006) -- (0.9749999867832889,0.47531248711370677);
\draw[line width=1.2pt,color=ccqqqq] (0.9749999867832889,0.47531248711370677) -- (0.9999999864443989,0.499999986444399);
\draw[line width=1.2pt,color=ccqqqq] (0.9999999864443989,0.499999986444399) -- (1.0249999861055088,0.5253124857581466);
\draw[line width=1.2pt,color=ccqqqq] (1.0249999861055088,0.5253124857581466) -- (1.0499999857666187,0.5512499850549497);
\draw[line width=1.2pt,color=ccqqqq] (1.0499999857666187,0.5512499850549497) -- (1.0749999854277286,0.5778124843348084);
\draw[line width=1.2pt,color=ccqqqq] (1.0749999854277286,0.5778124843348084) -- (1.0999999850888385,0.6049999835977224);
\draw[line width=1.2pt,color=ccqqqq] (1.0999999850888385,0.6049999835977224) -- (1.1249999847499483,0.632812482843692);
\draw[line width=1.2pt,color=ccqqqq] (1.1249999847499483,0.632812482843692) -- (1.1499999844110582,0.6612499820727171);
\draw[line width=1.2pt,color=ccqqqq] (1.1499999844110582,0.6612499820727171) -- (1.174999984072168,0.6903124812847976);
\draw[line width=1.2pt,color=ccqqqq] (1.174999984072168,0.6903124812847976) -- (1.199999983733278,0.7199999804799337);
\draw[line width=1.2pt,color=ccqqqq] (1.199999983733278,0.7199999804799337) -- (1.2249999833943879,0.7503124796581253);
\draw[line width=1.2pt,color=ccqqqq] (1.2249999833943879,0.7503124796581253) -- (1.2499999830554978,0.7812499788193723);
\draw[line width=1.2pt,color=ccqqqq] (1.2499999830554978,0.7812499788193723) -- (1.2749999827166076,0.8128124779636748);
\draw[line width=1.2pt,color=ccqqqq] (1.2749999827166076,0.8128124779636748) -- (1.2999999823777175,0.8449999770910329);
\draw[line width=1.2pt,color=ccqqqq] (1.2999999823777175,0.8449999770910329) -- (1.3249999820388274,0.8778124762014464);
\draw[line width=1.2pt,color=ccqqqq] (1.3249999820388274,0.8778124762014464) -- (1.3499999816999373,0.9112499752949155);
\draw[line width=1.2pt,color=ccqqqq] (1.3499999816999373,0.9112499752949155) -- (1.3749999813610472,0.9453124743714401);
\draw[line width=1.2pt,color=ccqqqq] (1.3749999813610472,0.9453124743714401) -- (1.399999981022157,0.9799999734310201);
\draw[line width=1.2pt,color=ccqqqq] (1.399999981022157,0.9799999734310201) -- (1.424999980683267,1.0153124724736555);
\draw[line width=1.2pt,color=ccqqqq] (1.424999980683267,1.0153124724736555) -- (1.4499999803443768,1.0512499714993466);
\draw[line width=1.2pt,color=ccqqqq] (1.4499999803443768,1.0512499714993466) -- (1.4749999800054867,1.0878124705080932);
\draw[line width=1.2pt,color=ccqqqq] (1.4749999800054867,1.0878124705080932) -- (1.4999999796665966,1.1249999694998951);
\draw[line width=1.2pt,color=ccqqqq] (1.4999999796665966,1.1249999694998951) -- (1.5249999793277065,1.1628124684747525);
\draw[line width=1.2pt,color=ccqqqq] (1.5249999793277065,1.1628124684747525) -- (1.5499999789888164,1.2012499674326655);
\draw[line width=1.2pt,color=ccqqqq] (1.5499999789888164,1.2012499674326655) -- (1.5749999786499262,1.240312466373634);
\draw[line width=1.2pt,color=ccqqqq] (1.5749999786499262,1.240312466373634) -- (1.5999999783110361,1.2799999652976581);
\draw[line width=1.2pt,color=ccqqqq] (1.5999999783110361,1.2799999652976581) -- (1.624999977972146,1.3203124642047375);
\draw[line width=1.2pt,color=ccqqqq] (1.624999977972146,1.3203124642047375) -- (1.649999977633256,1.3612499630948725);
\draw[line width=1.2pt,color=ccqqqq] (1.649999977633256,1.3612499630948725) -- (1.6749999772943658,1.4028124619680629);
\draw[line width=1.2pt,color=ccqqqq] (1.6749999772943658,1.4028124619680629) -- (1.6999999769554757,1.444999960824309);
\draw[line width=1.2pt,color=ccqqqq] (1.6999999769554757,1.444999960824309) -- (1.7249999766165856,1.4878124596636104);
\draw[line width=1.2pt,color=ccqqqq] (1.7249999766165856,1.4878124596636104) -- (1.7499999762776954,1.5312499584859673);
\draw[line width=1.2pt,color=ccqqqq] (1.7499999762776954,1.5312499584859673) -- (1.7749999759388053,1.5753124572913797);
\draw[line width=1.2pt,color=ccqqqq] (1.7749999759388053,1.5753124572913797) -- (1.7999999755999152,1.6199999560798477);
\draw[line width=1.2pt,color=ccqqqq] (1.7999999755999152,1.6199999560798477) -- (1.824999975261025,1.6653124548513711);
\draw[line width=1.2pt,color=ccqqqq] (1.824999975261025,1.6653124548513711) -- (1.849999974922135,1.71124995360595);
\draw[line width=1.2pt,color=ccqqqq] (1.849999974922135,1.71124995360595) -- (1.8749999745832449,1.7578124523435845);
\draw[line width=1.2pt,color=ccqqqq] (1.8749999745832449,1.7578124523435845) -- (1.8999999742443547,1.8049999510642742);
\draw[line width=1.2pt,color=ccqqqq] (1.8999999742443547,1.8049999510642742) -- (1.9249999739054646,1.8528124497680198);
\draw[line width=1.2pt,color=ccqqqq] (1.9249999739054646,1.8528124497680198) -- (1.9499999735665745,1.9012499484548206);
\draw[line width=1.2pt,color=ccqqqq] (1.9499999735665745,1.9012499484548206) -- (1.9749999732276844,1.950312447124677);
\draw[line width=1.2pt,color=ccqqqq] (1.9749999732276844,1.950312447124677) -- (1.9999999728887943,1.999999945777589);
\draw[line width=1.2pt,color=ccqqqq] (1.9999999728887943,1.999999945777589) -- (2.0249999725499044,2.0503124444135565);
\draw[line width=1.2pt,color=ccqqqq] (2.0249999725499044,2.0503124444135565) -- (2.0499999722110145,2.10124994303258);
\draw[line width=1.2pt,color=ccqqqq] (2.0499999722110145,2.10124994303258) -- (2.0749999718721246,2.152812441634659);
\draw[line width=1.2pt,color=ccqqqq] (2.0749999718721246,2.152812441634659) -- (2.0999999715332347,2.204999940219793);
\draw[line width=1.2pt,color=ccqqqq] (2.0999999715332347,2.204999940219793) -- (2.124999971194345,2.257812438787983);
\draw[line width=1.2pt,color=ccqqqq] (2.124999971194345,2.257812438787983) -- (2.149999970855455,2.3112499373392286);
\draw[line width=1.2pt,color=ccqqqq] (2.149999970855455,2.3112499373392286) -- (2.174999970516565,2.365312435873529);
\draw[line width=1.2pt,color=ccqqqq] (2.174999970516565,2.365312435873529) -- (2.199999970177675,2.419999934390886);
\draw[line width=1.2pt,color=ccqqqq] (2.199999970177675,2.419999934390886) -- (2.2249999698387852,2.4753124328912977);
\draw[line width=1.2pt,color=ccqqqq] (2.2249999698387852,2.4753124328912977) -- (2.2499999694998953,2.531249931374765);
\draw[line width=1.2pt,color=ccqqqq] (2.2499999694998953,2.531249931374765) -- (2.2749999691610054,2.587812429841288);
\draw[line width=1.2pt,color=ccqqqq] (2.2749999691610054,2.587812429841288) -- (2.2999999688221155,2.6449999282908663);
\draw[line width=1.2pt,color=ccqqqq] (2.2999999688221155,2.6449999282908663) -- (2.3249999684832257,2.7028124267235003);
\draw[line width=1.2pt,color=ccqqqq] (2.3249999684832257,2.7028124267235003) -- (2.3499999681443358,2.7612499251391895);
\draw[line width=1.2pt,color=ccqqqq] (2.3499999681443358,2.7612499251391895) -- (2.374999967805446,2.8203124235379344);
\draw[line width=1.2pt,color=ccqqqq] (2.374999967805446,2.8203124235379344) -- (2.399999967466556,2.879999921919735);
\draw[line width=1.2pt,color=ccqqqq] (2.399999967466556,2.879999921919735) -- (2.424999967127666,2.9403124202845907);
\draw[line width=1.2pt,color=ccqqqq] (2.424999967127666,2.9403124202845907) -- (2.449999966788776,3.001249918632502);
\draw[line width=1.2pt,color=ccqqqq] (2.449999966788776,3.001249918632502) -- (2.4749999664498863,3.062812416963469);
\draw[line width=1.2pt,color=ccqqqq] (2.4749999664498863,3.062812416963469) -- (2.4999999661109964,3.1249999152774914);
\draw[line width=1.2pt,color=ccqqqq] (2.4999999661109964,3.1249999152774914) -- (2.5249999657721065,3.1878124135745693);
\draw[line width=1.2pt,color=ccqqqq] (2.5249999657721065,3.1878124135745693) -- (2.5499999654332166,3.251249911854703);
\draw[line width=1.2pt,color=ccqqqq] (2.5499999654332166,3.251249911854703) -- (2.5749999650943267,3.3153124101178917);
\draw[line width=1.2pt,color=ccqqqq] (2.5749999650943267,3.3153124101178917) -- (2.599999964755437,3.3799999083641366);
\draw[line width=1.2pt,color=ccqqqq] (2.599999964755437,3.3799999083641366) -- (2.624999964416547,3.445312406593436);
\draw[line width=1.2pt,color=ccqqqq] (2.624999964416547,3.445312406593436) -- (2.649999964077657,3.511249904805792);
\draw[line width=1.2pt,color=ccqqqq] (2.649999964077657,3.511249904805792) -- (2.674999963738767,3.577812403001203);
\draw[line width=1.2pt,color=ccqqqq] (2.674999963738767,3.577812403001203) -- (2.6999999633998772,3.6449999011796694);
\draw[line width=1.2pt,color=ccqqqq] (2.6999999633998772,3.6449999011796694) -- (2.7249999630609874,3.7128123993411912);
\draw[line width=1.2pt,color=ccqqqq] (2.7249999630609874,3.7128123993411912) -- (2.7499999627220975,3.7812498974857687);
\draw[line width=1.2pt,color=ccqqqq] (2.7499999627220975,3.7812498974857687) -- (2.7749999623832076,3.8503123956134018);
\draw[line width=1.2pt,color=ccqqqq] (2.7749999623832076,3.8503123956134018) -- (2.7999999620443177,3.91999989372409);
\draw[line width=1.2pt,color=ccqqqq] (2.7999999620443177,3.91999989372409) -- (2.8249999617054278,3.990312391817834);
\draw[line width=1.2pt,color=ccqqqq] (2.8249999617054278,3.990312391817834) -- (2.849999961366538,4.061249889894634);
\draw[line width=1.2pt,color=ccqqqq] (2.849999961366538,4.061249889894634) -- (2.874999961027648,4.1328123879544885);
\draw[line width=1.2pt,color=ccqqqq] (2.874999961027648,4.1328123879544885) -- (2.899999960688758,4.204999885997399);
\draw[line width=1.2pt,color=ccqqqq] (2.899999960688758,4.204999885997399) -- (2.924999960349868,4.277812384023365);
\draw[line width=1.2pt,color=ccqqqq] (2.924999960349868,4.277812384023365) -- (2.9499999600109783,4.351249882032387);
\draw[line width=1.2pt,color=ccqqqq] (2.9499999600109783,4.351249882032387) -- (2.9749999596720884,4.425312380024464);
\draw[line width=1.2pt,color=ccqqqq] (2.9749999596720884,4.425312380024464) -- (2.9999999593331985,4.499999877999596);
\draw[line width=1.2pt,color=ccqqqq] (2.9999999593331985,4.499999877999596) -- (3.0249999589943086,4.575312375957784);
\draw[line width=1.2pt,color=ccqqqq] (3.0249999589943086,4.575312375957784) -- (3.0499999586554187,4.651249873899028);
\draw[line width=1.2pt,color=ccqqqq] (3.0499999586554187,4.651249873899028) -- (3.074999958316529,4.727812371823327);
\draw[line width=1.2pt,color=ccqqqq] (3.074999958316529,4.727812371823327) -- (3.099999957977639,4.804999869730682);
\draw[line width=1.2pt,color=ccqqqq] (3.099999957977639,4.804999869730682) -- (3.124999957638749,4.882812367621091);
\draw[line width=1.2pt,color=ccqqqq] (3.124999957638749,4.882812367621091) -- (3.149999957299859,4.961249865494557);
\draw[line width=1.2pt,color=ccqqqq] (3.149999957299859,4.961249865494557) -- (3.1749999569609693,5.040312363351078);
\draw[line width=1.2pt,color=ccqqqq] (3.1749999569609693,5.040312363351078) -- (3.1999999566220794,5.119999861190655);
\draw[line width=1.2pt,color=ccqqqq] (3.1999999566220794,5.119999861190655) -- (3.2249999562831895,5.200312359013287);
\draw[line width=1.2pt,color=ccqqqq] (3.2249999562831895,5.200312359013287) -- (3.2499999559442996,5.281249856818975);
\draw[line width=1.2pt,color=ccqqqq] (3.2499999559442996,5.281249856818975) -- (3.2749999556054097,5.362812354607717);
\draw[line width=1.2pt,color=ccqqqq] (3.2749999556054097,5.362812354607717) -- (3.29999995526652,5.4449998523795164);
\draw[line width=1.2pt,color=ccqqqq] (3.29999995526652,5.4449998523795164) -- (3.32499995492763,5.52781235013437);
\draw[line width=1.2pt,color=ccqqqq] (3.32499995492763,5.52781235013437) -- (3.34999995458874,5.61124984787228);
\draw[line width=1.2pt,color=ccqqqq] (3.34999995458874,5.61124984787228) -- (3.37499995424985,5.695312345593245);
\draw[line width=1.2pt,color=ccqqqq] (3.37499995424985,5.695312345593245) -- (3.39999995391096,5.779999843297266);
\draw[line width=1.2pt,color=ccqqqq] (3.39999995391096,5.779999843297266) -- (3.4249999535720703,5.8653123409843415);
\draw[line width=1.2pt,color=ccqqqq] (3.4249999535720703,5.8653123409843415) -- (3.4499999532331804,5.951249838654474);
\draw[line width=1.2pt,color=ccqqqq] (3.4499999532331804,5.951249838654474) -- (3.4749999528942905,6.037812336307661);
\draw[line width=1.2pt,color=ccqqqq] (3.4749999528942905,6.037812336307661) -- (3.4999999525554006,6.124999833943903);
\draw[line width=1.2pt,color=ccqqqq] (3.4999999525554006,6.124999833943903) -- (3.5249999522165107,6.212812331563201);
\draw[line width=1.2pt,color=ccqqqq] (3.5249999522165107,6.212812331563201) -- (3.549999951877621,6.301249829165555);
\draw[line width=1.2pt,color=ccqqqq] (3.549999951877621,6.301249829165555) -- (3.574999951538731,6.390312326750965);
\draw[line width=1.2pt,color=ccqqqq] (3.574999951538731,6.390312326750965) -- (3.599999951199841,6.479999824319429);
\draw[line width=1.2pt,color=ccqqqq] (3.599999951199841,6.479999824319429) -- (3.624999950860951,6.570312321870949);
\draw[line width=1.2pt,color=ccqqqq] (3.624999950860951,6.570312321870949) -- (3.6499999505220613,6.6612498194055245);
\draw[line width=1.2pt,color=ccqqqq] (3.6499999505220613,6.6612498194055245) -- (3.6749999501831714,6.752812316923156);
\draw[line width=1.2pt,color=ccqqqq] (3.6749999501831714,6.752812316923156) -- (3.6999999498442815,6.844999814423843);
\draw[line width=1.2pt,color=ccqqqq] (3.6999999498442815,6.844999814423843) -- (3.7249999495053916,6.937812311907585);
\draw[line width=1.2pt,color=ccqqqq] (3.7249999495053916,6.937812311907585) -- (3.7499999491665017,7.0312498093743825);
\draw[line width=1.2pt,color=ccqqqq] (3.7499999491665017,7.0312498093743825) -- (3.774999948827612,7.125312306824235);
\draw[line width=1.2pt,color=ccqqqq] (3.774999948827612,7.125312306824235) -- (3.799999948488722,7.219999804257145);
\draw[line width=1.2pt,color=ccqqqq] (3.799999948488722,7.219999804257145) -- (3.824999948149832,7.315312301673109);
\draw[line width=1.2pt,color=ccqqqq] (3.824999948149832,7.315312301673109) -- (3.849999947810942,7.411249799072128);
\draw[line width=1.2pt,color=ccqqqq] (3.849999947810942,7.411249799072128) -- (3.8749999474720522,7.507812296454204);
\draw[line width=1.2pt,color=ccqqqq] (3.8749999474720522,7.507812296454204) -- (3.8999999471331623,7.604999793819334);
\draw[line width=1.2pt,color=ccqqqq] (3.8999999471331623,7.604999793819334) -- (3.9249999467942724,7.702812291167521);
\draw[line width=1.2pt,color=ccqqqq] (3.9249999467942724,7.702812291167521) -- (3.9499999464553825,7.801249788498763);
\draw[line width=1.2pt,color=ccqqqq] (3.9499999464553825,7.801249788498763) -- (3.9749999461164927,7.90031228581306);
\draw[line width=1.2pt,color=ccqqqq] (3.9749999461164927,7.90031228581306) -- (3.9999999457776028,7.999999783110413);
\draw[line width=1.2pt,color=ccqqqq] (3.9999999457776028,7.999999783110413) -- (4.024999945438712,8.100312280390819);
\draw[line width=1.2pt,color=ccqqqq] (4.024999945438712,8.100312280390819) -- (4.049999945099822,8.201249777654281);
\draw[line width=1.2pt,color=ccqqqq] (4.049999945099822,8.201249777654281) -- (4.074999944760932,8.302812274900798);
\draw[line width=1.2pt,color=ccqqqq] (4.074999944760932,8.302812274900798) -- (4.099999944422041,8.404999772130372);
\draw[line width=1.2pt,color=ccqqqq] (4.099999944422041,8.404999772130372) -- (4.124999944083151,8.507812269342999);
\draw[line width=1.2pt,color=ccqqqq] (4.124999944083151,8.507812269342999) -- (4.149999943744261,8.611249766538684);
\draw[line width=1.2pt,color=ccqqqq] (4.149999943744261,8.611249766538684) -- (4.17499994340537,8.715312263717424);
\draw[line width=1.2pt,color=ccqqqq] (4.17499994340537,8.715312263717424) -- (4.19999994306648,8.819999760879218);
\draw[line width=1.2pt,color=ccqqqq] (4.19999994306648,8.819999760879218) -- (4.22499994272759,8.925312258024068);
\draw[line width=1.2pt,color=ccqqqq] (4.22499994272759,8.925312258024068) -- (4.249999942388699,9.031249755151974);
\draw[line width=1.2pt,color=ccqqqq] (4.249999942388699,9.031249755151974) -- (4.274999942049809,9.137812252262936);
\draw[line width=1.2pt,color=ccqqqq] (4.274999942049809,9.137812252262936) -- (4.299999941710919,9.244999749356952);
\draw[line width=1.2pt,color=ccqqqq] (4.299999941710919,9.244999749356952) -- (4.324999941372028,9.352812246434024);
\draw[line width=1.2pt,color=ccqqqq] (4.324999941372028,9.352812246434024) -- (4.349999941033138,9.461249743494152);
\draw[line width=1.2pt,color=ccqqqq] (4.349999941033138,9.461249743494152) -- (4.374999940694248,9.570312240537335);
\draw[line width=1.2pt,color=ccqqqq] (4.374999940694248,9.570312240537335) -- (4.399999940355357,9.679999737563573);
\draw[line width=1.2pt,color=ccqqqq] (4.399999940355357,9.679999737563573) -- (4.424999940016467,9.790312234572868);
\draw[line width=1.2pt,color=ccqqqq] (4.424999940016467,9.790312234572868) -- (4.449999939677577,9.901249731565217);
\draw[line width=1.2pt,color=ccqqqq] (4.449999939677577,9.901249731565217) -- (4.474999939338686,10.012812228540623);
\draw[line width=1.2pt,color=ccqqqq] (4.474999939338686,10.012812228540623) -- (4.499999938999796,10.124999725499084);
\draw[line width=1.2pt,color=ccqqqq] (4.499999938999796,10.124999725499084) -- (4.524999938660906,10.2378122224406);
\draw[line width=1.2pt,color=ccqqqq] (4.524999938660906,10.2378122224406) -- (4.549999938322015,10.351249719365171);
\draw[line width=1.2pt,color=ccqqqq] (4.549999938322015,10.351249719365171) -- (4.574999937983125,10.4653122162728);
\draw[line width=1.2pt,color=ccqqqq] (4.574999937983125,10.4653122162728) -- (4.599999937644235,10.579999713163481);
\draw[line width=1.2pt,color=ccqqqq] (4.599999937644235,10.579999713163481) -- (4.624999937305344,10.695312210037219);
\draw[line width=1.2pt,color=ccqqqq] (4.624999937305344,10.695312210037219) -- (4.649999936966454,10.811249706894014);
\draw[line width=1.2pt,color=ccqqqq] (4.649999936966454,10.811249706894014) -- (4.674999936627564,10.927812203733861);
\draw[line width=1.2pt,color=ccqqqq] (4.674999936627564,10.927812203733861) -- (4.699999936288673,11.044999700556767);
\draw[line width=1.2pt,color=ccqqqq] (4.699999936288673,11.044999700556767) -- (4.724999935949783,11.162812197362726);
\draw[line width=1.2pt,color=ccqqqq] (4.724999935949783,11.162812197362726) -- (4.749999935610893,11.281249694151741);
\draw[line width=1.2pt,color=ccqqqq] (4.749999935610893,11.281249694151741) -- (4.774999935272002,11.400312190923813);
\draw[line width=1.2pt,color=ccqqqq] (4.774999935272002,11.400312190923813) -- (4.799999934933112,11.51999968767894);
\draw[line width=1.2pt,color=ccqqqq] (4.799999934933112,11.51999968767894) -- (4.824999934594222,11.640312184417121);
\draw[line width=1.2pt,color=ccqqqq] (4.824999934594222,11.640312184417121) -- (4.849999934255331,11.76124968113836);
\draw[line width=1.2pt,color=ccqqqq] (4.849999934255331,11.76124968113836) -- (4.874999933916441,11.882812177842652);
\draw[line width=1.2pt,color=ccqqqq] (4.874999933916441,11.882812177842652) -- (4.899999933577551,12.00499967453);
\draw[line width=1.2pt,color=ccqqqq] (4.899999933577551,12.00499967453) -- (4.92499993323866,12.127812171200404);
\draw[line width=1.2pt,color=ccqqqq] (4.92499993323866,12.127812171200404) -- (4.94999993289977,12.251249667853862);
\draw[line width=1.2pt,color=ccqqqq] (4.94999993289977,12.251249667853862) -- (4.97499993256088,12.375312164490378);
\draw[line width=1.2pt,color=ccqqqq] (4.97499993256088,12.375312164490378) -- (4.999999932221989,12.499999661109948);
\draw[line width=1.2pt,color=ccqqqq] (4.999999932221989,12.499999661109948) -- (5.024999931883099,12.625312157712575);
\draw[line width=1.2pt,color=ccqqqq] (5.024999931883099,12.625312157712575) -- (5.049999931544209,12.751249654298256);
\draw[line width=1.2pt,color=ccqqqq] (5.049999931544209,12.751249654298256) -- (5.074999931205318,12.877812150866992);
\draw[line width=1.2pt,color=ccqqqq] (5.074999931205318,12.877812150866992) -- (5.099999930866428,13.004999647418785);
\draw[line width=1.2pt,color=ccqqqq] (5.099999930866428,13.004999647418785) -- (5.1249999305275376,13.132812143953632);
\draw[line width=1.2pt,color=ccqqqq] (5.1249999305275376,13.132812143953632) -- (5.149999930188647,13.261249640471535);
\draw[line width=1.2pt,color=ccqqqq] (5.149999930188647,13.261249640471535) -- (5.174999929849757,13.390312136972494);
\draw[line width=1.2pt,color=ccqqqq] (5.174999929849757,13.390312136972494) -- (5.1999999295108665,13.519999633456509);
\draw[line width=1.2pt,color=ccqqqq] (5.1999999295108665,13.519999633456509) -- (5.224999929171976,13.650312129923579);
\draw[line width=1.2pt,color=ccqqqq] (5.224999929171976,13.650312129923579) -- (5.249999928833086,13.781249626373704);
\draw[line width=1.2pt,color=ccqqqq] (5.249999928833086,13.781249626373704) -- (5.2749999284941955,13.912812122806884);
\draw[line width=1.2pt,color=ccqqqq] (5.2749999284941955,13.912812122806884) -- (5.299999928155305,14.04499961922312);
\draw[line width=1.2pt,color=ccqqqq] (5.299999928155305,14.04499961922312) -- (5.324999927816415,14.177812115622412);
\draw[line width=1.2pt,color=ccqqqq] (5.324999927816415,14.177812115622412) -- (5.3499999274775245,14.311249612004758);
\draw[line width=1.2pt,color=ccqqqq] (5.3499999274775245,14.311249612004758) -- (5.374999927138634,14.445312108370162);
\draw[line width=1.2pt,color=ccqqqq] (5.374999927138634,14.445312108370162) -- (5.399999926799744,14.579999604718619);
\draw[line width=1.2pt,color=ccqqqq] (5.399999926799744,14.579999604718619) -- (5.4249999264608535,14.715312101050133);
\draw[line width=1.2pt,color=ccqqqq] (5.4249999264608535,14.715312101050133) -- (5.449999926121963,14.851249597364703);
\draw[line width=1.2pt,color=ccqqqq] (5.449999926121963,14.851249597364703) -- (5.474999925783073,14.987812093662326);
\draw[line width=1.2pt,color=ccqqqq] (5.474999925783073,14.987812093662326) -- (5.4999999254441825,15.124999589943007);
\draw[line width=1.2pt,color=ccqqqq] (5.4999999254441825,15.124999589943007) -- (5.524999925105292,15.262812086206742);
\draw[line width=1.2pt,color=ccqqqq] (5.524999925105292,15.262812086206742) -- (5.549999924766402,15.401249582453532);
\draw[line width=1.2pt,color=ccqqqq] (5.549999924766402,15.401249582453532) -- (5.5749999244275115,15.54031207868338);
\draw[line width=1.2pt,color=ccqqqq] (5.5749999244275115,15.54031207868338) -- (5.599999924088621,15.67999957489628);
\draw[line width=1.2pt,color=ccqqqq] (5.599999924088621,15.67999957489628) -- (5.624999923749731,15.82031207109224);
\draw[line width=1.2pt,color=ccqqqq] (5.624999923749731,15.82031207109224) -- (5.6499999234108405,15.96124956727125);
\draw[line width=1.2pt,color=ccqqqq] (5.6499999234108405,15.96124956727125) -- (5.67499992307195,16.10281206343332);
\draw[line width=1.2pt,color=ccqqqq] (5.67499992307195,16.10281206343332) -- (5.69999992273306,16.244999559578442);
\draw[line width=1.2pt,color=ccqqqq] (5.69999992273306,16.244999559578442) -- (5.724999922394169,16.387812055706622);
\draw[line width=1.2pt,color=ccqqqq] (5.724999922394169,16.387812055706622) -- (5.749999922055279,16.531249551817858);
\draw[line width=1.2pt,color=ccqqqq] (5.749999922055279,16.531249551817858) -- (5.774999921716389,16.675312047912147);
\draw[line width=1.2pt,color=ccqqqq] (5.774999921716389,16.675312047912147) -- (5.799999921377498,16.819999543989493);
\draw[line width=1.2pt,color=ccqqqq] (5.799999921377498,16.819999543989493) -- (5.824999921038608,16.965312040049895);
\draw[line width=1.2pt,color=ccqqqq] (5.824999921038608,16.965312040049895) -- (5.849999920699718,17.111249536093354);
\draw[line width=1.2pt,color=ccqqqq] (5.849999920699718,17.111249536093354) -- (5.874999920360827,17.257812032119865);
\draw[line width=1.2pt,color=ccqqqq] (5.874999920360827,17.257812032119865) -- (5.899999920021937,17.404999528129434);
\draw[line width=1.2pt,color=ccqqqq] (5.899999920021937,17.404999528129434) -- (5.924999919683047,17.552812024122055);
\draw[line width=1.2pt,color=ccqqqq] (5.924999919683047,17.552812024122055) -- (5.949999919344156,17.701249520097733);
\draw[line width=1.2pt,color=ccqqqq] (5.949999919344156,17.701249520097733) -- (5.974999919005266,17.850312016056467);
\draw[line width=1.2pt,color=ccqqqq] (5.974999919005266,17.850312016056467) -- (5.999999918666376,17.999999511998258);
\draw[line width=1.2pt,color=ccqqqq] (5.999999918666376,17.999999511998258) -- (6.024999918327485,18.1503120079231);
\draw[line width=1.2pt,color=ccqqqq] (6.024999918327485,18.1503120079231) -- (6.049999917988595,18.301249503831002);
\draw[line width=1.2pt,color=ccqqqq] (6.049999917988595,18.301249503831002) -- (6.074999917649705,18.45281199972196);
\draw[line width=1.2pt,color=ccqqqq] (6.074999917649705,18.45281199972196) -- (6.099999917310814,18.604999495595973);
\draw[line width=1.2pt,color=ccqqqq] (6.099999917310814,18.604999495595973) -- (6.124999916971924,18.75781199145304);
\draw[line width=1.2pt,color=ccqqqq] (6.124999916971924,18.75781199145304) -- (6.149999916633034,18.911249487293162);
\draw[line width=1.2pt,color=ccqqqq] (6.149999916633034,18.911249487293162) -- (6.174999916294143,19.065311983116338);
\draw[line width=1.2pt,color=ccqqqq] (6.174999916294143,19.065311983116338) -- (6.199999915955253,19.21999947892257);
\draw[line width=1.2pt,color=ccqqqq] (6.199999915955253,19.21999947892257) -- (6.224999915616363,19.37531197471186);
\draw[line width=1.2pt,color=ccqqqq] (6.224999915616363,19.37531197471186) -- (6.249999915277472,19.531249470484205);
\draw[line width=1.2pt,color=ccqqqq] (6.249999915277472,19.531249470484205) -- (6.274999914938582,19.687811966239607);
\draw[line width=1.2pt,color=ccqqqq] (6.274999914938582,19.687811966239607) -- (6.299999914599692,19.844999461978063);
\draw[line width=1.2pt,color=ccqqqq] (6.299999914599692,19.844999461978063) -- (6.324999914260801,20.00281195769957);
\draw[line width=1.2pt,color=ccqqqq] (6.324999914260801,20.00281195769957) -- (6.349999913921911,20.16124945340414);
\draw[line width=1.2pt,color=ccqqqq] (6.349999913921911,20.16124945340414) -- (6.374999913583021,20.32031194909176);
\draw[line width=1.2pt,color=ccqqqq] (6.374999913583021,20.32031194909176) -- (6.39999991324413,20.479999444762438);
\draw[line width=1.2pt,color=ccqqqq] (6.39999991324413,20.479999444762438) -- (6.42499991290524,20.640311940416172);
\draw[line width=1.2pt,color=ccqqqq] (6.42499991290524,20.640311940416172) -- (6.44999991256635,20.80124943605296);
\draw[line width=1.2pt,color=ccqqqq] (6.44999991256635,20.80124943605296) -- (6.474999912227459,20.962811931672803);
\draw[line width=1.2pt,color=ccqqqq] (6.474999912227459,20.962811931672803) -- (6.499999911888569,21.124999427275704);
\draw[line width=1.2pt,color=ccqqqq] (6.499999911888569,21.124999427275704) -- (6.524999911549679,21.287811922861657);
\draw[line width=1.2pt,color=ccqqqq] (6.524999911549679,21.287811922861657) -- (6.549999911210788,21.451249418430667);
\draw[line width=1.2pt,color=ccqqqq] (6.549999911210788,21.451249418430667) -- (6.574999910871898,21.615311913982733);
\draw[line width=1.2pt,color=ccqqqq] (6.574999910871898,21.615311913982733) -- (6.599999910533008,21.779999409517853);
\draw[line width=1.2pt,color=ccqqqq] (6.599999910533008,21.779999409517853) -- (6.624999910194117,21.945311905036032);
\draw[line width=1.2pt,color=ccqqqq] (6.624999910194117,21.945311905036032) -- (6.649999909855227,22.111249400537265);
\draw[line width=1.2pt,color=ccqqqq] (6.649999909855227,22.111249400537265) -- (6.674999909516337,22.27781189602155);
\draw[line width=1.2pt,color=ccqqqq] (6.674999909516337,22.27781189602155) -- (6.699999909177446,22.444999391488896);
\draw[line width=1.2pt,color=ccqqqq] (6.699999909177446,22.444999391488896) -- (6.724999908838556,22.612811886939294);
\draw[line width=1.2pt,color=ccqqqq] (6.724999908838556,22.612811886939294) -- (6.749999908499666,22.781249382372746);
\draw[line width=1.2pt,color=ccqqqq] (6.749999908499666,22.781249382372746) -- (6.774999908160775,22.950311877789257);
\draw[line width=1.2pt,color=ccqqqq] (6.774999908160775,22.950311877789257) -- (6.799999907821885,23.119999373188822);
\draw[line width=1.2pt,color=ccqqqq] (6.799999907821885,23.119999373188822) -- (6.824999907482995,23.290311868571443);
\draw[line width=1.2pt,color=ccqqqq] (6.824999907482995,23.290311868571443) -- (6.849999907144104,23.461249363937117);
\draw[line width=1.2pt,color=ccqqqq] (6.849999907144104,23.461249363937117) -- (6.874999906805214,23.63281185928585);
\draw[line width=1.2pt,color=ccqqqq] (6.874999906805214,23.63281185928585) -- (6.8999999064663236,23.804999354617635);
\draw[line width=1.2pt,color=ccqqqq] (6.8999999064663236,23.804999354617635) -- (6.924999906127433,23.97781184993248);
\draw[line width=1.2pt,color=ccqqqq] (6.924999906127433,23.97781184993248) -- (6.949999905788543,24.151249345230376);
\draw[line width=1.2pt,color=ccqqqq] (6.949999905788543,24.151249345230376) -- (6.9749999054496525,24.325311840511333);
\draw[line width=1.2pt,color=ccqqqq] (6.9749999054496525,24.325311840511333) -- (6.999999905110762,24.49999933577534);
\draw[line width=1.2pt,color=ccqqqq] (6.999999905110762,24.49999933577534) -- (7.024999904771872,24.675311831022405);
\draw[line width=1.2pt,color=ccqqqq] (7.024999904771872,24.675311831022405) -- (7.0499999044329815,24.851249326252525);
\draw[line width=1.2pt,color=ccqqqq] (7.0499999044329815,24.851249326252525) -- (7.074999904094091,25.0278118214657);
\draw[line width=1.2pt,color=ccqqqq] (7.074999904094091,25.0278118214657) -- (7.099999903755201,25.20499931666193);
\draw[line width=1.2pt,color=ccqqqq] (7.099999903755201,25.20499931666193) -- (7.1249999034163105,25.38281181184122);
\draw[line width=1.2pt,color=ccqqqq] (7.1249999034163105,25.38281181184122) -- (7.14999990307742,25.56124930700356);
\draw[line width=1.2pt,color=ccqqqq] (7.14999990307742,25.56124930700356) -- (7.17499990273853,25.740311802148955);
\draw[line width=1.2pt,color=ccqqqq] (7.17499990273853,25.740311802148955) -- (7.1999999023996395,25.91999929727741);
\draw[line width=1.2pt,color=ccqqqq] (7.1999999023996395,25.91999929727741) -- (7.224999902060749,26.10031179238892);
\draw[line width=1.2pt,color=ccqqqq] (7.224999902060749,26.10031179238892) -- (7.249999901721859,26.28124928748348);
\draw[line width=1.2pt,color=ccqqqq] (7.249999901721859,26.28124928748348) -- (7.2749999013829685,26.4628117825611);
\draw[line width=1.2pt,color=ccqqqq] (7.2749999013829685,26.4628117825611) -- (7.299999901044078,26.644999277621775);
\draw[line width=1.2pt,color=ccqqqq] (7.299999901044078,26.644999277621775) -- (7.324999900705188,26.827811772665505);
\draw[line width=1.2pt,color=ccqqqq] (7.324999900705188,26.827811772665505) -- (7.3499999003662975,27.011249267692293);
\draw[line width=1.2pt,color=ccqqqq] (7.3499999003662975,27.011249267692293) -- (7.374999900027407,27.195311762702133);
\draw[line width=1.2pt,color=ccqqqq] (7.374999900027407,27.195311762702133) -- (7.399999899688517,27.37999925769503);
\draw[line width=1.2pt,color=ccqqqq] (7.399999899688517,27.37999925769503) -- (7.4249998993496265,27.565311752670983);
\draw[line width=1.2pt,color=ccqqqq] (7.4249998993496265,27.565311752670983) -- (7.449999899010736,27.75124924762999);
\draw[line width=1.2pt,color=ccqqqq] (7.449999899010736,27.75124924762999) -- (7.474999898671846,27.937811742572052);
\draw[line width=1.2pt,color=ccqqqq] (7.474999898671846,27.937811742572052) -- (7.499999898332955,28.12499923749717);
\draw[line width=1.2pt,color=ccqqqq] (7.499999898332955,28.12499923749717) -- (7.524999897994065,28.312811732405343);
\draw[line width=1.2pt,color=ccqqqq] (7.524999897994065,28.312811732405343) -- (7.549999897655175,28.501249227296576);
\draw[line width=1.2pt,color=ccqqqq] (7.549999897655175,28.501249227296576) -- (7.574999897316284,28.69031172217086);
\draw[line width=1.2pt,color=ccqqqq] (7.574999897316284,28.69031172217086) -- (7.599999896977394,28.8799992170282);
\draw[line width=1.2pt,color=ccqqqq] (7.599999896977394,28.8799992170282) -- (7.624999896638504,29.070311711868598);
\draw[line width=1.2pt,color=ccqqqq] (7.624999896638504,29.070311711868598) -- (7.649999896299613,29.26124920669205);
\draw[line width=1.2pt,color=ccqqqq] (7.649999896299613,29.26124920669205) -- (7.674999895960723,29.452811701498554);
\draw[line width=1.2pt,color=ccqqqq] (7.674999895960723,29.452811701498554) -- (7.699999895621833,29.64499919628812);
\draw[line width=1.2pt,color=ccqqqq] (7.699999895621833,29.64499919628812) -- (7.724999895282942,29.837811691060736);
\draw[line width=1.2pt,color=ccqqqq] (7.724999895282942,29.837811691060736) -- (7.749999894944052,30.03124918581641);
\draw[line width=1.2pt,color=ccqqqq] (7.749999894944052,30.03124918581641) -- (7.774999894605162,30.225311680555137);
\draw[line width=1.2pt,color=ccqqqq] (7.774999894605162,30.225311680555137) -- (7.799999894266271,30.41999917527692);
\draw[line width=1.2pt,color=ccqqqq] (7.799999894266271,30.41999917527692) -- (7.824999893927381,30.61531166998176);
\draw[line width=1.2pt,color=ccqqqq] (7.824999893927381,30.61531166998176) -- (7.849999893588491,30.811249164669658);
\draw[line width=1.2pt,color=ccqqqq] (7.849999893588491,30.811249164669658) -- (7.8749998932496,31.007811659340607);
\draw[line width=1.2pt,color=ccqqqq] (7.8749998932496,31.007811659340607) -- (7.89999989291071,31.204999153994613);
\draw[line width=1.2pt,color=ccqqqq] (7.89999989291071,31.204999153994613) -- (7.92499989257182,31.402811648631676);
\draw[line width=1.2pt,color=ccqqqq] (7.92499989257182,31.402811648631676) -- (7.949999892232929,31.601249143251795);
\draw[line width=1.2pt,color=ccqqqq] (7.949999892232929,31.601249143251795) -- (7.974999891894039,31.800311637854968);
\draw[line width=1.2pt,color=ccqqqq] (7.974999891894039,31.800311637854968) -- (7.999999891555149,31.999999132441197);
\draw[line width=1.2pt,color=ccqqqq] (7.999999891555149,31.999999132441197) -- (8.02499989121626,32.200311627010485);
\draw[line width=1.2pt,color=ccqqqq] (8.02499989121626,32.200311627010485) -- (8.04999989087737,32.401249121562834);
\draw[line width=1.2pt,color=ccqqqq] (8.04999989087737,32.401249121562834) -- (8.07499989053848,32.602811616098236);
\draw[line width=1.2pt,color=ccqqqq] (8.07499989053848,32.602811616098236) -- (8.09999989019959,32.80499911061669);
\draw[line width=1.2pt,color=ccqqqq] (8.09999989019959,32.80499911061669) -- (8.124999889860701,33.007811605118206);
\draw[line width=1.2pt,color=ccqqqq] (8.124999889860701,33.007811605118206) -- (8.149999889521812,33.211249099602774);
\draw[line width=1.2pt,color=ccqqqq] (8.149999889521812,33.211249099602774) -- (8.174999889182923,33.415311594070396);
\draw[line width=1.2pt,color=ccqqqq] (8.174999889182923,33.415311594070396) -- (8.199999888844033,33.61999908852108);
\draw[line width=1.2pt,color=ccqqqq] (8.199999888844033,33.61999908852108) -- (8.224999888505144,33.82531158295481);
\draw[line width=1.2pt,color=ccqqqq] (8.224999888505144,33.82531158295481) -- (8.249999888166254,34.031249077371605);
\draw[line width=1.2pt,color=ccqqqq] (8.249999888166254,34.031249077371605) -- (8.274999887827365,34.23781157177145);
\draw[line width=1.2pt,color=ccqqqq] (8.274999887827365,34.23781157177145) -- (8.299999887488475,34.44499906615435);
\draw[line width=1.2pt,color=ccqqqq] (8.299999887488475,34.44499906615435) -- (8.324999887149586,34.652811560520306);
\draw[line width=1.2pt,color=ccqqqq] (8.324999887149586,34.652811560520306) -- (8.349999886810696,34.86124905486932);
\draw[line width=1.2pt,color=ccqqqq] (8.349999886810696,34.86124905486932) -- (8.374999886471807,35.07031154920139);
\draw[line width=1.2pt,color=ccqqqq] (8.374999886471807,35.07031154920139) -- (8.399999886132917,35.27999904351651);
\draw[line width=1.2pt,color=ccqqqq] (8.399999886132917,35.27999904351651) -- (8.424999885794028,35.49031153781469);
\draw[line width=1.2pt,color=ccqqqq] (8.424999885794028,35.49031153781469) -- (8.449999885455139,35.70124903209593);
\draw[line width=1.2pt,color=ccqqqq] (8.449999885455139,35.70124903209593) -- (8.47499988511625,35.912811526360215);
\draw[line width=1.2pt,color=ccqqqq] (8.47499988511625,35.912811526360215) -- (8.49999988477736,36.12499902060756);
\draw[line width=1.2pt,color=ccqqqq] (8.49999988477736,36.12499902060756) -- (8.52499988443847,36.337811514837966);
\draw[line width=1.2pt,color=ccqqqq] (8.52499988443847,36.337811514837966) -- (8.54999988409958,36.551249009051425);
\draw[line width=1.2pt,color=ccqqqq] (8.54999988409958,36.551249009051425) -- (8.574999883760691,36.765311503247936);
\draw[line width=1.2pt,color=ccqqqq] (8.574999883760691,36.765311503247936) -- (8.599999883421802,36.9799989974275);
\draw[line width=1.2pt,color=ccqqqq] (8.599999883421802,36.9799989974275) -- (8.624999883082912,37.195311491590125);
\draw[line width=1.2pt,color=ccqqqq] (8.624999883082912,37.195311491590125) -- (8.649999882744023,37.4112489857358);
\draw[line width=1.2pt,color=ccqqqq] (8.649999882744023,37.4112489857358) -- (8.674999882405134,37.62781147986454);
\draw[line width=1.2pt,color=ccqqqq] (8.674999882405134,37.62781147986454) -- (8.699999882066244,37.84499897397633);
\draw[line width=1.2pt,color=ccqqqq] (8.699999882066244,37.84499897397633) -- (8.724999881727355,38.062811468071175);
\draw[line width=1.2pt,color=ccqqqq] (8.724999881727355,38.062811468071175) -- (8.749999881388465,38.28124896214908);
\draw[line width=1.2pt,color=ccqqqq] (8.749999881388465,38.28124896214908) -- (8.774999881049576,38.500311456210035);
\draw[line width=1.2pt,color=ccqqqq] (8.774999881049576,38.500311456210035) -- (8.799999880710686,38.719998950254045);
\draw[line width=1.2pt,color=ccqqqq] (8.799999880710686,38.719998950254045) -- (8.824999880371797,38.940311444281114);
\draw[line width=1.2pt,color=ccqqqq] (8.824999880371797,38.940311444281114) -- (8.849999880032907,39.16124893829124);
\draw[line width=1.2pt,color=ccqqqq] (8.849999880032907,39.16124893829124) -- (8.874999879694018,39.38281143228441);
\draw[line width=1.2pt,color=ccqqqq] (8.874999879694018,39.38281143228441) -- (8.899999879355128,39.60499892626065);
\draw[line width=1.2pt,color=ccqqqq] (8.899999879355128,39.60499892626065) -- (8.924999879016239,39.82781142021994);
\draw[line width=1.2pt,color=ccqqqq] (8.924999879016239,39.82781142021994) -- (8.94999987867735,40.05124891416229);
\draw[line width=1.2pt,color=ccqqqq] (8.94999987867735,40.05124891416229) -- (8.97499987833846,40.27531140808769);
\draw[line width=1.2pt,color=ccqqqq] (8.97499987833846,40.27531140808769) -- (8.99999987799957,40.49999890199614);
\draw[line width=1.2pt,color=ccqqqq] (8.99999987799957,40.49999890199614) -- (9.024999877660681,40.72531139588766);
\draw[line width=1.2pt,color=ccqqqq] (9.024999877660681,40.72531139588766) -- (9.049999877321792,40.951248889762226);
\draw[line width=1.2pt,color=ccqqqq] (9.049999877321792,40.951248889762226) -- (9.074999876982902,41.17781138361985);
\draw[line width=1.2pt,color=ccqqqq] (9.074999876982902,41.17781138361985) -- (9.099999876644013,41.40499887746053);
\draw[line width=1.2pt,color=ccqqqq] (9.099999876644013,41.40499887746053) -- (9.124999876305123,41.63281137128426);
\draw[line width=1.2pt,color=ccqqqq] (9.124999876305123,41.63281137128426) -- (9.149999875966234,41.86124886509105);
\draw[line width=1.2pt,color=ccqqqq] (9.149999875966234,41.86124886509105) -- (9.174999875627345,42.090311358880896);
\draw[line width=1.2pt,color=ccqqqq] (9.174999875627345,42.090311358880896) -- (9.199999875288455,42.3199988526538);
\draw[line width=1.2pt,color=ccqqqq] (9.199999875288455,42.3199988526538) -- (9.224999874949566,42.55031134640975);
\draw[line width=1.2pt,color=ccqqqq] (9.224999874949566,42.55031134640975) -- (9.249999874610676,42.78124884014876);
\draw[line width=1.2pt,color=ccqqqq] (9.249999874610676,42.78124884014876) -- (9.274999874271787,43.01281133387083);
\draw[line width=1.2pt,color=ccqqqq] (9.274999874271787,43.01281133387083) -- (9.299999873932897,43.244998827575955);
\draw[line width=1.2pt,color=ccqqqq] (9.299999873932897,43.244998827575955) -- (9.324999873594008,43.477811321264134);
\draw[line width=1.2pt,color=ccqqqq] (9.324999873594008,43.477811321264134) -- (9.349999873255118,43.71124881493537);
\draw[line width=1.2pt,color=ccqqqq] (9.349999873255118,43.71124881493537) -- (9.374999872916229,43.94531130858965);
\draw[line width=1.2pt,color=ccqqqq] (9.374999872916229,43.94531130858965) -- (9.39999987257734,44.179998802227);
\draw[line width=1.2pt,color=ccqqqq] (9.39999987257734,44.179998802227) -- (9.42499987223845,44.4153112958474);
\draw[line width=1.2pt,color=ccqqqq] (9.42499987223845,44.4153112958474) -- (9.44999987189956,44.651248789450854);
\draw[line width=1.2pt,color=ccqqqq] (9.44999987189956,44.651248789450854) -- (9.474999871560671,44.887811283037365);
\draw[line width=1.2pt,color=ccqqqq] (9.474999871560671,44.887811283037365) -- (9.499999871221782,45.12499877660694);
\draw[line width=1.2pt,color=ccqqqq] (9.499999871221782,45.12499877660694) -- (9.524999870882892,45.362811270159554);
\draw[line width=1.2pt,color=ccqqqq] (9.524999870882892,45.362811270159554) -- (9.549999870544003,45.60124876369523);
\draw[line width=1.2pt,color=ccqqqq] (9.549999870544003,45.60124876369523) -- (9.574999870205113,45.84031125721397);
\draw[line width=1.2pt,color=ccqqqq] (9.574999870205113,45.84031125721397) -- (9.599999869866224,46.07999875071576);
\draw[line width=1.2pt,color=ccqqqq] (9.599999869866224,46.07999875071576) -- (9.624999869527334,46.3203112442006);
\draw[line width=1.2pt,color=ccqqqq] (9.624999869527334,46.3203112442006) -- (9.649999869188445,46.5612487376685);
\draw[line width=1.2pt,color=ccqqqq] (9.649999869188445,46.5612487376685) -- (9.674999868849556,46.802811231119456);
\draw[line width=1.2pt,color=ccqqqq] (9.674999868849556,46.802811231119456) -- (9.699999868510666,47.04499872455347);
\draw[line width=1.2pt,color=ccqqqq] (9.699999868510666,47.04499872455347) -- (9.724999868171777,47.287811217970535);
\draw[line width=1.2pt,color=ccqqqq] (9.724999868171777,47.287811217970535) -- (9.749999867832887,47.53124871137066);
\draw[line width=1.2pt,color=ccqqqq] (9.749999867832887,47.53124871137066) -- (9.774999867493998,47.775311204753834);
\draw[line width=1.2pt,color=ccqqqq] (9.774999867493998,47.775311204753834) -- (9.799999867155108,48.01999869812007);
\draw[line width=1.2pt,color=ccqqqq] (9.799999867155108,48.01999869812007) -- (9.824999866816219,48.26531119146936);
\draw[line width=1.2pt,color=ccqqqq] (9.824999866816219,48.26531119146936) -- (9.84999986647733,48.5112486848017);
\draw[line width=1.2pt,color=ccqqqq] (9.84999986647733,48.5112486848017) -- (9.87499986613844,48.7578111781171);
\draw[line width=1.2pt,color=ccqqqq] (9.87499986613844,48.7578111781171) -- (9.89999986579955,49.00499867141556);
\draw[line width=1.2pt,color=ccqqqq] (9.89999986579955,49.00499867141556) -- (9.924999865460661,49.25281116469707);
\draw[line width=1.2pt,color=ccqqqq] (9.924999865460661,49.25281116469707) -- (9.949999865121772,49.50124865796164);
\draw[line width=1.2pt,color=ccqqqq] (9.949999865121772,49.50124865796164) -- (9.974999864782882,49.75031115120926);
\draw[line width=1.2pt,color=ccqqqq] (9.974999864782882,49.75031115120926) -- (9.999999864443993,49.999998644439934);
\draw (0.12,52.13004484304933) node[anchor=north west] {Tailles des pousses (mm)};
\draw (7.56,2.9147982062780033) node[anchor=north west] {Temps (jour)};
\end{tikzpicture}
\end{center}
\end{minipage}


\end{ExoCu}


%%%%%%%%%%%%%%%%%%%%%%%%%%%
\begin{ExoCu}{Raisonner.}{1234}{2}{0}{0}{0}{0}

 
Ce graphique représente le niveau sonore en fonction de la distance à laquelle se trouve une personne de la source émettrice du son. Le niveau sonore est exprimé en (dB) et la distance en mètre.

\begin{minipage}{7cm}
\begin{enumerate}
\item Complète le graphique en notant sur chaque axe la légende.
\item Comment se nomme la variable ? Quelle est son unité ?
\item Quel est le niveau sonore pour une personne situés à 300 mètres de la source sonore ?
\item Quel est le niveau sonore pour une personne situés à 600 mètres de la source sonore ?
\item Une personne perçoit un son à un niveau de  40 dB. A Quelle distance de la source se trouve-t-il ?

\end{enumerate}
\end{minipage}
\begin{minipage}{9cm}
\begin{center}
\definecolor{qqqqff}{rgb}{0.,0.,1.}
\definecolor{cqcqcq}{rgb}{0.7529411764705882,0.7529411764705882,0.7529411764705882}
\begin{tikzpicture}[line cap=round,line join=round,>=triangle 45,x=0.007174400979366021cm,y=0.07955814622432517cm]
\draw [color=cqcqcq,, xstep=0.717440097936602cm,ystep=1.5911629244865033cm] (-57.6948075297355,-11.76791192653425) grid (1336.1497836438775,113.92631895465824);
\draw[->,color=black] (-57.6948075297355,0.) -- (1336.1497836438775,0.);
\foreach \x in {,100.,200.,300.,400.,500.,600.,700.,800.,900.,1000.,1100.,1200.,1300.}
\draw[shift={(\x,0)},color=black] (0pt,2pt) -- (0pt,-2pt) node[below] {\footnotesize $\x$};
\draw[->,color=black] (0.,-11.76791192653425) -- (0.,113.92631895465824);
\foreach \y in {,20.,40.,60.,80.,100.}
\draw[shift={(0,\y)},color=black] (2pt,0pt) -- (-2pt,0pt) node[left] {\footnotesize $\y$};
\draw[color=black] (0pt,-10pt) node[right] {\footnotesize $0$};
\clip(-57.6948075297355,-11.76791192653425) rectangle (1336.1497836438775,113.92631895465824);
\draw [color=qqqqff] (59.68157909541087,107.80845815955595)-- (155.86500702435026,65.8176863386266);
\draw [color=qqqqff] (155.86500702435026,65.8176863386266)-- (351.49231806626085,31.335198220777333);
\draw [color=qqqqff] (351.49231806626085,31.335198220777333)-- (600.,20.);
\draw [color=qqqqff] (600.,20.)-- (1324.7381904997662,13.537784998661582);
\end{tikzpicture}
\end{center}
\end{minipage}


 

\end{ExoCu}

%%%%%%%%%%%%%%%%%%%%%%%%%%%
\begin{ExoCu}{Représenter.}{1234}{2}{0}{0}{0}{0}
  
  
\begin{minipage}{0.49\linewidth}

On propose la figure suivante.

\definecolor{qqqqff}{rgb}{0.,0.,1.}
\definecolor{ffffff}{rgb}{1.,1.,1.}
\definecolor{zzttqq}{rgb}{0.6,0.2,0.}
\begin{tikzpicture}[line cap=round,line join=round,>=triangle 45,x=1.0cm,y=1.0cm]
\clip(3.4041911842591923,2.866575798081113) rectangle (12.252666255261191,9.209615907843705);
\fill[color=zzttqq,fill=zzttqq,fill opacity=0.25] (5.,8.) -- (5.,4.) -- (10.,4.) -- (10.,8.) -- cycle;
\draw [color=zzttqq] (5.,8.)-- (5.,4.);
\draw [color=zzttqq] (5.,4.)-- (10.,4.);
\draw [shift={(10.,5.5)},color=zzttqq,fill=zzttqq,fill opacity=0.25]  plot[domain=-1.5707963267948966:1.5707963267948966,variable=\t]({1.*1.5*cos(\t r)+0.*1.5*sin(\t r)},{0.*1.5*cos(\t r)+1.*1.5*sin(\t r)});
\draw [shift={(7.498580889309367,8.)},color=zzttqq]  plot[domain=3.141592653589793:6.283185307179586,variable=\t]({1.*1.498580889309367*cos(\t r)+0.*1.498580889309367*sin(\t r)},{0.*1.498580889309367*cos(\t r)+1.*1.498580889309367*sin(\t r)});
\draw [shift={(7.5,8.)},color=ffffff,fill=ffffff,fill opacity=1.0]  plot[domain=3.141592653589793:6.283185307179586,variable=\t]({1.*1.5*cos(\t r)+0.*1.5*sin(\t r)},{0.*1.5*cos(\t r)+1.*1.5*sin(\t r)});
\draw (5.0005655527389345,8.376347904148352)-- (5.958390173826779,8.376347904148352);
\draw (4.3012015436906665,7.9852221064954305)-- (4.3164051091047595,4.0059422521135355);
\draw (11.,8.)-- (11.,7.);
\draw (8.983899691231244,8.34233696522201)-- (9.972131443147275,8.32533149575884);
\draw (5.0157691181530275,3.5127836376815917)-- (9.987335008561367,3.495778168218421);
\draw (5.350247557263068,9.02255574374883) node[anchor=north west] {$x$};
\draw (9.34878526116947,8.971539335359319) node[anchor=north west] {$x$};
\draw (11.158009545446511,7.866183820253237) node[anchor=north west] {$x$};
\draw (3.784280319611512,6.59077361051545) node[anchor=north west] {$5$};
\draw (7.2050825377823875,3.5808055155342737) node[anchor=north west] {$6$};
\begin{scriptsize}
\draw [fill=qqqqff,shift={(5.0005655527389345,8.376347904148352)},rotate=90] (0,0) ++(0 pt,2.25pt) -- ++(1.9485571585149868pt,-3.375pt)--++(-3.8971143170299736pt,0 pt) -- ++(1.9485571585149868pt,3.375pt);
\draw [fill=qqqqff,shift={(5.958390173826779,8.376347904148352)},rotate=270] (0,0) ++(0 pt,2.25pt) -- ++(1.9485571585149868pt,-3.375pt)--++(-3.8971143170299736pt,0 pt) -- ++(1.9485571585149868pt,3.375pt);
\draw [fill=qqqqff,shift={(4.3012015436906665,7.9852221064954305)}] (0,0) ++(0 pt,2.25pt) -- ++(1.9485571585149868pt,-3.375pt)--++(-3.8971143170299736pt,0 pt) -- ++(1.9485571585149868pt,3.375pt);
\draw [fill=qqqqff,shift={(4.3164051091047595,4.0059422521135355)},rotate=180] (0,0) ++(0 pt,2.25pt) -- ++(1.9485571585149868pt,-3.375pt)--++(-3.8971143170299736pt,0 pt) -- ++(1.9485571585149868pt,3.375pt);
\draw [fill=qqqqff,shift={(11.,8.)}] (0,0) ++(0 pt,2.25pt) -- ++(1.9485571585149868pt,-3.375pt)--++(-3.8971143170299736pt,0 pt) -- ++(1.9485571585149868pt,3.375pt);
\draw [fill=qqqqff,shift={(11.,7.)},rotate=180] (0,0) ++(0 pt,2.25pt) -- ++(1.9485571585149868pt,-3.375pt)--++(-3.8971143170299736pt,0 pt) -- ++(1.9485571585149868pt,3.375pt);
\draw [fill=qqqqff,shift={(8.983899691231244,8.34233696522201)},rotate=90] (0,0) ++(0 pt,2.25pt) -- ++(1.9485571585149868pt,-3.375pt)--++(-3.8971143170299736pt,0 pt) -- ++(1.9485571585149868pt,3.375pt);
\draw [fill=qqqqff,shift={(9.972131443147275,8.32533149575884)},rotate=270] (0,0) ++(0 pt,2.25pt) -- ++(1.9485571585149868pt,-3.375pt)--++(-3.8971143170299736pt,0 pt) -- ++(1.9485571585149868pt,3.375pt);
\draw [fill=qqqqff,shift={(5.0157691181530275,3.5127836376815917)},rotate=90] (0,0) ++(0 pt,2.25pt) -- ++(1.9485571585149868pt,-3.375pt)--++(-3.8971143170299736pt,0 pt) -- ++(1.9485571585149868pt,3.375pt);
\draw [fill=qqqqff,shift={(9.987335008561367,3.495778168218421)},rotate=270] (0,0) ++(0 pt,2.25pt) -- ++(1.9485571585149868pt,-3.375pt)--++(-3.8971143170299736pt,0 pt) -- ++(1.9485571585149868pt,3.375pt);
\end{scriptsize}
\end{tikzpicture}
\end{minipage}
\begin{minipage}{0.49\linewidth}
\begin{enumerate}
\item Quelle est la valeur la plus petite pour $x$ ? Et la plus grande valeur pour $x$ ? \point{3}
\item Détermine le périmètre $\mathscr{P}$ de cette surface en fonction de $x$.\point{5}
\item Calcule $\mathscr{P}(2)$.\point{2}
\end{enumerate}
\end{minipage}


\end{ExoCu}

 


\end{pageParcoursu}

  
%%%%%%%%%%%%%%%%%%%%%%%%%%%%%%%%%%%%%%%%%%%%%%%%%%%%%%%%%%%%%%%%%%%
%%%%  Niveau 2
%%%%%%%%%%%%%%%%%%%%%%%%%%%%%%%%%%%%%%%%%%%%%%%%%%%%%%%%%%%%%%%%%%%



\begin{pageParcoursd} 
 
%%%%%%%%%%%%%%%%%%%%%%%%%%%%%%%%%%%%%%%%%%%%%%%%%%%%%%%%%%%%%%%%%%%
\begin{ExoCd}{Représenter.}{1234}{2}{0}{0}{0}{0}


 
\end{ExoCd}

 
%%%%%%%%%%%%%%%%%%%%%%%%%%%%%%%%%%%%%%%%%%%%%%%%%%%%%%%%%%%%%%%%%%%
\begin{ExoCd}{Chercher.communiquer.}{1234}{2}{0}{0}{0}{0}



\end{ExoCd}


%%%%%%%%%%%%%%%%%%%%%%%%%%%%%%%%%%%%%%%%%%%%%%%%%%%%%%%%%%%%%%%%%%%
\begin{ExoCd}{Représenter. Raisonner.}{1234}{2}{0}{0}{0}{0}


\end{ExoCd}

 %%%%%%%%%%%%%%%%%%%%%%%%%%%%%%%%%%%%%%%%%%%%%%%%%%%%%%%%%%%%%%%%%%%
\begin{ExoCd}{Représenter. Raisonner.}{1234}{2}{0}{0}{0}{0}


\end{ExoCd}
 
%%%%%%%%%%%%%%%%%%%%%%%%%%%%%%%%%%%%%%%%%%%%%%%%%%%%%%%%%%%%%%%%%%%
\begin{ExoCd}{Représenter. Raisonner.}{1234}{2}{0}{0}{0}{0}


\end{ExoCd}
 
\end{pageParcoursd}

%%%%%%%%%%%%%%%%%%%%%%%%%%%%%%%%%%%%%%%%%%%%%%%%%%%%%%%%%%%%%%%%%%%
%%%%  Niveau 3
%%%%%%%%%%%%%%%%%%%%%%%%%%%%%%%%%%%%%%%%%%%%%%%%%%%%%%%%%%%%%%%%%%%
\begin{pageParcourst}

%%%%%%%%%%%%%%%%%%%%%%%%%%%%%%%%%%%%%%%%%%%%%%%%%%%%%%%%%%%%%%%%%%%
\begin{ExoCt}{Représenter.}{1234}{2}{0}{0}{0}{0}


\end{ExoCt}

%%%%%%%%%%%%%%%%%%%%%%%%%%%%%%%%%%%%%%%%%%%%%%%%%%%%%%%%%%%%%%%%%%%
\begin{ExoCt}{Représenter. Raisonner.}{1234}{2}{0}{0}{0}{0}
 
 


\end{ExoCt}


%%%%%%%%%%%%%%%%%%%%%%%%%%%%%%%%%%%%%%%%%%%%%%%%%%%%%%%%%%%%%%%%%%%
\begin{ExoCt}{Raisonner.}{1234}{2}{0}{0}{0}{0}
 
\end{ExoCt}

%%%%%%%%%%%%%%%%%%%%%%%%%%%%%%%%%%%%%%%%%%%%%%%%%%%%%%%%%%%%%%%%%%%
\begin{ExoCt}{Représenter.}{1234}{2}{0}{0}{0}{0}

 

\end{ExoCt}

%%%%%%%%%%%%%%%%%%%%%%%%%%%%%%%%%%%%%%%%%%%%%%%%%%%%%%%%%%%%%%%%%%%
\begin{ExoCt}{Représenter.}{1234}{2}{0}{0}{0}{0}

 

\end{ExoCt} 
 
\end{pageParcourst}

%%%%%%%%%%%%%%%%%%%%%%%%%%%%%%%%%%%%%%%%%%%%%%%%%%%%%%%%%%%%%%%%%%%
%%%%  Brouillon
%%%%%%%%%%%%%%%%%%%%%%%%%%%%%%%%%%%%%%%%%%%%%%%%%%%%%%%%%%%%%%%%%%%


\begin{pageBrouillon} 
 
\ligne{32}



\end{pageBrouillon}

%%%%%%%%%%%%%%%%%%%%%%%%%%%%%%%%%%%%%%%%%%%%%%%%%%%%%%%%%%%%%%%%%%%
%%%%  Auto
%%%%%%%%%%%%%%%%%%%%%%%%%%%%%%%%%%%%%%%%%%%%%%%%%%%%%%%%%%%%%%%%%%%


%%%%%%%%%%%%%%%%%%%%%%%%%%%%%%%%%%%%%%%%%%%%%%%%%%%%%%%%%%%%%%%%%%%
\begin{pageAuto} 


\begin{ExoAuto}{Raisonner.}{1234}{2}{0}{0}{0}{0}

  
La distance verticale dans une chute libre est donnée par $h(t)=\frac{1}{2}g\times t^2$, $g$ est une constante égale à $9,81$ et le temps $t$ est exprimé en seconde. La distance est exprimée en mètres.

\begin{enumerate}
\item Quelle est la distance parcourue en $5$ secondes ? \point{2}
\item Complète le tableau c-dessous.

 \begin{tabular}{|c|c|c|c|c|}
 \hline 
 $t$ & $0$ & $\dfrac{3}{5}$ &  $5$ & $10$    \\ 
 \hline 
 $h(t)$ &  &  & $122,625$ &  \\ 
 \hline 
 \end{tabular} 
\item La distance est elle proportionnelle au temps ?\point{3}
\end{enumerate}


%%%%%%%%%%%%%%%%%%%%%%%%%%%%%%%%%%%%%%%%%%%%%%%%%%%%%%%%%%%%%%%%%%%
\end{ExoAuto}

\begin{ExoAuto}{Raisonner.}{1234}{2}{0}{0}{0}{0}
  

\end{ExoAuto}

%%%%%%%%%%%%%%%%%%%%%%%%%%%%%%%%%%%%%%%%%%%%%%%%%%%%%%%%%%%%%%%%%%%
\begin{ExoAuto}{Raisonner.}{1234}{2}{0}{0}{0}{0}

   
Le son parcourt 330 m en 1 seconde. Exprime la distance en fonction du temps.
 

\end{ExoAuto}

 
%%%%%%%%%%%%%%%%%%%%%%%%%%%%%%%%%%%%%%%%%%%%%%%%%%%%%%%%%%%%%%%%%%%
\begin{ExoAuto}{Raisonner.}{1234}{2}{0}{0}{0}{0}

 
 

\end{ExoAuto}


\end{pageAuto}
