\documentclass[openany]{book}

\input{../../../latex_preambule_style/preambule}
\input{../../../latex_preambule_style/styleCoursCycle4}
\input{../../../latex_preambule_style/styleExercices}
\input{../../../latex_preambule_style/styleExercicesAideCompetences}
%\input{../../latex_preambule_style/styleCahier}
\input{../../../latex_preambule_style/bas_de_page_cycle4}
\input{../../../latex_preambule_style/algobox}



%%%%%%%%%%%%%%%  Affichage ou impression  %%%%%%%%%%%%%%%%%%
 \usepackage{geometry}
 \geometry{top=2.5cm, bottom=0cm, left=2cm , right=2cm}
%%%%%%%%%%%%%%%%%%%%%%%%%%%%%%%%%%%%%%%%%%%%%%%%

\begin{document}

\Exe

Trace un triangle $ABC$ tel que $AB = 3 cm$, $BC = 4 cm$ et $AC = 5 cm$.
\begin{enumerate}
\item Quelle est la nature du triangle $ABC$ ?
\item Soit $d$ une droite parallèle à $(AB)$. $A'$ et $B'$ sont deux points de $d$ tels que $A'B' = $10,5 cm.
\item 
\begin{enumerate}
\item On appelle l'homothétie $h$ qui transforme $A$ en $A'$ et $B$ en $B'$. Construire $C'$ l'image du point $C$ par $h$.
\item Calcule $B'C'$.
\end{enumerate}
\end{enumerate}



\end{document}
