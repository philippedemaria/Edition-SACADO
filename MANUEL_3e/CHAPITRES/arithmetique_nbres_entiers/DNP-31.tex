
Dans toute cette activité, $a$ et $b$ sont deux nombres entiers strictement positifs tels que $a>b$.
\begin{enumerate}
\item\begin{enumerate}
\item Donne un diviseur $d$ (différent de 1) commun à 45 et à 35.
\item $d$ est-il un diviseur de $45+35$ ? de $45-35$ ?
\end{enumerate}
\item Supposons que 2 soit un diviseur de $a$ alors $a=2\times n$ avec $n$ un nombre entier.
\par Supposons que 2 soit un diviseur de $b$ alors $b=2\times m$ avec $m$ un nombre entier.
\par Alors \\ 
$a+b=\ldots\times n+\ldots\times m=\ldots\times\ldots\ldots$ \hspace{1cm} d'où 2 est un diviseur de $\ldots\ldots\ldots$\\
$a-b=\ldots\times n-\ldots\times m=\ldots\times\ldots\ldots$ \hspace{1cm} d'où 2 est un diviseur de $\ldots\ldots\ldots$
\item Si $5$ est un diviseur commun à $a$ et $b$, prouve que 5 est aussi un diviseur de $a-b$.
\par 5 est un diviseur de \ldots alors $\ldots=5\times\ldots$ avec $n$ un nombre entier.
\par 5 est \dotfill.
\par Donc $a-b=\ldots\ldots\ldots\ldots\ldots=\ldots\times\ldots\ldots$ d'où \dotfill.
\end{enumerate}
\textbf{Bilan} Si $k$ est un diviseur commun de $a$ et de $b$ alors $a=k\times a'$ et $b=k\times b'$ avec $a'$ et $b'$ des nombres entiers.
\par Donc $a-b=$\dotfill
\par d'où $k$ est un diviseur de $a-b$.
\par\textbf{Application} Soit $k=PGCD(a,b)$. Donc
$k$ divise $a-b$ et 
$k$ divise $b$. \\
Donc $k$ est un diviseur commun à $b$ et $a-b$.\\
Est-ce le plus grand ?
Soit $\ell$ le $PGCD(b;a-b)$. Alors $k\leqslant\ell$ et $b=\ell\times b_1$ et $a-b=\ell\times c_1$.
\\On obtient alors que
$a=a-b+b=\ell\times c_1+\ell\times b_1=\ell\times(c_1+b_1)$\\
$\ell$ est donc un diviseur commun de $a$ et $b$ et $\ell\leqslant k$.
Donc $k$ est le $PGCD(b;a-b)$.