\begin{titre}[Arithmétique]

\Titre{Nombres premiers}{2}
\end{titre}

 

\begin{CpsCol}
\begin{description}
\item[$\square$] Savoir tester si un nombre est premier ou ne l'est pas en utilisant la calculatrice ou un logiciel
\item[$\square$] Donner la liste des nombres premiers inférieurs à un entier donné (crible d'Eratosthène)
\item[$\square$] Décomposer un nombre entier en produit de facteurs premiers (à la main ou à l'aide d'un logiciel) 
\end{description}
\end{CpsCol}




\begin{DefT}{Nombre premier}

Un \textbf{nombre premier} est un nombre entier positif qui admet exactement deux diviseurs positifs : 1 et lui-même.


\end{DefT}

\begin{Rq}

1 n'est pas premier.


\end{Rq}

\begin{Ex}

Les premiers nombres premiers sont $2$ (qui est le seul nombre premier pair), $3$,$5$,$7$,$11$,$13$,$17$,$19$,$23$,$29$, \ldots


\end{Ex}

\ExeComp{Chercher. Calculer.}

Décomposer en produit de facteurs premiers (à la main, à l’aide d’un tableur ou d’un logiciel de programmation) les entiers naturels suivants : 306 ; 124 ; \np{2 220}


\sacado{1ed97ed7}





\begin{Th}

 Tout entier $n$ avec $n\geq 2$ admet au moins un diviseur premier.

\end{Th}

\begin{Prop}

Si $n$ n'est pas premier et $n\geq 2$ alors il admet un diviseur premier compris entre 2 et $\sqrt{n}$


\end{Prop}

\begin{Pv}

Si $n$ est premier, il admet bien un diviseur premier: lui-même.

Si $n$ n'est pas premier alors il admet un plus petit diviseur positif $p\neq 1$.
$p$ est premier sinon $p$ aurait lui-même un diviseur positif différent de 1 qui serait un diviseur de $n$, mais plus petit que $p$.

De plus, $n$ peut s'écrire $n=p \times r$ avec $p\leq r$ donc $p^{2}\leq p \times r$ soit $p^{2} \leq n$ et $p\leq\sqrt{n}$


\end{Pv}

\begin{Rq}

On utilise ce théorème de la manière suivante:\\
Si un naturel $n\geq 2$ n'admet pas de diviseur premier compris entre 2 et $\sqrt{n}$ alors $n$ est premier.

\end{Rq}

\begin{Ex}

Pour savoir si 631 est premier, il suffit de tester tous les nombres premiers inférieurs à $\sqrt{631}\approx25,12$

\end{Ex}


\begin{Mt}[Crible d'Eratosthène]

\begin{enumerate}
\item Écrire dans un carré de $10 \times 10$ les $100$ premiers nombres entiers
\item Barrer le $1$, puis entourer le $2$ car c'est un nombre premier, puis barrer tous les multiples de $2$ qui, par définition, ne sont pas des nombres premiers.
\item Le nombre suivant dans la liste ($3$) est un nombre premier. L'entourer et barrer tous les multiples de $3$ dans la liste.
\item En procédant de même, donner la liste de tous les nombres premiers, inférieurs ou égaux à $100$
\end{enumerate}





\end{Mt}

\begin{Th}[Théorème fondamental de l'algèbre]

Tout entier naturel strictement supérieur à 1 admet une décomposition, unique à l'ordre des facteurs près, en produit de nombres premiers.





\end{Th}


\ExeComp{ Calculer.}

\begin{enumerate}
\item 217 est-il premier ?
\item 289 est-il premier ?
\end{enumerate}

\ExeComp{ Représenter.}

Retrouver chaque nombre décomposé en produit de facteur premier : 
\begin{enumerate}
\item $A = 2^2 \times 3^2 \times 5$
\item $B = 2^3 \times 2^3 \times 7$
\end{enumerate}


\ExeComp{ Représenter.}

Décomposer les nombres entiers suivants en produits de facteurs premiers : 
 
$$ 630 \quad ; \quad  75 \quad ; \quad 164 \quad ; \quad 3192 $$


\ExeComp{ Représenter.}

\begin{enumerate}
\item Décomposer 150 en produits de facteurs premiers : 
\item A l'aide de la question 1 et en construisant un  arbre, déterminer tous les diviseurs de 150.
\end{enumerate} 
 

\ExeComp{ Chercher. La conjecture de Goldbach}

Le 7 juin 1742, le mathématicien prussien Christian Goldbach   propose la conjecture suivante :

  $$ \text{ Tout nombre pair supérieur à 3 est la somme de deux nombres premiers.}$$

Vérifier cette conjecture pour 10, 26 et 138.




\sacado{24612830}

\sacado{089e514d}

\sacado{e305e112}
