
Effectue la division euclidienne de 434 par 126.\\
$434=126\times\ldots+\ldots$
\par Soit $d$ le $PGCD(434;126)$ donc $434=d\times n$ et $126=d\times m$ avec $n$ et $m$ deux nombres entiers.
\par Est-ce que $d$ divise 56 ?
\par$56=434-3\times126=\ldots\ldots-\ldots\ldots\ldots=\ldots\ldots-\ldots\ldots=\ldots\times\ldots\ldots$ donc
\dotfill\par\dotfill.
$d$ divise $56$ et 
$d$ divise $126$, donc
 $d$ est un diviseur commun de 126 et 56.\\
Est-ce le plus grand ?
Soit $\ell$ le $PGCD(126;56)$. Alors $d\leqslant\ell$ et $126=\ell\times n$ et $56=\ell\times m$.
\\On obtient\\
$434=126\times3+56=\ell\times n\times3+\ell\times m=\ell\times(3n+m)$
$\ell$ est donc un diviseur commun à 434 et 126 donc $\ell\leqslant d$.
\\$d$ est donc le $PGCD(126;56)$.\\
\textbf{Bilan}
Soit $(q;r)$ le couple obtenu par la division euclidienne de $a$ par $b$.
\[a=b\times q+r\]
\par Soit $d$ le $PGCD(a;b)$ donc $a=d\times n$ et $b=d\times m$ avec $n$ et $m$ deux nombres entiers.
\par$r=a-b\times q=\ldots\ldots\ldots-\ldots\ldots\ldots\ldots=\ldots\times\ldots\ldots\ldots$ donc $d$ divise $r$.
$d$ divise $r$ et 
$d$ divise $b$, donc $d$ est un diviseur commun à $b$ et $r$.\\
Est-ce le plus grand ?
Soit $\ell$ le $PGCD(b;r)$. Alors $d\leqslant\ell$ et $b=\ell\times b_1$ et $r=\ell\times r_1$.
\\On obtient alors que \\
$a=b\times q+r=\ell\times b_1\times q+\ell\times r_1=\ell\times(b_1\times q+r_1)$
$\ell$ est donc un diviseur commun de $a$ et $b$ et $\ell\leqslant d$.
Donc $d$ est le $PGCD(b;r)$.