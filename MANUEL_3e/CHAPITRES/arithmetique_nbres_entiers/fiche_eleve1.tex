\documentclass[openany]{book}

\input{../../../latex_preambule_style/preambule}
\input{../../../latex_preambule_style/styleCoursCycle4}
\input{../../../latex_preambule_style/styleExercices}
\input{../../../latex_preambule_style/styleExercicesAideCompetences}
%\input{../../latex_preambule_style/styleCahier}
\input{../../../latex_preambule_style/bas_de_page_cycle4}
\input{../../../latex_preambule_style/algobox}



%%%%%%%%%%%%%%%  Affichage ou impression  %%%%%%%%%%%%%%%%%%
 \usepackage{geometry}
 \geometry{top=3cm, bottom=0cm, left=2cm , right=2cm}
%%%%%%%%%%%%%%%%%%%%%%%%%%%%%%%%%%%%%%%%%%%%%%%%

\begin{document}

\ExeComp{Représenter. Calculer.}

On souhaite ranger 142 bonbons dans des boites de 12 bonbons. Les boites doivent être complétées entièrement avant d'utiliser d'une nouvelle boite. 
\begin{enumerate}
\item Combien de bonbons ne sont-ils pas rangés dans une boite ?
\item Combien de boites sont-elles entièrement remplies ?
\item Quel est le nombre minimal de boites nécessaires pour ranger tous les 142 bonbons ? Combien de bonbons faudrait-il rajouter pour remplir la boite incomplète ? 
\end{enumerate}


\AD{1}{DNP-30}







\end{document}
