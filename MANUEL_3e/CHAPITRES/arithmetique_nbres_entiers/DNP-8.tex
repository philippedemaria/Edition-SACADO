
Cherchons le PGCD de 105 et 60. On note
 $PGCD(105;60)$
\begin{multicols}{2}
$105=1 \times 105$\\
$105=3 \times 35$\\
$105=5 \times 21$\\
$105=7 \times 15$\\
La liste des diviseurs de 105 est $\lbrace 1;3;5;7;15;21;35;105 \rbrace$\\
\columnbreak
\\
$60=1 \times 60$\\
$60=2 \times 30$\\
$60=3 \times 20$\\
$60=4 \times 15$\\
$60=5 \times 12$\\
$60=6 \times 10$\\
La liste des diviseurs de 60 est $\lbrace 1;2;3;4;5;6;10;12;15;20;30;60 \rbrace$
\end{multicols}
Les diviseurs communs à 60 et 105 sont $\lbrace 1;3;5;15 \rbrace$, donc $PGCD(60;105)=15$