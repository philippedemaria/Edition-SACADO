\begin{titre}[Arithmétique]

\Titre{Division euclidienne}{1,5}
\end{titre}

 

\begin{CpsCol}
\begin{description}
\item[$\square$] Utiliser et écrire en ligne la division euclidienne de $a$ par $b$, deux nombres entiers.
\end{description}
\end{CpsCol}


\ExeComp{Représenter. Calculer.}

On souhaite ranger 142 bonbons dans des boites de 12 bonbons. Les boites doivent être complétées entièrement avant d'utiliser d'une nouvelle boite. 
\begin{enumerate}
\item Combien de bonbons ne sont-ils pas rangés dans une boite complète ?
\item Combien de boites sont-elles entièrement remplies ?
\item Quel est le nombre minimal de boites nécessaires pour ranger tous les 142 bonbons ? Combien de bonbons faudrait-il rajouter pour remplir la boite incomplète ? 
\end{enumerate}






\begin{DefT}{Division euclidienne dans $\N$}\index{Division euclidienne}
Écrire la \textbf{division euclidienne} d´un nombre entier naturel $a$ par un entier naturel $b$, tous deux non nuls, c'est déterminer les nombres entiers $q$ et $r$ tels que $a=b \times q + r$ avec $0 \leq r < b$.\\
$q$ est appelé le \textbf{quotient}\index{Division euclidienne!Quotient} de la division euclidienne de $a$ par $b$.\\
$r$ est appelé le \textbf{reste} \index{Division euclidienne!Reste} de la division euclidienne de $a$ par $b$. 
\end{DefT}

\begin{Ex}
La division euclidienne de $254$ par $7$ s'écrit $254=7 \times 36 + 2$ (où 36 est le quotient et 2 le reste). 
\end{Ex}

\begin{Att}
$22 = 3 \times 5 + 7$ mais cette égalité n'est pas l'écriture de la division euclidienne de 22 par 5 ou par 3. En effet, $7>3$ et $7>5$ ! \\
L'égalité de la division euclidienne de 22 par 3 s'écrit : $22=3 \times 7 + 1$\\
L'égalité de la division euclidienne de 22 par 5 s'écrit : $22=5 \times 4 + 2$.
\end{Att}


 \AD{1}{DNP-30}
 
 
 \ExeComp{Communiquer.}

 Dans la division de 85 par 6, le reste est égal à 1 et le quotient est égal à 14. Ecrire ce résultat par une égalité.
 
  \ExeComp{Chercher.}

Cédric attend le bus qui peut contenir 53 personnes. Il passe un bus toutes les 17 minutes. Il y a 164 personnes devant Cédric. Dans combien de temps Cédric pourra monter dans un bus sachant qu'il vient d'en voir un partir ?
 
 
 \sacado{3fee8c7e}