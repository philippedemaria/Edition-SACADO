
On a représenté sur le diagramme suivant les vols du mois de février d’une compagnie aérienne.  
\begin{center}
\definecolor{xdxdff}{rgb}{0.49019607843137253,0.49019607843137253,1.}
\definecolor{ffdxqq}{rgb}{1.,0.8431372549019608,0.}
\definecolor{ffffqq}{rgb}{1.,1.,0.}
\definecolor{ffxfqq}{rgb}{1.,0.4980392156862745,0.}
\definecolor{ffqqqq}{rgb}{1.,0.,0.}
\definecolor{xfqqff}{rgb}{0.4980392156862745,0.,1.}
\definecolor{qqqqff}{rgb}{0.,0.,1.}
\begin{tikzpicture}[line cap=round,line join=round,>=triangle 45,x=1.0cm,y=1.0cm]
\clip(-0.096,-0.204) rectangle (18.304,8.336);
\draw[color=xfqqff,fill=xfqqff,fill opacity=1.0] (5.,3.5757359312880714) -- (5.424264068711929,3.5757359312880714) -- (5.424264068711929,4.) -- (5.,4.) -- cycle; 
\draw [shift={(5.,4.)},color=ffqqqq,fill=ffqqqq,fill opacity=1.0] (0,0) -- (0.:0.6) arc (0.:30.488940499830935:0.6) -- cycle;
\draw [shift={(5.,4.)},color=ffxfqq,fill=ffxfqq,fill opacity=0.95] (0,0) -- (30.488940499830935:0.6) arc (30.488940499830935:59.44286434307001:0.6) -- cycle;
\draw [shift={(5.,4.)},fill=black,fill opacity=1.0] (0,0) -- (59.44286434307001:0.6) arc (59.44286434307001:90.:0.6) -- cycle;
\draw [shift={(5.,4.)},color=qqqqff,fill=qqqqff,fill opacity=0.44]  plot[domain=1.5707963267948966:4.71238898038469,variable=\t]({1.*4.*cos(\t r)+0.*4.*sin(\t r)},{0.*4.*cos(\t r)+1.*4.*sin(\t r)});
\draw [shift={(5.,4.)},color=xfqqff,fill=xfqqff,fill opacity=0.66]  (0,0) --  plot[domain=-1.5707963267948966:0.,variable=\t]({1.*4.*cos(\t r)+0.*4.*sin(\t r)},{0.*4.*cos(\t r)+1.*4.*sin(\t r)}) -- cycle ;
\draw [shift={(5.,4.)},color=ffqqqq,fill=ffqqqq,fill opacity=0.6]  (0,0) --  plot[domain=0.:0.5321323971666955,variable=\t]({1.*4.*cos(\t r)+0.*4.*sin(\t r)},{0.*4.*cos(\t r)+1.*4.*sin(\t r)}) -- cycle ;
\draw [shift={(5.,4.)},color=ffxfqq,fill=ffxfqq,fill opacity=0.4]  (0,0) --  plot[domain=0.5321323971666955:1.037473699602908,variable=\t]({1.*3.973415659102379*cos(\t r)+0.*3.973415659102379*sin(\t r)},{0.*3.973415659102379*cos(\t r)+1.*3.973415659102379*sin(\t r)}) -- cycle ;
\draw [shift={(5.,4.)},color=ffffqq,fill=ffffqq,fill opacity=0.5]  (0,0) --  plot[domain=1.037473699602908:1.5707963267948966,variable=\t]({1.*4.*cos(\t r)+0.*4.*sin(\t r)},{0.*4.*cos(\t r)+1.*4.*sin(\t r)}) -- cycle ;
\draw (10.644,6.776) node[anchor=north west] {Vols vers l'Asie};
\draw (10.664,6.176) node[anchor=north west] {Vols vers l'Afrique};
\draw (10.664,5.616) node[anchor=north west] {Vols vers l'Amérique};
\draw (10.704,5.056) node[anchor=north west] {Vols vers l'Europe};
\draw (10.724,4.436) node[anchor=north west] {Vols vers la France};
\begin{scriptsize}
\draw [fill=ffffqq] (10.304,6.636) circle (2.5pt);
\draw [fill=ffdxqq] (10.304,6.036) circle (2.5pt);
\draw [fill=ffqqqq] (10.324,5.436) circle (2.5pt);
\draw [fill=xfqqff] (10.344,4.836) circle (2.5pt);
\draw [fill=xdxdff] (10.364,4.236) circle (2.5pt);
\end{scriptsize}
\end{tikzpicture}
\end{center}

\begin{minipage}{8cm}
Dans chaque cas, quelle fréquence représentent les vols vers la France, l’Europe et l’Asie. 
\end{minipage}
\begin{minipage}{1cm}
$~~$
\end{minipage}
\begin{minipage}{8cm}
Au mois de février, cette compagnie a affrété 576 vols. Calculer le nombre de vols vers la France, l’Europe et l’Asie. 
\end{minipage}
