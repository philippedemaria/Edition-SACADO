
Lors d'un test d'endurance, plusieurs élèves ont eu 12 minutes pour
parcourir la plus grande distance possible. Voici les résultats des
élèves :
2230 -- 2450 -- 1890 -- 1850 -- 2650 -- 2630 -- 2110 -- 2250 -- 2180 --
1980 -- 2000 -- 2850 -- 1950 -- 2920 -- 1975 -- 1910 -- 1860 -- 1930 --
2010 -- 2400 -- 2650 -- 2320 -- 2190 -- 2730 -- 2120 -- 2380 -- 2220.

\begin{enumerate}
\item
Calcule la moyenne des distances parcourues.

\item
On veut calculer la moyenne approximative des distances parcourues.
Pour cela, dénombrer le nombre d'élèves dans chacun des intervalles
suivants :\\
\hskip 0pt plus 500pt minus 500pt\begin{tabular}{|*{6}{c|}}
\hline
$[1800 ; 2000 [$ & $[2000 ; 2200 [$ &
$[2200 ; 2400 [$ & $[2400 ; 2600 [$ &
$[2600 ; 2800 [$ & $[2800 ; 3000 [$ \\
\hline
&&
 &&& \\
\hline
\end{tabular}\hskip 0pt plus 500 pt minus 500 pt\strut
\item
Calculer une moyenne approchée en remplaçant la distance de chaque élève
	par le début de chaque intervalle (1800 pour le premier, 2000 pour
	le deuxième,\dots) en pensant à remplacer chaque série de nombres
	identiques par une multiplication.
	
\item Reprendre la question précédente en 
	remplaçant la distance de chaque élève
	par le milieu de chaque intervalle (1900 pour le premier, 2100 pour
	le deuxième,\dots)
	
\item Reprendre la question précédente en 
	remplaçant la distance de chaque élève
	par la fin de chaque intervalle (2000 pour le premier, 2200 pour
	le deuxième,\dots)
	
\end{enumerate}