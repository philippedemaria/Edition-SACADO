
Lors de la première édition de la Course aux Nombres, les 24 élèves de la classe de Cinquième 2 en Colombie ont obtenu les résultats suivants. Les notes sont évaluées sur 30 points.


\begin{enumerate}
\item Reproduire la feuille de calcul comme indiqué ci dessous.

\begin{tabular}{|c|c|c|c|c|c|c|c|c|c|}
\hline 
\rowcolor{gray} & A & B & C & D & E & F & G & H & I \\ 
\hline 
\cellcolor{gray} 1 & 11 & 16 & 22 & 20 & 26 &  &  & Nombres d'élèves total &  \\ 
\hline 
\cellcolor{gray}2& 26 & 18 & 26 & 19& 16 &  &  & Nombre d'élèves dont la nombre est égale à 22 &  \\ 
\hline 
\cellcolor{gray}3 & 17 &27 & 18 & 16 & 22 &  & & Fréquence d'élèves dont la note est égale à 22 &  \\ 
\hline 
\cellcolor{gray}4 & 16 & 19 & 11 & 21 & 17 & &  & Nombre d'élèves dont la note est égale à 19 &  \\ 
\hline 
\cellcolor{gray}5 & 22 & 15 & 22 & 23 &  &  &  & Fréquence d'élèves dont la note est égale à 19 &  \\ 
\hline 
\end{tabular} 

\item  
\begin{enumerate}
\item  Déterminer dans la cellule I1 le nombre d'élèves participants à ce jeu. 
Rappel : Pour déterminer le nombre de cases remplies d'un tableau, on utilise la syntaxe : =NB(A1:E5). 
\item  Déterminer dans la cellule I2 le nombre d'élèves dont la note est égale à 22 ? 
Rappel : Pour déterminer le nombre de cases remplies avec la valeur 22, on utilise la syntaxe : =NB.SI(A1:E5;22). 
\item  Calcule dans la cellule I3 la fréquence des élèves ayant obtenu 22.
\end{enumerate}
\item  
\begin{enumerate}
\item Détermine la moyenne de cette classe. On pourra utiliser la syntaxe "=MOYENNE(A1:E5)".
\item Explique par une phrase le calcul du tableur pour donner la moyenne.
\item 
\begin{enumerate}
\item Complète le tableau ci-dessous.

\begin{tabular}{|c|c|c|c|c|c|c|}
\hline 
Notes & $[0;5[$ &  $[5;10[$  &  $[10;15[$  &  $[15;20[$  &  $[20;25[$  &  $[25;30]$  \\ 
\hline 
Fréquence &  &  &  &  &  &  \\ 
\hline 
\end{tabular} 

\item Calcule la moyenne avec ce regroupement.	
\end{enumerate}
\item Création d'un diagramme avec un tableur
\begin{enumerate}
\item  Sélectionne la plage de données A1:E5 puis clique sur l'icône graphique
\item  Quel est le problème de la plage de données A1:E5 ?
\end{enumerate}

\end{enumerate}
\end{enumerate}



