\begin{titre}[Les statistiques]

{\color{bleu3}{\LARGE Utilisation d'un tableur} \hfill{Niveau 1}}
\end{titre}



\begin{CpsCol}
\textbf{Interpréter, représenter et traiter des données}
\begin{description}
\item[$\square$] Calculer une fréquence
\item[$\square$] Utiliser un tableur pour effectuer des calcul
\end{description}
\end{CpsCol}

\begin{Rec}

Le travail de statistiques est souvent fastidieux, on a toujours recours à un tableur lorsque les données sont trop importantes. Une feuille de calcul d'un tableur se présente comme cela, en cellules. La cellule rouge est la cellule C3.

L'intérêt d'un tableur est le calcul automatique des caractéristiques demandées : moyenne, fréquence, effectif cumulés,... Pour cela, il faut utiliser des formules mathématiques.

\vspace{0.4cm}

\begin{tabular}{|c|m{2cm}|m{2cm}|m{2cm}|m{2cm}|m{2cm}|}
\hline 
\rowcolor{gray} & A & B & C & D &...\\ 
\hline 
\cellcolor{gray}1 & &4 &  & &  \\ 
\hline 
\cellcolor{gray}2 & & & 5 & &   \\ 
\hline 
\cellcolor{gray}3 & & & \cellcolor{red} & &  \\ 
\hline 
\cellcolor{gray}... & & &  & & \\ 
\hline 
\end{tabular} 

\subsubsection*{Quelques formules : Utilisation du signe = avant toute formule}

\begin{description}
\item[=a+b] calcule la somme de  $a$ et de $b$. 
\item[=SOMME(plage)] calcule la somme des cellules dans une plage rectangulaire.
\item[=NB.SI(plage;critère] renvoie parmi les cellules de la plage celle qui vérifie le critères.
\item[=a/b] calcule le quotient de  $a$ par $b$. $b$ doit être non nul.
\end{description}

\subsubsection*{Utilisations}
\begin{description}[leftmargin=*]
\item Pour calculer 4+5 dans la cellule D3, on tapera dans la cellule D3, \textbf{=B1+C2}. Lorsque une des valeurs des cellules B1 ou C2 change, la somme change alors. 
\item Pour calculer 4/5 dans la cellule B3, on tapera dans la cellule DB3, \textbf{=B1/C2}. 
\end{description}

\subsection*{Application concrète}

Voici les notes obtenues à un contrôle sur 10 par une classe de quatrième :\\
0 -- 1 -- 2 -- 2 -- 3 -- 3 -- 3 -- 3 -- 4 -- 4 -- 5 -- 5 -- 5 -- 6 -- 6 -- 6 -- 7 -- 7 -- 8 -- 8 -- 8 -- 9 -- 9 -- 10 -- 10.

\begin{enumerate}
\item Créer un tableau sur une feuille de calcul et complète le tableau ci-dessous avec les formules adéquates.
\item Combien d'élèves ont obtenu moins de 5 ?
\item Quel est le pourcentage d'élèves qui ont obtenu 8 ?
\item Quel est le pourcentage d'élèves qui ont obtenu au moins 8 ?
\item Change un 3 et par un 8 dans la série statistique. Remarque le changement de résultats.
\end{enumerate}

\begin{tabular}{|p{4.2cm}*{11}{|p{5mm}}|p{8mm}|}
\hline
Note&0 & 1 & 2& 3 & 4 & 5 & 6&7 &8 &9 &10&Total \\
\hline
Effectifs&1 & 1 & 2& &&&& & &&& \\
\hline
Effectifs cumulés&1 & 2 & 4& &  & & & && & & \\
\hline
Fréquences &0,04 & 0,04 & 0,08 & &  & & & & & & & \\
\hline
Fréquences cumulées &0,04  & 0,08  & 0,16  &  &  & & & & & & & \\
\hline
\end{tabular}




\subsubsection*{Effectifs -- Effectifs cumulés}
	Pour chaque note, {\em l'effectif} est le nombre d'élèves ayant eu cette note.	Par exemple (dans le tableau), 1 élève a eu 0 ;	2 élèves ont eu 1\dots\\ 
	Pour chaque note, {\em l'effectif cumulé} est le nombre d'élèves ayant eu cette note ou une note inférieure. Pour le calculer, il suffit à chaque fois de cumuler les effectifs.\\
	Par exemple (dans le tableau), 1 élève a eu 0 ;	$1+2=3$ élèves ont eu 1 au plus ; $3+4=7$ élèves ont eu 2 au plus\dots
	
\subsubsection*{Fréquences -- Fréquences cumulées}
	Pour chaque note, {\em la fréquence} exprime la proportion d'élèves. Par exemple, sur les 20 élèves, 4 élèves ont eu une note de 2.
	La proportion est de 4 sur 20 ou $4\over20$ que l'on exprime en \%\ par le calcul $100\times{4\over20}=20$.\\
	Comme pour les effectifs cumulés, les {\em fréquences cumulées}	sont obtenues en cumulant les fréquences.
\end{Rec}

\begin{DefT}{Fréquence}
La population $P$ étudiée a un effectif total égal à $N$.
La fréquence $f$ d'un sous ensemble de la population, appelé $A$, d'effectif $n$ est le quotient de cet effectif sur l'effectif total $N$. On écrit alors $f_A=\frac{n}{N}$.
\end{DefT}


\begin{Ex}
Une classe de Quatrième compte 25 élèves dont 12 filles.\\
La population est l'ensemble des élèves de la classe. L'effectif de la population est égal à 25, c'est l'effectif total : $N=25$.\\
Pour calculer la fréquence des filles dans cette classe, on détermine l'effectif de ce sous ensemble : les filles. $n=12$\\
Donc $f_{filles}=\frac{12}{25}$.
\end{Ex}

\begin{AD}

Lors du concours Algoréa, les 12\% meilleurs élèves sont qualifiés pour le tour suivant. 
\begin{description}
\item[•] En sixième, Mathis est arrivée $682^{ième}$ sur 6800 participants.
\item[•] En cinquième, Pol est arrivé $524^{ième}$ sur 5200 participants.
\item[•] En quatrième, Luisa est arrivée $855^{ième}$ sur 8200 participants.
\item[•] En troisième, Rafaela est arrivée $423^{ième}$ sur 4500 participants.
\end{description}
Quel élèves ont été sélectionnés pour le tour suivant ?

\end{AD}


\begin{AD}

Lors de la première édition de la Course aux Nombres, les 24 élèves de la classe de Cinquième 2 en Colombie ont obtenu les résultats suivants. Les notes sont évaluées sur 30 points.


\begin{enumerate}
\item Reproduire la feuille de calcul comme indiqué ci dessous.

\begin{tabular}{|c|c|c|c|c|c|c|c|c|c|}
\hline 
\rowcolor{gray} & A & B & C & D & E & F & G & H & I \\ 
\hline 
\cellcolor{gray} 1 & 11 & 16 & 22 & 20 & 26 &  &  & Nombres d'élèves total &  \\ 
\hline 
\cellcolor{gray}2& 26 & 18 & 26 & 19& 16 &  &  & Nombre d'élèves dont la nombre est égale à 22 &  \\ 
\hline 
\cellcolor{gray}3 & 17 &27 & 18 & 16 & 22 &  & & Fréquence d'élèves dont la note est égale à 22 &  \\ 
\hline 
\cellcolor{gray}4 & 16 & 19 & 11 & 21 & 17 & &  & Nombre d'élèves dont la note est égale à 19 &  \\ 
\hline 
\cellcolor{gray}5 & 22 & 15 & 22 & 23 &  &  &  & Fréquence d'élèves dont la note est égale à 19 &  \\ 
\hline 
\end{tabular} 

\item  
\begin{enumerate}
\item  Déterminer dans la cellule I1 le nombre d'élèves participants à ce jeu. 
Rappel : Pour déterminer le nombre de cases remplies d'un tableau, on utilise la syntaxe : =NB(A1:E5). 
\item  Déterminer dans la cellule I2 le nombre d'élèves dont la note est égale à 22 ? 
Rappel : Pour déterminer le nombre de cases remplies avec la valeur 22, on utilise la syntaxe : =NB.SI(A1:E5;22). 
\item  Calcule dans la cellule I3 la fréquence des élèves ayant obtenu 22.
\end{enumerate}
\item  
\begin{enumerate}
\item Détermine la moyenne de cette classe. On pourra utiliser la syntaxe "=MOYENNE(A1:E5)".
\item Explique par une phrase le calcul du tableur pour donner la moyenne.
\item 
\begin{enumerate}
\item Complète le tableau ci-dessous.

\begin{tabular}{|c|c|c|c|c|c|c|}
\hline 
Notes & $[0;5[$ &  $[5;10[$  &  $[10;15[$  &  $[15;20[$  &  $[20;25[$  &  $[25;30]$  \\ 
\hline 
Fréquence &  &  &  &  &  &  \\ 
\hline 
\end{tabular} 

\item Calcule la moyenne avec ce regroupement.	
\end{enumerate}
\item Création d'un diagramme avec un tableur
\begin{enumerate}
\item  Sélectionne la plage de données A1:E5 puis clique sur l'icône graphique
\item  Quel est le problème de la plage de données A1:E5 ?
\end{enumerate}

\end{enumerate}
\end{enumerate}




\end{AD}


\begin{autoeval}
\begin{tabular}{p{12cm}p{0.5cm}p{0.5cm}p{0.5cm}p{1cm}}
\textbf{Compétences visées} &  M I & MF & MF  & TBM \vcomp \\ 
Calculer une fréquence & $\square$ & $\square$  & $\square$ & $\square$ \vcomp \\ 
Utiliser un tableur pour effectuer des calcul & $\square$ & $\square$ & $\square$ & $\square$ \vcomp \\ 
\end{tabular}
{\footnotesize MI : maitrise insuffisante ; MF = Maitrise fragile ; MS = Maitrise satisfaisante ; TBM = Très bonne maitrise}
 
\end{autoeval}