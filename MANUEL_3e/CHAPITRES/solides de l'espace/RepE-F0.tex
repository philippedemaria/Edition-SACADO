\begin{titre}[Repérage dans l'espace]

\Titre{Coordonnées dans un repère}{1}
\end{titre}

 
\impress{1}{
\begin{CpsCol}
\textbf{Repérage dans l'espace}
\begin{description}
\item[$\square$] Se repérer sur une droite graduée, dans le plan muni d'un repère orthogonal
\item[$\square$] Connaître la définition d'abscisse et ordonnée
\end{description}
\end{CpsCol}

\begin{DefT}{Repère d'une droite}\index{Repère d'une droite}
Sur une droite $d$, on peut repérer tous les points à partir de deux points distincts O et I appartenant à la droite $d$.\\
Ainsi, un tel couple $\left(O;I\right)$ est appelé \textbf{repère} de la droite $d$.

\begin{tikzpicture}[line cap=round,line join=round,>=triangle 45,x=1.0cm,y=1.0cm]
\draw[->,color=black] (-5.5,0.) -- (2.505759637776914,0.);
\foreach \x in {-5,-4,-3,-2,-1,0,1,2}
\clip(-5.5,-0.5) rectangle (2.505759637776914,0.5);
\draw [domain=-5.5:2.505759637776914] plot(\x,{(-0.-0.*\x)/1.});
\begin{scriptsize}
\draw [fill=black] (-2.,0.) -- ++(-2.5pt,0 pt) -- ++(5.0pt,0 pt) ++(-2.5pt,-2.5pt) -- ++(0 pt,5.0pt);
\draw[color=red] (-1.9202930204239919,0.20528779975855618) node {$O$};
\draw [fill=black] (-1.,0.) -- ++(-2.5pt,0 pt) -- ++(5.0pt,0 pt) ++(-2.5pt,-2.5pt) -- ++(0 pt,5.0pt);
\draw[color=red] (-0.9186074188311553,0.20528779975855618) node {$I$};
\draw[color=black] (-5.228185007079405,0.19364029276329062) node {$d$};
\end{scriptsize}
\end{tikzpicture}

Le point O est appelé \og Origine du repère \fg{}.
\end{DefT}

\begin{DefT}{Abscisse d'un point sur une droite orientée}\index{Repère!Abscisse}
Étant donné un repère $\left(O;I\right)$ de la droite $d$, tout point $M$ peut être repéré par un nombre relatif $x$, appelé \textbf{abscisse du point M} qui donne sa position sur la droite $d$.

\definecolor{xdxdff}{rgb}{0.49019607843137253,0.49019607843137253,1.}
\begin{tikzpicture}[line cap=round,line join=round,>=triangle 45,x=1.0cm,y=0.14601238469314765cm]
\draw[->,color=black] (-5.251480021069936,0.) -- (5.114801204716395,0.);
\foreach \x in {-5,-4,-3,-2,-1,0,1,2,3,4,5}
\draw[shift={(\x,0)},color=black] (0pt,2pt) -- (0pt,-2pt) node[below] {\footnotesize $\x$};
\clip(-5.251480021069936,-2.904596567977349) rectangle (5.114801204716395,7.368504601846877);
\draw [domain=-5.251480021069936:5.114801204716395] plot(\x,{(-0.-0.*\x)/1.});
\begin{scriptsize}
\draw [color=red] (-0.1,0.) -- ++(5.0pt,0 pt) ++(-2.5pt,-2.5pt) -- ++(0 pt,5.0pt);

\draw[color=red] (-0.09163442216730192,1.) node {$O$};

\draw [color=red] (0.9,0.)-- ++(5.0pt,0 pt) ++(-2.5pt,-2.5pt) -- ++(0 pt,5.0pt);
\draw [color=blue] (2.9,0.)-- ++(5.0pt,0 pt) ++(-2.5pt,-2.5pt) -- ++(0 pt,5.0pt);
\draw [color=blue] (3.9,0.)-- ++(5.0pt,0 pt) ++(-2.5pt,-2.5pt) -- ++(0 pt,5.0pt);
\draw [color=blue] (-2.6,0.)-- ++(5.0pt,0 pt) ++(-2.5pt,-2.5pt) -- ++(0 pt,5.0pt);
\draw [color=blue] (0.4,0.)-- ++(5.0pt,0 pt) ++(-2.5pt,-2.5pt) -- ++(0 pt,5.0pt);

\draw[color=red] (0.9682887144018623,1.) node {$I$};

\draw[color=blue] (2.971659917587535,1.) node {$A$};

\draw[color=blue] (3.9384029981945754,1.) node {$B$};

\draw[color=blue] (-2.6307909471351896,1.) node {$C$};

\draw[color=blue] (0.39756087163385084,1.) node {$D$};
\end{scriptsize}
\end{tikzpicture}
\end{DefT}

\begin{Rq}
\begin{itemize}
\item L'abscisse du point O est par définition $0$ et l'abscisse du point $I$ est $1$.
\item Un repère $\left(O;I\right)$ définit ainsi un \og sens \fg{} : on dit alors que la droite $d$ est une droite \textbf{orientée}.
\item Les points situés sur la demie-droite $\left[O;I\right)$ ont une abscisse positive et les autres une abscisse négative.
\end{itemize}



\end{Rq}

\begin{Ex}
\begin{itemize}
\item L'abscisse du point A est $3$.
\item $4$ est l'abscisse du point $D$.
\item L'abscisse du point $C$ se situe entre $-3$ et $-2$.
\end{itemize}



\end{Ex}

\begin{DefT}{Repère du plan}\index{Repère du plan}
Tout point du plan peut être repéré par trois autres points $O$,$I$ et $J$ à condition que ces trois points soient distincts et ne soient pas alignés. \\
Un tel \og triplet \fg{} $\left(O;I;J\right)$ est appelé \textbf{repère du plan}.




\end{DefT}

\begin{DefT}{Origine; axe des abscisses, des ordonnées}\index{Repère!Origine}\index{Repère!Ordonnée}
\begin{itemize}
\item Le point $O$ est appelé \textbf{origine du repère}
\item La droite $\left(OI\right)$ est appelée \textbf{axe des abscisses}
\item La droite $\left(OJ\right)$ est appelée \textbf{axe des ordonnées}
\end{itemize}




\end{DefT}

\begin{DefT}{Repère orthonormal}\index{Repère orthonormal}
\begin{itemize}
\item Un repère est dit \textbf{orthogonal} si l'axe des abscisses et celui des ordonnées sont perpendiculaires
\item Un repère est dit \textbf{orthonormal} s'il est orthogonal et si en plus $OI=OJ$
\end{itemize}



\end{DefT}

\begin{Rq}

\begin{minipage}{0.48\linewidth}

Étant donné un repère orthonormal (O,I,J), tout point $M$ du plan peut être repéré par un couple de nombres relatifs $\left(x;y\right)$ appelé \textbf{coordonnées cartésiennes} du point $M$ (voir figure ci-dessous).

\begin{description}
\item $A(-1;2)$
\item $B(4;3)$
\item $C(2;-1)$
\end{description}

\end{minipage}
\hfill
\begin{minipage}{0.48\linewidth}

\definecolor{qqqqff}{rgb}{0.,0.,1.}
\definecolor{cqcqcq}{rgb}{0.7529411764705882,0.7529411764705882,0.7529411764705882}
\begin{tikzpicture}[line cap=round,line join=round,>=triangle 45,x=1.0cm,y=1.0cm]
\draw [color=cqcqcq,, xstep=1.0cm,ystep=1.0cm] (-2.073026147439821,-1.4530480083130866) grid (4.6098020671918265,3.6082115954152947);
\draw[->,color=black] (-2.073026147439821,0.) -- (4.6098020671918265,0.);
\foreach \x in {-2.,-1.,1.,2.,3.,4.}
\draw[shift={(\x,0)},color=black] (0pt,2pt) -- (0pt,-2pt) node[below] {\footnotesize $\x$};
\draw[->,color=black] (0.,-1.4530480083130866) -- (0.,3.6082115954152947);
\foreach \y in {-1.,1.,2.,3.}
\draw[shift={(0,\y)},color=black] (2pt,0pt) -- (-2pt,0pt) node[left] {\footnotesize $\y$};
\draw[color=black] (0pt,-10pt) node[right] {\footnotesize $0$};
\clip(-2.073026147439821,-1.4530480083130866) rectangle (4.6098020671918265,3.6082115954152947);
\begin{scriptsize}
\draw [color=qqqqff] (-1.,2.)-- ++(-2.5pt,0 pt) -- ++(5.0pt,0 pt) ++(-2.5pt,-2.5pt) -- ++(0 pt,5.0pt);
\draw[color=qqqqff] (-0.8773240404101388,2.2978531219581084) node {$A$};
\draw [color=qqqqff] (4.,3.)-- ++(-2.5pt,0 pt) -- ++(5.0pt,0 pt) ++(-2.5pt,-2.5pt) -- ++(0 pt,5.0pt);
\draw[color=qqqqff] (4.118417639645382,3.297001457969213) node {$B$};
\draw [color=qqqqff] (2.,-1.)-- ++(-2.5pt,0 pt) -- ++(5.0pt,0 pt) ++(-2.5pt,-2.5pt) -- ++(0 pt,5.0pt);
\draw[color=qqqqff] (2.1201209676231736,-0.6995918860752046) node {$C$};
\end{scriptsize}
\end{tikzpicture}
\end{minipage}
\end{Rq}

\begin{Ex}

\begin{enumerate}
\item Sur la figure ci-dessus, l'abscisse du point $M$ est $1,4$ et son ordonnée est $1,2$. On note $M\left(1,4;1,2\right)$.
\item \textbf{Attention}, il n'y a pas de signe $=$ entre $M$ et $\left(1,4;1,2\right)$ \ldots{}
\item On a aussi $N\left(-1;0,6\right)$
\end{enumerate}


\end{Ex}
}
%%%%%%%%%%%%%%%%%%%%%%%%%%%%%%%%%%%%%%%%%%%%%%%%%%%%%%%%%%%%%%%%%%%%%%%%%%%%%%%%%%%%%EXERCICES%%%%%%%%%%%%%%%%%%%%%%%%%%%%%%%%%%%%%%%%%%%%%%%%%%%%%%%%%%%%%%%%%%%%%%%%%%%%%%%%%%%%%%%%%%%%%%%%%%%%%%%%%%%%%%%%%%%%%%%%%%%%%%%%%%%%%%%%%%%%

\mini{
\AD{1}{RepE-9}

\AD{1}{RepE-19}
}{
\AD{1}{RepE-20}
}

\impress{0}{

\vspace{2cm}


{\Large {\color{violet}Aucun exercice sur ce chapitre dans le cahier}}


\begin{autoeval}
\begin{tabular}{p{12cm}p{0.5cm}p{0.5cm}p{0.5cm}p{1cm}}
\textbf{Compétences visées} &  M I & MF & MS  & TBM \vcomp \\ 
Se repérer sur une droite graduée, dans le plan muni d'un repère orthogonal & $\square$ & $\square$  & $\square$ & $\square$ \vcomp \\
Connaître la définition d'abscisse, ordonnée & $\square$ & $\square$  & $\square$ & $\square$ \vcomp \\  
\end{tabular}
{\footnotesize MI : maitrise insuffisante ; MF = Maitrise fragile ; MS = Maitrise satisfaisante ; TBM = Très bonne maitrise}
\end{autoeval}
}