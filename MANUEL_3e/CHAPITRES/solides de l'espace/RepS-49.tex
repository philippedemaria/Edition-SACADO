
Une boule de laiton mesure 10~cm de diamètre. Le laiton est un alliage
constitué de 40\% de zinc et de 60\% de cuivre.
\begin{enumerate}
  \item Calcule le volume de cette boule. (Arrondir au dixième de
    cm$^3$ près.)
  \item Sachant que la masse volumique du cuivre est 8,94~g/cm$^3$ et
    que la masse volumique du zinc est 7,1~g/cm$^3$; quelle est la
    masse de cette boule ?
  \item On veut recouvrir cette boule de peinture dorée.\par Calcule
    l'aire de la surface de la boule. Quelle quantité de peinture est
    nécessaire si 1~dL recouvre 0,1~m$^2$ ?
  \item La boule est sciée selon un plan situé à 3~cm de son
    centre.\par Fais le schéma.\par Calcule le rayon du cercle de
    section, la longueur de ce cercle et l'aire du disque de section.
\end{enumerate}

