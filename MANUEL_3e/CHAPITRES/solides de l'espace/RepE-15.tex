
\textbf{HISTORIQUE}
\begin{itemize}
\item IVe siècle av J.-C: ARISTOTE prouve la rotondité de la Terre en donnant deux
indices.
\item IVe siècle av J.-C: ANAXIMANDRE (à Milet) est l’introducteur du gnomon en
Grèce.
\item IIIe siècle av J.-C: ERATOSTHENE (à Alexandrie) mesure un méridien et donne la
première estimation précise du rayon terrestre.
\item IIe siècle av J.-C: HIPPARQUE DE NICEE construit un instrument de navigation appelé astrolabe. Il lance l’idée d’un quadrillage par méridiens et
parallèles et mesure les coordonnées géographiques de plusieurs
points.
\item IIe siècle: PTOLEMEE (à Alexandrie) calcule la longitude et la latitude de
8 000 points et publie un guide géographique copié pendant des
siècles.
\item Ve siècle: Les astronomes et géographes arabes perfectionnent les instruments
de mesure et prolongent la tradition grecque.
\item 1714: Apparition du premier chronomètre de marine permettant une
bonne détermination de la longitude en mer.
\item 1730: Le mathématicien anglais John Hadley et l'inventeur américain
Thomas Godfrey inventent en même temps et indépendamment le
sextant.
\item 1945: Les systèmes hyperboliques permettent par émission de signaux
radio de connaître sa position avec une précision de l’ordre de la
centaine de mètres.
\item 1980: Le système GPS, mis en place pour l'armée des USA, permet de
connaître sa position avec une précision de quelques mètres
(fonctionnement en PPS) ou d'une centaine de mètres
(fonctionnement en SPS).
\end{itemize}