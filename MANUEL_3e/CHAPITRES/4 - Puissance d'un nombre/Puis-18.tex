
Clara essaie d’évaluer le nombre de ses ancêtres, jusqu’à l’époque de Charlemagne. Elle part du principe que 25 ans séparent chaque génération de la suivante, que chaque individu a deux parents, qui ont à leur tour deux parents chacun, etc. Elle suppose également que tous les ascendants d’une génération donnée sont des personnes distinctes. Après quelques calculs, Clara va consulter l’encyclopédie en ligne Wikipédia, et déclare : « En fait, en 2015, nous sommes tous cousins ! » 

\vspace{0.2cm}
Évolution de la population mondiale (Source Wikipédia)
\vspace{0.2cm}

\begin{tabular}{|>{\centering\arraybackslash}p{2cm}|>{\centering\arraybackslash}p{2cm}|>{\centering\arraybackslash}p{2cm}|>{\centering\arraybackslash}p{2cm}|>{\centering\arraybackslash}p{2cm}|>{\centering\arraybackslash}p{2cm}|>{\centering\arraybackslash}p{2cm}|}
\hline 
ANNEE & $-5000$ & 400 & 1000 & 1500 & 1900 & 2000 \\ 
\hline 
Population mondiale & Entre 5 et 20 millions & Entre 190 et 206 millions & Entre 254 et 345 millions & Entre 425 et 540 millions & Entre 1,55 et 1,762 milliards & 6,127 miliards \\ 
\hline 
\end{tabular} 


\vspace{0.4cm}
Comment Clara est-elle parvenue à cette conclusion ?

