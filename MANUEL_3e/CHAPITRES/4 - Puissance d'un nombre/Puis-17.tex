
En homéopathie, on dilue dans de l’eau une teinture mère contenant une substance active. Le degré de dilution s’exprime en CH, abréviation de centésimale hahnemannienne (du nom de Samuel Hahnemann, l’un des pères de l’homéopathie).
\vspace{0.2cm}
On obtient la dilution 1 CH en mélangeant 1 volume de teinture mère, contenant la substance, avec 99 volumes d’eau. Ainsi une solution à la dilution 1 CH contient 1\% de la substance active. Autrement dit, dans un volume de solution à la dilution 1 CH, le nombre de molécules de substance active présentes est multiplié par 0,01, ou encore divisé par 100, par rapport au nombre de molécules présentes dans un volume égal de teinture mère.
\vspace{0.2cm}
On recommence ce procédé pour obtenir les dilutions suivantes : 2 CH (mélange d’un volume de solution 1 CH et de 99 volumes d’eau), 3 CH (1 volume de solution 2 CH et 99 volumes d’eau), etc.

\vspace{0.2cm}
Voici deux extraits adaptés de l’encyclopédie en ligne Wikipédia.

\vspace{0.4cm}
\textbf{Extrait 1 }

Il existe des dilutions pouvant atteindre 30 CH, soit une dilution par $10^{-60}$ de la teinture mère. Il est impossible de se représenter concrètement la petitesse extrême d’un tel chiffre. À titre de comparaison, le Soleil contient environ $10^{57}$ atomes et on estime que la partie observable de notre univers contient $10^80$ atomes. Un seul atome dilué dans la masse de mille soleils représente donc 30 CH et un seul atome dilué dans l’univers représente 40 CH. Pour un volume de teinture mère à l’état pur, et contenant $10^{+24}$ molécules de substance active (un peu plus d’une mole), les dilutions successives contiennent les nombres suivants de molécules de substance active :
 
\begin{tabular}{|>{\centering\arraybackslash}p{4cm}|c|c|c|c|c|c|c|c|c|c|c|c|c|}
 \hline 
CH & 1 & 2 & 3 & 4 & 5 & 6 & 7 & 8 & 9 & 10 & 11 & 12 & >12 \\ 
 \hline 
Nombre de molécules de substance active & $10^{22}$ &  & &  &  &  &  &  &  & $10^4$ & 100 & 1 & 0\\ 
 \hline 
 \end{tabular}  

À une dilution de 12 CH, et a fortiori de 15 CH et plus, la totalité des flacons ou des granules 
fabriqués ne comprend statistiquement plus une seule molécule de substance active.

\vspace{0.4cm}

\textbf{Extrait 2 }

Divers travaux expérimentaux ont été menés pour tenter de mettre en évidence une influence des dilutions extrêmes sur des phénomènes physiques ou chimiques observables. Ces travaux constituent des pistes de recherche pour étudier un éventuel effet physique mesurable des solutions très diluées. L’hypothèse étant que l’oxygène dissous dans l’eau conserverait une « mémoire » statique de la substance ayant subi la méthode de préparation homéopathique et qu’elle transmettrait cette mémoire aux granules de sucre.

\begin{enumerate}
\item Compléter le tableau figurant dans l'extrait 1.
\item Réaliser un petit texte qui explique le principe de la dilution et des unités CH et, en utilisant les deux extraits de Wikipédia, qui présente les arguments des adversaires de l’homéopathie et ceux de ses adeptes.

\Point{10}
\end{enumerate}
