
\begin{minipage}{0.48\linewidth}

\begin{enumerate}
\item En informatique, l'information est codée à partir de bits, qui ne prennent que deux valeurs : 0 et 1. Un octet est un regroupement de 8 bits. Combien d'informations différentes peuvent être codées sur un octet ?
\item Les capacités de stockage des mémoires informatiques (disques durs, clé USB, ...) utilisent un grand nombre d'octets. Cela conduit à utiliser des multiples de l'octet, dont voici les principaux ci-contre.
\end{enumerate}

À l'aide des unités précédentes, donner un ordre de grandeur de la taille d'un fichier relatif aux données suivantes :
\begin{description}
\item[•] une photographie numérique ;
\item[•] l’ensemble des données circulant sur le web en 2015 ;
\item[•] un texte de dix lignes sur un traitement de textes ;
\item[•] l’ensemble des données générées chaque année à travers le monde ;
\item[•] la capacité d’un disque dur vendu en 2015 ;
\item[•] un DVD
\end{description}

\end{minipage}
\hfill
\begin{minipage}{0.48\linewidth}

\begin{center}
\begin{tabular}{|c|c|c|}
\hline 
NOM & SYMBOLE & NOMBRE D'OCTETS \\ 
\hline 
Kilooctet & Ko & $10^{3}$ \\ 
\hline 
Megaoctet & Mo & $10^{6}$  \\ 
\hline 
Gigaoctet& Go & $10^{9}$  \\ 
\hline 
Teraoctet & To & $10^{12}$  \\ 
\hline 
Petaoctet & Po & $10^{15}$  \\ 
\hline 
Exaoctet & Eo & $10^{18}$  \\ 
\hline 
\end{tabular} 
\end{center}

\end{minipage}


