\begin{titre}[Les puissances]

\TitreSansTemps{Les puissances de base quelconque} 
\end{titre}


\begin{CpsCol}
\textbf{}
\begin{description}
\item[$\square$] Calculer avec des puissances de base quelconque et exposant entier
\end{description}
\end{CpsCol}


\Dec{1}{Puis-2}

\begin{DefT}{Puissance de base a}\index{Puissance!de base a}
Le produit $\underbrace{a \times a \times a \times \cdots \times a \times a}_n$ se note $a^n$ et se lit "a exposant $n$".  
La puissance du nombre $a$, $a^n$, est un produit de $n$ fois le même nombre $a$.
\end{DefT}
 
\begin{minipage}[t]{0.49\linewidth}
\begin{Prop}[Produit de puissances]\index{Puissances!Produit}
Soit $n$ et $m$ deux nombres entiers et $a$ un nombre.\\
$a^n \times a^m = a^{n+m}$.
\end{Prop}
 \begin{Ex}
 \begin{description}
 \item[•] $10^3 \times 10^4 = 10^{3+4}=10^7$
 \item[•] $2^2 \times 2^3 = 2^{2+3}=2^5$ 
  \end{description}
 \end{Ex}
\end{minipage}
 \hfill
\begin{minipage}[t]{0.49\linewidth}
\begin{Prop}[Quotient de puissances]\index{Puissances!Quotient}
Soit $n$ et $m$ deux nombres entiers et $a$ un nombre.\\
$\frac{a^n}{a^m} = a^{n-m}$.
\end{Prop}
 \begin{Ex}
$\frac{10^8}{10^2} = 10^6$ et $\frac{5^9}{5^3} = 5^6$
 \end{Ex}
\end{minipage}
 

\begin{minipage}{0.48\linewidth}
\Exo{1}{Puis-13}

\end{minipage}
\hfill
\begin{minipage}{0.48\linewidth}

\Exo{1}{Puis-20}

\end{minipage}



 
\Exo{1}{Puis-21}

 

\Exo{1}{Puis-22}


 
\begin{minipage}{0.48\linewidth}
\Exo{1}{DNP-56}

\end{minipage}
\hfill
\begin{minipage}{0.48\linewidth}

\Fl{1}{Puis-12}

\end{minipage}

%\PO{1}{Puis-23}

%\App{1}{Puis-24}

\App{1}{Puis-14}

\App{1}{Puis-15}

%\PO{1}{Puis-18}



%\PO{1}{Puis-19}
%\begin{autoeval}
%\begin{tabular}{p{12cm}p{0.5cm}p{0.5cm}p{0.5cm}p{1cm}}
%\textbf{Compétences visées} &  M I & MF & MS  & TBM \vcomp \\ 
%Calculer avec des puissances de base quelconque et exposant entier & $\square$ & $\square$  & $\square$ & $\square$ \vcomp \\  
%\end{tabular}
%{\footnotesize MI : maitrise insuffisante ; MF = Maitrise fragile ; MS = Maitrise satisfaisante ; TBM = Très bonne maitrise}
% 
%\end{autoeval}



