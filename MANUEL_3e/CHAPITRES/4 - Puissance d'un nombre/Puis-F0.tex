\begin{titre}[Les puissances]

\TitreSansTemps{Les puissances de 10} 
\end{titre}


\begin{CpsCol}
\textbf{Utiliser le calcul littéral}
\begin{description}
\item[$\square$] Écrire un nombre avec une puissance de base 10
\item[$\square$] Calculer avec des nombres écrits en puissance de 10
\item[$\square$] Écrire un nombre en écriture scientifique
\end{description}
\end{CpsCol}

\Rec{1}{Puis-0}


\begin{DefT}{Puissance de base 10}\index{Puissance!de base 10}
Le produit $\underbrace{10 \times 10 \times 10 \times \cdots \times 10 \times 10}_n$ se note $10^n$ et se lit "10 exposant $n$". 
\end{DefT}

\begin{Ex}
$100 = 10^2$ donc 100 est une puissance de 10 et $\np{10000} = 10^4$ donc 10000 est aussi une puissance de 10. 
\end{Ex}

\begin{Rq}
Le nombre de zéros est égal à l'exposant. $10^n= 1\underbrace{0000 \cdots 000}_{n \text{zéros}}$
\end{Rq}

\begin{Def}
Par convention, $10^0=1$
\end{Def}

\begin{minipage}[t]{0.48\linewidth}
\AD{1}{Puis-7}

\PO{1}{Puis-5}
\end{minipage}
\hfill
\begin{minipage}[t]{0.48\linewidth}
\Fl{1}{Puis-11}
\end{minipage}

\Fl{1}{Puis-10}

\Exo{1}{Puis-16}


\begin{DefT}{Puissance de base 10 d'exposant négatif}\index{Puissance!de base d'exposant négatif}
L'écriture $10^{-n}$ désigne l'inverse de $10^n$, c'est à dire : $10^{-n}= \frac{1}{10^n}$. 
\end{DefT}


\begin{Rq}
L'exposant correspond au nombre de chiffres après la virgule. $10^{-n}= 0,\underbrace{0000 \cdots 001}_{n \text{chiffres}}$
\end{Rq}


\begin{minipage}[t]{0.49\linewidth}
\AD{1}{Puis-10}
\end{minipage}
\hfill
\begin{minipage}[t]{0.49\linewidth}
\AD{1}{Puis-8}
\end{minipage}



