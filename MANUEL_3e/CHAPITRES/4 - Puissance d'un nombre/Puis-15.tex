
\begin{minipage}{0.59 \linewidth}
Un laboratoire effectue des recherches sur le développement d’une population de bactéries dans un milieu clos. Les chercheurs observent que le nombre de bactéries triple toutes les heures. À 0 heure, il y a 4 bactéries.
\begin{enumerate}
\item Déterminer le nombre de bactéries ; à 1 heure ; à 2 heures ; à 5 heures.
\item Exprimer le nombre de bactéries à 24 heures.
\item Afin d'afficher le nombre de bactéries à chaque heure, l'un des chercheurs utilise un tableur (voir ci-contre). Quelle formule a-t-il entré dans la cellule B3 afin d'afficher dans la colonne B, par recopie vers le bas, les résultats voulus ?
\item  Calculer les valeurs donnant le nombre de bactéries sur une calculatrice ou un tableur, de 0 heure à 24 heures.
\end{enumerate}
\end{minipage}
\hfill
\begin{minipage}{0.39 \linewidth}

\begin{tabular}{|c|c|c|}
\hline 
\cellcolor{gray} & \cellcolor{gray}Heure & \cellcolor{gray}Nombre de bactéries\\ 
\hline 
\cellcolor{gray} 1& 0  &4\\ 
\hline 
\cellcolor{gray}2 & 1  &12\\ 
\hline 
\cellcolor{gray}3 & 2  &\\ 
\hline 
\cellcolor{gray}4 & 3  &\\ 
\hline 
\cellcolor{gray}5 & 4  &\\ 
\hline 
\cellcolor{gray}6 & 5  &\\ 
\hline 
\cellcolor{gray}7 & 6  &\\ 
\hline 
\cellcolor{gray}8 & 7  &\\ 
\hline 

\end{tabular} 

\end{minipage}