\chapter{Arithmétique}
{https://sacado.xyz/qcm/parcours_show_course/0/117129}
{


 \begin{CpsCol}
\textbf{Les savoir-faire du parcours}
 \begin{itemize}
 \item \textbf{Utiliser des nombres pour calculer et résoudre des problèmes}
 \item Connaitre les bases de l'arithmétique
 \item Simplifier une fraction pour la rendre irréductible
 \end{itemize}
 \end{CpsCol}

\begin{His}

  
L'arithmétique est une branche des mathématiques qui correspond à la science des nombres. De nombreux nombres entiers ont des propriétés particulières. Ces propriétés font l'objet de la théorie des nombres. Parmi ces nombres particuliers, les nombres premiers sont sans doute les plus importants.

On connait aussi les nombres pairs et les nombres impairs. 

\end{His}

 

\begin{ExoDec}{Chercher.}{1234}{1}{0}{0}{0}

Pour fêter les 25 ans de sa boutique, un chocolatier souhaite offrir aux premiers clients de la journée une boîte contenant des truffes au chocolat.
Il a confectionné 300 truffes : 125 truffes parfumées au café et 175 truffes enrobées de noix de coco. Combien y aura-t-il de truffes de chaque sorte dans chaque boîte ?
 
\end{ExoDec}



}

\begin{pageCours}
 
 
\section{Nombre premier}

 
\begin{DefT}{Nombre premier}\index{Nombre premier}

Un \textbf{nombre premier} est un nombre entier supérieur à 2 avec exactement 2 diviseurs : 1 et lui-même. 

\end{DefT}
 

\section{Décomposition en produit de facteurs premiers.}


\begin{DefT}{Décomposition}\index{Décomposition}

Tout nombre se décompose de façon unique en produit de facteurs premiers.

\end{DefT}

 

\begin{Ex} 

La décomposition du nombre $18$ est $2 \times 3 \times 3 = 2 \times 3^2$. On écrit : $18 = 2 \times 3^2$.

\end{Ex}
 
 

 
\section{Diviseurs communs}
 
 
   
\begin{minipage}[t]{0.5\linewidth}
\begin{MtT}{Recherche des diviseurs communs}

Pour déterminer les diviseurs communs de deux nombres, 
\begin{enumerate}[leftmargin=*]
\item On écrit les \textbf{diviseurs} de chaque nombre.
\item On récupère les diviseurs \textbf{communs} aux $2$ nombres.
\end{enumerate}


\end{MtT}
\end{minipage}
\begin{minipage}[t]{0.5\linewidth}
\begin{MtT}{Recherche du pgcd}

Pour déterminer le plus grand diviseur commun de deux nombres, 
\begin{enumerate}[leftmargin=*]
\item Méthode $8$  
\item On garde \textbf{le plus grand} nombre parmi les diviseurs communs.
\end{enumerate}


\end{MtT}
\end{minipage}
  
 

\begin{Ex} 
\begin{minipage}{0.6\linewidth}
On cherche les diviseurs communs de $12$ et de $18$.

$12 = 4 \times 3 = 2^2 \times 3$. Les diviseurs de $12$ sont $1\,;\,2\,;\,3\,;\,4\,;\,6\,;\,12$

$18 = 2 \times 9 = 2 \times 3^2$. Les diviseurs de $18$ sont $1\,;\,2\,;\,3\,;\,6\,;\,9\,;\,18$

\end{minipage}
\begin{minipage}{0.4\linewidth}

Les diviseurs communs sont $1\,;\,2\,;\,3;\,6$
\end{minipage}
\end{Ex} 
 
 
 
\section{Fractions irréductibles}


\begin{DefT}{Fraction irréductible}\index{Fraction irréductible}

Une fraction est dite \textbf{irréductible} lorsque le numérateur et le  dénominateur n'ont pas de diviseur commun autre que $1$. 

\end{DefT}
 

\begin{MtT}{Rendre une fraction irréductible}

Déterminer la fraction $A=\dfrac{1575}{2550}$ irréductible.

\begin{enumerate}[leftmargin=*]
\begin{minipage}{0.6\linewidth}
\item On détermine les diviseurs du numérateur et du 

dénominateur
 
$1575 = 3^2 \times 5^2 \times 7$

$2550 = 2 \times 3 \times 5^2 \times 17$

\item On détermine le pgcd du numérateur et du dénominateur

Le pgcd de $1575$ et $2550$ est $3 \times 5^2=75$

\end{minipage}
\begin{minipage}{0.4\linewidth}


\item On décompose la fraction

$A=\dfrac{1575}{2550}=\dfrac{3 \times 75 \times 7}{2 \times 75 \times 17}$

\item On simplifie la fraction $A=\dfrac{3 \times 7}{2\times 17}=\dfrac{21}{34}$
\end{minipage}
\end{enumerate}
\end{MtT}
 



\end{pageCours} 
\begin{pageAD} 
 
\Sf{Connaitre les nombres premiers}

\begin{ExoCad}{Communiquer.}{1234}{0}{0}{0}{0}{0}
Donner trois nombres premiers plus petits que $40$. 

$$  \cdots\cdots\cdots  \quad-\quad \cdots\cdots\cdots \quad-\quad  \cdots\cdots\cdots $$
\end{ExoCad}


\begin{ExoCad}{Calculer.}{1234}{0}{0}{0}{0}{0}

$51$ est-il un nombre premier ?  
\point{3}
\end{ExoCad}

\Sf{Décomposer un nombre en produit de facteurs premiers.}

\begin{ExoCad}{Calculer.}{1234}{0}{0}{0}{0}{0}

Quel nombre se cache sous cette décomposition en facteurs premiers $2^3 \times 5 \times 7^2  \times 11$ ?  

\point{1}

\end{ExoCad}


\begin{ExoCad}{Calculer.}{1234}{0}{0}{0}{0}{0}

Décomposer les nombres suivants en produit de facteurs premiers.
\begin{enumerate}[leftmargin=*]
\item $6 =$  \point{1}
\item $90 =$  \point{1}
\item $720 =$  \point{1}
\end{enumerate}
\end{ExoCad}



 
\Sf{Déterminer les diviseurs communs}


\begin{ExoCad}{Calculer.}{1234}{0}{0}{0}{0}{0}

Déterminer les diviseurs communs de $50$ et $70$ 

\point{3}

\end{ExoCad}


\Sf{Rendre une fraction irréductible}


\begin{ExoCad}{Calculer.}{1234}{0}{0}{0}{0}{0}

\begin{minipage}{0.48\linewidth}
Simplifie la fraction $\dfrac{735}{840}$ 

\point{6}
\end{minipage}
\hfill
\begin{minipage}{0.48\linewidth}
Simplifie la fraction $\dfrac{135}{315}$ 

\point{6}
\end{minipage}


\end{ExoCad}


 
\end{pageAD}


%%%%%%%%%%%%%%%%%%%%%%%%%%%%%%%%%%%%%%%%%%%%%%%%%%%%%%%%%%%%%%%%%%%
%%%%  Niveau 1
%%%%%%%%%%%%%%%%%%%%%%%%%%%%%%%%%%%%%%%%%%%%%%%%%%%%%%%%%%%%%%%%%%%
\begin{pageParcoursu} 

\begin{ExoCu}{Représenter. Calculer.}{1234}{0}{0}{0}{0}{0}

Marcel a trouvé la décomposition de $180$ et a donné le produit suivant : $180 = 2^2 \times 5 \times 9$. Qu'en penses-tu ?  
\point{2}
\end{ExoCu}
 
%%%%%%%%%%%%%%%%%%%%%%%%%%%
\begin{ExoCu}{Représenter.}{1234}{2}{0}{0}{0}{0}
\begin{enumerate}
\item Décomposer $36$ en produit de facteurs premiers. \point{2}
\item Décomposer $45$ en produit de facteurs premiers. \point{2}
\item Décomposer $126$ en produit de facteurs premiers. \point{2}
\item Décomposer $256$ en produit de facteurs premiers. \point{2}
\end{enumerate}
\end{ExoCu}


 %%%%%%%%%%%%%%%%%%%%%%%%%%%
\begin{ExoCu}{Représenter.}{1234}{2}{0}{0}{0}{0}

Déterminer le pgcd de $252$ et $288$.\point{6}

\end{ExoCu}

 

%%%%%%%%%%%%%%%%%%%%%%%%%%%
\begin{ExoCu}{Raisonner.}{1234}{1}{0}{0}{0}{0}
\begin{enumerate}
\item Simplifier la fraction $\dfrac{126}{168}$. \point{4}
\item Simplifier la fraction $\dfrac{435}{135}$. \point{4}
\item Simplifier la fraction $\dfrac{378}{540}$. \point{4}
\end{enumerate}
\end{ExoCu}





\end{pageParcoursu} 
 
%%%%%%%%%%%%%%%%%%%%%%%%%%%%%%%%%%%%%%%%%%%%%%%%%%%%%%%%%%%%%%%%%%%
%%%%  Niveau 2
%%%%%%%%%%%%%%%%%%%%%%%%%%%%%%%%%%%%%%%%%%%%%%%%%%%%%%%%%%%%%%%%%%%
\begin{pageParcoursd} 
 
\begin{ExoCd}{DNB 2023 - Représenter. Calculer.}{1234}{2}{0}{0}{0}{0}

Un professionnel et un amateur vont faire une séance de karting sur la piste. 
Ils partent partent en même temps de la ligne de départ et font plusieurs tours de circuit.
Le professionnel effectue un tour en $60$ s et l'amateur en $72$ s.

\begin{enumerate}[leftmargin=*]
\item Décomposer $60$ et $72$ en produit de facteurs premiers.\point{3}
\item Au bout de combien de temps se retrouveront-ils pour la première fois sur la ligne de départ ensemble ?\point{3}
\item Combien auront-ils alors effectué de tours chacun ?\point{3}
\end{enumerate}

\end{ExoCd}



\begin{ExoCd}{DNB 2023 - Représenter. Calculer.}{1234}{2}{0}{0}{0}{0}

Pour constituer des lots d'une tombolas, on dispose de $195$ figurines et $234$ autocollants.
Chaque lot sera composé de figurines ainsi que d'autocollants.
Tous les lots sont identiques.
Toutes les figurines et tous les autocollants doivent être utilisés.

\begin{enumerate}[leftmargin=*]
\item Peut-on faire $3$ lots ? \point{2}
\item Décomposer $195$ en produit de facteurs premiers. \point{2}
\item Sachant que la décomposition en produit de facteurs premiers de $234$ est $2 \times 3^2 \times 13$ :
\begin{enumerate}[leftmargin=*]
\item Combien de lots peut-on constituer au maximum ? \point{3}
\item De combien de figurines et d'autocollants sera alors composé chaque lot ? \point{3}
\end{enumerate}
\end{enumerate}
\end{ExoCd}


\begin{ExoCd}{Représenter. Calculer. }{1234}{1}{0}{0}{0}

\begin{minipage}{0.55\linewidth}

Pour déterminer tous les diviseurs d'un nombre, on utilise un arbre de diviseurs. C'est un représentation qui propose tous les calculs possibles avec les facteurs premiers du nombre.

On donne en exemple l'arbre de diviseurs de $36$.

Construire l'arbre des diviseurs de $20$.

\vspace{4cm}

\end{minipage}
\begin{minipage}{0.4\linewidth}

\definecolor{ududff}{rgb}{0.30196078431372547,0.30196078431372547,1.}
\begin{tikzpicture}[line cap=round,line join=round,>=triangle 45,x=1.0cm,y=1.0cm]
\clip(-3.6,-1.) rectangle (3.9,5.42);
\draw (-3.34,2.36) node[anchor=north west] {$36=2^2\times 3^2$};
\draw [line width=1.pt] (-1.,2.)-- (1.,4.);
\draw [line width=1.pt] (-1.,2.)-- (1.,2.);
\draw [line width=1.pt] (-1.,2.)-- (1.,0.);
\draw [line width=1.pt] (1.,2.)-- (2.04,2.78);
\draw [line width=1.pt] (1.,2.)-- (2.,2.);
\draw [line width=1.pt] (1.,2.)-- (1.98,1.24);
\draw [line width=1.pt] (1.,0.)-- (2.02,0.64);
\draw [line width=1.pt] (1.,0.)-- (2.,0.);
\draw [line width=1.pt] (1.,0.)-- (2.02,-0.74);
\draw [line width=1.pt] (1.,4.)-- (1.96,4.66);
\draw [line width=1.pt] (1.,4.)-- (2.,4.);
\draw [line width=1.pt] (1.,4.)-- (2.06,3.5);
\begin{scriptsize}
\draw[color=ududff] (2.9,2.91) node {$2^1 \times 3^0=2$};
\draw[color=ududff] (2.9,2.09) node {$2^1 \times 3^1=6$};
\draw[color=ududff] (2.9,1.31) node {$2^1 \times 3^2=18$};
\draw[color=ududff] (2.9,0.77) node {$2^2 \times 3^0=4$};
\draw[color=ududff] (2.9,0.07) node {$2^2 \times 3^1=12$};
\draw[color=ududff] (2.9,-0.63) node {$2^2 \times 3^2=36$};
\draw[color=ududff] (2.9,4.73) node {$2^0 \times 3^0=1$};
\draw[color=ududff] (2.9,4.09) node {$2^0 \times 3^1=3$}; 

\draw[color=black] (0.16,3.37) node {$2^0$};
\draw[color=black] (0.16,2.35) node {$2^1$};
\draw[color=black] (0.16,1.23) node {$2^2$};

\draw[color=black] (1.56,4.5) node {$3^0$};
\draw[color=black] (1.56,4.15) node {$3^1$};
\draw[color=black] (1.56,3.55) node {$3^2$};

\draw[color=black] (1.56,2.61) node {$3^0$};
\draw[color=black] (1.56,2.2) node {$3^1$};
\draw[color=black] (1.56,1.35) node {$3^2$};

\draw[color=black] (1.56,0.63) node {$3^0$};
\draw[color=black] (1.56,0.15) node {$3^1$};
\draw[color=black] (1.56,-0.63) node {$3^2$};

\end{scriptsize}
\end{tikzpicture}
\end{minipage}

\end{ExoCd}





\end{pageParcoursd}
 
%
%%%%%%%%%%%%%%%%%%%%%%%%%%%%%%%%%%%%%%%%%%%%%%%%%%%%%%%%%%%%%%%%%%%%
%%%%%  Niveau 3
%%%%%%%%%%%%%%%%%%%%%%%%%%%%%%%%%%%%%%%%%%%%%%%%%%%%%%%%%%%%%%%%%%%%
\begin{pageParcourst}



 
\begin{ExoCt}{Chercher. Raisonner.}{1234}{1}{0}{0}{0}{0}

Pour déterminer le PGCD de deux nombres $a$ et $b$, $a>b$, on effectue la division euclidienne de $a$ par $b$. On appelle $r_0$ le reste. \\
Puis on divise $b$ par $r_0$ et on appelle $r_1$ le reste. \\
On divise alors $r_0$ par $r_1$ et on appelle $r_2$ le reste.\\ 
On divise alors $r_1$ par $r_2$ et on appelle $r_3$ le reste. Et ainsi de suite. \\ 
Le PGCD de $a$ et de $b$ est alors le dernier reste non nul.
On appelle ce procédé, la méthode par divisions successives\index{Divisions successives!PGCD}.

\begin{enumerate}[leftmargin=*]

\item Déterminer à l'aide de ce procédé le PGCD de $2 622$ et de $2 530$. \vspace{4cm}
\item Imaginer un algorithme qui détermine le pgcd selon la méthode des divisions successives.\point{6}
 
\end{enumerate}
 
\end{ExoCt}


 


%%%%%%%%%%%%%%%%%%%%%%%%%%%%%%%%%%%%%%%%%%%%%%%%%%%%%%%%%%%%%%%%%%%
\begin{ExoCt}{Chercher.}{1234}{2}{0}{0}{0}{0} 
Je suis un nombre à trois chiffres non nuls. Je suis divisible par 94. Changez l'ordre de mes chiffres et je deviens divisible par 49.
Qui suis-je ?   \point{5}
\end{ExoCt}


 %%%%%%%%%%%%%%%%%%%%%%%%%%%%%%%%%%%%%%%%%%%%%%%%%%%%%%%%%%%%%%%%%%%
\begin{ExoCt}{Raisonner.}{1234}{2}{1}{0}{0}{0}
 
 \begin{minipage}{0.5\linewidth} 
 
\textbf{Le crible d'Eratosthène}

L'algorithme procède par élimination : il s'agit de supprimer d'une table d'entiers tous les multiples d'un entier $n$ (autres que lui-même).

En supprimant tous ces multiples, à la fin il ne restera que les entiers qui ne sont multiples d'aucun entier à part 1 et eux-mêmes, et qui sont donc les nombres premiers.

On commence par rayer les multiples de 2, puis les multiples de 3 restants, puis les multiples de 5 restants, et ainsi de suite en rayant à chaque fois tous les multiples du plus petit entier restant.

\begin{enumerate}

\item Faire fonctionner le crible sur la table ci-contre.
\item Écrire un algorithme de ce crible. \point{5}

\end{enumerate}

\end{minipage}
\begin{minipage}{0.5\linewidth}

 


\begin{tabular}{|c|c|c|c|c|c|c|c|c|c|}
 \hline 
 &  & 2 & 3 & 4 & 5 & 6 & 7 & 8 & 9 \\ 
 \hline 
 10&11 & 12 & 13 & 14 & 15 & 16 & 17 & 18 & 19 \\
 \hline 
 20&21 & 22 & 23 & 24 & 25 & 26 & 27 & 28 & 29 \\
 \hline 
 30&31 & 32 & 33 & 34 & 35 & 36 & 37 & 38 & 39 \\
 \hline 
 40&41 & 42 & 43 & 44 & 45 & 46 & 47 & 48 & 49 \\
 \hline 
 50&51 & 52 & 53 & 54 & 55 & 56 & 57 & 58 & 59 \\
 \hline 
 60&61 & 62 & 63 & 64 & 65 & 66 & 67 & 68 & 69 \\
 \hline 
 70&71 & 72 & 73 & 74 & 75 & 76 & 77 & 78 & 79 \\
 \hline 
 80&81 & 82 & 83 & 84 & 85 & 86 & 87 & 88 & 89 \\
 \hline 
 90&91 & 92 & 93 & 94 & 95 & 96 & 97 & 98 & 99 \\
 \hline 
 \end{tabular}  
 
 \end{minipage}


 
\end{ExoCt}
 
\end{pageParcourst}
%
%%%%%%%%%%%%%%%%%%%%%%%%%%%%%%%%%%%%%%%%%%%%%%%%%%%%%%%%%%%%%%%%%%%%
%%%%%  Brouillon
%%%%%%%%%%%%%%%%%%%%%%%%%%%%%%%%%%%%%%%%%%%%%%%%%%%%%%%%%%%%%%%%%%%%


%%%%%%%%%%%%%%%%%%%%%%%%%%%%%%%%%%%%%%%%%%%%%%%%%%%%%%%%%%%%%%%%%%%
%%%%  Auto
%%%%%%%%%%%%%%%%%%%%%%%%%%%%%%%%%%%%%%%%%%%%%%%%%%%%%%%%%%%%%%%%%%%


%%%%%%%%%%%%%%%%%%%%%%%%%%%%%%%%%%%%%%%%%%%%%%%%%%%%%%%%%%%%%%%%%%%
\begin{pageAuto} 




%%%%%%%%%%%%%%%%%%%%%%%%%%%%%%%%%%%%%%%%%%%%%%%%%%%%%%%%%%%%%%%%%%%
\begin{ExoAuto}{Raisonner.}{1234}{2}{0}{0}{0}{0}

\begin{enumerate}
\item Décomposer $186$ et $155$ en produit de facteurs premiers. \point{3}
\item Déterminer le PGCD de $186$ et $155$. \point{3}
\item Un chocolatier a fabriqué $186$ pralines et $155$ chocolats.
Les colis sont constitués ainsi :
\begin{description}[leftmargin=*]
\item Le nombre de pralines est le même dans chaque colis.
\item Le nombre de chocolats est le même dans chaque colis.
\item Tous les chocolats et toutes les pralines sont utilisés.
\end{description}
\begin{enumerate}
\item Quel nombre maximal de colis pourra-t-il réaliser ?  \point{3}
\item Combien y aura-t-il de chocolats et de pralines dans chaque colis.   \point{3}
\end{enumerate}
\end{enumerate}

\end{ExoAuto}


%%%%%%%%%%%%%%%%%%%%%%%%%%%%%%%%%%%%%%%%%%%%%%%%%%%%%%%%%%%%%%%%%%%
\begin{ExoAuto}{DNB Métropole 2022 - Raisonner.}{1234}{2}{0}{0}{0}{0}

Une collectionneuse compte ses cartes Pokémon afin de les revendre. Elle possède 252 cartes de type \og feu \fg{} et 156 cartes de type \og terre \fg.
\begin{enumerate}
\item  \begin{enumerate}
\item  Parmi les trois propositions suivantes, laquelle correspond à la décomposition en produit de facteurs premiers du nombre $252$ :

\begin{center}
\begin{tabularx}{9cm}{|*{3}{>{\centering \arraybackslash}X|}}\hline
Proposition 1 &Proposition 2& Proposition 3\\
 $2^2\times 9\times 7$ &$2\times 2\times 3\times 21$ &$2^2 \times 3^2\times  7$\\\hline
\end{tabularx}
\end{center}
\item Donner la décomposition en produit de facteurs premiers du nombre $156$.\point{3}
\end{enumerate}
\item Elle veut réaliser des paquets identiques, c'est-à-dire contenant chacun le même nombre de cartes \og terre \fg{} et le même nombre de cartes \og feu \fg{} en utilisant toutes ses cartes.
\begin{enumerate}
\item Peut-elle faire $36$ paquets ? \point{3}
\item Quel est le nombre maximum de paquets qu'elle peut réaliser ? \point{3}
\item Combien de cartes de chaque type contient alors chaque paquet ? \point{3}
\end{enumerate}
\end{enumerate}


\end{ExoAuto}




\end{pageAuto}
