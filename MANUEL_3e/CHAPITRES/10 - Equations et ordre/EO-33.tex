
\begin{enumerate}
\item Voici un programme de calcul:

\begin{center}
\begin{tabular}{|l|}\hline
\multicolumn{1}{|c|}{\textbf{Programme A}}\\
$\bullet~~$Choisir un nombre.\\
$\bullet~~$Ajouter 3.\\
$\bullet~~$Calculer le carré du résultat obtenu.\\
$\bullet~~$Soustraire le carré du nombre de départ.\\ \hline
\end{tabular}
\end{center}

	\begin{enumerate}
		\item Eugénie choisit 4 comme nombre de départ. Vérifier qu'elle obtient 33 comme résultat du programme.
		\item Elle choisit ensuite $- 5$ comme nombre de départ. Quel résultat obtient-elle ?
	\end{enumerate}
\item Voici un deuxième programme de calcul:

\begin{center}
\begin{tabular}{|l|}\hline
\multicolumn{1}{|c|}{\textbf{Programme B}}\\
$\bullet~~$Choisir un nombre.\\
$\bullet~~$Multiplier par 6.\\
$\bullet~~$Ajouter 9 au résultat obtenu.\\ \hline
\end{tabular}
\end{center}

Clément affirme: \og Si on choisit n'importe quel nombre et qu'on lui applique les deux programmes, on
obtient le même résultat. \fg
Prouver que Clément a raison.
\item Quel nombre de départ faut-il choisir pour que le résultat des programmes soit $54$ ?
\end{enumerate}
 
