
Vanessa a acheté 4 classeurs à 3 \euro{} et 5 cahiers pour un total de 27,75 €. On appelle $x$ le prix d'un cahier.
\begin{enumerate}
\item Exprimer la somme totale payée en fonction de $x$.
\item A l'aide d'un tableur, on souhaite conjecturer le prix d'un cahier.
\begin{enumerate}
\item Reproduire ce tableau sur une feuille de calcul.

\begin{tabular}{|c|c|c|}
\hline 
&A & B \\ 
\hline 
1&Prix d'un cahier & Prix total \\ 
\hline 
2&1 & =12+5*A2 \\ 
\hline 
3&& \\ 
\hline 
\end{tabular} 
\item Que représente la formule =12+5*A2 ?
\item Sélectionner les cases A2 et B2 puis en utilisant la croix, glisse pour copier jusqu'à la ligne 10.
\item Déterminer un encadrement à l'unité du prix d'un cahier.
\item Modifier les valeurs de la colonne A pour obtenir une meilleure approximation de la solution.
\item Peux-tu déterminer le prix exact d'un cahier ?
\end{enumerate}
\item Quel est le problème de cette méthode ? 
\end{enumerate}