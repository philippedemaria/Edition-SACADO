
Soient les fonctions $f$, $g$ et $h$ définies par :

\[f(x) = 6x \qquad g(x) = 3x^2 - 9x - 7\qquad \text{et} \quad  h(x) = 5x - 7.\]

À l'aide d'un tableur, Pauline a construit un tableau de valeurs de ces fonctions.

Elle a étiré vers la droite les formules qu'elle avait saisies dans les cellules B2, B3 et B4.

\begin{center}
\begin{tabularx}{\linewidth}{|c|m{2.75cm}|*{7}{>{\centering \arraybackslash}X|}}\hline
\multicolumn{2}{|c|}{B3}&\multicolumn{7}{l|}{$=3*\text{B}1*\text{B}1-9*\text{B}1-7$} \\ \hline
	&A						&B		&C		&D		&E		&F		&G		&H\\ \hline
1	&$x$					&$-3$	&$-2$	&$-1$	&0		&1		&2		&3\\ \hline
2	&$f(x) = 6x$			&$-18$	&$-12$	&$-6$	&0		&6		&12		& 18\\ \hline
3	&$g(x) = 3x^2 - 9x - 7$	&47 	&23 	&5 		&$-7$ 	&$- 13$	& $-13$	& $-7$\\ \hline
4	&$h(x) = 5x - 7$			&$-22$ 	&$-17$ 	&$-12$ 	&$-7$ 	&$-2$ 	&3 		&8\\ \hline
\end{tabularx}
\end{center}

\medskip

\begin{enumerate}
\item Utiliser le tableur pour déterminer la valeur de $h(-2)$.
\item Écrire les calculs montrant que : $g(- 3) = 47$.
\item Faire une phrase avec le mot \og antécédent\fg{} ou le mot \og image \fg{} pour traduire
l'égalité $g(- 3) = 47$.
\item Quelle formule Pauline a-t-elle saisie dans la cellule B4 ?
\item  
	\begin{enumerate}
		\item Déduire du tableau ci-dessus une solution de l'équation ci-dessous :

\[3x^2 - 9x - 7 = 5x - 7.\]

		\item Cette équation a-t-elle une autre solution que celle trouvée grâce au tableur ?

Justifier la réponse.

\emph{Dans cette question, toute trace de recherche, même inaboutie sera prise en compte
et valorisée.}
	\end{enumerate}
\end{enumerate}