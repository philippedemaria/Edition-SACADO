\documentclass[10pt]{article}

\input{../preambule}
\input{../styles}
\input{../bas_de_page_quatrieme} 
 
%%%%%%%%%%%   Marges de pages  %%%%%%%%%%%%%%%% 
 \usepackage{geometry}
 \geometry{top=2cm, bottom=0cm, left=2cm , right=2cm}
%%%%%%%%%%%%%%%%%%%%%%%%%%%%%%%%%%%%%%%%%%%%%%%

%%%%%%%%%%%%%%%  Indentation  %%%%%%%%%%%%%%%%%%
\parindent=0pt
%%%%%%%%%%%%%%%%%%%%%%%%%%%%%%%%%%%%%%%%%%%%%%%%


\begin{document}

%%%%%%%%%%%%%%%%%%%%%%%%%%%%%%%%%%%%%%%%%%%%%%%
%%%%		 Titre encadré
%%%%%%%%%%%%%%%%%%%%%%%%%%%%%%%%%%%%%%%%%%%%%%%
\begin{encadrementombre} {Thème 6 Calcul littéral}
{\LARGE DTL13: Equations}\\

%% Laisse la ligne vide ci-dessus
{\Large Calcul littéral}
\end{encadrementombre}


%%%%%%%%%%%%%%%%%%%%%%%%%%%%%%%%%%%%%%%%%%%%%%%
%%%%		 Corps du document
%%%%%%%%%%%%%%%%%%%%%%%%%%%%%%%%%%%%%%%%%%%%%%%

%%%%%%%%%%%%%%%%%%%%%%%%%%%%%%%%%%%%%%%%%%%%%%%
%%%% \renewcommand{\arraystretch}{1.8}
\definecolor{shadecolor}{gray}{0.9}
%%%%%%%%%%%%%%%%%%%%%%%%%%%%%%%%%%%%%%%%%%%%%%%
%%%%%%%%%%%   Hauteur de ligne  %%%%%%%%%%%%%%%%
{\setlength{\baselineskip}{1.5\baselineskip}
%%%%%%%%%%%%%%%%%%%%%%%%%%%%%%%%%%%%%%%%%%%%%%%%
\begin{multicols}{2}
\begin{exo} \textbf{1}(voir sesamath)\\
\includegraphics[scale=0.3]{DTL13_equations_cor.eps}  
\end{exo}


\begin{shaded}
\begin{exo} \textbf{2}(voir sesamath)\\
Soit $s$ le nombre de jour de soleil, c'est-à-dire la nombre de jours de vacances où il a fait soleil toute la journée. On cherche le nombre de vacances $j$ tel que $j=s+11$.\\
Appelons maintenant $p$ le nombre de matinée de pluie.\\ D'après l'énoncé, on sait que le nombre d'après-midi de pluie est $11-p$.\\
Par ailleurs, comme il y eu $12$ après-midi sans pluie, et que le nombre d'après-midi sans pluie est la somme du nombre de jours où il a plus le matin ($p$) et le nombre de jour de soleil ($s$), on déduit que $12=p+s$.\\
De même, comme  il y eu $9$ matinée sans pluie, et que le nombre de matinée sans pluie est la somme du nombre de jours où il a plus l'après-midi ($11-p$) et le nombre de jour de soleil ($s$), on déduit que $9=11-p+s$.\\ 
\vspace{0.1cm}\\
Comme $12=p+s$, on déduit que $s=12-p$ , et comme $9=11-p+s$, on déduit que $9=11-p+12-p$, soit en réduisant:$9=23-2p$, donc $9-23=23-2p-23$, soit $-14=-2p$, donc $\frac{-14}{-2}=\frac{-2p}{-2}$, soit finalement $p=7$.\\
Comme $s=12-p$, on déduit que $s=12-7=5$ et donc $j=5+11=16$.\\
Ce garçon a eu $16$ jours de vacances, dont $5$ de beau temps toute la journée. 
\end{exo}
\end{shaded}
\end{multicols}

\end{document}