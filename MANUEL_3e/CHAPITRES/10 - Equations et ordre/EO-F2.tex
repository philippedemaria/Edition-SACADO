\begin{titre}[Équations et ordre]

\Titre{Équation du premier degré}{2,5}
\end{titre}


\begin{CpsCol}
\textbf{Utiliser le calcul littéral}
\begin{description}
\item[$\square$] Résoudre de façon exacte ou approchée des problèmes du premier degré
\end{description}
\end{CpsCol}




\begin{DefT}{Équation du premier degré}
On appelle \textbf{équation du premier degré}\index{Équation du premier degré} toute équation de la forme $ax+b=0$ ou qui peut s'y ramener, avec $a$ et $b$ deux réels. $x$ est appelé l'\textbf{inconnue} \index{Inconnue}.
\end{DefT}

  
\begin{Ex}   
\begin{description}
\item L'équation $2x+5=0$ est une équation du premier degré d'inconnue $x$.
\item L'équation $x^2+5x+4=0$ n'est pas une équation du premier degré. L'inconnue est $x$.
\end{description}
\end{Ex}


\begin{DefT}{Résoudre une équation}
\textbf{Résoudre une équation} \index{Équation!Résoudre}, c'est déterminer toutes les valeurs de l'inconnue pour lesquelles les deux membres de l'équation sont égaux.
\end{DefT}


\begin{Mt}   
Pour déterminer une valeur exacte de la solution, si elle existe, il faut isoler l'inconnue.

On souhaite résoudre l'équation $4x + 5 = 0$
\begin{description}
\item[1a. Isoler $4x$.] En soustrayant 5 à chaque membre, on obtient : $4x + 5 -5 = 0 - 5$ 
\item[1b. Simplification.]  $4x = - 5$ 
\item[2a. Isoler.]  En divisant par $4 \neq 0$ chaque membre, on obtient : $\frac{4}{4}x = \frac{-5}{4}$ 
\item[2b. Simplification.]  $\frac{4}{4}=1$, on obtient : $x = \frac{-5}{4}$ 
\item[3. Écriture de la solution] la solution est donc $ \frac{-5}{4}$. On peut écrire : $S=\left\lbrace  \frac{-5}{4} \right\rbrace $.
\end{description}
\end{Mt}

\begin{minipage}{0.49\linewidth}


\AD{1}{EO-43}
\end{minipage}
\begin{minipage}{0.49\linewidth}


\Exo{1}{EO-9}
\end{minipage}






\begin{minipage}{0.49\linewidth}


\Exo{1}{EO-10}

\Exo{1}{EO-8}


\App{1}{EO-36}

\App{1}{EO-2}


\end{minipage}
\hfill
\begin{minipage}{0.49\linewidth}
\PO{1}{EO-39}

\PO{1}{EO-36}

\PO{1}{EO-11}

\PO{1}{EO-40}

\PO{1}{EO-41}

\PO{1}{EO-42}

\PO{1}{EO-44}
\end{minipage}



%\begin{autoeval}
%\begin{tabular}{p{12cm}p{0.5cm}p{0.5cm}p{0.5cm}p{1cm}}
%\textbf{Compétences visées} &  M I & MF & MS  & TBM \vcomp \\ 
%Résoudre de façon exacte ou approchée des problèmes du premier degré & $\square$ & $\square$  & $\square$ & $\square$ \vcomp \\ 
%
%\end{tabular}
%{\footnotesize MI : maitrise insuffisante ; MF = Maitrise fragile ; MS = Maitrise satisfaisante ; TBM = Très bonne maitrise}
% 
%\end{autoeval}