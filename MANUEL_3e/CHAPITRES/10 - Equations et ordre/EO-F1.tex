\begin{titre}[Équations et ordre]

\Titre{Équation et inconnue}{1}
\end{titre}

\impress{1}{
\begin{CpsCol}
\textbf{Utiliser le calcul littéral}
\begin{description}
\item[$\square$] Résoudre de façon exacte ou approchée des problèmes du premier degré
\end{description}
\end{CpsCol}
}

\Rec{1}{EO-16}

\impress{1}{
\begin{DefT}{Équation}
On appelle \textbf{équation}\index{Équation} une proposition mathématique  dans laquelle intervient un ou plusieurs nombres inconnus, appelés "inconnues de l'équation" ou plus simplement "inconnues"\index{Inconnue}.

Les deux expressions de part et d'autre du signe $=$ sont appelées les \index{Membre} \textbf{membres} de l'équation.
\end{DefT}

  
\begin{Ex}   
\begin{description}[leftmargin=*]
\item L'équation $2x+5=0$ est une équation du premier degré d'inconnue $x$. Le membre de gauche est $2x+5$, celui de droite est $0$.
\item L'équation $x^2+5x=-4$ n'est pas une équation du premier degré. L'inconnue est $x$. Les deux membres sont : $x^2+5x$ et $-4$.  
\end{description}
\end{Ex}

\begin{DefT}{Résoudre une équation}
\textbf{Résoudre une équation} \index{Résoudre une équation}c'est déterminer toutes les valeurs de l'inconnue pour lesquelles les deux membres de l'équation sont égaux.
\end{DefT}

\begin{Mt}   
On souhaite résoudre l'équation $x + 7 = 4$
\begin{description}
\item[1a. Isoler l'inconnue.] En soustrayant 7 à chaque membre, on obtient : $x + 7 - 7 = 4 - 7$ 
\item[1b. Simplifier.]  $x +0 = - 3$ 
\item[1c. Réduire.]  $x = - 3$  
\item[2. Conclure] La solution est donc $-3$. On peut écrire : $S=\left\lbrace -3 \right\rbrace $.
\end{description}
\end{Mt}
}


\AD{1}{EO-14}

\AD{1}{EO-15}



\begin{minipage}{0.49\linewidth}
\AD{1}{EO-20}
\end{minipage}
\begin{minipage}{0.49\linewidth}
\AD{1}{EO-24}
\end{minipage}


\Exo{1}{EO-17}

\Exo{1}{EO-18}

\Exo{1}{EO-19}

\PO{1}{EO-23}

\impress{0}{

\vspace{2cm}

{\Large {\color{violet}Aucun exercice sur ce chapitre dans le cahier}}



\begin{autoeval}
\begin{tabular}{p{12cm}p{0.5cm}p{0.5cm}p{0.5cm}p{1cm}}
\textbf{Compétences visées} &  M I & MF & MS  & TBM \vcomp \\ 
Résoudre de façon exacte ou approchée des problèmes du premier degré & $\square$ & $\square$  & $\square$ & $\square$ \vcomp \\ 
\end{tabular}
{\footnotesize MI : maitrise insuffisante ; MF = Maitrise fragile ; MS = Maitrise satisfaisante ; TBM = Très bonne maitrise}
\end{autoeval}
}