\begin{titre}[Équations et ordre]

\Titre{Équation produit nul}{3}
\end{titre}


\begin{CpsCol}
\textbf{Utiliser le calcul littéral}
\begin{description}
\item[$\square$] Résoudre une équation produit
\item[$\square$] Faire le lien entre forme algébrique et représentation graphique
\end{description}
\end{CpsCol}


\begin{DefT}{Équation produit nul}
On appelle \textbf{équation produit nul}\index{Équation produit nul} une équation dont l'un des membres est un produit de facteurs et l'autre est 0. 
\end{DefT}


\begin{Mt}[. Résolution d'une équation produit nul]\index{Équation produit nul!Résolution}

\begin{tabular}{ccc}
\multicolumn{3}{c}{$(2x-1)(x+3)=0$} \\ 
\multicolumn{3}{c}{un produit est nul lorsqu'au moins un de ses facteurs est nul. }\\  
$2x-1=0$ & ou & $x+3=0$ \\  
$2x=1$ & ou & $x=-3$ \\ 
$x=\frac{1}{2}$ &  &  \\ 
\multicolumn{3}{c}{$S=\left\lbrace \frac{1}{2}; -3 \right\rbrace $}\\  
\end{tabular} 
\end{Mt}


\AD{1}{EO-45}

\mini{
\Exo{1}{EO-51}

}{
\App{1}{EO-49}

\App{1}{EO-50}
}
