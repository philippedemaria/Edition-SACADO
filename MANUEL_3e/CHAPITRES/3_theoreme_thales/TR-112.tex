
Les points O, A, C ainsi que les points O, B, D étant  alignés dans
le même ordre,\\
si $\frac{OB}{OD}=\frac{OA}{OC}$
alors les droites (AB) et (CD) sont parallèles.

%\begin{pspicture*}(0,0)(6,4.2)
%\psline(3,7.275)(6.65,-1.725)%d= droite (CD)
%\psline(-0.15,7.275)(3.525,-1.725)%@9= droite parallele a d passant par A
%\psline(0.65,0.6)(13.325,1.25)%@3= demi-droite [OD)
%\psline(0.65,0.6)(8.125,7.275)%@5= demi-droite [OC)
%\uput{-3pt}[u](0.425,0.525){O}
%\uput{-3pt}[u](5.45,0.675){D}
%\uput{-3pt}[u](4.6,3.85){C}
%\uput{-3pt}[u](2.275,1.8){A}
%\uput{-3pt}[u](2.35,0.51){B}
%\end{pspicture*}