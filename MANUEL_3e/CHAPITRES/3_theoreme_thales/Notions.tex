\documentclass[10pt]{article}

\input{../../../latex_preambule_style/preambule}
\input{../../../latex_preambule_style/styleCoursCycle4}
\input{../../../latex_preambule_style/styleExercices}
\input{../../../latex_preambule_style/styleExercicesAideCompetences}
%\input{../../latex_preambule_style/styleCahier}
\input{../../../latex_preambule_style/bas_de_page_cycle4}
\input{../../../latex_preambule_style/algobox}
%%%%%%%%%%%   Marges de pages  %%%%%%%%%%%%%%%% 
 \usepackage{geometry}
 \geometry{top=2cm, bottom=1cm, left=2cm , right=2cm}
%%%%%%%%%%%%%%%%%%%%%%%%%%%%%%%%%%%%%%%%%%%%%%%

%%%%%%%%%%%%%%%  Indentation  %%%%%%%%%%%%%%%%%%
\parindent=0pt
%%%%%%%%%%%%%%%%%%%%%%%%%%%%%%%%%%%%%%%%%%%%%%%%


\begin{document}

%%%%%%%%%%%%%%%%%%%%%%%%%%%%%%%%%%%%%%%%%%%%%%%
%%%%		 Titre encadré
%%%%%%%%%%%%%%%%%%%%%%%%%%%%%%%%%%%%%%%%%%%%%%%


%%%%%%%%%%%%%%%%%%%%%%%%%%%%%%%%%%%%%%%%%%%%%%%
%%%%		 Corps du document
%%%%%%%%%%%%%%%%%%%%%%%%%%%%%%%%%%%%%%%%%%%%%%%

%%%%%%%%%%%%%%%%%%%%%%%%%%%%%%%%%%%%%%%%%%%%%%%
%%%% \renewcommand{\arraystretch}{1.8}
%\definecolor{shadecolor}{gray}{0.9}
%%%%%%%%%%%%%%%%%%%%%%%%%%%%%%%%%%%%%%%%%%%%%%%

\Exe
%%%%%%%%%%%   Hauteur de ligne  %%%%%%%%%%%%%%%%
{\setlength{\baselineskip}{1.5\baselineskip}
%%%%%%%%%%%%%%%%%%%%%%%%%%%%%%%%%%%%%%%%%%%%%%%%


%%%%%%%%%%%%%%%%%%%%%%%%%%%%%%%%%%%%%%%%%%%%%%%%%%%%%%%
%     Notions
%%%%%%%%%%%%%%%%%%%%%%%%%%%%%%%%%%%%%%%%%%%%%%%%%%%%%%%
\emph{Pour illustrer l'exercice, la figure ci-dessous a été faite à main levée.}

\begin{center}
\psset{unit=1cm}
\begin{pspicture}(8,5)
%\psgrid
\pslineByHand(0.5,2.5)(7.5,4)
\pslineByHand(2,0.5)(6,4.8)
\pslineByHand(2,0.5)(0.5,2.5)
\pslineByHand(6,4.8)(7.5,4)
\pslineByHand(3,3)(4,2.6)
\uput[ul](4.75,3.4){A}\uput[r](7.5,4){B}\uput[u](6,4.8){C}
\uput[l](0.5,2.5){D} \uput[d](2,0.5){E}\uput[ul](3,3){F}\uput[dr](3.8,2.6){G}
\rput{-50}(1.2,1.2){8,1 cm}\rput{-27}(3.3,2.6){3 cm}
\rput{46}(3.2,1.4){6,8 cm}\rput{45}(4.4,2.78){4 cm}
\rput{15}(3.6,3.4){5 cm}\rput{45}(5.2,4.3){5 cm}
\rput{15}(6,3.4){6,25 cm}
\end{pspicture}
\end{center}

Les points D, F, A et B sont alignés, ainsi que les points E, G, A et C.

De plus, les droites (DE) et (FG) sont parallèles.

\medskip

\begin{enumerate}
\item Montrer que le triangle AFG est un triangle rectangle.
\item  Calculer la longueur du segment [AD]. En déduire la longueur du segment [FD].
\item  Les droites (FG) et (BC) sont-elles parallèles ? Justifier.
\end{enumerate}


\Exe

Voici deux programmes de calcul :

\medskip


\begin{tabularx}{\linewidth}{|X|p{0.4cm}|X|}\cline{1-1}\cline{3-3}
\multicolumn{1}{|c|}{\textbf{Programme A}}&~&\multicolumn{1}{|c|}{\textbf{Programme B}}\\
Choisir un nombre de départ&&Choisir un nombre de départ\\
Multiplier ce nombre par - 3&&Multiplier ce nombre par 2\\
Soustraire 12 au résultat&&Ajouter 5 au résultat\\
&&Multiplier le tout par 3\\
Écrire le résultat.&&Écrire le résultat.\\\cline{1-1}\cline{3-3} 
\end{tabularx}

\medskip

\begin{enumerate}
\item On choisit $- 8$ comme nombre de départ.
	\begin{enumerate}
		\item Prouver par le calcul que le résultat obtenu avec le programme A est $12$.
		\item Calculer le résultat final avec le programme B.
	\end{enumerate}
\item Sandro affirme : \og Si on choisit le même nombre de départ pour les deux programmes, le résultat du programme A est toujours supérieur à celui du programme B. \fg
	
Prouver qu'il se trompe.
\item Anne affirme: \og Avec le programme B j'ai trouvé un résultat égal à mon nombre de départ \fg.
	
Quel était son nombre de départ ?
\end{enumerate}



\end{document}