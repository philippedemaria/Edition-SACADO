
En France, il existe deux types d'imposition : Le réel et le forfait. On ne considère que le type d'imposition au forfait.

\medskip
Voici le barème de l'impôt sur le revenu des personnes physiques en France en 2013. L'impôt est dit par tranches.
\medskip

\begin{minipage}{10cm}
\begin{enumerate}
\item Que représente la variable $x$ ?
\item Déterminer le montant de l'impôt d'une personne physique dont le salaire est égal à \np{1430,22} euros brut par mois.
\item Déterminer le montant de l'impôt d'une personne physique dont le revenu annuel est \np{56423} euros annuel.
\end{enumerate}
\end{minipage}
\begin{minipage}{7cm}

\begin{flushright}
\begin{tabular}{|l|c|c|} 
\hline
Quotient familial annuel & Impôt à payer \\
\hline
Jusqu'à 5963 euros & $f(x)=0$  \\
\hline
De 5964 à 11896 euros & $f(x)=0,055x$   \\
\hline
De 11897 à 26420 euros & $f(x)=0,14x$   \\
\hline
De 26421 à 70830 euros & $f(x)=0,3x$    \\
\hline
De 70831 à 150000 euros & $f(x)=0,41x$ \\
\hline
De 150001 à 250000 euros & $f(x)=0,45x$  \\
\hline
Au dessus de 250000 & $f(x)=0,42x$   \\
\hline
\end{tabular}
\end{flushright}

\end{minipage}

