
Le kelvin (symbole K, du nom de William Thomson, Lord Kelvin) est l'unité de température thermodynamique. Le kelvin, n'étant pas une mesure relative, n'est jamais précédé du mot "degré" ni du symbole "$^\text{o}$", contrairement aux degrés Celsius ou Fahrenheit. 

L'échelle des températures Celsius est, par définition, la température absolue décalée en origine de $273,15$ K. 

La fonction de conversion est $f(x)=x + 273,15$, où $x$ est en degré Celsius et $f(x)$ exprimée en kelvin.  

\begin{enumerate}
\item Quelle est la température, en degré Celsius, du zéro absolue de $f(x)$ ?
\item $300$ K est une température plutôt chaude ou froide ? justifier.
\item A Bogotá, l'eau entre en ébullition à $93^\text{o} C$. Exprime en kelvin la température de l'eau.
\end{enumerate}

