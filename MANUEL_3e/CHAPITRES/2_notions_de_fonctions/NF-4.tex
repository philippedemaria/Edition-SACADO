
\begin{minipage}{0.49\linewidth}

On propose la figure suivante.

\definecolor{qqqqff}{rgb}{0.,0.,1.}
\definecolor{ffffff}{rgb}{1.,1.,1.}
\definecolor{zzttqq}{rgb}{0.6,0.2,0.}
\begin{tikzpicture}[line cap=round,line join=round,>=triangle 45,x=1.0cm,y=1.0cm]
\clip(3.4041911842591923,2.866575798081113) rectangle (12.252666255261191,9.209615907843705);
\fill[color=zzttqq,fill=zzttqq,fill opacity=0.25] (5.,8.) -- (5.,4.) -- (10.,4.) -- (10.,8.) -- cycle;
\draw [color=zzttqq] (5.,8.)-- (5.,4.);
\draw [color=zzttqq] (5.,4.)-- (10.,4.);
\draw [shift={(10.,5.5)},color=zzttqq,fill=zzttqq,fill opacity=0.25]  plot[domain=-1.5707963267948966:1.5707963267948966,variable=\t]({1.*1.5*cos(\t r)+0.*1.5*sin(\t r)},{0.*1.5*cos(\t r)+1.*1.5*sin(\t r)});
\draw [shift={(7.498580889309367,8.)},color=zzttqq]  plot[domain=3.141592653589793:6.283185307179586,variable=\t]({1.*1.498580889309367*cos(\t r)+0.*1.498580889309367*sin(\t r)},{0.*1.498580889309367*cos(\t r)+1.*1.498580889309367*sin(\t r)});
\draw [shift={(7.5,8.)},color=ffffff,fill=ffffff,fill opacity=1.0]  plot[domain=3.141592653589793:6.283185307179586,variable=\t]({1.*1.5*cos(\t r)+0.*1.5*sin(\t r)},{0.*1.5*cos(\t r)+1.*1.5*sin(\t r)});
\draw (5.0005655527389345,8.376347904148352)-- (5.958390173826779,8.376347904148352);
\draw (4.3012015436906665,7.9852221064954305)-- (4.3164051091047595,4.0059422521135355);
\draw (11.,8.)-- (11.,7.);
\draw (8.983899691231244,8.34233696522201)-- (9.972131443147275,8.32533149575884);
\draw (5.0157691181530275,3.5127836376815917)-- (9.987335008561367,3.495778168218421);
\draw (5.350247557263068,9.02255574374883) node[anchor=north west] {$x$};
\draw (9.34878526116947,8.971539335359319) node[anchor=north west] {$x$};
\draw (11.158009545446511,7.866183820253237) node[anchor=north west] {$x$};
\draw (3.784280319611512,6.59077361051545) node[anchor=north west] {$5$};
\draw (7.2050825377823875,3.5808055155342737) node[anchor=north west] {$6$};
\begin{scriptsize}
\draw [fill=qqqqff,shift={(5.0005655527389345,8.376347904148352)},rotate=90] (0,0) ++(0 pt,2.25pt) -- ++(1.9485571585149868pt,-3.375pt)--++(-3.8971143170299736pt,0 pt) -- ++(1.9485571585149868pt,3.375pt);
\draw [fill=qqqqff,shift={(5.958390173826779,8.376347904148352)},rotate=270] (0,0) ++(0 pt,2.25pt) -- ++(1.9485571585149868pt,-3.375pt)--++(-3.8971143170299736pt,0 pt) -- ++(1.9485571585149868pt,3.375pt);
\draw [fill=qqqqff,shift={(4.3012015436906665,7.9852221064954305)}] (0,0) ++(0 pt,2.25pt) -- ++(1.9485571585149868pt,-3.375pt)--++(-3.8971143170299736pt,0 pt) -- ++(1.9485571585149868pt,3.375pt);
\draw [fill=qqqqff,shift={(4.3164051091047595,4.0059422521135355)},rotate=180] (0,0) ++(0 pt,2.25pt) -- ++(1.9485571585149868pt,-3.375pt)--++(-3.8971143170299736pt,0 pt) -- ++(1.9485571585149868pt,3.375pt);
\draw [fill=qqqqff,shift={(11.,8.)}] (0,0) ++(0 pt,2.25pt) -- ++(1.9485571585149868pt,-3.375pt)--++(-3.8971143170299736pt,0 pt) -- ++(1.9485571585149868pt,3.375pt);
\draw [fill=qqqqff,shift={(11.,7.)},rotate=180] (0,0) ++(0 pt,2.25pt) -- ++(1.9485571585149868pt,-3.375pt)--++(-3.8971143170299736pt,0 pt) -- ++(1.9485571585149868pt,3.375pt);
\draw [fill=qqqqff,shift={(8.983899691231244,8.34233696522201)},rotate=90] (0,0) ++(0 pt,2.25pt) -- ++(1.9485571585149868pt,-3.375pt)--++(-3.8971143170299736pt,0 pt) -- ++(1.9485571585149868pt,3.375pt);
\draw [fill=qqqqff,shift={(9.972131443147275,8.32533149575884)},rotate=270] (0,0) ++(0 pt,2.25pt) -- ++(1.9485571585149868pt,-3.375pt)--++(-3.8971143170299736pt,0 pt) -- ++(1.9485571585149868pt,3.375pt);
\draw [fill=qqqqff,shift={(5.0157691181530275,3.5127836376815917)},rotate=90] (0,0) ++(0 pt,2.25pt) -- ++(1.9485571585149868pt,-3.375pt)--++(-3.8971143170299736pt,0 pt) -- ++(1.9485571585149868pt,3.375pt);
\draw [fill=qqqqff,shift={(9.987335008561367,3.495778168218421)},rotate=270] (0,0) ++(0 pt,2.25pt) -- ++(1.9485571585149868pt,-3.375pt)--++(-3.8971143170299736pt,0 pt) -- ++(1.9485571585149868pt,3.375pt);
\end{scriptsize}
\end{tikzpicture}
\end{minipage}
\begin{minipage}{0.49\linewidth}
\begin{enumerate}
\item Quelle est la valeur la plus petite pour $x$ ? Et la plus grande valeur pour $x$ ? \point{3}
\item Détermine le périmètre $\mathscr{P}$ de cette surface en fonction de $x$.\point{5}
\item Calcule $\mathscr{P}(2)$.\point{2}
\end{enumerate}
\end{minipage}