\documentclass[10pt]{article}

\input{../../../latex_preambule_style/preambule}
\input{../../../latex_preambule_style/styleCoursCycle4}
\input{../../../latex_preambule_style/styleExercices}
\input{../../../latex_preambule_style/bas_de_page_perso}

\usepackage{multicol}

 % pied de page et entete
  \renewcommand{\headrulewidth}{0pt}
    \rhead{} 
 \cfoot{\thepage}
 \lfoot{Cycle 4 - Niveau 3} 
  \rfoot{Lycée Gustave Flaubert - La Marsa}
 \renewcommand{\footrulewidth}{0.4pt}
 \parindent=0cm
 
%%%%%%%%%%%   Marges de pages  %%%%%%%%%%%%%%%% 
 \usepackage{geometry}
 \geometry{top=3cm, bottom=0cm, left=2cm , right=2cm}
%%%%%%%%%%%%%%%%%%%%%%%%%%%%%%%%%%%%%%%%%%%%%%%
\begin{document}

\section{Lecture d'un tableau}

%%%%%%%%%%%%%%%%%%%%%%%%%%%%%%%%%%%%%%%%%%%%%%%
%%%%		 Corps du document
%%%%%%%%%%%%%%%%%%%%%%%%%%%%%%%%%%%%%%%%%%%%%%%

\begin{cadre}[eduscol4P]
\begin{description}
\item {\color{eduscol4P} COMPETENCES}
\begin{description}
\item[1] CHERCHER
\item[2] CALCULER
\end{description}
\item {\color{eduscol4P} ÉLÉMENT SIGNIFIANT}
\begin{description}
\item[1] Extraire l'information d'un tableau (D1-3)
\item[2] Utiliser le calcul littéral (D1-3)
\end{description}
\end{description}
\end{cadre}


Soient les fonctions $f$, $g$ et $h$ définies par :

\[f(x) = 6x \qquad g(x) = 3x^2 - 9x - 7\qquad \text{et} \quad  h(x) = 5x - 7.\]

À l'aide d'un tableur, Pauline a construit un tableau de valeurs de ces fonctions.

Elle a étiré vers la droite les formules qu'elle avait saisies dans les cellules B2, B3 et B4.

\begin{center}
\begin{tabularx}{\linewidth}{|c|m{2.75cm}|*{7}{>{\centering \arraybackslash}X|}}\hline
\multicolumn{2}{|c|}{B3}&\multicolumn{7}{l|}{$=3*\text{B}1*\text{B}1-9*\text{B}1-7$} \\ \hline
	&A						&B		&C		&D		&E		&F		&G		&H\\ \hline
1	&$x$					&$-3$	&$-2$	&$-1$	&0		&1		&2		&3\\ \hline
2	&$f(x) = 6x$			&$-18$	&$-12$	&$-6$	&0		&6		&12		& 18\\ \hline
3	&$g(x) = 3x^2 - 9x - 7$	&47 	&23 	&5 		&$-7$ 	&$- 13$	& $-13$	& $-7$\\ \hline
4	&$h(x) = 5x - 7$			&$-22$ 	&$-17$ 	&$-12$ 	&$-7$ 	&$-2$ 	&3 		&8\\ \hline
\end{tabularx}
\end{center}

\medskip

\begin{enumerate}
\item Utiliser le tableur pour déterminer la valeur de $h(-2)$. Entourer en Rouge.
\item Écrire les calculs montrant que : $g(- 3) = 47$.

\point{4}

\item Faire une phrase avec le mot \og antécédent\fg{} ou le mot \og image \fg{} pour traduire
l'égalité $g(- 3) = 47$.
\point{1}

\item Quelle formule Pauline a-t-elle saisie dans la cellule B4 ?
\point{1}

\item  
	\begin{enumerate}
		\item Déduire du tableau ci-dessus une solution de l'équation ci-dessous $3x^2 - 9x - 7 = 5x - 7.$ Entourer en vert.

		\item \textbf{ Question Bonus :} Cette équation a-t-elle une autre solution que celle trouvée grâce au tableur ?
Justifier la réponse.


\point{6}
	\end{enumerate}
\end{enumerate}



\end{document}