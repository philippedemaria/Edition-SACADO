
La voiture de Marcel consomme 4,5 litres pour 100 km.

\begin{enumerate}
\item Complète le tableau ci dessous

\begin{tabular}{|c|>{\centering\arraybackslash}p{2cm}|>{\centering\arraybackslash}p{2cm}|>{\centering\arraybackslash}p{2cm}|>{\centering\arraybackslash}p{2cm}|}
\hline 
Volume d'essence $v$ en L & 4,5 & 2,25 & 10 & 18 \vplus \\ 
\hline 
Distance $d$ en km & 100 &  &  &  \vplus   \\ 
\hline 
\end{tabular} 

\item Détermine une formule qui lie le volume $v$ à la distance $d$.

\Point{1}
\item 
\begin{enumerate} 
\item Nomme les axes du repère par les données qu'ils représentent.
\item Trace dans le repère ci dessous les points dont l'abscisse est le volume d'essence et l'ordonnée la distance correspondante.

\definecolor{cqcqcq}{rgb}{0.7529411764705882,0.7529411764705882,0.7529411764705882}
\begin{tikzpicture}[line cap=round,line join=round,>=triangle 45,x=0.655603448275862cm,y=0.02758620689655112cm]
\draw [color=cqcqcq,, xstep=0.655603448275862cm,ystep=1.3879310344827556cm] (-1.222879684418147,-24.223602484473055) grid (21.65680473372781,516.1490683229815);
\draw[->,color=black] (-1.222879684418147,0.) -- (21.65680473372781,0.);
\foreach \x in {-1.,1.,2.,3.,4.,5.,6.,7.,8.,9.,10.,11.,12.,13.,14.,15.,16.,17.,18.,19.,20.,21.}
\draw[shift={(\x,0)},color=black] (0pt,2pt) -- (0pt,-2pt) node[below] {\footnotesize $\x$};
\draw[->,color=black] (0.,-24.223602484473055) -- (0.,516.1490683229815);
\foreach \y in {,50.,100.,150.,200.,250.,300.,350.,400.,450.,500.}
\draw[shift={(0,\y)},color=black] (2pt,0pt) -- (-2pt,0pt) node[left] {\footnotesize $\y$};
\draw[color=black] (0pt,-10pt) node[right] {\footnotesize $0$};
\clip(-1.222879684418147,-24.223602484473055) rectangle (21.65680473372781,516.1490683229815);
\end{tikzpicture}

\item Quel est le type de représentation graphique ?

\Point{1}
\end{enumerate}

\item Avec Géogébra
\begin{enumerate}
\item Trace les points ci dessus. 
\item Dans la fenêtre \textbf{algèbre}, recopie la relation entre $x$ et $y$.

\Point{1}
\item Compare la avec la formule entre $v$ et $d$.

\Point{1}
\end{enumerate}
\end{enumerate}