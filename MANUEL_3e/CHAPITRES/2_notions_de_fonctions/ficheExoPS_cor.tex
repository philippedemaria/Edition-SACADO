\documentclass[openany]{book}



\input{../../../latex_preambule_style/preambule}
\input{../../../latex_preambule_style/styleCoursCycle4}
\input{../../../latex_preambule_style/styleExercices}
\input{../../../latex_preambule_style/styleExercicesAideCompetences}
%\input{../../latex_preambule_style/styleCahier}
\input{../../../latex_preambule_style/bas_de_page_cycle4}
\input{../../../latex_preambule_style/algobox}


%%%%%%%%%%%%%%%  Affichage ou impression  %%%%%%%%%%%%%%%%%%
\newcommand{\impress}[2]{
\ifthenelse{\equal{#1}{1}}  %   1 imprime / affiche sur livre  -----    0 affiche sur cahier 
{%condition vraieé
#2
}% fin condition vraie
{%condition fausse
}% fin condition fausse
} % fin de la procédure
%%%%%%%%%%%%%%%  Affichage ou impression  %%%%%%%%%%%%%%%%%%
%%%%%%%%%%%   Marges de pages  %%%%%%%%%%%%%%%% 
 \usepackage{geometry}
 \geometry{top=1cm, bottom=0cm, left=2cm , right=2cm}
%%%%%%%%%%%%%%%%%%%%%%%%%%%%%%%%%%%%%%%%%%%%%%%
%%%%%%%%%%%%%%%%%%%%%%%%%%%%%%%%%%%%%%%%%%%%%%%%

\begin{document}


\begin{seance}[Notions de fonction.]

\section{Préparer le DNB}


\end{seance}

\vspace{0.4cm}
\section{Corrigé}

\vspace{0.4cm}




\begin{enumerate}
\item $\dfrac{18}{15} = \dfrac{x}{60}$. Sa fréquence cardiaque est donc  $\dfrac{18 \times 60}{15} = 72$ pulsations par minute.

Ou en supposant les pulsations régulières sur 60 secondes :

18 en 15~(s) donnent 36 en 30~(s) et 72 en 60~(s).
\item Il y a $\dfrac{60}{0,8} = \frac{600}{8} = \dfrac{8 \times 75}{8 \times 1} =  75$ intervalles donc $76$ pulsations/min.
\item 
	\begin{enumerate}
		\item L'étendue est la différence entre la plus haute et la plus basse fréquence : E $= 182 - 65 = 117$ pulsations /min.
		\item On divise le nombre total de pulsation par la fréquence moyenne, d'où
		
$\dfrac{\np{3640}}{130} = 28$ minutes.
		
L'entrainement a duré environ 28 minutes.
	\end{enumerate}		
\item
	\begin{enumerate} 
		\item Denis a 32 ans, donc sa FCMC est $f(32) = 220 - 32 = 188$ pulsations/minute.
		\item Pour une personne de 15 ans, la FCMC est $f(15) = 220 - 15 = 205$ pulsations/minute.

La FCMC de Denis est inférieure à la FCMC d'une personne de 15 ans.
		\item D'après les questions précédentes, on peut écrire :
			\begin{description}
			\item[•] $f: 15 \mapsto 205$
			\item[•] $f: 32 \mapsto 188$
			\end{description}
		
	Plusieurs façons sont possibles. Une seule est demandée : 	
			\begin{description}
			\item[Façon 1 :] $\frac{188}{32} \neq \frac{205}{15}$ 
			\item[Façon 2 :] $\frac{205}{188} \neq \frac{15}{32}$
			\item[Façon 3 :] $15 \times 188 \neq 32 \times 205$
			\end{description}		
		donc la FCMC n'est pas proportionnelle à l'age.


	\end{enumerate}
	
	
	
\item $=191,5 - 0,007*\text{A}2*\text{A}2$.
\end{enumerate}

\end{document}