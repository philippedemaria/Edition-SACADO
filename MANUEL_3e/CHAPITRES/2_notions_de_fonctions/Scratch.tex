\documentclass[10pt]{article}
 
 
 
 
 \input{../../../latex_preambule_style/preambule}
\input{../../../latex_preambule_style/styleCoursCycle4}
\input{../../../latex_preambule_style/styleExercices}
\input{../../../latex_preambule_style/styleExercicesAideCompetences}
\input{../../../latex_preambule_style/bas_de_page_cycle4}
\input{../../../latex_preambule_style/algobox}
 % scratch
\usepackage{scratch}
 
\begin{document}
 
 \Exe


\begin{enumerate}
\item On a utilisé une feuille de calcul pour obtenir les images de différentes valeurs de $x$ par une fonction $f$.

Voici une copie de l'écran obtenu :

\begin{center}
\begin{tabularx}{\linewidth}{|c|*{8}{>{\centering \arraybackslash}X|}}\hline
\multicolumn{2}{|l|}{B2}&\multicolumn{7}{l|}{=3*B1$-4$}\\ \hline
	&A		&B		&C		&D		&E		&F	&G	&H\\ \hline
1	&$x$	&$-2$	&$-1$	&0		&1		&2	&3	&4\\ \hline
2	&$f(x)$	&\multicolumn{1}{>{\columncolor{lightgray}}c|}{$- 10$}	&$- 7$	&$- 4$	&$- 1$	&2	&5	&8\\ \hline
\end{tabularx}
\end{center}

	\begin{enumerate}
		\item Quelle est l'image de $- 1$ par la fonction $f$ ?
		\item Quel est l'antécédent de $5$ par la fonction $f$?
		\item Donner l'expression de $f(x)$.
		\item Calculer $f(10)$.
 	\end{enumerate}
\item  On donne le programme suivant qui traduit un programme de calcul.

\begin{center}
\begin{scratch}
\blockinit{Quand \greenflag est cliqué}
\blocksensing{demander \txtbox{Choisir un nombre} et attendre}
\blockmove{mettre \ovalnum{A \selectarrownum} à \ovalvariable{réponse}}
\blockmove{mettre \ovalnum{A\selectarrownum} à \ovalnum{\ovalnum{A} + \ovalnum{3}}}
\blockmove{mettre \ovalnum{A\selectarrownum} à \ovalnum{\ovalnum{A} * \ovalnum{2}}}
\blockmove{mettre \ovalnum{A\selectarrownum} à \ovalnum{\ovalnum{A} $- \ovalnum{5}$}}
\blockmove{dire\ovalnum{regroupe} {\txtbox{Le programme de calcul donne }}\ovalnum{A}}
\end{scratch}
\end{center}
	\begin{enumerate}
		\item Écrire sur votre copie les deux dernières étapes du programme de calcul:
\begin{center}
\begin{tabularx}{0.4\linewidth}{|X|}\hline
$\bullet~~$ Choisir un nombre.\\
$\bullet~~$ Ajouter 3 à ce nombre.\\
$\bullet~~$ \ldots\\
$\bullet~~$ \ldots\\ \hline
\end{tabularx}
\end{center}
		\item  Si on choisit le nombre $8$ au départ, quel sera le résultat ?
		\item  Si on choisit $x$ comme nombre de départ, montrer que le résultat obtenu avec ce programme de calcul sera $2x + 1$.
		\item Quel nombre doit-on choisir au départ pour obtenir 6 ?
 	\end{enumerate}
\item  Quel nombre faudrait-il choisir pour que la fonction $f$ et le programme de calcul
donnent le même résultat ?
\end{enumerate}

 \newpage
 
 \Exe

On donne le programme ci-dessous où on considère $2$ lutins. Pour chaque lutin, on a écrit un
script correspondant à un programme de calcul différent.

\begin{center}
\begin{tabularx}{\linewidth}{|X|c|}\hline
Lutin \no 1 &Numéro d'instruction\\ \hline
\begin{scratch}
\blockinit{Quand \greenflag est cliqué}\end{scratch}&\raisebox{12pt}{1}\\
\begin{scratch}\blocksensing{demander \txtbox{Saisir un nombre} et attendre}\end{scratch}&\raisebox{12pt}{2}\\
\begin{scratch}\blockvariable{mettre \ovalvariable{x} à {\ovaloperator{\ovalvariable{réponse} + \ovalnum{5}}}}\end{scratch}&\raisebox{12pt}{3}\\
\begin{scratch}\blockvariable{mettre \ovalvariable{x} à {\ovaloperator{\ovalvariable{x} * \ovalnum{2}}}}\end{scratch}&\raisebox{12pt}{4}\\
\begin{scratch} \blockvariable{mettre \ovalvariable{x} à \ovaloperator{\ovalvariable{x} - \ovalvariable{réponse}}} \end{scratch}&\raisebox{12pt}{5}\\
\begin{scratch}\blocklook{dire \txtbox{regroupe} Le programme de calcul donne{\ovalvariable{x}}}\end{scratch}&\raisebox{12pt}{6}\\ 
\hline
\end{tabularx}
\end{center}

\medskip

\begin{flushleft}
\begin{tabularx}{0.75\linewidth}{|X|}
\hline
Lutin \no 2\\ 
\hline
\begin{scratch}\blockinit{Quand je reçois \ovalvariable{nombre saisi}}\end{scratch}\\
\begin{scratch}\blockvariable{mettre \ovalvariable{x} \`a {\ovaloperator{\ovalnum{7} * \ovalvariable{réponse}}}}\end{scratch}\\
\begin{scratch}\blockvariable{mettre \ovalvariable{x} \`a {\ovalvariable{x} - \ovalnum{8}}}\end{scratch}\\
\begin{scratch}\blocklook{dire \txtbox{regroupe} Le programme de calcul donne{\ovalvariable{x}}}\end{scratch}\\ 
\hline
\end{tabularx}
\end{flushleft}

\begin{enumerate}
\item Vérifier que si on saisit $7$ comme nombre, le lutin \no 1 affiche comme résultat $17$ et le lutin
\no 2 affiche $41$.
\item Quel résultat affiche le lutin \no 2 si on saisit le nombre $- 4$ ?
\item 
	\begin{enumerate}
		\item Si on appelle $x$ le nombre saisi, écrire en fonction de $x$ les expressions qui traduisent le programme de calcul du lutin \no 1, à chaque étape (instructions 3 à 5).
		\item Montrer que cette expression peut s'écrire $x + 10$.
	\end{enumerate}
\item Célia affirme que plusieurs instructions dans le script du lutin \no 1 peuvent être supprimées et remplacées 
par celle ci-contre.
\begin{minipage}[c][1cm][c]{6cm}
\hfill\begin{scratch}
\blockvariable{mettre \ovalvariable{x} à {\ovaloperator{\ovalvariable{réponse} + \ovalnum{10}}}}
\end{scratch}
\end{minipage}

Indiquer, sur la copie, les numéros des instructions qui sont alors inutiles.
\item  Paul a saisi un nombre pour lequel les lutins \no 1 et \no 2 affichent le même résultat. Quel
est ce nombre ?
\end{enumerate}
\end{document}