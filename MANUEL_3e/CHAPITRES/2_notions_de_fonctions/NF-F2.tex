\begin{titre}[Notions de fonction]

\Titre{Fonctions linéaires et affines}{2,5}
\end{titre}


\begin{CpsCol}
\textbf{Comprendre et utiliser la notion de fonction}
\begin{description}
\item[$\square$] Connaitre la fonction linéaire
\item[$\square$] Résoudre des problèmes liés à la fonction linéaire
\item[$\square$] Connaitre la fonction affine
\item[$\square$] Résoudre des problèmes liés à la fonction affine
\item[$\square$] Lire et interpréter les coefficients d'une fonction affine représenter par une droite
\end{description}
\end{CpsCol}



\begin{DefT}{Fonction linéaire}
Une \textbf{fonction linéaire} \index{Fonction linéaire} est une fonction dont l'image de $x$ est de la forme $ax$ où $a$ est un nombre. Les fonctions linéaires traduisent des situations de proportionnalité.
\end{DefT}

\begin{Ex}
La fonction $f$ définie par $f(x)=65x$ est une fonction linéaire. 
\end{Ex}

\begin{Rq}
La représentation d'une fonction linéaire est une droite qui passe par l'origine. 
\end{Rq}




\begin{DefT}{Fonction affine}
Une \textbf{fonction affine} \index{Fonction affine} est une fonction dont l'image de $x$ est de la forme $ax+b$ où $a$ et $b$ sont deux nombres.
\end{DefT}


\begin{Ex}
La fonction $f$ définie par $f(x)=2x+ 5$ est une fonction affine. $a=2$ et $b=5$.
\end{Ex}

\begin{Att}
La fonction $f$ définie par $f(x)=x^2+ 1$ n'est pas une fonction affine puisque $x$ est au carré.
\end{Att}

\begin{Rq}
La représentation d'une fonction affine est une droite dont l'ordonnée à l'origine \index{Ordonnée à l'origine} est $f(0)=b$. 
\end{Rq}


