\begin{titre}[Notions de fonction]

\Titre{Fonction, image, antécédent}{2}
\end{titre}

 
\begin{CpsCol}
\textbf{Comprendre et utiliser la notion de fonction}
\begin{description}
\item[$\square$] Connaitre la notion de fonction
\item[$\square$] Déterminer l'image par une fonction
\item[$\square$] Déterminer un antécédent par une fonction
\end{description}
\end{CpsCol}
 

\Rec{1}{NF-13}

 
\begin{DefT}{Variable}
Une variable \index{Variable} mathématique désigne une valeur arbitraire, pas totalement précisée, ou même inconnue - appartenant à un ensemble. Les valeurs de la variable varie durant l'exercice. Comme ces valeurs ne sont pas fixes, on note la variable par une lettre.
\end{DefT}

\begin{Ex}
Un carré a pour coté $x$. Son périmètre est $\mathscr P = 4x$. On préfère écrire $\mathscr{P}(x) = 4x$. $x$ est la variable car pour chaque valeur de $x$ positive, on obtient une valeur du périmètre. Le périmètre est en fonction de $x$.
\end{Ex}

\begin{DefT}{Fonction}
Une \textbf{fonction} \index{Fonction} est un procédé qui à un nombre donné associe un unique nombre. La fonction est explicitée par une expression littérale en fonction de la variable.
\end{DefT}

\begin{Nt}
On écrit $f : x \longmapsto f(x)$ et on lit $f$ est la fonction qui à $x$ associe le nombre $f(x)$.
\end{Nt}


\begin{DefT}{Image, antécédent}
L'\textbf{image} \index{Image} d'un nombre par $f$ est l'unique nombre obtenu après le procédé calculatoire de la fonction.\\ Lorsque $f : a \longmapsto b$, on dit que $b$ est l'image de $a$ par $f$. On note alors $f(a)=b$. \\
$a$ est appelé un \index{Antécédent} \textbf{antécédent} de $b$ par $f$.
\end{DefT}

\begin{Ex}
Un triangle équilatéral de coté 4 cm a un périmètre égal à 12 cm. \\
On peut alors dire que 12 est l'image de 4 par la fonction $f$, où $f : 4 \mapsto 12$. $4$ est un antécédent de 12 par $f$.\\Plus généralement, $f : x \mapsto 3x$ ou $f(x)=3x$. $f(x)$ est l'image de $x$ par la fonction $f$.\\
La fonction $f$ est la fonction qui a une longueur du coté d'un triangle équilatéral associe son périmètre.
\end{Ex}

\begin{Ety}
\textbf{Antécédent} est composé de anté - cédent : qui vient avant le procédé. \\Un antécédent vient donc avant la flèche qui symbolise  le procédé. Antécédent $\mapsto$  image.
\end{Ety}
 

\AD{1}{NF-15}

\AD{1}{NF-14}

\App{1}{NF-10}

 
 