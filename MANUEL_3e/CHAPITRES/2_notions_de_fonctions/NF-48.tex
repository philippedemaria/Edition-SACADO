
Lorsqu’un conducteur aperçoit un obstacle, son temps de réaction avant de commencer à freiner est estimé à 1 seconde. Durant cette seconde, le véhicule continue à avancer à vitesse constante. Ensuite, le conducteur freine jusqu’à l’arrêt total.
La distance d’arrêt est égale à la somme de la distance parcourue pendant le temps de réaction et de la distance de freinage.

\begin{center}
\includegraphics[scale=1]{../Troisième/NF-41.png} 
\end{center}

Le but du travail qui suit est d’étudier la distance d’arrêt d’un véhicule en fonction de sa vitesse à l’instant où le conducteur aperçoit l’obstacle.


\sectioncolor{eduscol4B}{Partie 1 : distance parcourue pendant le temps de réaction}

\begin{enumerate}
\item Compléter le tableau suivant où $v$ est la vitesse du véhicule en km/h et $D_r$ la distance parcourue en m pendant le temps de réaction.

\begin{tabular}{|c|c|c|c|c|c|c|c|c|c|c|c|}
\hline 
$v$ (km/h) & 0 & 5 & 10 & 15 & 20 & 25 & 30 & 35 & 40 & 45 & 50 \\ 
\hline 
$D_r$ (m) & • & • & • & • & • & • & • & • & • & • & • \\ 
\hline 
\end{tabular} 

\item Écrire une formule qui donne $D_r$ en fonction de $v$.
\item Construire une représentation graphique, à partir des valeurs du tableau, donnant la distance parcourue pendant le temps de réaction en fonction de la vitesse.
\item La distance parcourue pendant le temps de réaction est-elle proportionnelle à la vitesse du véhicule ? Justifier.
\end{enumerate}


\sectioncolor{eduscol4B}{Partie 2 : distance de freinage}


On admet que la distance de freinage $D_f$ en mètre d'un véhicule en fonction de sa vitesse en km/h peut être donnée par la formule suivante :
$$D_f=\frac{v^2}{2} \div 100$$

\begin{enumerate}
\item Compléter le tableau suivant où $v$ est la vitesse du véhicule et $D_f$ la distance de freinage.

\begin{tabular}{|c|c|c|c|c|c|c|c|c|c|c|c|}
\hline 
$v$ (km/h) & 0 & 5 & 10 & 15 & 20 & 25 & 30 & 35 & 40 & 45 & 50 \\ 
\hline 
$D_f$ (m) & • & • & • & • & • & • & • & • & • & • & • \\ 
\hline 
\end{tabular} 

\item Construire une représentation graphique, à partir des valeurs du tableau, donnant la distance de freinage $D_f$ en fonction de la vitesse $v$.
\item La distance de freinage est-elle proportionnelle à la vitesse du véhicule ? Justifier
\end{enumerate}

\sectioncolor{eduscol4B}{Partie 3 : distance d’arrêt}


La distance d’arrêt d’un véhicule est la somme de la distance parcourue pendant le temps de réaction et de la distance de freinage.

\begin{enumerate}
\item Compléter le tableau suivant où $v$ est la vitesse du véhicule et $D_a$ la distance d'arrêt.

\begin{tabular}{|c|c|c|c|c|c|c|c|c|c|c|c|}
\hline 
$v$ (km/h) & 0 & 5 & 10 & 15 & 20 & 25 & 30 & 35 & 40 & 45 & 50 \\ 
\hline 
$D_r$ (m) & • & • & • & • & • & • & • & • & • & • & • \\ 
\hline
$D_f$ (m) & • & • & • & • & • & • & • & • & • & • & • \\ 
\hline 
$D_a$ (m) & • & • & • & • & • & • & • & • & • & • & • \\ 
\hline
\end{tabular} 

\item Construire une représentation graphique, à partir des valeurs du tableau, donnant la distance d'arrêt $D_a$ en fonction de la vitesse $v$.
\item La distance d'arrêt est-elle proportionnelle à la vitesse du véhicule ? Justifier.
\item Un ballon traverse la route 10 mètres devant la voiture qui est en mouvement. Quelle est vitesse maximum à laquelle cette voiture peut rouler pour ne pas heurter le ballon ?
\end{enumerate}

