
A la séance de 18 heures, le prix des places de cinémas est de 5 \euro{} pour les moins de 12 ans (enfant) et de 8 \euro{} sinon (adultes).

\begin{enumerate}[leftmargin=*]
\item Pierre et Marie, deux élèves de CP, vont au cinéma à 18h00. Quel est le montant total des places de cinémas ? \point{2}
\item Sasha, Tristan et Colin, sont trois frères et  vont au cinéma à 18h00. Les deux cadets ont respectivement 8 et 11 ans et l'ainé a 16 ans. Quel est le montant total des places de cinéma ? \point{2}
\item Monsieur et Madame Cinefil vont au cinéma à la séance de 18h00 avec leur 3 enfants, Marie 17 ans, Audrey 14 ans et Tom 10 ans. Quel est le montant total des places de cinéma ? \point{3}
\item Exprimer le prix total des places $\mathcal{P}$ en fonction du nombres $n$ d'enfants de moins de 12 ans et du nombre de personnes $m$ dont l'âge dépassent 12 ans. \point{2}
\item Un groupe d'amis va au cinéma. Le prix total des places est $68$ \euro{}. Quel est le nombre $e$ d'enfants de moins de 12 ans ?  et le nombre de personnes $a$ dont l'âge dépassent 12 ans. \point{7}
\end{enumerate}