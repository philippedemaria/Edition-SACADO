%----------------------------------------------------------------------------------------
%	PACKAGES AND OTHER DOCUMENT CONFIGURATIONS
%----------------------------------------------------------------------------------------

%----------------------------------------------------------------------------------------
%		Generals
%----------------------------------------------------------------------------------------
\usepackage{fourier}
\usepackage{frcursive}
\usepackage[T1]{fontenc} %Accents handling
\usepackage[utf8]{inputenc} % Use UTF-8 encoding
%\usepackage{microtype} % Slightly tweak font spacing for aesthetics
\usepackage[english, francais]{babel} % Language hyphenation and typographical rules

%----------------------------------------------------------------------------------------
%		Graphics
%----------------------------------------------------------------------------------------
\usepackage{xcolor}
\usepackage{graphicx, multicol} % Enhanced support for graphics
\graphicspath{{FIG/}}
\usepackage{wrapfig}

%----------------------------------------------------------------------------------------
%		Other packages
%----------------------------------------------------------------------------------------
\usepackage{hyperref}
\hypersetup{
	colorlinks=true, %colorise les liens
	breaklinks=true, %permet le retour à la ligne dans les liens trop longs
	urlcolor= bleu3,  %couleur des hyperliens
	linkcolor= bleu3, %couleur des liens internes
	plainpages=false  %pour palier à "Bookmark problems can occur when you have duplicate page numbers, for example, if you have a page i and a page 1."
}
\usepackage{tabularx}
\newcolumntype{M}[1]{>{\arraybackslash}m{#1}} %Defines a scalable column type in tabular
\usepackage{booktabs} % Enhances quality of tables
\usepackage{diagbox} % barre en diagonale dans un tableau
\usepackage{multicol}
\usepackage[toctitles,explicit]{titlesec}


%----------------------------------------------------------------------------------------
%		Headers and footers
%----------------------------------------------------------------------------------------
\usepackage{fancyhdr} % Headers and footers
\pagestyle{fancy} % All pages have headers and footers
\fancyhead{}\renewcommand{\headrulewidth}{0pt} % Blank out the default header
\renewcommand{\footrulewidth}{0pt}
\fancyfoot[L]{} % Custom footer text
\fancyfoot[C]{\href{https://sacado.xyz/}{sacado.xyz}} % Custom footer text
\fancyfoot[R]{\thepage} % Custom footer text

%----------------------------------------------------------------------------------------
%		Table of Contents
%----------------------------------------------------------------------------------------
%%%%% Pour la Table Of Contents TOC
% \usepackage{titletoc}
\renewcommand\contentsname{Table des maères}
% %%%%%%%%%%%%% Profondeur de TOC
% \setcounter{tocdepth}{3}     % Dans la table des matieres
% \setcounter{secnumdepth}{3}  % Avec un numero.

%----------------------------------------------------------------------------------------
%		Mathematics packages
%----------------------------------------------------------------------------------------
\usepackage{amsthm, amsmath, amssymb} % Mathematical typesetting
\usepackage{marvosym, wasysym} % More symbols
\usepackage[makeroom]{cancel}
\usepackage{xlop}
\usepackage{pgf,tikz,pgfplots}
\pgfplotsset{compat=1.15}
\usetikzlibrary{positioning}
%\usetikzlibrary{arrows}
\usepackage{pst-plot,pst-tree,pst-func, pstricks-add,pst-node,pst-text}
\usepackage{units}
\usepackage{nicefrac}
\usepackage[np]{numprint} %Séparation milliers dans un nombre

%----------------------------------------------------------------------------------------
%		New text commands
%----------------------------------------------------------------------------------------
\usepackage{calc}
\usepackage{boites}
 \renewcommand{\arraystretch}{1.6}

%%%%% Pour les imports.
\usepackage{import}

%%%%% Pour faire des boites
\usepackage[tikz]{bclogo}
\usepackage{bclogo}
\usepackage{framed}
\usepackage[skins]{tcolorbox}
\tcbuselibrary{breakable}
\tcbuselibrary{skins}
\usetikzlibrary{babel,arrows,shadows,decorations.pathmorphing,decorations.markings,patterns}

%%%%% Pour les symboles et les ensembles
\newcommand{\pp}{\leq}
\newcommand{\pg}{\geq}
%\newcommand{\euro}{\eurologo{}}
\newcommand{\R}{\mathbb{R}}
\newcommand{\N}{\mathbb{N}}
\newcommand{\D}{\mathbb{D}}
\newcommand{\Z}{\mathbb{Z}}
\newcommand{\Q}{\mathbb{Q}}
\newcommand{\C}{\mathbb{C}}

%%%%% Pour une double minipage
\newcommand{\mini}[2]{
\begin{minipage}[t]{0.48\linewidth}
#1
\end{minipage}
\hfill
\begin{minipage}[t]{0.48\linewidth}
#2
\end{minipage}
}


%\newcommand\hole[1]{\texttt{\_}}
%\newcommand{\PROP}[1]{\textbf{\underline{#1}}}
%\newcommand{\exercice}{\textcolor{OliveGreen}{Exercice : }}
%\newcommand{\correction}{\textcolor{BurntOrange}{Correction : }}
%\newcommand{\propriete}{\textbf{\underline{Propriété}} : }
%\newcommand{\prop}{\textbf{\underline{Propriété}} : }
%\newcommand{\vocabulaire}{\textbf{\underline{Vocabulaire}} : }
%\newcommand{\voca}{\textbf{\underline{Vocabulaire}} : }

\usepackage{enumitem}
\newlist{todolist}{itemize}{2} %Pour faire des QCM
\setlist[todolist]{label=$\square$} %Pour faire des QCM \begin{todolist} instead of itemize

%----------------------------------------------------------------------------------------
%		Définition de couleur pour geogebra
%----------------------------------------------------------------------------------------
\definecolor{zzttqq}{rgb}{0.6,0.2,0.} %rouge des polygones
\definecolor{qqqqff}{rgb}{0.,0.,1.}
\definecolor{xdxdff}{rgb}{0.49019607843137253,0.49019607843137253,1.}%bleu
\definecolor{qqwuqq}{rgb}{0.,0.39215686274509803,0.} %vert des angles
\definecolor{ffqqqq}{rgb}{1.,0.,0.} %rouge vif
\definecolor{uuuuuu}{rgb}{0.26666666666666666,0.26666666666666666,0.26666666666666666}
\definecolor{qqzzqq}{rgb}{0.,0.6,0.}
\definecolor{cqcqcq}{rgb}{0.7529411764705882,0.7529411764705882,0.7529411764705882} %gris
\definecolor{qqffqq}{rgb}{0.,1.,0.}
\definecolor{ffdxqq}{rgb}{1.,0.8431372549019608,0.}
\definecolor{ffffff}{rgb}{1.,1.,1.}
\definecolor{ududff}{rgb}{0.30196078431372547,0.30196078431372547,1.}

%-------------------------------------------------
%
%	EN TETE
%
%-------------------------------------------------

% Classe
\newcommand{\myClasse}   
{
    6e
}

% Discipline
\newcommand{\myDiscipline}   
{
    Mathématiques
}

% Parcours
\newcommand{\myParcours}
{
  Grandeurs et Mesures
}

%Titre de la séquence
\newcommand{\myTitle}
{
    \scshape\huge
\textcolor{sacado_purple}{
Calculer des longueurs
}
}

%----------------------------------------------------------------------------------------
