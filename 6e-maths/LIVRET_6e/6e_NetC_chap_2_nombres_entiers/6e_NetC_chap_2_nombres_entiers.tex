\input{../MISC/preambule_sacado.tex}
\input{../MISC/environnements_sacado.tex}
\input{../MISC/macro_sacado.tex}

\title{Mathématiques 6e  : le livre sacado}
\author{L'équipe SACADO}

\usepackage{multido}
\usepackage{multirow}


\begin{document}

%\maketitle

\chapter{Les nombres entiers}{2} %XX = numero du chapitre
{https://sacado.xyz/qcm/parcours_show/32270/147706}
{
 \begin{CpsCol}
	\textbf{Les savoir-faire du parcours}
 	\begin{itemize}
 		\item Savoir écrire un nombre entier en lettres.
		\item Savoir écrire un nombre entier en chiffres.
		\item Savoir déterminer la valeur d'un chiffre selon sa position.
		\item Savoir déterminer un nombre de ... dans un nombre entier.
		\item Savoir décomposer un nombre entier.
		\item Savoir comparer des nombres entiers.
		\item Savoir encadrer un nombre entier.
		\item Savoir repérer un nombre entier sur une demi-droite graduée.
		\item Savoir placer un nombre sur une demi-droite graduée.
 	\end{itemize}
 \end{CpsCol}

\begin{His}
	% Un paragraphe parlant de la vie d'un ou une mathématicien ou mathématicienne
\end{His}

\begin{ExoDec}{Compétence.}{1234}{1}{0}{0}{0}
% Un exercice de découverte/d'accroche
\end{ExoDec}
}

%%%%%%%%%%%%%%%%%%%%%%%%%%%%%%%%%%%%%%%%%%%%%%%%%%%%%%%%%%%%%%%%%%%
%%%% Page de cours 1
%%%%%%%%%%%%%%%%%%%%%%%%%%%%%%%%%%%%%%%%%%%%%%%%%%%%%%%%%%%%%%%%%%%

\begin{pageCours} % Début page de cours 1

\newgeometry{left=2cm,right=.8cm,top=1.5cm} %Ne pas toucher cette ligne

\section{Nombres entiers}

\mini{.25\linewidth}{
\begin{Def}
Un \textbf{nombre entier} est un nombre qui peut s'écrire \textbf{sans virgule}
\end{Def}
}{.75\linewidth}{
\begin{Rqs}
\begin{itemize}
\item $0$, $1$, $2$, $3$, $4$, $5$, $6$, $7$, $8$ et $9$ sont les dix \textbf{chiffres} qui permettent d'écrire tous les nombres entiers.
\item Pour pouvoir lire les grands nombres entiers facilement, on regroupe les chiffres par groupe de 3 : $345\,202$
\end{itemize}
\end{Rqs}
}

\mini{.52\linewidth}{
\begin{Reg}
Règles orthographiques pour l'écriture des nombres :
\begin{itemize}
    \item Un trait d'union entre chaque mot.
    \item Les mots servant à écrire les nombres sont tous invariable sauf :
    \begin{itemize}
        \item Au pluriel million et milliard prennent un 's'.
        \item Au pluriel cent et vingt prennent un 's' lorsqu'ils ne sont pas suivi par un autre nombre.
    \end{itemize}
\end{itemize}
\end{Reg}
}{.48\linewidth}{
\begin{Ex}
\begin{itemize}
\item $895$ s'écrit : 'huit-cent-quatre-vingt-quinze'
\item $1\,200$ s'écrit : 'mille-deux-cents'
\item $1\,230$ s'écrit : 'mille-deux-cent-trente'
\item $1\,280$ s'écrit : 'mille-deux-cent-quatre-vingts'
\item $1\,285$ s'écrit : 'mille-deux-cent-quatre-vingt-cinq'
\end{itemize}
\end{Ex}
}

\section{Position d'un chiffre dans un nombre}

\begin{Def}
\begin{itemize}
\item Notre système numérique est un \textbf{système décimal} (numération décimale).
\item Chaque \textbf{chiffre} à une valeur en fonction de sa \textbf{position} dans le nombre (numération de position)
\end{itemize}
\end{Def}

\mini{.22\linewidth}{
\begin{DefT}{Vocabulaire}
Chaque position (rang) possède un nom spécifique : unité, dizaine, centaines, milliers, millions, milliards\ldots
\end{DefT}
}{.78\linewidth}{
\begin{Mt}
Décomposition de $437\,640\,881$ :
\begin{itemize}
\item Décomposition 1 : 
$437\,000\,000+640\,000+881$
\item Décomposition 2 :
$(437\times1\,000\,000)+(640\times 1\,000)+(881\times1)$
\item Décomposition 3 :
$400\,000\,000+30\,000\,000+7\,000\,000+600\,000+40\,000+800+80+1$
\item Décomposition 4 :
$4\times100\,000\,000+3\times10\,000\,000+7\times1\,000\,000+6\times100\,000+4\times10\,000+8\times100+8\times10+1\times1$
\end{itemize}
\end{Mt}
}

\end{pageCours} % Fin page de cours 1

%%%%%%%%%%%%%%%%%%%%%%%%%%%%%%%%%%%%%%%%%%%%%%%%%%%%%%%%%%%%%%%%%%%
%%%% Application direct 1
%%%%%%%%%%%%%%%%%%%%%%%%%%%%%%%%%%%%%%%%%%%%%%%%%%%%%%%%%%%%%%%%%%%

\begin{pageAD}  % Début page d'exercice d'application direct 1
\restoregeometry %Ne pas toucher cette ligne

\Sf{Écrire un nombre entier en chiffres et en lettres}

\begin{ExoCadN}{Communiquer.}{0}{0}{0}{0}{0}
Écris les nombres suivants en chiffres :
\begin{itemize}
\item neuf-millions-sept-cent-neuf-mille-cinq-cents : $\ldots\ldots\ldots\ldots$
\item vingt-millions-quatre-cent-cinquante-mille-sept-cent-cinquante-six : $\ldots\ldots\ldots\ldots$
\item quatre-vingt-millions-deux-cent-vingt-mille-neuf-cents : $\ldots\ldots\ldots\ldots$
\end{itemize}
\end{ExoCadN}

\begin{ExoCadN}{Communiquer.}{0}{0}{0}{0}{0}
Écris les nombres suivants en lettres :
\begin{itemize}
\item $\np{90103116}$ : \point{1}
\item $\np{598819}$ : \point{1}
\item $\np{318380}$ : \point{1}
\end{itemize}
\end{ExoCadN}

\Sf{Déterminer la valeur d'un chiffre selon sa position}

\begin{ExoCadN}{Communiquer.}{0}{0}{0}{0}{0}
Considérons le nombre $\np{68347239}$, complète les phrases :
\begin{itemize}
\mini{.48\linewidth}{
\item Le chiffre des unités de milliers est : $\ldots\ldots$
\item Le chiffre des unités est : $\ldots\ldots$
\item Le chiffre des unités de millions est : $\ldots\ldots$
\item Le chiffre des dizaines est : $\ldots\ldots$
}{.48\linewidth}{
\item Le chiffre des centaines est : $\ldots\ldots$
\item Le chiffre des centaines de milliers est : $\ldots\ldots$
\item Le chiffre des dizaines de milliers est : $\ldots\ldots$
\item Le chiffre des dizaines de millions est : $\ldots\ldots$
}
\end{itemize}
\end{ExoCadN}

\begin{ExoCadN}{Communiquer.}{0}{0}{0}{0}{0}
Complète les phrases :
\begin{enumerate}
\item Dans le nombre $A=72\,505\,0\underline{1}2$, le $\underline{1}$ est le chiffre des  $\ldots\ldots\ldots\ldots$
\item Dans le nombre $A=914\,216\,\underline{5}54$, le $\underline{5}$ est le chiffre des  $\ldots\ldots\ldots\ldots$
\item Dans le nombre $A=2\underline{2}7\,074\,541$, le $\underline{2}$ est le chiffre des  $\ldots\ldots\ldots\ldots$
\end{enumerate}
\end{ExoCadN}

\begin{ExoCadN}{Communiquer.}{0}{0}{0}{0}{0}
Complète les phrases :
\begin{enumerate}
\item Dans le nombre $A=\np{33189207}$, le chiffre des centaines est $\ldots\ldots$
\item Dans le nombre $A=\np{944597319}$, le chiffre des dizaines de milliers est $\ldots\ldots$
\item Dans le nombre $A=\np{5664432}$, le chiffre des unités de millions est $\ldots\ldots$
\end{enumerate}
\end{ExoCadN}

\Sf{Décomposer un nombre entier}

\begin{ExoCadN}{Représenter.}{0}{0}{0}{0}{0}
Margaux a décomposé le nombre $\np{963652}=9\times\np{100000}+6\times\np{10000}+3\times\np{1000}+6\times100+5\times10+2$ décompose les nombres suivants avec la même méthode :
\begin{itemize}
\item $\np{949009}=$\point{1}
\item $\np{745017003}=$\point{1}
\item $\np{6009060}=$\point{1}
\end{itemize}
\end{ExoCadN}
 
\end{pageAD} % Fin page d'exercice d'application direct 1

%%%%%%%%%%%%%%%%%%%%%%%%%%%%%%%%%%%%%%%%%%%%%%%%%%%%%%%%%%%%%%%%%%%
%%%% Page de cours 1
%%%%%%%%%%%%%%%%%%%%%%%%%%%%%%%%%%%%%%%%%%%%%%%%%%%%%%%%%%%%%%%%%%%

\begin{pageCours} % Début page de cours 1

\newgeometry{left=2cm,right=.8cm,top=1.5cm} %Ne pas toucher cette ligne

\section{Comparer des nombres entiers}

\begin{Def}
\textbf{Comparer} deux nombres, c'est trouver le \textbf{plus grand} (ou le \textbf{plus petit}) ou dire s'ils sont \textbf{égaux}.

On utilise les \textbf{symboles de comparaison} :
\begin{center}
est supérieur à ($>$) \hspace{1cm} est inférieur à ($<$) \hspace{1cm} est égal à ($=$)
\end{center}
\end{Def}

\begin{Ex}
$29\,874\,492$ est plus grand que $27\,514\,420$ donc $29\,874\,492>27\,514\,420$.
\end{Ex}

\begin{Def}
\begin{itemize}
\item Ranger des nombres dans \textbf{l'ordre croissant} signifie les ranger \textbf{du plus petit au plus grand}.
\item Ranger des nombres dans \textbf{l'ordre décroissant} signifie les ranger \textbf{du plus grand au plus petit}.
\end{itemize}
\end{Def}


\section{Encadrer un nombre entier}

\begin{Def}
\textbf{Encadrer} un nombre, c'est trouver un nombre plus petit et un nombre plus grand.

La \textbf{précision de l'encadrement} est la \textbf{différence} entre les deux nombres trouvés.
\end{Def}

\begin{Ex}
Encadrement du nombre $56$ :
 \begin{itemize}
\item Encadrement à la dizaine : \(50 < 56< 60\)

\item Encadrement au centième : \(0 < 56< 100\)
 \end{itemize}
\end{Ex}

\section{Nombres entiers et demi-droite graduée}

\begin{Def}
Une \textbf{demi-droite graduée} est une \textbf{demi-droite} sur laquelle on a reporté une \textbf{unité de longueur} régulièrement à partir de son \textbf{origine}.

Sur une demi-droite graduée, \textbf{un point} est repéré par \textbf{un nombre}, son \textbf{abscisse}.

Si un point $A$ a pour abscisse $6$, on note : $A(6)$.

L'origine est repérée par le nombre $0$.

\begin{center}
\begin{tikzpicture}[line cap=round,line join=round,>=triangle 45,x=1.0cm,y=1.0cm]
\begin{axis}[
x=1.0cm,y=1.0cm,
axis lines=middle,
xmin=-0.1994343408245771,
xmax=9.60761325487766,
ymin=-0.4707007586686898,
ymax=0.44760621908013903,
xtick={-0.0,1.0,...,9.0},
ytick={-0.0,1.0,...,0.0},]
\clip(-0.1994343408245771,-0.4707007586686898) rectangle (9.60761325487766,0.44760621908013903);
\draw [->,line width=2.pt] (0.,0.) -- (9.619386421259055,0.);
\begin{scriptsize}
\draw [fill=xdxdff] (6.,0.) circle (2.5pt);
\draw[color=xdxdff] (6.087436506840482,0.21802947464293182) node {$A$};
\end{scriptsize}
\end{axis}
\end{tikzpicture}
\end{center}

\end{Def}

\end{pageCours} % Fin page de cours 1

%%%%%%%%%%%%%%%%%%%%%%%%%%%%%%%%%%%%%%%%%%%%%%%%%%%%%%%%%%%%%%%%%%%
%%%% Application direct 1
%%%%%%%%%%%%%%%%%%%%%%%%%%%%%%%%%%%%%%%%%%%%%%%%%%%%%%%%%%%%%%%%%%%

\begin{pageAD}  % Début page d'exercice d'application direct 1
\restoregeometry %Ne pas toucher cette ligne

\Sf{Comparer et ordonner des nombres entiers}

\begin{ExoCadN}{Représenter. Communiquer.}{0}{0}{0}{0}{0}
Range les nombres donnés dans l'ordre \textbf{décroissant} :
\begin{itemize}
\item $\np{678093}-\np{267879}-\np{803830}-\np{510819}$ :

\point{1}
\item $\np{294669}-\np{67611}-\np{336306}-\np{476448}$ :

\point{1}
\end{itemize}
Range les nombres donnés dans l'ordre \textbf{croissant} :
\begin{itemize}
\item $\np{417904}-\np{993961}-\np{357198}-\np{811940}$ :

\point{1}
\item $\np{426488}-\np{948073}-\np{670748}-\np{618317}$ :

\point{1}
\end{itemize}
\end{ExoCadN}

\Sf{Encadrer des nombres entiers}

\begin{ExoCadN}{Représenter.}{0}{0}{0}{0}{0}
\begin{enumerate}
\item Amir a placé les nombres $\np{750264}$, $\np{121696}$ et $\np{289054}$ dans un tableau, déduis-en un encadrement au rang indiqué.

\mini{.45\linewidth}{
\vspace{.2cm}
\includegraphics[width=\linewidth]{FIG/tableau_position_exo.png} 
}{.45\linewidth}{
\begin{itemize}
\item Encadrement de $\np{750264}$ aux \textbf{dizaines} près : 
$\ldots\ldots\ldots<\np{750264}<\ldots\ldots\ldots$
\item Encadrement de $\np{121696}$ aux \textbf{unités de milliers} près :
$\ldots\ldots\ldots<\np{121696}<\ldots\ldots\ldots$
\item Encadrement de $\np{289054}$ aux \textbf{centaines} près :
$\ldots\ldots\ldots<\np{289054}<\ldots\ldots\ldots$
\end{itemize}
}
\item Maintenant que tu as compris essaye sans le tableau :
\begin{itemize}
\item Encadrement de $\np{46121215}$ aux \textbf{centaine de milliers} près :

$\ldots\ldots\ldots<\np{46121215}<\ldots\ldots\ldots$
\item Encadrement de $\np{6571845}$ aux \textbf{dizaines} près :

$\ldots\ldots\ldots<\np{546571845}<\ldots\ldots\ldots$
\end{itemize}
\end{enumerate}
\end{ExoCadN}

\Sf{Repérer/Placer un nombre entier sur une demi-droite graduée.}

\begin{ExoCad}{Compétence.}{1234}{0}{0}{0}{0}{0}

\end{ExoCad}
 
\end{pageAD} % Fin page d'exercice d'application direct 1

%%%%%%%%%%%%%%%%%%%%%%%%%%%%%%%%%%%%%%%%%%%%%%%%%%%%%%%%%%%%%%%%%%%
%%%% Parcours niveau 1
%%%%%%%%%%%%%%%%%%%%%%%%%%%%%%%%%%%%%%%%%%%%%%%%%%%%%%%%%%%%%%%%%%%

\begin{pageParcoursu} % Début du parcours niveau 1

%Premier exo du parcours 1
\begin{ExoCu}{Compétence.}{1234}{2}{0}{0}{0}{0}
  
\end{ExoCu}

%Deuxième exo du parcours 1
\begin{ExoCuN}{Compétence.}{1}{0}{0}{0}{0}

\end{ExoCuN}

%Troisième exo du parcours 1
\begin{ExoCuN}{Compétence.}{1}{0}{0}{0}{0}

\end{ExoCuN}

%Quatrième exo du parcours 1
\begin{ExoCuN}{Compétence.}{1}{0}{0}{0}{0}

\end{ExoCuN}

%Cinquième exo du parcours 1
\begin{ExoCuN}{Compétence.}{1}{0}{0}{0}{0}

\end{ExoCuN}

%Sixième exo du parcours 1
\begin{ExoCuN}{Compétence.}{1}{0}{0}{0}{0}

\end{ExoCuN}


\end{pageParcoursu} % Fin du parcours niveau 1
 
%%%%%%%%%%%%%%%%%%%%%%%%%%%%%%%%%%%%%%%%%%%%%%%%%%%%%%%%%%%%%%%%%%%
%%%% Parcours Niveau 2
%%%%%%%%%%%%%%%%%%%%%%%%%%%%%%%%%%%%%%%%%%%%%%%%%%%%%%%%%%%%%%%%%%%

\begin{pageParcoursd} % Début du parcours niveau 2

%Premier exo du parcours 2
\begin{ExoCdN}{Compétence.}{2}{0}{0}{0}{0}

\end{ExoCdN}

%Deuxième exo du parcours 2
\begin{ExoCdN}{Compétence.}{2}{0}{0}{0}{0}

\end{ExoCdN}

%Troisième exo du parcours 2
\begin{ExoCdN}{Compétence.}{2}{0}{0}{0}{0}

\end{ExoCdN}

%Quatrième exo du parcours 2
\begin{ExoCdN}{Compétence.}{2}{0}{0}{0}{0}

\end{ExoCdN}

%Cinquième exo du parcours 2
\begin{ExoCdN}{Compétence.}{2}{0}{0}{0}{0}

\end{ExoCdN}

%Sixième exo du parcours 2
\begin{ExoCdN}{Compétence.}{2}{0}{0}{0}{0}

\end{ExoCdN}

\end{pageParcoursd} % Fin du parcours niveau 2

%%%%%%%%%%%%%%%%%%%%%%%%%%%%%%%%%%%%%%%%%%%%%%%%%%%%%%%%%%%%%%%%%%%%
%%%%% Parcours Niveau 3
%%%%%%%%%%%%%%%%%%%%%%%%%%%%%%%%%%%%%%%%%%%%%%%%%%%%%%%%%%%%%%%%%%%%

\begin{pageParcourst} % Début du parcours niveau 3

% Premier exo du parcours 3
\begin{ExoCtN}{Compétence.}{1}{1}{0}{0}{0}
 
\end{ExoCtN}

% Deuxième exo du parcours 3
\begin{ExoCtN}{Compétence.}{1}{1}{0}{0}{0}
 
\end{ExoCtN}

% Troisième exo du parcours 3
\begin{ExoCtN}{Compétence.}{1}{1}{0}{0}{0}
 
\end{ExoCtN}

% Quatrième exo du parcours 3
\begin{ExoCtN}{Compétence.}{1}{1}{0}{0}{0}
 
\end{ExoCtN}

% Cinquième exo du parcours 3
\begin{ExoCtN}{Compétence.}{1}{1}{0}{0}{0}
 
\end{ExoCtN}

% Sixième exo du parcours 3
\begin{ExoCtN}{Compétence.}{1}{1}{0}{0}{0}
 
\end{ExoCtN}
 
\end{pageParcourst} % Fin du parcours niveau 3

%%%%%%%%%%%%%%%%%%%%%%%%%%%%%%%%%%%%%%%%%%%%%%%%%%%%%%%%%%%%%%%%%%%
%%%%  Autoevaluation/exos ouverts
%%%%%%%%%%%%%%%%%%%%%%%%%%%%%%%%%%%%%%%%%%%%%%%%%%%%%%%%%%%%%%%%%%%

\begin{pageAuto} % Début de la page d'exos ouverts

%Premier exercice : "ce que je peux avoir en éval"
\begin{ExoAutoN}{Compétence.}{1}{0}{0}{0}{0}

\end{ExoAutoN}

%Deuxième exercice : "ce que je peux avoir en éval"
\begin{ExoAutoN}{Compétence.}{1}{0}{0}{0}{0}

\end{ExoAutoN}

%%%%%%%%%%%%%%%%%%%%%%%%%%%%%%%%%%%%%%%%%%%%%%%%%%%%%%%%%%%%%%%%%%%
%%%%%%%%%%%%%%%%%%%%%%%%%%%%%%%%%%%%%%%%%%%%%%%%%%%%%%%%%%%%%%%%%%%

%Problème ouvert
\begin{ExoAutoN}{Compétence.}{1}{0}{0}{0}{0}

\end{ExoAutoN}

\end{pageAuto} % Fin de la page d'exos ouverts

%%%%%%%%%%%%%%%%%%%%%%%%%%%%%%%%%%%%%%%%%%%%%%%%%%%%%%%%%%%%%%%%%%%
%%% Page algorithmique
%%%%%%%%%%%%%%%%%%%%%%%%%%%%%%%%%%%%%%%%%%%%%%%%%%%%%%%%%%%%%%%%%%%

\begin{pageAlgo} % Début de la page d'exos d'algorithmique

%Exercice d'algorithmique
\begin{ExoAlgoN}{Compétence.}{1}{0}{0}{0}{0}

\end{ExoAlgoN}

%Exercice d'algorithmique
\begin{ExoAlgo}{Compétence.}{1}{0}{0}{0}{0}

\end{ExoAlgo}

\end{pageAlgo} % Fin de la page d'exos d'algorithmique

%%%%%%%%%%%%%%%%%%%%%%%%%%%%%%%%%%%%%%%%%%%%%%%%%%%%%%%%%%%%%%%%%%%
%%%%  Page(s) blanche(s)
%%%%%%%%%%%%%%%%%%%%%%%%%%%%%%%%%%%%%%%%%%%%%%%%%%%%%%%%%%%%%%%%%%%

\begin{pageBrouillon}

\end{pageBrouillon}

\end{document}
