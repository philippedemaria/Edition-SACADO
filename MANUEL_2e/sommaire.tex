\documentclass[10pt,a4paper]{article}
\usepackage[utf8]{inputenc}
\usepackage[french]{babel}
\usepackage[T1]{fontenc}
\usepackage{amsmath}
\usepackage{amsfonts}
\usepackage{amssymb}
\usepackage{lmodern}
\usepackage[left=2cm,right=2cm,top=2cm,bottom=2cm]{geometry}


 

\begin{document}


\fbox{ {\Huge Contenus détaillés} }
 


\section*{Remarque}

Les algorithmes du programme, précédés de *, pourraient être décloisonnés dans la partie Algorithmique pour illustrer ou travailler les notions de condition, boucle.
Ce décloisonnement laisse le choix à l'enseignant une double entrée : 
\begin{itemize}
\item soit aborder la notion par le chapitre de la notion
\item soit par l'algorithmique puis revenir en synthèse sur la notion.
\end{itemize}

Dans les chapitres, on propose des exercices de lecture ou modification d'algorithme seulement. 

Dans la partie \textbf{Algorithmique et programmation}, on propose des exercices de  lecture, de  modification et de création.


\section{Algorithmique et programmation}

\textbf{Contenus}

\begin{enumerate} 
\item  Choisir ou déterminer le type d'une variable (entier, flottant ou chaîne de caractères).
\item  Concevoir et écrire une instruction d'affectation, une séquence d'instructions, une 
instruction conditionnelle.
\item Écrire une formule permettant un calcul combinant des variables.
\item Programmer, dans des cas simples, une boucle bornée, une boucle non bornée.
\item Dans des cas plus complexes : lire, comprendre, modifier ou compléter un algorithme 
ou un programme.
\item Fonctions à un ou plusieurs arguments.
\item Fonction renvoyant un nombre aléatoire. Série statistique obtenue par la répétition de 
l’appel d’une telle fonction.

\end{enumerate}


\textbf{Capacités attendues}

\begin{enumerate}
\item Choisir ou déterminer le type d’une variable (entier, flottant ou chaîne de caractères).
\item  Concevoir et écrire une instruction d’affectation, une séquence d’instructions, une 
instruction conditionnelle.
\item  Écrire une formule permettant un calcul combinant des variables.
\item  Programmer, dans des cas simples, une boucle bornée, une boucle non bornée.
\item  Dans des cas plus complexes : lire, comprendre, modifier ou compléter un algorithme 
ou un programme.

\item Écrire des fonctions simples ; lire, comprendre, modifier, compléter des fonctions plus 
complexes. Appeler une fonction.
\item  Lire et comprendre une fonction renvoyant une moyenne, un écart type. Aucune 
connaissance sur les listes n’est exigée.
\item Écrire des fonctions renvoyant le résultat numérique d’une expérience aléatoire, 
d’une répétition d’expériences aléatoires indépendantes.

\end{enumerate}




\section{Arithmétique}

\textbf{Contenus}

\begin{enumerate}
\item Notations $\mathbb{N}$ et $\mathbb{Z}$
\item Définition des notions de multiple, de diviseur, de nombre pair, de nombre impair.
\item Nombres premiers. Décomposition en facteurs premiers.
\end{enumerate}

\textbf{Capacités attendues}

\begin{enumerate}
\item Modéliser et résoudre des problèmes mobilisant les notions de multiple, de diviseur, 
de nombre pair, de nombre impair, de nombre premier.
\item Présenter les résultats fractionnaires sous forme irréductible.
\end{enumerate}

\textbf{Démo}

\begin{enumerate}
\item  Pour une valeur numérique de $a$, la somme de deux multiples de $a$ est multiple de $a$.
\item  Le carré d'un nombre impair est impair.
\end{enumerate}

\textbf{Algorithmique}

\begin{enumerate}
\item *Déterminer si un entier naturel $a$ est multiple d'un entier naturel $b$.
\item Pour des entiers $a$ et $b$ donnés, déterminer le plus grand multiple de $a$ inférieur ou égal à $b$.
\item *Déterminer si un entier naturel $n$ est premier.
 \end{enumerate}


\section{Ensembles de nombres}

\textbf{Contenus}

\begin{enumerate}
\item Ensemble $\mathbb{R}$ des nombres réels, droite numérique.
\item Ensemble $\mathbb{D}$ des nombres décimaux. 
\item Ensemble $\mathbb{Q}$ des nombres rationnels. Nombres irrationnels ; exemples fournis par la géométrie, par exemple $\sqrt 2$ et $\pi$.
\end{enumerate}

\textbf{Capacités attendues}

\begin{enumerate}
\item Associer à chaque point de la droite graduée un unique nombre réel et réciproquement.
\item Connaitre quelques idées ensemblistes et logiques
\end{enumerate}


\textbf{Démo}

\begin{enumerate}
\item   Le nombre réel $\sqrt 2 $ est irrationnel.
\item   Le nombre rationnel $\frac13$ n'est pas décimal.
\end{enumerate}



\textbf{Approfondissement}

\begin{enumerate}
\item Développement décimal illimité d'un nombre réel.
\item Observation, sur des exemples, de la périodicité du développement décimal de nombres rationnels, du fait qu'un développement décimal périodique correspond à un rationnel.
\item Vocabulaire ensembliste et logique
\end{enumerate}




\section{Intervalles de $\mathbb{R}$}

\textbf{Contenus}

\begin{enumerate}
\item Intervalles de  $\mathbb{R}$. Notations $\infty$.
\item Intersection, réunion d'intervalles.
\item Notation $|a|$. Distance entre deux nombres réels.
\item Représentation de l'intervalle $[a - r , a + r]$ puis caractérisation par la condition $|x - a| \leq r$.
\item Encadrement décimal d'un nombre réel à $10^{-n}$ près.
\end{enumerate}

\textbf{Capacités attendues}

\begin{enumerate}
\item Représenter un intervalle de la droite numérique. Déterminer si un nombre réel 
appartient à un intervalle donné.
\item Donner un encadrement, d’amplitude donnée, d’un nombre réel par des décimaux.
\item Dans le cadre de la résolution de problèmes, arrondir en donnant le nombre de 
chiffres significatifs adapté à la situation étudiée

\end{enumerate}




\textbf{Algorithmique}

\begin{enumerate}
\item Déterminer par balayage un encadrement de $\sqrt 2$ d’amplitude inférieure ou égale à $10^{-n}$.

\end{enumerate}




\section{Calculs numériques}

\textbf{Contenus}

\begin{enumerate}
\item Effectuer des calculs numériques mettant en jeu des puissances, des racines carrées, des écritures fractionnaires.
\item Calculs avec les relatifs, avec les rationnels.
\item Calculs avec les puissances.
\item Calculs avec les racines carrées.
\item Règles de calcul sur les puissances entières relatives, sur les racines carrées. $\sqrt {a^2} = |a|$ 
\end{enumerate}

\textbf{Capacités attendues}

\begin{enumerate}
\item Développer la pratique du calcul numérique (relatifs, rationnels, puissances)
\end{enumerate}


\textbf{Démo}

\begin{enumerate}
\item   Quels que soient les réels positifs $a$ et $b$, on a $\sqrt{ab} = \sqrt a  \sqrt b$.
\end{enumerate}



\textbf{Algorithmique}

\begin{enumerate}
\item *Déterminer la première puissance d'un nombre positif donné supérieure ou inférieure à une valeur donnée.
\end{enumerate}



\section{Calcul littéral, identités remarquables}

\textbf{Contenus}

\begin{enumerate}
\item Effectuer des calculs littéraux mettant en jeu des puissances, des racines carrées, des écritures fractionnaires.
\item Identités $a^2-b^2=(a-b)(a+b)$, $(a + b)^2= a^2+ 2ab + b^2$ et $(a - b)^2= a^2- 2ab + b^2$, à savoir utiliser dans les deux sens.
\item Exemples simples de calcul sur des expressions algébriques, en particulier sur des expressions fractionnaires.
\end{enumerate}


\textbf{Capacités attendues}

\begin{enumerate}
\item  Comparer deux quantités en utilisant leur différence, ou leur quotient dans le cas positif
\end{enumerate}

\textbf{Démo}

\begin{enumerate}
\item Si $a$ et $b$ sont des réels strictement positifs, $\sqrt{a+b} < \sqrt{a}\sqrt{b}$.
\item Pour $a$ et $b$ réels positifs, illustration géométrique de l'égalité $(a + b)^2= a^2+ 2ab + b^2$
\end{enumerate}


\textbf{Approfondissement}

\begin{enumerate}
\item Développement de $(a + b + c)^2$.
\item Développement de $(a + b)^3$
\item Inégalité entre moyennes géométrique et arithmétique de deux réels strictement positifs.
\end{enumerate}




\section{Équations et inéquations}

\textbf{Contenus}

\begin{enumerate}
\item Ensemble des solutions d'une équation (équation du premier degré, équation du second degré simple, équation-produit, équation-quotient).
\item  Résoudre un système de deux équations linéaires à deux inconnues.
\item  Modéliser un problème par une inéquation.
\item  Résoudre une inéquation du premier degré.
\end{enumerate}


\textbf{Capacités attendues}

\begin{enumerate}
\item Choisir la forme la plus adaptée (factorisée, développée réduite) d'une expression en vue de la résolution d'un problème.
\item  Comparer deux quantités en utilisant leur différence, ou leur quotient dans le cas positif
\end{enumerate}

\textbf{Démo}

\begin{enumerate}
\item Si $a$ et $b$ sont des réels strictement positifs, $\sqrt{a+b} < \sqrt{a}\sqrt{b}$ .
\item Pour $a$ et $b$ réels positifs, illustration géométrique de l'égalité $(a + b)^2= a^2+ 2ab + b^2$
\end{enumerate}


\textbf{Approfondissement}

\begin{enumerate}
\item Développement de $(a + b + c)^2$.
\item Développement de $(a + b)^3$
\item Inégalité entre moyennes géométrique et arithmétique de deux réels strictement positifs.
\end{enumerate}



\section{Généralités sur les fonctions}

\textbf{Contenus}

\begin{enumerate}
\item Fonction à valeurs réelles (4 représentations : expression, algo, courbe, tableau)
\item Courbe représentative : la courbe d'équation $y = f(x)$ est l'ensemble des points du plan dont les coordonnées $(x,y)$ vérifient $y = f(x)$.
\item Signes d'une fonction , tableau de signes.
\item Croissance, décroissance, monotonie d'une fonction définie sur un intervalle. Tableau de variations.
\item Maximum, minimum d'une fonction sur un intervalle.
\end{enumerate}

\textbf{Capacités attendues}

\begin{enumerate}
\item Sur des cas simples de relations entre variables (par exemple $U = RI$, $d = vt$, $S = \pi r^2$, $V = abc$, $V = \pi r^2h$), exprimer une variable en fonction des autres. Cas d’une relation du premier degré $ax + by = c$. 
\item Exploiter l'équation $y = f(x$) d'une courbe : appartenance, calcul de coordonnées.
\item Modéliser par des fonctions des situations issues des mathématiques, des autres disciplines.
\item Relier représentation graphique et tableau de variations.
\item Déterminer graphiquement les extremums d'une fonction sur un intervalle.
\item Exploiter un logiciel de géométrie dynamique ou de calcul formel, la calculatrice ou Python pour décrire les variations d'une fonction donnée par une formule.
\item  Résoudre une équation du type $f(x) = k$ ou une inéquation du type $f(x) < k$, en choisissant une méthode adaptée : graphique, algébrique, logicielle.
\item  Résoudre, graphiquement ou à l'aide d'un outil numérique une inéquation du type $f(x) < g(x)$.

\end{enumerate}
 
 
 
\textbf{Algorithmique}

\begin{enumerate}
\item *Pour une fonction dont le tableau de variations est donné, algorithmes d'approximation numérique d'un extremum (balayage, dichotomie).
\item *Algorithme de calcul approché de longueur d'une portion de courbe représentative de fonction.
\end{enumerate}


\section{Fonctions affines}

\textbf{Contenus}

\begin{enumerate}
\item Pour une fonction affine, interprétation du coefficient directeur comme taux d'accroissement, variations
selon son signe.
\end{enumerate}

\textbf{Capacités attendues}

\begin{enumerate}
\item Somme d'inégalités. Produit d'une inégalité par un réel positif, négatif, en liaison avec le sens de variation d'une fonction affine.
\item  Relier sens de variation, signe et droite représentative d'une fonction affine.
\item  Pour les fonctions affines, résoudre graphiquement ou algébriquement une équation du type $f(x) = k$ ou une inéquation du type $f(x) < k$. 
\item Pour deux nombres $a$ et $b$ donnés et une fonction de référence $f$, comparer $f(a)$ et $f(b)$ numériquement ou graphiquement.
\item Résoudre une équation, une inéquation produit ou quotient, à l'aide d'un tableau de signes. 
\end{enumerate}
 



\section{Fonctions de référence}

\textbf{Contenus}

\begin{enumerate}
\item Fonction paire, impaire. Traduction géométrique.
\item Fonctions carré, inverse, racine carrée, cube : définitions et courbes représentatives.
\item Variations des fonctions carré, inverse, racine carrée, cube.
\end{enumerate}

\textbf{Capacités attendues}

\begin{enumerate}
\item Somme d'inégalités. Produit d'une inégalité par un réel positif, négatif, en liaison avec le sens de variation d'une fonction affine.
\item  Pour les fonctions carré, inverse, racine carrée et cube, résoudre graphiquement ou algébriquement une équation du type $f(x) = k$ ou une inéquation du type $f(x) < k$. 
\item Pour deux nombres $a$ et $b$ donnés et une fonction de référence $f$, comparer $f(a)$ et $f(b)$ numériquement ou graphiquement.
\end{enumerate}
 
\textbf{Démo}

\begin{enumerate}
\item Variations des fonctions Carré, Inverse, Racine carrée.
\item Étudier la position relative des courbes d'équation $y = x, y = x^2, y = x^3$, pour $x \geq 0$
\end{enumerate}
 
\textbf{Algorithmique}

\begin{enumerate}
\item *Pour une fonction dont le tableau de variations est donné, algorithmes d'approximation numérique d'un extremum (balayage, dichotomie).
\item *Algorithme de calcul approché de longueur d'une portion de courbe représentative de fonction.
\end{enumerate}

\textbf{Approfondissement}

\begin{enumerate}
\item Relier les courbes représentatives de la fonction racine carrée et de la fonction carré sur $\mathbb R+$.
\item Étudier la parité d'une fonction dans des cas simples.
\end{enumerate}




 
 
 

\section{Configuration du plan}

\textbf{Contenus}

\begin{enumerate}
\item Les configurations de cycle 4.
\item Projeté orthogonal d’un point sur une droite.
\end{enumerate}

\textbf{Capacités attendues}
 
\begin{enumerate}
\item Résoudre des problèmes de géométrie plane sur des figures simples ou complexes 
(triangles, quadrilatères, cercles).
\item  Calculer des longueurs, des angles, des aires et des volumes.
\item  Traiter de problèmes d’optimisation.
\end{enumerate}

\textbf{Démo}
 
\begin{enumerate}
\item Le projeté orthogonal du point M sur une droite $\Delta$ est le point de la droite $\Delta$ le plus proche du point M.
\item Relation trigonométrique $cos^2a + sin^2 a = 1$ dans un triangle rectangle
\end{enumerate}


\textbf{Approfondissement}

\begin{enumerate}
\item Démontrer que les hauteurs d'un triangle sont concourantes.
\item Expression de l'aire d’un triangle : $\dfrac12 ab sin\widehat C$.
\item Formule d'Al-Kashi.
\item Le point de concours des médiatrices est le centre du cercle circonscrit.
\end{enumerate}

\section{Géométrie vectorielle}

\textbf{Contenus}

\begin{enumerate}
\item  Vecteur $\overrightarrow{MM'}$ associé à la translation qui transforme M en M'. Direction, sens et norme.
\item  Égalité de deux vecteurs. Notation $\vec{u}$. Vecteur nul.
\item  Somme de deux vecteurs en lien avec l'enchaînement des translations. Relation de Chasles.
\end{enumerate}

\textbf{Capacités attendues}
 
\begin{enumerate}
\item Représenter géométriquement des vecteurs.
\item Construire géométriquement la somme de deux vecteurs
\end{enumerate}




\textbf{Approfondissement}


Définition vectorielle des homothéties.


\section{Géométrie analytique}

\textbf{Contenus}

\begin{enumerate}
\item Base orthonormée. Coordonnées d’un vecteur. Expression de la norme d’un vecteur.
\item Expression des coordonnées de $\overrightarrow{AB}$ en fonction de celles de $A$ et de $B$.
\item Produit d’un vecteur par un nombre réel. Colinéarité de deux vecteurs.
\item Déterminant de deux vecteurs dans une base orthonormée, critère de colinéarité. 
Application à alignement, au parallélisme
\end{enumerate}

\textbf{Capacités attendues}
 
\begin{enumerate}
\item Calculer les coordonnées d'une somme de vecteurs, d'un produit d'un vecteur par un nombre réel.
\item Calculer la distance entre deux points. Calculer les coordonnées du milieu d'un segment.
\item Caractériser alignement et parallélisme par la colinéarité de vecteurs.
\item  Résoudre des problèmes en utilisant la représentation la plus adaptée des vecteurs
\end{enumerate}

\textbf{Démo}

\begin{enumerate}
\item Deux vecteurs sont colinéaires si et seulement si leur déterminant est nul
\end{enumerate}

 
\textbf{Algorithmique}

\begin{enumerate}
\item *Étudier l'alignement de trois points dans le plan.
\end{enumerate}

 

\section{Équations de droite}

\textbf{Contenus}

\begin{enumerate}
\item Vecteur directeur d'une droite.
\item Équation de droite : équation cartésienne, équation réduite.
\item Pente (ou coefficient directeur) d'une droite non parallèle à l'axe des ordonnées
\end{enumerate}

\textbf{Capacités attendues}
 
\begin{enumerate}
\item  Déterminer une équation de droite à partir de deux points, un point et un vecteur 
directeur ou un point et la pente.
\item  Déterminer la pente ou un vecteur directeur d’une droite donnée par une équation ou 
une représentation graphique.
\item  Tracer une droite connaissant son équation cartésienne ou réduite.
\item  Établir que trois points sont alignés ou non.
\item  Déterminer si deux droites sont parallèles ou sécantes.
\item  Déterminer le point d'intersection de deux droites sécantes
\end{enumerate}

\textbf{Démo}

\begin{enumerate}
\item  En utilisant le déterminant, établir la forme générale d'une équation de droite
\end{enumerate}

\textbf{Algorithmique}

\begin{enumerate}
\item *Déterminer une équation de droite passant par deux points donnés
\end{enumerate}

\textbf{Approfondissement}

\begin{enumerate}
\item Ensemble des points équidistants d'un point et de l'axe des abscisses.
\item Représentation, sur des exemples, de parties du plan décrites par des inégalités sur les coordonnées.
\end{enumerate}


\section{Proportions et pourcentages}

\textbf{Contenus}

\begin{enumerate}
\item Proportion, pourcentage d’une sous-population dans une population.
\item  Ensembles de référence inclus les uns dans les autres : pourcentage de 
pourcentage.
\item  Évolution : variation absolue, variation relative.
\item  Évolutions successives, évolution réciproque : relation sur les coefficients 
multiplicateurs (produit, inverse).
\end{enumerate}

\textbf{Capacités attendues}
 
\begin{enumerate}
\item  Exploiter la relation entre effectifs, proportions et pourcentages.
\item  Traiter des situations simples mettant en jeu des pourcentages de pourcentages.
\item  Exploiter la relation entre deux valeurs successives et leur taux d’évolution.
\item  Calculer le taux d’évolution global à partir des taux d’évolution successifs. Calculer un taux d’évolution réciproque.
\end{enumerate}

 
 
\section{Statistiques}

\textbf{Contenus}

\begin{enumerate}
\item Indicateurs de tendance centrale d’une série statistique : moyenne pondérée.
\item  Linéarité de la moyenne.
\item  Indicateurs de dispersion : écart interquartile, écart type.
\end{enumerate}

\textbf{Capacités attendues}
 
\begin{enumerate}
\item Décrire verbalement les différences entre deux séries statistiques, en s’appuyant sur 
des indicateurs ou sur des représentations graphiques données.
\item Pour des données réelles ou issues d’une simulation, lire et comprendre une fonction 
écrite en Python renvoyant la moyenne $m$, l’écart type $s$, et la proportion d’éléments 
appartenant à $[m - 2s,m + 2s]$
\end{enumerate}
 
 

\section{Probabilités}

\textbf{Contenu}

\begin{enumerate}
\item  Ensemble (univers) des issues. Événements. Réunion, intersection, complémentaire.
\item  Loi (distribution) de probabilité. Probabilité d’un événement : somme des probabilités 
des issues.
\item   Relation $P(A \cap B) + P(A \cup B) = P(A) + P(B)$.
\item  Dénombrement à l'aide de tableaux et d'arbres.
\end{enumerate}


\textbf{Capacités attendues}
 
\begin{enumerate}
\item Utiliser des modèles théoriques de référence (dé, pièce équilibrée, tirage au sort avec équiprobabilité dans une population) en comprenant que les probabilités sont définies \textit{a priori}.
\item Construire un modèle à partir de fréquences observées, en distinguant nettement modèle et réalité.
\item  Calculer des probabilités dans des cas simples : expérience aléatoire à deux ou trois épreuves

\end{enumerate}
 
 

  
\subsection*{Échantillonnage}

\textbf{Contenus} 
 
\begin{enumerate} 
\item Échantillon aléatoire de taille n pour une expérience à deux issues.
\item Version vulgarisée de la loi des grands nombres : « Lorsque $n$ est grand, sauf exception, la fréquence observée est proche de la probabilité. »
\item Principe de l'estimation d'une probabilité, ou d'une proportion dans une population,par une fréquence observée sur un échantillon.
\end{enumerate}

\textbf{Algo}

 
\begin{enumerate} 
\item Lire et comprendre une fonction Python renvoyant le nombre ou la fréquence de succès dans un échantillon de taille $n$ pour une expérience aléatoire à deux issues.
\item Observer la loi des grands nombres à l'aide d'une simulation sur Python ou tableur.
\item *Simuler $N$ échantillons de taille $n$ d'une expérience aléatoire à deux issues. Si $p$ est la probabilité d’une issue et $f$ sa fréquence observée dans un échantillon, calculer la proportion des cas où l'écart entre $p$ et $f$ est inférieur ou égal à $\dfrac{1}{\sqrt n}$
\end{enumerate}

\end{document}