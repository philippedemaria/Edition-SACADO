\documentclass[10pt,a4paper]{article}
\usepackage[utf8]{inputenc}
\usepackage[french]{babel}
\usepackage[T1]{fontenc}
\usepackage{amsmath}
\usepackage{amsfonts}
\usepackage{amssymb}
\usepackage{lmodern}
\usepackage[left=2cm,right=2cm,top=2cm,bottom=2cm]{geometry}
\begin{document}

\fbox{{\Huge Sommaire}}



\section{Algorithmique et programmation}


\begin{enumerate} 
\item Type d'une variable (entier, flottant ou chaîne de caractères).
\item Affectation et séquence d'instructions.
\item Instruction conditionnelle.
\item Boucle bornée, une boucle non bornée.
\item Notions de fonctions.
\end{enumerate}

\section{Arithmétique}



\begin{enumerate}
\item Les ensembles $\mathbb{N}$ et $\mathbb{Z}$
\item Notions de multiple, de diviseur, de nombre pair, de nombre impair.
\item Décomposition en facteurs premiers.
\item Nombres premiers entre eux.
\item Fractionnaires irréductibles.
\end{enumerate}
 

\section{Ensembles de nombres}



\begin{enumerate}
\item Les ensembles de nombres réels.
\item La droite numérique réelle.
\item Opérations sur les ensembles. Inclusion et appartenance
\end{enumerate}


\section{Intervalles de $\mathbb{R}$}



\begin{enumerate}
\item Intervalles de $\mathbb{R}$. Notations $\infty$.
\item Intersection, réunion d'intervalles.
\item Distance entre deux nombres réels.
\item Représentation de l'intervalle $[a - r , a + r]$.
\item Encadrement décimal d'un nombre réel à $10^{-n}$ près.
\end{enumerate}



\section{Calculs numériques}

 

\begin{enumerate}
\item Rappel des règles de calcul sur les relatifs et les rationnels.
\item Règles de calcul sur les puissances.
\item Règles de calcul sur les racines carrées.
\end{enumerate}




\section{Calcul littéral, identités remarquables}

\begin{enumerate}
\item Expressions littérales : réduire, ordonner, substituer.
\item Forme développée ou factorisée d'une expression.
\item Les identités remarquables.
\end{enumerate}


\section{Équations et inéquations}


\begin{enumerate}
\item Résolution d'équations et vocabulaire.
\item Types d'équations.
\item Systèmes de deux équations.
\item Résolution d'inéquations.
\end{enumerate}



\section{Généralités sur les fonctions}

 

\begin{enumerate}
\item Généralités sur les fonctions
\item Génération de fonction
\begin{enumerate}
\item Fonction générée par une expression algébrique.
\item Fonction représentée par une courbe.
\item Fonction représentée par un tableau de valeurs.
\item Fonction générée par un algorithme.
\end{enumerate}
\item Fonction paire, impaire. Traduction géométrique.
\item Variations et extremums
\end{enumerate}

 

\section{Fonctions affines}

 

\begin{enumerate}
\item Interprétation du coefficient directeur.
\item Variations.
\item Résolution d'une inéquation produit ou quotient.
\end{enumerate}




\section{Fonctions de référence}

 

\begin{enumerate}
\item Fonction Carré.
\begin{enumerate}
\item Définition, courbe.
\item Variations.
\end{enumerate} 
\item Fonction Inverse 
\begin{enumerate}
\item Définition, courbe.
\item Variations.
\end{enumerate} 
\item Fonction Racine carrée  
\begin{enumerate}
\item Définition, courbe.
\item Variations.
\end{enumerate}
\item Fonction Cube 
\begin{enumerate}
\item Définition, courbe.
\item Variations.
\end{enumerate}
\item Parité d'une fonction.
\end{enumerate}

 
 

\section{Configuration du plan}


\begin{enumerate}
\item Géométrie euclidienne.
\begin{enumerate}
\item Le théorème de Pythagore.
\item Le théorème de Thalès.
\item Les droites remarquables du triangles.
\item Les formules trigonométriques.
\end{enumerate}
\item Grandeurs et mesures.
\item Projeté orthogonal d'un point sur une droite.
\end{enumerate}

 

\section{Géométrie vectorielle}


\begin{enumerate}
\item  Vecteur du plan.
\item  Addition et soustraction de vecteurs.
\item  Multiplication d'un vecteurs par un réel.
\item  Vecteurs colinéaires.
\end{enumerate}


\section{Géométrie analytique}


\begin{enumerate}
\item Bases de vecteurs, repère du plan. 
\item Coordonnées d'un vecteur. 
\item Expression de la norme d'un vecteur.
\item Colinéarité de deux vecteurs.
\item Déterminant de deux vecteurs dans une base orthonormée, critère de colinéarité. 
\end{enumerate}


 
\section{Équations de droite}


\begin{enumerate}
\item Vecteur directeur d'une droite.
\item Équation cartésienne d'une droite.
\item Équation réduite d'une droite.
\item Position relative de deux droites.
\end{enumerate}


\section{Proportions et pourcentages}


\begin{enumerate}
\item Proportion.
\item Variations d'une quantité, taux d'évolution.
\item Évolutions successives.
\item Évolution réciproque.
\end{enumerate}

 
\section{Statistiques}

\textbf{Contenus}

\begin{enumerate}
\item Série statistique
\item Indicateur de position d'une série statistique.
\begin{enumerate}
\item Moyenne
\item Linéarité de la moyenne.
\item Médiane.
\end{enumerate}
\item Indicateurs de dispersion  d'une série statistique.
\begin{enumerate}
\item Écart inter quartile
\item Écart type
\end{enumerate}
\end{enumerate}

 
\section{Probabilités}


\begin{enumerate}
\item Expériences aléatoires.
\item Événements d'une expérience aléatoire.
\item Probabilité d'un événement : 
\item Calculs d'une probabilité. Loi de probabilité.
\item Formules de probabilités.
\item Dénombrement à l'aide de tableaux et d'arbres.
\end{enumerate}

  
\subsection*{Échantillonnage}

 
\begin{enumerate} 
\item Fonction Python renvoyant le nombre de succès dans un échantillon de taille $n$ pour une expérience aléatoire à deux issues.
\item Loi des grands nombres à l'aide d'une simulation sur Python.
\item Simulation de $N$ échantillons de taille $n$ d'une expérience aléatoire à deux issues.  
\item Calcul de la proportion des cas où l'écart entre $p$ et $f$ est inférieur ou égal à $\dfrac{1}{\sqrt n}$
\end{enumerate}

\end{document}
