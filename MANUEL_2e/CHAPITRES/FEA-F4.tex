\begin{titre}[Fonctions et expressions algébriques]

\Titre{Notion de fonction}{4}
\end{titre}


\begin{CpsCol}
\begin{description}
\item[$\square$] Relier représentation graphique et tableau de variations.
\end{description}
\end{CpsCol}



\begin{DefT}{Fonction} \index{Fonction!Antécédent}\index{Fonction!}\index{Fonction!Ensemble de définition}

\begin{minipage}{0.48\linewidth}
Définir une fonction $f$ d'un ensemble D de réels dans $\R$, c'est associer à chaque réel $x$ de D un
unique réel noté $f(x)$.
\begin{description}
\item On dit que D est l'\textbf{ensemble de définition} de $f$. 
\item $f(x)$ est l'\textbf{image} de $x$ par $f$.
\item $x$ est un \textbf{antécédent} de $f(x)$ par $f$.
\end{description}
\end{minipage}
\begin{minipage}{0.48\linewidth}
\begin{center}
\definecolor{sqsqsq}{rgb}{0.12549019607843137,0.12549019607843137,0.12549019607843137}
\definecolor{ffqqqq}{rgb}{1.,0.,0.}
\begin{tikzpicture}[line cap=round,line join=round,>=triangle 45,x=1.0cm,y=1.0cm]
\clip(1.84,4.9) rectangle (9.52,7.);
\draw [color=ffqqqq] (4.,6.62)-- (7.,6.62);
\draw [color=ffqqqq] (7.,6.62)-- (7.,5.32);
\draw [color=ffqqqq] (7.,5.32)-- (4.,5.32);
\draw [color=ffqqqq] (4.,5.32)-- (4.,6.62);
\draw [->] (2.62,6.) -- (4.,6.);
\draw [->] (7.,5.98) -- (8.3,6.);
\draw [color=sqsqsq](2.44,6.25) node[anchor=north west] {|};
\draw [color=sqsqsq](4.58,6.2) node[anchor=north west] {fonction};
\draw [color=sqsqsq](6.34,6.2) node[anchor=north west] {$f$};
\draw [color=sqsqsq](2.06,6.2) node[anchor=north west] {$x$};
\draw [color=sqsqsq](8.36,6.2) node[anchor=north west] {$f(x)$};
\end{tikzpicture}
\end{center}
\end{minipage}
\end{DefT}
 
\begin{Nt}
$f : D \longrightarrow \R$

$x \mapsto f(x)$

Ce qui se lit : la fonction $f$ qui à $x$ associe $f(x)$.
\end{Nt}

\begin{Mt}[Déterminer algébriquement une image, un antécédent par $f$]
Soit $f$ la fonction suivante :

$f$ : $[-4 ; 5] \longrightarrow \R$

$x \mapsto 2x^2 - 6x + 3$.

\begin{description}
\item Pour déterminer l'image d'un nombre $x$ par $f$, il faut que ce nombre soit dans l'ensemble de définition de $f$. Dans ce cas, on remplace $x$ par ce nombre dans l'expression de $f(x)$.

Image de -2 : $f(-2) = 2 \times (-2)^2 -6 \times (-2) + 3 = 2 \times 4 + 12 + 3 = 8 + 12 + 3 = 23$

Image de 6 : Impossible car 6 n'appartient pas à $[-4 ; 5]$.

\item  Pour déterminer le (ou les) antécédent(s) d'un nombre $a$ par $f$, il faut et il suffit de résoudre l'équation $f(x) = 3$.

$2x^2 - 6x + 3 = 3$

$2x^2 - 6x = 0$

$2x (x - 3) = 0$

$2x = 0$ ou $x - 3 = 0$ , $x = 0$ ou  $x = 3$. $\mathscr{S}=\left\lbrace 0;3\right\rbrace $. Les antécédents de 3 par $f$ sont 0 et 3.
\end{description}
\end{Mt}




\EPC{1}{FEA-60}{Calculer. Communiquer}

\mini{
\EPC{0}{FEA-60bis}{Calculer. Communiquer}
}{
\EPC{1}{FEA-46}{Chercher. Calculer.}
}


\EPC{1}{FEA-107}{Calculer. Communiquer}

\mini{
\EPC{1}{FEA-48}{Raisonner. Communiquer}
}{

\EPC{1}{FEA-50}{Modéliser. Calculer}

\EPC{1}{FEA-61}{Raisonner. Communiquer}

\EPC{0}{FEA-45}{Raisonner. Calculer}
}


 
\EPC{0}{FEA-53}{Calculer}
 

\begin{Rqs}
\begin{description}
\item[•] Une fonction peut être donnée, sur un ensemble de définition $D$, par une \textbf{formule algébrique}, un \textbf{tableau de valeurs}, une \textbf{courbe}. 
\item[•] Le seul mode de définition qui permet le calcul d'images et d'antécédents est la formule algébrique. La courbe est imité par la précision et le tableau de valeur par le nombre de valeurs proposées.
\end{description}
\end{Rqs}


\EPCP{1}{FEA-109}{Calculer}


\end{pageCours}

\begin{pageAD}


\begin{ExoCad}{Représenter. Chercher.}{1234}{2}{0}{0}{0}{0}
 
Soit $A$ et $B$ d'abscisse respective $4$ et $-2$.

\begin{enumerate}
\item Le point $I$ est le milieu de $[AB]$. Quelle est l'abscisse du point $I$ ? 
\item Soit $M$ le point d'abscisse $x$ de la droite $(AB)$. Calculer $IM$. 
\item Compléter le tableau suivant.

\begin{tabular}{|c|p{1.5cm}|c|p{1.5cm}|c|p{1.5cm}|}
\hline 
Abscisses de $M$ & $IM$ & Abscisses de $M$ & $IM$ & Abscisses de $M$ & $IM$ \\ 
\hline 
$1$ & & $-6$ &  & $-1$ & \\ 
\hline 
$-2$ & & $3$ &  & $1$ & \\ 
\hline 
$5$ & & $0$ &  & $4$ & \\ 
\hline 
\end{tabular}

\item  A quelle condition le point $M$ appartient-il au segment $[AB]$ ?

\end{enumerate}

On pourra se rendre à la page Se rendre à la page : \url{https://www.geogebra.org/m/jsvqnzbq} pour visualiser la situation. 
\end{ExoCad}
 

\begin{ExoCad}{Chercher. Communiquer.}{1234}{2}{0}{0}{0}{0}

 
\begin{enumerate}
\item Soit $A$ et $B$ d'abscisse respective $3$ et $-1$. Déterminer le rayon de l'intervalle $[AB]$.
\item Représenter le segment $[AB]$ par un intervalle puis par une inégalité.

\item Représenter sur la droite graduée le segment $[AB]$.
 
\end{enumerate}
  
\end{ExoCad}


 
 
\begin{ExoCad}{Représenter. Chercher.}{1234}{2}{0}{0}{0}{0}

On donne les segments $[AB]$ et $[EF]$ représentés ci-dessous.

\definecolor{ttqqqq}{rgb}{0.2,0.,0.}
\definecolor{qqzzff}{rgb}{0.,0.6,1.}
\definecolor{qqzzcc}{rgb}{0.,0.6,0.8}
\begin{tikzpicture}[line cap=round,line join=round,>=triangle 45,x=1.0cm,y=1.0cm]
\begin{axis}[
x=1.0cm,y=1.0cm,
axis lines=middle,
xmin=-2.200000000000002,
xmax=8.480000000000008,
ymin=-0.8000000000000048,
ymax=0.7799999999999957,
xtick={-2.0,-1.0,...,8.0},
ytick={-0.0,1.0,...,0.0},]
\clip(-2.2,-0.8) rectangle (8.48,0.78);
\draw [line width=2.pt,color=qqzzff] (-2.,0.)-- (4.,0.);
\draw [line width=2.pt] (5.,0.)-- (8.,0.);
\begin{scriptsize}
\draw [color=qqzzcc] (-2.,0.)-- ++(-2.5pt,0 pt) -- ++(5.0pt,0 pt) ++(-2.5pt,-2.5pt) -- ++(0 pt,5.0pt);
\draw[color=qqzzcc] (-1.86,0.37) node {$A$};
\draw [color=qqzzcc] (4.,0.)-- ++(-2.5pt,0 pt) -- ++(5.0pt,0 pt) ++(-2.5pt,-2.5pt) -- ++(0 pt,5.0pt);
\draw[color=qqzzcc] (3.96,0.37) node {$B$};
\draw [color=ttqqqq] (5.,0.)-- ++(-2.5pt,0 pt) -- ++(5.0pt,0 pt) ++(-2.5pt,-2.5pt) -- ++(0 pt,5.0pt);
\draw[color=ttqqqq] (5.14,0.37) node {$E$};
\draw [color=ttqqqq] (8.,0.)-- ++(-2.5pt,0 pt) -- ++(5.0pt,0 pt) ++(-2.5pt,-2.5pt) -- ++(0 pt,5.0pt);
\draw[color=ttqqqq] (8.14,0.37) node {$F$};
\end{scriptsize}
\end{axis}
\end{tikzpicture}


\begin{enumerate}
\item 
	\begin{enumerate}
		\item Déterminer le rayon de l'intervalle $[AB]$.
		\item Représenter $[AB]$ par une inégalité.
	\end{enumerate}
\item 
	\begin{enumerate}
		\item Déterminer le rayon de l'intervalle $[EF]$.
		\item Représenter $[EF]$ par une inégalité.
	\end{enumerate}
\end{enumerate}
 
\end{ExoCad}

\begin{ExoCad}{Représenter. Chercher.}{1234}{2}{0}{0}{0}{0}

Soit $x$ un réel.   


\begin{enumerate}
	\item Déterminer puis représenter l'ensemble des points $M$ d'abscisse $x$ tel que $\vert x- 3 \vert \leq 3$.
	\item Déterminer puis représenter l'ensemble des points $M$ d'abscisse $x$ tel que $\vert x+4 \vert \leq 1$.
	\item Déterminer puis représenter l'ensemble des points $M$ d'abscisse $x$ tel que $\vert x+\frac{2}{3} \vert \leq 4$.
\end{enumerate}
 
\end{ExoCad}

\begin{ExoCad}{Représenter. Chercher.}{1234}{2}{0}{0}{0}{0}

Soit $x$ un réel.   


\begin{enumerate}
	\item Déterminer puis représenter l'ensemble des points $M$ d'abscisse $x$ tel que $\vert x - \frac{4}{5} \vert \leq \frac{1}{2}$.
    \item Déterminer puis représenter l'ensemble des points $M$ d'abscisse $x$ tel que $\vert x- \pi \vert \leq 1$.
	\item Déterminer puis représenter l'ensemble des points $M$ d'abscisse $x$ tel que $\vert x - \sqrt{2}  \vert \leq  1$.
	\item Écrire à l'aide d'une double inégalité puis représenter  l'ensemble tel  que $\vert x + \frac{2}{3} \vert \leq 5$.
	\item Écrire à l'aide d'une double inégalité puis représenter  l'ensemble tel  que $\vert x+ \pi \vert \leq 10^{-1}$.	
\end{enumerate}
 
\end{ExoCad}

\begin{ExoCad}{Représenter. Chercher.}{1234}{2}{0}{0}{0}{0}

\begin{enumerate}
\item On s'intéresse à $\frac{3}{7}$.
\begin{enumerate}
	\item 1 est-elle une valeur approchée de $\frac{3}{7}$ à $ 10^{-1}$ près ?
	\item Déterminer une valeur approchée $a$ à $ 10^{-1}$ près de $\frac{3}{7}$.	
\end{enumerate}

\item
On s'intéresse à $\sqrt{10}$.
\begin{enumerate}
	\item Déterminer un encadrement de $\sqrt{10}$ à $ 10^{-2}$ près .
	\item Déterminer à la calculatrice  une valeur approchée de  $\sqrt{10}$.	
\end{enumerate}
\end{enumerate}
  
\end{ExoCad}


\end{pageAD}

%%%%%%%%%%%%%%%%%%%%%%%%%%%%%%%%%%%%%%%%%%%%%%%%%%%%%%%%%%%%%%%%%%%%%%%%%%%%%%%%%%%%%%%%%%%%%%%%%%%%%%%%%%%%%%%%%%%%%%%%%%%
%%%%%%%%%%%%%%%%%%%%%%%%%%%%%%%%%%%%%%%%%%%%%%%%%%%%%%%%%%%%%%%%%%%%%%%%%%%%%%%%%%%%%%%%%%%%%%%%%%%%%%%%%%%%%%%%%%%%%%%%%%%
%%%%%%%%%%%%%%%              pageParcoursu                         %%%%%%%%%%%%%%%%%%%%%%%%%%%%%%%%%%%%%%%%%%%%%%%%%%%%%%%%
%%%%%%%%%%%%%%%%%%%%%%%%%%%%%%%%%%%%%%%%%%%%%%%%%%%%%%%%%%%%%%%%%%%%%%%%%%%%%%%%%%%%%%%%%%%%%%%%%%%%%%%%%%%%%%%%%%%%%%%%%%%
%%%%%%%%%%%%%%%%%%%%%%%%%%%%%%%%%%%%%%%%%%%%%%%%%%%%%%%%%%%%%%%%%%%%%%%%%%%%%%%%%%%%%%%%%%%%%%%%%%%%%%%%%%%%%%%%%%%%%%%%%%%
\begin{pageParcoursu}

\begin{ExoCu}{Représenter. Chercher.}{1234}{2}{0}{0}{0}{0}
 
\end{ExoCu}

\begin{ExoCu}{Représenter. Chercher.}{1234}{2}{0}{0}{0}{0}
 
\end{ExoCu}

\begin{ExoCu}{Représenter. Chercher.}{1234}{2}{0}{0}{0}{0}
 
\end{ExoCu}


\end{pageParcoursu}
%%%%%%%%%%%%%%%%%%%%%%%%%%%%%%%%%%%%%%%%%%%%%%%%%%%%%%%%%%%%%%%%%%%%%%%%%%%%%%%%%%%%%%%%%%%%%%%%%%%%%%%%%%%%%%%%%%%%%%%%%%%
%%%%%%%%%%%%%%%%%%%%%%%%%%%%%%%%%%%%%%%%%%%%%%%%%%%%%%%%%%%%%%%%%%%%%%%%%%%%%%%%%%%%%%%%%%%%%%%%%%%%%%%%%%%%%%%%%%%%%%%%%%%
%%%%%%%%%%%%%%%              pageParcoursd                    %%%%%%%%%%%%%%%%%%%%%%%%%%%%%%%%%%%%%%%%%%%%%%%%%%%%%%%%%%%%%
%%%%%%%%%%%%%%%%%%%%%%%%%%%%%%%%%%%%%%%%%%%%%%%%%%%%%%%%%%%%%%%%%%%%%%%%%%%%%%%%%%%%%%%%%%%%%%%%%%%%%%%%%%%%%%%%%%%%%%%%%%%
%%%%%%%%%%%%%%%%%%%%%%%%%%%%%%%%%%%%%%%%%%%%%%%%%%%%%%%%%%%%%%%%%%%%%%%%%%%%%%%%%%%%%%%%%%%%%%%%%%%%%%%%%%%%%%%%%%%%%%%%%%%

\begin{pageParcoursd}

\begin{ExoCd}{Représenter. Chercher.}{1234}{2}{0}{0}{0}{0}
 
\end{ExoCd}

\begin{ExoCd}{Représenter. Chercher.}{1234}{2}{0}{0}{0}{0}
 
\end{ExoCd}

\begin{ExoCd}{Représenter. Chercher.}{1234}{2}{0}{0}{0}{0}
 
\end{ExoCd}


\end{pageParcoursd}

%%%%%%%%%%%%%%%%%%%%%%%%%%%%%%%%%%%%%%%%%%%%%%%%%%%%%%%%%%%%%%%%%%%%%%%%%%%%%%%%%%%%%%%%%%%%%%%%%%%%%%%%%%%%%%%%%%%%%%%%%%%
%%%%%%%%%%%%%%%%%%%%%%%%%%%%%%%%%%%%%%%%%%%%%%%%%%%%%%%%%%%%%%%%%%%%%%%%%%%%%%%%%%%%%%%%%%%%%%%%%%%%%%%%%%%%%%%%%%%%%%%%%%%
%%%%%%%%%%%%%%%            pageParcourst                      %%%%%%%%%%%%%%%%%%%%%%%%%%%%%%%%%%%%%%%%%%%%%%%%%%%%%%%%%%%%%
%%%%%%%%%%%%%%%%%%%%%%%%%%%%%%%%%%%%%%%%%%%%%%%%%%%%%%%%%%%%%%%%%%%%%%%%%%%%%%%%%%%%%%%%%%%%%%%%%%%%%%%%%%%%%%%%%%%%%%%%%%%
%%%%%%%%%%%%%%%%%%%%%%%%%%%%%%%%%%%%%%%%%%%%%%%%%%%%%%%%%%%%%%%%%%%%%%%%%%%%%%%%%%%%%%%%%%%%%%%%%%%%%%%%%%%%%%%%%%%%%%%%%%%

\begin{pageParcourst}

\begin{ExoCt}{Représenter. Chercher.}{1234}{2}{0}{0}{0}{0}
 
\end{ExoCt}

\begin{ExoCt}{Représenter. Chercher.}{1234}{2}{0}{0}{0}{0}
 
\end{ExoCt}

\begin{ExoCt}{Représenter. Chercher.}{1234}{2}{0}{0}{0}{0}
 
\end{ExoCt}


\end{pageParcourst}
%%%%%%%%%%%%%%%%%%%%%%%%%%%%%%%%%%%%%%%%%%%%%%%%%%%%%%%%%%%%%%%%%%%%%%%%%%%%%%%%%%%%%%%%%%%%%%%%%%%%%%%%%%%%%%%%%%%%%%%%%%%
%%%%%%%%%%%%%%%%%%%%%%%%%%%%%%%%%%%%%%%%%%%%%%%%%%%%%%%%%%%%%%%%%%%%%%%%%%%%%%%%%%%%%%%%%%%%%%%%%%%%%%%%%%%%%%%%%%%%%%%%%%%
%%%%%%%%%%%%%%%              pageAuto                         %%%%%%%%%%%%%%%%%%%%%%%%%%%%%%%%%%%%%%%%%%%%%%%%%%%%%%%%%%%%%
%%%%%%%%%%%%%%%%%%%%%%%%%%%%%%%%%%%%%%%%%%%%%%%%%%%%%%%%%%%%%%%%%%%%%%%%%%%%%%%%%%%%%%%%%%%%%%%%%%%%%%%%%%%%%%%%%%%%%%%%%%%
%%%%%%%%%%%%%%%%%%%%%%%%%%%%%%%%%%%%%%%%%%%%%%%%%%%%%%%%%%%%%%%%%%%%%%%%%%%%%%%%%%%%%%%%%%%%%%%%%%%%%%%%%%%%%%%%%%%%%%%%%%%
\begin{pageAuto}

\begin{ExoAuto}{Représenter. Chercher.}{1234}{2}{0}{0}{0}{0}
 
\end{ExoAuto}

\begin{ExoAuto}{Représenter. Chercher.}{1234}{2}{0}{0}{0}{0}
 
\end{ExoAuto}

\begin{ExoAuto}{Représenter. Chercher.}{1234}{2}{0}{0}{0}{0}
 
\end{ExoAuto}


\end{pageAuto}