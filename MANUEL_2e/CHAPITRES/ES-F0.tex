\begin{titre}[Probabilités]

\Titre{Intervalle de fluctuation}{4}
\end{titre}


\begin{CpsCol}
\begin{description}
\item[$\square$] Mettre en œuvre une simulation
\item[$\square$] Exploiter et faire une analyse critique d'un résultat d'échatillonnage
\end{description}
\end{CpsCol}


\Rec{1}{ES-0}

\begin{DefT}{Échantillon}\index{Échantillon}
Lorsqu'on répète $n$ fois, de façon identique et indépendante, une même expérience aléatoire, on obtient une série
de $n$ résultats que l'on appelle \textbf{échantillon de taille $n$}.
\end{DefT}

\begin{Rq}
\begin{enumerate}
\item Pour obtenir un échantillon à l'aide d'un tirage, celui ci doit s'effectuer avec remise pour que la
proportion du caractère ne change pas.
\item Si la taille de l'échantillon est négligeable devant l'effectif total, on peut assimiler un tirage sans
remise à un tirage avec remise.
\end{enumerate}
\end{Rq}



\begin{Ex}
\begin{enumerate}
\item  On lance 5 fois un même dé et on note les 5 faces obtenues. On dit que l'on a un échantillon de
taille 5.
\item On choisit 10 élèves dans le lycée. On dit que l'on a un échantillon de taille 10 des élèves du lycée, car le nombre de
lycéen choisi par rapport au nombre total de lycéen est négligeable. Par contre, si on choisit 10 élèves
dans la classe, on n'a plus un échantillon de la classe.
\end{enumerate}
\end{Ex}

\begin{DefT}{Fluctuation d'échantillonnage}\index{Fluctuation d'échantillonnage}
Lorsqu'on effectue plusieurs échantillons de même taille, la fréquence observée $f$ du caractère
varie : on l'appelle la \textbf{fluctuation d'échantillonnage}.
\end{DefT}


\begin{DefT}{Intervalle de fluctuation}\index{Intervalle de fluctuation}
Pour un grand nombre de tirages d'échantillon, l'intervalle centré en $p$, qui contient au moins
95\% des fréquences observées, $f$ est appelé \textbf{intervalle de fluctuation} de la fréquence $f$ au seuil de 95\%
des échantillons.

$$I_f=\left[ p - \frac{1}{\sqrt{n}}; p + \frac{1}{\sqrt{n}} \right]$$
\end{DefT}

\begin{Ex}
On jette une pièce de monnaie équilibrée, $p=0,5$ et on réalise 200 échantillons de taille 100 et pour chaque
échantillon, on note la fréquence d'apparition du coté pile.
Au moins 190 fréquences observées appartiennent à l'intervalle de fluctuation $[0,4 ; 0,6]$.
\end{Ex}

\mini{
\EPC{1}{ES-2}{Modéliser.}

\EPC{1}{ES-4}{Modéliser.}

\EPC{1}{ES-5}{Modéliser.}
}{
\EPC{1}{ES-8}{Modéliser.}

\CR{1}{ES-3}{Communiquer.}
}

\mini{
\EPC{1}{ES-9}{Modéliser.}

\EPC{1}{ES-6}{Modéliser.}

\EPC{1}{ES-10}{Modéliser.}

\EPC{1}{ES-13}{Modéliser.}
}{
\EPC{1}{ES-7}{Modéliser.}

\CR{1}{ES-11}{Communiquer.}

\EPCP{1}{ES-14}{Représenter.}
}


