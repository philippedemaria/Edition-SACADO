\begin{titre}[Fonctions de référence]

\Titre{Les fonctions du second degré}{4}
\end{titre}


\begin{CpsCol}
\textbf{Variations de fonctions}
\begin{description}
\item[$\square$] Connaitre les variations des fonctions du second degré
\item[$\square$] Déterminer l'axe de symétrie d'une courbe d'une fonction du second degré
\item[$\square$] Déterminer l'extremum d'une fonction du second degré
\end{description}
\end{CpsCol}

\mini{
\EPCB{1}{FR-22}{Chercher. Représenter. Modéliser}

\EPCB{1}{FR-25}{Calculer}
}{
\EPCB{1}{FR-23}{Calculer. Communiquer}

\EPCB{1}{FR-24}{Calculer. Représenter}
}

\begin{DefT}{Fonction du second degré} \index{Fonctions ! Second degré}
Une \textbf{fonction polynôme du second degré} est une fonction définie sur $\R$ par $f(x) = ax^2+bx+c$, où $a \in \R^*$ et $b$ et $c$ deux réels.
\end{DefT}


\EPCC{1}{FPD-12}{Chercher. Représenter. Modéliser}


\begin{Pp}[Forme canonique] \index{Fonctions ! Forme canonique}
Soit $f$ une fonction polynôme du second degré définie sur $R$ par $f(x) = ax^2+bx+c$, où $a \in \R^*$ et $b$ et $c$ deux réels.

$f$ peut s'écrire sous la forme $f(x)=a(x-\alpha)^2+ \beta$. Cette forme est la \textbf{forme canonique} de $f$.

Par changement de variable, toute fonction du second degré a une courbe d'équation $Y=aX^2$.

\begin{description}
\item[•] $\beta$ est le minimum ou le maximum de $f$ selon le signe de $a$.
\item[•] La droite d'équation $x=\alpha$ est l'axe de symétrie de la parabole.
\end{description}
\end{Pp}

\begin{Rq}
La forme canonique d'une fonction du second degré permet de :

\begin{description}
\item[•] déterminer l' \textit{extremum}.
\item[•] obtenir les coordonnées du sommet.
\item[•] étudier les variations.
\end{description}
\end{Rq}


\Rec{1}{FR-27}

\begin{Pp}
Soit $f$ une fonction polynôme du second degré définie sur $R$ par $f(x) = ax^2+bx+c$, où $a \in \R^*$ et $b$ et $c$ deux réels.
\begin{description}
\item[•] Si $a>0$, $f$ est décroissante sur $\left]-\infty; \frac{-b}{2a} \right]$ puis croissante sur $\left[\frac{-b}{2a};+\infty \right[$.
\item[•] Si $a<0$, $f$ est croissante sur $\left]-\infty; \frac{-b}{2a} \right]$ puis décroissante sur $\left[\frac{-b}{2a};+\infty \right[$.
\end{description}
\end{Pp}

\ROC


\mini{
\Exo{1}{FR-30}

\Exo{1}{FR-40}
}{
\Exo{1}{FR-31}
}

\vspace{0.4cm}

\Exo{1}{FR-28}

\Exo{1}{FR-29}

\mini{
\App{1}{FR-36}
}{
\Exo{1}{FR-44}
}

\App{1}{FR-37}

\App{1}{FR-41}

\begin{DTL}

On souhaite construire un terrain de jeu
rectangulaire sur le sable avec une élastique
rouge de 32 m.
On souhaite déterminer les dimensions du
rectangle pour que son aire soit maximale.
On note $AB = x$ en m et $AD = d (x)$ en cm.
\begin{enumerate}
\item Montrer que $d (x) = 16 - x$ et vérifier que $x$ appartient à l'intervalle $[ 0;16 ]$.
\item  On note $S(x)$ l’aire en $cm^2$ du rectangle $ABCD$. Montrer que $S(x) = -x^2 + 16x$, pour tout $x \in [0;16]$.
\item  \begin{enumerate}
		\item Démontrer que, pour tout $x$ de $[ 0 ; 16 ]$, $S(x)= - ( x - 8)^2 + 64$.
		\item  Déterminer les variations de $S$ sur $[0 ; 16]$.
		\item  Reproduire et compléter le tableau de valeurs suivant :
		\begin{tabular}{|c|c|c|c|c|c|c|c|c|}
		\hline 
		$x$ & 0 & 1 & 2 & 4 & 8 & 10 & 15 & 16 \\ 
		\hline 
		$S(x)$ &  &  & 28 &  & & 60 & 15 &  \\ 
		\hline 
		\end{tabular}		
		\item  Dans un repère orthogonal , tracer la courbe représentative de la fonction $S$.
On prendra 1 cm pour 2 unités en abscisses et 1 cm pour 8 unités en ordonnées .
 		\end{enumerate}
\item Pour quelle valeur de $x$ l’aire du rectangle $ABCD$ est-elle maximale ? Que vaut cette aire et quelle est alors la nature de $ABCD$ ? 		
 \item 
\begin{enumerate}
		\item Établir le tableau de signes de l’expression $( 12 -x )( x - 4 )$ , pour $x \in [0;16]$.
		\item Vérifier que l'inéquation $-x^2 + 16x \geq 48$ équivaut à $( 12 - x )( x - 4 )\geq0 $. 
		\item  En déduire les valeurs de x pour lesquelles l'aire de $ABCD$ est supérieure ou égale à 48 $cm^2$ .
 \end{enumerate} 			
 \end{enumerate}
\end{DTL}
