\begin{titre}[Calcul littéral]

\Titre{Résolution d'équation, d'inéquation}{3}
\end{titre}

\begin{CpsCol}
\begin{description}
\item[$\square$] Modéliser un problème par une inéquation.
\item[$\square$] Résoudre une équation, une inéquation du premier degré
\end{description}
\end{CpsCol}



\begin{minipage}{0.48\linewidth}

\EPCN{Calculer }

Résoudre les équations suivantes dans $\R$.

\begin{enumerate}
\item $3x+7=0$
\item $3(x+7)=9$
\item $5x-4=2x+2$
\item $2(x-4)=3(5-2x)+1$
\end{enumerate}

\end{minipage}
\hfill
\begin{minipage}{0.48\linewidth}

\EPCNM{Calculer }

Résoudre les équations suivantes dans $\R$.

\begin{enumerate}
\item $(x+3)(x-7)=0$
\item $x^3-x=0$
\item $(x-3)(2-x)=0$
\item $\left( x-\sqrt{3} \right)(5-2x)=0$
\end{enumerate}
\end{minipage}

\EPCN{Représenter. Calculer. Communiquer. }

La somme de trois nombres entiers consécutifs est égale à 147. Quels sont ces trois nombres ?


\begin{minipage}{0.48\linewidth}

\EPCNM{Représenter. Raisonner. Calculer }

Résoudre les équation suivantes dans $\R$ : 

\begin{enumerate}
\item $(x+3)^2=x(x+3)$.
\item $\frac{2}{5}x+8=\frac{1}{3}(x-7)$.
\end{enumerate}
\end{minipage}
\hfill
\begin{minipage}{0.48\linewidth}

\EPCN{Calculer }

Résoudre les inéquations suivantes dans $\R$.

\begin{enumerate}
\item $2x+1>3$
\item $3x-2 \leq 7$
\item $-5x + \frac{1}{2} \geq \frac{5}{2}$
\item $2(x-4) > 3(5-2x)+1$
\end{enumerate}


\end{minipage}


\EPCN{Représenter. Raisonner. Calculer }

On considère un triangle $ABC$ et un nombre réel $x$. On a $AB=x+1$, $BC=4$ et $CA=15$.

\begin{enumerate}
\item Montrer que $x+1 \leq 19$ et $ x+5 > 15$
\item Donner le plus grand intervalle de $\R$ auquel appartient $x$.
\end{enumerate}


\EPCN{ Raisonner. Calculer }

Soit $k$ un nombre réel. On considère l'équation suivante d'inconnue $x$ : $k^2x+7=x-2k$.

\begin{enumerate}
\item Résoudre cette équation dans $\R$ en fonction de $k$.
\item Pour quelles valeurs de $k$ n'existe-t-il pas de solution ?
\item A quel plus petit ensemble de nombres appartient $k$ lorsque 0 est une solution de l'équation ?
\end{enumerate}


\EPCN{ Raisonner. Calculer }

Soit $m$ un nombre réel. On considère l'équation suivante d'inconnue $m$ : $5mx+7 = x+m$.

Résoudre cette équation dans $\R$ et discuter l'existence de solution selon la valeur de $m$.



\EPCN{ Raisonner. Représenter. Calculer }

On considère un triangle $ABC$ tel que $AB = x+9$, $AC= 2x-1$ et $BC = 3x+6$.

\begin{enumerate}
\item Pour quelle valeurs de $x$, le triangle $ABC$ est-il isocèle en $A$ ?
\item Existe-t-il un réel $x$ tel que le triangle $ABC$ est équilatéral ?
\end{enumerate}
