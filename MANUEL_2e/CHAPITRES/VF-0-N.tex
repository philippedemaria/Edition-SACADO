
\vspace{0.2cm}

\begin{minipage}{0.63\linewidth}
Un gardien est chargé de la surveillance d'une propriété rectangulaire de 5 hm sur 4 hm. Il dispose d'un talkie-walkie pour communiquer avec un autre gardien situé à l'intérieur de la propriété. La qualité de la communication dépend de la distance entre les deux gardiens.

Le schéma ci-dessous illustre cette situation : On note G la position du premier gardien qui se déplace à partir du point $A$ en direction du point $B$ jusqu'à compléter le tour de la propriété.

Le point $O$ symbolise le deuxième gardien. Le point $O$ est fixe.(1 unité = 100m)

\end{minipage}
\begin{minipage}{0.33\linewidth}
\begin{center}
\definecolor{ffqqqq}{rgb}{1.,0.,0.}
\definecolor{xdxdff}{rgb}{0.49019607843137253,0.49019607843137253,1.}
\definecolor{qqqqff}{rgb}{0.,0.,1.}
\definecolor{cqcqcq}{rgb}{0.7529411764705882,0.7529411764705882,0.7529411764705882}
\begin{tikzpicture}[line cap=round,line join=round,>=triangle 45,x=0.8cm,y=0.8cm]
\draw [color=cqcqcq,, xstep=0.8cm,ystep=0.8cm] (-2.56,-1.3) grid (3.54,3.62);
\clip(-2.56,-1.3) rectangle (3.54,3.62);
\draw (-2.,3.)-- (3.,3.);
\draw (3.,3.)-- (3.,-1.);
\draw (3.,-1.)-- (-2.,-1.);
\draw (-2.,-1.)-- (-2.,3.);
\draw [color=ffqqqq] (-0.36,3.)-- (1.,2.);
\begin{scriptsize}
\draw [color=qqqqff] (-2.,3.)-- ++(-1.5pt,0 pt) -- ++(3.0pt,0 pt) ++(-1.5pt,-1.5pt) -- ++(0 pt,3.0pt);
\draw[color=qqqqff] (-2.28,3.18) node {$A$};
\draw [color=qqqqff] (3.,3.)-- ++(-1.5pt,0 pt) -- ++(3.0pt,0 pt) ++(-1.5pt,-1.5pt) -- ++(0 pt,3.0pt);
\draw[color=qqqqff] (3.22,3.14) node {$B$};
\draw [color=qqqqff] (3.,-1.)-- ++(-1.5pt,0 pt) -- ++(3.0pt,0 pt) ++(-1.5pt,-1.5pt) -- ++(0 pt,3.0pt);
\draw[color=qqqqff] (3.2,-0.78) node {$C$};
\draw [color=qqqqff] (-2.,-1.)-- ++(-1.5pt,0 pt) -- ++(3.0pt,0 pt) ++(-1.5pt,-1.5pt) -- ++(0 pt,3.0pt);
\draw[color=qqqqff] (-2.3,-0.8) node {$D$};
\draw [color=qqqqff] (1.,2.)-- ++(-1.5pt,0 pt) -- ++(3.0pt,0 pt) ++(-1.5pt,-1.5pt) -- ++(0 pt,3.0pt);
\draw[color=qqqqff] (1.14,2.28) node {$O$};
\draw [color=xdxdff] (-0.36,3.)-- ++(-2.5pt,0 pt) -- ++(5.0pt,0 pt) ++(-2.5pt,-2.5pt) -- ++(0 pt,5.0pt);
\draw[color=xdxdff] (-0.22,3.36) node {$G$};
\end{scriptsize}
\end{tikzpicture}
\end{center}
\end{minipage}

\begin{enumerate}
\item Laquelle de ces courbes décrit la distance OG selon la distance parcourue par le gardien G sur $[AB]$? Justifier.

\begin{minipage}{0.33\linewidth}
\begin{center}
\textbf{ Courbe 1}
 \end{center} 

\begin{tikzpicture}[line cap=round,line join=round,>=triangle 45,x=0.7cm,y=0.7cm]
\begin{axis}[
x=0.7cm,y=0.7cm,
axis lines=middle,
ymajorgrids=true,
xmajorgrids=true,
xmin=-0.8906752411575632,
xmax=5.7073954983923,
ymin=-0.7160883280756861,
ymax=3.6750788643533183,
xtick={-0.0,1.0,...,5.0},
ytick={-0.0,1.0,...,3.0},]
\clip(-0.8906752411575632,-0.7160883280756861) rectangle (5.7073954983923,3.6750788643533183);
\draw[line width=2.pt] (3.0000025337620646,1.0000000000032099) -- (3.0000025337620646,1.0000000000032099);
\draw[line width=2.pt] (3.0000025337620646,1.0000000000032099) -- (3.0050025267836964,1.0000125125588286);
\draw[line width=2.pt] (3.0050025267836964,1.0000125125588286) -- (3.010002519805328,1.0000500239500303);
\draw[line width=2.pt] (3.010002519805328,1.0000500239500303) -- (3.01500251282696,1.0001125313639077);
\draw[line width=2.pt] (3.01500251282696,1.0001125313639077) -- (3.0200025058485918,1.0002000301140883);
\draw[line width=2.pt] (3.0200025058485918,1.0002000301140883) -- (3.0250024988702235,1.0003125136424893);
\draw[line width=2.pt] (3.0250024988702235,1.0003125136424893) -- (3.0300024918918553,1.0004499735217753);
\draw[line width=2.pt] (3.0300024918918553,1.0004499735217753) -- (3.035002484913487,1.000612399458511);
\draw[line width=2.pt] (3.035002484913487,1.000612399458511) -- (3.040002477935119,1.000799779297013);
\draw[line width=2.pt] (3.040002477935119,1.000799779297013) -- (3.0450024709567507,1.0010120990238895);
\draw[line width=2.pt] (3.0450024709567507,1.0010120990238895) -- (3.0500024639783825,1.0012493427732723);
\draw[line width=2.pt] (3.0500024639783825,1.0012493427732723) -- (3.0550024570000143,1.0015114928327276);
\draw[line width=2.pt] (3.0550024570000143,1.0015114928327276) -- (3.060002450021646,1.0017985296498495);
\draw[line width=2.pt] (3.060002450021646,1.0017985296498495) -- (3.065002443043278,1.0021104318395226);
\draw[line width=2.pt] (3.065002443043278,1.0021104318395226) -- (3.0700024360649096,1.0024471761918539);
\draw[line width=2.pt] (3.0700024360649096,1.0024471761918539) -- (3.0750024290865414,1.0028087376807613);
\draw[line width=2.pt] (3.0750024290865414,1.0028087376807613) -- (3.080002422108173,1.0031950894732162);
\draw[line width=2.pt] (3.080002422108173,1.0031950894732162) -- (3.085002415129805,1.0036062029391308);
\draw[line width=2.pt] (3.085002415129805,1.0036062029391308) -- (3.090002408151437,1.0040420476618785);
\draw[line width=2.pt] (3.090002408151437,1.0040420476618785) -- (3.0950024011730686,1.004502591449444);
\draw[line width=2.pt] (3.0950024011730686,1.004502591449444) -- (3.1000023941947004,1.0049878003461894);
\draw[line width=2.pt] (3.1000023941947004,1.0049878003461894) -- (3.105002387216332,1.0054976386452275);
\draw[line width=2.pt] (3.105002387216332,1.0054976386452275) -- (3.110002380237964,1.0060320689013933);
\draw[line width=2.pt] (3.110002380237964,1.0060320689013933) -- (3.1150023732595957,1.0065910519448003);
\draw[line width=2.pt] (3.1150023732595957,1.0065910519448003) -- (3.1200023662812275,1.0071745468949729);
\draw[line width=2.pt] (3.1200023662812275,1.0071745468949729) -- (3.1250023593028593,1.0077825111755419);
\draw[line width=2.pt] (3.1250023593028593,1.0077825111755419) -- (3.130002352324491,1.00841490052949);
\draw[line width=2.pt] (3.130002352324491,1.00841490052949) -- (3.135002345346123,1.0090716690349373);
\draw[line width=2.pt] (3.135002345346123,1.0090716690349373) -- (3.1400023383677547,1.0097527691214514);
\draw[line width=2.pt] (3.1400023383677547,1.0097527691214514) -- (3.1450023313893865,1.0104581515868718);
\draw[line width=2.pt] (3.1450023313893865,1.0104581515868718) -- (3.1500023244110182,1.0111877656146302);
\draw[line width=2.pt] (3.1500023244110182,1.0111877656146302) -- (3.15500231743265,1.01194155879156);
\draw[line width=2.pt] (3.15500231743265,1.01194155879156) -- (3.160002310454282,1.0127194771261725);
\draw[line width=2.pt] (3.160002310454282,1.0127194771261725) -- (3.1650023034759136,1.0135214650673945);
\draw[line width=2.pt] (3.1650023034759136,1.0135214650673945) -- (3.1700022964975454,1.014347465523742);
\draw[line width=2.pt] (3.1700022964975454,1.014347465523742) -- (3.175002289519177,1.0151974198829279);
\draw[line width=2.pt] (3.175002289519177,1.0151974198829279) -- (3.180002282540809,1.0160712680318744);
\draw[line width=2.pt] (3.180002282540809,1.0160712680318744) -- (3.1850022755624408,1.0169689483771278);
\draw[line width=2.pt] (3.1850022755624408,1.0169689483771278) -- (3.1900022685840725,1.0178903978656513);
\draw[line width=2.pt] (3.1900022685840725,1.0178903978656513) -- (3.1950022616057043,1.0188355520059846);
\draw[line width=2.pt] (3.1950022616057043,1.0188355520059846) -- (3.200002254627336,1.019804344889753);
\draw[line width=2.pt] (3.200002254627336,1.019804344889753) -- (3.205002247648968,1.0207967092135088);
\draw[line width=2.pt] (3.205002247648968,1.0207967092135088) -- (3.2100022406705997,1.021812576300895);
\draw[line width=2.pt] (3.2100022406705997,1.021812576300895) -- (3.2150022336922315,1.022851876125106);
\draw[line width=2.pt] (3.2150022336922315,1.022851876125106) -- (3.2200022267138633,1.0239145373316358);
\draw[line width=2.pt] (3.2200022267138633,1.0239145373316358) -- (3.225002219735495,1.0250004872612988);
\draw[line width=2.pt] (3.225002219735495,1.0250004872612988) -- (3.230002212757127,1.0261096519734987);
\draw[line width=2.pt] (3.230002212757127,1.0261096519734987) -- (3.2350022057787586,1.02724195626974);
\draw[line width=2.pt] (3.2350022057787586,1.02724195626974) -- (3.2400021988003904,1.028397323717357);
\draw[line width=2.pt] (3.2400021988003904,1.028397323717357) -- (3.245002191822022,1.0295756766734512);
\draw[line width=2.pt] (3.245002191822022,1.0295756766734512) -- (3.250002184843654,1.0307769363090156);
\draw[line width=2.pt] (3.250002184843654,1.0307769363090156) -- (3.255002177865286,1.032001022633233);
\draw[line width=2.pt] (3.255002177865286,1.032001022633233) -- (3.2600021708869176,1.0332478545179322);
\draw[line width=2.pt] (3.2600021708869176,1.0332478545179322) -- (3.2650021639085494,1.0345173497221851);
\draw[line width=2.pt] (3.2650021639085494,1.0345173497221851) -- (3.270002156930181,1.0358094249170309);
\draw[line width=2.pt] (3.270002156930181,1.0358094249170309) -- (3.275002149951813,1.0371239957103102);
\draw[line width=2.pt] (3.275002149951813,1.0371239957103102) -- (3.2800021429734447,1.0384609766715942);
\draw[line width=2.pt] (3.2800021429734447,1.0384609766715942) -- (3.2850021359950765,1.039820281357195);
\draw[line width=2.pt] (3.2850021359950765,1.039820281357195) -- (3.2900021290167083,1.0412018223352395);
\draw[line width=2.pt] (3.2900021290167083,1.0412018223352395) -- (3.29500212203834,1.0426055112107953);
\draw[line width=2.pt] (3.29500212203834,1.0426055112107953) -- (3.300002115059972,1.0440312586510312);
\draw[line width=2.pt] (3.300002115059972,1.0440312586510312) -- (3.3050021080816037,1.0454789744104);
\draw[line width=2.pt] (3.3050021080816037,1.0454789744104) -- (3.3100021011032355,1.0469485673558279);
\draw[line width=2.pt] (3.3100021011032355,1.0469485673558279) -- (3.3150020941248672,1.048439945491897);
\draw[line width=2.pt] (3.3150020941248672,1.048439945491897) -- (3.320002087146499,1.0499530159860084);
\draw[line width=2.pt] (3.320002087146499,1.0499530159860084) -- (3.325002080168131,1.0514876851935129);
\draw[line width=2.pt] (3.325002080168131,1.0514876851935129) -- (3.3300020731897626,1.0530438586827908);
\draw[line width=2.pt] (3.3300020731897626,1.0530438586827908) -- (3.3350020662113944,1.0546214412602768);
\draw[line width=2.pt] (3.3350020662113944,1.0546214412602768) -- (3.340002059233026,1.0562203369954104);
\draw[line width=2.pt] (3.340002059233026,1.0562203369954104) -- (3.345002052254658,1.057840449245502);
\draw[line width=2.pt] (3.345002052254658,1.057840449245502) -- (3.3500020452762898,1.0594816806805043);
\draw[line width=2.pt] (3.3500020452762898,1.0594816806805043) -- (3.3550020382979215,1.0611439333076729);
\draw[line width=2.pt] (3.3550020382979215,1.0611439333076729) -- (3.3600020313195533,1.0628271084961112);
\draw[line width=2.pt] (3.3600020313195533,1.0628271084961112) -- (3.365002024341185,1.0645311070011825);
\draw[line width=2.pt] (3.365002024341185,1.0645311070011825) -- (3.370002017362817,1.0662558289887818);
\draw[line width=2.pt] (3.370002017362817,1.0662558289887818) -- (3.3750020103844487,1.0680011740594568);
\draw[line width=2.pt] (3.3750020103844487,1.0680011740594568) -- (3.3800020034060805,1.0697670412723672);
\draw[line width=2.pt] (3.3800020034060805,1.0697670412723672) -- (3.3850019964277123,1.0715533291690733);
\draw[line width=2.pt] (3.3850019964277123,1.0715533291690733) -- (3.390001989449344,1.0733599357971426);
\draw[line width=2.pt] (3.390001989449344,1.0733599357971426) -- (3.395001982470976,1.0751867587335706);
\draw[line width=2.pt] (3.395001982470976,1.0751867587335706) -- (3.4000019754926076,1.0770336951079982);
\draw[line width=2.pt] (3.4000019754926076,1.0770336951079982) -- (3.4050019685142394,1.0789006416257285);
\draw[line width=2.pt] (3.4050019685142394,1.0789006416257285) -- (3.410001961535871,1.080787494590524);
\draw[line width=2.pt] (3.410001961535871,1.080787494590524) -- (3.415001954557503,1.082694149927184);
\draw[line width=2.pt] (3.415001954557503,1.082694149927184) -- (3.420001947579135,1.0846205032038931);
\draw[line width=2.pt] (3.420001947579135,1.0846205032038931) -- (3.4250019406007666,1.0865664496543308);
\draw[line width=2.pt] (3.4250019406007666,1.0865664496543308) -- (3.4300019336223984,1.0885318841995404);
\draw[line width=2.pt] (3.4300019336223984,1.0885318841995404) -- (3.43500192664403,1.0905167014695456);
\draw[line width=2.pt] (3.43500192664403,1.0905167014695456) -- (3.440001919665662,1.0925207958247145);
\draw[line width=2.pt] (3.440001919665662,1.0925207958247145) -- (3.4450019126872937,1.0945440613768591);
\draw[line width=2.pt] (3.4450019126872937,1.0945440613768591) -- (3.4500019057089255,1.0965863920100707);
\draw[line width=2.pt] (3.4500019057089255,1.0965863920100707) -- (3.4550018987305573,1.0986476814012818);
\draw[line width=2.pt] (3.4550018987305573,1.0986476814012818) -- (3.460001891752189,1.100727823040552);
\draw[line width=2.pt] (3.460001891752189,1.100727823040552) -- (3.465001884773821,1.1028267102510738);
\draw[line width=2.pt] (3.465001884773821,1.1028267102510738) -- (3.4700018777954527,1.104944236208892);
\draw[line width=2.pt] (3.4700018777954527,1.104944236208892) -- (3.4750018708170844,1.107080293962335);
\draw[line width=2.pt] (3.4750018708170844,1.107080293962335) -- (3.4800018638387162,1.1092347764511539);
\draw[line width=2.pt] (3.4800018638387162,1.1092347764511539) -- (3.485001856860348,1.1114075765253653);
\draw[line width=2.pt] (3.485001856860348,1.1114075765253653) -- (3.49000184988198,1.1135985869637957);
\draw[line width=2.pt] (3.49000184988198,1.1135985869637957) -- (3.4950018429036116,1.115807700492326);
\draw[line width=2.pt] (3.4950018429036116,1.115807700492326) -- (3.5000018359252434,1.11803480980183);
\draw[line width=2.pt] (3.5000018359252434,1.11803480980183) -- (3.505001828946875,1.1202798075658102);
\draw[line width=2.pt] (3.505001828946875,1.1202798075658102) -- (3.510001821968507,1.1225425864577239);
\draw[line width=2.pt] (3.510001821968507,1.1225425864577239) -- (3.5150018149901388,1.1248230391680005);
\draw[line width=2.pt] (3.5150018149901388,1.1248230391680005) -- (3.5200018080117705,1.1271210584207494);
\draw[line width=2.pt] (3.5200018080117705,1.1271210584207494) -- (3.5250018010334023,1.129436536990156);
\draw[line width=2.pt] (3.5250018010334023,1.129436536990156) -- (3.530001794055034,1.1317693677165657);
\draw[line width=2.pt] (3.530001794055034,1.1317693677165657) -- (3.535001787076666,1.1341194435222535);
\draw[line width=2.pt] (3.535001787076666,1.1341194435222535) -- (3.5400017800982977,1.1364866574268835);
\draw[line width=2.pt] (3.5400017800982977,1.1364866574268835) -- (3.5450017731199295,1.1388709025626509);
\draw[line width=2.pt] (3.5450017731199295,1.1388709025626509) -- (3.5500017661415613,1.141272072189115);
\draw[line width=2.pt] (3.5500017661415613,1.141272072189115) -- (3.555001759163193,1.143690059707716);
\draw[line width=2.pt] (3.555001759163193,1.143690059707716) -- (3.560001752184825,1.1461247586759804);
\draw[line width=2.pt] (3.560001752184825,1.1461247586759804) -- (3.5650017452064566,1.1485760628214143);
\draw[line width=2.pt] (3.5650017452064566,1.1485760628214143) -- (3.5700017382280884,1.1510438660550875);
\draw[line width=2.pt] (3.5700017382280884,1.1510438660550875) -- (3.57500173124972,1.1535280624849034);
\draw[line width=2.pt] (3.57500173124972,1.1535280624849034) -- (3.580001724271352,1.1560285464285653);
\draw[line width=2.pt] (3.580001724271352,1.1560285464285653) -- (3.585001717292984,1.1585452124262308);
\draw[line width=2.pt] (3.585001717292984,1.1585452124262308) -- (3.5900017103146156,1.1610779552528638);
\draw[line width=2.pt] (3.5900017103146156,1.1610779552528638) -- (3.5950017033362474,1.1636266699302813);
\draw[line width=2.pt] (3.5950017033362474,1.1636266699302813) -- (3.600001696357879,1.1661912517388957);
\draw[line width=2.pt] (3.600001696357879,1.1661912517388957) -- (3.605001689379511,1.1687715962291616);
\draw[line width=2.pt] (3.605001689379511,1.1687715962291616) -- (3.6100016824011427,1.1713675992327195);
\draw[line width=2.pt] (3.6100016824011427,1.1713675992327195) -- (3.6150016754227745,1.173979156873247);
\draw[line width=2.pt] (3.6150016754227745,1.173979156873247) -- (3.6200016684444063,1.1766061655770157);
\draw[line width=2.pt] (3.6200016684444063,1.1766061655770157) -- (3.625001661466038,1.1792485220831561);
\draw[line width=2.pt] (3.625001661466038,1.1792485220831561) -- (3.63000165448767,1.1819061234536359);
\draw[line width=2.pt] (3.63000165448767,1.1819061234536359) -- (3.6350016475093017,1.184578867082951);
\draw[line width=2.pt] (3.6350016475093017,1.184578867082951) -- (3.6400016405309334,1.1872666507075342);
\draw[line width=2.pt] (3.6400016405309334,1.1872666507075342) -- (3.6450016335525652,1.1899693724148859);
\draw[line width=2.pt] (3.6450016335525652,1.1899693724148859) -- (3.650001626574197,1.1926869306524248);
\draw[line width=2.pt] (3.650001626574197,1.1926869306524248) -- (3.655001619595829,1.1954192242360664);
\draw[line width=2.pt] (3.655001619595829,1.1954192242360664) -- (3.6600016126174606,1.1981661523585319);
\draw[line width=2.pt] (3.6600016126174606,1.1981661523585319) -- (3.6650016056390924,1.2009276145973873);
\draw[line width=2.pt] (3.6650016056390924,1.2009276145973873) -- (3.670001598660724,1.2037035109228211);
\draw[line width=2.pt] (3.670001598660724,1.2037035109228211) -- (3.675001591682356,1.2064937417051587);
\draw[line width=2.pt] (3.675001591682356,1.2064937417051587) -- (3.6800015847039877,1.2092982077221213);
\draw[line width=2.pt] (3.6800015847039877,1.2092982077221213) -- (3.6850015777256195,1.2121168101658304);
\draw[line width=2.pt] (3.6850015777256195,1.2121168101658304) -- (3.6900015707472513,1.2149494506495626);
\draw[line width=2.pt] (3.6900015707472513,1.2149494506495626) -- (3.695001563768883,1.217796031214256);
\draw[line width=2.pt] (3.695001563768883,1.217796031214256) -- (3.700001556790515,1.2206564543347749);
\draw[line width=2.pt] (3.700001556790515,1.2206564543347749) -- (3.7050015498121467,1.223530622925936);
\draw[line width=2.pt] (3.7050015498121467,1.223530622925936) -- (3.7100015428337785,1.2264184403482956);
\draw[line width=2.pt] (3.7100015428337785,1.2264184403482956) -- (3.7150015358554103,1.2293198104137082);
\draw[line width=2.pt] (3.7150015358554103,1.2293198104137082) -- (3.720001528877042,1.2322346373906545);
\draw[line width=2.pt] (3.720001528877042,1.2322346373906545) -- (3.725001521898674,1.2351628260093457);
\draw[line width=2.pt] (3.725001521898674,1.2351628260093457) -- (3.7300015149203056,1.2381042814666061);
\draw[line width=2.pt] (3.7300015149203056,1.2381042814666061) -- (3.7350015079419374,1.2410589094305402);
\draw[line width=2.pt] (3.7350015079419374,1.2410589094305402) -- (3.740001500963569,1.2440266160449844);
\draw[line width=2.pt] (3.740001500963569,1.2440266160449844) -- (3.745001493985201,1.2470073079337514);
\draw[line width=2.pt] (3.745001493985201,1.2470073079337514) -- (3.750001487006833,1.2500008922046657);
\draw[line width=2.pt] (3.750001487006833,1.2500008922046657) -- (3.7550014800284646,1.253007276453402);
\draw[line width=2.pt] (3.7550014800284646,1.253007276453402) -- (3.7600014730500964,1.2560263687671196);
\draw[line width=2.pt] (3.7600014730500964,1.2560263687671196) -- (3.765001466071728,1.2590580777279075);
\draw[line width=2.pt] (3.765001466071728,1.2590580777279075) -- (3.77000145909336,1.262102312416035);
\draw[line width=2.pt] (3.77000145909336,1.262102312416035) -- (3.7750014521149917,1.265158982413019);
\draw[line width=2.pt] (3.7750014521149917,1.265158982413019) -- (3.7800014451366235,1.2682279978045041);
\draw[line width=2.pt] (3.7800014451366235,1.2682279978045041) -- (3.7850014381582553,1.2713092691829668);
\draw[line width=2.pt] (3.7850014381582553,1.2713092691829668) -- (3.790001431179887,1.2744027076502427);
\draw[line width=2.pt] (3.790001431179887,1.2744027076502427) -- (3.795001424201519,1.277508224819881);
\draw[line width=2.pt] (3.795001424201519,1.277508224819881) -- (3.8000014172231507,1.2806257328193313);
\draw[line width=2.pt] (3.8000014172231507,1.2806257328193313) -- (3.8050014102447824,1.2837551442919668);
\draw[line width=2.pt] (3.8050014102447824,1.2837551442919668) -- (3.8100014032664142,1.2868963723989435);
\draw[line width=2.pt] (3.8100014032664142,1.2868963723989435) -- (3.815001396288046,1.290049330820905);
\draw[line width=2.pt] (3.815001396288046,1.290049330820905) -- (3.820001389309678,1.2932139337595314);
\draw[line width=2.pt] (3.820001389309678,1.2932139337595314) -- (3.8250013823313096,1.2963900959389392);
\draw[line width=2.pt] (3.8250013823313096,1.2963900959389392) -- (3.8300013753529414,1.299577732606932);
\draw[line width=2.pt] (3.8300013753529414,1.299577732606932) -- (3.835001368374573,1.3027767595361108);
\draw[line width=2.pt] (3.835001368374573,1.3027767595361108) -- (3.840001361396205,1.305987093024842);
\draw[line width=2.pt] (3.840001361396205,1.305987093024842) -- (3.8450013544178367,1.3092086498980897);
\draw[line width=2.pt] (3.8450013544178367,1.3092086498980897) -- (3.8500013474394685,1.3124413475081131);
\draw[line width=2.pt] (3.8500013474394685,1.3124413475081131) -- (3.8550013404611003,1.3156851037350383);
\draw[line width=2.pt] (3.8550013404611003,1.3156851037350383) -- (3.860001333482732,1.3189398369872969);
\draw[line width=2.pt] (3.860001333482732,1.3189398369872969) -- (3.865001326504364,1.3222054662019473);
\draw[line width=2.pt] (3.865001326504364,1.3222054662019473) -- (3.8700013195259957,1.3254819108448723);
\draw[line width=2.pt] (3.8700013195259957,1.3254819108448723) -- (3.8750013125476275,1.3287690909108592);
\draw[line width=2.pt] (3.8750013125476275,1.3287690909108592) -- (3.8800013055692593,1.332066926923569);
\draw[line width=2.pt] (3.8800013055692593,1.332066926923569) -- (3.885001298590891,1.3353753399353918);
\draw[line width=2.pt] (3.885001298590891,1.3353753399353918) -- (3.890001291612523,1.3386942515271958);
\draw[line width=2.pt] (3.890001291612523,1.3386942515271958) -- (3.8950012846341546,1.34202358380797);
\draw[line width=2.pt] (3.8950012846341546,1.34202358380797) -- (3.9000012776557864,1.3453632594143665);
\draw[line width=2.pt] (3.9000012776557864,1.3453632594143665) -- (3.905001270677418,1.3487132015101437);
\draw[line width=2.pt] (3.905001270677418,1.3487132015101437) -- (3.91000126369905,1.3520733337855118);
\draw[line width=2.pt] (3.91000126369905,1.3520733337855118) -- (3.9150012567206818,1.3554435804563858);
\draw[line width=2.pt] (3.9150012567206818,1.3554435804563858) -- (3.9200012497423136,1.3588238662635488);
\draw[line width=2.pt] (3.9200012497423136,1.3588238662635488) -- (3.9250012427639454,1.3622141164717254);
\draw[line width=2.pt] (3.9250012427639454,1.3622141164717254) -- (3.930001235785577,1.3656142568685714);
\draw[line width=2.pt] (3.930001235785577,1.3656142568685714) -- (3.935001228807209,1.3690242137635809);
\draw[line width=2.pt] (3.935001228807209,1.3690242137635809) -- (3.9400012218288407,1.3724439139869118);
\draw[line width=2.pt] (3.9400012218288407,1.3724439139869118) -- (3.9450012148504725,1.3758732848881357);
\draw[line width=2.pt] (3.9450012148504725,1.3758732848881357) -- (3.9500012078721043,1.3793122543349121);
\draw[line width=2.pt] (3.9500012078721043,1.3793122543349121) -- (3.955001200893736,1.3827607507115893);
\draw[line width=2.pt] (3.955001200893736,1.3827607507115893) -- (3.960001193915368,1.3862187029177364);
\draw[line width=2.pt] (3.960001193915368,1.3862187029177364) -- (3.9650011869369997,1.3896860403666067);
\draw[line width=2.pt] (3.9650011869369997,1.3896860403666067) -- (3.9700011799586314,1.3931626929835357);
\draw[line width=2.pt] (3.9700011799586314,1.3931626929835357) -- (3.9750011729802632,1.396648591204276);
\draw[line width=2.pt] (3.9750011729802632,1.396648591204276) -- (3.980001166001895,1.400143665973272);
\draw[line width=2.pt] (3.980001166001895,1.400143665973272) -- (3.985001159023527,1.403647848741874);
\draw[line width=2.pt] (3.985001159023527,1.403647848741874) -- (3.9900011520451586,1.4071610714664975);
\draw[line width=2.pt] (3.9900011520451586,1.4071610714664975) -- (3.9950011450667904,1.4106832666067264);
\draw[line width=2.pt] (3.9950011450667904,1.4106832666067264) -- (4.000001138088423,1.4142143671233651);
\draw[line width=2.pt] (4.000001138088423,1.4142143671233651) -- (4.005001131110054,1.417754306476439);
\draw[line width=2.pt] (4.005001131110054,1.417754306476439) -- (4.010001124131686,1.421303018623147);
\draw[line width=2.pt] (4.010001124131686,1.421303018623147) -- (4.015001117153318,1.424860438015767);
\draw[line width=2.pt] (4.015001117153318,1.424860438015767) -- (4.02000111017495,1.4284264995995173);
\draw[line width=2.pt] (4.02000111017495,1.4284264995995173) -- (4.0250011031965816,1.4320011388103744);
\draw[line width=2.pt] (4.0250011031965816,1.4320011388103744) -- (4.030001096218213,1.4355842915728498);
\draw[line width=2.pt] (4.030001096218213,1.4355842915728498) -- (4.035001089239845,1.4391758942977282);
\draw[line width=2.pt] (4.035001089239845,1.4391758942977282) -- (4.040001082261477,1.4427758838797673);
\draw[line width=2.pt] (4.040001082261477,1.4427758838797673) -- (4.045001075283109,1.446384197695361);
\draw[line width=2.pt] (4.045001075283109,1.446384197695361) -- (4.0500010683047405,1.4500007736001717);
\draw[line width=2.pt] (4.0500010683047405,1.4500007736001717) -- (4.055001061326372,1.4536255499267243);
\draw[line width=2.pt] (4.055001061326372,1.4536255499267243) -- (4.060001054348004,1.457258465481975);
\draw[line width=2.pt] (4.060001054348004,1.457258465481975) -- (4.065001047369636,1.4608994595448455);
\draw[line width=2.pt] (4.065001047369636,1.4608994595448455) -- (4.070001040391268,1.464548471863733);
\draw[line width=2.pt] (4.070001040391268,1.464548471863733) -- (4.075001033412899,1.468205442653991);
\draw[line width=2.pt] (4.075001033412899,1.468205442653991) -- (4.080001026434531,1.4718703125953867);
\draw[line width=2.pt] (4.080001026434531,1.4718703125953867) -- (4.085001019456163,1.475543022829532);
\draw[line width=2.pt] (4.085001019456163,1.475543022829532) -- (4.090001012477795,1.4792235149572959);
\draw[line width=2.pt] (4.090001012477795,1.4792235149572959) -- (4.095001005499427,1.482911731036192);
\draw[line width=2.pt] (4.095001005499427,1.482911731036192) -- (4.100000998521058,1.4866076135777475);
\draw[line width=2.pt] (4.100000998521058,1.4866076135777475) -- (4.10500099154269,1.4903111055448552);
\draw[line width=2.pt] (4.10500099154269,1.4903111055448552) -- (4.110000984564322,1.4940221503491051);
\draw[line width=2.pt] (4.110000984564322,1.4940221503491051) -- (4.115000977585954,1.4977406918481024);
\draw[line width=2.pt] (4.115000977585954,1.4977406918481024) -- (4.1200009706075855,1.5014666743427687);
\draw[line width=2.pt] (4.1200009706075855,1.5014666743427687) -- (4.125000963629217,1.50520004257463);
\draw[line width=2.pt] (4.125000963629217,1.50520004257463) -- (4.130000956650849,1.5089407417230916);
\draw[line width=2.pt] (4.130000956650849,1.5089407417230916) -- (4.135000949672481,1.5126887174027026);
\draw[line width=2.pt] (4.135000949672481,1.5126887174027026) -- (4.140000942694113,1.5164439156604064);
\draw[line width=2.pt] (4.140000942694113,1.5164439156604064) -- (4.1450009357157445,1.5202062829727847);
\draw[line width=2.pt] (4.1450009357157445,1.5202062829727847) -- (4.150000928737376,1.5239757662432918);
\draw[line width=2.pt] (4.150000928737376,1.5239757662432918) -- (4.155000921759008,1.5277523127994794);
\draw[line width=2.pt] (4.155000921759008,1.5277523127994794) -- (4.16000091478064,1.5315358703902175);
\draw[line width=2.pt] (4.16000091478064,1.5315358703902175) -- (4.165000907802272,1.5353263871829068);
\draw[line width=2.pt] (4.165000907802272,1.5353263871829068) -- (4.170000900823903,1.5391238117606865);
\draw[line width=2.pt] (4.170000900823903,1.5391238117606865) -- (4.175000893845535,1.5429280931196394);
\draw[line width=2.pt] (4.175000893845535,1.5429280931196394) -- (4.180000886867167,1.546739180665991);
\draw[line width=2.pt] (4.180000886867167,1.546739180665991) -- (4.185000879888799,1.5505570242133073);
\draw[line width=2.pt] (4.185000879888799,1.5505570242133073) -- (4.1900008729104306,1.5543815739796927);
\draw[line width=2.pt] (4.1900008729104306,1.5543815739796927) -- (4.195000865932062,1.5582127805849812);
\draw[line width=2.pt] (4.195000865932062,1.5582127805849812) -- (4.200000858953694,1.5620505950479335);
\draw[line width=2.pt] (4.200000858953694,1.5620505950479335) -- (4.205000851975326,1.5658949687834307);
\draw[line width=2.pt] (4.205000851975326,1.5658949687834307) -- (4.210000844996958,1.5697458535996685);
\draw[line width=2.pt] (4.210000844996958,1.5697458535996685) -- (4.2150008380185895,1.5736032016953558);
\draw[line width=2.pt] (4.2150008380185895,1.5736032016953558) -- (4.220000831040221,1.5774669656569138);
\draw[line width=2.pt] (4.220000831040221,1.5774669656569138) -- (4.225000824061853,1.581337098455677);
\draw[line width=2.pt] (4.225000824061853,1.581337098455677) -- (4.230000817083485,1.5852135534450997);
\draw[line width=2.pt] (4.230000817083485,1.5852135534450997) -- (4.235000810105117,1.5890962843579661);
\draw[line width=2.pt] (4.235000810105117,1.5890962843579661) -- (4.240000803126748,1.592985245303603);
\draw[line width=2.pt] (4.240000803126748,1.592985245303603) -- (4.24500079614838,1.5968803907651008);
\draw[line width=2.pt] (4.24500079614838,1.5968803907651008) -- (4.250000789170012,1.6007816755965358);
\draw[line width=2.pt] (4.250000789170012,1.6007816755965358) -- (4.255000782191644,1.6046890550202049);
\draw[line width=2.pt] (4.255000782191644,1.6046890550202049) -- (4.260000775213276,1.6086024846238598);
\draw[line width=2.pt] (4.260000775213276,1.6086024846238598) -- (4.265000768234907,1.6125219203579546);
\draw[line width=2.pt] (4.265000768234907,1.6125219203579546) -- (4.270000761256539,1.6164473185328958);
\draw[line width=2.pt] (4.270000761256539,1.6164473185328958) -- (4.275000754278171,1.6203786358163035);
\draw[line width=2.pt] (4.275000754278171,1.6203786358163035) -- (4.280000747299803,1.6243158292302804);
\draw[line width=2.pt] (4.280000747299803,1.6243158292302804) -- (4.2850007403214345,1.6282588561486884);
\draw[line width=2.pt] (4.2850007403214345,1.6282588561486884) -- (4.290000733343066,1.6322076742944351);
\draw[line width=2.pt] (4.290000733343066,1.6322076742944351) -- (4.295000726364698,1.636162241736771);
\draw[line width=2.pt] (4.295000726364698,1.636162241736771) -- (4.30000071938633,1.6401225168885938);
\draw[line width=2.pt] (4.30000071938633,1.6401225168885938) -- (4.305000712407962,1.6440884585037654);
\draw[line width=2.pt] (4.305000712407962,1.6440884585037654) -- (4.3100007054295935,1.648060025674439);
\draw[line width=2.pt] (4.3100007054295935,1.648060025674439) -- (4.315000698451225,1.6520371778283955);
\draw[line width=2.pt] (4.315000698451225,1.6520371778283955) -- (4.320000691472857,1.6560198747263937);
\draw[line width=2.pt] (4.320000691472857,1.6560198747263937) -- (4.325000684494489,1.6600080764595286);
\draw[line width=2.pt] (4.325000684494489,1.6600080764595286) -- (4.330000677516121,1.6640017434466046);
\draw[line width=2.pt] (4.330000677516121,1.6640017434466046) -- (4.335000670537752,1.6680008364315195);
\draw[line width=2.pt] (4.335000670537752,1.6680008364315195) -- (4.340000663559384,1.6720053164806592);
\draw[line width=2.pt] (4.340000663559384,1.6720053164806592) -- (4.345000656581016,1.6760151449803085);
\draw[line width=2.pt] (4.345000656581016,1.6760151449803085) -- (4.350000649602648,1.6800302836340693);
\draw[line width=2.pt] (4.350000649602648,1.6800302836340693) -- (4.3550006426242795,1.6840506944602975);
\draw[line width=2.pt] (4.3550006426242795,1.6840506944602975) -- (4.360000635645911,1.6880763397895495);
\draw[line width=2.pt] (4.360000635645911,1.6880763397895495) -- (4.365000628667543,1.6921071822620422);
\draw[line width=2.pt] (4.365000628667543,1.6921071822620422) -- (4.370000621689175,1.6961431848251274);
\draw[line width=2.pt] (4.370000621689175,1.6961431848251274) -- (4.375000614710807,1.7001843107307797);
\draw[line width=2.pt] (4.375000614710807,1.7001843107307797) -- (4.3800006077324385,1.7042305235330986);
\draw[line width=2.pt] (4.3800006077324385,1.7042305235330986) -- (4.38500060075407,1.7082817870858238);
\draw[line width=2.pt] (4.38500060075407,1.7082817870858238) -- (4.390000593775702,1.7123380655398641);
\draw[line width=2.pt] (4.390000593775702,1.7123380655398641) -- (4.395000586797334,1.7163993233408434);
\draw[line width=2.pt] (4.395000586797334,1.7163993233408434) -- (4.400000579818966,1.720465525226658);
\draw[line width=2.pt] (4.400000579818966,1.720465525226658) -- (4.405000572840597,1.724536636225049);
\draw[line width=2.pt] (4.405000572840597,1.724536636225049) -- (4.410000565862229,1.728612621651192);
\draw[line width=2.pt] (4.410000565862229,1.728612621651192) -- (4.415000558883861,1.7326934471052977);
\draw[line width=2.pt] (4.415000558883861,1.7326934471052977) -- (4.420000551905493,1.7367790784702308);
\draw[line width=2.pt] (4.420000551905493,1.7367790784702308) -- (4.425000544927125,1.7408694819091413);
\draw[line width=2.pt] (4.425000544927125,1.7408694819091413) -- (4.430000537948756,1.7449646238631122);
\draw[line width=2.pt] (4.430000537948756,1.7449646238631122) -- (4.435000530970388,1.7490644710488221);
\draw[line width=2.pt] (4.435000530970388,1.7490644710488221) -- (4.44000052399202,1.7531689904562229);
\draw[line width=2.pt] (4.44000052399202,1.7531689904562229) -- (4.445000517013652,1.7572781493462328);
\draw[line width=2.pt] (4.445000517013652,1.7572781493462328) -- (4.4500005100352835,1.761391915248444);
\draw[line width=2.pt] (4.4500005100352835,1.761391915248444) -- (4.455000503056915,1.7655102559588478);
\draw[line width=2.pt] (4.455000503056915,1.7655102559588478) -- (4.460000496078547,1.769633139537572);
\draw[line width=2.pt] (4.460000496078547,1.769633139537572) -- (4.465000489100179,1.7737605343066363);
\draw[line width=2.pt] (4.465000489100179,1.7737605343066363) -- (4.470000482121811,1.7778924088477221);
\draw[line width=2.pt] (4.470000482121811,1.7778924088477221) -- (4.4750004751434425,1.7820287319999588);
\draw[line width=2.pt] (4.4750004751434425,1.7820287319999588) -- (4.480000468165074,1.7861694728577238);
\draw[line width=2.pt] (4.480000468165074,1.7861694728577238) -- (4.485000461186706,1.7903146007684598);
\draw[line width=2.pt] (4.485000461186706,1.7903146007684598) -- (4.490000454208338,1.794464085330507);
\draw[line width=2.pt] (4.490000454208338,1.794464085330507) -- (4.49500044722997,1.798617896390951);
\draw[line width=2.pt] (4.49500044722997,1.798617896390951) -- (4.500000440251601,1.8027760040434857);
\draw[line width=2.pt] (4.500000440251601,1.8027760040434857) -- (4.505000433273233,1.8069383786262938);
\draw[line width=2.pt] (4.505000433273233,1.8069383786262938) -- (4.510000426294865,1.8111049907199401);
\draw[line width=2.pt] (4.510000426294865,1.8111049907199401) -- (4.515000419316497,1.8152758111452818);
\draw[line width=2.pt] (4.515000419316497,1.8152758111452818) -- (4.5200004123381285,1.8194508109613958);
\draw[line width=2.pt] (4.5200004123381285,1.8194508109613958) -- (4.52500040535976,1.8236299614635183);
\draw[line width=2.pt] (4.52500040535976,1.8236299614635183) -- (4.530000398381392,1.8278132341810032);
\draw[line width=2.pt] (4.530000398381392,1.8278132341810032) -- (4.535000391403024,1.8320006008752936);
\draw[line width=2.pt] (4.535000391403024,1.8320006008752936) -- (4.540000384424656,1.8361920335379107);
\draw[line width=2.pt] (4.540000384424656,1.8361920335379107) -- (4.5450003774462875,1.8403875043884566);
\draw[line width=2.pt] (4.5450003774462875,1.8403875043884566) -- (4.550000370467919,1.8445869858726334);
\draw[line width=2.pt] (4.550000370467919,1.8445869858726334) -- (4.555000363489551,1.8487904506602786);
\draw[line width=2.pt] (4.555000363489551,1.8487904506602786) -- (4.560000356511183,1.8529978716434128);
\draw[line width=2.pt] (4.560000356511183,1.8529978716434128) -- (4.565000349532815,1.8572092219343066);
\draw[line width=2.pt] (4.565000349532815,1.8572092219343066) -- (4.570000342554446,1.8614244748635598);
\draw[line width=2.pt] (4.570000342554446,1.8614244748635598) -- (4.575000335576078,1.865643603978198);
\draw[line width=2.pt] (4.575000335576078,1.865643603978198) -- (4.58000032859771,1.8698665830397825);
\draw[line width=2.pt] (4.58000032859771,1.8698665830397825) -- (4.585000321619342,1.8740933860225366);
\draw[line width=2.pt] (4.585000321619342,1.8740933860225366) -- (4.590000314640974,1.8783239871114874);
\draw[line width=2.pt] (4.590000314640974,1.8783239871114874) -- (4.595000307662605,1.88255836070062);
\draw[line width=2.pt] (4.595000307662605,1.88255836070062) -- (4.600000300684237,1.8867964813910507);
\draw[line width=2.pt] (4.600000300684237,1.8867964813910507) -- (4.605000293705869,1.8910383239892115);
\draw[line width=2.pt] (4.605000293705869,1.8910383239892115) -- (4.610000286727501,1.8952838635050515);
\draw[line width=2.pt] (4.610000286727501,1.8952838635050515) -- (4.6150002797491325,1.8995330751502528);
\draw[line width=2.pt] (4.6150002797491325,1.8995330751502528) -- (4.620000272770764,1.9037859343364607);
\draw[line width=2.pt] (4.620000272770764,1.9037859343364607) -- (4.625000265792396,1.9080424166735281);
\draw[line width=2.pt] (4.625000265792396,1.9080424166735281) -- (4.630000258814028,1.9123024979677765);
\draw[line width=2.pt] (4.630000258814028,1.9123024979677765) -- (4.63500025183566,1.9165661542202685);
\draw[line width=2.pt] (4.63500025183566,1.9165661542202685) -- (4.6400002448572915,1.9208333616250983);
\draw[line width=2.pt] (4.6400002448572915,1.9208333616250983) -- (4.645000237878923,1.925104096567693);
\draw[line width=2.pt] (4.645000237878923,1.925104096567693) -- (4.650000230900555,1.9293783356231315);
\draw[line width=2.pt] (4.650000230900555,1.9293783356231315) -- (4.655000223922187,1.9336560555544744);
\draw[line width=2.pt] (4.655000223922187,1.9336560555544744) -- (4.660000216943819,1.9379372333111113);
\draw[line width=2.pt] (4.660000216943819,1.9379372333111113) -- (4.66500020996545,1.9422218460271201);
\draw[line width=2.pt] (4.66500020996545,1.9422218460271201) -- (4.670000202987082,1.94650987101964);
\draw[line width=2.pt] (4.670000202987082,1.94650987101964) -- (4.675000196008714,1.9508012857872608);
\draw[line width=2.pt] (4.675000196008714,1.9508012857872608) -- (4.680000189030346,1.9550960680084233);
\draw[line width=2.pt] (4.680000189030346,1.9550960680084233) -- (4.6850001820519775,1.959394195539835);
\draw[line width=2.pt] (4.6850001820519775,1.959394195539835) -- (4.690000175073609,1.9636956464148996);
\draw[line width=2.pt] (4.690000175073609,1.9636956464148996) -- (4.695000168095241,1.9680003988421586);
\draw[line width=2.pt] (4.695000168095241,1.9680003988421586) -- (4.700000161116873,1.972308431203749);
\draw[line width=2.pt] (4.700000161116873,1.972308431203749) -- (4.705000154138505,1.9766197220538717);
\draw[line width=2.pt] (4.705000154138505,1.9766197220538717) -- (4.7100001471601365,1.9809342501172744);
\draw[line width=2.pt] (4.7100001471601365,1.9809342501172744) -- (4.715000140181768,1.9852519942877491);
\draw[line width=2.pt] (4.715000140181768,1.9852519942877491) -- (4.7200001332034,1.98957293362664);
\draw[line width=2.pt] (4.7200001332034,1.98957293362664) -- (4.725000126225032,1.9938970473613666);
\draw[line width=2.pt] (4.725000126225032,1.9938970473613666) -- (4.730000119246664,1.9982243148839598);
\draw[line width=2.pt] (4.730000119246664,1.9982243148839598) -- (4.735000112268295,2.0025547157496093);
\draw[line width=2.pt] (4.735000112268295,2.0025547157496093) -- (4.740000105289927,2.0068882296752246);
\draw[line width=2.pt] (4.740000105289927,2.0068882296752246) -- (4.745000098311559,2.011224836538011);
\draw[line width=2.pt] (4.745000098311559,2.011224836538011) -- (4.750000091333191,2.0155645163740545);
\draw[line width=2.pt] (4.750000091333191,2.0155645163740545) -- (4.755000084354823,2.01990724937692);
\draw[line width=2.pt] (4.755000084354823,2.01990724937692) -- (4.760000077376454,2.0242530158962655);
\draw[line width=2.pt] (4.760000077376454,2.0242530158962655) -- (4.765000070398086,2.028601796436464);
\draw[line width=2.pt] (4.765000070398086,2.028601796436464) -- (4.770000063419718,2.032953571655242);
\draw[line width=2.pt] (4.770000063419718,2.032953571655242) -- (4.77500005644135,2.037308322362326);
\draw[line width=2.pt] (4.77500005644135,2.037308322362326) -- (4.7800000494629815,2.0416660295181033);
\draw[line width=2.pt] (4.7800000494629815,2.0416660295181033) -- (4.785000042484613,2.0460266742322966);
\draw[line width=2.pt] (4.785000042484613,2.0460266742322966) -- (4.790000035506245,2.0503902377626457);
\draw[line width=2.pt] (4.790000035506245,2.0503902377626457) -- (4.795000028527877,2.0547567015136075);
\draw[line width=2.pt] (4.795000028527877,2.0547567015136075) -- (4.800000021549509,2.05912604703506);
\draw[line width=2.pt] (4.800000021549509,2.05912604703506) -- (4.8050000145711405,2.063498256021026);
\draw[line width=2.pt] (4.8050000145711405,2.063498256021026) -- (4.810000007592772,2.0678733103084035);
\draw[line width=2.pt] (4.810000007592772,2.0678733103084035) -- (4.815000000614404,2.0722511918757065);
\draw[line width=2.pt] (4.815000000614404,2.0722511918757065) -- (4.819999993636036,2.0766318828418218);
\draw[line width=2.pt] (4.819999993636036,2.0766318828418218) -- (4.824999986657668,2.0810153654647743);
\draw[line width=2.pt] (4.824999986657668,2.0810153654647743) -- (4.829999979679299,2.085401622140502);
\draw[line width=2.pt] (4.829999979679299,2.085401622140502) -- (4.834999972700931,2.0897906354016467);
\draw[line width=2.pt] (4.834999972700931,2.0897906354016467) -- (4.839999965722563,2.0941823879163515);
\draw[line width=2.pt] (4.839999965722563,2.0941823879163515) -- (4.844999958744195,2.098576862487071);
\draw[line width=2.pt] (4.844999958744195,2.098576862487071) -- (4.8499999517658265,2.1029740420493925);
\draw[line width=2.pt] (4.8499999517658265,2.1029740420493925) -- (4.854999944787458,2.1073739096708666);
\draw[line width=2.pt] (4.854999944787458,2.1073739096708666) -- (4.85999993780909,2.1117764485498505);
\draw[line width=2.pt] (4.85999993780909,2.1117764485498505) -- (4.864999930830722,2.1161816420143613);
\draw[line width=2.pt] (4.864999930830722,2.1161816420143613) -- (4.869999923852354,2.120589473520938);
\draw[line width=2.pt] (4.869999923852354,2.120589473520938) -- (4.8749999168739855,2.124999926653517);
\draw[line width=2.pt] (4.8749999168739855,2.124999926653517) -- (4.879999909895617,2.129412985122315);
\draw[line width=2.pt] (4.879999909895617,2.129412985122315) -- (4.884999902917249,2.1338286327627247);
\draw[line width=2.pt] (4.884999902917249,2.1338286327627247) -- (4.889999895938881,2.138246853534217);
\draw[line width=2.pt] (4.889999895938881,2.138246853534217) -- (4.894999888960513,2.14266763151926);
\draw[line width=2.pt] (4.894999888960513,2.14266763151926) -- (4.899999881982144,2.1470909509222387);
\draw[line width=2.pt] (4.899999881982144,2.1470909509222387) -- (4.904999875003776,2.151516796068393);
\draw[line width=2.pt] (4.904999875003776,2.151516796068393) -- (4.909999868025408,2.1559451514027614);
\draw[line width=2.pt] (4.909999868025408,2.1559451514027614) -- (4.91499986104704,2.160376001489135);
\draw[line width=2.pt] (4.91499986104704,2.160376001489135) -- (4.919999854068672,2.1648093310090197);
\draw[line width=2.pt] (4.919999854068672,2.1648093310090197) -- (4.924999847090303,2.169245124760614);
\draw[line width=2.pt] (4.924999847090303,2.169245124760614) -- (4.929999840111935,2.1736833676577865);
\draw[line width=2.pt] (4.929999840111935,2.1736833676577865) -- (4.934999833133567,2.178124044729072);
\draw[line width=2.pt] (4.934999833133567,2.178124044729072) -- (4.939999826155199,2.182567141116672);
\draw[line width=2.pt] (4.939999826155199,2.182567141116672) -- (4.9449998191768305,2.187012642075464);
\draw[line width=2.pt] (4.9449998191768305,2.187012642075464) -- (4.949999812198462,2.1914605329720267);
\draw[line width=2.pt] (4.949999812198462,2.1914605329720267) -- (4.954999805220094,2.1959107992836606);
\draw[line width=2.pt] (4.954999805220094,2.1959107992836606) -- (4.959999798241726,2.200363426597435);
\draw[line width=2.pt] (4.959999798241726,2.200363426597435) -- (4.964999791263358,2.2048184006092293);
\draw[line width=2.pt] (4.964999791263358,2.2048184006092293) -- (4.9699997842849895,2.2092757071227904);
\draw[line width=2.pt] (4.9699997842849895,2.2092757071227904) -- (4.974999777306621,2.2137353320487985);
\draw[line width=2.pt] (4.974999777306621,2.2137353320487985) -- (4.979999770328253,2.218197261403939);
\draw[line width=2.pt] (4.979999770328253,2.218197261403939) -- (4.984999763349885,2.2226614813099856);
\draw[line width=2.pt] (4.984999763349885,2.2226614813099856) -- (4.989999756371517,2.227127977992889);
\draw[line width=2.pt] (4.989999756371517,2.227127977992889) -- (4.994999749393148,2.2315967377818793);
\draw[line width=2.pt] (4.994999749393148,2.2315967377818793) -- (4.99999974241478,2.2360677471085677);
\draw[line width=2.pt] (4.926045012605778E-6,3.162272986912134) -- (0.0,3.162272986912134);
\draw[line width=2.pt] (0.0,3.162272986912134) -- (0.007499982071786537,3.1551634438330383);
\draw[line width=2.pt] (0.007499982071786537,3.1551634438330383) -- (0.014999964143573075,3.1480510183386277);
\draw[line width=2.pt] (0.014999964143573075,3.1480510183386277) -- (0.022499946215359612,3.1409403958508246);
\draw[line width=2.pt] (0.022499946215359612,3.1409403958508246) -- (0.02999992828714615,3.1338315886426247);
\draw[line width=2.pt] (0.02999992828714615,3.1338315886426247) -- (0.03749991035893269,3.126724609095488);
\draw[line width=2.pt] (0.03749991035893269,3.126724609095488) -- (0.044999892430719224,3.1196194697005053);
\draw[line width=2.pt] (0.044999892430719224,3.1196194697005053) -- (0.05249987450250576,3.112516183059575);
\draw[line width=2.pt] (0.05249987450250576,3.112516183059575) -- (0.0599998565742923,3.1054147618865957);
\draw[line width=2.pt] (0.0599998565742923,3.1054147618865957) -- (0.06749983864607884,3.0983152190086747);
\draw[line width=2.pt] (0.06749983864607884,3.0983152190086747) -- (0.07499982071786537,3.0912175673673508);
\draw[line width=2.pt] (0.07499982071786537,3.0912175673673508) -- (0.08249980278965191,3.0841218200198286);
\draw[line width=2.pt] (0.08249980278965191,3.0841218200198286) -- (0.08999978486143845,3.077027990140238);
\draw[line width=2.pt] (0.08999978486143845,3.077027990140238) -- (0.09749976693322498,3.0699360910209);
\draw[line width=2.pt] (0.09749976693322498,3.0699360910209) -- (0.10499974900501152,3.0628461360736106);
\draw[line width=2.pt] (0.10499974900501152,3.0628461360736106) -- (0.11249973107679806,3.055758138830946);
\draw[line width=2.pt] (0.11249973107679806,3.055758138830946) -- (0.1199997131485846,3.0486721129475756);
\draw[line width=2.pt] (0.1199997131485846,3.0486721129475756) -- (0.12749969522037113,3.0415880722016024);
\draw[line width=2.pt] (0.12749969522037113,3.0415880722016024) -- (0.13499967729215767,3.0345060304959093);
\draw[line width=2.pt] (0.13499967729215767,3.0345060304959093) -- (0.1424996593639442,3.0274260018595296);
\draw[line width=2.pt] (0.1424996593639442,3.0274260018595296) -- (0.14999964143573075,3.0203480004490317);
\draw[line width=2.pt] (0.14999964143573075,3.0203480004490317) -- (0.15749962350751728,3.0132720405499245);
\draw[line width=2.pt] (0.15749962350751728,3.0132720405499245) -- (0.16499960557930382,3.006198136578077);
\draw[line width=2.pt] (0.16499960557930382,3.006198136578077) -- (0.17249958765109036,2.9991263030811584);
\draw[line width=2.pt] (0.17249958765109036,2.9991263030811584) -- (0.1799995697228769,2.9920565547400937);
\draw[line width=2.pt] (0.1799995697228769,2.9920565547400937) -- (0.18749955179466343,2.984988906370544);
\draw[line width=2.pt] (0.18749955179466343,2.984988906370544) -- (0.19499953386644997,2.9779233729244);
\draw[line width=2.pt] (0.19499953386644997,2.9779233729244) -- (0.2024995159382365,2.970859969491292);
\draw[line width=2.pt] (0.2024995159382365,2.970859969491292) -- (0.20999949801002304,2.963798711300132);
\draw[line width=2.pt] (0.20999949801002304,2.963798711300132) -- (0.21749948008180958,2.9567396137206603);
\draw[line width=2.pt] (0.21749948008180958,2.9567396137206603) -- (0.22499946215359612,2.9496826922650223);
\draw[line width=2.pt] (0.22499946215359612,2.9496826922650223) -- (0.23249944422538266,2.9426279625893614);
\draw[line width=2.pt] (0.23249944422538266,2.9426279625893614) -- (0.2399994262971692,2.935575440495433);
\draw[line width=2.pt] (0.2399994262971692,2.935575440495433) -- (0.24749940836895573,2.9285251419322402);
\draw[line width=2.pt] (0.24749940836895573,2.9285251419322402) -- (0.25499939044074227,2.921477082997691);
\draw[line width=2.pt] (0.25499939044074227,2.921477082997691) -- (0.2624993725125288,2.914431279940273);
\draw[line width=2.pt] (0.2624993725125288,2.914431279940273) -- (0.26999935458431534,2.9073877491607574);
\draw[line width=2.pt] (0.26999935458431534,2.9073877491607574) -- (0.2774993366561019,2.9003465072139165);
\draw[line width=2.pt] (0.2774993366561019,2.9003465072139165) -- (0.2849993187278884,2.8933075708102707);
\draw[line width=2.pt] (0.2849993187278884,2.8933075708102707) -- (0.29249930079967495,2.886270956817854);
\draw[line width=2.pt] (0.29249930079967495,2.886270956817854) -- (0.2999992828714615,2.8792366822640028);
\draw[line width=2.pt] (0.2999992828714615,2.8792366822640028) -- (0.30749926494324803,2.8722047643371718);
\draw[line width=2.pt] (0.30749926494324803,2.8722047643371718) -- (0.31499924701503457,2.865175220388769);
\draw[line width=2.pt] (0.31499924701503457,2.865175220388769) -- (0.3224992290868211,2.8581480679350166);
\draw[line width=2.pt] (0.3224992290868211,2.8581480679350166) -- (0.32999921115860764,2.851123324658836);
\draw[line width=2.pt] (0.32999921115860764,2.851123324658836) -- (0.3374991932303942,2.8441010084117617);
\draw[line width=2.pt] (0.3374991932303942,2.8441010084117617) -- (0.3449991753021807,2.837081137215871);
\draw[line width=2.pt] (0.3449991753021807,2.837081137215871) -- (0.35249915737396725,2.8300637292657482);
\draw[line width=2.pt] (0.35249915737396725,2.8300637292657482) -- (0.3599991394457538,2.82304880293047);
\draw[line width=2.pt] (0.3599991394457538,2.82304880293047) -- (0.36749912151754033,2.81603637675562);
\draw[line width=2.pt] (0.36749912151754033,2.81603637675562) -- (0.37499910358932687,2.809026469465327);
\draw[line width=2.pt] (0.37499910358932687,2.809026469465327) -- (0.3824990856611134,2.802019099964329);
\draw[line width=2.pt] (0.3824990856611134,2.802019099964329) -- (0.38999906773289994,2.7950142873400723);
\draw[line width=2.pt] (0.38999906773289994,2.7950142873400723) -- (0.3974990498046865,2.788012050864829);
\draw[line width=2.pt] (0.3974990498046865,2.788012050864829) -- (0.404999031876473,2.7810124099978486);
\draw[line width=2.pt] (0.404999031876473,2.7810124099978486) -- (0.41249901394825955,2.7740153843875364);
\draw[line width=2.pt] (0.41249901394825955,2.7740153843875364) -- (0.4199989960200461,2.7670209938736585);
\draw[line width=2.pt] (0.4199989960200461,2.7670209938736585) -- (0.4274989780918326,2.7600292584895842);
\draw[line width=2.pt] (0.4274989780918326,2.7600292584895842) -- (0.43499896016361916,2.7530401984645474);
\draw[line width=2.pt] (0.43499896016361916,2.7530401984645474) -- (0.4424989422354057,2.746053834225946);
\draw[line width=2.pt] (0.4424989422354057,2.746053834225946) -- (0.44999892430719224,2.7390701864016695);
\draw[line width=2.pt] (0.44999892430719224,2.7390701864016695) -- (0.4574989063789788,2.7320892758224593);
\draw[line width=2.pt] (0.4574989063789788,2.7320892758224593) -- (0.4649988884507653,2.7251111235242966);
\draw[line width=2.pt] (0.4649988884507653,2.7251111235242966) -- (0.47249887052255185,2.7181357507508297);
\draw[line width=2.pt] (0.47249887052255185,2.7181357507508297) -- (0.4799988525943384,2.7111631789558244);
\draw[line width=2.pt] (0.4799988525943384,2.7111631789558244) -- (0.4874988346661249,2.7041934298056565);
\draw[line width=2.pt] (0.4874988346661249,2.7041934298056565) -- (0.49499881673791146,2.697226525181833);
\draw[line width=2.pt] (0.49499881673791146,2.697226525181833) -- (0.502498798809698,2.6902624871835465);
\draw[line width=2.pt] (0.502498798809698,2.6902624871835465) -- (0.5099987808814845,2.6833013381302693);
\draw[line width=2.pt] (0.5099987808814845,2.6833013381302693) -- (0.517498762953271,2.6763431005643765);
\draw[line width=2.pt] (0.517498762953271,2.6763431005643765) -- (0.5249987450250575,2.6693877972538083);
\draw[line width=2.pt] (0.5249987450250575,2.6693877972538083) -- (0.532498727096844,2.6624354511947694);
\draw[line width=2.pt] (0.532498727096844,2.6624354511947694) -- (0.5399987091686305,2.6554860856144598);
\draw[line width=2.pt] (0.5399987091686305,2.6554860856144598) -- (0.547498691240417,2.648539723973848);
\draw[line width=2.pt] (0.547498691240417,2.648539723973848) -- (0.5549986733122034,2.6415963899704824);
\draw[line width=2.pt] (0.5549986733122034,2.6415963899704824) -- (0.5624986553839899,2.6346561075413346);
\draw[line width=2.pt] (0.5624986553839899,2.6346561075413346) -- (0.5699986374557764,2.627718900865689);
\draw[line width=2.pt] (0.5699986374557764,2.627718900865689) -- (0.5774986195275629,2.6207847943680656);
\draw[line width=2.pt] (0.5774986195275629,2.6207847943680656) -- (0.5849986015993494,2.613853812721189);
\draw[line width=2.pt] (0.5849986015993494,2.613853812721189) -- (0.5924985836711358,2.6069259808489935);
\draw[line width=2.pt] (0.5924985836711358,2.6069259808489935) -- (0.5999985657429223,2.6000013239296687);
\draw[line width=2.pt] (0.5999985657429223,2.6000013239296687) -- (0.6074985478147088,2.5930798673987514);
\draw[line width=2.pt] (0.6074985478147088,2.5930798673987514) -- (0.6149985298864953,2.586161636952257);
\draw[line width=2.pt] (0.6149985298864953,2.586161636952257) -- (0.6224985119582818,2.5792466585498537);
\draw[line width=2.pt] (0.6224985119582818,2.5792466585498537) -- (0.6299984940300682,2.57233495841808);
\draw[line width=2.pt] (0.6299984940300682,2.57233495841808) -- (0.6374984761018547,2.5654265630536104);
\draw[line width=2.pt] (0.6374984761018547,2.5654265630536104) -- (0.6449984581736412,2.5585214992265604);
\draw[line width=2.pt] (0.6449984581736412,2.5585214992265604) -- (0.6524984402454277,2.551619793983843);
\draw[line width=2.pt] (0.6524984402454277,2.551619793983843) -- (0.6599984223172142,2.5447214746525657);
\draw[line width=2.pt] (0.6599984223172142,2.5447214746525657) -- (0.6674984043890007,2.5378265688434776);
\draw[line width=2.pt] (0.6674984043890007,2.5378265688434776) -- (0.6749983864607871,2.530935104454467);
\draw[line width=2.pt] (0.6749983864607871,2.530935104454467) -- (0.6824983685325736,2.524047109674101);
\draw[line width=2.pt] (0.6824983685325736,2.524047109674101) -- (0.6899983506043601,2.5171626129852194);
\draw[line width=2.pt] (0.6899983506043601,2.5171626129852194) -- (0.6974983326761466,2.5102816431685757);
\draw[line width=2.pt] (0.6974983326761466,2.5102816431685757) -- (0.7049983147479331,2.503404229306531);
\draw[line width=2.pt] (0.7049983147479331,2.503404229306531) -- (0.7124982968197195,2.4965304007867966);
\draw[line width=2.pt] (0.7124982968197195,2.4965304007867966) -- (0.719998278891506,2.4896601873062307);
\draw[line width=2.pt] (0.719998278891506,2.4896601873062307) -- (0.7274982609632925,2.482793618874686);
\draw[line width=2.pt] (0.7274982609632925,2.482793618874686) -- (0.734998243035079,2.4759307258189147);
\draw[line width=2.pt] (0.734998243035079,2.4759307258189147) -- (0.7424982251068655,2.469071538786524);
\draw[line width=2.pt] (0.7424982251068655,2.469071538786524) -- (0.749998207178652,2.4622160887499867);
\draw[line width=2.pt] (0.749998207178652,2.4622160887499867) -- (0.7574981892504384,2.4553644070107117);
\draw[line width=2.pt] (0.7574981892504384,2.4553644070107117) -- (0.7649981713222249,2.448516525203168);
\draw[line width=2.pt] (0.7649981713222249,2.448516525203168) -- (0.7724981533940114,2.4416724752990704);
\draw[line width=2.pt] (0.7724981533940114,2.4416724752990704) -- (0.7799981354657979,2.434832289611614);
\draw[line width=2.pt] (0.7799981354657979,2.434832289611614) -- (0.7874981175375844,2.4279960007997814);
\draw[line width=2.pt] (0.7874981175375844,2.4279960007997814) -- (0.7949980996093708,2.421163641872702);
\draw[line width=2.pt] (0.7949980996093708,2.421163641872702) -- (0.8024980816811573,2.4143352461940726);
\draw[line width=2.pt] (0.8024980816811573,2.4143352461940726) -- (0.8099980637529438,2.4075108474866433);
\draw[line width=2.pt] (0.8099980637529438,2.4075108474866433) -- (0.8174980458247303,2.4006904798367636);
\draw[line width=2.pt] (0.8174980458247303,2.4006904798367636) -- (0.8249980278965168,2.3938741776989954);
\draw[line width=2.pt] (0.8249980278965168,2.3938741776989954) -- (0.8324980099683033,2.387061975900786);
\draw[line width=2.pt] (0.8324980099683033,2.387061975900786) -- (0.8399979920400897,2.3802539096472133);
\draw[line width=2.pt] (0.8399979920400897,2.3802539096472133) -- (0.8474979741118762,2.373450014525791);
\draw[line width=2.pt] (0.8474979741118762,2.373450014525791) -- (0.8549979561836627,2.3666503265113463);
\draw[line width=2.pt] (0.8549979561836627,2.3666503265113463) -- (0.8624979382554492,2.359854881970967);
\draw[line width=2.pt] (0.8624979382554492,2.359854881970967) -- (0.8699979203272357,2.35306371766901);
\draw[line width=2.pt] (0.8699979203272357,2.35306371766901) -- (0.8774979023990221,2.3462768707721926);
\draw[line width=2.pt] (0.8774979023990221,2.3462768707721926) -- (0.8849978844708086,2.339494378854746);
\draw[line width=2.pt] (0.8849978844708086,2.339494378854746) -- (0.8924978665425951,2.332716279903648);
\draw[line width=2.pt] (0.8924978665425951,2.332716279903648) -- (0.8999978486143816,2.3259426123239213);
\draw[line width=2.pt] (0.8999978486143816,2.3259426123239213) -- (0.9074978306861681,2.3191734149440166);
\draw[line width=2.pt] (0.9074978306861681,2.3191734149440166) -- (0.9149978127579546,2.3124087270212663);
\draw[line width=2.pt] (0.9149978127579546,2.3124087270212663) -- (0.922497794829741,2.305648588247413);
\draw[line width=2.pt] (0.922497794829741,2.305648588247413) -- (0.9299977769015275,2.2988930387542217);
\draw[line width=2.pt] (0.9299977769015275,2.2988930387542217) -- (0.937497758973314,2.292142119119166);
\draw[line width=2.pt] (0.937497758973314,2.292142119119166) -- (0.9449977410451005,2.2853958703712007);
\draw[line width=2.pt] (0.9449977410451005,2.2853958703712007) -- (0.952497723116887,2.2786543339966094);
\draw[line width=2.pt] (0.952497723116887,2.2786543339966094) -- (0.9599977051886734,2.271917551944938);
\draw[line width=2.pt] (0.9599977051886734,2.271917551944938) -- (0.9674976872604599,2.2651855666350116);
\draw[line width=2.pt] (0.9674976872604599,2.2651855666350116) -- (0.9749976693322464,2.25845842096104);
\draw[line width=2.pt] (0.9749976693322464,2.25845842096104) -- (0.9824976514040329,2.251736158298801);
\draw[line width=2.pt] (0.9824976514040329,2.251736158298801) -- (0.9899976334758194,2.24501882251192);
\draw[line width=2.pt] (0.9899976334758194,2.24501882251192) -- (0.9974976155476059,2.2383064579582315);
\draw[line width=2.pt] (0.9974976155476059,2.2383064579582315) -- (1.0049975976193923,2.2315991094962366);
\draw[line width=2.pt] (1.0049975976193923,2.2315991094962366) -- (1.0124975796911788,2.2248968224916457);
\draw[line width=2.pt] (1.0124975796911788,2.2248968224916457) -- (1.0199975617629653,2.2181996428240183);
\draw[line width=2.pt] (1.0199975617629653,2.2181996428240183) -- (1.0274975438347518,2.2115076168934933);
\draw[line width=2.pt] (1.0274975438347518,2.2115076168934933) -- (1.0349975259065383,2.2048207916276157);
\draw[line width=2.pt] (1.0349975259065383,2.2048207916276157) -- (1.0424975079783247,2.1981392144882608);
\draw[line width=2.pt] (1.0424975079783247,2.1981392144882608) -- (1.0499974900501112,2.1914629334786535);
\draw[line width=2.pt] (1.0499974900501112,2.1914629334786535) -- (1.0574974721218977,2.184791997150488);
\draw[line width=2.pt] (1.0574974721218977,2.184791997150488) -- (1.0649974541936842,2.1781264546111467);
\draw[line width=2.pt] (1.0649974541936842,2.1781264546111467) -- (1.0724974362654707,2.171466355531023);
\draw[line width=2.pt] (1.0724974362654707,2.171466355531023) -- (1.0799974183372572,2.1648117501509447);
\draw[line width=2.pt] (1.0799974183372572,2.1648117501509447) -- (1.0874974004090436,2.158162689289704);
\draw[line width=2.pt] (1.0874974004090436,2.158162689289704) -- (1.0949973824808301,2.151519224351688);
\draw[line width=2.pt] (1.0949973824808301,2.151519224351688) -- (1.1024973645526166,2.144881407334626);
\draw[line width=2.pt] (1.1024973645526166,2.144881407334626) -- (1.109997346624403,2.138249290837436);
\draw[line width=2.pt] (1.109997346624403,2.138249290837436) -- (1.1174973286961896,2.131622928068185);
\draw[line width=2.pt] (1.1174973286961896,2.131622928068185) -- (1.124997310767976,2.1250023728521628);
\draw[line width=2.pt] (1.124997310767976,2.1250023728521628) -- (1.1324972928397625,2.118387679640064);
\draw[line width=2.pt] (1.1324972928397625,2.118387679640064) -- (1.139997274911549,2.11177890351629);
\draw[line width=2.pt] (1.139997274911549,2.11177890351629) -- (1.1474972569833355,2.1051761002073595);
\draw[line width=2.pt] (1.1474972569833355,2.1051761002073595) -- (1.154997239055122,2.0985793260904444);
\draw[line width=2.pt] (1.154997239055122,2.0985793260904444) -- (1.1624972211269085,2.091988638202018);
\draw[line width=2.pt] (1.1624972211269085,2.091988638202018) -- (1.169997203198695,2.085404094246628);
\draw[line width=2.pt] (1.169997203198695,2.085404094246628) -- (1.1774971852704814,2.078825752605787);
\draw[line width=2.pt] (1.1774971852704814,2.078825752605787) -- (1.184997167342268,2.0722536723469913);
\draw[line width=2.pt] (1.184997167342268,2.0722536723469913) -- (1.1924971494140544,2.065687913232858);
\draw[line width=2.pt] (1.1924971494140544,2.065687913232858) -- (1.1999971314858409,2.059128535730395);
\draw[line width=2.pt] (1.1999971314858409,2.059128535730395) -- (1.2074971135576273,2.0525756010203953);
\draw[line width=2.pt] (1.2074971135576273,2.0525756010203953) -- (1.2149970956294138,2.04602917100696);
\draw[line width=2.pt] (1.2149970956294138,2.04602917100696) -- (1.2224970777012003,2.0394893083271537);
\draw[line width=2.pt] (1.2224970777012003,2.0394893083271537) -- (1.2299970597729868,2.0329560763607932);
\draw[line width=2.pt] (1.2299970597729868,2.0329560763607932) -- (1.2374970418447733,2.0264295392403664);
\draw[line width=2.pt] (1.2374970418447733,2.0264295392403664) -- (1.2449970239165598,2.0199097618610917);
\draw[line width=2.pt] (1.2449970239165598,2.0199097618610917) -- (1.2524970059883462,2.013396809891109);
\draw[line width=2.pt] (1.2524970059883462,2.013396809891109) -- (1.2599969880601327,2.006890749781814);
\draw[line width=2.pt] (1.2599969880601327,2.006890749781814) -- (1.2674969701319192,2.000391648778329);
\draw[line width=2.pt] (1.2674969701319192,2.000391648778329) -- (1.2749969522037057,1.9938995749301178);
\draw[line width=2.pt] (1.2749969522037057,1.9938995749301178) -- (1.2824969342754922,1.987414597101743);
\draw[line width=2.pt] (1.2824969342754922,1.987414597101743) -- (1.2899969163472786,1.9809367849837651);
\draw[line width=2.pt] (1.2899969163472786,1.9809367849837651) -- (1.2974968984190651,1.974466209103793);
\draw[line width=2.pt] (1.2974968984190651,1.974466209103793) -- (1.3049968804908516,1.968002940837677);
\draw[line width=2.pt] (1.3049968804908516,1.968002940837677) -- (1.312496862562638,1.9615470524208538);
\draw[line width=2.pt] (1.312496862562638,1.9615470524208538) -- (1.3199968446344246,1.9550986169598428);
\draw[line width=2.pt] (1.3199968446344246,1.9550986169598428) -- (1.327496826706211,1.9486577084438903);
\draw[line width=2.pt] (1.327496826706211,1.9486577084438903) -- (1.3349968087779975,1.9422244017567722);
\draw[line width=2.pt] (1.3349968087779975,1.9422244017567722) -- (1.342496790849784,1.9357987726887484);
\draw[line width=2.pt] (1.342496790849784,1.9357987726887484) -- (1.3499967729215705,1.929380897948674);
\draw[line width=2.pt] (1.3499967729215705,1.929380897948674) -- (1.357496754993357,1.9229708551762694);
\draw[line width=2.pt] (1.357496754993357,1.9229708551762694) -- (1.3649967370651435,1.9165687229545483);
\draw[line width=2.pt] (1.3649967370651435,1.9165687229545483) -- (1.37249671913693,1.9101745808224067);
\draw[line width=2.pt] (1.37249671913693,1.9101745808224067) -- (1.3799967012087164,1.9037885092873739);
\draw[line width=2.pt] (1.3799967012087164,1.9037885092873739) -- (1.387496683280503,1.897410589838525);
\draw[line width=2.pt] (1.387496683280503,1.897410589838525) -- (1.3949966653522894,1.8910409049595598);
\draw[line width=2.pt] (1.3949966653522894,1.8910409049595598) -- (1.4024966474240759,1.8846795381420465);
\draw[line width=2.pt] (1.4024966474240759,1.8846795381420465) -- (1.4099966294958624,1.8783265738988302);
\draw[line width=2.pt] (1.4099966294958624,1.8783265738988302) -- (1.4174966115676488,1.8719820977776132);
\draw[line width=2.pt] (1.4174966115676488,1.8719820977776132) -- (1.4249965936394353,1.865646196374699);
\draw[line width=2.pt] (1.4249965936394353,1.865646196374699) -- (1.4324965757112218,1.8593189573489122);
\draw[line width=2.pt] (1.4324965757112218,1.8593189573489122) -- (1.4399965577830083,1.8530004694356834);
\draw[line width=2.pt] (1.4399965577830083,1.8530004694356834) -- (1.4474965398547948,1.8466908224613114);
\draw[line width=2.pt] (1.4474965398547948,1.8466908224613114) -- (1.4549965219265812,1.8403901073573943);
\draw[line width=2.pt] (1.4549965219265812,1.8403901073573943) -- (1.4624965039983677,1.8340984161754355);
\draw[line width=2.pt] (1.4624965039983677,1.8340984161754355) -- (1.4699964860701542,1.827815842101626);
\draw[line width=2.pt] (1.4699964860701542,1.827815842101626) -- (1.4774964681419407,1.8215424794717978);
\draw[line width=2.pt] (1.4774964681419407,1.8215424794717978) -- (1.4849964502137272,1.8152784237865573);
\draw[line width=2.pt] (1.4849964502137272,1.8152784237865573) -- (1.4924964322855137,1.809023771726592);
\draw[line width=2.pt] (1.4924964322855137,1.809023771726592) -- (1.4999964143573001,1.8027786211681558);
\draw[line width=2.pt] (1.4999964143573001,1.8027786211681558) -- (1.5074963964290866,1.7965430711987291);
\draw[line width=2.pt] (1.5074963964290866,1.7965430711987291) -- (1.514996378500873,1.7903172221328605);
\draw[line width=2.pt] (1.514996378500873,1.7903172221328605) -- (1.5224963605726596,1.7841011755281808);
\draw[line width=2.pt] (1.5224963605726596,1.7841011755281808) -- (1.529996342644446,1.7778950342015991);
\draw[line width=2.pt] (1.529996342644446,1.7778950342015991) -- (1.5374963247162325,1.7716989022456744);
\draw[line width=2.pt] (1.5374963247162325,1.7716989022456744) -- (1.544996306788019,1.7655128850451658);
\draw[line width=2.pt] (1.544996306788019,1.7655128850451658) -- (1.5524962888598055,1.7593370892937588);
\draw[line width=2.pt] (1.5524962888598055,1.7593370892937588) -- (1.559996270931592,1.7531716230109706);
\draw[line width=2.pt] (1.559996270931592,1.7531716230109706) -- (1.5674962530033785,1.7470165955592296);
\draw[line width=2.pt] (1.5674962530033785,1.7470165955592296) -- (1.574996235075165,1.7408721176611321);
\draw[line width=2.pt] (1.574996235075165,1.7408721176611321) -- (1.5824962171469514,1.7347383014168745);
\draw[line width=2.pt] (1.5824962171469514,1.7347383014168745) -- (1.589996199218738,1.728615260321858);
\draw[line width=2.pt] (1.589996199218738,1.728615260321858) -- (1.5974961812905244,1.7225031092844685);
\draw[line width=2.pt] (1.5974961812905244,1.7225031092844685) -- (1.6049961633623109,1.716401964644026);
\draw[line width=2.pt] (1.6049961633623109,1.716401964644026) -- (1.6124961454340974,1.7103119441889065);
\draw[line width=2.pt] (1.6124961454340974,1.7103119441889065) -- (1.6199961275058838,1.7042331671748314);
\draw[line width=2.pt] (1.6199961275058838,1.7042331671748314) -- (1.6274961095776703,1.698165754343324);
\draw[line width=2.pt] (1.6274961095776703,1.698165754343324) -- (1.6349960916494568,1.6921098279403315);
\draw[line width=2.pt] (1.6349960916494568,1.6921098279403315) -- (1.6424960737212433,1.6860655117350096);
\draw[line width=2.pt] (1.6424960737212433,1.6860655117350096) -- (1.6499960557930298,1.6800329310386675);
\draw[line width=2.pt] (1.6499960557930298,1.6800329310386675) -- (1.6574960378648163,1.67401221272387);
\draw[line width=2.pt] (1.6574960378648163,1.67401221272387) -- (1.6649960199366027,1.6680034852436945);
\draw[line width=2.pt] (1.6649960199366027,1.6680034852436945) -- (1.6724960020083892,1.6620068786511417);
\draw[line width=2.pt] (1.6724960020083892,1.6620068786511417) -- (1.6799959840801757,1.6560225246186913);
\draw[line width=2.pt] (1.6799959840801757,1.6560225246186913) -- (1.6874959661519622,1.6500505564580048);
\draw[line width=2.pt] (1.6874959661519622,1.6500505564580048) -- (1.6949959482237487,1.644091109139768);
\draw[line width=2.pt] (1.6949959482237487,1.644091109139768) -- (1.7024959302955351,1.6381443193136704);
\draw[line width=2.pt] (1.7024959302955351,1.6381443193136704) -- (1.7099959123673216,1.6322103253285156);
\draw[line width=2.pt] (1.7099959123673216,1.6322103253285156) -- (1.717495894439108,1.62628926725246);
\draw[line width=2.pt] (1.717495894439108,1.62628926725246) -- (1.7249958765108946,1.6203812868933725);
\draw[line width=2.pt] (1.7249958765108946,1.6203812868933725) -- (1.732495858582681,1.614486527819311);
\draw[line width=2.pt] (1.732495858582681,1.614486527819311) -- (1.7399958406544676,1.608605135379109);
\draw[line width=2.pt] (1.7399958406544676,1.608605135379109) -- (1.747495822726254,1.6027372567230673);
\draw[line width=2.pt] (1.747495822726254,1.6027372567230673) -- (1.7549958047980405,1.5968830408237413);
\draw[line width=2.pt] (1.7549958047980405,1.5968830408237413) -- (1.762495786869827,1.591042638496822);
\draw[line width=2.pt] (1.762495786869827,1.591042638496822) -- (1.7699957689416135,1.5852162024220964);
\draw[line width=2.pt] (1.7699957689416135,1.5852162024220964) -- (1.7774957510134,1.5794038871644869);
\draw[line width=2.pt] (1.7774957510134,1.5794038871644869) -- (1.7849957330851864,1.5736058491951546);
\draw[line width=2.pt] (1.7849957330851864,1.5736058491951546) -- (1.792495715156973,1.5678222469126626);
\draw[line width=2.pt] (1.792495715156973,1.5678222469126626) -- (1.7999956972287594,1.562053240664188);
\draw[line width=2.pt] (1.7999956972287594,1.562053240664188) -- (1.8074956793005459,1.5562989927667712);
\draw[line width=2.pt] (1.8074956793005459,1.5562989927667712) -- (1.8149956613723324,1.550559667528598);
\draw[line width=2.pt] (1.8149956613723324,1.550559667528598) -- (1.8224956434441189,1.5448354312702954);
\draw[line width=2.pt] (1.8224956434441189,1.5448354312702954) -- (1.8299956255159053,1.5391264523462385);
\draw[line width=2.pt] (1.8299956255159053,1.5391264523462385) -- (1.8374956075876918,1.533432901165848);
\draw[line width=2.pt] (1.8374956075876918,1.533432901165848) -- (1.8449955896594783,1.5277549502148753);
\draw[line width=2.pt] (1.8449955896594783,1.5277549502148753) -- (1.8524955717312648,1.5220927740766517);
\draw[line width=2.pt] (1.8524955717312648,1.5220927740766517) -- (1.8599955538030513,1.5164465494532973);
\draw[line width=2.pt] (1.8599955538030513,1.5164465494532973) -- (1.8674955358748377,1.5108164551868706);
\draw[line width=2.pt] (1.8674955358748377,1.5108164551868706) -- (1.8749955179466242,1.5052026722804421);
\draw[line width=2.pt] (1.8749955179466242,1.5052026722804421) -- (1.8824955000184107,1.4996053839190835);
\draw[line width=2.pt] (1.8824955000184107,1.4996053839190835) -- (1.8899954820901972,1.4940247754907459);
\draw[line width=2.pt] (1.8899954820901972,1.4940247754907459) -- (1.8974954641619837,1.48846103460702);
\draw[line width=2.pt] (1.8974954641619837,1.48846103460702) -- (1.9049954462337702,1.4829143511237526);
\draw[line width=2.pt] (1.9049954462337702,1.4829143511237526) -- (1.9124954283055566,1.4773849171615077);
\draw[line width=2.pt] (1.9124954283055566,1.4773849171615077) -- (1.9199954103773431,1.4718729271258453);
\draw[line width=2.pt] (1.9199954103773431,1.4718729271258453) -- (1.9274953924491296,1.4663785777274048);
\draw[line width=2.pt] (1.9274953924491296,1.4663785777274048) -- (1.934995374520916,1.4609020680017684);
\draw[line width=2.pt] (1.934995374520916,1.4609020680017684) -- (1.9424953565927026,1.4554435993290826);
\draw[line width=2.pt] (1.9424953565927026,1.4554435993290826) -- (1.949995338664489,1.4500033754534163);
\draw[line width=2.pt] (1.949995338664489,1.4500033754534163) -- (1.9574953207362755,1.444581602501832);
\draw[line width=2.pt] (1.9574953207362755,1.444581602501832) -- (1.964995302808062,1.439178489003145);
\draw[line width=2.pt] (1.964995302808062,1.439178489003145) -- (1.9724952848798485,1.4337942459063449);
\draw[line width=2.pt] (1.9724952848798485,1.4337942459063449) -- (1.979995266951635,1.4284290865986544);
\draw[line width=2.pt] (1.979995266951635,1.4284290865986544) -- (1.9874952490234215,1.4230832269231983);
\draw[line width=2.pt] (1.9874952490234215,1.4230832269231983) -- (1.994995231095208,1.4177568851962505);
\draw[line width=2.pt] (1.994995231095208,1.4177568851962505) -- (2.0024952131669944,1.4124502822240363);
\draw[line width=2.pt] (2.0024952131669944,1.4124502822240363) -- (2.009995195238781,1.407163641319054);
\draw[line width=2.pt] (2.009995195238781,1.407163641319054) -- (2.017495177310568,1.4018971883158882);
\draw[line width=2.pt] (2.017495177310568,1.4018971883158882) -- (2.0249951593823545,1.39665115158648);
\draw[line width=2.pt] (2.0249951593823545,1.39665115158648) -- (2.0324951414541412,1.391425762054822);
\draw[line width=2.pt] (2.0324951414541412,1.391425762054822) -- (2.039995123525928,1.3862212532110445);
\draw[line width=2.pt] (2.039995123525928,1.3862212532110445) -- (2.0474951055977146,1.381037861124853);
\draw[line width=2.pt] (2.0474951055977146,1.381037861124853) -- (2.0549950876695013,1.3758758244582878);
\draw[line width=2.pt] (2.0549950876695013,1.3758758244582878) -- (2.062495069741288,1.3707353844777599);
\draw[line width=2.pt] (2.062495069741288,1.3707353844777599) -- (2.0699950518130747,1.3656167850653291);
\draw[line width=2.pt] (2.0699950518130747,1.3656167850653291) -- (2.0774950338848615,1.3605202727291839);
\draw[line width=2.pt] (2.0774950338848615,1.3605202727291839) -- (2.084995015956648,1.355446096613279);
\draw[line width=2.pt] (2.084995015956648,1.355446096613279) -- (2.092494998028435,1.3503945085060922);
\draw[line width=2.pt] (2.092494998028435,1.3503945085060922) -- (2.0999949801002216,1.3453657628484532);
\draw[line width=2.pt] (2.0999949801002216,1.3453657628484532) -- (2.1074949621720083,1.3403601167404022);
\draw[line width=2.pt] (2.1074949621720083,1.3403601167404022) -- (2.114994944243795,1.3353778299470318);
\draw[line width=2.pt] (2.114994944243795,1.3353778299470318) -- (2.1224949263155817,1.3304191649032633);
\draw[line width=2.pt] (2.1224949263155817,1.3304191649032633) -- (2.1299949083873684,1.3254843867175137);
\draw[line width=2.pt] (2.1299949083873684,1.3254843867175137) -- (2.137494890459155,1.320573763174199);
\draw[line width=2.pt] (2.137494890459155,1.320573763174199) -- (2.144994872530942,1.315687564735025);
\draw[line width=2.pt] (2.144994872530942,1.315687564735025) -- (2.1524948546027285,1.3108260645390182);
\draw[line width=2.pt] (2.1524948546027285,1.3108260645390182) -- (2.159994836674515,1.3059895384012363);
\draw[line width=2.pt] (2.159994836674515,1.3059895384012363) -- (2.167494818746302,1.3011782648101116);
\draw[line width=2.pt] (2.167494818746302,1.3011782648101116) -- (2.1749948008180886,1.296392524923368);
\draw[line width=2.pt] (2.1749948008180886,1.296392524923368) -- (2.1824947828898753,1.2916326025624594);
\draw[line width=2.pt] (2.1824947828898753,1.2916326025624594) -- (2.189994764961662,1.2868987842054687);
\draw[line width=2.pt] (2.189994764961662,1.2868987842054687) -- (2.1974947470334487,1.2821913589784126);
\draw[line width=2.pt] (2.1974947470334487,1.2821913589784126) -- (2.2049947291052354,1.2775106186448932);
\draw[line width=2.pt] (2.2049947291052354,1.2775106186448932) -- (2.212494711177022,1.272856857594035);
\draw[line width=2.pt] (2.212494711177022,1.272856857594035) -- (2.219994693248809,1.2682303728266484);
\draw[line width=2.pt] (2.219994693248809,1.2682303728266484) -- (2.2274946753205955,1.2636314639395585);
\draw[line width=2.pt] (2.2274946753205955,1.2636314639395585) -- (2.2349946573923822,1.2590604331080373);
\draw[line width=2.pt] (2.2349946573923822,1.2590604331080373) -- (2.242494639464169,1.2545175850662753);
\draw[line width=2.pt] (2.242494639464169,1.2545175850662753) -- (2.2499946215359556,1.2500032270858321);
\draw[line width=2.pt] (2.2499946215359556,1.2500032270858321) -- (2.2574946036077423,1.245517668952);
\draw[line width=2.pt] (2.2574946036077423,1.245517668952) -- (2.264994585679529,1.2410612229380173);
\draw[line width=2.pt] (2.264994585679529,1.2410612229380173) -- (2.2724945677513158,1.2366342037770688);
\draw[line width=2.pt] (2.2724945677513158,1.2366342037770688) -- (2.2799945498231025,1.232236928632005);
\draw[line width=2.pt] (2.2799945498231025,1.232236928632005) -- (2.287494531894889,1.2278697170627195);
\draw[line width=2.pt] (2.287494531894889,1.2278697170627195) -- (2.294994513966676,1.2235328909911183);
\draw[line width=2.pt] (2.294994513966676,1.2235328909911183) -- (2.3024944960384626,1.2192267746636136);
\draw[line width=2.pt] (2.3024944960384626,1.2192267746636136) -- (2.3099944781102493,1.214951694611085);
\draw[line width=2.pt] (2.3099944781102493,1.214951694611085) -- (2.317494460182036,1.2107079796062346);
\draw[line width=2.pt] (2.317494460182036,1.2107079796062346) -- (2.3249944422538227,1.2064959606182808);
\draw[line width=2.pt] (2.3249944422538227,1.2064959606182808) -- (2.3324944243256094,1.2023159707649231);
\draw[line width=2.pt] (2.3324944243256094,1.2023159707649231) -- (2.339994406397396,1.1981683452615186);
\draw[line width=2.pt] (2.339994406397396,1.1981683452615186) -- (2.347494388469183,1.194053421367405);
\draw[line width=2.pt] (2.347494388469183,1.194053421367405) -- (2.3549943705409695,1.1899715383293166);
\draw[line width=2.pt] (2.3549943705409695,1.1899715383293166) -- (2.362494352612756,1.1859230373218277);
\draw[line width=2.pt] (2.362494352612756,1.1859230373218277) -- (2.369994334684543,1.181908261384771);
\draw[line width=2.pt] (2.369994334684543,1.181908261384771) -- (2.3774943167563296,1.1779275553575732);
\draw[line width=2.pt] (2.3774943167563296,1.1779275553575732) -- (2.3849942988281163,1.1739812658104558);
\draw[line width=2.pt] (2.3849942988281163,1.1739812658104558) -- (2.392494280899903,1.1700697409724456);
\draw[line width=2.pt] (2.392494280899903,1.1700697409724456) -- (2.3999942629716897,1.1661933306561505);
\draw[line width=2.pt] (2.3999942629716897,1.1661933306561505) -- (2.4074942450434764,1.1623523861792515);
\draw[line width=2.pt] (2.4074942450434764,1.1623523861792515) -- (2.414994227115263,1.1585472602826647);
\draw[line width=2.pt] (2.414994227115263,1.1585472602826647) -- (2.42249420918705,1.1547783070453355);
\draw[line width=2.pt] (2.42249420918705,1.1547783070453355) -- (2.4299941912588365,1.151045881795625);
\draw[line width=2.pt] (2.4299941912588365,1.151045881795625) -- (2.4374941733306232,1.1473503410192543);
\draw[line width=2.pt] (2.4374941733306232,1.1473503410192543) -- (2.44499415540241,1.1436920422637749);
\draw[line width=2.pt] (2.44499415540241,1.1436920422637749) -- (2.4524941374741966,1.1400713440395402);
\draw[line width=2.pt] (2.4524941374741966,1.1400713440395402) -- (2.4599941195459833,1.1364886057171528);
\draw[line width=2.pt] (2.4599941195459833,1.1364886057171528) -- (2.46749410161777,1.1329441874213688);
\draw[line width=2.pt] (2.46749410161777,1.1329441874213688) -- (2.4749940836895568,1.1294384499214503);
\draw[line width=2.pt] (2.4749940836895568,1.1294384499214503) -- (2.4824940657613435,1.1259717545179473);
\draw[line width=2.pt] (2.4824940657613435,1.1259717545179473) -- (2.48999404783313,1.1225444629259171);
\draw[line width=2.pt] (2.48999404783313,1.1225444629259171) -- (2.497494029904917,1.119156937154571);
\draw[line width=2.pt] (2.497494029904917,1.119156937154571) -- (2.5049940119767036,1.1158095393833662);
\draw[line width=2.pt] (2.5049940119767036,1.1158095393833662) -- (2.5124939940484903,1.112502631834547);
\draw[line width=2.pt] (2.5124939940484903,1.112502631834547) -- (2.519993976120277,1.1092365766421612);
\draw[line width=2.pt] (2.519993976120277,1.1092365766421612) -- (2.5274939581920637,1.1060117357175752);
\draw[line width=2.pt] (2.5274939581920637,1.1060117357175752) -- (2.5349939402638504,1.102828470611518);
\draw[line width=2.pt] (2.5349939402638504,1.102828470611518) -- (2.542493922335637,1.0996871423726977);
\draw[line width=2.pt] (2.542493922335637,1.0996871423726977) -- (2.549993904407424,1.096588111403035);
\draw[line width=2.pt] (2.549993904407424,1.096588111403035) -- (2.5574938864792105,1.0935317373095643);
\draw[line width=2.pt] (2.5574938864792105,1.0935317373095643) -- (2.564993868550997,1.0905183787530712);
\draw[line width=2.pt] (2.564993868550997,1.0905183787530712) -- (2.572493850622784,1.0875483932935281);
\draw[line width=2.pt] (2.572493850622784,1.0875483932935281) -- (2.5799938326945706,1.0846221372324079);
\draw[line width=2.pt] (2.5799938326945706,1.0846221372324079) -- (2.5874938147663573,1.0817399654519622);
\draw[line width=2.pt] (2.5874938147663573,1.0817399654519622) -- (2.594993796838144,1.0789022312515544);
\draw[line width=2.pt] (2.594993796838144,1.0789022312515544) -- (2.6024937789099307,1.0761092861811512);
\draw[line width=2.pt] (2.6024937789099307,1.0761092861811512) -- (2.6099937609817174,1.0733614798720819);
\draw[line width=2.pt] (2.6099937609817174,1.0733614798720819) -- (2.617493743053504,1.0706591598651827);
\draw[line width=2.pt] (2.617493743053504,1.0706591598651827) -- (2.624993725125291,1.0680026714364557);
\draw[line width=2.pt] (2.624993725125291,1.0680026714364557) -- (2.6324937071970775,1.0653923574203765);
\draw[line width=2.pt] (2.6324937071970775,1.0653923574203765) -- (2.6399936892688642,1.0628285580309946);
\draw[line width=2.pt] (2.6399936892688642,1.0628285580309946) -- (2.647493671340651,1.0603116106809796);
\draw[line width=2.pt] (2.647493671340651,1.0603116106809796) -- (2.6549936534124376,1.0578418497987765);
\draw[line width=2.pt] (2.6549936534124376,1.0578418497987765) -- (2.6624936354842244,1.0554196066440378);
\draw[line width=2.pt] (2.6624936354842244,1.0554196066440378) -- (2.669993617556011,1.0530452091215117);
\draw[line width=2.pt] (2.669993617556011,1.0530452091215117) -- (2.6774935996277978,1.050718981593573);
\draw[line width=2.pt] (2.6774935996277978,1.050718981593573) -- (2.6849935816995845,1.0484412446915927);
\draw[line width=2.pt] (2.6849935816995845,1.0484412446915927) -- (2.692493563771371,1.0462123151263474);
\draw[line width=2.pt] (2.692493563771371,1.0462123151263474) -- (2.699993545843158,1.0440325054976791);
\draw[line width=2.pt] (2.699993545843158,1.0440325054976791) -- (2.7074935279149446,1.0419021241036248);
\draw[line width=2.pt] (2.7074935279149446,1.0419021241036248) -- (2.7149935099867313,1.0398214747492396);
\draw[line width=2.pt] (2.7149935099867313,1.0398214747492396) -- (2.722493492058518,1.0377908565553446);
\draw[line width=2.pt] (2.722493492058518,1.0377908565553446) -- (2.7299934741303047,1.0358105637674402);
\draw[line width=2.pt] (2.7299934741303047,1.0358105637674402) -- (2.7374934562020914,1.0338808855650263);
\draw[line width=2.pt] (2.7374934562020914,1.0338808855650263) -- (2.744993438273878,1.0320021058715814);
\draw[line width=2.pt] (2.744993438273878,1.0320021058715814) -- (2.752493420345665,1.030174503165453);
\draw[line width=2.pt] (2.752493420345665,1.030174503165453) -- (2.7599934024174515,1.0283983502919243);
\draw[line width=2.pt] (2.7599934024174515,1.0283983502919243) -- (2.767493384489238,1.0266739142767138);
\draw[line width=2.pt] (2.767493384489238,1.0266739142767138) -- (2.774993366561025,1.0250014561411809);
\draw[line width=2.pt] (2.774993366561025,1.0250014561411809) -- (2.7824933486328116,1.0233812307195045);
\draw[line width=2.pt] (2.7824933486328116,1.0233812307195045) -- (2.7899933307045983,1.0218134864781088);
\draw[line width=2.pt] (2.7899933307045983,1.0218134864781088) -- (2.797493312776385,1.0202984653376104);
\draw[line width=2.pt] (2.797493312776385,1.0202984653376104) -- (2.8049932948481717,1.018836402497561);
\draw[line width=2.pt] (2.8049932948481717,1.018836402497561) -- (2.8124932769199584,1.017427526264262);
\draw[line width=2.pt] (2.8124932769199584,1.017427526264262) -- (2.819993258991745,1.0160720578819264);
\draw[line width=2.pt] (2.819993258991745,1.0160720578819264) -- (2.827493241063532,1.0147702113674626);
\draw[line width=2.pt] (2.827493241063532,1.0147702113674626) -- (2.8349932231353185,1.0135221933491494);
\draw[line width=2.pt] (2.8349932231353185,1.0135221933491494) -- (2.8424932052071052,1.012328202909477);
\draw[line width=2.pt] (2.8424932052071052,1.012328202909477) -- (2.849993187278892,1.011188431432414);
\draw[line width=2.pt] (2.849993187278892,1.011188431432414) -- (2.8574931693506787,1.0101030624553686);
\draw[line width=2.pt] (2.8574931693506787,1.0101030624553686) -- (2.8649931514224654,1.0090722715260971);
\draw[line width=2.pt] (2.8649931514224654,1.0090722715260971) -- (2.872493133494252,1.008096226064811);
\draw[line width=2.pt] (2.872493133494252,1.008096226064811) -- (2.8799931155660388,1.0071750852317318);
\draw[line width=2.pt] (2.8799931155660388,1.0071750852317318) -- (2.8874930976378255,1.0063089998003256);
\draw[line width=2.pt] (2.8874930976378255,1.0063089998003256) -- (2.894993079709612,1.0054981120364532);
\draw[line width=2.pt] (2.894993079709612,1.0054981120364532) -- (2.902493061781399,1.004742555583651);
\draw[line width=2.pt] (2.902493061781399,1.004742555583651) -- (2.9099930438531856,1.0040424553547596);
\draw[line width=2.pt] (2.9099930438531856,1.0040424553547596) -- (2.9174930259249723,1.0033979274300986);
\draw[line width=2.pt] (2.9174930259249723,1.0033979274300986) -- (2.924993007996759,1.0028090789623787);
\draw[line width=2.pt] (2.924993007996759,1.0028090789623787) -- (2.9324929900685457,1.0022760080885331);
\draw[line width=2.pt] (2.9324929900685457,1.0022760080885331) -- (2.9399929721403324,1.0017988038486325);
\draw[line width=2.pt] (2.9399929721403324,1.0017988038486325) -- (2.947492954212119,1.00137754611204);
\draw[line width=2.pt] (2.947492954212119,1.00137754611204) -- (2.954992936283906,1.0010123055109486);
\draw[line width=2.pt] (2.954992936283906,1.0010123055109486) -- (2.9624929183556925,1.000703143381429);
\draw[line width=2.pt] (2.9624929183556925,1.000703143381429) -- (2.969992900427479,1.000450111712101);
\draw[line width=2.pt] (2.969992900427479,1.000450111712101) -- (2.977492882499266,1.0002532531005295);
\draw[line width=2.pt] (2.977492882499266,1.0002532531005295) -- (2.9849928645710526,1.0001126007174306);
\draw[line width=2.pt] (2.9849928645710526,1.0001126007174306) -- (2.9924928466428393,1.0000281782787563);
\draw[line width=2.pt] (2.9924928466428393,1.0000281782787563) -- (2.999992828714626,1.0000000000257137);
\begin{scriptsize}
\draw [color=uuuuuu] (0.,0.)-- ++(-2.0pt,0 pt) -- ++(4.0pt,0 pt) ++(-2.0pt,-2.0pt) -- ++(0 pt,4.0pt);
\draw[color=uuuuuu] (-0.1926045016077218,0.17760252365932675) node {$O$};
\end{scriptsize}
\end{axis}
\end{tikzpicture}

\end{minipage}
\begin{minipage}{0.33\linewidth}
\begin{center}
\textbf{ Courbe 2}
 \end{center} 
 
\definecolor{qqwuqq}{rgb}{0.,0.39215686274509803,0.}
\begin{tikzpicture}[line cap=round,line join=round,>=triangle 45,x=0.7cm,y=0.7cm]
\begin{axis}[
x=0.7cm,y=0.7cm,
axis lines=middle,
ymajorgrids=true,
xmajorgrids=true,
xmin=-0.7749196141479164,
xmax=5.726688102893908,
ymin=-0.7728706624605428,
ymax=3.637223974763414,
xtick={-0.0,1.0,...,5.0},
ytick={-0.0,1.0,...,3.0},]
\clip(-0.7749196141479164,-0.7728706624605428) rectangle (5.726688102893908,3.637223974763414);
\draw[line width=2.pt,color=qqwuqq] (5.080385816016104E-7,0.8377228217993338) -- (0.0,0.8377228217993338);
\draw[line width=2.pt,color=qqwuqq] (0.0,0.8377228217993338) -- (0.007499985042040242,0.8448365589840847);
\draw[line width=2.pt,color=qqwuqq] (0.007499985042040242,0.8448365589840847) -- (0.014999970084080484,0.8519489872941959);
\draw[line width=2.pt,color=qqwuqq] (0.014999970084080484,0.8519489872941959) -- (0.022499955126120727,0.8590596125962602);
\draw[line width=2.pt,color=qqwuqq] (0.022499955126120727,0.8590596125962602) -- (0.029999940168160967,0.8661684226172701);
\draw[line width=2.pt,color=qqwuqq] (0.029999940168160967,0.8661684226172701) -- (0.03749992521020121,0.8732754049757503);
\draw[line width=2.pt,color=qqwuqq] (0.03749992521020121,0.8732754049757503) -- (0.04499991025224145,0.8803805471805921);
\draw[line width=2.pt,color=qqwuqq] (0.04499991025224145,0.8803805471805921) -- (0.05249989529428169,0.887483836629885);
\draw[line width=2.pt,color=qqwuqq] (0.05249989529428169,0.887483836629885) -- (0.05999988033632193,0.8945852606097136);
\draw[line width=2.pt,color=qqwuqq] (0.05999988033632193,0.8945852606097136) -- (0.06749986537836217,0.9016848062929546);
\draw[line width=2.pt,color=qqwuqq] (0.06749986537836217,0.9016848062929546) -- (0.07499985042040241,0.9087824607380561);
\draw[line width=2.pt,color=qqwuqq] (0.07499985042040241,0.9087824607380561) -- (0.08249983546244266,0.9158782108877936);
\draw[line width=2.pt,color=qqwuqq] (0.08249983546244266,0.9158782108877936) -- (0.08999982050448291,0.9229720435680235);
\draw[line width=2.pt,color=qqwuqq] (0.08999982050448291,0.9229720435680235) -- (0.09749980554652316,0.9300639454864097);
\draw[line width=2.pt,color=qqwuqq] (0.09749980554652316,0.9300639454864097) -- (0.1049997905885634,0.9371539032311365);
\draw[line width=2.pt,color=qqwuqq] (0.1049997905885634,0.9371539032311365) -- (0.11249977563060365,0.9442419032696137);
\draw[line width=2.pt,color=qqwuqq] (0.11249977563060365,0.9442419032696137) -- (0.1199997606726439,0.9513279319471524);
\draw[line width=2.pt,color=qqwuqq] (0.1199997606726439,0.9513279319471524) -- (0.12749974571468414,0.9584119754856335);
\draw[line width=2.pt,color=qqwuqq] (0.12749974571468414,0.9584119754856335) -- (0.1349997307567244,0.9654940199821578);
\draw[line width=2.pt,color=qqwuqq] (0.1349997307567244,0.9654940199821578) -- (0.14249971579876464,0.9725740514076744);
\draw[line width=2.pt,color=qqwuqq] (0.14249971579876464,0.9725740514076744) -- (0.14999970084080488,0.9796520556055963);
\draw[line width=2.pt,color=qqwuqq] (0.14999970084080488,0.9796520556055963) -- (0.15749968588284513,0.9867280182903961);
\draw[line width=2.pt,color=qqwuqq] (0.15749968588284513,0.9867280182903961) -- (0.16499967092488538,0.9938019250461876);
\draw[line width=2.pt,color=qqwuqq] (0.16499967092488538,0.9938019250461876) -- (0.17249965596692562,1.0008737613252823);
\draw[line width=2.pt,color=qqwuqq] (0.17249965596692562,1.0008737613252823) -- (0.17999964100896587,1.0079435124467384);
\draw[line width=2.pt,color=qqwuqq] (0.17999964100896587,1.0079435124467384) -- (0.18749962605100612,1.015011163594874);
\draw[line width=2.pt,color=qqwuqq] (0.18749962605100612,1.015011163594874) -- (0.19499961109304637,1.0220766998177808);
\draw[line width=2.pt,color=qqwuqq] (0.19499961109304637,1.0220766998177808) -- (0.2024995961350866,1.0291401060258067);
\draw[line width=2.pt,color=qqwuqq] (0.2024995961350866,1.0291401060258067) -- (0.20999958117712686,1.0362013669900225);
\draw[line width=2.pt,color=qqwuqq] (0.20999958117712686,1.0362013669900225) -- (0.2174995662191671,1.0432604673406685);
\draw[line width=2.pt,color=qqwuqq] (0.2174995662191671,1.0432604673406685) -- (0.22499955126120735,1.050317391565577);
\draw[line width=2.pt,color=qqwuqq] (0.22499955126120735,1.050317391565577) -- (0.2324995363032476,1.0573721240085856);
\draw[line width=2.pt,color=qqwuqq] (0.2324995363032476,1.0573721240085856) -- (0.23999952134528785,1.0644246488679192);
\draw[line width=2.pt,color=qqwuqq] (0.23999952134528785,1.0644246488679192) -- (0.2474995063873281,1.0714749501945517);
\draw[line width=2.pt,color=qqwuqq] (0.2474995063873281,1.0714749501945517) -- (0.25499949142936834,1.0785230118905567);
\draw[line width=2.pt,color=qqwuqq] (0.25499949142936834,1.0785230118905567) -- (0.2624994764714086,1.085568817707422);
\draw[line width=2.pt,color=qqwuqq] (0.2624994764714086,1.085568817707422) -- (0.26999946151344884,1.0926123512443588);
\draw[line width=2.pt,color=qqwuqq] (0.26999946151344884,1.0926123512443588) -- (0.2774994465554891,1.0996535959465694);
\draw[line width=2.pt,color=qqwuqq] (0.2774994465554891,1.0996535959465694) -- (0.28499943159752933,1.1066925351035124);
\draw[line width=2.pt,color=qqwuqq] (0.28499943159752933,1.1066925351035124) -- (0.2924994166395696,1.1137291518471324);
\draw[line width=2.pt,color=qqwuqq] (0.2924994166395696,1.1137291518471324) -- (0.2999994016816098,1.1207634291500703);
\draw[line width=2.pt,color=qqwuqq] (0.2999994016816098,1.1207634291500703) -- (0.30749938672365007,1.1277953498238462);
\draw[line width=2.pt,color=qqwuqq] (0.30749938672365007,1.1277953498238462) -- (0.3149993717656903,1.1348248965170322);
\draw[line width=2.pt,color=qqwuqq] (0.3149993717656903,1.1348248965170322) -- (0.32249935680773056,1.1418520517133803);
\draw[line width=2.pt,color=qqwuqq] (0.32249935680773056,1.1418520517133803) -- (0.3299993418497708,1.1488767977299439);
\draw[line width=2.pt,color=qqwuqq] (0.3299993418497708,1.1488767977299439) -- (0.33749932689181106,1.1558991167151689);
\draw[line width=2.pt,color=qqwuqq] (0.33749932689181106,1.1558991167151689) -- (0.3449993119338513,1.1629189906469497);
\draw[line width=2.pt,color=qqwuqq] (0.3449993119338513,1.1629189906469497) -- (0.35249929697589155,1.1699364013306788);
\draw[line width=2.pt,color=qqwuqq] (0.35249929697589155,1.1699364013306788) -- (0.3599992820179318,1.1769513303972534);
\draw[line width=2.pt,color=qqwuqq] (0.3599992820179318,1.1769513303972534) -- (0.36749926705997205,1.183963759301066);
\draw[line width=2.pt,color=qqwuqq] (0.36749926705997205,1.183963759301066) -- (0.3749992521020123,1.1909736693179624);
\draw[line width=2.pt,color=qqwuqq] (0.3749992521020123,1.1909736693179624) -- (0.38249923714405254,1.1979810415431755);
\draw[line width=2.pt,color=qqwuqq] (0.38249923714405254,1.1979810415431755) -- (0.3899992221860928,1.2049858568892358);
\draw[line width=2.pt,color=qqwuqq] (0.3899992221860928,1.2049858568892358) -- (0.39749920722813303,1.211988096083843);
\draw[line width=2.pt,color=qqwuqq] (0.39749920722813303,1.211988096083843) -- (0.4049991922701733,1.2189877396677211);
\draw[line width=2.pt,color=qqwuqq] (0.4049991922701733,1.2189877396677211) -- (0.4124991773122135,1.2259847679924372);
\draw[line width=2.pt,color=qqwuqq] (0.4124991773122135,1.2259847679924372) -- (0.4199991623542538,1.2329791612181973);
\draw[line width=2.pt,color=qqwuqq] (0.4199991623542538,1.2329791612181973) -- (0.427499147396294,1.2399708993116043);
\draw[line width=2.pt,color=qqwuqq] (0.427499147396294,1.2399708993116043) -- (0.43499913243833427,1.2469599620433964);
\draw[line width=2.pt,color=qqwuqq] (0.43499913243833427,1.2469599620433964) -- (0.4424991174803745,1.2539463289861459);
\draw[line width=2.pt,color=qqwuqq] (0.4424991174803745,1.2539463289861459) -- (0.44999910252241476,1.2609299795119346);
\draw[line width=2.pt,color=qqwuqq] (0.44999910252241476,1.2609299795119346) -- (0.457499087564455,1.2679108927899918);
\draw[line width=2.pt,color=qqwuqq] (0.457499087564455,1.2679108927899918) -- (0.46499907260649526,1.2748890477843053);
\draw[line width=2.pt,color=qqwuqq] (0.46499907260649526,1.2748890477843053) -- (0.4724990576485355,1.2818644232511982);
\draw[line width=2.pt,color=qqwuqq] (0.4724990576485355,1.2818644232511982) -- (0.47999904269057575,1.2888369977368725);
\draw[line width=2.pt,color=qqwuqq] (0.47999904269057575,1.2888369977368725) -- (0.487499027732616,1.295806749574922);
\draw[line width=2.pt,color=qqwuqq] (0.487499027732616,1.295806749574922) -- (0.49499901277465624,1.302773656883808);
\draw[line width=2.pt,color=qqwuqq] (0.49499901277465624,1.302773656883808) -- (0.5024989978166965,1.3097376975643051);
\draw[line width=2.pt,color=qqwuqq] (0.5024989978166965,1.3097376975643051) -- (0.5099989828587367,1.3166988492969098);
\draw[line width=2.pt,color=qqwuqq] (0.5099989828587367,1.3166988492969098) -- (0.5174989679007769,1.3236570895392141);
\draw[line width=2.pt,color=qqwuqq] (0.5174989679007769,1.3236570895392141) -- (0.5249989529428171,1.3306123955232447);
\draw[line width=2.pt,color=qqwuqq] (0.5249989529428171,1.3306123955232447) -- (0.5324989379848573,1.337564744252763);
\draw[line width=2.pt,color=qqwuqq] (0.5324989379848573,1.337564744252763) -- (0.5399989230268974,1.3445141125005344);
\draw[line width=2.pt,color=qqwuqq] (0.5399989230268974,1.3445141125005344) -- (0.5474989080689376,1.3514604768055563);
\draw[line width=2.pt,color=qqwuqq] (0.5474989080689376,1.3514604768055563) -- (0.5549988931109778,1.358403813470245);
\draw[line width=2.pt,color=qqwuqq] (0.5549988931109778,1.358403813470245) -- (0.562498878153018,1.3653440985575944);
\draw[line width=2.pt,color=qqwuqq] (0.562498878153018,1.3653440985575944) -- (0.5699988631950582,1.372281307888283);
\draw[line width=2.pt,color=qqwuqq] (0.5699988631950582,1.372281307888283) -- (0.5774988482370984,1.3792154170377557);
\draw[line width=2.pt,color=qqwuqq] (0.5774988482370984,1.3792154170377557) -- (0.5849988332791386,1.3861464013332494);
\draw[line width=2.pt,color=qqwuqq] (0.5849988332791386,1.3861464013332494) -- (0.5924988183211788,1.3930742358507935);
\draw[line width=2.pt,color=qqwuqq] (0.5924988183211788,1.3930742358507935) -- (0.599998803363219,1.3999988954121614);
\draw[line width=2.pt,color=qqwuqq] (0.599998803363219,1.3999988954121614) -- (0.6074987884052592,1.406920354581775);
\draw[line width=2.pt,color=qqwuqq] (0.6074987884052592,1.406920354581775) -- (0.6149987734472994,1.4138385876635837);
\draw[line width=2.pt,color=qqwuqq] (0.6149987734472994,1.4138385876635837) -- (0.6224987584893396,1.4207535686978776);
\draw[line width=2.pt,color=qqwuqq] (0.6224987584893396,1.4207535686978776) -- (0.6299987435313797,1.4276652714580784);
\draw[line width=2.pt,color=qqwuqq] (0.6299987435313797,1.4276652714580784) -- (0.6374987285734199,1.434573669447472);
\draw[line width=2.pt,color=qqwuqq] (0.6374987285734199,1.434573669447472) -- (0.6449987136154601,1.4414787358959007);
\draw[line width=2.pt,color=qqwuqq] (0.6449987136154601,1.4414787358959007) -- (0.6524986986575003,1.4483804437564114);
\draw[line width=2.pt,color=qqwuqq] (0.6524986986575003,1.4483804437564114) -- (0.6599986836995405,1.4552787657018538);
\draw[line width=2.pt,color=qqwuqq] (0.6599986836995405,1.4552787657018538) -- (0.6674986687415807,1.4621736741214342);
\draw[line width=2.pt,color=qqwuqq] (0.6674986687415807,1.4621736741214342) -- (0.6749986537836209,1.469065141117225);
\draw[line width=2.pt,color=qqwuqq] (0.6749986537836209,1.469065141117225) -- (0.6824986388256611,1.4759531385006115);
\draw[line width=2.pt,color=qqwuqq] (0.6824986388256611,1.4759531385006115) -- (0.6899986238677013,1.4828376377887116);
\draw[line width=2.pt,color=qqwuqq] (0.6899986238677013,1.4828376377887116) -- (0.6974986089097415,1.4897186102007254);
\draw[line width=2.pt,color=qqwuqq] (0.6974986089097415,1.4897186102007254) -- (0.7049985939517817,1.496596026654248);
\draw[line width=2.pt,color=qqwuqq] (0.7049985939517817,1.496596026654248) -- (0.7124985789938219,1.5034698577615204);
\draw[line width=2.pt,color=qqwuqq] (0.7124985789938219,1.5034698577615204) -- (0.719998564035862,1.5103400738256383);
\draw[line width=2.pt,color=qqwuqq] (0.719998564035862,1.5103400738256383) -- (0.7274985490779022,1.5172066448366994);
\draw[line width=2.pt,color=qqwuqq] (0.7274985490779022,1.5172066448366994) -- (0.7349985341199424,1.5240695404679059);
\draw[line width=2.pt,color=qqwuqq] (0.7349985341199424,1.5240695404679059) -- (0.7424985191619826,1.5309287300716008);
\draw[line width=2.pt,color=qqwuqq] (0.7424985191619826,1.5309287300716008) -- (0.7499985042040228,1.53778418267526);
\draw[line width=2.pt,color=qqwuqq] (0.7499985042040228,1.53778418267526) -- (0.757498489246063,1.544635866977428);
\draw[line width=2.pt,color=qqwuqq] (0.757498489246063,1.544635866977428) -- (0.7649984742881032,1.5514837513435804);
\draw[line width=2.pt,color=qqwuqq] (0.7649984742881032,1.5514837513435804) -- (0.7724984593301434,1.558327803801955);
\draw[line width=2.pt,color=qqwuqq] (0.7724984593301434,1.558327803801955) -- (0.7799984443721836,1.5651679920393011);
\draw[line width=2.pt,color=qqwuqq] (0.7799984443721836,1.5651679920393011) -- (0.7874984294142238,1.572004283396586);
\draw[line width=2.pt,color=qqwuqq] (0.7874984294142238,1.572004283396586) -- (0.794998414456264,1.5788366448646247);
\draw[line width=2.pt,color=qqwuqq] (0.794998414456264,1.5788366448646247) -- (0.8024983994983041,1.5856650430796648);
\draw[line width=2.pt,color=qqwuqq] (0.8024983994983041,1.5856650430796648) -- (0.8099983845403443,1.592489444318904);
\draw[line width=2.pt,color=qqwuqq] (0.8099983845403443,1.592489444318904) -- (0.8174983695823845,1.5993098144959337);
\draw[line width=2.pt,color=qqwuqq] (0.8174983695823845,1.5993098144959337) -- (0.8249983546244247,1.6061261191561371);
\draw[line width=2.pt,color=qqwuqq] (0.8249983546244247,1.6061261191561371) -- (0.8324983396664649,1.6129383234720072);
\draw[line width=2.pt,color=qqwuqq] (0.8324983396664649,1.6129383234720072) -- (0.8399983247085051,1.6197463922384099);
\draw[line width=2.pt,color=qqwuqq] (0.8399983247085051,1.6197463922384099) -- (0.8474983097505453,1.6265502898677715);
\draw[line width=2.pt,color=qqwuqq] (0.8474983097505453,1.6265502898677715) -- (0.8549982947925855,1.6333499803852032);
\draw[line width=2.pt,color=qqwuqq] (0.8549982947925855,1.6333499803852032) -- (0.8624982798346257,1.64014542742356);
\draw[line width=2.pt,color=qqwuqq] (0.8624982798346257,1.64014542742356) -- (0.8699982648766659,1.6469365942184186);
\draw[line width=2.pt,color=qqwuqq] (0.8699982648766659,1.6469365942184186) -- (0.8774982499187061,1.6537234436030026);
\draw[line width=2.pt,color=qqwuqq] (0.8774982499187061,1.6537234436030026) -- (0.8849982349607463,1.6605059380030145);
\draw[line width=2.pt,color=qqwuqq] (0.8849982349607463,1.6605059380030145) -- (0.8924982200027864,1.6672840394314132);
\draw[line width=2.pt,color=qqwuqq] (0.8924982200027864,1.6672840394314132) -- (0.8999982050448266,1.6740577094831117);
\draw[line width=2.pt,color=qqwuqq] (0.8999982050448266,1.6740577094831117) -- (0.9074981900868668,1.680826909329591);
\draw[line width=2.pt,color=qqwuqq] (0.9074981900868668,1.680826909329591) -- (0.914998175128907,1.6875915997134534);
\draw[line width=2.pt,color=qqwuqq] (0.914998175128907,1.6875915997134534) -- (0.9224981601709472,1.6943517409428868);
\draw[line width=2.pt,color=qqwuqq] (0.9224981601709472,1.6943517409428868) -- (0.9299981452129874,1.701107292886057);
\draw[line width=2.pt,color=qqwuqq] (0.9299981452129874,1.701107292886057) -- (0.9374981302550276,1.7078582149654213);
\draw[line width=2.pt,color=qqwuqq] (0.9374981302550276,1.7078582149654213) -- (0.9449981152970678,1.7146044661519522);
\draw[line width=2.pt,color=qqwuqq] (0.9449981152970678,1.7146044661519522) -- (0.952498100339108,1.7213460049592957);
\draw[line width=2.pt,color=qqwuqq] (0.952498100339108,1.7213460049592957) -- (0.9599980853811482,1.7280827894378321);
\draw[line width=2.pt,color=qqwuqq] (0.9599980853811482,1.7280827894378321) -- (0.9674980704231884,1.734814777168661);
\draw[line width=2.pt,color=qqwuqq] (0.9674980704231884,1.734814777168661) -- (0.9749980554652286,1.7415419252574988);
\draw[line width=2.pt,color=qqwuqq] (0.9749980554652286,1.7415419252574988) -- (0.9824980405072687,1.7482641903284897);
\draw[line width=2.pt,color=qqwuqq] (0.9824980405072687,1.7482641903284897) -- (0.9899980255493089,1.7549815285179324);
\draw[line width=2.pt,color=qqwuqq] (0.9899980255493089,1.7549815285179324) -- (0.9974980105913491,1.7616938954679133);
\draw[line width=2.pt,color=qqwuqq] (0.9974980105913491,1.7616938954679133) -- (1.0049979956333894,1.7684012463198506);
\draw[line width=2.pt,color=qqwuqq] (1.0049979956333894,1.7684012463198506) -- (1.0124979806754297,1.775103535707955);
\draw[line width=2.pt,color=qqwuqq] (1.0124979806754297,1.775103535707955) -- (1.01999796571747,1.7818007177525832);
\draw[line width=2.pt,color=qqwuqq] (1.01999796571747,1.7818007177525832) -- (1.0274979507595103,1.7884927460535138);
\draw[line width=2.pt,color=qqwuqq] (1.0274979507595103,1.7884927460535138) -- (1.0349979358015506,1.7951795736831162);
\draw[line width=2.pt,color=qqwuqq] (1.0349979358015506,1.7951795736831162) -- (1.042497920843591,1.8018611531794306);
\draw[line width=2.pt,color=qqwuqq] (1.042497920843591,1.8018611531794306) -- (1.0499979058856312,1.8085374365391447);
\draw[line width=2.pt,color=qqwuqq] (1.0499979058856312,1.8085374365391447) -- (1.0574978909276715,1.815208375210477);
\draw[line width=2.pt,color=qqwuqq] (1.0574978909276715,1.815208375210477) -- (1.0649978759697118,1.8218739200859542);
\draw[line width=2.pt,color=qqwuqq] (1.0649978759697118,1.8218739200859542) -- (1.0724978610117522,1.8285340214950931);
\draw[line width=2.pt,color=qqwuqq] (1.0724978610117522,1.8285340214950931) -- (1.0799978460537925,1.8351886291969741);
\draw[line width=2.pt,color=qqwuqq] (1.0799978460537925,1.8351886291969741) -- (1.0874978310958328,1.841837692372712);
\draw[line width=2.pt,color=qqwuqq] (1.0874978310958328,1.841837692372712) -- (1.094997816137873,1.8484811596178217);
\draw[line width=2.pt,color=qqwuqq] (1.094997816137873,1.8484811596178217) -- (1.1024978011799134,1.855118978934481);
\draw[line width=2.pt,color=qqwuqq] (1.1024978011799134,1.855118978934481) -- (1.1099977862219537,1.8617510977236735);
\draw[line width=2.pt,color=qqwuqq] (1.1099977862219537,1.8617510977236735) -- (1.117497771263994,1.8683774627772323);
\draw[line width=2.pt,color=qqwuqq] (1.117497771263994,1.8683774627772323) -- (1.1249977563060343,1.8749980202697678);
\draw[line width=2.pt,color=qqwuqq] (1.1249977563060343,1.8749980202697678) -- (1.1324977413480746,1.8816127157504834);
\draw[line width=2.pt,color=qqwuqq] (1.1324977413480746,1.8816127157504834) -- (1.1399977263901149,1.888221494134875);
\draw[line width=2.pt,color=qqwuqq] (1.1399977263901149,1.888221494134875) -- (1.1474977114321552,1.8948242996963174);
\draw[line width=2.pt,color=qqwuqq] (1.1474977114321552,1.8948242996963174) -- (1.1549976964741955,1.9014210760575327);
\draw[line width=2.pt,color=qqwuqq] (1.1549976964741955,1.9014210760575327) -- (1.1624976815162358,1.9080117661819394);
\draw[line width=2.pt,color=qqwuqq] (1.1624976815162358,1.9080117661819394) -- (1.169997666558276,1.9145963123648801);
\draw[line width=2.pt,color=qqwuqq] (1.169997666558276,1.9145963123648801) -- (1.1774976516003164,1.921174656224732);
\draw[line width=2.pt,color=qqwuqq] (1.1774976516003164,1.921174656224732) -- (1.1849976366423567,1.9277467386938847);
\draw[line width=2.pt,color=qqwuqq] (1.1849976366423567,1.9277467386938847) -- (1.192497621684397,1.9343125000096073);
\draw[line width=2.pt,color=qqwuqq] (1.192497621684397,1.9343125000096073) -- (1.1999976067264373,1.9408718797047726);
\draw[line width=2.pt,color=qqwuqq] (1.1999976067264373,1.9408718797047726) -- (1.2074975917684776,1.947424816598474);
\draw[line width=2.pt,color=qqwuqq] (1.2074975917684776,1.947424816598474) -- (1.214997576810518,1.9539712487864875);
\draw[line width=2.pt,color=qqwuqq] (1.214997576810518,1.9539712487864875) -- (1.2224975618525582,1.9605111136316289);
\draw[line width=2.pt,color=qqwuqq] (1.2224975618525582,1.9605111136316289) -- (1.2299975468945985,1.9670443477539559);
\draw[line width=2.pt,color=qqwuqq] (1.2299975468945985,1.9670443477539559) -- (1.2374975319366388,1.9735708870208564);
\draw[line width=2.pt,color=qqwuqq] (1.2374975319366388,1.9735708870208564) -- (1.244997516978679,1.9800906665369844);
\draw[line width=2.pt,color=qqwuqq] (1.244997516978679,1.9800906665369844) -- (1.2524975020207194,1.9866036206340727);
\draw[line width=2.pt,color=qqwuqq] (1.2524975020207194,1.9866036206340727) -- (1.2599974870627597,1.9931096828605925);
\draw[line width=2.pt,color=qqwuqq] (1.2599974870627597,1.9931096828605925) -- (1.2674974721048,1.99960878597129);
\draw[line width=2.pt,color=qqwuqq] (1.2674974721048,1.99960878597129) -- (1.2749974571468403,2.006100861916565);
\draw[line width=2.pt,color=qqwuqq] (1.2749974571468403,2.006100861916565) -- (1.2824974421888806,2.0125858418317186);
\draw[line width=2.pt,color=qqwuqq] (1.2824974421888806,2.0125858418317186) -- (1.289997427230921,2.0190636560260526);
\draw[line width=2.pt,color=qqwuqq] (1.289997427230921,2.0190636560260526) -- (1.2974974122729612,2.025534233971816);
\draw[line width=2.pt,color=qqwuqq] (1.2974974122729612,2.025534233971816) -- (1.3049973973150015,2.031997504293016);
\draw[line width=2.pt,color=qqwuqq] (1.3049973973150015,2.031997504293016) -- (1.3124973823570418,2.038453394754069);
\draw[line width=2.pt,color=qqwuqq] (1.3124973823570418,2.038453394754069) -- (1.3199973673990821,2.04490183224831);
\draw[line width=2.pt,color=qqwuqq] (1.3199973673990821,2.04490183224831) -- (1.3274973524411224,2.0513427427863427);
\draw[line width=2.pt,color=qqwuqq] (1.3274973524411224,2.0513427427863427) -- (1.3349973374831627,2.057776051484238);
\draw[line width=2.pt,color=qqwuqq] (1.3349973374831627,2.057776051484238) -- (1.342497322525203,2.0642016825515834);
\draw[line width=2.pt,color=qqwuqq] (1.342497322525203,2.0642016825515834) -- (1.3499973075672433,2.0706195592793666);
\draw[line width=2.pt,color=qqwuqq] (1.3499973075672433,2.0706195592793666) -- (1.3574972926092836,2.0770296040277083);
\draw[line width=2.pt,color=qqwuqq] (1.3574972926092836,2.0770296040277083) -- (1.364997277651324,2.0834317382134335);
\draw[line width=2.pt,color=qqwuqq] (1.364997277651324,2.0834317382134335) -- (1.3724972626933643,2.0898258822974825);
\draw[line width=2.pt,color=qqwuqq] (1.3724972626933643,2.0898258822974825) -- (1.3799972477354046,2.09621195577216);
\draw[line width=2.pt,color=qqwuqq] (1.3799972477354046,2.09621195577216) -- (1.3874972327774449,2.1025898771482225);
\draw[line width=2.pt,color=qqwuqq] (1.3874972327774449,2.1025898771482225) -- (1.3949972178194852,2.1089595639417986);
\draw[line width=2.pt,color=qqwuqq] (1.3949972178194852,2.1089595639417986) -- (1.4024972028615255,2.1153209326611466);
\draw[line width=2.pt,color=qqwuqq] (1.4024972028615255,2.1153209326611466) -- (1.4099971879035658,2.121673898793245);
\draw[line width=2.pt,color=qqwuqq] (1.4099971879035658,2.121673898793245) -- (1.417497172945606,2.128018376790213);
\draw[line width=2.pt,color=qqwuqq] (1.417497172945606,2.128018376790213) -- (1.4249971579876464,2.1343542800555646);
\draw[line width=2.pt,color=qqwuqq] (1.4249971579876464,2.1343542800555646) -- (1.4324971430296867,2.1406815209302916);
\draw[line width=2.pt,color=qqwuqq] (1.4324971430296867,2.1406815209302916) -- (1.439997128071727,2.147000010678775);
\draw[line width=2.pt,color=qqwuqq] (1.439997128071727,2.147000010678775) -- (1.4474971131137673,2.153309659474527);
\draw[line width=2.pt,color=qqwuqq] (1.4474971131137673,2.153309659474527) -- (1.4549970981558076,2.159610376385757);
\draw[line width=2.pt,color=qqwuqq] (1.4549970981558076,2.159610376385757) -- (1.4624970831978479,2.1659020693607647);
\draw[line width=2.pt,color=qqwuqq] (1.4624970831978479,2.1659020693607647) -- (1.4699970682398882,2.1721846452131612);
\draw[line width=2.pt,color=qqwuqq] (1.4699970682398882,2.1721846452131612) -- (1.4774970532819285,2.1784580096069126);
\draw[line width=2.pt,color=qqwuqq] (1.4774970532819285,2.1784580096069126) -- (1.4849970383239688,2.1847220670412075);
\draw[line width=2.pt,color=qqwuqq] (1.4849970383239688,2.1847220670412075) -- (1.492497023366009,2.190976720835151);
\draw[line width=2.pt,color=qqwuqq] (1.492497023366009,2.190976720835151) -- (1.4999970084080494,2.1972218731122783);
\draw[line width=2.pt,color=qqwuqq] (1.4999970084080494,2.1972218731122783) -- (1.5074969934500897,2.2034574247848946);
\draw[line width=2.pt,color=qqwuqq] (1.5074969934500897,2.2034574247848946) -- (1.51499697849213,2.209683275538235);
\draw[line width=2.pt,color=qqwuqq] (1.51499697849213,2.209683275538235) -- (1.5224969635341703,2.215899323814447);
\draw[line width=2.pt,color=qqwuqq] (1.5224969635341703,2.215899323814447) -- (1.5299969485762106,2.222105466796399);
\draw[line width=2.pt,color=qqwuqq] (1.5299969485762106,2.222105466796399) -- (1.537496933618251,2.228301600391303);
\draw[line width=2.pt,color=qqwuqq] (1.537496933618251,2.228301600391303) -- (1.5449969186602912,2.2344876192141707);
\draw[line width=2.pt,color=qqwuqq] (1.5449969186602912,2.2344876192141707) -- (1.5524969037023315,2.240663416571082);
\draw[line width=2.pt,color=qqwuqq] (1.5524969037023315,2.240663416571082) -- (1.5599968887443718,2.2468288844422832);
\draw[line width=2.pt,color=qqwuqq] (1.5599968887443718,2.2468288844422832) -- (1.5674968737864121,2.252983913465104);
\draw[line width=2.pt,color=qqwuqq] (1.5674968737864121,2.252983913465104) -- (1.5749968588284524,2.259128392916704);
\draw[line width=2.pt,color=qqwuqq] (1.5749968588284524,2.259128392916704) -- (1.5824968438704927,2.2652622106966387);
\draw[line width=2.pt,color=qqwuqq] (1.5824968438704927,2.2652622106966387) -- (1.589996828912533,2.271385253309254);
\draw[line width=2.pt,color=qqwuqq] (1.589996828912533,2.271385253309254) -- (1.5974968139545733,2.277497405845909);
\draw[line width=2.pt,color=qqwuqq] (1.5974968139545733,2.277497405845909) -- (1.6049967989966136,2.2835985519670245);
\draw[line width=2.pt,color=qqwuqq] (1.6049967989966136,2.2835985519670245) -- (1.612496784038654,2.2896885738839616);
\draw[line width=2.pt,color=qqwuqq] (1.612496784038654,2.2896885738839616) -- (1.6199967690806942,2.2957673523407305);
\draw[line width=2.pt,color=qqwuqq] (1.6199967690806942,2.2957673523407305) -- (1.6274967541227345,2.3018347665955385);
\draw[line width=2.pt,color=qqwuqq] (1.6274967541227345,2.3018347665955385) -- (1.6349967391647748,2.307890694402162);
\draw[line width=2.pt,color=qqwuqq] (1.6349967391647748,2.307890694402162) -- (1.6424967242068151,2.3139350119911666);
\draw[line width=2.pt,color=qqwuqq] (1.6424967242068151,2.3139350119911666) -- (1.6499967092488554,2.31996759405096);
\draw[line width=2.pt,color=qqwuqq] (1.6499967092488554,2.31996759405096) -- (1.6574966942908957,2.3259883137086907);
\draw[line width=2.pt,color=qqwuqq] (1.6574966942908957,2.3259883137086907) -- (1.664996679332936,2.331997042510989);
\draw[line width=2.pt,color=qqwuqq] (1.664996679332936,2.331997042510989) -- (1.6724966643749763,2.3379936504045586);
\draw[line width=2.pt,color=qqwuqq] (1.6724966643749763,2.3379936504045586) -- (1.6799966494170167,2.34397800571662);
\draw[line width=2.pt,color=qqwuqq] (1.6799966494170167,2.34397800571662) -- (1.687496634459057,2.3499499751352078);
\draw[line width=2.pt,color=qqwuqq] (1.687496634459057,2.3499499751352078) -- (1.6949966195010973,2.355909423689326);
\draw[line width=2.pt,color=qqwuqq] (1.6949966195010973,2.355909423689326) -- (1.7024966045431376,2.3618562147289737);
\draw[line width=2.pt,color=qqwuqq] (1.7024966045431376,2.3618562147289737) -- (1.7099965895851779,2.367790209905028);
\draw[line width=2.pt,color=qqwuqq] (1.7099965895851779,2.367790209905028) -- (1.7174965746272182,2.3737112691490116);
\draw[line width=2.pt,color=qqwuqq] (1.7174965746272182,2.3737112691490116) -- (1.7249965596692585,2.379619250652728);
\draw[line width=2.pt,color=qqwuqq] (1.7249965596692585,2.379619250652728) -- (1.7324965447112988,2.385514010847788);
\draw[line width=2.pt,color=qqwuqq] (1.7324965447112988,2.385514010847788) -- (1.739996529753339,2.3913954043850216);
\draw[line width=2.pt,color=qqwuqq] (1.739996529753339,2.3913954043850216) -- (1.7474965147953794,2.3972632841137878);
\draw[line width=2.pt,color=qqwuqq] (1.7474965147953794,2.3972632841137878) -- (1.7549964998374197,2.4031175010611845);
\draw[line width=2.pt,color=qqwuqq] (1.7549964998374197,2.4031175010611845) -- (1.76249648487946,2.408957904411171);
\draw[line width=2.pt,color=qqwuqq] (1.76249648487946,2.408957904411171) -- (1.7699964699215003,2.414784341483604);
\draw[line width=2.pt,color=qqwuqq] (1.7699964699215003,2.414784341483604) -- (1.7774964549635406,2.4205966577132014);
\draw[line width=2.pt,color=qqwuqq] (1.7774964549635406,2.4205966577132014) -- (1.7849964400055809,2.426394696628436);
\draw[line width=2.pt,color=qqwuqq] (1.7849964400055809,2.426394696628436) -- (1.7924964250476212,2.4321782998303743);
\draw[line width=2.pt,color=qqwuqq] (1.7924964250476212,2.4321782998303743) -- (1.7999964100896615,2.4379473069714646);
\draw[line width=2.pt,color=qqwuqq] (1.7999964100896615,2.4379473069714646) -- (1.8074963951317018,2.4437015557342843);
\draw[line width=2.pt,color=qqwuqq] (1.8074963951317018,2.4437015557342843) -- (1.814996380173742,2.449440881810263);
\draw[line width=2.pt,color=qqwuqq] (1.814996380173742,2.449440881810263) -- (1.8224963652157824,2.455165118878382);
\draw[line width=2.pt,color=qqwuqq] (1.8224963652157824,2.455165118878382) -- (1.8299963502578227,2.4608740985838695);
\draw[line width=2.pt,color=qqwuqq] (1.8299963502578227,2.4608740985838695) -- (1.837496335299863,2.466567650516904);
\draw[line width=2.pt,color=qqwuqq] (1.837496335299863,2.466567650516904) -- (1.8449963203419033,2.472245602191326);
\draw[line width=2.pt,color=qqwuqq] (1.8449963203419033,2.472245602191326) -- (1.8524963053839436,2.4779077790233934);
\draw[line width=2.pt,color=qqwuqq] (1.8524963053839436,2.4779077790233934) -- (1.859996290425984,2.4835540043105664);
\draw[line width=2.pt,color=qqwuqq] (1.859996290425984,2.4835540043105664) -- (1.8674962754680242,2.4891840992103647);
\draw[line width=2.pt,color=qqwuqq] (1.8674962754680242,2.4891840992103647) -- (1.8749962605100645,2.494797882719288);
\draw[line width=2.pt,color=qqwuqq] (1.8749962605100645,2.494797882719288) -- (1.8824962455521048,2.5003951716518307);
\draw[line width=2.pt,color=qqwuqq] (1.8824962455521048,2.5003951716518307) -- (1.8899962305941451,2.5059757806196026);
\draw[line width=2.pt,color=qqwuqq] (1.8899962305941451,2.5059757806196026) -- (1.8974962156361854,2.5115395220105663);
\draw[line width=2.pt,color=qqwuqq] (1.8974962156361854,2.5115395220105663) -- (1.9049962006782257,2.517086205968425);
\draw[line width=2.pt,color=qqwuqq] (1.9049962006782257,2.517086205968425) -- (1.912496185720266,2.5226156403721576);
\draw[line width=2.pt,color=qqwuqq] (1.912496185720266,2.5226156403721576) -- (1.9199961707623063,2.5281276308157414);
\draw[line width=2.pt,color=qqwuqq] (1.9199961707623063,2.5281276308157414) -- (1.9274961558043466,2.53362198058807);
\draw[line width=2.pt,color=qqwuqq] (1.9274961558043466,2.53362198058807) -- (1.934996140846387,2.539098490653087);
\draw[line width=2.pt,color=qqwuqq] (1.934996140846387,2.539098490653087) -- (1.9424961258884272,2.5445569596301665);
\draw[line width=2.pt,color=qqwuqq] (1.9424961258884272,2.5445569596301665) -- (1.9499961109304675,2.549997183774755);
\draw[line width=2.pt,color=qqwuqq] (1.9499961109304675,2.549997183774755) -- (1.9574960959725078,2.555418956959298);
\draw[line width=2.pt,color=qqwuqq] (1.9574960959725078,2.555418956959298) -- (1.9649960810145481,2.560822070654485);
\draw[line width=2.pt,color=qqwuqq] (1.9649960810145481,2.560822070654485) -- (1.9724960660565884,2.5662063139108238);
\draw[line width=2.pt,color=qqwuqq] (1.9724960660565884,2.5662063139108238) -- (1.9799960510986288,2.5715714733405837);
\draw[line width=2.pt,color=qqwuqq] (1.9799960510986288,2.5715714733405837) -- (1.987496036140669,2.576917333100126);
\draw[line width=2.pt,color=qqwuqq] (1.987496036140669,2.576917333100126) -- (1.9949960211827094,2.5822436748726583);
\draw[line width=2.pt,color=qqwuqq] (1.9949960211827094,2.5822436748726583) -- (2.0024960062247494,2.587550277851429);
\draw[line width=2.pt,color=qqwuqq] (2.0024960062247494,2.587550277851429) -- (2.0099959912667895,2.5928369187234104);
\draw[line width=2.pt,color=qqwuqq] (2.0099959912667895,2.5928369187234104) -- (2.0174959763088296,2.5981033716534805);
\draw[line width=2.pt,color=qqwuqq] (2.0174959763088296,2.5981033716534805) -- (2.0249959613508697,2.603349408269157);
\draw[line width=2.pt,color=qqwuqq] (2.0249959613508697,2.603349408269157) -- (2.0324959463929098,2.6085747976458986);
\draw[line width=2.pt,color=qqwuqq] (2.0324959463929098,2.6085747976458986) -- (2.03999593143495,2.613779306293024);
\draw[line width=2.pt,color=qqwuqq] (2.03999593143495,2.613779306293024) -- (2.04749591647699,2.6189626981402676);
\draw[line width=2.pt,color=qqwuqq] (2.04749591647699,2.6189626981402676) -- (2.05499590151903,2.624124734525026);
\draw[line width=2.pt,color=qqwuqq] (2.05499590151903,2.624124734525026) -- (2.06249588656107,2.6292651741803184);
\draw[line width=2.pt,color=qqwuqq] (2.06249588656107,2.6292651741803184) -- (2.06999587160311,2.63438377322351);
\draw[line width=2.pt,color=qqwuqq] (2.06999587160311,2.63438377322351) -- (2.0774958566451502,2.6394802851458325);
\draw[line width=2.pt,color=qqwuqq] (2.0774958566451502,2.6394802851458325) -- (2.0849958416871903,2.6445544608027465);
\draw[line width=2.pt,color=qqwuqq] (2.0849958416871903,2.6445544608027465) -- (2.0924958267292304,2.6496060484051838);
\draw[line width=2.pt,color=qqwuqq] (2.0924958267292304,2.6496060484051838) -- (2.0999958117712705,2.654634793511719);
\draw[line width=2.pt,color=qqwuqq] (2.0999958117712705,2.654634793511719) -- (2.1074957968133106,2.6596404390217128);
\draw[line width=2.pt,color=qqwuqq] (2.1074957968133106,2.6596404390217128) -- (2.1149957818553506,2.6646227251694663);
\draw[line width=2.pt,color=qqwuqq] (2.1149957818553506,2.6646227251694663) -- (2.1224957668973907,2.6695813895194496);
\draw[line width=2.pt,color=qqwuqq] (2.1224957668973907,2.6695813895194496) -- (2.129995751939431,2.674516166962631);
\draw[line width=2.pt,color=qqwuqq] (2.129995751939431,2.674516166962631) -- (2.137495736981471,2.6794267897139754);
\draw[line width=2.pt,color=qqwuqq] (2.137495736981471,2.6794267897139754) -- (2.144995722023511,2.684312987311155);
\draw[line width=2.pt,color=qqwuqq] (2.144995722023511,2.684312987311155) -- (2.152495707065551,2.6891744866145153);
\draw[line width=2.pt,color=qqwuqq] (2.152495707065551,2.6891744866145153) -- (2.159995692107591,2.6940110118083673);
\draw[line width=2.pt,color=qqwuqq] (2.159995692107591,2.6940110118083673) -- (2.167495677149631,2.698822284403644);
\draw[line width=2.pt,color=qqwuqq] (2.167495677149631,2.698822284403644) -- (2.1749956621916713,2.703608023241983);
\draw[line width=2.pt,color=qqwuqq] (2.1749956621916713,2.703608023241983) -- (2.1824956472337114,2.708367944501288);
\draw[line width=2.pt,color=qqwuqq] (2.1824956472337114,2.708367944501288) -- (2.1899956322757514,2.7131017617028297);
\draw[line width=2.pt,color=qqwuqq] (2.1899956322757514,2.7131017617028297) -- (2.1974956173177915,2.7178091857199442);
\draw[line width=2.pt,color=qqwuqq] (2.1974956173177915,2.7178091857199442) -- (2.2049956023598316,2.722489924788377);
\draw[line width=2.pt,color=qqwuqq] (2.2049956023598316,2.722489924788377) -- (2.2124955874018717,2.7271436845183494);
\draw[line width=2.pt,color=qqwuqq] (2.2124955874018717,2.7271436845183494) -- (2.2199955724439118,2.7317701679083948);
\draw[line width=2.pt,color=qqwuqq] (2.2199955724439118,2.7317701679083948) -- (2.227495557485952,2.7363690753610292);
\draw[line width=2.pt,color=qqwuqq] (2.227495557485952,2.7363690753610292) -- (2.234995542527992,2.7409401047003197);
\draw[line width=2.pt,color=qqwuqq] (2.234995542527992,2.7409401047003197) -- (2.242495527570032,2.745482951191414);
\draw[line width=2.pt,color=qqwuqq] (2.242495527570032,2.745482951191414) -- (2.249995512612072,2.7499973075620883);
\draw[line width=2.pt,color=qqwuqq] (2.249995512612072,2.7499973075620883) -- (2.257495497654112,2.754482864026386);
\draw[line width=2.pt,color=qqwuqq] (2.257495497654112,2.754482864026386) -- (2.2649954826961523,2.758939308310403);
\draw[line width=2.pt,color=qqwuqq] (2.2649954826961523,2.758939308310403) -- (2.2724954677381923,2.763366325680288);
\draw[line width=2.pt,color=qqwuqq] (2.2724954677381923,2.763366325680288) -- (2.2799954527802324,2.7677635989725258);
\draw[line width=2.pt,color=qqwuqq] (2.2799954527802324,2.7677635989725258) -- (2.2874954378222725,2.7721308086265557);
\draw[line width=2.pt,color=qqwuqq] (2.2874954378222725,2.7721308086265557) -- (2.2949954228643126,2.7764676327198083);
\draw[line width=2.pt,color=qqwuqq] (2.2949954228643126,2.7764676327198083) -- (2.3024954079063527,2.780773747005206);
\draw[line width=2.pt,color=qqwuqq] (2.3024954079063527,2.780773747005206) -- (2.3099953929483927,2.7850488249512066);
\draw[line width=2.pt,color=qqwuqq] (2.3099953929483927,2.7850488249512066) -- (2.317495377990433,2.7892925377844477);
\draw[line width=2.pt,color=qqwuqq] (2.317495377990433,2.7892925377844477) -- (2.324995363032473,2.793504554535052);
\draw[line width=2.pt,color=qqwuqq] (2.324995363032473,2.793504554535052) -- (2.332495348074513,2.7976845420846637);
\draw[line width=2.pt,color=qqwuqq] (2.332495348074513,2.7976845420846637) -- (2.339995333116553,2.8018321652172724);
\draw[line width=2.pt,color=qqwuqq] (2.339995333116553,2.8018321652172724) -- (2.347495318158593,2.8059470866728913);
\draw[line width=2.pt,color=qqwuqq] (2.347495318158593,2.8059470866728913) -- (2.354995303200633,2.810028967204141);
\draw[line width=2.pt,color=qqwuqq] (2.354995303200633,2.810028967204141) -- (2.3624952882426733,2.8140774656358065);
\draw[line width=2.pt,color=qqwuqq] (2.3624952882426733,2.8140774656358065) -- (2.3699952732847134,2.8180922389274183);
\draw[line width=2.pt,color=qqwuqq] (2.3699952732847134,2.8180922389274183) -- (2.3774952583267535,2.8220729422389197);
\draw[line width=2.pt,color=qqwuqq] (2.3774952583267535,2.8220729422389197) -- (2.3849952433687935,2.8260192289994652);
\draw[line width=2.pt,color=qqwuqq] (2.3849952433687935,2.8260192289994652) -- (2.3924952284108336,2.829930750979411);
\draw[line width=2.pt,color=qqwuqq] (2.3924952284108336,2.829930750979411) -- (2.3999952134528737,2.8338071583655378);
\draw[line width=2.pt,color=qqwuqq] (2.3999952134528737,2.8338071583655378) -- (2.407495198494914,2.8376480998395617);
\draw[line width=2.pt,color=qqwuqq] (2.407495198494914,2.8376480998395617) -- (2.414995183536954,2.8414532226599727);
\draw[line width=2.pt,color=qqwuqq] (2.414995183536954,2.8414532226599727) -- (2.422495168578994,2.84522217274724);
\draw[line width=2.pt,color=qqwuqq] (2.422495168578994,2.84522217274724) -- (2.429995153621034,2.848954594772427);
\draw[line width=2.pt,color=qqwuqq] (2.429995153621034,2.848954594772427) -- (2.437495138663074,2.852650132249245);
\draw[line width=2.pt,color=qqwuqq] (2.437495138663074,2.852650132249245) -- (2.444995123705114,2.85630842762959);
\draw[line width=2.pt,color=qqwuqq] (2.444995123705114,2.85630842762959) -- (2.4524951087471543,2.859929122402563);
\draw[line width=2.pt,color=qqwuqq] (2.4524951087471543,2.859929122402563) -- (2.4599950937891943,2.8635118571970315);
\draw[line width=2.pt,color=qqwuqq] (2.4599950937891943,2.8635118571970315) -- (2.4674950788312344,2.8670562718877193);
\draw[line width=2.pt,color=qqwuqq] (2.4674950788312344,2.8670562718877193) -- (2.4749950638732745,2.8705620057048606);
\draw[line width=2.pt,color=qqwuqq] (2.4749950638732745,2.8705620057048606) -- (2.4824950489153146,2.8740286973474136);
\draw[line width=2.pt,color=qqwuqq] (2.4824950489153146,2.8740286973474136) -- (2.4899950339573547,2.8774559850998447);
\draw[line width=2.pt,color=qqwuqq] (2.4899950339573547,2.8774559850998447) -- (2.4974950189993947,2.8808435069524823);
\draw[line width=2.pt,color=qqwuqq] (2.4974950189993947,2.8808435069524823) -- (2.504995004041435,2.8841909007254247);
\draw[line width=2.pt,color=qqwuqq] (2.504995004041435,2.8841909007254247) -- (2.512494989083475,2.887497804196);
\draw[line width=2.pt,color=qqwuqq] (2.512494989083475,2.887497804196) -- (2.519994974125515,2.8907638552297508);
\draw[line width=2.pt,color=qqwuqq] (2.519994974125515,2.8907638552297508) -- (2.527494959167555,2.8939886919149194);
\draw[line width=2.pt,color=qqwuqq] (2.527494959167555,2.8939886919149194) -- (2.534994944209595,2.897171952700405);
\draw[line width=2.pt,color=qqwuqq] (2.534994944209595,2.897171952700405) -- (2.5424949292516352,2.9003132765371467);
\draw[line width=2.pt,color=qqwuqq] (2.5424949292516352,2.9003132765371467) -- (2.5499949142936753,2.9034123030228924);
\draw[line width=2.pt,color=qqwuqq] (2.5499949142936753,2.9034123030228924) -- (2.5574948993357154,2.9064686725502975);
\draw[line width=2.pt,color=qqwuqq] (2.5574948993357154,2.9064686725502975) -- (2.5649948843777555,2.9094820264582877);
\draw[line width=2.pt,color=qqwuqq] (2.5649948843777555,2.9094820264582877) -- (2.5724948694197955,2.9124520071866264);
\draw[line width=2.pt,color=qqwuqq] (2.5724948694197955,2.9124520071866264) -- (2.5799948544618356,2.915378258433598);
\draw[line width=2.pt,color=qqwuqq] (2.5799948544618356,2.915378258433598) -- (2.5874948395038757,2.918260425316734);
\draw[line width=2.pt,color=qqwuqq] (2.5874948395038757,2.918260425316734) -- (2.594994824545916,2.921098154536478);
\draw[line width=2.pt,color=qqwuqq] (2.594994824545916,2.921098154536478) -- (2.602494809587956,2.9238910945426966);
\draw[line width=2.pt,color=qqwuqq] (2.602494809587956,2.9238910945426966) -- (2.609994794629996,2.9266388957039204);
\draw[line width=2.pt,color=qqwuqq] (2.609994794629996,2.9266388957039204) -- (2.617494779672036,2.9293412104792003);
\draw[line width=2.pt,color=qqwuqq] (2.617494779672036,2.9293412104792003) -- (2.624994764714076,2.9319976935924474);
\draw[line width=2.pt,color=qqwuqq] (2.624994764714076,2.9319976935924474) -- (2.632494749756116,2.9346080022091305);
\draw[line width=2.pt,color=qqwuqq] (2.632494749756116,2.9346080022091305) -- (2.6399947347981563,2.9371717961151718);
\draw[line width=2.pt,color=qqwuqq] (2.6399947347981563,2.9371717961151718) -- (2.6474947198401964,2.939688737897903);
\draw[line width=2.pt,color=qqwuqq] (2.6474947198401964,2.939688737897903) -- (2.6549947048822364,2.9421584931289115);
\draw[line width=2.pt,color=qqwuqq] (2.6549947048822364,2.9421584931289115) -- (2.6624946899242765,2.9445807305486076);
\draw[line width=2.pt,color=qqwuqq] (2.6624946899242765,2.9445807305486076) -- (2.6699946749663166,2.946955122252339);
\draw[line width=2.pt,color=qqwuqq] (2.6699946749663166,2.946955122252339) -- (2.6774946600083567,2.949281343877855);
\draw[line width=2.pt,color=qqwuqq] (2.6774946600083567,2.949281343877855) -- (2.6849946450503968,2.9515590747939466);
\draw[line width=2.pt,color=qqwuqq] (2.6849946450503968,2.9515590747939466) -- (2.692494630092437,2.9537879982900277);
\draw[line width=2.pt,color=qqwuqq] (2.692494630092437,2.9537879982900277) -- (2.699994615134477,2.9559678017664828);
\draw[line width=2.pt,color=qqwuqq] (2.699994615134477,2.9559678017664828) -- (2.707494600176517,2.9580981769255343);
\draw[line width=2.pt,color=qqwuqq] (2.707494600176517,2.9580981769255343) -- (2.714994585218557,2.9601788199624215);
\draw[line width=2.pt,color=qqwuqq] (2.714994585218557,2.9601788199624215) -- (2.722494570260597,2.9622094317566523);
\draw[line width=2.pt,color=qqwuqq] (2.722494570260597,2.9622094317566523) -- (2.7299945553026372,2.96418971806309);
\draw[line width=2.pt,color=qqwuqq] (2.7299945553026372,2.96418971806309) -- (2.7374945403446773,2.9661193897026346);
\draw[line width=2.pt,color=qqwuqq] (2.7374945403446773,2.9661193897026346) -- (2.7449945253867174,2.9679981627522434);
\draw[line width=2.pt,color=qqwuqq] (2.7449945253867174,2.9679981627522434) -- (2.7524945104287575,2.969825758734038);
\draw[line width=2.pt,color=qqwuqq] (2.7524945104287575,2.969825758734038) -- (2.7599944954707976,2.9716019048032436);
\draw[line width=2.pt,color=qqwuqq] (2.7599944954707976,2.9716019048032436) -- (2.7674944805128376,2.9733263339346845);
\draw[line width=2.pt,color=qqwuqq] (2.7674944805128376,2.9733263339346845) -- (2.7749944655548777,2.974998785107581);
\draw[line width=2.pt,color=qqwuqq] (2.7749944655548777,2.974998785107581) -- (2.782494450596918,2.97661900348837);
\draw[line width=2.pt,color=qqwuqq] (2.782494450596918,2.97661900348837) -- (2.789994435638958,2.97818674061128);
\draw[line width=2.pt,color=qqwuqq] (2.789994435638958,2.97818674061128) -- (2.797494420680998,2.979701754556382);
\draw[line width=2.pt,color=qqwuqq] (2.797494420680998,2.979701754556382) -- (2.804994405723038,2.9811638101248508);
\draw[line width=2.pt,color=qqwuqq] (2.804994405723038,2.9811638101248508) -- (2.812494390765078,2.9825726790111444);
\draw[line width=2.pt,color=qqwuqq] (2.812494390765078,2.9825726790111444) -- (2.819994375807118,2.983928139971848);
\draw[line width=2.pt,color=qqwuqq] (2.819994375807118,2.983928139971848) -- (2.8274943608491583,2.9852299789908847);
\draw[line width=2.pt,color=qqwuqq] (2.8274943608491583,2.9852299789908847) -- (2.8349943458911984,2.986477989440844);
\draw[line width=2.pt,color=qqwuqq] (2.8349943458911984,2.986477989440844) -- (2.8424943309332384,2.9876719722401397);
\draw[line width=2.pt,color=qqwuqq] (2.8424943309332384,2.9876719722401397) -- (2.8499943159752785,2.9888117360057405);
\draw[line width=2.pt,color=qqwuqq] (2.8499943159752785,2.9888117360057405) -- (2.8574943010173186,2.9898970972012098);
\draw[line width=2.pt,color=qqwuqq] (2.8574943010173186,2.9898970972012098) -- (2.8649942860593587,2.9909278802797976);
\draw[line width=2.pt,color=qqwuqq] (2.8649942860593587,2.9909278802797976) -- (2.8724942711013988,2.991903917822332);
\draw[line width=2.pt,color=qqwuqq] (2.8724942711013988,2.991903917822332) -- (2.879994256143439,2.992825050669663);
\draw[line width=2.pt,color=qqwuqq] (2.879994256143439,2.992825050669663) -- (2.887494241185479,2.993691128049429);
\draw[line width=2.pt,color=qqwuqq] (2.887494241185479,2.993691128049429) -- (2.894994226227519,2.994502007696904);
\draw[line width=2.pt,color=qqwuqq] (2.894994226227519,2.994502007696904) -- (2.902494211269559,2.9952575559697174);
\draw[line width=2.pt,color=qqwuqq] (2.902494211269559,2.9952575559697174) -- (2.909994196311599,2.9959576479562253);
\draw[line width=2.pt,color=qqwuqq] (2.909994196311599,2.9959576479562253) -- (2.9174941813536392,2.996602167577333);
\draw[line width=2.pt,color=qqwuqq] (2.9174941813536392,2.996602167577333) -- (2.9249941663956793,2.9971910076815833);
\draw[line width=2.pt,color=qqwuqq] (2.9249941663956793,2.9971910076815833) -- (2.9324941514377194,2.997724070133322);
\draw[line width=2.pt,color=qqwuqq] (2.9324941514377194,2.997724070133322) -- (2.9399941364797595,2.9982012658937878);
\draw[line width=2.pt,color=qqwuqq] (2.9399941364797595,2.9982012658937878) -- (2.9474941215217996,2.998622515094948);
\draw[line width=2.pt,color=qqwuqq] (2.9474941215217996,2.998622515094948) -- (2.9549941065638397,2.998987747105967);
\draw[line width=2.pt,color=qqwuqq] (2.9549941065638397,2.998987747105967) -- (2.9624940916058797,2.999296900592155);
\draw[line width=2.pt,color=qqwuqq] (2.9624940916058797,2.999296900592155) -- (2.96999407664792,2.999549923566293);
\draw[line width=2.pt,color=qqwuqq] (2.96999407664792,2.999549923566293) -- (2.97749406168996,2.9997467734322396);
\draw[line width=2.pt,color=qqwuqq] (2.97749406168996,2.9997467734322396) -- (2.984994046732,2.999887417020723);
\draw[line width=2.pt,color=qqwuqq] (2.984994046732,2.999887417020723) -- (2.99249403177404,2.9999718306172527);
\draw[line width=2.pt,color=qqwuqq] (2.99249403177404,2.9999718306172527) -- (2.99999401681608,2.9999999999821005);
\draw[line width=2.pt,color=qqwuqq] (3.0000048488746094,2.999999999988244) -- (3.0000048488746094,2.999999999988244);
\draw[line width=2.pt,color=qqwuqq] (3.0000048488746094,2.999999999988244) -- (3.005004836124586,2.99998747588611);
\draw[line width=2.pt,color=qqwuqq] (3.005004836124586,2.99998747588611) -- (3.0100048233745627,2.999949953006973);
\draw[line width=2.pt,color=qqwuqq] (3.0100048233745627,2.999949953006973) -- (3.0150048106245393,2.9998874341645942);
\draw[line width=2.pt,color=qqwuqq] (3.0150048106245393,2.9998874341645942) -- (3.020004797874516,2.9997999240461937);
\draw[line width=2.pt,color=qqwuqq] (3.020004797874516,2.9997999240461937) -- (3.0250047851244926,2.999687429210688);
\draw[line width=2.pt,color=qqwuqq] (3.0250047851244926,2.999687429210688) -- (3.030004772374469,2.99954995808624);
\draw[line width=2.pt,color=qqwuqq] (3.030004772374469,2.99954995808624) -- (3.035004759624446,2.9993875209671);
\draw[line width=2.pt,color=qqwuqq] (3.035004759624446,2.9993875209671) -- (3.0400047468744225,2.9992001300097573);
\draw[line width=2.pt,color=qqwuqq] (3.0400047468744225,2.9992001300097573) -- (3.045004734124399,2.998987799228397);
\draw[line width=2.pt,color=qqwuqq] (3.045004734124399,2.998987799228397) -- (3.0500047213743757,2.9987505444896714);
\draw[line width=2.pt,color=qqwuqq] (3.0500047213743757,2.9987505444896714) -- (3.0550047086243524,2.9984883835067864);
\draw[line width=2.pt,color=qqwuqq] (3.0550047086243524,2.9984883835067864) -- (3.060004695874329,2.9982013358329076);
\draw[line width=2.pt,color=qqwuqq] (3.060004695874329,2.9982013358329076) -- (3.0650046831243056,2.9978894228538993);
\draw[line width=2.pt,color=qqwuqq] (3.0650046831243056,2.9978894228538993) -- (3.0700046703742823,2.9975526677803908);
\draw[line width=2.pt,color=qqwuqq] (3.0700046703742823,2.9975526677803908) -- (3.075004657624259,2.997191095639188);
\draw[line width=2.pt,color=qqwuqq] (3.075004657624259,2.997191095639188) -- (3.0800046448742355,2.996804733264031);
\draw[line width=2.pt,color=qqwuqq] (3.0800046448742355,2.996804733264031) -- (3.085004632124212,2.9963936092857058);
\draw[line width=2.pt,color=qqwuqq] (3.085004632124212,2.9963936092857058) -- (3.090004619374189,2.995957754121525);
\draw[line width=2.pt,color=qqwuqq] (3.090004619374189,2.995957754121525) -- (3.0950046066241654,2.9954971999641753);
\draw[line width=2.pt,color=qqwuqq] (3.0950046066241654,2.9954971999641753) -- (3.100004593874142,2.9950119807699536);
\draw[line width=2.pt,color=qqwuqq] (3.100004593874142,2.9950119807699536) -- (3.1050045811241187,2.9945021322463923);
\draw[line width=2.pt,color=qqwuqq] (3.1050045811241187,2.9945021322463923) -- (3.1100045683740953,2.9939676918392877);
\draw[line width=2.pt,color=qqwuqq] (3.1100045683740953,2.9939676918392877) -- (3.115004555624072,2.9934086987191426);
\draw[line width=2.pt,color=qqwuqq] (3.115004555624072,2.9934086987191426) -- (3.1200045428740486,2.9928251937670356);
\draw[line width=2.pt,color=qqwuqq] (3.1200045428740486,2.9928251937670356) -- (3.1250045301240252,2.992217219559925);
\draw[line width=2.pt,color=qqwuqq] (3.1250045301240252,2.992217219559925) -- (3.130004517374002,2.991584820355402);
\draw[line width=2.pt,color=qqwuqq] (3.130004517374002,2.991584820355402) -- (3.1350045046239785,2.9909280420759052);
\draw[line width=2.pt,color=qqwuqq] (3.1350045046239785,2.9909280420759052) -- (3.140004491873955,2.990246932292412);
\draw[line width=2.pt,color=qqwuqq] (3.140004491873955,2.990246932292412) -- (3.1450044791239318,2.989541540207613);
\draw[line width=2.pt,color=qqwuqq] (3.1450044791239318,2.989541540207613) -- (3.1500044663739084,2.9888119166385905);
\draw[line width=2.pt,color=qqwuqq] (3.1500044663739084,2.9888119166385905) -- (3.155004453623885,2.988058113999011);
\draw[line width=2.pt,color=qqwuqq] (3.155004453623885,2.988058113999011) -- (3.1600044408738617,2.9872801862808465);
\draw[line width=2.pt,color=qqwuqq] (3.1600044408738617,2.9872801862808465) -- (3.1650044281238383,2.9864781890356404);
\draw[line width=2.pt,color=qqwuqq] (3.1650044281238383,2.9864781890356404) -- (3.170004415373815,2.9856521793553297);
\draw[line width=2.pt,color=qqwuqq] (3.170004415373815,2.9856521793553297) -- (3.1750044026237916,2.98480221585264);
\draw[line width=2.pt,color=qqwuqq] (3.1750044026237916,2.98480221585264) -- (3.180004389873768,2.983928358641071);
\draw[line width=2.pt,color=qqwuqq] (3.180004389873768,2.983928358641071) -- (3.185004377123745,2.983030669314485);
\draw[line width=2.pt,color=qqwuqq] (3.185004377123745,2.983030669314485) -- (3.1900043643737215,2.9821092109263088);
\draw[line width=2.pt,color=qqwuqq] (3.1900043643737215,2.9821092109263088) -- (3.195004351623698,2.981164047968379);
\draw[line width=2.pt,color=qqwuqq] (3.195004351623698,2.981164047968379) -- (3.2000043388736747,2.9801952463494317);
\draw[line width=2.pt,color=qqwuqq] (3.2000043388736747,2.9801952463494317) -- (3.2050043261236514,2.979202873373258);
\draw[line width=2.pt,color=qqwuqq] (3.2050043261236514,2.979202873373258) -- (3.210004313373628,2.9781869977165445);
\draw[line width=2.pt,color=qqwuqq] (3.210004313373628,2.9781869977165445) -- (3.2150043006236046,2.97714768940641);
\draw[line width=2.pt,color=qqwuqq] (3.2150043006236046,2.97714768940641) -- (3.2200042878735813,2.9760850197976585);
\draw[line width=2.pt,color=qqwuqq] (3.2200042878735813,2.9760850197976585) -- (3.225004275123558,2.9749990615497577);
\draw[line width=2.pt,color=qqwuqq] (3.225004275123558,2.9749990615497577) -- (3.2300042623735346,2.97388988860357);
\draw[line width=2.pt,color=qqwuqq] (3.2300042623735346,2.97388988860357) -- (3.235004249623511,2.972757576157843);
\draw[line width=2.pt,color=qqwuqq] (3.235004249623511,2.972757576157843) -- (3.240004236873488,2.971602200645477);
\draw[line width=2.pt,color=qqwuqq] (3.240004236873488,2.971602200645477) -- (3.2450042241234645,2.970423839709592);
\draw[line width=2.pt,color=qqwuqq] (3.2450042241234645,2.970423839709592) -- (3.250004211373441,2.9692225721793983);
\draw[line width=2.pt,color=qqwuqq] (3.250004211373441,2.9692225721793983) -- (3.2550041986234177,2.9679984780459034);
\draw[line width=2.pt,color=qqwuqq] (3.2550041986234177,2.9679984780459034) -- (3.2600041858733944,2.9667516384374535);
\draw[line width=2.pt,color=qqwuqq] (3.2600041858733944,2.9667516384374535) -- (3.265004173123371,2.9654821355951357);
\draw[line width=2.pt,color=qqwuqq] (3.265004173123371,2.9654821355951357) -- (3.2700041603733476,2.9641900528480543);
\draw[line width=2.pt,color=qqwuqq] (3.2700041603733476,2.9641900528480543) -- (3.2750041476233243,2.962875474588499);
\draw[line width=2.pt,color=qqwuqq] (3.2750041476233243,2.962875474588499) -- (3.280004134873301,2.9615384862470124);
\draw[line width=2.pt,color=qqwuqq] (3.280004134873301,2.9615384862470124) -- (3.2850041221232775,2.9601791742673838);
\draw[line width=2.pt,color=qqwuqq] (3.2850041221232775,2.9601791742673838) -- (3.290004109373254,2.95879762608157);
\draw[line width=2.pt,color=qqwuqq] (3.290004109373254,2.95879762608157) -- (3.295004096623231,2.957393930084575);
\draw[line width=2.pt,color=qqwuqq] (3.295004096623231,2.957393930084575) -- (3.3000040838732074,2.955968175609286);
\draw[line width=2.pt,color=qqwuqq] (3.3000040838732074,2.955968175609286) -- (3.305004071123184,2.954520452901294);
\draw[line width=2.pt,color=qqwuqq] (3.305004071123184,2.954520452901294) -- (3.3100040583731607,2.9530508530937);
\draw[line width=2.pt,color=qqwuqq] (3.3100040583731607,2.9530508530937) -- (3.3150040456231373,2.9515594681819364);
\draw[line width=2.pt,color=qqwuqq] (3.3150040456231373,2.9515594681819364) -- (3.320004032873114,2.9500463909986037);
\draw[line width=2.pt,color=qqwuqq] (3.320004032873114,2.9500463909986037) -- (3.3250040201230906,2.948511715188338);
\draw[line width=2.pt,color=qqwuqq] (3.3250040201230906,2.948511715188338) -- (3.3300040073730672,2.9469555351827337);
\draw[line width=2.pt,color=qqwuqq] (3.3300040073730672,2.9469555351827337) -- (3.335003994623044,2.945377946175315);
\draw[line width=2.pt,color=qqwuqq] (3.335003994623044,2.945377946175315) -- (3.3400039818730205,2.943779044096592);
\draw[line width=2.pt,color=qqwuqq] (3.3400039818730205,2.943779044096592) -- (3.345003969122997,2.9421589255891876);
\draw[line width=2.pt,color=qqwuqq] (3.345003969122997,2.9421589255891876) -- (3.3500039563729738,2.940517687983072);
\draw[line width=2.pt,color=qqwuqq] (3.3500039563729738,2.940517687983072) -- (3.3550039436229504,2.9388554292708995);
\draw[line width=2.pt,color=qqwuqq] (3.3550039436229504,2.9388554292708995) -- (3.360003930872927,2.9371722480834634);
\draw[line width=2.pt,color=qqwuqq] (3.360003930872927,2.9371722480834634) -- (3.3650039181229037,2.9354682436652855);
\draw[line width=2.pt,color=qqwuqq] (3.3650039181229037,2.9354682436652855) -- (3.3700039053728803,2.9337435158503453);
\draw[line width=2.pt,color=qqwuqq] (3.3700039053728803,2.9337435158503453) -- (3.375003892622857,2.931998165037955);
\draw[line width=2.pt,color=qqwuqq] (3.375003892622857,2.931998165037955) -- (3.3800038798728336,2.9302322921688058);
\draw[line width=2.pt,color=qqwuqq] (3.3800038798728336,2.9302322921688058) -- (3.38500386712281,2.928445998701177);
\draw[line width=2.pt,color=qqwuqq] (3.38500386712281,2.928445998701177) -- (3.390003854372787,2.926639386587327);
\draw[line width=2.pt,color=qqwuqq] (3.390003854372787,2.926639386587327) -- (3.3950038416227635,2.9248125582500784);
\draw[line width=2.pt,color=qqwuqq] (3.3950038416227635,2.9248125582500784) -- (3.40000382887274,2.9229656165595954);
\draw[line width=2.pt,color=qqwuqq] (3.40000382887274,2.9229656165595954) -- (3.4050038161227167,2.9210986648103714);
\draw[line width=2.pt,color=qqwuqq] (3.4050038161227167,2.9210986648103714) -- (3.4100038033726934,2.9192118066984287);
\draw[line width=2.pt,color=qqwuqq] (3.4100038033726934,2.9192118066984287) -- (3.41500379062267,2.9173051462987436);
\draw[line width=2.pt,color=qqwuqq] (3.41500379062267,2.9173051462987436) -- (3.4200037778726466,2.9153787880428967);
\draw[line width=2.pt,color=qqwuqq] (3.4200037778726466,2.9153787880428967) -- (3.4250037651226233,2.9134328366969644);
\draw[line width=2.pt,color=qqwuqq] (3.4250037651226233,2.9134328366969644) -- (3.4300037523726,2.9114673973396497);
\draw[line width=2.pt,color=qqwuqq] (3.4300037523726,2.9114673973396497) -- (3.4350037396225765,2.909482575340666);
\draw[line width=2.pt,color=qqwuqq] (3.4350037396225765,2.909482575340666) -- (3.440003726872553,2.9074784763393735);
\draw[line width=2.pt,color=qqwuqq] (3.440003726872553,2.9074784763393735) -- (3.44500371412253,2.9054552062236803);
\draw[line width=2.pt,color=qqwuqq] (3.44500371412253,2.9054552062236803) -- (3.4500037013725064,2.9034128711092055);
\draw[line width=2.pt,color=qqwuqq] (3.4500037013725064,2.9034128711092055) -- (3.455003688622483,2.9013515773187195);
\draw[line width=2.pt,color=qqwuqq] (3.455003688622483,2.9013515773187195) -- (3.4600036758724597,2.899271431361857);
\draw[line width=2.pt,color=qqwuqq] (3.4600036758724597,2.899271431361857) -- (3.4650036631224364,2.8971725399151125);
\draw[line width=2.pt,color=qqwuqq] (3.4650036631224364,2.8971725399151125) -- (3.470003650372413,2.8950550098021197);
\draw[line width=2.pt,color=qqwuqq] (3.470003650372413,2.8950550098021197) -- (3.4750036376223896,2.8929189479742226);
\draw[line width=2.pt,color=qqwuqq] (3.4750036376223896,2.8929189479742226) -- (3.4800036248723663,2.8907644614913335);
\draw[line width=2.pt,color=qqwuqq] (3.4800036248723663,2.8907644614913335) -- (3.485003612122343,2.8885916575030937);
\draw[line width=2.pt,color=qqwuqq] (3.485003612122343,2.8885916575030937) -- (3.4900035993723195,2.8864006432303273);
\draw[line width=2.pt,color=qqwuqq] (3.4900035993723195,2.8864006432303273) -- (3.495003586622296,2.884191525946797);
\draw[line width=2.pt,color=qqwuqq] (3.495003586622296,2.884191525946797) -- (3.500003573872273,2.8819644129612665);
\draw[line width=2.pt,color=qqwuqq] (3.500003573872273,2.8819644129612665) -- (3.5050035611222494,2.8797194115998646);
\draw[line width=2.pt,color=qqwuqq] (3.5050035611222494,2.8797194115998646) -- (3.510003548372226,2.8774566291887598);
\draw[line width=2.pt,color=qqwuqq] (3.510003548372226,2.8774566291887598) -- (3.5150035356222027,2.875176173037142);
\draw[line width=2.pt,color=qqwuqq] (3.5150035356222027,2.875176173037142) -- (3.5200035228721793,2.872878150420516);
\draw[line width=2.pt,color=qqwuqq] (3.5200035228721793,2.872878150420516) -- (3.525003510122156,2.870562668564305);
\draw[line width=2.pt,color=qqwuqq] (3.525003510122156,2.870562668564305) -- (3.5300034973721326,2.868229834627767);
\draw[line width=2.pt,color=qqwuqq] (3.5300034973721326,2.868229834627767) -- (3.5350034846221092,2.8658797556882254);
\draw[line width=2.pt,color=qqwuqq] (3.5350034846221092,2.8658797556882254) -- (3.540003471872086,2.863512538725611);
\draw[line width=2.pt,color=qqwuqq] (3.540003471872086,2.863512538725611) -- (3.5450034591220625,2.861128290607316);
\draw[line width=2.pt,color=qqwuqq] (3.5450034591220625,2.861128290607316) -- (3.550003446372039,2.858727118073368);
\draw[line width=2.pt,color=qqwuqq] (3.550003446372039,2.858727118073368) -- (3.5550034336220158,2.8563091277219064);
\draw[line width=2.pt,color=qqwuqq] (3.5550034336220158,2.8563091277219064) -- (3.5600034208719924,2.8538744259949813);
\draw[line width=2.pt,color=qqwuqq] (3.5600034208719924,2.8538744259949813) -- (3.565003408121969,2.851423119164659);
\draw[line width=2.pt,color=qqwuqq] (3.565003408121969,2.851423119164659) -- (3.5700033953719457,2.8489553133194407);
\draw[line width=2.pt,color=qqwuqq] (3.5700033953719457,2.8489553133194407) -- (3.5750033826219223,2.8464711143509875);
\draw[line width=2.pt,color=qqwuqq] (3.5750033826219223,2.8464711143509875) -- (3.580003369871899,2.8439706279411587);
\draw[line width=2.pt,color=qqwuqq] (3.580003369871899,2.8439706279411587) -- (3.5850033571218756,2.841453959549356);
\draw[line width=2.pt,color=qqwuqq] (3.5850033571218756,2.841453959549356) -- (3.590003344371852,2.8389212144001723);
\draw[line width=2.pt,color=qqwuqq] (3.590003344371852,2.8389212144001723) -- (3.595003331621829,2.836372497471345);
\draw[line width=2.pt,color=qqwuqq] (3.595003331621829,2.836372497471345) -- (3.6000033188718055,2.8338079134820107);
\draw[line width=2.pt,color=qqwuqq] (3.6000033188718055,2.8338079134820107) -- (3.605003306121782,2.8312275668812656);
\draw[line width=2.pt,color=qqwuqq] (3.605003306121782,2.8312275668812656) -- (3.6100032933717587,2.828631561837015);
\draw[line width=2.pt,color=qqwuqq] (3.6100032933717587,2.828631561837015) -- (3.6150032806217354,2.8260200022251247);
\draw[line width=2.pt,color=qqwuqq] (3.6150032806217354,2.8260200022251247) -- (3.620003267871712,2.8233929916188663);
\draw[line width=2.pt,color=qqwuqq] (3.620003267871712,2.8233929916188663) -- (3.6250032551216886,2.8207506332786494);
\draw[line width=2.pt,color=qqwuqq] (3.6250032551216886,2.8207506332786494) -- (3.6300032423716653,2.8180930301420455);
\draw[line width=2.pt,color=qqwuqq] (3.6300032423716653,2.8180930301420455) -- (3.635003229621642,2.815420284814097);
\draw[line width=2.pt,color=qqwuqq] (3.635003229621642,2.815420284814097) -- (3.6400032168716185,2.812732499557905);
\draw[line width=2.pt,color=qqwuqq] (3.6400032168716185,2.812732499557905) -- (3.645003204121595,2.8100297762855053);
\draw[line width=2.pt,color=qqwuqq] (3.645003204121595,2.8100297762855053) -- (3.650003191371572,2.807312216549013);
\draw[line width=2.pt,color=qqwuqq] (3.650003191371572,2.807312216549013) -- (3.6550031786215484,2.804579921532045);
\draw[line width=2.pt,color=qqwuqq] (3.6550031786215484,2.804579921532045) -- (3.660003165871525,2.801832992041412);
\draw[line width=2.pt,color=qqwuqq] (3.660003165871525,2.801832992041412) -- (3.6650031531215017,2.799071528499079);
\draw[line width=2.pt,color=qqwuqq] (3.6650031531215017,2.799071528499079) -- (3.6700031403714783,2.796295630934388);
\draw[line width=2.pt,color=qqwuqq] (3.6700031403714783,2.796295630934388) -- (3.675003127621455,2.793505398976545);
\draw[line width=2.pt,color=qqwuqq] (3.675003127621455,2.793505398976545) -- (3.6800031148714316,2.7907009318473577);
\draw[line width=2.pt,color=qqwuqq] (3.6800031148714316,2.7907009318473577) -- (3.6850031021214082,2.787882328354234);
\draw[line width=2.pt,color=qqwuqq] (3.6850031021214082,2.787882328354234) -- (3.690003089371385,2.7850496868834282);
\draw[line width=2.pt,color=qqwuqq] (3.690003089371385,2.7850496868834282) -- (3.6950030766213615,2.782203105393531);
\draw[line width=2.pt,color=qqwuqq] (3.6950030766213615,2.782203105393531) -- (3.700003063871338,2.779342681409209);
\draw[line width=2.pt,color=qqwuqq] (3.700003063871338,2.779342681409209) -- (3.705003051121315,2.7764685120151738);
\draw[line width=2.pt,color=qqwuqq] (3.705003051121315,2.7764685120151738) -- (3.7100030383712914,2.7735806938504);
\draw[line width=2.pt,color=qqwuqq] (3.7100030383712914,2.7735806938504) -- (3.715003025621268,2.770679323102565);
\draw[line width=2.pt,color=qqwuqq] (3.715003025621268,2.770679323102565) -- (3.7200030128712447,2.7677644955027185);
\draw[line width=2.pt,color=qqwuqq] (3.7200030128712447,2.7677644955027185) -- (3.7250030001212213,2.7648363063201815);
\draw[line width=2.pt,color=qqwuqq] (3.7250030001212213,2.7648363063201815) -- (3.730002987371198,2.7618948503576632);
\draw[line width=2.pt,color=qqwuqq] (3.730002987371198,2.7618948503576632) -- (3.7350029746211746,2.7589402219465917);
\draw[line width=2.pt,color=qqwuqq] (3.7350029746211746,2.7589402219465917) -- (3.740002961871151,2.7559725149426653);
\draw[line width=2.pt,color=qqwuqq] (3.740002961871151,2.7559725149426653) -- (3.745002949121128,2.752991822721608);
\draw[line width=2.pt,color=qqwuqq] (3.745002949121128,2.752991822721608) -- (3.7500029363711045,2.74999823817513);
\draw[line width=2.pt,color=qqwuqq] (3.7500029363711045,2.74999823817513) -- (3.755002923621081,2.7469918537070956);
\draw[line width=2.pt,color=qqwuqq] (3.755002923621081,2.7469918537070956) -- (3.7600029108710578,2.7439727612298843);
\draw[line width=2.pt,color=qqwuqq] (3.7600029108710578,2.7439727612298843) -- (3.7650028981210344,2.740941052160948);
\draw[line width=2.pt,color=qqwuqq] (3.7650028981210344,2.740941052160948) -- (3.770002885371011,2.737896817419557);
\draw[line width=2.pt,color=qqwuqq] (3.770002885371011,2.737896817419557) -- (3.7750028726209877,2.7348401474237405);
\draw[line width=2.pt,color=qqwuqq] (3.7750028726209877,2.7348401474237405) -- (3.7800028598709643,2.7317711320873967);
\draw[line width=2.pt,color=qqwuqq] (3.7800028598709643,2.7317711320873967) -- (3.785002847120941,2.728689860817596);
\draw[line width=2.pt,color=qqwuqq] (3.785002847120941,2.728689860817596) -- (3.7900028343709176,2.7255964225120506);
\draw[line width=2.pt,color=qqwuqq] (3.7900028343709176,2.7255964225120506) -- (3.795002821620894,2.7224909055567617);
\draw[line width=2.pt,color=qqwuqq] (3.795002821620894,2.7224909055567617) -- (3.800002808870871,2.71937339782383);
\draw[line width=2.pt,color=qqwuqq] (3.800002808870871,2.71937339782383) -- (3.8050027961208475,2.7162439866694363);
\draw[line width=2.pt,color=qqwuqq] (3.8050027961208475,2.7162439866694363) -- (3.810002783370824,2.7131027589319796);
\draw[line width=2.pt,color=qqwuqq] (3.810002783370824,2.7131027589319796) -- (3.8150027706208007,2.7099498009303744);
\draw[line width=2.pt,color=qqwuqq] (3.8150027706208007,2.7099498009303744) -- (3.8200027578707774,2.706785198462498);
\draw[line width=2.pt,color=qqwuqq] (3.8200027578707774,2.706785198462498) -- (3.825002745120754,2.7036090368037966);
\draw[line width=2.pt,color=qqwuqq] (3.825002745120754,2.7036090368037966) -- (3.8300027323707306,2.7004214007060297);
\draw[line width=2.pt,color=qqwuqq] (3.8300027323707306,2.7004214007060297) -- (3.8350027196207073,2.6972223743961607);
\draw[line width=2.pt,color=qqwuqq] (3.8350027196207073,2.6972223743961607) -- (3.840002706870684,2.694012041575392);
\draw[line width=2.pt,color=qqwuqq] (3.840002706870684,2.694012041575392) -- (3.8450026941206605,2.690790485418329);
\draw[line width=2.pt,color=qqwuqq] (3.8450026941206605,2.690790485418329) -- (3.850002681370637,2.6875577885722843);
\draw[line width=2.pt,color=qqwuqq] (3.850002681370637,2.6875577885722843) -- (3.855002668620614,2.684314033156707);
\draw[line width=2.pt,color=qqwuqq] (3.855002668620614,2.684314033156707) -- (3.8600026558705904,2.6810593007627412);
\draw[line width=2.pt,color=qqwuqq] (3.8600026558705904,2.6810593007627412) -- (3.865002643120567,2.6777936724529083);
\draw[line width=2.pt,color=qqwuqq] (3.865002643120567,2.6777936724529083) -- (3.8700026303705437,2.6745172287609074);
\draw[line width=2.pt,color=qqwuqq] (3.8700026303705437,2.6745172287609074) -- (3.8750026176205203,2.671230049691534);
\draw[line width=2.pt,color=qqwuqq] (3.8750026176205203,2.671230049691534) -- (3.880002604870497,2.6679322147207145);
\draw[line width=2.pt,color=qqwuqq] (3.880002604870497,2.6679322147207145) -- (3.8850025921204736,2.6646238027956475);
\draw[line width=2.pt,color=qqwuqq] (3.8850025921204736,2.6646238027956475) -- (3.8900025793704502,2.661304892335057);
\draw[line width=2.pt,color=qqwuqq] (3.8900025793704502,2.661304892335057) -- (3.895002566620427,2.657975561229546);
\draw[line width=2.pt,color=qqwuqq] (3.895002566620427,2.657975561229546) -- (3.9000025538704035,2.654635886842061);
\draw[line width=2.pt,color=qqwuqq] (3.9000025538704035,2.654635886842061) -- (3.90500254112038,2.651285946008441);
\draw[line width=2.pt,color=qqwuqq] (3.90500254112038,2.651285946008441) -- (3.910002528370357,2.6479258150380796);
\draw[line width=2.pt,color=qqwuqq] (3.910002528370357,2.6479258150380796) -- (3.9150025156203334,2.6445555697146643);
\draw[line width=2.pt,color=qqwuqq] (3.9150025156203334,2.6445555697146643) -- (3.92000250287031,2.6411752852970203);
\draw[line width=2.pt,color=qqwuqq] (3.92000250287031,2.6411752852970203) -- (3.9250024901202867,2.637785036520032);
\draw[line width=2.pt,color=qqwuqq] (3.9250024901202867,2.637785036520032) -- (3.9300024773702633,2.634384897595656);
\draw[line width=2.pt,color=qqwuqq] (3.9300024773702633,2.634384897595656) -- (3.93500246462024,2.6309749422140136);
\draw[line width=2.pt,color=qqwuqq] (3.93500246462024,2.6309749422140136) -- (3.9400024518702166,2.627555243544565);
\draw[line width=2.pt,color=qqwuqq] (3.9400024518702166,2.627555243544565) -- (3.945002439120193,2.624125874237358);
\draw[line width=2.pt,color=qqwuqq] (3.945002439120193,2.624125874237358) -- (3.95000242637017,2.620686906424357);
\draw[line width=2.pt,color=qqwuqq] (3.95000242637017,2.620686906424357) -- (3.9550024136201465,2.61723841172084);
\draw[line width=2.pt,color=qqwuqq] (3.9550024136201465,2.61723841172084) -- (3.960002400870123,2.613780461226866);
\draw[line width=2.pt,color=qqwuqq] (3.960002400870123,2.613780461226866) -- (3.9650023881200998,2.6103131255288137);
\draw[line width=2.pt,color=qqwuqq] (3.9650023881200998,2.6103131255288137) -- (3.9700023753700764,2.606836474700981);
\draw[line width=2.pt,color=qqwuqq] (3.9700023753700764,2.606836474700981) -- (3.975002362620053,2.6033505783072526);
\draw[line width=2.pt,color=qqwuqq] (3.975002362620053,2.6033505783072526) -- (3.9800023498700297,2.5998555054028243);
\draw[line width=2.pt,color=qqwuqq] (3.9800023498700297,2.5998555054028243) -- (3.9850023371200063,2.596351324535988);
\draw[line width=2.pt,color=qqwuqq] (3.9850023371200063,2.596351324535988) -- (3.990002324369983,2.592838103749974);
\draw[line width=2.pt,color=qqwuqq] (3.990002324369983,2.592838103749974) -- (3.9950023116199596,2.589315910584846);
\draw[line width=2.pt,color=qqwuqq] (3.9950023116199596,2.589315910584846) -- (4.000002298869936,2.5857848120794498);
\draw[line width=2.pt,color=qqwuqq] (4.000002298869936,2.5857848120794498) -- (4.005002286119913,2.582244874773415);
\draw[line width=2.pt,color=qqwuqq] (4.005002286119913,2.582244874773415) -- (4.0100022733698895,2.5786961647091973);
\draw[line width=2.pt,color=qqwuqq] (4.0100022733698895,2.5786961647091973) -- (4.015002260619866,2.57513874743418);
\draw[line width=2.pt,color=qqwuqq] (4.015002260619866,2.57513874743418) -- (4.020002247869843,2.5715726880028047);
\draw[line width=2.pt,color=qqwuqq] (4.020002247869843,2.5715726880028047) -- (4.025002235119819,2.567998050978762);
\draw[line width=2.pt,color=qqwuqq] (4.025002235119819,2.567998050978762) -- (4.030002222369796,2.5644149004372054);
\draw[line width=2.pt,color=qqwuqq] (4.030002222369796,2.5644149004372054) -- (4.035002209619773,2.560823299967022);
\draw[line width=2.pt,color=qqwuqq] (4.035002209619773,2.560823299967022) -- (4.040002196869749,2.557223312673127);
\draw[line width=2.pt,color=qqwuqq] (4.040002196869749,2.557223312673127) -- (4.045002184119726,2.5536150011788017);
\draw[line width=2.pt,color=qqwuqq] (4.045002184119726,2.5536150011788017) -- (4.0500021713697025,2.549998427628063);
\draw[line width=2.pt,color=qqwuqq] (4.0500021713697025,2.549998427628063) -- (4.055002158619679,2.546373653688066);
\draw[line width=2.pt,color=qqwuqq] (4.055002158619679,2.546373653688066) -- (4.060002145869656,2.5427407405515394);
\draw[line width=2.pt,color=qqwuqq] (4.060002145869656,2.5427407405515394) -- (4.065002133119632,2.5390997489392486);
\draw[line width=2.pt,color=qqwuqq] (4.065002133119632,2.5390997489392486) -- (4.070002120369609,2.535450739102484);
\draw[line width=2.pt,color=qqwuqq] (4.070002120369609,2.535450739102484) -- (4.075002107619586,2.531793770825586);
\draw[line width=2.pt,color=qqwuqq] (4.075002107619586,2.531793770825586) -- (4.080002094869562,2.528128903428482);
\draw[line width=2.pt,color=qqwuqq] (4.080002094869562,2.528128903428482) -- (4.085002082119539,2.5244561957692566);
\draw[line width=2.pt,color=qqwuqq] (4.085002082119539,2.5244561957692566) -- (4.090002069369516,2.5207757062467415);
\draw[line width=2.pt,color=qqwuqq] (4.090002069369516,2.5207757062467415) -- (4.095002056619492,2.5170874928031264);
\draw[line width=2.pt,color=qqwuqq] (4.095002056619492,2.5170874928031264) -- (4.100002043869469,2.513391612926589);
\draw[line width=2.pt,color=qqwuqq] (4.100002043869469,2.513391612926589) -- (4.1050020311194455,2.5096881236539446);
\draw[line width=2.pt,color=qqwuqq] (4.1050020311194455,2.5096881236539446) -- (4.110002018369422,2.5059770815733144);
\draw[line width=2.pt,color=qqwuqq] (4.110002018369422,2.5059770815733144) -- (4.115002005619399,2.502258542826806);
\draw[line width=2.pt,color=qqwuqq] (4.115002005619399,2.502258542826806) -- (4.120001992869375,2.4985325631132147);
\draw[line width=2.pt,color=qqwuqq] (4.120001992869375,2.4985325631132147) -- (4.125001980119352,2.4947991976907327);
\draw[line width=2.pt,color=qqwuqq] (4.125001980119352,2.4947991976907327) -- (4.130001967369329,2.4910585013796744);
\draw[line width=2.pt,color=qqwuqq] (4.130001967369329,2.4910585013796744) -- (4.135001954619305,2.487310528565217);
\draw[line width=2.pt,color=qqwuqq] (4.135001954619305,2.487310528565217) -- (4.140001941869282,2.4835553332001417);
\draw[line width=2.pt,color=qqwuqq] (4.140001941869282,2.4835553332001417) -- (4.145001929119259,2.4797929688075957);
\draw[line width=2.pt,color=qqwuqq] (4.145001929119259,2.4797929688075957) -- (4.150001916369235,2.4760234884838566);
\draw[line width=2.pt,color=qqwuqq] (4.150001916369235,2.4760234884838566) -- (4.155001903619212,2.4722469449011064);
\draw[line width=2.pt,color=qqwuqq] (4.155001903619212,2.4722469449011064) -- (4.1600018908691885,2.468463390310211);
\draw[line width=2.pt,color=qqwuqq] (4.1600018908691885,2.468463390310211) -- (4.165001878119165,2.464672876543509);
\draw[line width=2.pt,color=qqwuqq] (4.165001878119165,2.464672876543509) -- (4.170001865369142,2.460875455017603);
\draw[line width=2.pt,color=qqwuqq] (4.170001865369142,2.460875455017603) -- (4.175001852619118,2.457071176736153);
\draw[line width=2.pt,color=qqwuqq] (4.175001852619118,2.457071176736153) -- (4.180001839869095,2.4532600922926795);
\draw[line width=2.pt,color=qqwuqq] (4.180001839869095,2.4532600922926795) -- (4.185001827119072,2.449442251873366);
\draw[line width=2.pt,color=qqwuqq] (4.185001827119072,2.449442251873366) -- (4.190001814369048,2.445617705259859);
\draw[line width=2.pt,color=qqwuqq] (4.190001814369048,2.445617705259859) -- (4.195001801619025,2.4417865018320772);
\draw[line width=2.pt,color=qqwuqq] (4.195001801619025,2.4417865018320772) -- (4.2000017888690016,2.437948690571016);
\draw[line width=2.pt,color=qqwuqq] (4.2000017888690016,2.437948690571016) -- (4.205001776118978,2.434104320061553);
\draw[line width=2.pt,color=qqwuqq] (4.205001776118978,2.434104320061553) -- (4.210001763368955,2.430253438495252);
\draw[line width=2.pt,color=qqwuqq] (4.210001763368955,2.430253438495252) -- (4.2150017506189315,2.4263960936731674);
\draw[line width=2.pt,color=qqwuqq] (4.2150017506189315,2.4263960936731674) -- (4.220001737868908,2.4225323330086423);
\draw[line width=2.pt,color=qqwuqq] (4.220001737868908,2.4225323330086423) -- (4.225001725118885,2.4186622035301113);
\draw[line width=2.pt,color=qqwuqq] (4.225001725118885,2.4186622035301113) -- (4.230001712368861,2.4147857518838878);
\draw[line width=2.pt,color=qqwuqq] (4.230001712368861,2.4147857518838878) -- (4.235001699618838,2.4109030243369607);
\draw[line width=2.pt,color=qqwuqq] (4.235001699618838,2.4109030243369607) -- (4.240001686868815,2.4070140667797766);
\draw[line width=2.pt,color=qqwuqq] (4.240001686868815,2.4070140667797766) -- (4.245001674118791,2.4031189247290223);
\draw[line width=2.pt,color=qqwuqq] (4.245001674118791,2.4031189247290223) -- (4.250001661368768,2.3992176433303998);
\draw[line width=2.pt,color=qqwuqq] (4.250001661368768,2.3992176433303998) -- (4.2550016486187445,2.395310267361392);
\draw[line width=2.pt,color=qqwuqq] (4.2550016486187445,2.395310267361392) -- (4.260001635868721,2.391396841234031);
\draw[line width=2.pt,color=qqwuqq] (4.260001635868721,2.391396841234031) -- (4.265001623118698,2.387477408997648);
\draw[line width=2.pt,color=qqwuqq] (4.265001623118698,2.387477408997648) -- (4.270001610368674,2.3835520143416224);
\draw[line width=2.pt,color=qqwuqq] (4.270001610368674,2.3835520143416224) -- (4.275001597618651,2.3796207005981245);
\draw[line width=2.pt,color=qqwuqq] (4.275001597618651,2.3796207005981245) -- (4.280001584868628,2.3756835107448433);
\draw[line width=2.pt,color=qqwuqq] (4.280001584868628,2.3756835107448433) -- (4.285001572118604,2.3717404874077093);
\draw[line width=2.pt,color=qqwuqq] (4.285001572118604,2.3717404874077093) -- (4.290001559368581,2.3677916728636106);
\draw[line width=2.pt,color=qqwuqq] (4.290001559368581,2.3677916728636106) -- (4.295001546618558,2.363837109043095);
\draw[line width=2.pt,color=qqwuqq] (4.295001546618558,2.363837109043095) -- (4.300001533868534,2.3598768375330645);
\draw[line width=2.pt,color=qqwuqq] (4.300001533868534,2.3598768375330645) -- (4.305001521118511,2.3559108995794578);
\draw[line width=2.pt,color=qqwuqq] (4.305001521118511,2.3559108995794578) -- (4.3100015083684875,2.3519393360899263);
\draw[line width=2.pt,color=qqwuqq] (4.3100015083684875,2.3519393360899263) -- (4.315001495618464,2.3479621876364947);
\draw[line width=2.pt,color=qqwuqq] (4.315001495618464,2.3479621876364947) -- (4.320001482868441,2.3439794944582113);
\draw[line width=2.pt,color=qqwuqq] (4.320001482868441,2.3439794944582113) -- (4.325001470118417,2.3399912964637903);
\draw[line width=2.pt,color=qqwuqq] (4.325001470118417,2.3399912964637903) -- (4.330001457368394,2.3359976332342396);
\draw[line width=2.pt,color=qqwuqq] (4.330001457368394,2.3359976332342396) -- (4.335001444618371,2.3319985440254745);
\draw[line width=2.pt,color=qqwuqq] (4.335001444618371,2.3319985440254745) -- (4.340001431868347,2.3279940677709243);
\draw[line width=2.pt,color=qqwuqq] (4.340001431868347,2.3279940677709243) -- (4.345001419118324,2.3239842430841215);
\draw[line width=2.pt,color=qqwuqq] (4.345001419118324,2.3239842430841215) -- (4.350001406368301,2.319969108261283);
\draw[line width=2.pt,color=qqwuqq] (4.350001406368301,2.319969108261283) -- (4.355001393618277,2.3159487012838733);
\draw[line width=2.pt,color=qqwuqq] (4.355001393618277,2.3159487012838733) -- (4.360001380868254,2.3119230598211593);
\draw[line width=2.pt,color=qqwuqq] (4.360001380868254,2.3119230598211593) -- (4.3650013681182305,2.3078922212327484);
\draw[line width=2.pt,color=qqwuqq] (4.3650013681182305,2.3078922212327484) -- (4.370001355368207,2.303856222571116);
\draw[line width=2.pt,color=qqwuqq] (4.370001355368207,2.303856222571116) -- (4.375001342618184,2.2998151005841136);
\draw[line width=2.pt,color=qqwuqq] (4.375001342618184,2.2998151005841136) -- (4.38000132986816,2.295768891717473);
\draw[line width=2.pt,color=qqwuqq] (4.38000132986816,2.295768891717473) -- (4.385001317118137,2.2917176321172854);
\draw[line width=2.pt,color=qqwuqq] (4.385001317118137,2.2917176321172854) -- (4.390001304368114,2.2876613576324756);
\draw[line width=2.pt,color=qqwuqq] (4.390001304368114,2.2876613576324756) -- (4.39500129161809,2.283600103817254);
\draw[line width=2.pt,color=qqwuqq] (4.39500129161809,2.283600103817254) -- (4.400001278868067,2.2795339059335626);
\draw[line width=2.pt,color=qqwuqq] (4.400001278868067,2.2795339059335626) -- (4.4050012661180435,2.2754627989534972);
\draw[line width=2.pt,color=qqwuqq] (4.4050012661180435,2.2754627989534972) -- (4.41000125336802,2.271386817561723);
\draw[line width=2.pt,color=qqwuqq] (4.41000125336802,2.271386817561723) -- (4.415001240617997,2.2673059961578703);
\draw[line width=2.pt,color=qqwuqq] (4.415001240617997,2.2673059961578703) -- (4.4200012278679734,2.263220368858918);
\draw[line width=2.pt,color=qqwuqq] (4.4200012278679734,2.263220368858918) -- (4.42500121511795,2.2591299695015614);
\draw[line width=2.pt,color=qqwuqq] (4.42500121511795,2.2591299695015614) -- (4.430001202367927,2.2550348316445636);
\draw[line width=2.pt,color=qqwuqq] (4.430001202367927,2.2550348316445636) -- (4.435001189617903,2.2509349885710943);
\draw[line width=2.pt,color=qqwuqq] (4.435001189617903,2.2509349885710943) -- (4.44000117686788,2.2468304732910513);
\draw[line width=2.pt,color=qqwuqq] (4.44000117686788,2.2468304732910513) -- (4.445001164117857,2.2427213185433676);
\draw[line width=2.pt,color=qqwuqq] (4.445001164117857,2.2427213185433676) -- (4.450001151367833,2.2386075567983035);
\draw[line width=2.pt,color=qqwuqq] (4.450001151367833,2.2386075567983035) -- (4.45500113861781,2.2344892202597224);
\draw[line width=2.pt,color=qqwuqq] (4.45500113861781,2.2344892202597224) -- (4.4600011258677865,2.230366340867352);
\draw[line width=2.pt,color=qqwuqq] (4.4600011258677865,2.230366340867352) -- (4.465001113117763,2.2262389502990305);
\draw[line width=2.pt,color=qqwuqq] (4.465001113117763,2.2262389502990305) -- (4.47000110036774,2.2221070799729343);
\draw[line width=2.pt,color=qqwuqq] (4.47000110036774,2.2221070799729343) -- (4.475001087617716,2.217970761049795);
\draw[line width=2.pt,color=qqwuqq] (4.475001087617716,2.217970761049795) -- (4.480001074867693,2.213830024435097);
\draw[line width=2.pt,color=qqwuqq] (4.480001074867693,2.213830024435097) -- (4.48500106211767,2.20968490078126);
\draw[line width=2.pt,color=qqwuqq] (4.48500106211767,2.20968490078126) -- (4.490001049367646,2.2055354204898086);
\draw[line width=2.pt,color=qqwuqq] (4.490001049367646,2.2055354204898086) -- (4.495001036617623,2.201381613713524);
\draw[line width=2.pt,color=qqwuqq] (4.495001036617623,2.201381613713524) -- (4.5000010238676,2.197223510358578);
\draw[line width=2.pt,color=qqwuqq] (4.5000010238676,2.197223510358578) -- (4.505001011117576,2.1930611400866584);
\draw[line width=2.pt,color=qqwuqq] (4.505001011117576,2.1930611400866584) -- (4.510000998367553,2.18889453231707);
\draw[line width=2.pt,color=qqwuqq] (4.510000998367553,2.18889453231707) -- (4.5150009856175295,2.1847237162288256);
\draw[line width=2.pt,color=qqwuqq] (4.5150009856175295,2.1847237162288256) -- (4.520000972867506,2.1805487207627223);
\draw[line width=2.pt,color=qqwuqq] (4.520000972867506,2.1805487207627223) -- (4.525000960117483,2.176369574623399);
\draw[line width=2.pt,color=qqwuqq] (4.525000960117483,2.176369574623399) -- (4.530000947367459,2.1721863062813753);
\draw[line width=2.pt,color=qqwuqq] (4.530000947367459,2.1721863062813753) -- (4.535000934617436,2.1679989439750855);
\draw[line width=2.pt,color=qqwuqq] (4.535000934617436,2.1679989439750855) -- (4.540000921867413,2.163807515712887);
\draw[line width=2.pt,color=qqwuqq] (4.540000921867413,2.163807515712887) -- (4.545000909117389,2.159612049275056);
\draw[line width=2.pt,color=qqwuqq] (4.545000909117389,2.159612049275056) -- (4.550000896367366,2.155412572215771);
\draw[line width=2.pt,color=qqwuqq] (4.550000896367366,2.155412572215771) -- (4.555000883617343,2.1512091118650774);
\draw[line width=2.pt,color=qqwuqq] (4.555000883617343,2.1512091118650774) -- (4.560000870867319,2.1470016953308364);
\draw[line width=2.pt,color=qqwuqq] (4.560000870867319,2.1470016953308364) -- (4.565000858117296,2.142790349500662);
\draw[line width=2.pt,color=qqwuqq] (4.565000858117296,2.142790349500662) -- (4.5700008453672725,2.1385751010438403);
\draw[line width=2.pt,color=qqwuqq] (4.5700008453672725,2.1385751010438403) -- (4.575000832617249,2.1343559764132314);
\draw[line width=2.pt,color=qqwuqq] (4.575000832617249,2.1343559764132314) -- (4.580000819867226,2.1301330018471623);
\draw[line width=2.pt,color=qqwuqq] (4.580000819867226,2.1301330018471623) -- (4.585000807117202,2.125906203371298);
\draw[line width=2.pt,color=qqwuqq] (4.585000807117202,2.125906203371298) -- (4.590000794367179,2.1216756068005025);
\draw[line width=2.pt,color=qqwuqq] (4.590000794367179,2.1216756068005025) -- (4.595000781617156,2.1174412377406817);
\draw[line width=2.pt,color=qqwuqq] (4.595000781617156,2.1174412377406817) -- (4.600000768867132,2.113203121590611);
\draw[line width=2.pt,color=qqwuqq] (4.600000768867132,2.113203121590611) -- (4.605000756117109,2.108961283543753);
\draw[line width=2.pt,color=qqwuqq] (4.605000756117109,2.108961283543753) -- (4.6100007433670855,2.104715748590052);
\draw[line width=2.pt,color=qqwuqq] (4.6100007433670855,2.104715748590052) -- (4.615000730617062,2.100466541517722);
\draw[line width=2.pt,color=qqwuqq] (4.615000730617062,2.100466541517722) -- (4.620000717867039,2.096213686915015);
\draw[line width=2.pt,color=qqwuqq] (4.620000717867039,2.096213686915015) -- (4.625000705117015,2.091957209171975);
\draw[line width=2.pt,color=qqwuqq] (4.625000705117015,2.091957209171975) -- (4.630000692366992,2.087697132482181);
\draw[line width=2.pt,color=qqwuqq] (4.630000692366992,2.087697132482181) -- (4.635000679616969,2.0834334808444686);
\draw[line width=2.pt,color=qqwuqq] (4.635000679616969,2.0834334808444686) -- (4.640000666866945,2.079166278064646);
\draw[line width=2.pt,color=qqwuqq] (4.640000666866945,2.079166278064646) -- (4.645000654116922,2.074895547757187);
\draw[line width=2.pt,color=qqwuqq] (4.645000654116922,2.074895547757187) -- (4.650000641366899,2.0706213133469165);
\draw[line width=2.pt,color=qqwuqq] (4.650000641366899,2.0706213133469165) -- (4.655000628616875,2.066343598070678);
\draw[line width=2.pt,color=qqwuqq] (4.655000628616875,2.066343598070678) -- (4.660000615866852,2.0620624249789863);
\draw[line width=2.pt,color=qqwuqq] (4.660000615866852,2.0620624249789863) -- (4.6650006031168285,2.05777781693767);
\draw[line width=2.pt,color=qqwuqq] (4.6650006031168285,2.05777781693767) -- (4.670000590366805,2.0534897966294965);
\draw[line width=2.pt,color=qqwuqq] (4.670000590366805,2.0534897966294965) -- (4.675000577616782,2.0491983865557852);
\draw[line width=2.pt,color=qqwuqq] (4.675000577616782,2.0491983865557852) -- (4.680000564866758,2.044903609038003);
\draw[line width=2.pt,color=qqwuqq] (4.680000564866758,2.044903609038003) -- (4.685000552116735,2.040605486219352);
\draw[line width=2.pt,color=qqwuqq] (4.685000552116735,2.040605486219352) -- (4.690000539366712,2.036304040066341);
\draw[line width=2.pt,color=qqwuqq] (4.690000539366712,2.036304040066341) -- (4.695000526616688,2.0319992923703376);
\draw[line width=2.pt,color=qqwuqq] (4.695000526616688,2.0319992923703376) -- (4.700000513866665,2.0276912647491203);
\draw[line width=2.pt,color=qqwuqq] (4.700000513866665,2.0276912647491203) -- (4.705000501116642,2.0233799786484);
\draw[line width=2.pt,color=qqwuqq] (4.705000501116642,2.0233799786484) -- (4.710000488366618,2.019065455343344);
\draw[line width=2.pt,color=qqwuqq] (4.710000488366618,2.019065455343344) -- (4.715000475616595,2.0147477159400755);
\draw[line width=2.pt,color=qqwuqq] (4.715000475616595,2.0147477159400755) -- (4.7200004628665715,2.010426781377167);
\draw[line width=2.pt,color=qqwuqq] (4.7200004628665715,2.010426781377167) -- (4.725000450116548,2.0061026724271147);
\draw[line width=2.pt,color=qqwuqq] (4.725000450116548,2.0061026724271147) -- (4.730000437366525,2.001775409697807);
\draw[line width=2.pt,color=qqwuqq] (4.730000437366525,2.001775409697807) -- (4.735000424616501,1.997445013633973);
\draw[line width=2.pt,color=qqwuqq] (4.735000424616501,1.997445013633973) -- (4.740000411866478,1.9931115045186214);
\draw[line width=2.pt,color=qqwuqq] (4.740000411866478,1.9931115045186214) -- (4.745000399116455,1.9887749024744679);
\draw[line width=2.pt,color=qqwuqq] (4.745000399116455,1.9887749024744679) -- (4.750000386366431,1.9844352274653492);
\draw[line width=2.pt,color=qqwuqq] (4.750000386366431,1.9844352274653492) -- (4.755000373616408,1.980092499297621);
\draw[line width=2.pt,color=qqwuqq] (4.755000373616408,1.980092499297621) -- (4.760000360866385,1.9757467376215487);
\draw[line width=2.pt,color=qqwuqq] (4.760000360866385,1.9757467376215487) -- (4.765000348116361,1.9713979619326825);
\draw[line width=2.pt,color=qqwuqq] (4.765000348116361,1.9713979619326825) -- (4.770000335366338,1.9670461915732202);
\draw[line width=2.pt,color=qqwuqq] (4.770000335366338,1.9670461915732202) -- (4.7750003226163145,1.9626914457333609);
\draw[line width=2.pt,color=qqwuqq] (4.7750003226163145,1.9626914457333609) -- (4.780000309866291,1.9583337434526449);
\draw[line width=2.pt,color=qqwuqq] (4.780000309866291,1.9583337434526449) -- (4.785000297116268,1.9539731036212729);
\draw[line width=2.pt,color=qqwuqq] (4.785000297116268,1.9539731036212729) -- (4.790000284366244,1.9496095449814357);
\draw[line width=2.pt,color=qqwuqq] (4.790000284366244,1.9496095449814357) -- (4.795000271616221,1.9452430861286025);
\draw[line width=2.pt,color=qqwuqq] (4.795000271616221,1.9452430861286025) -- (4.800000258866198,1.9408737455128255);
\draw[line width=2.pt,color=qqwuqq] (4.800000258866198,1.9408737455128255) -- (4.805000246116174,1.9365015414400109);
\draw[line width=2.pt,color=qqwuqq] (4.805000246116174,1.9365015414400109) -- (4.810000233366151,1.9321264920731922);
\draw[line width=2.pt,color=qqwuqq] (4.810000233366151,1.9321264920731922) -- (4.8150002206161275,1.927748615433786);
\draw[line width=2.pt,color=qqwuqq] (4.8150002206161275,1.927748615433786) -- (4.820000207866104,1.9233679294028363);
\draw[line width=2.pt,color=qqwuqq] (4.820000207866104,1.9233679294028363) -- (4.825000195116081,1.9189844517222525);
\draw[line width=2.pt,color=qqwuqq] (4.825000195116081,1.9189844517222525) -- (4.830000182366057,1.9145981999960284);
\draw[line width=2.pt,color=qqwuqq] (4.830000182366057,1.9145981999960284) -- (4.835000169616034,1.9102091916914574);
\draw[line width=2.pt,color=qqwuqq] (4.835000169616034,1.9102091916914574) -- (4.840000156866011,1.9058174441403288);
\draw[line width=2.pt,color=qqwuqq] (4.840000156866011,1.9058174441403288) -- (4.845000144115987,1.9014229745401252);
\draw[line width=2.pt,color=qqwuqq] (4.845000144115987,1.9014229745401252) -- (4.850000131365964,1.8970257999551965);
\draw[line width=2.pt,color=qqwuqq] (4.850000131365964,1.8970257999551965) -- (4.855000118615941,1.8926259373179253);
\draw[line width=2.pt,color=qqwuqq] (4.855000118615941,1.8926259373179253) -- (4.860000105865917,1.888223403429894);
\draw[line width=2.pt,color=qqwuqq] (4.860000105865917,1.888223403429894) -- (4.865000093115894,1.8838182149630214);
\draw[line width=2.pt,color=qqwuqq] (4.865000093115894,1.8838182149630214) -- (4.8700000803658705,1.879410388460709);
\draw[line width=2.pt,color=qqwuqq] (4.8700000803658705,1.879410388460709) -- (4.875000067615847,1.8749999403389577);
\draw[line width=2.pt,color=qqwuqq] (4.875000067615847,1.8749999403389577) -- (4.880000054865824,1.8705868868874926);
\draw[line width=2.pt,color=qqwuqq] (4.880000054865824,1.8705868868874926) -- (4.8850000421158,1.86617124427086);
\draw[line width=2.pt,color=qqwuqq] (4.8850000421158,1.86617124427086) -- (4.890000029365777,1.8617530285295292);
\draw[line width=2.pt,color=qqwuqq] (4.890000029365777,1.8617530285295292) -- (4.895000016615754,1.8573322555809764);
\draw[line width=2.pt,color=qqwuqq] (4.895000016615754,1.8573322555809764) -- (4.90000000386573,1.8529089412207558);
\draw[line width=2.pt,color=qqwuqq] (4.90000000386573,1.8529089412207558) -- (4.904999991115707,1.8484831011235716);
\draw[line width=2.pt,color=qqwuqq] (4.904999991115707,1.8484831011235716) -- (4.909999978365684,1.844054750844328);
\draw[line width=2.pt,color=qqwuqq] (4.909999978365684,1.844054750844328) -- (4.91499996561566,1.8396239058191792);
\draw[line width=2.pt,color=qqwuqq] (4.91499996561566,1.8396239058191792) -- (4.919999952865637,1.835190581366561);
\draw[line width=2.pt,color=qqwuqq] (4.919999952865637,1.835190581366561) -- (4.9249999401156135,1.8307547926882233);
\draw[line width=2.pt,color=qqwuqq] (4.9249999401156135,1.8307547926882233) -- (4.92999992736559,1.8263165548702398);
\draw[line width=2.pt,color=qqwuqq] (4.92999992736559,1.8263165548702398) -- (4.934999914615567,1.821875882884024);
\draw[line width=2.pt,color=qqwuqq] (4.934999914615567,1.821875882884024) -- (4.939999901865543,1.8174327915873203);
\draw[line width=2.pt,color=qqwuqq] (4.939999901865543,1.8174327915873203) -- (4.94499988911552,1.812987295725197);
\draw[line width=2.pt,color=qqwuqq] (4.94499988911552,1.812987295725197) -- (4.949999876365497,1.8085394099310266);
\draw[line width=2.pt,color=qqwuqq] (4.949999876365497,1.8085394099310266) -- (4.954999863615473,1.8040891487274537);
\draw[line width=2.pt,color=qqwuqq] (4.954999863615473,1.8040891487274537) -- (4.95999985086545,1.7996365265273582);
\draw[line width=2.pt,color=qqwuqq] (4.95999985086545,1.7996365265273582) -- (4.9649998381154266,1.7951815576348125);
\draw[line width=2.pt,color=qqwuqq] (4.9649998381154266,1.7951815576348125) -- (4.969999825365403,1.7907242562460155);
\draw[line width=2.pt,color=qqwuqq] (4.969999825365403,1.7907242562460155) -- (4.97499981261538,1.786264636450241);
\draw[line width=2.pt,color=qqwuqq] (4.97499981261538,1.786264636450241) -- (4.9799997998653565,1.7818027122307516);
\draw[line width=2.pt,color=qqwuqq] (4.9799997998653565,1.7818027122307516) -- (4.984999787115333,1.7773384974657258);
\draw[line width=2.pt,color=qqwuqq] (4.984999787115333,1.7773384974657258) -- (4.98999977436531,1.7728720059291643);
\draw[line width=2.pt,color=qqwuqq] (4.98999977436531,1.7728720059291643) -- (4.994999761615286,1.768403251291791);
\begin{scriptsize}
\draw [color=uuuuuu] (0.,0.)-- ++(-2.0pt,0 pt) -- ++(4.0pt,0 pt) ++(-2.0pt,-2.0pt) -- ++(0 pt,4.0pt);
\draw[color=uuuuuu] (-0.1926045016077218,0.17760252365932675) node {$O$};
\end{scriptsize}
\end{axis}
\end{tikzpicture} 

\end{minipage}
\begin{minipage}{0.33\linewidth}
\begin{center}
\textbf{ Courbe 3}
 \end{center} 

\begin{tikzpicture}[line cap=round,line join=round,>=triangle 45,x=0.7cm,y=0.7cm]
\begin{axis}[
x=0.7cm,y=0.7cm,
axis lines=middle,
ymajorgrids=true,
xmajorgrids=true,
xmin=-0.779421221865035,
xmax=6.3633440514469655,
ymin=-0.6593059936908272,
ymax=5.8706624605678215,
xtick={-1.0,0.0,...,6.0},
ytick={-0.0,1.0,...,5.0},]
\clip(-0.77421221865035,-0.7793059936908272) rectangle (6.3633440514469655,5.8706624605678215);
\draw[line width=2.pt] (3.0000029581993837,3.000000000004375) -- (3.0000029581993837,3.000000000004375);
\draw[line width=2.pt] (3.0000029581993837,3.000000000004375) -- (3.005002943110369,3.0000125146415746);
\draw[line width=2.pt] (3.005002943110369,3.0000125146415746) -- (3.0100029280213545,3.0000500280330984);
\draw[line width=2.pt] (3.0100029280213545,3.0000500280330984) -- (3.01500291293234,3.000112537365898);
\draw[line width=2.pt] (3.01500291293234,3.000112537365898) -- (3.0200028978433253,3.0002000379534737);
\draw[line width=2.pt] (3.0200028978433253,3.0002000379534737) -- (3.0250028827543107,3.0003125232376258);
\draw[line width=2.pt] (3.0250028827543107,3.0003125232376258) -- (3.030002867665296,3.0004499847909143);
\draw[line width=2.pt] (3.030002867665296,3.0004499847909143) -- (3.0350028525762816,3.0006124123198137);
\draw[line width=2.pt] (3.0350028525762816,3.0006124123198137) -- (3.040002837487267,3.00079979366856);
\draw[line width=2.pt] (3.040002837487267,3.00079979366856) -- (3.0450028223982524,3.001012114823696);
\draw[line width=2.pt] (3.0450028223982524,3.001012114823696) -- (3.050002807309238,3.0012493599192984);
\draw[line width=2.pt] (3.050002807309238,3.0012493599192984) -- (3.055002792220223,3.001511511242892);
\draw[line width=2.pt] (3.055002792220223,3.001511511242892) -- (3.0600027771312086,3.0017985492420407);
\draw[line width=2.pt] (3.0600027771312086,3.0017985492420407) -- (3.065002762042194,3.002110452531613);
\draw[line width=2.pt] (3.065002762042194,3.002110452531613) -- (3.0700027469531794,3.00244719790171);
\draw[line width=2.pt] (3.0700027469531794,3.00244719790171) -- (3.075002731864165,3.0028087603262588);
\draw[line width=2.pt] (3.075002731864165,3.0028087603262588) -- (3.0800027167751503,3.0031951129722496);
\draw[line width=2.pt] (3.0800027167751503,3.0031951129722496) -- (3.0850027016861357,3.0036062272096276);
\draw[line width=2.pt] (3.0850027016861357,3.0036062272096276) -- (3.090002686597121,3.00404207262181);
\draw[line width=2.pt] (3.090002686597121,3.00404207262181) -- (3.0950026715081065,3.0045026170168385);
\draw[line width=2.pt] (3.0950026715081065,3.0045026170168385) -- (3.100002656419092,3.0049878264391436);
\draw[line width=2.pt] (3.100002656419092,3.0049878264391436) -- (3.1050026413300773,3.0054976651819203);
\draw[line width=2.pt] (3.1050026413300773,3.0054976651819203) -- (3.1100026262410627,3.006032095800095);
\draw[line width=2.pt] (3.1100026262410627,3.006032095800095) -- (3.115002611152048,3.006591079123886);
\draw[line width=2.pt] (3.115002611152048,3.006591079123886) -- (3.1200025960630335,3.007174574272935);
\draw[line width=2.pt] (3.1200025960630335,3.007174574272935) -- (3.125002580974019,3.007782538671);
\draw[line width=2.pt] (3.125002580974019,3.007782538671) -- (3.1300025658850044,3.008414928061205);
\draw[line width=2.pt] (3.1300025658850044,3.008414928061205) -- (3.1350025507959898,3.00907169652182);
\draw[line width=2.pt] (3.1350025507959898,3.00907169652182) -- (3.140002535706975,3.009752796482576);
\draw[line width=2.pt] (3.140002535706975,3.009752796482576) -- (3.1450025206179606,3.0104581787414864);
\draw[line width=2.pt] (3.1450025206179606,3.0104581787414864) -- (3.150002505528946,3.011187792482169);
\draw[line width=2.pt] (3.150002505528946,3.011187792482169) -- (3.1550024904399314,3.011941585291652);
\draw[line width=2.pt] (3.1550024904399314,3.011941585291652) -- (3.160002475350917,3.0127195031786544);
\draw[line width=2.pt] (3.160002475350917,3.0127195031786544) -- (3.1650024602619022,3.0135214905923213);
\draw[line width=2.pt] (3.1650024602619022,3.0135214905923213) -- (3.1700024451728877,3.0143474904413976);
\draw[line width=2.pt] (3.1700024451728877,3.0143474904413976) -- (3.175002430083873,3.015197444113834);
\draw[line width=2.pt] (3.175002430083873,3.015197444113834) -- (3.1800024149948585,3.016071291496803);
\draw[line width=2.pt] (3.1800024149948585,3.016071291496803) -- (3.185002399905844,3.0169689709971106);
\draw[line width=2.pt] (3.185002399905844,3.0169689709971106) -- (3.1900023848168293,3.017890419561989);
\draw[line width=2.pt] (3.1900023848168293,3.017890419561989) -- (3.1950023697278147,3.018835572700258);
\draw[line width=2.pt] (3.1950023697278147,3.018835572700258) -- (3.2000023546388,3.0198043645038317);
\draw[line width=2.pt] (3.2000023546388,3.0198043645038317) -- (3.2050023395497855,3.0207967276695618);
\draw[line width=2.pt] (3.2050023395497855,3.0207967276695618) -- (3.210002324460771,3.021812593521399);
\draw[line width=2.pt] (3.210002324460771,3.021812593521399) -- (3.2150023093717564,3.0228518920328535);
\draw[line width=2.pt] (3.2150023093717564,3.0228518920328535) -- (3.2200022942827418,3.0239145518497477);
\draw[line width=2.pt] (3.2200022942827418,3.0239145518497477) -- (3.225002279193727,3.0250005003132303);
\draw[line width=2.pt] (3.225002279193727,3.0250005003132303) -- (3.2300022641047126,3.026109663483048);
\draw[line width=2.pt] (3.2300022641047126,3.026109663483048) -- (3.235002249015698,3.027241966161058);
\draw[line width=2.pt] (3.235002249015698,3.027241966161058) -- (3.2400022339266834,3.028397331914955);
\draw[line width=2.pt] (3.2400022339266834,3.028397331914955) -- (3.245002218837669,3.0295756831022094);
\draw[line width=2.pt] (3.245002218837669,3.0295756831022094) -- (3.2500022037486542,3.0307769408941896);
\draw[line width=2.pt] (3.2500022037486542,3.0307769408941896) -- (3.2550021886596396,3.032001025300463);
\draw[line width=2.pt] (3.2550021886596396,3.032001025300463) -- (3.260002173570625,3.033247855193249);
\draw[line width=2.pt] (3.260002173570625,3.033247855193249) -- (3.2650021584816105,3.0345173483320194);
\draw[line width=2.pt] (3.2650021584816105,3.0345173483320194) -- (3.270002143392596,3.0358094213882185);
\draw[line width=2.pt] (3.270002143392596,3.0358094213882185) -- (3.2750021283035813,3.0371239899701);
\draw[line width=2.pt] (3.2750021283035813,3.0371239899701) -- (3.2800021132145667,3.0384609686476534);
\draw[line width=2.pt] (3.2800021132145667,3.0384609686476534) -- (3.285002098125552,3.039820270977618);
\draw[line width=2.pt] (3.285002098125552,3.039820270977618) -- (3.2900020830365375,3.041201809528552);
\draw[line width=2.pt] (3.2900020830365375,3.041201809528552) -- (3.295002067947523,3.0426054959059607);
\draw[line width=2.pt] (3.295002067947523,3.0426054959059607) -- (3.3000020528585083,3.0440312407774583);
\draw[line width=2.pt] (3.3000020528585083,3.0440312407774583) -- (3.3050020377694938,3.045478953897946);
\draw[line width=2.pt] (3.3050020377694938,3.045478953897946) -- (3.310002022680479,3.046948544134805);
\draw[line width=2.pt] (3.310002022680479,3.046948544134805) -- (3.3150020075914646,3.0484399194930782);
\draw[line width=2.pt] (3.3150020075914646,3.0484399194930782) -- (3.32000199250245,3.049952987140633);
\draw[line width=2.pt] (3.32000199250245,3.049952987140633) -- (3.3250019774134354,3.0514876534332886);
\draw[line width=2.pt] (3.3250019774134354,3.0514876534332886) -- (3.330001962324421,3.053043823939901);
\draw[line width=2.pt] (3.330001962324421,3.053043823939901) -- (3.335001947235406,3.0546214034673835);
\draw[line width=2.pt] (3.335001947235406,3.0546214034673835) -- (3.3400019321463916,3.05622029608566);
\draw[line width=2.pt] (3.3400019321463916,3.05622029608566) -- (3.345001917057377,3.057840405152528);
\draw[line width=2.pt] (3.345001917057377,3.057840405152528) -- (3.3500019019683624,3.059481633338432);
\draw[line width=2.pt] (3.3500019019683624,3.059481633338432) -- (3.355001886879348,3.0611438826511215);
\draw[line width=2.pt] (3.355001886879348,3.0611438826511215) -- (3.3600018717903333,3.0628270544601994);
\draw[line width=2.pt] (3.3600018717903333,3.0628270544601994) -- (3.3650018567013187,3.0645310495215297);
\draw[line width=2.pt] (3.3650018567013187,3.0645310495215297) -- (3.370001841612304,3.066255768001513);
\draw[line width=2.pt] (3.370001841612304,3.066255768001513) -- (3.3750018265232895,3.0680011095012043);
\draw[line width=2.pt] (3.3750018265232895,3.0680011095012043) -- (3.380001811434275,3.0697669730802737);
\draw[line width=2.pt] (3.380001811434275,3.0697669730802737) -- (3.3850017963452603,3.071553257280793);
\draw[line width=2.pt] (3.3850017963452603,3.071553257280793) -- (3.3900017812562457,3.0733598601508465);
\draw[line width=2.pt] (3.3900017812562457,3.0733598601508465) -- (3.395001766167231,3.0751866792679454);
\draw[line width=2.pt] (3.395001766167231,3.0751866792679454) -- (3.4000017510782166,3.077033611762251);
\draw[line width=2.pt] (3.4000017510782166,3.077033611762251) -- (3.405001735989202,3.0789005543395866);
\draw[line width=2.pt] (3.405001735989202,3.0789005543395866) -- (3.4100017209001874,3.080787403304237);
\draw[line width=2.pt] (3.4100017209001874,3.080787403304237) -- (3.415001705811173,3.0826940545815256);
\draw[line width=2.pt] (3.415001705811173,3.0826940545815256) -- (3.420001690722158,3.0846204037401614);
\draw[line width=2.pt] (3.420001690722158,3.0846204037401614) -- (3.4250016756331436,3.0865663460143518);
\draw[line width=2.pt] (3.4250016756331436,3.0865663460143518) -- (3.430001660544129,3.0885317763256657);
\draw[line width=2.pt] (3.430001660544129,3.0885317763256657) -- (3.4350016454551144,3.090516589304655);
\draw[line width=2.pt] (3.4350016454551144,3.090516589304655) -- (3.4400016303661,3.0925206793122157);
\draw[line width=2.pt] (3.4400016303661,3.0925206793122157) -- (3.4450016152770853,3.0945439404606905);
\draw[line width=2.pt] (3.4450016152770853,3.0945439404606905) -- (3.4500016001880707,3.0965862666346977);
\draw[line width=2.pt] (3.4500016001880707,3.0965862666346977) -- (3.455001585099056,3.0986475515117);
\draw[line width=2.pt] (3.455001585099056,3.0986475515117) -- (3.4600015700100415,3.1007276885822863);
\draw[line width=2.pt] (3.4600015700100415,3.1007276885822863) -- (3.465001554921027,3.102826571170178);
\draw[line width=2.pt] (3.465001554921027,3.102826571170178) -- (3.4700015398320123,3.10494409245195);
\draw[line width=2.pt] (3.4700015398320123,3.10494409245195) -- (3.4750015247429977,3.1070801454764565);
\draw[line width=2.pt] (3.4750015247429977,3.1070801454764565) -- (3.480001509653983,3.109234623183979);
\draw[line width=2.pt] (3.480001509653983,3.109234623183979) -- (3.4850014945649685,3.111407418425059);
\draw[line width=2.pt] (3.4850014945649685,3.111407418425059) -- (3.490001479475954,3.1135984239790497);
\draw[line width=2.pt] (3.490001479475954,3.1135984239790497) -- (3.4950014643869394,3.115807532572358);
\draw[line width=2.pt] (3.4950014643869394,3.115807532572358) -- (3.500001449297925,3.1180346368963825);
\draw[line width=2.pt] (3.500001449297925,3.1180346368963825) -- (3.50500143420891,3.1202796296251467);
\draw[line width=2.pt] (3.50500143420891,3.1202796296251467) -- (3.5100014191198956,3.122542403432631);
\draw[line width=2.pt] (3.5100014191198956,3.122542403432631) -- (3.515001404030881,3.124822851009784);
\draw[line width=2.pt] (3.515001404030881,3.124822851009784) -- (3.5200013889418664,3.1271208650812343);
\draw[line width=2.pt] (3.5200013889418664,3.1271208650812343) -- (3.525001373852852,3.1294363384216846);
\draw[line width=2.pt] (3.525001373852852,3.1294363384216846) -- (3.5300013587638372,3.1317691638719944);
\draw[line width=2.pt] (3.5300013587638372,3.1317691638719944) -- (3.5350013436748227,3.1341192343549533);
\draw[line width=2.pt] (3.5350013436748227,3.1341192343549533) -- (3.540001328585808,3.1364864428907357);
\draw[line width=2.pt] (3.540001328585808,3.1364864428907357) -- (3.5450013134967935,3.138870682612047);
\draw[line width=2.pt] (3.5450013134967935,3.138870682612047) -- (3.550001298407779,3.1412718467789533);
\draw[line width=2.pt] (3.550001298407779,3.1412718467789533) -- (3.5550012833187643,3.1436898287933994);
\draw[line width=2.pt] (3.5550012833187643,3.1436898287933994) -- (3.5600012682297497,3.1461245222134147);
\draw[line width=2.pt] (3.5600012682297497,3.1461245222134147) -- (3.565001253140735,3.1485758207670056);
\draw[line width=2.pt] (3.565001253140735,3.1485758207670056) -- (3.5700012380517205,3.1510436183657395);
\draw[line width=2.pt] (3.5700012380517205,3.1510436183657395) -- (3.575001222962706,3.153527809118015);
\draw[line width=2.pt] (3.575001222962706,3.153527809118015) -- (3.5800012078736914,3.1560282873420276);
\draw[line width=2.pt] (3.5800012078736914,3.1560282873420276) -- (3.5850011927846768,3.1585449475784246);
\draw[line width=2.pt] (3.5850011927846768,3.1585449475784246) -- (3.590001177695662,3.161077684602658);
\draw[line width=2.pt] (3.590001177695662,3.161077684602658) -- (3.5950011626066476,3.163626393437027);
\draw[line width=2.pt] (3.5950011626066476,3.163626393437027) -- (3.600001147517633,3.1661909693624266);
\draw[line width=2.pt] (3.600001147517633,3.1661909693624266) -- (3.6050011324286184,3.1687713079297897);
\draw[line width=2.pt] (3.6050011324286184,3.1687713079297897) -- (3.610001117339604,3.1713673049712314);
\draw[line width=2.pt] (3.610001117339604,3.1713673049712314) -- (3.6150011022505892,3.1739788566109013);
\draw[line width=2.pt] (3.6150011022505892,3.1739788566109013) -- (3.6200010871615746,3.176605859275541);
\draw[line width=2.pt] (3.6200010871615746,3.176605859275541) -- (3.62500107207256,3.1792482097047463);
\draw[line width=2.pt] (3.62500107207256,3.1792482097047463) -- (3.6300010569835455,3.181905804960947);
\draw[line width=2.pt] (3.6300010569835455,3.181905804960947) -- (3.635001041894531,3.184578542439099);
\draw[line width=2.pt] (3.635001041894531,3.184578542439099) -- (3.6400010268055163,3.18726631987609);
\draw[line width=2.pt] (3.6400010268055163,3.18726631987609) -- (3.6450010117165017,3.1899690353598746);
\draw[line width=2.pt] (3.6450010117165017,3.1899690353598746) -- (3.650000996627487,3.1926865873383194);
\draw[line width=2.pt] (3.650000996627487,3.1926865873383194) -- (3.6550009815384725,3.1954188746277863);
\draw[line width=2.pt] (3.6550009815384725,3.1954188746277863) -- (3.660000966449458,3.1981657964214376);
\draw[line width=2.pt] (3.660000966449458,3.1981657964214376) -- (3.6650009513604433,3.20092725229728);
\draw[line width=2.pt] (3.6650009513604433,3.20092725229728) -- (3.6700009362714288,3.203703142225936);
\draw[line width=2.pt] (3.6700009362714288,3.203703142225936) -- (3.675000921182414,3.2064933665781625);
\draw[line width=2.pt] (3.675000921182414,3.2064933665781625) -- (3.6800009060933996,3.209297826132109);
\draw[line width=2.pt] (3.6800009060933996,3.209297826132109) -- (3.685000891004385,3.2121164220803218);
\draw[line width=2.pt] (3.685000891004385,3.2121164220803218) -- (3.6900008759153704,3.2149490560364984);
\draw[line width=2.pt] (3.6900008759153704,3.2149490560364984) -- (3.695000860826356,3.2177956300419934);
\draw[line width=2.pt] (3.695000860826356,3.2177956300419934) -- (3.7000008457373412,3.2206560465720853);
\draw[line width=2.pt] (3.7000008457373412,3.2206560465720853) -- (3.7050008306483266,3.223530208542);
\draw[line width=2.pt] (3.7050008306483266,3.223530208542) -- (3.710000815559312,3.2264180193127006);
\draw[line width=2.pt] (3.710000815559312,3.2264180193127006) -- (3.7150008004702975,3.2293193826964437);
\draw[line width=2.pt] (3.7150008004702975,3.2293193826964437) -- (3.720000785381283,3.232234202962109);
\draw[line width=2.pt] (3.720000785381283,3.232234202962109) -- (3.7250007702922683,3.2351623848403017);
\draw[line width=2.pt] (3.7250007702922683,3.2351623848403017) -- (3.7300007552032537,3.238103833528239);
\draw[line width=2.pt] (3.7300007552032537,3.238103833528239) -- (3.735000740114239,3.241058454694411);
\draw[line width=2.pt] (3.735000740114239,3.241058454694411) -- (3.7400007250252245,3.2440261544830387);
\draw[line width=2.pt] (3.7400007250252245,3.2440261544830387) -- (3.74500070993621,3.247006839518315);
\draw[line width=2.pt] (3.74500070993621,3.247006839518315) -- (3.7500006948471953,3.250000416908441);
\draw[line width=2.pt] (3.7500006948471953,3.250000416908441) -- (3.7550006797581807,3.2530067942494627);
\draw[line width=2.pt] (3.7550006797581807,3.2530067942494627) -- (3.760000664669166,3.256025879628909);
\draw[line width=2.pt] (3.760000664669166,3.256025879628909) -- (3.7650006495801516,3.2590575816292335);
\draw[line width=2.pt] (3.7650006495801516,3.2590575816292335) -- (3.770000634491137,3.2621018093310674);
\draw[line width=2.pt] (3.770000634491137,3.2621018093310674) -- (3.7750006194021224,3.265158472316284);
\draw[line width=2.pt] (3.7750006194021224,3.265158472316284) -- (3.780000604313108,3.268227480670882);
\draw[line width=2.pt] (3.780000604313108,3.268227480670882) -- (3.785000589224093,3.271308744987689);
\draw[line width=2.pt] (3.785000589224093,3.271308744987689) -- (3.7900005741350786,3.274402176368886);
\draw[line width=2.pt] (3.7900005741350786,3.274402176368886) -- (3.795000559046064,3.277507686428365);
\draw[line width=2.pt] (3.795000559046064,3.277507686428365) -- (3.8000005439570494,3.280625187293915);
\draw[line width=2.pt] (3.8000005439570494,3.280625187293915) -- (3.805000528868035,3.2837545916092434);
\draw[line width=2.pt] (3.805000528868035,3.2837545916092434) -- (3.8100005137790203,3.286895812535839);
\draw[line width=2.pt] (3.8100005137790203,3.286895812535839) -- (3.8150004986900057,3.2900487637546725);
\draw[line width=2.pt] (3.8150004986900057,3.2900487637546725) -- (3.820000483600991,3.293213359467748);
\draw[line width=2.pt] (3.820000483600991,3.293213359467748) -- (3.8250004685119765,3.296389514399504);
\draw[line width=2.pt] (3.8250004685119765,3.296389514399504) -- (3.830000453422962,3.29957714379806);
\draw[line width=2.pt] (3.830000453422962,3.29957714379806) -- (3.8350004383339473,3.3027761634363304);
\draw[line width=2.pt] (3.8350004383339473,3.3027761634363304) -- (3.8400004232449327,3.305986489612992);
\draw[line width=2.pt] (3.8400004232449327,3.305986489612992) -- (3.845000408155918,3.309208039153315);
\draw[line width=2.pt] (3.845000408155918,3.309208039153315) -- (3.8500003930669036,3.312440729409862);
\draw[line width=2.pt] (3.8500003930669036,3.312440729409862) -- (3.855000377977889,3.3156844782630572);
\draw[line width=2.pt] (3.855000377977889,3.3156844782630572) -- (3.8600003628888744,3.318939204121629);
\draw[line width=2.pt] (3.8600003628888744,3.318939204121629) -- (3.86500034779986,3.3222048259229275);
\draw[line width=2.pt] (3.86500034779986,3.3222048259229275) -- (3.870000332710845,3.325481263133124);
\draw[line width=2.pt] (3.870000332710845,3.325481263133124) -- (3.8750003176218306,3.3287684357472918);
\draw[line width=2.pt] (3.8750003176218306,3.3287684357472918) -- (3.880000302532816,3.3320662642893737);
\draw[line width=2.pt] (3.880000302532816,3.3320662642893737) -- (3.8850002874438014,3.3353746698120386);
\draw[line width=2.pt] (3.8850002874438014,3.3353746698120386) -- (3.890000272354787,3.338693573896429);
\draw[line width=2.pt] (3.890000272354787,3.338693573896429) -- (3.8950002572657723,3.3420228986518072);
\draw[line width=2.pt] (3.8950002572657723,3.3420228986518072) -- (3.9000002421767577,3.345362566715093);
\draw[line width=2.pt] (3.9000002421767577,3.345362566715093) -- (3.905000227087743,3.3487125012503096);
\draw[line width=2.pt] (3.905000227087743,3.3487125012503096) -- (3.9100002119987285,3.3520726259479297);
\draw[line width=2.pt] (3.9100002119987285,3.3520726259479297) -- (3.915000196909714,3.3554428650241275);
\draw[line width=2.pt] (3.915000196909714,3.3554428650241275) -- (3.9200001818206993,3.358823143219941);
\draw[line width=2.pt] (3.9200001818206993,3.358823143219941) -- (3.9250001667316847,3.362213385800347);
\draw[line width=2.pt] (3.9250001667316847,3.362213385800347) -- (3.93000015164267,3.3656135185532507);
\draw[line width=2.pt] (3.93000015164267,3.3656135185532507) -- (3.9350001365536555,3.3690234677883923);
\draw[line width=2.pt] (3.9350001365536555,3.3690234677883923) -- (3.940000121464641,3.3724431603361724);
\draw[line width=2.pt] (3.940000121464641,3.3724431603361724) -- (3.9450001063756264,3.375872523546402);
\draw[line width=2.pt] (3.9450001063756264,3.375872523546402) -- (3.950000091286612,3.3793114852869786);
\draw[line width=2.pt] (3.950000091286612,3.3793114852869786) -- (3.955000076197597,3.382759973942483);
\draw[line width=2.pt] (3.955000076197597,3.382759973942483) -- (3.9600000611085826,3.386217918412716);
\draw[line width=2.pt] (3.9600000611085826,3.386217918412716) -- (3.965000046019568,3.389685248111157);
\draw[line width=2.pt] (3.965000046019568,3.389685248111157) -- (3.9700000309305534,3.3931618929633682);
\draw[line width=2.pt] (3.9700000309305534,3.3931618929633682) -- (3.975000015841539,3.3966477834053226);
\draw[line width=2.pt] (3.975000015841539,3.3966477834053226) -- (3.9800000007525242,3.4001428503816844);
\draw[line width=2.pt] (3.9800000007525242,3.4001428503816844) -- (3.9849999856635097,3.4036470253440196);
\draw[line width=2.pt] (3.9849999856635097,3.4036470253440196) -- (3.989999970574495,3.4071602402489565);
\draw[line width=2.pt] (3.989999970574495,3.4071602402489565) -- (3.9949999554854805,3.4106824275562904);
\draw[line width=2.pt] (3.9949999554854805,3.4106824275562904) -- (3.999999940396466,3.4142135202270323);
\draw[line width=2.pt] (3.999999940396466,3.4142135202270323) -- (4.004999925307452,3.417753451721414);
\draw[line width=2.pt] (4.004999925307452,3.417753451721414) -- (4.009999910218437,3.421302155996835);
\draw[line width=2.pt] (4.009999910218437,3.421302155996835) -- (4.014999895129423,3.424859567505773);
\draw[line width=2.pt] (4.014999895129423,3.424859567505773) -- (4.019999880040408,3.428425621193644);
\draw[line width=2.pt] (4.019999880040408,3.428425621193644) -- (4.024999864951393,3.432000252496617);
\draw[line width=2.pt] (4.024999864951393,3.432000252496617) -- (4.029999849862379,3.4355833973393963);
\draw[line width=2.pt] (4.029999849862379,3.4355833973393963) -- (4.034999834773364,3.439174992132955);
\draw[line width=2.pt] (4.034999834773364,3.439174992132955) -- (4.03999981968435,3.442774973772237);
\draw[line width=2.pt] (4.03999981968435,3.442774973772237) -- (4.044999804595335,3.446383279633821);
\draw[line width=2.pt] (4.044999804595335,3.446383279633821) -- (4.0499997895063204,3.4499998475735496);
\draw[line width=2.pt] (4.0499997895063204,3.4499998475735496) -- (4.054999774417306,3.4536246159241273);
\draw[line width=2.pt] (4.054999774417306,3.4536246159241273) -- (4.059999759328291,3.457257523492686);
\draw[line width=2.pt] (4.059999759328291,3.457257523492686) -- (4.064999744239277,3.460898509558321);
\draw[line width=2.pt] (4.064999744239277,3.460898509558321) -- (4.069999729150262,3.4645475138696025);
\draw[line width=2.pt] (4.069999729150262,3.4645475138696025) -- (4.0749997140612475,3.468204476642053);
\draw[line width=2.pt] (4.0749997140612475,3.468204476642053) -- (4.079999698972233,3.4718693385556048);
\draw[line width=2.pt] (4.079999698972233,3.4718693385556048) -- (4.084999683883218,3.4755420407520363);
\draw[line width=2.pt] (4.084999683883218,3.4755420407520363) -- (4.089999668794204,3.4792225248323776);
\draw[line width=2.pt] (4.089999668794204,3.4792225248323776) -- (4.094999653705189,3.482910732854302);
\draw[line width=2.pt] (4.094999653705189,3.482910732854302) -- (4.099999638616175,3.486606607329496);
\draw[line width=2.pt] (4.099999638616175,3.486606607329496) -- (4.10499962352716,3.4903100912210068);
\draw[line width=2.pt] (4.10499962352716,3.4903100912210068) -- (4.109999608438145,3.4940211279405777);
\draw[line width=2.pt] (4.109999608438145,3.4940211279405777) -- (4.114999593349131,3.4977396613459657);
\draw[line width=2.pt] (4.114999593349131,3.4977396613459657) -- (4.119999578260116,3.5014656357382403);
\draw[line width=2.pt] (4.119999578260116,3.5014656357382403) -- (4.124999563171102,3.5051989958590757);
\draw[line width=2.pt] (4.124999563171102,3.5051989958590757) -- (4.129999548082087,3.508939686888022);
\draw[line width=2.pt] (4.129999548082087,3.508939686888022) -- (4.134999532993072,3.5126876544397696);
\draw[line width=2.pt] (4.134999532993072,3.5126876544397696) -- (4.139999517904058,3.5164428445614044);
\draw[line width=2.pt] (4.139999517904058,3.5164428445614044) -- (4.144999502815043,3.5202052037296463);
\draw[line width=2.pt] (4.144999502815043,3.5202052037296463) -- (4.149999487726029,3.5239746788480866);
\draw[line width=2.pt] (4.149999487726029,3.5239746788480866) -- (4.154999472637014,3.527751217244411);
\draw[line width=2.pt] (4.154999472637014,3.527751217244411) -- (4.1599994575479995,3.531534766667624);
\draw[line width=2.pt] (4.1599994575479995,3.531534766667624) -- (4.164999442458985,3.5353252752852553);
\draw[line width=2.pt] (4.164999442458985,3.5353252752852553) -- (4.16999942736997,3.539122691680575);
\draw[line width=2.pt] (4.16999942736997,3.539122691680575) -- (4.174999412280956,3.5429269648497916);
\draw[line width=2.pt] (4.174999412280956,3.5429269648497916) -- (4.179999397191941,3.546738044199258);
\draw[line width=2.pt] (4.179999397191941,3.546738044199258) -- (4.1849993821029265,3.550555879542662);
\draw[line width=2.pt] (4.1849993821029265,3.550555879542662) -- (4.189999367013912,3.55438042109823);
\draw[line width=2.pt] (4.189999367013912,3.55438042109823) -- (4.194999351924897,3.5582116194859172);
\draw[line width=2.pt] (4.194999351924897,3.5582116194859172) -- (4.199999336835883,3.562049425724602);
\draw[line width=2.pt] (4.199999336835883,3.562049425724602) -- (4.204999321746868,3.565893791229281);
\draw[line width=2.pt] (4.204999321746868,3.565893791229281) -- (4.209999306657854,3.569744667808267);
\draw[line width=2.pt] (4.209999306657854,3.569744667808267) -- (4.214999291568839,3.573602007660381);
\draw[line width=2.pt] (4.214999291568839,3.573602007660381) -- (4.219999276479824,3.577465763372155);
\draw[line width=2.pt] (4.219999276479824,3.577465763372155) -- (4.22499926139081,3.581335887915034);
\draw[line width=2.pt] (4.22499926139081,3.581335887915034) -- (4.229999246301795,3.5852123346425815);
\draw[line width=2.pt] (4.229999246301795,3.5852123346425815) -- (4.234999231212781,3.589095057287688);
\draw[line width=2.pt] (4.234999231212781,3.589095057287688) -- (4.239999216123766,3.5929840099597845);
\draw[line width=2.pt] (4.239999216123766,3.5929840099597845) -- (4.2449992010347515,3.596879147142065);
\draw[line width=2.pt] (4.2449992010347515,3.596879147142065) -- (4.249999185945737,3.6007804236887098);
\draw[line width=2.pt] (4.249999185945737,3.6007804236887098) -- (4.254999170856722,3.6046877948221145);
\draw[line width=2.pt] (4.254999170856722,3.6046877948221145) -- (4.259999155767708,3.608601216130131);
\draw[line width=2.pt] (4.259999155767708,3.608601216130131) -- (4.264999140678693,3.61252064356331);
\draw[line width=2.pt] (4.264999140678693,3.61252064356331) -- (4.2699991255896785,3.616446033432155);
\draw[line width=2.pt] (4.2699991255896785,3.616446033432155) -- (4.274999110500664,3.620377342404381);
\draw[line width=2.pt] (4.274999110500664,3.620377342404381) -- (4.279999095411649,3.6243145275021833);
\draw[line width=2.pt] (4.279999095411649,3.6243145275021833) -- (4.284999080322635,3.628257546099516);
\draw[line width=2.pt] (4.284999080322635,3.628257546099516) -- (4.28999906523362,3.6322063559193776);
\draw[line width=2.pt] (4.28999906523362,3.6322063559193776) -- (4.294999050144606,3.636160915031107);
\draw[line width=2.pt] (4.294999050144606,3.636160915031107) -- (4.299999035055591,3.6401211818476913);
\draw[line width=2.pt] (4.299999035055591,3.6401211818476913) -- (4.304999019966576,3.644087115123078);
\draw[line width=2.pt] (4.304999019966576,3.644087115123078) -- (4.309999004877562,3.648058673949506);
\draw[line width=2.pt] (4.309999004877562,3.648058673949506) -- (4.314998989788547,3.652035817754839);
\draw[line width=2.pt] (4.314998989788547,3.652035817754839) -- (4.319998974699533,3.6560185062999198);
\draw[line width=2.pt] (4.319998974699533,3.6560185062999198) -- (4.324998959610518,3.6600066996759244);
\draw[line width=2.pt] (4.324998959610518,3.6600066996759244) -- (4.3299989445215035,3.664000358301738);
\draw[line width=2.pt] (4.3299989445215035,3.664000358301738) -- (4.334998929432489,3.667999442921337);
\draw[line width=2.pt] (4.334998929432489,3.667999442921337) -- (4.339998914343474,3.672003914601186);
\draw[line width=2.pt] (4.339998914343474,3.672003914601186) -- (4.34499889925446,3.6760137347276447);
\draw[line width=2.pt] (4.34499889925446,3.6760137347276447) -- (4.349998884165445,3.6800288650043926);
\draw[line width=2.pt] (4.349998884165445,3.6800288650043926) -- (4.3549988690764305,3.6840492674498586);
\draw[line width=2.pt] (4.3549988690764305,3.6840492674498586) -- (4.359998853987416,3.6880749043946732);
\draw[line width=2.pt] (4.359998853987416,3.6880749043946732) -- (4.364998838898401,3.6921057384791247);
\draw[line width=2.pt] (4.364998838898401,3.6921057384791247) -- (4.369998823809387,3.6961417326506365);
\draw[line width=2.pt] (4.369998823809387,3.6961417326506365) -- (4.374998808720372,3.7001828501612533);
\draw[line width=2.pt] (4.374998808720372,3.7001828501612533) -- (4.379998793631358,3.7042290545651433);
\draw[line width=2.pt] (4.379998793631358,3.7042290545651433) -- (4.384998778542343,3.708280309716114);
\draw[line width=2.pt] (4.384998778542343,3.708280309716114) -- (4.389998763453328,3.7123365797651413);
\draw[line width=2.pt] (4.389998763453328,3.7123365797651413) -- (4.394998748364314,3.7163978291579145);
\draw[line width=2.pt] (4.394998748364314,3.7163978291579145) -- (4.399998733275299,3.7204640226323953);
\draw[line width=2.pt] (4.399998733275299,3.7204640226323953) -- (4.404998718186285,3.724535125216388);
\draw[line width=2.pt] (4.404998718186285,3.724535125216388) -- (4.40999870309727,3.7286111022251314);
\draw[line width=2.pt] (4.40999870309727,3.7286111022251314) -- (4.4149986880082555,3.7326919192588983);
\draw[line width=2.pt] (4.4149986880082555,3.7326919192588983) -- (4.419998672919241,3.7367775422006138);
\draw[line width=2.pt] (4.419998672919241,3.7367775422006138) -- (4.424998657830226,3.7408679372134888);
\draw[line width=2.pt] (4.424998657830226,3.7408679372134888) -- (4.429998642741212,3.744963070738664);
\draw[line width=2.pt] (4.429998642741212,3.744963070738664) -- (4.434998627652197,3.7490629094928773);
\draw[line width=2.pt] (4.434998627652197,3.7490629094928773) -- (4.4399986125631825,3.753167420466138);
\draw[line width=2.pt] (4.4399986125631825,3.753167420466138) -- (4.444998597474168,3.7572765709194194);
\draw[line width=2.pt] (4.444998597474168,3.7572765709194194) -- (4.449998582385153,3.76139032838237);
\draw[line width=2.pt] (4.449998582385153,3.76139032838237) -- (4.454998567296139,3.7655086606510366);
\draw[line width=2.pt] (4.454998567296139,3.7655086606510366) -- (4.459998552207124,3.7696315357855994);
\draw[line width=2.pt] (4.459998552207124,3.7696315357855994) -- (4.46499853711811,3.7737589221081316);
\draw[line width=2.pt] (4.46499853711811,3.7737589221081316) -- (4.469998522029095,3.7778907882003674);
\draw[line width=2.pt] (4.469998522029095,3.7778907882003674) -- (4.47499850694008,3.7820271029014867);
\draw[line width=2.pt] (4.47499850694008,3.7820271029014867) -- (4.479998491851066,3.786167835305918);
\draw[line width=2.pt] (4.479998491851066,3.786167835305918) -- (4.484998476762051,3.790312954761154);
\draw[line width=2.pt] (4.484998476762051,3.790312954761154) -- (4.489998461673037,3.7944624308655825);
\draw[line width=2.pt] (4.489998461673037,3.7944624308655825) -- (4.494998446584022,3.7986162334663387);
\draw[line width=2.pt] (4.494998446584022,3.7986162334663387) -- (4.4999984314950074,3.802774332657164);
\draw[line width=2.pt] (4.4999984314950074,3.802774332657164) -- (4.504998416405993,3.806936698776287);
\draw[line width=2.pt] (4.504998416405993,3.806936698776287) -- (4.509998401316978,3.811103302404319);
\draw[line width=2.pt] (4.509998401316978,3.811103302404319) -- (4.514998386227964,3.815274114362163);
\draw[line width=2.pt] (4.514998386227964,3.815274114362163) -- (4.519998371138949,3.819449105708939);
\draw[line width=2.pt] (4.519998371138949,3.819449105708939) -- (4.5249983560499345,3.8236282477399284);
\draw[line width=2.pt] (4.5249983560499345,3.8236282477399284) -- (4.52999834096092,3.827811511984528);
\draw[line width=2.pt] (4.52999834096092,3.827811511984528) -- (4.534998325871905,3.831998870204224);
\draw[line width=2.pt] (4.534998325871905,3.831998870204224) -- (4.539998310782891,3.836190294390578);
\draw[line width=2.pt] (4.539998310782891,3.836190294390578) -- (4.544998295693876,3.840385756763234);
\draw[line width=2.pt] (4.544998295693876,3.840385756763234) -- (4.5499982806048616,3.8445852297679357);
\draw[line width=2.pt] (4.5499982806048616,3.8445852297679357) -- (4.554998265515847,3.848788686074559);
\draw[line width=2.pt] (4.554998265515847,3.848788686074559) -- (4.559998250426832,3.852996098575164);
\draw[line width=2.pt] (4.559998250426832,3.852996098575164) -- (4.564998235337818,3.8572074403820604);
\draw[line width=2.pt] (4.564998235337818,3.8572074403820604) -- (4.569998220248803,3.861422684825886);
\draw[line width=2.pt] (4.569998220248803,3.861422684825886) -- (4.574998205159789,3.865641805453704);
\draw[line width=2.pt] (4.574998205159789,3.865641805453704) -- (4.579998190070774,3.869864776027112);
\draw[line width=2.pt] (4.579998190070774,3.869864776027112) -- (4.584998174981759,3.8740915705203705);
\draw[line width=2.pt] (4.584998174981759,3.8740915705203705) -- (4.589998159892745,3.8783221631185407);
\draw[line width=2.pt] (4.589998159892745,3.8783221631185407) -- (4.59499814480373,3.8825565282156447);
\draw[line width=2.pt] (4.59499814480373,3.8825565282156447) -- (4.599998129714716,3.886794640412832);
\draw[line width=2.pt] (4.599998129714716,3.886794640412832) -- (4.604998114625701,3.891036474516569);
\draw[line width=2.pt] (4.604998114625701,3.891036474516569) -- (4.6099980995366865,3.895282005536839);
\draw[line width=2.pt] (4.6099980995366865,3.895282005536839) -- (4.614998084447672,3.8995312086853557);
\draw[line width=2.pt] (4.614998084447672,3.8995312086853557) -- (4.619998069358657,3.9037840593737982);
\draw[line width=2.pt] (4.619998069358657,3.9037840593737982) -- (4.624998054269643,3.9080405332120502);
\draw[line width=2.pt] (4.624998054269643,3.9080405332120502) -- (4.629998039180628,3.9123006060064647);
\draw[line width=2.pt] (4.629998039180628,3.9123006060064647) -- (4.6349980240916135,3.916564253758136);
\draw[line width=2.pt] (4.6349980240916135,3.916564253758136) -- (4.639998009002599,3.920831452661188);
\draw[line width=2.pt] (4.639998009002599,3.920831452661188) -- (4.644997993913584,3.925102179101078);
\draw[line width=2.pt] (4.644997993913584,3.925102179101078) -- (4.64999797882457,3.929376409652913);
\draw[line width=2.pt] (4.64999797882457,3.929376409652913) -- (4.654997963735555,3.9336541210797846);
\draw[line width=2.pt] (4.654997963735555,3.9336541210797846) -- (4.659997948646541,3.93793529033111);
\draw[line width=2.pt] (4.659997948646541,3.93793529033111) -- (4.664997933557526,3.9422198945409948);
\draw[line width=2.pt] (4.664997933557526,3.9422198945409948) -- (4.669997918468511,3.9465079110266057);
\draw[line width=2.pt] (4.669997918468511,3.9465079110266057) -- (4.674997903379497,3.9507993172865605);
\draw[line width=2.pt] (4.674997903379497,3.9507993172865605) -- (4.679997888290482,3.9550940909993257);
\draw[line width=2.pt] (4.679997888290482,3.9550940909993257) -- (4.684997873201468,3.959392210021636);
\draw[line width=2.pt] (4.684997873201468,3.959392210021636) -- (4.689997858112453,3.9636936523869193);
\draw[line width=2.pt] (4.689997858112453,3.9636936523869193) -- (4.6949978430234385,3.967998396303744);
\draw[line width=2.pt] (4.6949978430234385,3.967998396303744) -- (4.699997827934424,3.972306420154272);
\draw[line width=2.pt] (4.699997827934424,3.972306420154272) -- (4.704997812845409,3.9766177024927276);
\draw[line width=2.pt] (4.704997812845409,3.9766177024927276) -- (4.709997797756395,3.980932222043884);
\draw[line width=2.pt] (4.709997797756395,3.980932222043884) -- (4.71499778266738,3.985249957701556);
\draw[line width=2.pt] (4.71499778266738,3.985249957701556) -- (4.7199977675783655,3.989570888527112);
\draw[line width=2.pt] (4.7199977675783655,3.989570888527112) -- (4.724997752489351,3.9938949937479937);
\draw[line width=2.pt] (4.724997752489351,3.9938949937479937) -- (4.729997737400336,3.998222252756255);
\draw[line width=2.pt] (4.729997737400336,3.998222252756255) -- (4.734997722311322,4.002552645107108);
\draw[line width=2.pt] (4.734997722311322,4.002552645107108) -- (4.739997707222307,4.006886150517484);
\draw[line width=2.pt] (4.739997707222307,4.006886150517484) -- (4.744997692133293,4.0112227488646095);
\draw[line width=2.pt] (4.744997692133293,4.0112227488646095) -- (4.749997677044278,4.015562420184592);
\draw[line width=2.pt] (4.749997677044278,4.015562420184592) -- (4.754997661955263,4.019905144671016);
\draw[line width=2.pt] (4.754997661955263,4.019905144671016) -- (4.759997646866249,4.024250902673563);
\draw[line width=2.pt] (4.759997646866249,4.024250902673563) -- (4.764997631777234,4.028599674696624);
\draw[line width=2.pt] (4.764997631777234,4.028599674696624) -- (4.76999761668822,4.032951441397945);
\draw[line width=2.pt] (4.76999761668822,4.032951441397945) -- (4.774997601599205,4.037306183587271);
\draw[line width=2.pt] (4.774997601599205,4.037306183587271) -- (4.7799975865101905,4.041663882225011);
\draw[line width=2.pt] (4.7799975865101905,4.041663882225011) -- (4.784997571421176,4.046024518420905);
\draw[line width=2.pt] (4.784997571421176,4.046024518420905) -- (4.789997556332161,4.050388073432712);
\draw[line width=2.pt] (4.789997556332161,4.050388073432712) -- (4.794997541243147,4.0547545286649065);
\draw[line width=2.pt] (4.794997541243147,4.0547545286649065) -- (4.799997526154132,4.0591238656673845);
\draw[line width=2.pt] (4.799997526154132,4.0591238656673845) -- (4.8049975110651175,4.063496066134189);
\draw[line width=2.pt] (4.8049975110651175,4.063496066134189) -- (4.809997495976103,4.06787111190223);
\draw[line width=2.pt] (4.809997495976103,4.06787111190223) -- (4.814997480887088,4.072248984950042);
\draw[line width=2.pt] (4.814997480887088,4.072248984950042) -- (4.819997465798074,4.076629667396527);
\draw[line width=2.pt] (4.819997465798074,4.076629667396527) -- (4.824997450709059,4.081013141499728);
\draw[line width=2.pt] (4.824997450709059,4.081013141499728) -- (4.829997435620045,4.085399389655597);
\draw[line width=2.pt] (4.829997435620045,4.085399389655597) -- (4.83499742053103,4.089788394396795);
\draw[line width=2.pt] (4.83499742053103,4.089788394396795) -- (4.839997405442015,4.0941801383914775);
\draw[line width=2.pt] (4.839997405442015,4.0941801383914775) -- (4.844997390353001,4.098574604442116);
\draw[line width=2.pt] (4.844997390353001,4.098574604442116) -- (4.849997375263986,4.102971775484312);
\draw[line width=2.pt] (4.849997375263986,4.102971775484312) -- (4.854997360174972,4.107371634585631);
\draw[line width=2.pt] (4.854997360174972,4.107371634585631) -- (4.859997345085957,4.111774164944445);
\draw[line width=2.pt] (4.859997345085957,4.111774164944445) -- (4.8649973299969425,4.116179349888785);
\draw[line width=2.pt] (4.8649973299969425,4.116179349888785) -- (4.869997314907928,4.120587172875206);
\draw[line width=2.pt] (4.869997314907928,4.120587172875206) -- (4.874997299818913,4.124997617487656);
\draw[line width=2.pt] (4.874997299818913,4.124997617487656) -- (4.879997284729899,4.129410667436367);
\draw[line width=2.pt] (4.879997284729899,4.129410667436367) -- (4.884997269640884,4.133826306556742);
\draw[line width=2.pt] (4.884997269640884,4.133826306556742) -- (4.8899972545518695,4.138244518808269);
\draw[line width=2.pt] (4.8899972545518695,4.138244518808269) -- (4.894997239462855,4.142665288273426);
\draw[line width=2.pt] (4.894997239462855,4.142665288273426) -- (4.89999722437384,4.147088599156611);
\draw[line width=2.pt] (4.89999722437384,4.147088599156611) -- (4.904997209284826,4.151514435783078);
\draw[line width=2.pt] (4.904997209284826,4.151514435783078) -- (4.909997194195811,4.155942782597876);
\draw[line width=2.pt] (4.909997194195811,4.155942782597876) -- (4.914997179106797,4.160373624164809);
\draw[line width=2.pt] (4.914997179106797,4.160373624164809) -- (4.919997164017782,4.1648069451653935);
\draw[line width=2.pt] (4.919997164017782,4.1648069451653935) -- (4.924997148928767,4.169242730397841);
\draw[line width=2.pt] (4.924997148928767,4.169242730397841) -- (4.929997133839753,4.1736809647760325);
\draw[line width=2.pt] (4.929997133839753,4.1736809647760325) -- (4.934997118750738,4.1781216333285105);
\draw[line width=2.pt] (4.934997118750738,4.1781216333285105) -- (4.939997103661724,4.18256472119749);
\draw[line width=2.pt] (4.939997103661724,4.18256472119749) -- (4.944997088572709,4.187010213637859);
\draw[line width=2.pt] (4.944997088572709,4.187010213637859) -- (4.949997073483694,4.191458096016206);
\draw[line width=2.pt] (4.949997073483694,4.191458096016206) -- (4.95499705839468,4.195908353809842);
\draw[line width=2.pt] (4.95499705839468,4.195908353809842) -- (4.959997043305665,4.200360972605847);
\draw[line width=2.pt] (4.959997043305665,4.200360972605847) -- (4.964997028216651,4.2048159381001105);
\draw[line width=2.pt] (4.964997028216651,4.2048159381001105) -- (4.969997013127636,4.209273236096388);
\draw[line width=2.pt] (4.969997013127636,4.209273236096388) -- (4.9749969980386215,4.213732852505371);
\draw[line width=2.pt] (4.9749969980386215,4.213732852505371) -- (4.979996982949607,4.218194773343754);
\draw[line width=2.pt] (4.979996982949607,4.218194773343754) -- (4.984996967860592,4.222658984733318);
\draw[line width=2.pt] (4.984996967860592,4.222658984733318) -- (4.989996952771578,4.227125472900026);
\draw[line width=2.pt] (4.989996952771578,4.227125472900026) -- (4.994996937682563,4.231594224173114);
\draw[line width=2.pt] (4.994996937682563,4.231594224173114) -- (4.9999969225935486,4.2360652249842055);
\draw[line width=2.pt] (3.3440514434734525E-6,5.162274487722804) -- (0.0,5.162274487722804);
\draw[line width=2.pt] (0.0,5.162274487722804) -- (0.007499976532902266,5.155163449086367);
\draw[line width=2.pt] (0.007499976532902266,5.155163449086367) -- (0.014999953065804532,5.148051028842631);
\draw[line width=2.pt] (0.014999953065804532,5.148051028842631) -- (0.022499929598706798,5.140940411602823);
\draw[line width=2.pt] (0.022499929598706798,5.140940411602823) -- (0.029999906131609064,5.133831609639907);
\draw[line width=2.pt] (0.029999906131609064,5.133831609639907) -- (0.03749988266451133,5.12672463533532);
\draw[line width=2.pt] (0.03749988266451133,5.12672463533532) -- (0.0449998591974136,5.119619501180121);
\draw[line width=2.pt] (0.0449998591974136,5.119619501180121) -- (0.05249983573031587,5.112516219776182);
\draw[line width=2.pt] (0.05249983573031587,5.112516219776182) -- (0.05999981226321814,5.105414803837373);
\draw[line width=2.pt] (0.05999981226321814,5.105414803837373) -- (0.06749978879612041,5.098315266190772);
\draw[line width=2.pt] (0.06749978879612041,5.098315266190772) -- (0.07499976532902268,5.091217619777888);
\draw[line width=2.pt] (0.07499976532902268,5.091217619777888) -- (0.08249974186192495,5.084121877655896);
\draw[line width=2.pt] (0.08249974186192495,5.084121877655896) -- (0.08999971839482722,5.077028052998897);
\draw[line width=2.pt] (0.08999971839482722,5.077028052998897) -- (0.09749969492772949,5.069936159099179);
\draw[line width=2.pt] (0.09749969492772949,5.069936159099179) -- (0.10499967146063176,5.0628462093685105);
\draw[line width=2.pt] (0.10499967146063176,5.0628462093685105) -- (0.11249964799353403,5.055758217339433);
\draw[line width=2.pt] (0.11249964799353403,5.055758217339433) -- (0.1199996245264363,5.048672196666587);
\draw[line width=2.pt] (0.1199996245264363,5.048672196666587) -- (0.12749960105933855,5.0415881611280415);
\draw[line width=2.pt] (0.12749960105933855,5.0415881611280415) -- (0.13499957759224082,5.034506124626648);
\draw[line width=2.pt] (0.13499957759224082,5.034506124626648) -- (0.1424995541251431,5.027426101191407);
\draw[line width=2.pt] (0.1424995541251431,5.027426101191407) -- (0.14999953065804536,5.020348104978855);
\draw[line width=2.pt] (0.14999953065804536,5.020348104978855) -- (0.15749950719094763,5.013272150274466);
\draw[line width=2.pt] (0.15749950719094763,5.013272150274466) -- (0.1649994837238499,5.006198251494076);
\draw[line width=2.pt] (0.1649994837238499,5.006198251494076) -- (0.17249946025675217,4.999126423185318);
\draw[line width=2.pt] (0.17249946025675217,4.999126423185318) -- (0.17999943678965444,4.992056680029084);
\draw[line width=2.pt] (0.17999943678965444,4.992056680029084) -- (0.1874994133225567,4.984989036841);
\draw[line width=2.pt] (0.1874994133225567,4.984989036841) -- (0.19499938985545898,4.977923508572919);
\draw[line width=2.pt] (0.19499938985545898,4.977923508572919) -- (0.20249936638836125,4.9708601103144385);
\draw[line width=2.pt] (0.20249936638836125,4.9708601103144385) -- (0.20999934292126352,4.963798857294432);
\draw[line width=2.pt] (0.20999934292126352,4.963798857294432) -- (0.21749931945416578,4.956739764882603);
\draw[line width=2.pt] (0.21749931945416578,4.956739764882603) -- (0.22499929598706805,4.94968284859106);
\draw[line width=2.pt] (0.22499929598706805,4.94968284859106) -- (0.23249927251997032,4.942628124075907);
\draw[line width=2.pt] (0.23249927251997032,4.942628124075907) -- (0.2399992490528726,4.935575607138864);
\draw[line width=2.pt] (0.2399992490528726,4.935575607138864) -- (0.24749922558577486,4.928525313728892);
\draw[line width=2.pt] (0.24749922558577486,4.928525313728892) -- (0.2549992021186771,4.921477259943863);
\draw[line width=2.pt] (0.2549992021186771,4.921477259943863) -- (0.26249917865157935,4.914431462032223);
\draw[line width=2.pt] (0.26249917865157935,4.914431462032223) -- (0.2699991551844816,4.907387936394702);
\draw[line width=2.pt] (0.2699991551844816,4.907387936394702) -- (0.27749913171738383,4.900346699586034);
\draw[line width=2.pt] (0.27749913171738383,4.900346699586034) -- (0.28499910825028607,4.893307768316697);
\draw[line width=2.pt] (0.28499910825028607,4.893307768316697) -- (0.2924990847831883,4.886271159454681);
\draw[line width=2.pt] (0.2924990847831883,4.886271159454681) -- (0.29999906131609055,4.879236890027285);
\draw[line width=2.pt] (0.29999906131609055,4.879236890027285) -- (0.3074990378489928,4.872204977222918);
\draw[line width=2.pt] (0.3074990378489928,4.872204977222918) -- (0.31499901438189504,4.865175438392944);
\draw[line width=2.pt] (0.31499901438189504,4.865175438392944) -- (0.3224989909147973,4.858148291053542);
\draw[line width=2.pt] (0.3224989909147973,4.858148291053542) -- (0.3299989674476995,4.851123552887589);
\draw[line width=2.pt] (0.3299989674476995,4.851123552887589) -- (0.33749894398060176,4.844101241746575);
\draw[line width=2.pt] (0.33749894398060176,4.844101241746575) -- (0.344998920513504,4.837081375652531);
\draw[line width=2.pt] (0.344998920513504,4.837081375652531) -- (0.35249889704640625,4.830063972799996);
\draw[line width=2.pt] (0.35249889704640625,4.830063972799996) -- (0.3599988735793085,4.823049051558);
\draw[line width=2.pt] (0.3599988735793085,4.823049051558) -- (0.36749885011221073,4.816036630472078);
\draw[line width=2.pt] (0.36749885011221073,4.816036630472078) -- (0.37499882664511297,4.809026728266311);
\draw[line width=2.pt] (0.37499882664511297,4.809026728266311) -- (0.3824988031780152,4.80201936384539);
\draw[line width=2.pt] (0.3824988031780152,4.80201936384539) -- (0.38999877971091745,4.795014556296711);
\draw[line width=2.pt] (0.38999877971091745,4.795014556296711) -- (0.3974987562438197,4.788012324892497);
\draw[line width=2.pt] (0.3974987562438197,4.788012324892497) -- (0.40499873277672194,4.781012689091947);
\draw[line width=2.pt] (0.40499873277672194,4.781012689091947) -- (0.4124987093096242,4.774015668543413);
\draw[line width=2.pt] (0.4124987093096242,4.774015668543413) -- (0.4199986858425264,4.767021283086613);
\draw[line width=2.pt] (0.4199986858425264,4.767021283086613) -- (0.42749866237542866,4.7600295527548635);
\draw[line width=2.pt] (0.42749866237542866,4.7600295527548635) -- (0.4349986389083309,4.7530404977773415);
\draw[line width=2.pt] (0.4349986389083309,4.7530404977773415) -- (0.44249861544123315,4.746054138581395);
\draw[line width=2.pt] (0.44249861544123315,4.746054138581395) -- (0.4499985919741354,4.739070495794858);
\draw[line width=2.pt] (0.4499985919741354,4.739070495794858) -- (0.45749856850703763,4.7320895902484175);
\draw[line width=2.pt] (0.45749856850703763,4.7320895902484175) -- (0.46499854503993987,4.7251114429779975);
\draw[line width=2.pt] (0.46499854503993987,4.7251114429779975) -- (0.4724985215728421,4.718136075227189);
\draw[line width=2.pt] (0.4724985215728421,4.718136075227189) -- (0.47999849810574435,4.711163508449703);
\draw[line width=2.pt] (0.47999849810574435,4.711163508449703) -- (0.4874984746386466,4.7041937643118565);
\draw[line width=2.pt] (0.4874984746386466,4.7041937643118565) -- (0.49499845117154884,4.6972268646950965);
\draw[line width=2.pt] (0.49499845117154884,4.6972268646950965) -- (0.5024984277044511,4.690262831698558);
\draw[line width=2.pt] (0.5024984277044511,4.690262831698558) -- (0.5099984042373533,4.6833016876416504);
\draw[line width=2.pt] (0.5099984042373533,4.6833016876416504) -- (0.5174983807702556,4.676343455066689);
\draw[line width=2.pt] (0.5174983807702556,4.676343455066689) -- (0.5249983573031578,4.669388156741553);
\draw[line width=2.pt] (0.5249983573031578,4.669388156741553) -- (0.53249833383606,4.662435815662383);
\draw[line width=2.pt] (0.53249833383606,4.662435815662383) -- (0.5399983103689623,4.655486455056316);
\draw[line width=2.pt] (0.5399983103689623,4.655486455056316) -- (0.5474982869018645,4.648540098384257);
\draw[line width=2.pt] (0.5474982869018645,4.648540098384257) -- (0.5549982634347668,4.641596769343687);
\draw[line width=2.pt] (0.5549982634347668,4.641596769343687) -- (0.562498239967669,4.634656491871514);
\draw[line width=2.pt] (0.562498239967669,4.634656491871514) -- (0.5699982165005713,4.627719290146953);
\draw[line width=2.pt] (0.5699982165005713,4.627719290146953) -- (0.5774981930334735,4.620785188594457);
\draw[line width=2.pt] (0.5774981930334735,4.620785188594457) -- (0.5849981695663757,4.613854211886684);
\draw[line width=2.pt] (0.5849981695663757,4.613854211886684) -- (0.592498146099278,4.606926384947495);
\draw[line width=2.pt] (0.592498146099278,4.606926384947495) -- (0.5999981226321802,4.600001732955011);
\draw[line width=2.pt] (0.5999981226321802,4.600001732955011) -- (0.6074980991650825,4.593080281344697);
\draw[line width=2.pt] (0.6074980991650825,4.593080281344697) -- (0.6149980756979847,4.586162055812496);
\draw[line width=2.pt] (0.6149980756979847,4.586162055812496) -- (0.622498052230887,4.579247082318002);
\draw[line width=2.pt] (0.622498052230887,4.579247082318002) -- (0.6299980287637892,4.57233538708768);
\draw[line width=2.pt] (0.6299980287637892,4.57233538708768) -- (0.6374980052966914,4.565426996618129);
\draw[line width=2.pt] (0.6374980052966914,4.565426996618129) -- (0.6449979818295937,4.558521937679387);
\draw[line width=2.pt] (0.6449979818295937,4.558521937679387) -- (0.6524979583624959,4.551620237318291);
\draw[line width=2.pt] (0.6524979583624959,4.551620237318291) -- (0.6599979348953982,4.544721922861868);
\draw[line width=2.pt] (0.6599979348953982,4.544721922861868) -- (0.6674979114283004,4.537827021920789);
\draw[line width=2.pt] (0.6674979114283004,4.537827021920789) -- (0.6749978879612026,4.530935562392861);
\draw[line width=2.pt] (0.6749978879612026,4.530935562392861) -- (0.6824978644941049,4.52404757246657);
\draw[line width=2.pt] (0.6824978644941049,4.52404757246657) -- (0.6899978410270071,4.517163080624671);
\draw[line width=2.pt] (0.6899978410270071,4.517163080624671) -- (0.6974978175599094,4.510282115647837);
\draw[line width=2.pt] (0.6974978175599094,4.510282115647837) -- (0.7049977940928116,4.5034047066183405);
\draw[line width=2.pt] (0.7049977940928116,4.5034047066183405) -- (0.7124977706257138,4.496530882923809);
\draw[line width=2.pt] (0.7124977706257138,4.496530882923809) -- (0.7199977471586161,4.48966067426101);
\draw[line width=2.pt] (0.7199977471586161,4.48966067426101) -- (0.7274977236915183,4.482794110639711);
\draw[line width=2.pt] (0.7274977236915183,4.482794110639711) -- (0.7349977002244206,4.475931222386572);
\draw[line width=2.pt] (0.7349977002244206,4.475931222386572) -- (0.7424976767573228,4.4690720401491095);
\draw[line width=2.pt] (0.7424976767573228,4.4690720401491095) -- (0.749997653290225,4.462216594899704);
\draw[line width=2.pt] (0.749997653290225,4.462216594899704) -- (0.7574976298231273,4.455364917939672);
\draw[line width=2.pt] (0.7574976298231273,4.455364917939672) -- (0.7649976063560295,4.448517040903386);
\draw[line width=2.pt] (0.7649976063560295,4.448517040903386) -- (0.7724975828889318,4.441672995762465);
\draw[line width=2.pt] (0.7724975828889318,4.441672995762465) -- (0.779997559421834,4.434832814830006);
\draw[line width=2.pt] (0.779997559421834,4.434832814830006) -- (0.7874975359547363,4.4279965307648945);
\draw[line width=2.pt] (0.7874975359547363,4.4279965307648945) -- (0.7949975124876385,4.421164176576157);
\draw[line width=2.pt] (0.7949975124876385,4.421164176576157) -- (0.8024974890205407,4.41433578562739);
\draw[line width=2.pt] (0.8024974890205407,4.41433578562739) -- (0.809997465553443,4.407511391641241);
\draw[line width=2.pt] (0.809997465553443,4.407511391641241) -- (0.8174974420863452,4.400691028703953);
\draw[line width=2.pt] (0.8174974420863452,4.400691028703953) -- (0.8249974186192475,4.3938747312699835);
\draw[line width=2.pt] (0.8249974186192475,4.3938747312699835) -- (0.8324973951521497,4.387062534166672);
\draw[line width=2.pt] (0.8324973951521497,4.387062534166672) -- (0.839997371685052,4.380254472598987);
\draw[line width=2.pt] (0.839997371685052,4.380254472598987) -- (0.8474973482179542,4.373450582154332);
\draw[line width=2.pt] (0.8474973482179542,4.373450582154332) -- (0.8549973247508564,4.366650898807423);
\draw[line width=2.pt] (0.8549973247508564,4.366650898807423) -- (0.8624973012837587,4.359855458925232);
\draw[line width=2.pt] (0.8624973012837587,4.359855458925232) -- (0.8699972778166609,4.353064299272001);
\draw[line width=2.pt] (0.8699972778166609,4.353064299272001) -- (0.8774972543495632,4.346277457014333);
\draw[line width=2.pt] (0.8774972543495632,4.346277457014333) -- (0.8849972308824654,4.3394949697263385);
\draw[line width=2.pt] (0.8849972308824654,4.3394949697263385) -- (0.8924972074153676,4.332716875394874);
\draw[line width=2.pt] (0.8924972074153676,4.332716875394874) -- (0.8999971839482699,4.325943212424843);
\draw[line width=2.pt] (0.8999971839482699,4.325943212424843) -- (0.9074971604811721,4.319174019644571);
\draw[line width=2.pt] (0.9074971604811721,4.319174019644571) -- (0.9149971370140744,4.312409336311266);
\draw[line width=2.pt] (0.9149971370140744,4.312409336311266) -- (0.9224971135469766,4.305649202116541);
\draw[line width=2.pt] (0.9224971135469766,4.305649202116541) -- (0.9299970900798789,4.2988936571920355);
\draw[line width=2.pt] (0.9299970900798789,4.2988936571920355) -- (0.9374970666127811,4.2921427421150895);
\draw[line width=2.pt] (0.9374970666127811,4.2921427421150895) -- (0.9449970431456833,4.285396497914527);
\draw[line width=2.pt] (0.9449970431456833,4.285396497914527) -- (0.9524970196785856,4.278654966076496);
\draw[line width=2.pt] (0.9524970196785856,4.278654966076496) -- (0.9599969962114878,4.271918188550405);
\draw[line width=2.pt] (0.9599969962114878,4.271918188550405) -- (0.9674969727443901,4.265186207754942);
\draw[line width=2.pt] (0.9674969727443901,4.265186207754942) -- (0.9749969492772923,4.258459066584177);
\draw[line width=2.pt] (0.9749969492772923,4.258459066584177) -- (0.9824969258101945,4.251736808413744);
\draw[line width=2.pt] (0.9824969258101945,4.251736808413744) -- (0.9899969023430968,4.245019477107125);
\draw[line width=2.pt] (0.9899969023430968,4.245019477107125) -- (0.997496878875999,4.238307117022006);
\draw[line width=2.pt] (0.997496878875999,4.238307117022006) -- (1.0049968554089013,4.231599773016741);
\draw[line width=2.pt] (1.0049968554089013,4.231599773016741) -- (1.0124968319418035,4.22489749045689);
\draw[line width=2.pt] (1.0124968319418035,4.22489749045689) -- (1.0199968084747058,4.218200315221859);
\draw[line width=2.pt] (1.0199968084747058,4.218200315221859) -- (1.027496785007608,4.2115082937116295);
\draw[line width=2.pt] (1.027496785007608,4.2115082937116295) -- (1.0349967615405102,4.204821472853592);
\draw[line width=2.pt] (1.0349967615405102,4.204821472853592) -- (1.0424967380734125,4.198139900109461);
\draw[line width=2.pt] (1.0424967380734125,4.198139900109461) -- (1.0499967146063147,4.191463623482299);
\draw[line width=2.pt] (1.0499967146063147,4.191463623482299) -- (1.057496691139217,4.1847926915236355);
\draw[line width=2.pt] (1.057496691139217,4.1847926915236355) -- (1.0649966676721192,4.178127153340687);
\draw[line width=2.pt] (1.0649966676721192,4.178127153340687) -- (1.0724966442050214,4.171467058603676);
\draw[line width=2.pt] (1.0724966442050214,4.171467058603676) -- (1.0799966207379237,4.164812457553262);
\draw[line width=2.pt] (1.0799966207379237,4.164812457553262) -- (1.087496597270826,4.158163401008059);
\draw[line width=2.pt] (1.087496597270826,4.158163401008059) -- (1.0949965738037282,4.151519940372279);
\draw[line width=2.pt] (1.0949965738037282,4.151519940372279) -- (1.1024965503366304,4.144882127643472);
\draw[line width=2.pt] (1.1024965503366304,4.144882127643472) -- (1.1099965268695327,4.138250015420374);
\draw[line width=2.pt] (1.1099965268695327,4.138250015420374) -- (1.117496503402435,4.131623656910867);
\draw[line width=2.pt] (1.117496503402435,4.131623656910867) -- (1.1249964799353371,4.1250031059400545);
\draw[line width=2.pt] (1.1249964799353371,4.1250031059400545) -- (1.1324964564682394,4.118388416958439);
\draw[line width=2.pt] (1.1324964564682394,4.118388416958439) -- (1.1399964330011416,4.11177964505023);
\draw[line width=2.pt] (1.1399964330011416,4.11177964505023) -- (1.1474964095340439,4.105176845941751);
\draw[line width=2.pt] (1.1474964095340439,4.105176845941751) -- (1.154996386066946,4.098580076009974);
\draw[line width=2.pt] (1.154996386066946,4.098580076009974) -- (1.1624963625998483,4.091989392291172);
\draw[line width=2.pt] (1.1624963625998483,4.091989392291172) -- (1.1699963391327506,4.085404852489687);
\draw[line width=2.pt] (1.1699963391327506,4.085404852489687) -- (1.1774963156656528,4.078826514986826);
\draw[line width=2.pt] (1.1774963156656528,4.078826514986826) -- (1.184996292198555,4.072254438849871);
\draw[line width=2.pt] (1.184996292198555,4.072254438849871) -- (1.1924962687314573,4.065688683841228);
\draw[line width=2.pt] (1.1924962687314573,4.065688683841228) -- (1.1999962452643596,4.0591293104276875);
\draw[line width=2.pt] (1.1999962452643596,4.0591293104276875) -- (1.2074962217972618,4.052576379789822);
\draw[line width=2.pt] (1.2074962217972618,4.052576379789822) -- (1.214996198330164,4.046029953831509);
\draw[line width=2.pt] (1.214996198330164,4.046029953831509) -- (1.2224961748630663,4.039490095189587);
\draw[line width=2.pt] (1.2224961748630663,4.039490095189587) -- (1.2299961513959685,4.032956867243643);
\draw[line width=2.pt] (1.2299961513959685,4.032956867243643) -- (1.2374961279288708,4.026430334125929);
\draw[line width=2.pt] (1.2374961279288708,4.026430334125929) -- (1.244996104461773,4.019910560731428);
\draw[line width=2.pt] (1.244996104461773,4.019910560731428) -- (1.2524960809946752,4.01339761272804);
\draw[line width=2.pt] (1.2524960809946752,4.01339761272804) -- (1.2599960575275775,4.006891556566915);
\draw[line width=2.pt] (1.2599960575275775,4.006891556566915) -- (1.2674960340604797,4.000392459492929);
\draw[line width=2.pt] (1.2674960340604797,4.000392459492929) -- (1.274996010593382,3.993900389555293);
\draw[line width=2.pt] (1.274996010593382,3.993900389555293) -- (1.2824959871262842,3.9874154156183144);
\draw[line width=2.pt] (1.2824959871262842,3.9874154156183144) -- (1.2899959636591865,3.9809376073722955);
\draw[line width=2.pt] (1.2899959636591865,3.9809376073722955) -- (1.2974959401920887,3.9744670353445812);
\draw[line width=2.pt] (1.2974959401920887,3.9744670353445812) -- (1.304995916724991,3.9680037709107556);
\draw[line width=2.pt] (1.304995916724991,3.9680037709107556) -- (1.3124958932578932,3.9615478863059845);
\draw[line width=2.pt] (1.3124958932578932,3.9615478863059845) -- (1.3199958697907954,3.955099454636512);
\draw[line width=2.pt] (1.3199958697907954,3.955099454636512) -- (1.3274958463236977,3.948658549891305);
\draw[line width=2.pt] (1.3274958463236977,3.948658549891305) -- (1.3349958228566,3.942225246953857);
\draw[line width=2.pt] (1.3349958228566,3.942225246953857) -- (1.3424957993895021,3.9357996216141395);
\draw[line width=2.pt] (1.3424957993895021,3.9357996216141395) -- (1.3499957759224044,3.929381750580716);
\draw[line width=2.pt] (1.3499957759224044,3.929381750580716) -- (1.3574957524553066,3.9229717114930107);
\draw[line width=2.pt] (1.3574957524553066,3.9229717114930107) -- (1.3649957289882089,3.916569582933737);
\draw[line width=2.pt] (1.3649957289882089,3.916569582933737) -- (1.372495705521111,3.910175444441485);
\draw[line width=2.pt] (1.372495705521111,3.910175444441485) -- (1.3799956820540134,3.9037893765234752);
\draw[line width=2.pt] (1.3799956820540134,3.9037893765234752) -- (1.3874956585869156,3.897411460668467);
\draw[line width=2.pt] (1.3874956585869156,3.897411460668467) -- (1.3949956351198178,3.891041779359842);
\draw[line width=2.pt] (1.3949956351198178,3.891041779359842) -- (1.40249561165272,3.8846804160888437);
\draw[line width=2.pt] (1.40249561165272,3.8846804160888437) -- (1.4099955881856223,3.878327455367989);
\draw[line width=2.pt] (1.4099955881856223,3.878327455367989) -- (1.4174955647185246,3.8719829827446457);
\draw[line width=2.pt] (1.4174955647185246,3.8719829827446457) -- (1.4249955412514268,3.8656470848147797);
\draw[line width=2.pt] (1.4249955412514268,3.8656470848147797) -- (1.432495517784329,3.8593198492368703);
\draw[line width=2.pt] (1.432495517784329,3.8593198492368703) -- (1.4399954943172313,3.8530013647460004);
\draw[line width=2.pt] (1.4399954943172313,3.8530013647460004) -- (1.4474954708501335,3.8466917211681135);
\draw[line width=2.pt] (1.4474954708501335,3.8466917211681135) -- (1.4549954473830358,3.8403910094344478);
\draw[line width=2.pt] (1.4549954473830358,3.8403910094344478) -- (1.462495423915938,3.8340993215961428);
\draw[line width=2.pt] (1.462495423915938,3.8340993215961428) -- (1.4699954004488403,3.827816750839018);
\draw[line width=2.pt] (1.4699954004488403,3.827816750839018) -- (1.4774953769817425,3.8215433914985297);
\draw[line width=2.pt] (1.4774953769817425,3.8215433914985297) -- (1.4849953535146447,3.8152793390749027);
\draw[line width=2.pt] (1.4849953535146447,3.8152793390749027) -- (1.492495330047547,3.809024690248437);
\draw[line width=2.pt] (1.492495330047547,3.809024690248437) -- (1.4999953065804492,3.802779542894993);
\draw[line width=2.pt] (1.4999953065804492,3.802779542894993) -- (1.5074952831133515,3.7965439961016525);
\draw[line width=2.pt] (1.5074952831133515,3.7965439961016525) -- (1.5149952596462537,3.790318150182558);
\draw[line width=2.pt] (1.5149952596462537,3.790318150182558) -- (1.522495236179156,3.78410210669493);
\draw[line width=2.pt] (1.522495236179156,3.78410210669493) -- (1.5299952127120582,3.77789596845526);
\draw[line width=2.pt] (1.5299952127120582,3.77789596845526) -- (1.5374951892449604,3.7716998395556836);
\draw[line width=2.pt] (1.5374951892449604,3.7716998395556836) -- (1.5449951657778627,3.765513825380529);
\draw[line width=2.pt] (1.5449951657778627,3.765513825380529) -- (1.552495142310765,3.759338032623047);
\draw[line width=2.pt] (1.552495142310765,3.759338032623047) -- (1.5599951188436672,3.7531725693023104);
\draw[line width=2.pt] (1.5599951188436672,3.7531725693023104) -- (1.5674950953765694,3.7470175447802987);
\draw[line width=2.pt] (1.5674950953765694,3.7470175447802987) -- (1.5749950719094716,3.740873069779153);
\draw[line width=2.pt] (1.5749950719094716,3.740873069779153) -- (1.5824950484423739,3.7347392563986057);
\draw[line width=2.pt] (1.5824950484423739,3.7347392563986057) -- (1.5899950249752761,3.728616218133589);
\draw[line width=2.pt] (1.5899950249752761,3.728616218133589) -- (1.5974950015081784,3.722504069892012);
\draw[line width=2.pt] (1.5974950015081784,3.722504069892012) -- (1.6049949780410806,3.71640292801271);
\draw[line width=2.pt] (1.6049949780410806,3.71640292801271) -- (1.6124949545739828,3.7103129102835695);
\draw[line width=2.pt] (1.6124949545739828,3.7103129102835695) -- (1.619994931106885,3.704234135959813);
\draw[line width=2.pt] (1.619994931106885,3.704234135959813) -- (1.6274949076397873,3.6981667257824586);
\draw[line width=2.pt] (1.6274949076397873,3.6981667257824586) -- (1.6349948841726896,3.6921108019969404);
\draw[line width=2.pt] (1.6349948841726896,3.6921108019969404) -- (1.6424948607055918,3.6860664883718943);
\draw[line width=2.pt] (1.6424948607055918,3.6860664883718943) -- (1.649994837238494,3.6800339102181003);
\draw[line width=2.pt] (1.649994837238494,3.6800339102181003) -- (1.6574948137713963,3.6740131944075882);
\draw[line width=2.pt] (1.6574948137713963,3.6740131944075882) -- (1.6649947903042985,3.6680044693928924);
\draw[line width=2.pt] (1.6649947903042985,3.6680044693928924) -- (1.6724947668372008,3.662007865226461);
\draw[line width=2.pt] (1.6724947668372008,3.662007865226461) -- (1.679994743370103,3.656023513580215);
\draw[line width=2.pt] (1.679994743370103,3.656023513580215) -- (1.6874947199030053,3.6500515477652478);
\draw[line width=2.pt] (1.6874947199030053,3.6500515477652478) -- (1.6949946964359075,3.6440921027516704);
\draw[line width=2.pt] (1.6949946964359075,3.6440921027516704) -- (1.7024946729688097,3.6381453151885874);
\draw[line width=2.pt] (1.7024946729688097,3.6381453151885874) -- (1.709994649501712,3.63221132342421);
\draw[line width=2.pt] (1.709994649501712,3.63221132342421) -- (1.7174946260346142,3.6262902675260937);
\draw[line width=2.pt] (1.7174946260346142,3.6262902675260937) -- (1.7249946025675165,3.6203822893014985);
\draw[line width=2.pt] (1.7249946025675165,3.6203822893014985) -- (1.7324945791004187,3.6144875323178636);
\draw[line width=2.pt] (1.7324945791004187,3.6144875323178636) -- (1.739994555633321,3.6086061419233957);
\draw[line width=2.pt] (1.739994555633321,3.6086061419233957) -- (1.7474945321662232,3.6027382652677598);
\draw[line width=2.pt] (1.7474945321662232,3.6027382652677598) -- (1.7549945086991254,3.596884051322867);
\draw[line width=2.pt] (1.7549945086991254,3.596884051322867) -- (1.7624944852320277,3.591043650903753);
\draw[line width=2.pt] (1.7624944852320277,3.591043650903753) -- (1.76999446176493,3.5852172166895437);
\draw[line width=2.pt] (1.76999446176493,3.5852172166895437) -- (1.7774944382978322,3.579404903244489);
\draw[line width=2.pt] (1.7774944382978322,3.579404903244489) -- (1.7849944148307344,3.5736068670390675);
\draw[line width=2.pt] (1.7849944148307344,3.5736068670390675) -- (1.7924943913636366,3.5678232664711524);
\draw[line width=2.pt] (1.7924943913636366,3.5678232664711524) -- (1.7999943678965389,3.56205426188722);
\draw[line width=2.pt] (1.7999943678965389,3.56205426188722) -- (1.8074943444294411,3.556300015603601);
\draw[line width=2.pt] (1.8074943444294411,3.556300015603601) -- (1.8149943209623434,3.5505606919277612);
\draw[line width=2.pt] (1.8149943209623434,3.5505606919277612) -- (1.8224942974952456,3.544836457179599);
\draw[line width=2.pt] (1.8224942974952456,3.544836457179599) -- (1.8299942740281478,3.5391274797127497);
\draw[line width=2.pt] (1.8299942740281478,3.5391274797127497) -- (1.83749425056105,3.5334339299358857);
\draw[line width=2.pt] (1.83749425056105,3.5334339299358857) -- (1.8449942270939523,3.5277559803339984);
\draw[line width=2.pt] (1.8449942270939523,3.5277559803339984) -- (1.8524942036268546,3.522093805489651);
\draw[line width=2.pt] (1.8524942036268546,3.522093805489651) -- (1.8599941801597568,3.516447582104184);
\draw[line width=2.pt] (1.8599941801597568,3.516447582104184) -- (1.867494156692659,3.510817489018866);
\draw[line width=2.pt] (1.867494156692659,3.510817489018866) -- (1.8749941332255613,3.505203707235969);
\draw[line width=2.pt] (1.8749941332255613,3.505203707235969) -- (1.8824941097584635,3.499606419939755);
\draw[line width=2.pt] (1.8824941097584635,3.499606419939755) -- (1.8899940862913658,3.494025812517354);
\draw[line width=2.pt] (1.8899940862913658,3.494025812517354) -- (1.897494062824268,3.4884620725795266);
\draw[line width=2.pt] (1.897494062824268,3.4884620725795266) -- (1.9049940393571703,3.482915389981278);
\draw[line width=2.pt] (1.9049940393571703,3.482915389981278) -- (1.9124940158900725,3.4773859568423218);
\draw[line width=2.pt] (1.9124940158900725,3.4773859568423218) -- (1.9199939924229747,3.4718739675673547);
\draw[line width=2.pt] (1.9199939924229747,3.4718739675673547) -- (1.927493968955877,3.466379618866144);
\draw[line width=2.pt] (1.927493968955877,3.466379618866144) -- (1.9349939454887792,3.4609031097733887);
\draw[line width=2.pt] (1.9349939454887792,3.4609031097733887) -- (1.9424939220216815,3.455444641668341);
\draw[line width=2.pt] (1.9424939220216815,3.455444641668341) -- (1.9499938985545837,3.450004418294166);
\draw[line width=2.pt] (1.9499938985545837,3.450004418294166) -- (1.957493875087486,3.44458264577701);
\draw[line width=2.pt] (1.957493875087486,3.44458264577701) -- (1.9649938516203882,3.439179532644763);
\draw[line width=2.pt] (1.9649938516203882,3.439179532644763) -- (1.9724938281532904,3.433795289845478);
\draw[line width=2.pt] (1.9724938281532904,3.433795289845478) -- (1.9799938046861927,3.428430130765432);
\draw[line width=2.pt] (1.9799938046861927,3.428430130765432) -- (1.987493781219095,3.4230842712467897);
\draw[line width=2.pt] (1.987493781219095,3.4230842712467897) -- (1.9949937577519972,3.417757929604857);
\draw[line width=2.pt] (1.9949937577519972,3.417757929604857) -- (2.0024937342848994,3.4124513266448813);
\draw[line width=2.pt] (2.0024937342848994,3.4124513266448813) -- (2.0099937108178016,3.4071646856783704);
\draw[line width=2.pt] (2.0099937108178016,3.4071646856783704) -- (2.017493687350704,3.401898232538909);
\draw[line width=2.pt] (2.017493687350704,3.401898232538909) -- (2.024993663883606,3.3966521955974276);
\draw[line width=2.pt] (2.024993663883606,3.3966521955974276) -- (2.0324936404165084,3.3914268057768977);
\draw[line width=2.pt] (2.0324936404165084,3.3914268057768977) -- (2.0399936169494106,3.3862222965664186);
\draw[line width=2.pt] (2.0399936169494106,3.3862222965664186) -- (2.047493593482313,3.3810389040346536);
\draw[line width=2.pt] (2.047493593482313,3.3810389040346536) -- (2.054993570015215,3.3758768668425922);
\draw[line width=2.pt] (2.054993570015215,3.3758768668425922) -- (2.0624935465481173,3.370736426255583);
\draw[line width=2.pt] (2.0624935465481173,3.370736426255583) -- (2.0699935230810196,3.3656178261546144);
\draw[line width=2.pt] (2.0699935230810196,3.3656178261546144) -- (2.077493499613922,3.360521313046793);
\draw[line width=2.pt] (2.077493499613922,3.360521313046793) -- (2.084993476146824,3.3554471360749827);
\draw[line width=2.pt] (2.084993476146824,3.3554471360749827) -- (2.0924934526797263,3.3503955470265607);
\draw[line width=2.pt] (2.0924934526797263,3.3503955470265607) -- (2.0999934292126285,3.345366800341247);
\draw[line width=2.pt] (2.0999934292126285,3.345366800341247) -- (2.107493405745531,3.3403611531179616);
\draw[line width=2.pt] (2.107493405745531,3.3403611531179616) -- (2.114993382278433,3.3353788651206697);
\draw[line width=2.pt] (2.114993382278433,3.3353788651206697) -- (2.1224933588113353,3.330420198783156);
\draw[line width=2.pt] (2.1224933588113353,3.330420198783156) -- (2.1299933353442375,3.3254854192126917);
\draw[line width=2.pt] (2.1299933353442375,3.3254854192126917) -- (2.1374933118771398,3.320574794192538);
\draw[line width=2.pt] (2.1374933118771398,3.320574794192538) -- (2.144993288410042,3.315688594183241);
\draw[line width=2.pt] (2.144993288410042,3.315688594183241) -- (2.1524932649429442,3.310827092322657);
\draw[line width=2.pt] (2.1524932649429442,3.310827092322657) -- (2.1599932414758465,3.305990564424665);
\draw[line width=2.pt] (2.1599932414758465,3.305990564424665) -- (2.1674932180087487,3.301179288976515);
\draw[line width=2.pt] (2.1674932180087487,3.301179288976515) -- (2.174993194541651,3.2963935471347385);
\draw[line width=2.pt] (2.174993194541651,3.2963935471347385) -- (2.182493171074553,3.2916336227195933);
\draw[line width=2.pt] (2.182493171074553,3.2916336227195933) -- (2.1899931476074554,3.2868998022079565);
\draw[line width=2.pt] (2.1899931476074554,3.2868998022079565) -- (2.1974931241403577,3.2821923747246364);
\draw[line width=2.pt] (2.1974931241403577,3.2821923747246364) -- (2.20499310067326,3.277511632032021);
\draw[line width=2.pt] (2.20499310067326,3.277511632032021) -- (2.212493077206162,3.272857868518013);
\draw[line width=2.pt] (2.212493077206162,3.272857868518013) -- (2.2199930537390644,3.2682313811821997);
\draw[line width=2.pt] (2.2199930537390644,3.2682313811821997) -- (2.2274930302719667,3.2636324696201777);
\draw[line width=2.pt] (2.2274930302719667,3.2636324696201777) -- (2.234993006804869,3.259061436005986);
\draw[line width=2.pt] (2.234993006804869,3.259061436005986) -- (2.242492983337771,3.2545185850725806);
\draw[line width=2.pt] (2.242492983337771,3.2545185850725806) -- (2.2499929598706734,3.250004224090284);
\draw[line width=2.pt] (2.2499929598706734,3.250004224090284) -- (2.2574929364035756,3.2455186628431485);
\draw[line width=2.pt] (2.2574929364035756,3.2455186628431485) -- (2.264992912936478,3.2410622136031715);
\draw[line width=2.pt] (2.264992912936478,3.2410622136031715) -- (2.27249288946938,3.2366351911022955);
\draw[line width=2.pt] (2.27249288946938,3.2366351911022955) -- (2.2799928660022823,3.2322379125021303);
\draw[line width=2.pt] (2.2799928660022823,3.2322379125021303) -- (2.2874928425351846,3.227870697361327);
\draw[line width=2.pt] (2.2874928425351846,3.227870697361327) -- (2.294992819068087,3.223533867600551);
\draw[line width=2.pt] (2.294992819068087,3.223533867600551) -- (2.302492795600989,3.219227747464978);
\draw[line width=2.pt] (2.302492795600989,3.219227747464978) -- (2.3099927721338913,3.2149526634842496);
\draw[line width=2.pt] (2.3099927721338913,3.2149526634842496) -- (2.3174927486667936,3.2107089444298365);
\draw[line width=2.pt] (2.3174927486667936,3.2107089444298365) -- (2.324992725199696,3.2064969212697285);
\draw[line width=2.pt] (2.324992725199696,3.2064969212697285) -- (2.332492701732598,3.2023169271204024);
\draw[line width=2.pt] (2.332492701732598,3.2023169271204024) -- (2.3399926782655003,3.1981692971959963);
\draw[line width=2.pt] (2.3399926782655003,3.1981692971959963) -- (2.3474926547984025,3.194054368754638);
\draw[line width=2.pt] (2.3474926547984025,3.194054368754638) -- (2.3549926313313048,3.1899724810418575);
\draw[line width=2.pt] (2.3549926313313048,3.1899724810418575) -- (2.362492607864207,3.1859239752310344);
\draw[line width=2.pt] (2.362492607864207,3.1859239752310344) -- (2.3699925843971092,3.1819091943608164);
\draw[line width=2.pt] (2.3699925843971092,3.1819091943608164) -- (2.3774925609300115,3.1779284832694534);
\draw[line width=2.pt] (2.3774925609300115,3.1779284832694534) -- (2.3849925374629137,3.173982188526004);
\draw[line width=2.pt] (2.3849925374629137,3.173982188526004) -- (2.392492513995816,3.170070658358342);
\draw[line width=2.pt] (2.392492513995816,3.170070658358342) -- (2.399992490528718,3.166194242577938);
\draw[line width=2.pt] (2.399992490528718,3.166194242577938) -- (2.4074924670616205,3.162353292501348);
\draw[line width=2.pt] (2.4074924670616205,3.162353292501348) -- (2.4149924435945227,3.158548160868381);
\draw[line width=2.pt] (2.4149924435945227,3.158548160868381) -- (2.422492420127425,3.1547792017568894);
\draw[line width=2.pt] (2.422492420127425,3.1547792017568894) -- (2.429992396660327,3.151046770494161);
\draw[line width=2.pt] (2.429992396660327,3.151046770494161) -- (2.4374923731932294,3.1473512235648613);
\draw[line width=2.pt] (2.4374923731932294,3.1473512235648613) -- (2.4449923497261317,3.143692918515508);
\draw[line width=2.pt] (2.4449923497261317,3.143692918515508) -- (2.452492326259034,3.14007221385544);
\draw[line width=2.pt] (2.452492326259034,3.14007221385544) -- (2.459992302791936,3.136489468954269);
\draw[line width=2.pt] (2.459992302791936,3.136489468954269) -- (2.4674922793248384,3.132945043935784);
\draw[line width=2.pt] (2.4674922793248384,3.132945043935784) -- (2.4749922558577406,3.1294392995683054);
\draw[line width=2.pt] (2.4749922558577406,3.1294392995683054) -- (2.482492232390643,3.1259725971514674);
\draw[line width=2.pt] (2.482492232390643,3.1259725971514674) -- (2.489992208923545,3.1225452983994386);
\draw[line width=2.pt] (2.489992208923545,3.1225452983994386) -- (2.4974921854564474,3.1191577653205727);
\draw[line width=2.pt] (2.4974921854564474,3.1191577653205727) -- (2.5049921619893496,3.1158103600934965);
\draw[line width=2.pt] (2.5049921619893496,3.1158103600934965) -- (2.512492138522252,3.1125034449396582);
\draw[line width=2.pt] (2.512492138522252,3.1125034449396582) -- (2.519992115055154,3.109237381992342);
\draw[line width=2.pt] (2.519992115055154,3.109237381992342) -- (2.5274920915880563,3.1060125331621835);
\draw[line width=2.pt] (2.5274920915880563,3.1060125331621835) -- (2.5349920681209586,3.1028292599992184);
\draw[line width=2.pt] (2.5349920681209586,3.1028292599992184) -- (2.542492044653861,3.099687923551498);
\draw[line width=2.pt] (2.542492044653861,3.099687923551498) -- (2.549992021186763,3.0965888842203237);
\draw[line width=2.pt] (2.549992021186763,3.0965888842203237) -- (2.5574919977196653,3.0935325016121524);
\draw[line width=2.pt] (2.5574919977196653,3.0935325016121524) -- (2.5649919742525675,3.0905191343872325);
\draw[line width=2.pt] (2.5649919742525675,3.0905191343872325) -- (2.5724919507854698,3.0875491401050406);
\draw[line width=2.pt] (2.5724919507854698,3.0875491401050406) -- (2.579991927318372,3.084622875066599);
\draw[line width=2.pt] (2.579991927318372,3.084622875066599) -- (2.5874919038512743,3.0817406941537544);
\draw[line width=2.pt] (2.5874919038512743,3.0817406941537544) -- (2.5949918803841765,3.0789029506655106);
\draw[line width=2.pt] (2.5949918803841765,3.0789029506655106) -- (2.6024918569170787,3.0761099961515237);
\draw[line width=2.pt] (2.6024918569170787,3.0761099961515237) -- (2.609991833449981,3.0733621802428606);
\draw[line width=2.pt] (2.609991833449981,3.0733621802428606) -- (2.617491809982883,3.070659850480147);
\draw[line width=2.pt] (2.617491809982883,3.070659850480147) -- (2.6249917865157855,3.0680033521392254);
\draw[line width=2.pt] (2.6249917865157855,3.0680033521392254) -- (2.6324917630486877,3.065393028054465);
\draw[line width=2.pt] (2.6324917630486877,3.065393028054465) -- (2.63999173958159,3.062829218439863);
\draw[line width=2.pt] (2.63999173958159,3.062829218439863) -- (2.647491716114492,3.060312260708092);
\draw[line width=2.pt] (2.647491716114492,3.060312260708092) -- (2.6549916926473944,3.0578424892876583);
\draw[line width=2.pt] (2.6549916926473944,3.0578424892876583) -- (2.6624916691802967,3.055420235438331);
\draw[line width=2.pt] (2.6624916691802967,3.055420235438331) -- (2.669991645713199,3.0530458270650347);
\draw[line width=2.pt] (2.669991645713199,3.0530458270650347) -- (2.677491622246101,3.05071958853038);
\draw[line width=2.pt] (2.677491622246101,3.05071958853038) -- (2.6849915987790034,3.048441840466036);
\draw[line width=2.pt] (2.6849915987790034,3.048441840466036) -- (2.6924915753119056,3.046212899583136);
\draw[line width=2.pt] (2.6924915753119056,3.046212899583136) -- (2.699991551844808,3.0440330784819447);
\draw[line width=2.pt] (2.699991551844808,3.0440330784819447) -- (2.70749152837771,3.041902685460983);
\draw[line width=2.pt] (2.70749152837771,3.041902685460983) -- (2.7149915049106124,3.0398220243258542);
\draw[line width=2.pt] (2.7149915049106124,3.0398220243258542) -- (2.7224914814435146,3.0377913941979937);
\draw[line width=2.pt] (2.7224914814435146,3.0377913941979937) -- (2.729991457976417,3.03581108932358);
\draw[line width=2.pt] (2.729991457976417,3.03581108932358) -- (2.737491434509319,3.033881398882858);
\draw[line width=2.pt] (2.737491434509319,3.033881398882858) -- (2.7449914110422213,3.032002606800117);
\draw[line width=2.pt] (2.7449914110422213,3.032002606800117) -- (2.7524913875751236,3.030174991554584);
\draw[line width=2.pt] (2.7524913875751236,3.030174991554584) -- (2.759991364108026,3.0283988259924874);
\draw[line width=2.pt] (2.759991364108026,3.0283988259924874) -- (2.767491340640928,3.0266743771405578);
\draw[line width=2.pt] (2.767491340640928,3.0266743771405578) -- (2.7749913171738303,3.0250019060212363);
\draw[line width=2.pt] (2.7749913171738303,3.0250019060212363) -- (2.7824912937067325,3.02338166746985);
\draw[line width=2.pt] (2.7824912937067325,3.02338166746985) -- (2.789991270239635,3.0218139099540395);
\draw[line width=2.pt] (2.789991270239635,3.0218139099540395) -- (2.797491246772537,3.0202988753957056);
\draw[line width=2.pt] (2.797491246772537,3.0202988753957056) -- (2.8049912233054393,3.0188367989957516);
\draw[line width=2.pt] (2.8049912233054393,3.0188367989957516) -- (2.8124911998383415,3.0174279090618974);
\draw[line width=2.pt] (2.8124911998383415,3.0174279090618974) -- (2.8199911763712437,3.0160724268398433);
\draw[line width=2.pt] (2.8199911763712437,3.0160724268398433) -- (2.827491152904146,3.0147705663480493);
\draw[line width=2.pt] (2.827491152904146,3.0147705663480493) -- (2.8349911294370482,3.0135225342164134);
\draw[line width=2.pt] (2.8349911294370482,3.0135225342164134) -- (2.8424911059699505,3.0123285295291096);
\draw[line width=2.pt] (2.8424911059699505,3.0123285295291096) -- (2.8499910825028527,3.0111887436718554);
\draw[line width=2.pt] (2.8499910825028527,3.0111887436718554) -- (2.857491059035755,3.010103360183873);
\draw[line width=2.pt] (2.857491059035755,3.010103360183873) -- (2.864991035568657,3.0090725546147925);
\draw[line width=2.pt] (2.864991035568657,3.0090725546147925) -- (2.8724910121015594,3.008096494386765);
\draw[line width=2.pt] (2.8724910121015594,3.008096494386765) -- (2.8799909886344617,3.0071753386620097);
\draw[line width=2.pt] (2.8799909886344617,3.0071753386620097) -- (2.887490965167364,3.006309238216052);
\draw[line width=2.pt] (2.887490965167364,3.006309238216052) -- (2.894990941700266,3.00549833531687);
\draw[line width=2.pt] (2.894990941700266,3.00549833531687) -- (2.9024909182331684,3.0047427636101744);
\draw[line width=2.pt] (2.9024909182331684,3.0047427636101744) -- (2.9099908947660706,3.004042648011036);
\draw[line width=2.pt] (2.9099908947660706,3.004042648011036) -- (2.917490871298973,3.0033981046020584);
\draw[line width=2.pt] (2.917490871298973,3.0033981046020584) -- (2.924990847831875,3.00280924053829);
\draw[line width=2.pt] (2.924990847831875,3.00280924053829) -- (2.9324908243647774,3.002276153959051);
\draw[line width=2.pt] (2.9324908243647774,3.002276153959051) -- (2.9399908008976796,3.0017989339068505);
\draw[line width=2.pt] (2.9399908008976796,3.0017989339068505) -- (2.947490777430582,3.001377660253535);
\draw[line width=2.pt] (2.947490777430582,3.001377660253535) -- (2.954990753963484,3.001012403633829);
\draw[line width=2.pt] (2.954990753963484,3.001012403633829) -- (2.9624907304963863,3.000703225386375);
\draw[line width=2.pt] (2.9624907304963863,3.000703225386375) -- (2.9699907070292886,3.000450177502409);
\draw[line width=2.pt] (2.9699907070292886,3.000450177502409) -- (2.977490683562191,3.0002533025821494);
\draw[line width=2.pt] (2.977490683562191,3.0002533025821494) -- (2.984990660095093,3.000112633799004);
\draw[line width=2.pt] (2.984990660095093,3.000112633799004) -- (2.9924906366279953,3.0000281948716507);
\draw[line width=2.pt] (2.9924906366279953,3.0000281948716507) -- (2.9999906131608975,3.0000000000440563);
\begin{scriptsize}
\draw [color=uuuuuu] (0.,0.)-- ++(-2.0pt,0 pt) -- ++(4.0pt,0 pt) ++(-2.0pt,-2.0pt) -- ++(0 pt,4.0pt);
\draw[color=uuuuuu] (-0.1926045016077218,0.17760252365932675) node {$O$};
\end{scriptsize}
\end{axis}
\end{tikzpicture}

\end{minipage}

\item Décrire l'évolution de la distance OG selon la distance parcourue par le gardien G sur $[AB]$.
%\item Soit $x$ la distance $AG$. Exprimer OG en fonction de $x$ lorsque $G$ décrit $[AB]$.
%\item Le gardien G continue sa ronde. Le talkie-walkie ne fonctionne pas lorsque la distance est supérieure à 300 mètres. Sur quelle portion du périmètre de la propriété les 2 gardiens ne pourront plus communiquer ? Utiliser un argument géométrique.
%\item \textbf{Pour aller beaucoup plus loin :} Soit $x$ la distance $AG$. Déterminer la distance $OG$ quelque soit la position du gardien sur sa ronde.
\item Arrivé en B, le gardien $G$ continue sa ronde jusqu'au point $C$. Compléter à main levée la courbe qui représente la distance $OG$ en fonction de la distance parcourue par le gardien $G$.
\item Compléter à main levée la courbe sur l'intervalle $[0;18]$.

\begin{tikzpicture}[line cap=round,line join=round,>=triangle 45,x=0.8cm,y=0.8cm]
\begin{axis}[
x=0.8cm,y=0.8cm,
axis lines=middle,
ymajorgrids=true,
xmajorgrids=true,
xmin=-0.7683400000000034,
xmax=20.403433333333343,
ymin=-0.48380000000000695,
ymax=7.2005066666666675,
xtick={-0.0,1.0,...,20.0},
ytick={-0.0,1.0,...,7.0},]
\clip(-0.76834,-0.4838) rectangle (20.403433333333343,7.2005066666666675);
\end{axis}
\end{tikzpicture}


\end{enumerate}