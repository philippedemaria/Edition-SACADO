\begin{seance}[Étude qualitative de fonctions]

\Titre{Comparaison d'images}{1}
\end{seance}


\begin{CpsCol}
\textbf{Variations de fonctions}
\begin{description}
\item[$\square$] Comparer deux images sur un intervalle donné
\item[$\square$] Déterminer tous les nombres dont l'image est supérieure (inférieure) à une image donnée
\end{description}
\end{CpsCol}


\EPC{1}{VF-26}{Raisonner. Communiquer.}

\EPC{1}{VF-12}{Raisonner. Représenter.}


\EPC{1}{VF-4}{Raisonner. Communiquer.}


\begin{seance}[Étude quantitative de fonctions]

\Titre{Applications}{1}
\end{seance}



\EPC{1}{VF-13}{Raisonner. Communiquer.}

\EPC{1}{VF-27}{Représenter. Raisonner. }

\EPC{1}{VF-7}{Raisonner. Communiquer.}


\begin{titreDTL}[Étude quantitative de fonctions]

\Titre{S'auto évaluer}{1}
\end{titreDTL}


\EPC{1}{VF-28}{Raisonner. Communiquer. }

\EPC{1}{VF-29}{Raisonner. Représenter. Communiquer. }


\begin{titreDTL}[Étude quantitative de fonctions]

\Titre{S'auto évaluer. Corrigé}{1}
\end{titreDTL}


\EPC{1}{VF-28_cor}{Raisonner. Communiquer. }

\EPC{1}{VF-29_cor}{Raisonner. Représenter. Communiquer. }