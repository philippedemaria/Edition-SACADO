
Dans un repère \Oij, on donne les points :	$A(-2;4)$ , $B(3;-4)$ , $C(4;6)$.

Déterminer les coordonnées du point $D$ tel que $ABCD$ soit un parallélogramme des deux façons suivantes :
\begin{enumerate}
\item utiliser $\overrightarrow{AB}=\overrightarrow{DC}$

$\overrightarrow{AB} (x_B-x_A;y_B-y_A)$

$\overrightarrow{AB} (5;-8)$

Soit $D$ le point du plan de coordonnées $(x;y)$.

$\overrightarrow{DC} (x_C-x_D;y_C-y_D)$

$\overrightarrow{DC} (4-x_D;6-y_D)$


$\overrightarrow{AB}=\overrightarrow{DC} \Longleftrightarrow $ \begin{tabular}{c}
$4-x_D = 5$ \\ 
$6-y_D = -8$ \\ 
\end{tabular} 

$\Longleftrightarrow $ \begin{tabular}{c}
$x_D = -1$ \\ 
$y_D = 14$ \\ 
\end{tabular} 

Donc $D$ a pour coordonnées $(-1;14)$

\item Utiliser l'égalité $\overrightarrow{AC}=\overrightarrow{AB}+\overrightarrow{AD}$

$\overrightarrow{AC}(x_C-x_A;y_C-y_A)$, $\overrightarrow{AB}(x_B-x_A;y_B-y_A)$, $\overrightarrow{AD}(x_D-x_A;y_D-y_A)$

$\overrightarrow{AC}(4-(-2);6-4)$, $\overrightarrow{AB}(3-(-2);-4-4)$, $\overrightarrow{AD}(x_D-(-2);y_D-4)$

$\overrightarrow{AC}(6;2)$, $\overrightarrow{AB}(5;-8)$, $\overrightarrow{AD}(x_D-(-2);y_D-4)$


\begin{tabular}{c}
$6=5+x_D+2 $ \\ 
$2=-8+y_D-4  $ \\ 
\end{tabular} $\Longleftrightarrow $
\begin{tabular}{c}
$x_D = -1$ \\ 
$y_D = 14$ \\ 
\end{tabular}$\Longleftrightarrow $




\end{enumerate}