\documentclass[20pt]{article}

\input{../../../latex_preambule_style/preambule}
\input{../../../latex_preambule_style/styleCourslycee}
\input{../../../latex_preambule_style/styleExercices}
%\input{../../latex_preambule_style/styleCahier}
\input{../../../latex_preambule_style/bas_de_page_seconde}
\input{../../../latex_preambule_style/algobox}



%%%%%%%%%%%%%%%  Affichage ou impression  %%%%%%%%%%%%%%%%%%
\newcommand{\impress}[2]{
\ifthenelse{\equal{#1}{1}}  %   1 imprime / affiche  -----    0 n'affiche pas
{%condition vraie
#2
}% fin condition vraie
{%condition fausse
}% fin condition fausse
} % fin de la procédure
%%%%%%%%%%%%%%%  Affichage ou impression  %%%%%%%%%%%%%%%%%%



%%%%%%%%%%%%%%%  Indentation  %%%%%%%%%%%%%%%%%%
\parindent=0pt
%%%%%%%%%%%%%%%%%%%%%%%%%%%%%%%%%%%%%%%%%%%%%%%%



\begin{document}

 \begin{titre}[Sciences Numériques et Technologie]

\Titre{Raspberry Pi3, Apache2, Django}{4}
\end{titre}


\begin{CpsCol}
\textbf{Utiliser des nombres pour calculer et résoudre des problèmes}
\begin{description}
\item[$\square$] Manipuler le calcul littéral et les racines carrées
\item[$\square$] Développer des expressions polynomiales simples
\item[$\square$] Factoriser des expressions polynomiales simples
\end{description}
\end{CpsCol}


\begin{CpsCol}
\textbf{Matériel et liens utiles}
\begin{description}
\item[$\square$] Carte Raspberry Pi3 Modèle B+ 1GB 40€
\item[$\square$] https://raspbian-france.fr/installer-serveur-web-raspberry-lamp/
\item[$\square$] https://raspbian-france.fr/mettre-en-ligne-serveur-web-raspbian-dydns-port-forwarding/
\end{description}
\end{CpsCol}



\subsection{Installer un serveur apache}

sudo apt update
sudo apt upgrade
sudo apt update

sudo apt install apache2

sudo chown -R pi:www-data /var/www/html/
sudo chmod -R 770 /var/www/html/


\subsubsection{Vérifier qu'Apache fonctionne}

wget -O verif_apache.html http://127.0.0.1


\begin{Rq}
Apache utilise le répertoire /var/www/html comme répertoire par défaut.
\end{Rq}

\subsection{Installer php}

sudo apt install php php-mbstring

\subsection{Installer phpMyadmin - base MySQL}

sudo apt install mysql-server php-mysql

sudo mysql --user=root


DROP USER 'root'@'localhost';
CREATE USER 'root'@'localhost' IDENTIFIED BY 'password';
GRANT ALL PRIVILEGES ON *.* TO 'root'@'localhost';

sudo apt install phpmyadmin

 
\end{document}
