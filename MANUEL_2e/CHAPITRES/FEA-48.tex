
Une balle est jetée en l’air et sa trajectoire est décrite par le tableau suivant

\begin{center}
\begin{tabular}{|c|c|c|c|c|c|c|c|c|c|}
\hline 
Temps (s) & 0 & 0,5 & 1 & 1,5 & 2 & 2,5 & 3  \\ 
\hline 
Hauteur (m) & 2 & 2,25 & 3 & 4,25 & 6 & 8,25 & 11  \\ 
\hline 
\end{tabular} 
\end{center}


\begin{enumerate}
\item Décrire la trajectoire de la balle en utilisant les mots hauteur et seconde.
\item Quelle quantité est donnée en fonction de l'autre ? comment se nomme la variable ?
\item A quels intervalles appartiennent le temps et la hauteur ?
\item Expliquer pourquoi $t \geq 0$.
\item Ces 2 quantités sont-elles proportionnelles ? Expliquer.
\item Traduire le lien entre ces 2 quantités par une formule. \textbf{On dit que l'on modélise la situation par une fonction.}
\item Écrire un algorithme qui permet de calculer les images des valeurs du tableau par cette fonction.
\end{enumerate}
 