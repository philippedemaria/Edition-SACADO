\begin{titre}[Nombres et calculs]

\Titre{Ensembles de nombres}{3}
\end{titre}


\begin{CpsCol}
\begin{description}
\item[$\square$] Associer à chaque point de la droite graduée un unique nombre réel et
réciproquement.
\end{description}
\end{CpsCol}




\begin{DefT}{Ensemble des réels}
L'ensemble de tous les nombres connus en Seconde est appelé ensemble des nombres réels, noté $\R$.
\end{DefT}

\begin{DefT}{Ensemble de nombres}\index{Ensemble de nombres!Réels $\R$}
Les autres ensembles de nombres, inclus dans $\R$.
\begin{enumerate}
\item On appelle \textbf{entiers naturels} les nombres : 0 ; 1 ; 2 ; 3 . . . Leur ensemble est noté $\N$.\index{Ensemble de nombres! Entiers naturels $\N$}
On a donc : $\N = \lbrace 0 ; 1 ; 2 ; 3 \cdots \rbrace $
\item  On appelle \textbf{entiers relatifs} les nombres entiers naturels et leurs symétriques par rapport à 0. Leur ensemble est noté $\Z$.\index{Ensemble de nombres! Entiers $\Z$}
On a donc : $\Z = \lbrace \cdots -3 ; -2 ; -1 ; 0 ; 1 ; 2 ; 3  \cdots \rbrace $
\item  On appelle \textbf{nombres rationnels} les nombres de la forme $\frac{a}{b}$, $a$ et $b$ entiers et $b$ non nuls.  Leur ensemble est noté $\Q$. \index{Ensemble de nombres! Rationnels $\Q$}
On a donc : $\Q = \lbrace \cdots \frac{5}{3} ; -\frac{5}{7} ; -\frac{13}{22} \cdots \rbrace$
\item Par construction, $\N$ est inclus dans $\Z$  est inclus dans $\D$  est inclus dans $\Q$  est inclus dans $\R$. Ces ensembles sont dits "emboités".
\item  On peut représenter l'ensemble des réels sur une droite graduée.
\begin{center}
\begin{tikzpicture}[line cap=round,line join=round,>=triangle 45,x=1.0cm,y=1.0cm]
\draw[->,color=black] (-4.36,0.) -- (10.66,0.);
\foreach \x in {-4.,-3.,-2.,-1.,1.,2.,3.,4.,5.,6.,7.,8.,9.,10.}
\draw[shift={(\x,0)},color=black] (0pt,2pt) -- (0pt,-2pt) node[below] {\footnotesize $\x$};
\draw[color=black] (0pt,-10pt) node[right] {\footnotesize $0$};
\clip(-4.36,-0.5) rectangle (10.66,0.5);
\end{tikzpicture}
 \end{center} 
\end{enumerate}
\end{DefT}


\EPC{1}{FEA-2}{Représenter.}

\begin{DefT}{Appartenance}\index{Ensemble!Appartenance}
Pour symboliser l'appartenance d'un nombre à un ensemble, on utilise le symbole $\in$. On écrit que :
\begin{description}
\item[•] $-2 \in \Z$, $\frac{5}{3} \in \Q$
\item[•] $-5 \not \in \N$, $\pi \not\in \Q$
\end{description}
\end{DefT}

\begin{Nt}
 $ \not\in$ signifie n'appartient pas.
\end{Nt}



\begin{DefT}{Nombres décimaux}
Les \textbf{nombres décimaux} sont des nombres rationnels dont le dénominateur est une puissance de 2, de 5 ou de 10 ou un produit de puissances de ces nombres.
\end{DefT}

\begin{Ex}
\begin{description}
\item[•] $A=\frac{3}{25}$ est un nombre décimal car $25 = 5^2$, 25 est donc une puissance de 5.
\item[•] $B=\frac{13}{20}$ est un nombre décimal car $20 = 2^2 \times 5$, 20 est donc un produit d'une puissance de 2 et de 5.
\end{description}
On écrit $A=\frac{3}{25} \in \D$, $B=\frac{13}{20} \in \D$
\end{Ex}



\begin{Att}
$A=\frac{7}{15}$ n'est pas un nombre décimal. $15=3 \times 5$ n'est pas une puissance de 5 mais un multiple de 5 !

$A=\frac{7}{15} \not \in \D$
\end{Att}

  
\begin{Log}
Un nombre $x$ est décimal s'il existe un nombre $a \in \Z$ et $n \in \N$ tel que $x = \frac{a}{10^n}$
\end{Log}


 
\begin{minipage}{0.47\linewidth}

\EPC{1}{FEA-88}{Représenter. Chercher.}
\end{minipage}
\hfill
\begin{minipage}{0.47\linewidth}

\EPC{0}{FEA-88bis}{Représenter. Chercher}
\end{minipage}

 
 
 
 
 
\EPC{1}{FEA-89}{Communiquer. Chercher.}
 

\begin{DefT}{Inclusion}\index{Ensemble!Inclusion}
Un ensemble $A$ est inclus dans un ensemble $B$ lorsque tous les éléments de $A$ sont contenus dans $B$. On note $A \subset B$.
\end{DefT}


\begin{Rq}
\begin{description}
\item[•] Un ensemble  $A$ \textbf{est inclus dans} un ensemble $B$  se note : $A \subset B$.
\item[•] Un élément $x$ \textbf{appartient à} un ensemble $B$ se note  $x \in B$.
\end{description}
\end{Rq}

\EPC{1}{FEA-15}{Représenter.}
 

\begin{DefT}{Complémentaire}\index{Ensemble!Complémentaire}
Soit $\Omega$ un ensemble contenant un ensemble $A$. On appelle complémentaire de $A$ dans $\Omega$, tous les éléments de $\Omega$ qui n'appartiennent pas à $A$. On note le complémentaire de $A$, $\Omega \ A$ ou $\overline{A}$.
\end{DefT}



 
\EPC{1}{FEA-90}{Représenter. Chercher.}

\EPC{0}{FEA-104}{Représenter. Chercher.}
  
\EPCC{1}{FEA-91}{Modéliser. Calculer. Chercher.}  
  






  
\begin{Approfondissement}


\begin{DefT}{Nombre décimal périodique}
Le nombre $a_0,\underline{a_1a_2a_3}$ est un nombre décimal périodique de période $a_1a_2a_3$. Les chiffres $a_1$, $a_2$, $a_3$ se répètent indéfiniment.
\end{DefT}


On démontre que 
\begin{description}
\item[•] $0,\underline{12}$ est un nombre rationnel.
\item[•] $0,\underline{9}=1$.
\end{description}

\EPC{1}{FEA-4}{Chercher.}

\end{Approfondissement}



%
%\begin{Log}\index{Démonstration par l'absurde}\index{Contraposée}
%\begin{description}[leftmargin=*]
%\item[•] La \textbf{contraposée} d'une implication "si A alors B" est l'implication : "si non(B) alors non(A)". Pour démontrer que "si A alors B" , on peut démonter que : "si non(B) alors non(A)".  En langage symbolique, on écrit  : $A \Longrightarrow B \Longleftrightarrow \rceil B \Longrightarrow \rceil A$ 
%\item[•] Une démonstration est appelée \textbf{démonstration par l'absurde} lorsqu'on démontre que la supposition posée \textit{a priori} mène à une absurdité.
%\end{description}
%\end{Log}
