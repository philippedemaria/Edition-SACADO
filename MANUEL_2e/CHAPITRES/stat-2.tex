
Dans un groupe de personnes, on a relevé les âges. Compléter le tableau ci-dessous :


\begin{tabular}{|c|c|>{\centering\arraybackslash}p{2cm}|>{\centering\arraybackslash}p{2cm}|>{\centering\arraybackslash}p{2cm}|>{\centering\arraybackslash}p{2cm}|>{\centering\arraybackslash}p{2cm}|}
\hline 
Age & Effectifs & Effectifs
cumulés
croissants & Effectifs
cumulés décroissants & Fréquence & Fréquence
cumulées croissantes & Fréquence
cumulées décroissantes \\ 
\hline 
0 à 10 ans & 10 &  &  &  &  &  \\ 
\hline 
10 à 20 ans & 14 &  &  &  &  &  \\ 
\hline 
20 à 30 ans & 17 &  &  &  &  &  \\ 
\hline 
30 à 40 ans & 24 &  &  &  &  &  \\ 
\hline 
40 à 50 ans & 13 & &  &  &  &  \\ 
\hline
50 à 60 ans & 11 & &  &  &  &  \\ 
\hline 
60 à 70 ans & 7 &  &  &  &  &  \\ 
\hline 
Total &  &  &  &  &  &  \\ 
\hline 
\end{tabular} 

\begin{enumerate}
\item Combien de personnes ont plus de 30 ans ?
\item Combien de personnes ont moins de 40 ans ? 
\item Combien de personnes ont au moins de 50 ans ? 
\item Combien de personnes ont entre 30 ans et 60 ans ?
\item Quel est le pourcentage des personnes de plus de 40 ans ?
\item Quel est le pourcentage des personnes d'au plus de 20 ans ?
\end{enumerate}
