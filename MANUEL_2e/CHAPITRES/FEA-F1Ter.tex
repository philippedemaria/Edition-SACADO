\begin{titre}{Les ensembles de nombres et intervalles}

\Titre{Les ensembles et intervalles}{2}
\end{titre}

\begin{CpsCol}
\begin{description}
\item[$\square$] \textbf{Chercher :}  Tester, essayer plusieurs pistes de résolution.
\item[$\square$] \textbf{Représenter :} Produire et utiliser plusieurs représentations des nombres.
\item[$\square$] \textbf{Raisonner :} Mener collectivement une investigation en sachant prendre en compte le point de vue d’autrui.
\item[$\square$] \textbf{Communiquer :} Expliquer à l’oral ou à l’écrit (sa démarche, son raisonnement, un calcul, un protocole de construction géométrique, un algorithme), comprendre les explications d’un autre
et argumenter dans l’échange.
\end{description}
\end{CpsCol}


\EPCN{Représenter. Raisonner. }

\begin{tabular}{ccc}

$\frac{2}{10}..........\Z$ & $-\sqrt{25}..........\Z$ & $\frac{\sqrt{3}}{4}..........\Q$ \\ 

$\pi..........\R$  & $-\frac{5}{3}..........\Q$  &  $\sqrt{11}..........\R$ \\ 

\end{tabular} 

\EPCN{Représenter. Raisonner. }

Recopier et compléter le tableau.

\begin{tabular}{|c|c|c|}
\hline 
Intervalle & Inégalité & Représentation  \vplus \\ 
\hline 
$x\in \left[ -5 ; \frac{2}{3}\right]$ & $-5  \leq x \leq  \frac{2}{3} $  &  \vplus \\ 
\hline 
 & $-1 \leq x <4$ &  \vplus  \\ 
\hline 
$x\in \left[ 3 ; 6 \right[ $  &  &  \vplus  \\ 
\hline 
 &  & \definecolor{ffdxqq}{rgb}{1.,0.8431372549019608,0.}
\definecolor{ffxfqq}{rgb}{1.,0.4980392156862745,0.}
\begin{tikzpicture}[line cap=round,line join=round,>=triangle 45,x=1.0cm,y=1.0cm]
\draw[->,color=black] (-5.174092090680384,0.) -- (2.566282833730012,0.);
\foreach \x in {-5.,-4.,-3.,-2.,-1.,1.,2.}
\draw[shift={(\x,0)},color=black] (0pt,2pt) -- (0pt,-2pt) node[below] {\footnotesize $\x$};
\draw[color=black] (0pt,-10pt) node[right] {\footnotesize $0$};
\clip(-5.174092090680384,-0.4115875953650586) rectangle (2.566282833730012,0.4791698364123281);
\draw [line width=2.4pt,color=ffxfqq] (-3.,0.)-- (2.,0.);
\end{tikzpicture}  \vplus \\ 
\hline 
\end{tabular} 

\mini{
\EPCN{Représenter. Raisonner. Communiquer.}

Déterminer l'ensemble des valeurs de $x$ dans chaque cas.
\begin{enumerate}
\item On jette un dé à 6 face et on regarde la face obtenue. Soit $x$ le numéro de la face. 
\item $[-1,1;3]$ et $[2,9;6]$
\item $x > -4$ et $x \leq 10$
\item $x \leq -3$ et $x \leq 5$
\item $x \leq 5$ ou $x \geq 2$
\end{enumerate}
}{
\EPCN{Représenter. Raisonner. Communiquer.}


On propose dans chaque cas deux ensembles. Lequel est inclus dans l'autre ? Écrire ensuite une phrase :" $x$ appartient à .... donc $x$ appartient à ....."

\begin{enumerate}
\item $\left[ -\frac{11}{10};\frac{29}{10}\right]$ et $\left[-\frac{3}{2};3 \right]$
\item $\left[ \frac{1}{2}; +\infty \right[$ et $[0,7;0,8]$.
\item $[1;2]$ et $]1;2[$. 
\end{enumerate}
}


\mini{

\EPCN{Représenter. Raisonner. Communiquer.}


Déterminer les intersections des ensembles suivants. On écrira : $A \cap B = $ où $A$ et $B$ sont les ensembles ci-dessous.
 

\textit{{\small On pourra représenter chaque intervalle sur une droite graduée.}}



\begin{minipage}{0.48\linewidth}

\begin{enumerate}
\item $\Z$ et $\Q$
\item $[-5;2[$ et $[0;7]$
\item $[-1;4]$ et $[-3;-1]$
\item $\N$ et $]-\infty;5]$
\item $[-5;0[$ et $[0;3]$
\end{enumerate}

\end{minipage}
\hfill
\begin{minipage}{0.48\linewidth}
 
\begin{enumerate}
\item

\begin{tikzpicture}[line cap=round,line join=round,>=triangle 45,x=1.0cm,y=1.0cm]
\draw [->,line width=1.pt,domain=0.34:6.36] plot(\x,{(-14.-0.*\x)/7.});
\end{tikzpicture}
\item

\begin{tikzpicture}[line cap=round,line join=round,>=triangle 45,x=1.0cm,y=1.0cm]
\draw [->,line width=1.pt,domain=0.34:6.36] plot(\x,{(-14.-0.*\x)/7.});
\end{tikzpicture}
\item

\begin{tikzpicture}[line cap=round,line join=round,>=triangle 45,x=1.0cm,y=1.0cm]
\draw [->,line width=1.pt,domain=0.34:6.36] plot(\x,{(-14.-0.*\x)/7.});
\end{tikzpicture}
\item

\begin{tikzpicture}[line cap=round,line join=round,>=triangle 45,x=1.0cm,y=1.0cm]
\draw [->,line width=1.pt,domain=0.34:6.36] plot(\x,{(-14.-0.*\x)/7.});
\end{tikzpicture}
\item

\begin{tikzpicture}[line cap=round,line join=round,>=triangle 45,x=1.0cm,y=1.0cm]
\draw [->,line width=1.pt,domain=0.34:6.36] plot(\x,{(-14.-0.*\x)/7.});
\end{tikzpicture}
\end{enumerate}

\end{minipage}
}{

\EPCN{Représenter. Raisonner. Communiquer.}


Déterminer, dans chaque cas, la réunion des ensembles suivants. On écrira : $A \cup B = $ où $A$ et $B$ sont les ensembles ci-dessous.
\begin{enumerate}
\item $A=\left\lbrace 1;3;5;7  \right\rbrace $ et $B=\left\lbrace 0;2;4;5;7;8  \right\rbrace $\point{1}
\item $A=[-3;4]$ et $B=[2;6]$\point{1}
\item $A=[0;+\infty[$ et $B=]-\infty;5]$\point{1}
\end{enumerate}
On pourra représenter les intervalles sur une droite graduée tracée à main levée.
}



\EPCNA{Calculer. Représenter. Raisonner. Chercher.}

\begin{enumerate}
\item On considère le nombre $\frac{19}{11}$.

\begin{enumerate}
\item Donner le développement décimal de $\frac{19}{11}$ avec 8 chiffres significatifs. $\frac{19}{11}$ semble-t-il décimal ?
\item On dit que $\frac{19}{11}$ a une écriture périodique.
Préciser sa période (série de chiffres qui se répète à l'infini dans le développement décimal).
\end{enumerate}
\item On considère le nombre $x=0,13131313....$ dont le développement décimal a pour période 13.
\begin{enumerate}
\item Démontrer que $100x = 13 + x$. 
\item  En déduire une écriture fractionnaire de $x$. Quelle est la nature du nombre $x$ ?
\end{enumerate}
\item Démontrer que $x=3,412412412...$ est un nombre rationnel. 
\item Estimer le résultat avec la calculatrice.
\end{enumerate}
