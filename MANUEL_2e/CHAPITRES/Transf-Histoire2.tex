
\begin{His}

{\large {\color{brown}Empilements de disques et de sphères}}

\bigskip


Comment remplir le plan avec des disques de manière
optimale, c'est-à-dire en laissant le moins possible
d'espace libre ? La Fig. 1 montre deux exemples 
remplissages, celui de droite est irrégulier et ne semble pas optimum, celui de gauche est appelé {\em arrangement hexagonal}, et à l'air meilleur que le 
précédent.

Cette question,
d'apparence simple, se trouve être extrêmement compliquée.
C. F. \textsc{Gauss} a démontré que, parmi tous les remplissages
réguliers (c'est-à-dire, invariants par un certaines transformations
du plan), l'arrangement hexagonal est le plus dense, mais
il n'a pas su prouver qu'il n'y a pas de remplissage non régulier
meilleur.  Il a fallu attendre 1910 pour que \textsc{Thue} démontre que 
l'arrangement hexagonal est le meilleur parmi tous les remplissages,
réguliers ou non.

\medskip
\centerline{
\definecolor{ffqqqq}{rgb}{1,0,0}
\definecolor{zzttqq}{rgb}{0.6,0.2,0}
\definecolor{qqqqff}{rgb}{0,0,1}
\newcommand{\disque}[1]{\draw [color=ffqqqq,fill=ffqqqq,fill opacity=1.0]#1 circle (0.5cm);}
\begin{tikzpicture}[x=0.5cm,y=0.5cm]
\clip(-3.93,-1.15) rectangle (8.75,6.05);
\disque{(0.41,3.75)}
\disque{(0.52,1.74)}
\disque{(2.32,3.1)}
\disque{(2.43,1.1)}
\disque{(4.08,2.18)}
\disque{(4.21,0.2)}
\disque{(2.54,-0.9)}
\disque{(0.63,-0.29)}
\disque{(-1.13,-1.28)}
\disque{(-1.22,0.72)}
\disque{(-2.91,-0.35)}
\disque{(6.13,0.86)}
\disque{(5.73,-1.12)}
\disque{(7.6,-0.49)}
\disque{(8.06,1.47)}
\disque{(6.57,2.84)}
\disque{(4.96,3.98)}
\disque{(8.48,3.41)}
\disque{(7.03,4.79)}
\disque{(9.01,5.34)}
\disque{(5.4,5.96)}
\disque{(3.16,4.9)}
\disque{(1.24,5.54)}
\disque{(-1.31,2.7)}
\disque{(-3.02,1.67)}
\disque{(-3.09,3.67)}
\disque{(-1.37,4.71)}
\disque{(-3.16,5.67)}
\disque{(-4.75,0.55)}
\disque{(-4.83,2.66)}
\disque{(-0.53,6.55)}
\disque{(3.6,6.87)}
\disque{(7.58,6.73)}
\disque{(9.54,0.08)}
\end{tikzpicture}
\hskip 2em
\begin{tikzpicture}[x=0.5cm,y=0.5cm]
\clip(-3.93,-1.15) rectangle (8.75,6.05);
\disque{(-4.000,-1.000)}
\disque{(-3.000,0.732)}
\disque{(-4.000,2.464)}
\disque{(-3.000,4.196)}
\disque{(-4.000,5.928)}
\disque{(-2.000,-1.000)}
\disque{(-1.000,0.732)}
\disque{(-2.000,2.464)}
\disque{(-1.000,4.196)}
\disque{(-2.000,5.928)}
\disque{(0.000,-1.000)}
\disque{(1.000,0.732)}
\disque{(0.000,2.464)}
\disque{(1.000,4.196)}
\disque{(0.000,5.928)}
\disque{(2.000,-1.000)}
\disque{(3.000,0.732)}
\disque{(2.000,2.464)}
\disque{(3.000,4.196)}
\disque{(2.000,5.928)}
\disque{(4.000,-1.000)}
\disque{(5.000,0.732)}
\disque{(4.000,2.464)}
\disque{(5.000,4.196)}
\disque{(4.000,5.928)}
\disque{(6.000,-1.000)}
\disque{(7.000,0.732)}
\disque{(6.000,2.464)}
\disque{(7.000,4.196)}
\disque{(6.000,5.928)}
\disque{(8.000,-1.000)}
\disque{(9.000,0.732)}
\disque{(8.000,2.464)}
\disque{(9.000,4.196)}
\disque{(8.000,5.928)}
\end{tikzpicture}
}
\centerline{Fig. 1 : Deux exemples de remplissages : un non régulier et un 
régulier.}
\medskip

Le problème se pose aussi en dimension 3 : comment empiler des
sphères dans l'espace de la manière la plus dense possible ?
\textsc{Kepler} conjectura en 1611 que l'empilement optimal est régulier,
et qu'il est obtenu par exemple en prenant un arrangement hexagonal
dans un plan, puis, au dessus,  en plaçant une sphère dans chaque 
creux formé par 3 sphères adjacentes du plan inférieur, et en répétant
ainsi à l'infini (Fig. 2). Une démonstration de cette conjecture a enfin 
été publiée en 1998 par Thomas \textsc{Hales} mais la démonstration
était tellement difficile à vérifier que ce n'est qu'en 2014 qu'elle
a été officiellement déclarée valide !


\begin{center}
\includegraphics[scale=0.1]{image_chapitres/spheres.png} 

Fig. 2 : un empilement optimal de sphères
\end{center}


Pour les dimensions supérieures, il n'y a pour l'instant
que des résultats partiels, sauf pour les dimensions 8 et 24, pour
lesquels l'empilement optimal est connu, et se trouve être régulier.
Ce résultat est très récent, il a été obtenu en Avril 2016 par l'Ukrainienne
Maryna \textsc{Viazovska}


\begin{center}
\includegraphics[height=4cm]{image_chapitres/Hale.jpg} 
\hspace{2cm}
\includegraphics[height=4cm]{image_chapitres/Viazovska.jpg}

Thomas \textsc{Hale} et Maryna \textsc{Viazovska} (Sources : Wikipedia)
\end{center}


Ces résultats ne sont pas que théoriques : les empilements
de sphères sont utilisés en informatique pour obtenir de
bons {\em codes correcteurs d'erreur} qui servent par exemple
à obtenir des communications fiables sur internet. 

\PESP{https://fr.wikipedia.org/wiki/Fraction\_\%C3\%A9gyptienne}
\end{His}
