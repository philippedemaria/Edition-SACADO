
En utilisant le tableau à l'URL suivante \href{http://www.insee.fr/fr/themes/document.asp?reg\_id=0&ref\_id=ip1332}{Source INSEE},

\begin{enumerate}
\item Année 2001
\begin{enumerate}
\item Compléter le tableau suivant.


\begin{tabular}{|>{\centering\arraybackslash}p{2cm}|>{\centering\arraybackslash}p{1.5cm}|>{\centering\arraybackslash}p{1.5cm}|>{\centering\arraybackslash}p{1.5cm}|>{\centering\arraybackslash}p{1.5cm}|}
\hline 
Année 2001 & Moins de 20 ans & 20-59 ans & 60-64 ans & 65 ou plus \\ 
\hline 
Milieu de la classe &  &  &  & 80 \\ 
\hline 
Effectif &  &  &  &  \\ 
\hline 
Effectifs cumulés croissants &  &  &  &  \\ 
\hline 
Fréquence &  &  &  &  \\ 
\hline 
Fréquences cumulées croissantes &  &  &  &  \\ 
\hline 
\end{tabular} 

\item Calculer la moyenne d'age de la population.
\item Tracer le polygone des fréquences cumulées croissantes.
\item En déduire le plus précisément possible la médiane, le
premier quartile et le troisième quartile.
\end{enumerate}

\item Reprendre cette étude pour l'année 2010.

\item Commenter.
\end{enumerate}