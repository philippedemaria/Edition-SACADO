
Calculer le salaire net annuel moyen en France en 2005.

\begin{tabular}{|c|c|c|}
\hline 
Régions & Fréquences (\%) & Salaires (en euros) \\ 
\hline 
Régions parisiennes & 25,3 & 29237 \\ 
\hline 
Bassins parisiens & 15,7 & 20318 \\ 
\hline 
Nord & 5,8 & 20501 \\ 
\hline 
Est & 8 & 20946 \\ 
\hline 
Ouest & 13 & 19891 \\ 
\hline 
Sud-Ouest & 9,3 & 20542 \\ 
\hline 
Centre-Est & 11,1 & 25811 \\ 
\hline 
Méditerranées & 10 & 20993 \\ 
\hline 
DOM & 1,8 & 20495 \\ 
\hline 
\end{tabular}
 
\vspace{0.5cm}
Soit $\overline{s}$ le salaire net en France.

\vspace{0.5cm}

$\overline{s} = \frac{25,3}{100}\times 29237 + \frac{15,7}{100}\times 20318 +\frac{5,8}{100}\times 20501 +\frac{8}{100}\times 20946 +\frac{13}{100}\times 19891 +\frac{9,3}{100}\times 20542 +\frac{11,1}{100}\times 25811 +\frac{10}{100}\times 20993 +\frac{10}{1,8}\times 20495  = 22210$, arrondi à l'unité près.

\vspace{1cm}

Déterminer les régions dont le salaire net annuel moyen est supérieur à la moyenne des salaires nets annuels.

\vspace{0.5cm}

Régions parisiennes et Centre-Est