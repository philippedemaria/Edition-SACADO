
Dans sa penderie, Guillaume a deux pantalons, un noir et un blanc, deux vestes, une noire et une blanche, et trois chemises, deux blanches et une noire. Il prend au hasard un pantalon, une chemise et une veste.
\begin{enumerate}
\item A l'aide d'un arbre de dénombrement, déterminer le nombre de façon différentes de s'habiller.
\item
\begin{enumerate}
\item Calculer la probabilité de l'événement $A$ : "Il est habillé tout en noir".
\item Calculer la probabilité de l'événement $B$ : "Il est habillé tout en blanc".
\item $A$ et $B$ sont-ils contraires ?
Calculer la probabilité de l'événement $C$ : "Il une veste et un pantalon de couleurs différentes".
\end{enumerate}
\end{enumerate}