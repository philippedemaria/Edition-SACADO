\begin{titre}[Géométrie dans l'espace]

\Titre{Parallélisme}{4}
\end{titre}


\begin{CpsCol}
\textbf{Parallélisme}
\begin{description}
\item[$\square$] Déterminer l'intersection d'une droite et d'un plan
\item[$\square$] Déterminer l'intersection de deux plans
\end{description}
\end{CpsCol}


\begin{Reg}[Parallélisme entre droites]
\begin{description}
\item[P1] Deux droites parallèles à une même droite sont parallèles entre elles.
	$$ \text{ Si } d// d' \text{ et } d'// d" \text{ alors } d// d''$$
\item[P2]   Si deux droites sont parallèles, alors tout plan qui coupe l'une, coupe l'autre.
\end{description}
\end{Reg}


\begin{Reg}[Parallélisme entre plans]
\begin{description}
\item[P3]   Deux plans parallèles à un même plan sont parallèles entre eux.
	$$\text{ Si } \mathscr{P} // \mathscr{P}' \text{ et }  \mathscr{P}'// \mathscr{Q} \text{ alors } \mathscr{P} //\mathscr{Q}$$

\item[P4]  Si deux droites sécantes $d$ et $d'$ d'un plan $\mathscr{P}$   sont parallèles à deux droites sécantes $\Delta$ et $\Delta '$ d'un plan $\mathscr{Q}$ alors $\mathscr{P}$  et $\mathscr{Q}$ sont parallèles.

\definecolor{ffqqqq}{rgb}{1.,0.,0.}
\definecolor{qqqqff}{rgb}{0.,0.,1.}
\definecolor{qqwuqq}{rgb}{0.,0.39215686274509803,0.}
\begin{tikzpicture}[line cap=round,line join=round,>=triangle 45,x=0.5cm,y=0.5cm]
\clip(4.28,-2.3) rectangle (15.16,3.98);
\fill[color=qqwuqq,fill=qqwuqq,fill opacity=0.11] (5.,1.) -- (11.,1.) -- (15.,3.) -- (9.,3.) -- cycle;
\fill[color=qqwuqq,fill=qqwuqq,fill opacity=0.11] (9.,0.) -- (15.,0.) -- (11.,-2.) -- (5.,-2.) -- cycle;
\draw [color=qqqqff] (6.12,1.26)-- (13.22,2.62);
\draw [color=qqqqff] (6.2,-1.72)-- (13.18,-0.34);
\draw [color=ffqqqq] (8.46,2.5)-- (12.,2.);
\draw [color=ffqqqq] (8.78,-0.44)-- (12.,-1.);
\end{tikzpicture}


\item[P5]  Si deux plans $\mathscr{P}$  et $\mathscr{P}'$ sont parallèles alors tout plan qui coupe $\mathscr{P}$  coupe $\mathscr{P'}$ et les droites d'intersection $d$ et $d'$ sont parallèles.

\AV{https://www.geogebra.org/m/Hj7trpR8}{Visualisation}
\end{description}
\end{Reg}


\AD{1}{GE-6}

\Exo{1}{GE-13}



\begin{Reg}[Parallélisme entre droite et plan]
\begin{description}
\item[P6] Si deux plans $\mathscr{P}$ et $\mathscr{P}'$ sont parallèles et si une droite $d$ est parallèle à $\mathscr{P}$ alors $d$ est parallèle à $\mathscr{P}'$.
\item[P7]  Si deux droites $d$ et $d'$ sont parallèles et si $d$ est contenue dans un plan $\mathscr{P}$, alors $d'$ est parallèle à $\mathscr{P}$.
\item[P8]  Si deux plans $\mathscr{P}$ et $\mathscr{P}'$ sont sécants selon une droite $\Delta$ et si une droite $d$ est parallèle à $\mathscr{P}$ et à $\mathscr{P}'$ alors d est parallèle à $\Delta$.

\AV{https://www.geogebra.org/m/yCxHKp2T}{Visualisation}

\end{description}
\end{Reg}



\AD{1}{GE-14}

\Exo{1}{GE-15}



\begin{ThT}{Théorème du toit}
Si 
$ \left\lbrace  \begin{tabular}{l}
$d$ et $d'$ sont des droites parallèles \\ 
 $\mathscr{P}$  est un plan qui contient $d$ et  $\mathscr{P}'$ un plan qui contient $d'$ \\ 
 $\mathscr{P}$  et  $\mathscr{P}'$ sont sécants selon une droite $\Delta$ \\ 
\end{tabular} 
 \right\rbrace  $
alors $\Delta$  est parallèle à $d$ et à $d'$.

\AV{https://www.geogebra.org/m/KQJzHRS3}{Visualisation}
\end{ThT}


\AD{1}{GE-18}




