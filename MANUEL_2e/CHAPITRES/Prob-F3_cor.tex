\documentclass[openany]{book}

%%%%%%%%%%%%%%%%%%%%%%%%%%%%%%%%%%%%%%%%%%%%%%%%%%%%%%%%%%%%%%%%%%%%%%%%%%%%
%% Pour passer du manuel au cahier :
%%  Dans le préambule choisir styleExercices ou styleCahier
%%  1 imprime  -----     0 n'imprime pas %%%%%%%%%

%%%%%%%%%%%%%%%%%%%%%%%%%%%%%%%%%%%%%%%%%%%%%%%%%%%%%%%%%%%%%%%%%%%
\input{../../../latex_preambule_style/preambule}

%\input{../../../latex_preambule_style/accessibilite}
%\input{../../../latex_preambule_style/styleExercices_access}

\input{../../../latex_preambule_style/styleCoursLycee}
\input{../../../latex_preambule_style/styleExercices}

%\input{../../latex_preambule_style/styleCahier}
%\input{../../../latex_preambule_style/groupeEEMCP}
\input{../../../latex_preambule_style/bas_de_page_vierge}
\input{../../../latex_preambule_style/algobox}


 

 
        
%%%%%%%%%%%%%%%%%%%%%%%%%%%%%%%%%%%%%%%%%%%%%%%%%%%%%%%%%%%%%%%%%%%%%%%%%%%%
%%  Impression conditionnelle :
%%  1 imprime  -----     0 n'imprime pas 
%%%%%%%%%%%%%%%%%%%%%%%%%%%%%%%%%%%%%%%%%%%%%%%%%%%%%%%%%%%%%%%%%%%%%%%%%%%%

\begin{document}


\subsection{Exercice 6}

 


\begin{enumerate}
\item Calculer $p(A \cup B)$.

$A$ et $B$ sont \textbf{incompatibles} donc d'après la relation fondamentale, propriété 1, $p(A \cup B)=p(A)+p(B)$.

D'après l'énoncé, \fbox{$p(A \cup B)=0,4+0,5 = 0,9$}

\item $\overline{A}$ et $\overline{B}$ sont-ils incompatibles ?

Supposons que $\overline{A}$ et $\overline{B}$ soient incompatibles. Alors $p(\overline{A} \cup \overline{B})=p(\overline{A})+p(\overline{B})$.

Or, $p(\overline{A})= 1-p(A)=1-0,4=0,6$ 

et $p(\overline{B})= 1-p(B)=0,5$ 

On aurait alors : $p(\overline{A} \cup \overline{B})=0,6+0,5 = 1,1$ Or par définition de la probabilité, définition 10, $0\leq p \leq 1$ donc la supposition initiale n'est pas possible donc \fbox{$\overline{A}$ et $\overline{B}$ ne sont pas incompatibles}.

\end{enumerate}

 

\newpage

 

\subsection{Exercice 7}
 
 

 $p(A \cap B) = 0,1$ donc $A \cap B$ n'est pas l'ensemble vide. Les événements $A$ et $B$ ne sont pas incompatibles.
 
 
\definecolor{fffteu}{rgb}{1.,0.9529411764705882,0.8941176470588236}
\definecolor{qqwwzz}{rgb}{0.,0.4,0.6}
\definecolor{ffdxqq}{rgb}{1.,0.8431372549019608,0.}
\begin{tikzpicture}[line cap=round,line join=round,>=triangle 45,x=1.0cm,y=1.0cm]
\clip(-1.14,-0.56) rectangle (12.58,7.1);
\draw [rotate around={-2.726310993906266:(5.13,3.33)},line width=2.pt,color=fffteu,fill=fffteu,fill opacity=1.0] (5.13,3.33) ellipse (5.80158019894905cm and 3.2057031685478448cm);
\draw [line width=2.pt,color=ffdxqq,fill=ffdxqq,fill opacity=0.5] (3.5,3.66) circle (1.1497825881443848cm);
\draw [line width=2.pt,color=qqwwzz,fill=qqwwzz,fill opacity=0.6499999761581421] (5.5,3.56) circle (1.6100931650062982cm);
\draw (2.88,4.28) node[anchor=north west] {A};
\draw (5.88,3.88) node[anchor=north west] {B};

\draw (9.52,4.66) node[anchor=north west] {$\Omega$};
\end{tikzpicture}

	$A$ et $B$ sont deux événements tels que :
 $ p(A)=0,2$,  $p(B)=0,5$ , $p(A \cap B) = 0,1$.
\begin{enumerate}
\item Déterminer les probabilités des événements suivants :
 

 

	\begin{enumerate}
	\item $p(A \cup B)$	; d'après la relation fondamentale,  $ p(A \cup B)=p(A)+p(B)-p(A \cap B)=0,2+0,5- 0,1=0,6$
	\item $p(A \cap \overline{B})$
	
	D'après le diagramme de Venn, $A \cap \overline{B}$ est la partie entièrement jaune sans superposition.
	
	\fbox{$p(A \cap \overline{B}) = 0,1$}.
	
	\textbf{Par le calcul}. \textbf{Niveau 2}
	
	$A = (A \cap \overline{B}) \cup (A \cap B)$ avec  $A \cap \overline{B}$ et $A \cap B $ \textbf{incompatibles} donc $p(A) = p(A \cap \overline{B}) + p(A \cap B)$, d'après la relation fondamentale. 
	
	donc \fbox{ $p(A \cap \overline{B}) =  p(A)- p(A \cap B) =  0,2-0,1=0,1 $}
	
	\item $p(\overline{A} \cap B)$
	
		D'après le diagramme de Venn, $B \cap \overline{A}$ est la partie entièrement bleue sans superposition.
	
	\fbox{$p(B \cap \overline{A}) = 0,5-0,1=0,4$}.
	
	\textbf{Par le calcul}. \textbf{Niveau 2} Même idée de démonstration. 

$B = (B \cap \overline{A}) \cup (B \cap A)$ avec  $B \cap \overline{A}$ et $B \cap A $ \textbf{incompatibles} donc $p(B) = p(B \cap \overline{A}) + p(B \cap A)$, d'après la relation fondamentale. Or $A \cap B = B \cap A$ donc $p(A \cap B)=p(B \cap A)=0,1$
	
	donc \fbox{$p(\overline{A} \cap B) =  p(B)- p(A \cap B) =  0,5-0,1=0,4 $}
	
	
	\item $p(\overline{A} \cup B)$
	
$p(\overline{A} \cup B) = p(\overline{A})+p(B)-p(\overline{A} \cap B)$

D'après ce qui précède, $p(\overline{A} \cup B) = 1- p(A)+p(B)-p(\overline{A} \cap B) = 0,8+0,5-0,4=0,9$
	
	
	\item $p(\overline{A} \cap \overline{B})$
	
	$\overline{A} \cap \overline{B} = \Omega \setminus (A \cup B)$  donc 
	
	
	$p(\overline{A} \cap \overline{B}) = 1- p(A \cup B)$ et d'après les résultats précédents, \fbox{$p(\overline{A} \cap \overline{B}) = 1- 0,6=0,4$} 
	
	\end{enumerate}
	
\item Identifier parmi les événements étudiés dans la question 1, l'événement contraire de $A \cup B$


\fbox{$\overline{A \cup B } = \overline{A} \cap \overline{B}$ }

\end{enumerate}

 

\newpage



\subsection{Exercice 8}
 
On lance un dé à 12 faces numérotées de 1 à 12. On considère les événements :

\begin{description}
\item[A] On obtient un diviseur de 12 ;
\item[B] On obtient un multiple de 3 ;
\item[C] On obtient un nombre premier.
\end{description}

Avant de commencer l'exercice, définissons le cadre de notre expérience.

on s'intéresse à un lancer d'un dé à 12 faces, non truqué, on regarde à la face obtenue.
Une issue est le numéro d'une face. Par exemple, le 2. Il y a donc 12 issues possibles. L'univers est donc l'ensemble de ces douze issues possibles.
On écrit $$ \Omega = \lbrace 1;2;3;4;5;6;7;8;9;10;11;12\rbrace$$


Chaque issue a une probabilité égale à $\frac{1}{12}$.
\begin{enumerate}
\item Déterminer $p(A)$, $p(B)$, $p(C)$. 

$A$ est l'événement, :"Obtenir un diviseur de 12". 
Parmi les 12 faces, on a les diviseurs de 12 suivants : $  1;2;3;4;6;12$  qui sont les issues favorables à $A$

Donc \fbox{$ A = \lbrace 1;2;3;4;6;12\rbrace$}


  et \fbox{$p(A)=\frac{1}{12}+\frac{1}{12}+\frac{1}{12}+\frac{1}{12}+\frac{1}{12}+\frac{1}{12}=\frac{6}{12}=\frac{1}{2}=0,5$}


Avec un raisonnement analogue,  \fbox{$ B = \lbrace 3;6;9;12\rbrace$ } 

et \fbox{$p(B)=\frac{1}{12}+\frac{1}{12}+\frac{1}{12}+\frac{1}{12}=\frac{4}{12}=\frac{1}{3}$}

et avec ce même raisonnement analogue, ,  \fbox{$C = \lbrace 2;3;5;7;11 \rbrace$} 

et \fbox{$p(C)=\frac{1}{12}+\frac{1}{12}+\frac{1}{12}+\frac{1}{12}+\frac{1}{12}=\frac{5}{12}$}

\item Donner l'écriture ensembliste de $A \cap B$ et en déduire sa probabilité. 

$A \cap B = \lbrace 1;2;3;4;6;12\rbrace \cap \lbrace 3;6;9;12\rbrace =  \lbrace 3;6;12\rbrace$ 


\fbox{$p(A \cap B )=\frac{1}{12}+\frac{1}{12}+\frac{1}{12}=\frac{3}{12}=\frac{1}{4}$}

\item Donner l'écriture ensembliste de $B \cap C$ et en déduire sa probabilité.

$B \cap C = \lbrace 1;2;3;4;6;12\rbrace \cap \lbrace 2;3;5;7;11 \rbrace =  \lbrace 2;3\rbrace$ 

\fbox{$p(B \cap C)=\frac{1}{12}+\frac{1}{12}=\frac{2}{12}=\frac{1}{6}$}

\end{enumerate}

\end{minipage}

\newpage


 

\subsection{Exercice 9}

Dans le mot BOSTON, il y 6 lettres. L'expérience consiste à tirer une lettre.
Les issues sont B, O, S, T, O, N.

L'univers est donc $\Omega = \lbrace  B; O; S; T; O; N \rbrace$. Chaque issue a la même probabilité et un événement élémentaire $\omega_i$ est composé d'une lettre. Par exemple : $\omega = \lbrace  N \rbrace$
\begin{enumerate}
\item On considère l'évènement A : "tirer une voyelle".

$A = \lbrace  O;  O  \rbrace$

\fbox{$p(A)=\frac{1}{6}+\frac{1}{6}=\frac{2}{6}=\frac{1}{3}$}

\item On considère l'évènement B : "tirer une lettre du mot PROBABILITES".

Les lettres de probabilités sont $\lbrace  P;R;O;B;A;B;I;L;I;T;E;S \rbrace$. Pour tirer une lettre de ce mot, il faut donc choisir les lettres dans l'ensemble $\lbrace  B; O; S; T; O; N \rbrace \cap \lbrace  P;R;O;B;A;B;I;L;I;T;E;S \rbrace = \lbrace  B; O; S; T; O; N \rbrace$.

Donc quelque soit la lettre tirée dans BOSTON, cette lettre appartient à PROBABILITES. Donc \fbox{$p(B) = 1$}

\item On considère l'évènement C : "tirer une lettre du mot MILIEU ".

Les lettres de probabilités sont $\lbrace  M;I;L;I;E;U \rbrace$. Pour tirer une lettre de ce mot, il faut donc choisir les lettres dans l'ensemble $\lbrace  B; O; S; T; O; N \rbrace \cap \lbrace M;I;L;I;E;U \rbrace = \lbrace  \rbrace$, l'ensemble vide.

Donc quelque soit la lettre tirée dans BOSTON, cette lettre ne peut pas appartient à MILIEU. Donc \fbox{$p(C) = 0$}


\end{enumerate}


\end{minipage}

\newpage


 
 
 \subsection{Exercice 10}

On appelle :
\begin{description}
\item R, l'événement : "la personne a un tarif réduit", avec $p(R)=0,47$
\item A, l'événement : "la personne a un abonnement", avec $p(A)=0,32$
\item N, l'événement : "la personne a un tarif normal".
\end{description}
Un client ne peut bénéficier d’un tarif réduit s’il possède un abonnement, c'est à dire que R et A sont incompatibles ou encore :  $A \cap R = \oslash $.

Or d'après la formule des probabilités, R, A et N sont les seuls trois événements possibles et ils sont disjoints. On dit qu'ils \textbf{forment une partition de l'univers}.

Donc $p(R) + p(A)+ p(N)=1 \Longleftrightarrow p(N)=1 -p(R) -p(A) \Longleftrightarrow p(N)=1 - 0,47 - 0,32 =0,21$

\fbox{$p(N)=0,21$}
 
 
 \end{minipage}

\newpage


 
 
 \subsection{Exercice 11}
 
On sait que $p(A)=0,6$ et $p(B) = 0,8$. De plus, il y a toujours au moins une des deux boulangeries ouvertes.
 
 \begin{enumerate}
\item $D = A \cup B$. Or, l'énoncé nous dit qu'il y a toujours une boulangerie d'ouverte donc $D = A \cup B$ est un événement \textbf{certain} donc $p(D)=1$.
\item $D$ et $E$ sont deux événements \textbf{contraires}. Donc $p(D)+p(E) =1$. Comme $p(D) = 1$ alors $p(E)=0$. \\
Note : On retrouve : $\overline{A} \cap \overline{B} = \overline{A \cup B}$ 
\item On s'intéresse à l'événement $F = A \cap B$.

D'après la relation fondamentale des probabilités, $p(A \cup B) = p(A) + p(B) - p(A \cap B)$  

$p(A \cap B) = p(A) + p(B) - p(A \cup B) = 0,6 + 0,8 - 1 = 0,4$ 

\fbox{$p(F) = 0,4$ }
\end{enumerate}

\end{minipage}


\newpage

\begin{minipage}{0.5\linewidth}
 
\subsection{Exercice 13}
 
Le jeu d'échecs est composé de 16 pièces blanches et de 16 pièces noires réparties comme suit : un roi, une dame, deux fous, deux cavaliers, deux tours et huit pions.

Les pièces mineurs sont le fou et le cavalier, les pièces lourdes sont la tour et la dame. On choisit une pièce au hasard.


\textbf{On considère l'expérience, tirer une pièce au hasard. On suppose qu'on ne peut pas toucher la pièce. Chaque pièce à la même probabilité d'être choisie et l'expérience est soumise au hasard. Les issues sont équiprobables.}  

Quelle est la probabilité que ce soit
\begin{enumerate}
\item un pion ?

Il y a 32 pièces sur le plateau dont 16 pièces sont des pions. Donc $p = \frac{16}{32}= \frac{1}{2}$


\item une pièce mineure ?

Chaque couleur possède deux fous et deux cavaliers. Il y a donc 8 pièces mineures.

Donc $p = \frac{8}{32}= \frac{1}{4}$

\item un roi ?

Chaque couleur possède UN seul roi. Il y a donc 2 rois.

Donc $p = \frac{2}{32}= \frac{1}{16}$

\end{enumerate} 
 
 
\end{minipage}


\newpage

\begin{minipage}{0.5\linewidth}
 
 
\subsection{Exercice 14}
 
On suppose qu'un couple attend un enfant, il est aussi probable qu'il s'agisse d'une fille ou d'un garçon. 

\begin{enumerate}
\item Représenter, à l'aide d'un arbre, les possibilités pour une famille de 3 enfants.



\begin{tikzpicture}[line cap=round,line join=round,>=triangle 45,x=1.0cm,y=1.0cm]
\clip(-6.2033910390798,-5.588114617475448) rectangle (2.6714450001398293,4.830172845618551);
\draw [line width=2.pt] (-5.,0.)-- (-4.,2.);
\draw [line width=2.pt] (-5.,0.)-- (-4.,-3.);
\draw [line width=2.pt] (-3.,-3.)-- (-1.0424809510553414,-1.5124419756632121);
\draw [line width=2.pt] (-3.,-3.)-- (-1.0183645487748534,-4.430527121576114);
\draw [line width=2.pt] (0.,-4.3822943092469755)-- (1.5138576906763994,-3.4658708749933367);
\draw [line width=2.pt] (0.,-4.3822943092469755)-- (1.5138576906763994,-5.009320869525781);
\draw [line width=2.pt] (0.,-1.4159763510049344)-- (1.63443970207884,-2.115352129777448);
\draw [line width=2.pt] (0.,-1.4159763510049344)-- (1.5862068975178638,-0.7648333845615595);
\draw [line width=2.pt] (-3.140607949457808,2.0326697305284958)-- (-0.9942481464943652,3.57611972506094);
\draw [line width=2.pt] (-3.140607949457808,2.0326697305284958)-- (-0.9942481464943652,0.4651033298314821);
\draw [line width=2.pt] (0.,3.7449345682129263)-- (1.5620904952373758,4.44431034698544);
\draw [line width=2.pt] (0.,3.7449345682129263)-- (1.6585561043593282,2.9973259771112732);
\draw [line width=2.pt] (0.,0.513336142160621)-- (1.6585561043593282,1.3574103579205514);
\draw [line width=2.pt] (0.,0.513336142160621)-- (1.6585561043593282,0.15159004969207934);
\draw (-3.743518006470011,2.3461830106678985) node[anchor=north west] {F};
\draw (-0.6083857100065553,3.8655165990357734) node[anchor=north west] {F};
\draw (1.8032545180422568,4.661358002466565) node[anchor=north west] {F};
\draw (1.8514873226032331,1.5744580134016763) node[anchor=north west] {F};
\draw (-0.5842693077260671,-1.2712779140175177) node[anchor=north west] {F};
\draw (1.8514873226032331,-0.5477857290804344) node[anchor=north west] {F};
\draw (1.8032545180422568,-3.200590407183073) node[anchor=north west] {F};
\draw (-3.6952852019090345,-2.8870771270436704) node[anchor=north west] {G};
\draw (-0.6807349168480196,0.7062673914771765) node[anchor=north west] {G};
\draw (-0.7289677214089958,-4.165246653765851) node[anchor=north west] {G};
\draw (1.8756037248837212,0.3204048928440654) node[anchor=north west] {G};
\draw (1.827370920322745,3.14202441409869) node[anchor=north west] {G};
\draw (1.7309053112007924,-1.994770098954601) node[anchor=north west] {G};
\draw (1.6826725066398163,-4.912855244867504) node[anchor=north west] {G};
\draw (-5.55224817750662,0.7786166099708848) node[anchor=north west] {$\Omega$};
\end{tikzpicture}


\item Quelle est la probabilité de n'avoir que des garçons ?


D'après l'arbre, il y a 1 seule façon d'obtenir que des garçons sur les 8 possibilités. 

Donc $p = \frac{1}{8}$


\item Quelle est la probabilité d'avoir au moins une fille ?

D'après l'arbre, l'événement $\overline{G_3}$ :"avoir au moins une fille" est l'événement contraire de l'événement $G_3$ : "n'avoir que des garçons".

Soit $G_3$ avoir 3 garçons, et  $\overline{G_3}$  l'événement contraire.

Donc $p(G_3)+ p(\overline{G_3})= 1$. Donc $p(G_3)+ p(\overline{G_3})= 1$. Donc $p(\overline{G_3})= 1 - p(G_3) = \frac{7}{8} $;

\textbf{On peut aussi dénombrer les branches de l'arbre.}

\end{enumerate}
 
 
\end{minipage}


\newpage

\begin{minipage}{0.5\linewidth}
 
 
\subsection{Exercice 15}
 
 Dans le population mondiale, les groupes sanguins sont répartis comme l'indique le tableau :

\begin{tabular}{|c|c|c|c|c|}
\hline 
  & O & A & B & AB \\ 
\hline 
Rhésus + & 38 \% & 34 \% & 9 \% & 3 \% \\ 
\hline 
Rhésus - & 7\% & 6 \% & 2 \% & 1 \% \\ 
\hline 
\end{tabular} 

On appelle $Rh^-$ et $Rh^+$, le groupe sanguin est respectivement de Rhésus$+$ ou de Rhésus$-$


On choisit une personne au hasard dans cette population. Quelle est la probabilité qu'elle soit :
\begin{enumerate}
\item donneur universel, c'est à dire du groupe O et de rhésus $-$ ?

En lisant le tableau à double entrée, $p(O \cap Rh^-)=0,07$

\item du groupe AB ?

En lisant le tableau à double entrée, $p(AB)=p(AB \cap Rh^+)+p(AB \cap Rh^-) = 0,03+0,01 = 0,04$

\item de rhésus + ?

En lisant le tableau à double entrée, $$ p(Rh^+)= p(A \cap Rh^+)+p(B \cap Rh^+)+p(AB \cap Rh^+)+p(O \cap Rh^+) = 0,38+0,34 +0,09+0,03 = 0,83 $$


\end{enumerate}
 
 
 
 
 
\end{minipage}




\newpage

\begin{minipage}{0.5\linewidth}
 
 
\subsection{Exercice 16}
 
\textbf{L'expérience consiste à un tirage d'un nom parmi 30 noms}. et non un jour parmi 31 car les jours sont redondants ! Il n'y a que 7 jours...

On peut dire que l'expérience s'apparente à un tirage d'un nombre parmi les nombres de 1 à 30. C'est le même modèle.

Une issue est le 7. Et au 7 novembre on associe Karine. Rassurez-vous, je ne connais pas les noms sur un calendrier.... J'ai regardé sur Internet mais j'ai bien choisi le 7 au hasard.

$$\Omega = \lbrace 1;2;3;4;5;6;7;8;9;10;11;12;13;14;15;16;17;18;19;20;21;22;23;24;25;26;27;28;29;30\rbrace$$
 
L'énoncé nous dit que l'expérience est équiprobable. En appelant E un événement quelconque, $p(E)= \frac{\text{Card} E}{\text{card}\Omega}$
 
\begin{enumerate}
\item Soit I :"Tomber sur un jour impair".
 
 $$I = \lbrace 1;3;5;7;9;11;13;15;17;19;21;23;25;27;29\rbrace$$
 
On peut écrire : Card I = 15. Donc $p(E)= \frac{\text{Card} E}{\text{card}\Omega}= \frac{15}{30}= \frac{1}{2}$
 
\item Soit U :"Tomber sur une date comportant un 1". Il suffit de ne garder que les dates qui contiennent un 1.

$$I = \lbrace 1;11;13;15;17;19;21\rbrace$$
 
  $p(I)= \frac{\text{Card} I}{\text{card}\Omega}= \frac{7}{30}$
 
  
\end{enumerate} 
 
\end{minipage}



\newpage

\begin{minipage}{0.5\linewidth}
 
 
\subsection{Exercice 17}
 
L'expérience consiste à choisir une suite de 3 réponses de façon aléatoire. Une issue est (2,3,1). Ce qui signifie que j'ai choisi, la réponse 2 puis la réponse 3 puis la réponse 1. 

Pour dénombrer le nombre d'issues possibles, on peut dire que pour la première question il y 3 réponses possibles, puis pour la seconde réponse encore 3 réponses possible, et encore 3 réponses possibles pour la question 3.
Soit $3\times 3\times 3 = 27$. En construisant un arbre, on voit bien la situation.


$$\Omega = \lbrace (1;1;1); (1;1;2); (1;1;3); (1;2;1).......(3;3;2);(3;3;3)\rbrace$$
 
 
  
\begin{enumerate}
\item Soit J :"Répondre juste à toutes les questions".
 
Parmi tous ces 27 triplets de réponses, un seul est tout juste.
 
On peut écrire : Card J = 1. Donc $p(J)= \frac{\text{Card} J}{\text{card}\Omega}= \frac{1}{27}$
 
\item Soit J :"Répondre juste à toutes les questions".

Quelque soit le nombres de questions, il n'y a qu'une seule possibilité de répondre juste à toutes les questions puisqu'il n'y a qu'une seule réponse juste par question. 


Le cardinal de $\Omega$ est $3^n$. Alors  $p(J)= \frac{\text{Card} J}{\text{card}\Omega}= \frac{1}{3^n}$ et on souhaite que $p(J)= \frac{1}{3^n} \leq \frac{1}{1000}$

$ \frac{1}{3^n} \leq \frac{1}{1000} \Longleftrightarrow 3^n \geq 1000$.

Avec la calculatrice, on trouve, $n \geq 10$. Donc à partir de 10 questions, l'élève à une chance sur 1000 de réponse tout juste !
 
  
\end{enumerate}
 
 
 
\end{minipage}
 






\end{document}
