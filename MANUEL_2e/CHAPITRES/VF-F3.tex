\begin{titre}[Généralités sur les fonctions]

\Titre{Optimisation}{4}
\end{titre}


\begin{CpsCol}
\textbf{Variations de fonctions}
\begin{description}
\item[$\square$] Déterminer l'extremum d'une fonction
\item[$\square$] Déterminer l'optimum d'une situation donnée
\end{description}
\end{CpsCol}


\mini{
\App{1}{VF-20}

\App{1}{VF-21}

\App{1}{VF-23}
}{
\App{1}{VF-22}

\PO{1}{VF-24}
}

\begin{DefT}{Extremum} \index{Extremum}
Un extremum est soit un maximum, soit un minimum.
\end{DefT}


\begin{DefT}{Majorant} \index{Extremum}
On dit qu'une fonction $f$ est majorée sur $I$ par $M$ lorsque pour tout réel $x \in I$, $f(x) \leq M$. $M$ est appelé le \textbf{majorant}.
\end{DefT}

\begin{DefT}{Minorant} \index{Minorant}
On dit qu'une fonction $f$ est majorée sur $I$ par $M$ lorsque pour tout réel $x \in I$, $f(x) \geq M$. $M$ est appelé le \textbf{minorant}.
\end{DefT}


\begin{DefT}{Fonction bornée} \index{Fonction!Bornée}
Une fonction est dite bornée sur un intervalle $I$ lorsqu'elle admet sur cet intervalle $I$ un majorant et un minorant.
\end{DefT}

\AD{1}{VF-25}






