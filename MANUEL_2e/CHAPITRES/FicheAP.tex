\documentclass[openany]{book}

%%%%%%%%%%%%%%%%%%%%%%%%%%%%%%%%%%%%%%%%%%%%%%%%%%%%%%%%%%%%%%%%%%%%%%%%%%%%
%% Pour passer du manuel au cahier :
%%  Dans le préambule choisir styleExercices ou styleCahier
%%  1 imprime / affiche le manuel  -----    0 affiche le cahier élève 
%%	Choisir le fihier ND-F1  ou ND-C1
%%%%%%%%%%%%%%%%%%%%%%%%%%%%%%%%%%%%%%%%%%%%%%%%%%%%%%%%%%%%%%%%%%%%%%%%%%%%


\input{../../../latex_preambule_style/preambule}
\input{../../../latex_preambule_style/styleCoursLycee}
\input{../../../latex_preambule_style/styleExercices}
%\input{../../latex_preambule_style/styleCahier}
\input{../../../latex_preambule_style/bas_de_page_Seconde}
\input{../../../latex_preambule_style/algobox}

%%%%%%%%%%%%%%%  Affichage ou impression  %%%%%%%%%%%%%%%%%%
\newcommand{\impress}[2]{
\ifthenelse{\equal{#1}{1}}  %   1 imprime / affiche sur livre  -----    0 affiche sur cahier 
{%condition vraie
#2
}% fin condition vraie
{%condition fausse
}% fin condition fausse
} % fin de la procédure
%%%%%%%%%%%%%%%  Affichage ou impression  %%%%%%%%%%%%%%%%%%

%%%%%%%%%%%%%%%%%%%%%%%%%%%%%%%%%%%%%%%%%%%%%%%%

\begin{document}

\begin{titre}[Fonctions et expressions algébriques]

\Titre{Calculs littéraux}{4}

\end{titre}


\begin{CpsCol}
\textbf{Utiliser des nombres pour calculer et résoudre des problèmes}
\begin{description}
\item[$\square$] Développer des expressions polynomiales simples
\item[$\square$] Factoriser des expressions polynomiales simples
\end{description}
\end{CpsCol}


\mini{
\begin{ThT}{Les identités remarquables\index{Identités remarquables}}
Pour tous nombres $a$ et $b$, 
\begin{description}
\item $(a+b)^2=a^2+2ab+b^2$
\item $(a-b)^2=a^2-2ab+b^2$
\item $(a-b)(a+b)=a^2-b^2$
\end{description}
\end{ThT}
}{
\begin{ThT}{Les racines carrées\index{Racines carrées}}
Pour tous nombres $a$ et $b$ positifs, 
\begin{description}
\item $\sqrt{ab}=\sqrt{a}\sqrt{b}$
\item $\sqrt{\frac{a}{b}}=\frac{\sqrt{a}}{\sqrt{b}}$, $b >0$
\end{description}
\end{ThT}
}

\begin{Rq}
Dans la propriété ci dessus, le membre de gauche est la forme factorisée\index{identités remarquables!forme factorisée} et dans celui de droite est la forme développée\index{identités remarquables!forme développée}.
\end{Rq}

\AD{1}{FEA-18}

\AD{1}{FEA-19}


\begin{minipage}{0.49\linewidth}
\AD{1}{FEA-20}

\AD{1}{FEA-51}
\end{minipage}
\hfill
\begin{minipage}{0.49\linewidth}
\AD{1}{FEA-22}
\end{minipage}

\begin{minipage}{0.49\linewidth}
\Exo{1}{FEA-21}

\AD{1}{FEA-24}

\Exo{1}{FEA-25}

\end{minipage}
\hfill
\begin{minipage}{0.49\linewidth}
\Exo{1}{FEA-23}

\AD{1}{FEA-26}

\Exo{1}{FEA-27}

\Exo{1}{FEA-28}
\end{minipage}


\begin{minipage}{0.49\linewidth}
\PO{1}{FEA-29}

\PO{1}{FEA-63}
\end{minipage}
\hfill
\begin{minipage}{0.49\linewidth}
\PO{1}{FEA-17}

\PO{1}{FEA-30}
\end{minipage}


\end{document}