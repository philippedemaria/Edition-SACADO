\begin{titre}[Statistiques descriptives]

\Titre{Médiane et quartiles}{4}
\end{titre}


\begin{CpsCol}
\textbf{Utiliser des nombres pour calculer et résoudre des problèmes}
\begin{description}
\item[$\square$] Calculer les caractéristiques d'une série définie par effectifs ou par fréquences
\end{description}
\end{CpsCol}


\begin{DefT}{Médiane}\index{Médiane}
La médiane d'une série statistique est une valeur $m$ du caractère qui partage la population en deux sous-ensembles
de même effectif.
\end{DefT}

\begin{Rq}
50\% de l'effectif total se situe en dessous de $m$ et 50\% au dessus de $m$.
\end{Rq}



\begin{Mt}
La liste des $N$ données est rangée dans l'ordre croissant.
\begin{description}
\item[•] Si la série est de taille impaire ($N= 2n + 1$), la médiane est la donnée de rang $n + 1$.
\item[•] Si la série est de taille paire ($N = 2n$), la médiane est la demi-somme des données de rang $n$ et de rang $n+1$.
\end{description}
\end{Mt}

\begin{minipage}{0.48\linewidth}
\EPC{1}{stat-16}{Calculer}
\end{minipage}
\hfill
\begin{minipage}{0.48\linewidth}
\EPC{1}{stat-20}{Chercher. Raisonner. Calculer}
\end{minipage}

\vspace{0.4cm}

Pour en connaître un peu plus sur une série statistique, on peut étudier la répartition des valeurs et plus
précisément leur dispersion. Il existe plusieurs indicateurs de dispersion. En seconde, on s'intéresse aux
quartiles. On divise alors la population en 4 groupes (quartiles) mais aussi en dix groupes (déciles).


\begin{DefT}{Premier quartile}\index{Quartile!Premier}
On appelle premier quartile noté $Q_1$ le plus petit nombre tel qu'au moins 25\% des valeurs de la série
soient inférieures ou égales à ce nombre.
\end{DefT}


\begin{DefT}{Troisième quartile}\index{Quartile!Troisième}
On appelle \textbf{troisième quartile} $Q_3$ le plus petit nombre tel qu'au moins 75\% des valeurs de la série
soient inférieures ou égales à ce nombre.
\end{DefT}


\begin{Rq}
\begin{description}
\item[•] $Q_1$ et $Q_3$ sont obligatoirement des valeurs de la série étudiée.
\item[•] $Q_1$ et $Q_3$ sont les médianes des sous-séries obtenues.
\end{description}
\end{Rq}


\begin{Mt}
Soit une série ordonnée de $N$ valeurs.
\begin{description}
\item[•] Pour déterminer le \textbf{rang} du premier quartile, on calcule $\frac{N}{4}$

Remarque : dans le cas où $\frac{N}{4}$ n'est pas un entier, on prend le plus petit entier supérieur à $\frac{N}{4}$.

\item[•] Pour déterminer le \textbf{rang} du troisième quartile, on calcule $\frac{3N}{4}$

Remarque : dans le cas où $\frac{3N}{4}$ n'est pas un entier, on prend le plus petit entier supérieur à $\frac{3N}{4}$.
\end{description}
\end{Mt}


\EPC{0}{stat-0}{Représenter. Calculer}






\mini{
\EPC{1}{stat-8bis}{Raisonner. Calculer}

\EPC{1}{stat-7}{Raisonner. Calculer}

}{
\EPC{1}{stat-22}{Représenter. Calculer}

\EPC{1}{stat-1}{Représenter}
}

\mini{
\EPC{1}{stat-30}{Représenter. Calculer}
}{
\EPC{1}{stat-31}{Représenter. Calculer}
}