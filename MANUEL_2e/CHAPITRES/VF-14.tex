
$[AC]$ est un segment de longueur $10 cm$ et $B$ est le point de $[AC]$ tel que $AB = 4 cm$. $M$ est un point du
demi-cercle de diamètre $[AC]$. On note $x$ la longueur de l'arc $AM$. On considère la fonction $f$ qui à $x$
associe l'aire en $cm^2$ du triangle $ABM$.
\begin{enumerate}
\item Réaliser la figure dans la fenêtre \texttt{Graphique} avec le logiciel de géométrie dynamique Geogebra.
\begin{Rq}
Pour construire le triangle $ABM$, on utilisera l'outil \texttt{Polygone} et on prendra soin de renommer l'aire \texttt{aire}. 
\end{Rq}
\item Ouvrir la fenêtre \texttt{Graphique 2} : Affichage > Graphique 2. Le nom \textbf{Graphique 2} apparait en gras ce qui signifie que cette fenêtre contient les éléments construits.
\begin{enumerate}
\item Construire le point $N$ de coordonnées ($x$, \texttt{aire}). Pour construire ce point, taper dans la barre de saisie $N =$($x$, \texttt{aire})
\item Sélectionner le point $N$ et avec le clic droit de la souris appliquer \texttt{Trace active}.
\item Sélectionner le point M dans la fenêtre \texttt{Graphique}, \textbf{Graphique} apparait en gras. Faire varier le point $M$. Regarder sur la fenêtre \textbf{Graphique 2} la trace du point $N$.
\end{enumerate}
\item Conjectures :
	\begin{enumerate}
		\item Conjecturer les variations de la fonction $f$.
		\item $f$ admet elle un maximum? Un minimum ?
	\end{enumerate}
\end{enumerate}

\begin{CdP}
Construire le point $A$ à droite du point $C$.  
\end{CdP}
\begin{CdP}
Pour renommer un point, sélectionner le point avec le clic droit.  
\end{CdP}
\begin{CdP}
Pour construire le point $M$ utiliser l'outil Arc de cercle (centre-2points) .  
\end{CdP}