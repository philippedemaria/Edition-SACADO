\begin{titre}[Fonctions et expressions algébriques]

\Titre{Ensembles de nombres}{4}
\end{titre}


\begin{CpsCol}
\textbf{Utiliser des nombres pour calculer et résoudre des problèmes}
\begin{description}
\item[$\square$] Connaitre les ensembles de nombres
\item[$\square$] Identifier l'ensemble d'un nombre
\end{description}
\end{CpsCol}




\begin{DefT}{Ensemble des réels}
L'ensemble de tous les nombres connus en Seconde est appelé ensemble des nombres réels, noté $\R$.
\end{DefT}

\begin{DefT}{Ensemble de nombres}\index{Ensemble de nombres!Réels $\R$}
Les autres ensembles de nombres, inclus dans $\R$.
\begin{enumerate}
\item On appelle \textbf{entiers naturels} les nombres : 0 ; 1 ; 2 ; 3 . . . Leur ensemble est noté $\N$.\index{Ensemble de nombres! Entiers naturels $\N$}
On a donc : $\N =  \lbrace 0 ; 1 ; 2 ; 3 \cdots \rbrace $
\item  On appelle \textbf{entiers relatifs} les nombres entiers naturels et leurs symétriques par rapport à 0. Leur ensemble est noté $\Z$.\index{Ensemble de nombres! Entiers $\Z$}
On a donc : $\Z = \lbrace \cdots -3 ; -2 ; -1 ; 0 ; 1 ; 2 ; 3  \cdots  \rbrace$
\item  On appelle \textbf{nombres rationnels} les nombres de la forme $\frac{a}{b}$, $a$ et $b$ entiers et $b$ non nuls.  Leur ensemble est noté $\Q$. \index{Ensemble de nombres! Rationnels $\Q$}
On a donc : $\Q = \lbrace \cdots \frac{5}{3} ; -\frac{5}{7} ; -\frac{13}{22} \cdots   \rbrace$
\item Par construction, $\N$ est inclus dans $\Z$  est inclus dans $\D$  est inclus dans $\Q$  est inclus dans $\R$. Ces ensembles sont dits "emboités".
\item  On peut représenter l'ensemble des réels sur une droite graduée.
\begin{center}
\begin{tikzpicture}[line cap=round,line join=round,>=triangle 45,x=1.0cm,y=1.0cm]
\draw[->,color=black] (-4.36,0.) -- (10.66,0.);
\foreach \x in {-4.,-3.,-2.,-1.,1.,2.,3.,4.,5.,6.,7.,8.,9.,10.}
\draw[shift={(\x,0)},color=black] (0pt,2pt) -- (0pt,-2pt) node[below] {\footnotesize $\x$};
\draw[color=black] (0pt,-10pt) node[right] {\footnotesize $0$};
\clip(-4.36,-0.5) rectangle (10.66,0.5);
\end{tikzpicture}
 \end{center} 
\end{enumerate}
\end{DefT}


\Exo{1}{FEA-2}



\begin{Nt}
Soit $M$ un point d'abscisse $a$, sur la droite graduée.\\
On note $\vert a\vert$ la distance de $M$ à $O$.

Ainsi, on peut établir que $\vert -a \vert = \vert a \vert$. En effet, deux nombres opposés sont à la même distance de l'origine.
\end{Nt}






\begin{Nt}
\begin{description}
\item[•] On appelle $\R^+$ l'ensemble des réels positifs, $\R^-$ l'ensemble des réels négatifs.
\item[•] On utilise une étoile pour enlever 0 d'un ensemble. $\R^*$ l'ensemble des réels non nuls.
\end{description}
\end{Nt}





\begin{DefT}{Nombres décimaux}
Les \textbf{nombres décimaux} sont des nombres rationnels dont le dénominateur est une puissance de 2, de 5 ou de 10 ou un produit de puissances de ces nombres.
\end{DefT}

\begin{Ex}
\begin{description}
\item[•] $A=\frac{3}{25}$ est un nombre décimal car $25 = 5^2$, 25 est donc une puissance de 5.
\item[•] $B=\frac{13}{20}$ est un nombre décimal car $20 = 2^2 \times 5$, 20 est donc un produit d'une puissance de 2 et de 5.
\end{description}
\end{Ex}

\begin{Att}
$A=\frac{7}{15}$ n'est pas un nombre décimal. $15=3 \times 5$ n'est pas une puissance de 5 mais un multiple de 5 !
\end{Att}



\begin{DefT}{Nombre décimal périodique}
Le nombre $a_0,\underline{a_1a_2a_3}$ est un nombre décimal périodique de période $a_1a_2a_3$. Les chiffres $a_1$, $a_2$, $a_3$ se répètent indéfiniment.
\end{DefT}
\begin{Dem}

On démontre que 
\begin{description}
\item[•] $0,\underline{12}$ est un nombre rationnel.
\item[•] $0,\underline{9}=1$.
\end{description}
\end{Dem}

\PESP{https://fr.wikipedia.org/wiki/D\%C3\%A9veloppement\_d\%C3\%A9cimal\_p\%C3\%A9riodique}

\AV{https://www.youtube.com/watch?v=N_cDA6tF-40}{Conjecture de Cantor}

%
%
%
%\mini{
%\AD{1}{FEA-3}
%
%\AD{1}{FEA-4}
%
%\AD{1}{FEA-6}
%
%\PO{1}{FEA-7}
%}{
%\PO{1}{FEA-8}
%
%
%\begin{DTL}
%
%
\begin{enumerate}
\item Pré-requis à démontrer.
	\begin{enumerate}
	\item Démontrer que "si un nombre $a$ est pair alors $a^2$ est pair".
	\item Démontrer que "si un nombre $a$ est impair alors $a^2$ est impair".
	\item Justifier que ces deux propositions se traduisent par "un nombre $a$ est pair si et seulement si $a^2$ est pair".
	\end{enumerate}
\item Irrationalité de $\sqrt{2}$
	\begin{description}
	 \item En supposant que $\sqrt{2}=\frac{p}{q}$, $\frac{p}{q}$ irréductible, démontrer que $\sqrt{2}$ est irrationnel.
	\end{description}
\end{enumerate}
%
%\end{DTL}
%}

\begin{DefT}{Inclusion}\index{Ensemble!Inclusion}
Un ensemble $A$ est inclus dans un ensemble $B$ lorsque tous les éléments de $A$ sont contenus dans $B$. On note $A \subset B$.
\end{DefT}

\AD{1}{FEA-15}

\begin{Rq}
\begin{description}
\item[•] Un ensemble \textbf{est inclus dans} un ensemble. $A$ et $B$ deux ensembles : $A \subset B$.
\item[•] Un élément \textbf{appartient à} un ensemble. $x$ un élément et $A$ un ensemble : $x \in B$.
\end{description}
\end{Rq}


\begin{DefT}{Complémentaire}\index{Ensemble!Complémentaire}
Soit $\Omega$ un ensemble contenant un ensemble $A$. On appelle complémentaire de $A$ dans $\Omega$, tous les éléments de $\Omega$ qui n'appartiennent pas à $A$.
\end{DefT}


%
%\begin{Log}\index{Démonstration par l'absurde}\index{Contraposée}
%\begin{description}[leftmargin=*]
%\item[•] La \textbf{contraposée} d'une implication "si A alors B" est l'implication : "si non(B) alors non(A)". Pour démontrer que "si A alors B" , on peut démonter que : "si non(B) alors non(A)".  En langage symbolique, on écrit  : $A \Longrightarrow B \Longleftrightarrow \rceil B \Longrightarrow \rceil A$ 
%\item[•] Une démonstration est appelée \textbf{démonstration par l'absurde} lorsqu'on démontre que la supposition posée \textit{a priori} mène à une absurdité.
%\end{description}
%\end{Log}
