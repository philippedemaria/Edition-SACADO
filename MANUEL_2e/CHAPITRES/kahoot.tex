\documentclass[20pt]{article}

\input{../../../latex_preambule_style/preambule}
\input{../../../latex_preambule_style/styleCourslycee}
\input{../../../latex_preambule_style/styleExercices}
%\input{../../latex_preambule_style/styleCahier}
\input{../../../latex_preambule_style/bas_de_page_seconde}
\input{../../../latex_preambule_style/algobox}



%%%%%%%%%%%%%%%  Affichage ou impression  %%%%%%%%%%%%%%%%%%
\newcommand{\impress}[2]{
\ifthenelse{\equal{#1}{1}}  %   1 imprime / affiche  -----    0 n'affiche pas
{%condition vraie
#2
}% fin condition vraie
{%condition fausse
}% fin condition fausse
} % fin de la procédure
%%%%%%%%%%%%%%%  Affichage ou impression  %%%%%%%%%%%%%%%%%%



%%%%%%%%%%%%%%%  Indentation  %%%%%%%%%%%%%%%%%%
\parindent=0pt
%%%%%%%%%%%%%%%%%%%%%%%%%%%%%%%%%%%%%%%%%%%%%%%%



\begin{document}

Soit A(3;-2) et B(-1;5); Les coordonnées du vecteur $\overrightarrow{AB}$ sont

\vspace{0.4cm}

$\overrightarrow{AB}(x_B-x_A;y_B-y_A)$, $\overrightarrow{AB}(-1-3;(5-(-2))$, $\overrightarrow{AB}(-4;7)$

\vspace{1cm}
\hrule 
\vspace{0.4cm}

A, B et C sont trois points distincts et non alignés du plan.

On donne les vecteurs $\overrightarrow{AB}(3;2)$ et $\overrightarrow{DC}(6;4)$. Que dire ? 
\vspace{0.4cm}

$dét\left(\overrightarrow{AB},\overrightarrow{DC}\right)= 3 \times 4 - 6 \times 2  =0$

Les vecteurs sont colinéaires ou les droites $(AB)$ et $(CD)$ sont paralèles.

\vspace{1cm}
\hrule 
\vspace{0.4cm} 

Les vecteurs $\overrightarrow{AB}(k;2)$ et $\overrightarrow{DC}(5;-3)$ sont colinéaires pour $k=$

\vspace{0.4cm}

Les vecteurs $\overrightarrow{AB}(k;2)$ et $\overrightarrow{DC}(5;-3)$ sont colinéaires donc 
$dét(\overrightarrow{AB},\overrightarrow{DC})=$ \begin{tabular}{|c c|}
$k$ & $5$ \\ 
$2$ & $-3$ \\
\end{tabular} $ -3k-10 =0 \Longleftrightarrow  k = -\frac{10}{3} $

\vspace{1cm}
\hrule 
\vspace{0.4cm}

On a : 
$\overrightarrow{AB}= \overrightarrow{AC} - \overrightarrow{\ldots C}$.

$\overrightarrow{AB}= \overrightarrow{AC} + \overrightarrow{ C\ldots}= \overrightarrow{AC} + \overrightarrow{ CB}$.

Le point est donc \textbf{B}.

 


\vspace{1cm}
\hrule 
\vspace{0.4cm}

$\overrightarrow{AB}= 2\overrightarrow{u} - 3\overrightarrow{v}$ et $\overrightarrow{DC}= -6\overrightarrow{u} + k\overrightarrow{v}$  

$\overrightarrow{AB}$ et $\overrightarrow{DC}$ sont colinéaires lorsque $k =$

\vspace{0.4cm}

Considérons la base $(\vec u ; \vec v)$. Dans cette base, $\overrightarrow{AB}(2; - 3)$ et $\overrightarrow{DC}(-6;k)$.

$dét(\overrightarrow{AB},\overrightarrow{DC}) = 2k-18 = 0$ donc $k = 9$

\vspace{1cm}
\hrule 
\vspace{0.4cm} 

On donne : $\overrightarrow{OC}= \frac{1}{2}\overrightarrow{OA}+ \frac{1}{4}\overrightarrow{OB}$ et $\overrightarrow{OD}= 2\overrightarrow{OA}+ k\overrightarrow{OB}$.

$O$,$C$ et $D$ sont alignés lorsque $k$ = 

\vspace{0.4cm}

Considérons la base $(\overrightarrow{OA} ; \overrightarrow{OB})$. Dans cette base, $\overrightarrow{OC}\left(\frac{1}{2}; \frac{1}{4}\right)$ et $\overrightarrow{DC}\left(2;k\right)$.

$dét\left(\overrightarrow{OC},\overrightarrow{OD}\right) = \frac{1}{2}k- \frac{1}{4}\times 2 = \frac{1}{2}k-\frac{1}{2} =0$ 

$\Longleftrightarrow   k = 1$


\end{document}
