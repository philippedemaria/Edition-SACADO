\begin{titre}[Fonctions de référence]

\Titre{Les fonctions homographiques}{4}
\end{titre}


\begin{CpsCol}
\textbf{Variations de fonctions}
\begin{description}
\item[$\square$] Connaitre les variations des fonctions homographiques
\item[$\square$] Déterminer l'ensemble de définition d'une fonction homographique
\item[$\square$] Transformer des fonctions rationnelles simples
\end{description}
\end{CpsCol}


\Rec{1}{FR-35}


\begin{DefT}{Ensemble de définition} \index{Ensemble de définition}
L'ensemble de définition d'une fonction $f$ est l'ensemble de tous les réels qui possède une image par $f$.
\end{DefT}


\begin{DefT}{Fonction homographique} \index{Fonctions ! Homographique}
Une \textbf{fonction homographique} est une fonction définie sur $\R-\left\lbrace \frac{-d}{c} \right\rbrace$ par $f(x) = \frac{ax+b}{cx+d}$, où $a$, $b$, $c$ et $d$ des réels et $c \neq 0$.
\end{DefT}


\Rec{1}{FR-33}

\Exo{1}{FR-34}

\App{1}{FR-38}


\App{1}{FR-39}


