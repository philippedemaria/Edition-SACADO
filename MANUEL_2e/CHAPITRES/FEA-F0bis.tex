\chapter{Distance entre deux nombres}
{https://sacado.xyz/qcm/parcours_show_course/0/117129}
{


 \begin{CpsCol}
\textbf{Les savoir-faire du parcours}
 \begin{itemize}
 \item \textbf{Utiliser des nombres pour calculer et résoudre des problèmes}
\item[$\square$] Notation $\vert a\vert$. Distance entre deux nombres réels
\item[$\square$] Représentation de l'intervalle $[a-r;a+r]$ puis caractérisation par la condition $\vert x-	a\vert \leq r$.
\item[$\square$] Donner un encadrement, d’amplitude donnée, d’un nombre réel par des décimaux.
 \end{itemize}
 \end{CpsCol}

\begin{His}
\textbf{Georg Ferdinand Ludwig Philipp Cantor} (3 mars 1845, Saint-Pétersbourg – 6 janvier 1918, Halle) est un mathématicien allemand, connu pour être le créateur de la théorie des ensembles. Il établit l'importance de la bijection entre les ensembles, définit les ensembles infinis et les ensembles bien ordonnés. Il prouva également que les nombres réels sont « plus nombreux » que les entiers naturels. En fait, le théorème de Cantor implique l'existence d'une « infinité d'infinis ». Il définit les nombres cardinaux, les nombres ordinaux et leur arithmétique. Le travail de Cantor est d'un grand intérêt philosophique...https://fr.wikipedia.org/wiki/Georg\_Cantor
\end{His}


\begin{ExoDec}{Chercher.}{1234}{1}{0}{0}{0}

\end{ExoDec}

}

\begin{pageCours}
 

\section{Distance à 0}

\begin{DefT}{Valeur absolue}\index{Valeur absolue}
Soit $M$ un point d'abscisse $x$ sur la droite graduée d'origine $O$ d'abscisse 0 et $x$ un réel.\\
On note $\vert x \vert$ la distance de $M$ à $O$.\\
L'écriture $\vert x\vert$ est appelée \textbf{valeur absolue} de $x$. On peut alors écrite : $OM = d(O,M)=\vert x\vert$
\end{DefT}

 

\begin{Rq}
 La valeur absolue d'un nombre est un nombre positif.
\end{Rq}


\begin{Ex}
Sur cet exemple, $OA = OB = 4$. On peut écrire que $\vert -4 \vert = \vert 4 \vert = 4$. 

\definecolor{xfqqff}{rgb}{0.4980392156862745,0.,1.}
\definecolor{ffxfqq}{rgb}{1.,0.4980392156862745,0.}
\begin{tikzpicture}[line cap=round,line join=round,>=triangle 45,x=1.0cm,y=1.0cm]
\clip(-3.54,1.14) rectangle (7.68,2.56);
\draw [ line width=1.pt,color=xfqqff,domain=-3.54:7.68] plot(\x,{(--16.-0.*\x)/8.});
\draw (1.84,1.86) node[anchor=north west] {0};
\draw (2.84,1.86) node[anchor=north west] {1};
\begin{scriptsize}
\draw [color=ffxfqq] (-2.,2.)-- ++(-2.5pt,0 pt) -- ++(5.0pt,0 pt) ++(-2.5pt,-2.5pt) -- ++(0 pt,5.0pt);
\draw[color=ffxfqq] (-1.86,2.20) node {${\large  A}$};
\draw [color=ffxfqq] (6.,2.)-- ++(-2.5pt,0 pt) -- ++(5.0pt,0 pt) ++(-2.5pt,-2.5pt) -- ++(0 pt,5.0pt);
\draw[color=ffxfqq] (6.14,2.20) node {${\large B}$};
\draw [color=xfqqff] (2.02,2.)-- ++(-2.5pt,0 pt) -- ++(5.0pt,0 pt) ++(-2.5pt,-2.5pt) -- ++(0 pt,5.0pt);
\draw[color=xfqqff] (2.04,2.20) node {$O$};
\draw [color=xfqqff] (3.,2.)-- ++(-2.5pt,0 pt) -- ++(5.0pt,0 pt) ++(-2.5pt,-2.5pt) -- ++(0 pt,5.0pt);
\draw [color=xfqqff] (4.,2.)-- ++(-2.5pt,0 pt) -- ++(5.0pt,0 pt) ++(-2.5pt,-2.5pt) -- ++(0 pt,5.0pt);
\draw [color=xfqqff] (5.,2.)-- ++(-2.5pt,0 pt) -- ++(5.0pt,0 pt) ++(-2.5pt,-2.5pt) -- ++(0 pt,5.0pt);
\draw [color=xfqqff] (1.,2.)-- ++(-2.5pt,0 pt) -- ++(5.0pt,0 pt) ++(-2.5pt,-2.5pt) -- ++(0 pt,5.0pt);
\draw [color=xfqqff] (0.,2.)-- ++(-2.5pt,0 pt) -- ++(5.0pt,0 pt) ++(-2.5pt,-2.5pt) -- ++(0 pt,5.0pt);
\draw [color=xfqqff] (-1.,2.)-- ++(-2.5pt,0 pt) -- ++(5.0pt,0 pt) ++(-2.5pt,-2.5pt) -- ++(0 pt,5.0pt);
\draw [color=xfqqff] (-3.,2.)-- ++(-2.5pt,0 pt) -- ++(5.0pt,0 pt) ++(-2.5pt,-2.5pt) -- ++(0 pt,5.0pt);
\draw [color=xfqqff] (7.,2.)-- ++(-2.5pt,0 pt) -- ++(5.0pt,0 pt) ++(-2.5pt,-2.5pt) -- ++(0 pt,5.0pt);
\end{scriptsize}
\end{tikzpicture}

\end{Ex}



\section{Distance entre deux nombres}

\begin{DefT}{Généralisation. Distance entre deux nombres}\index{Distance entre deux nombres}
Soit $a$ et $b$ deux réels.\\
La distance entre $a$ et $b$ est égale à $\vert b-a\vert=\vert a-b\vert$. On peut écrire $d(a,b)=\vert b-a\vert=\vert a-b\vert$.
\end{DefT}

\begin{DefT}{Généralisation. Distance entre deux points}\index{Distance entre deux points}
Soit $a$ et $x$ deux réels.

Soit $A$ le point d'abscisse $a$ et $M$ le point d'abscisse $x$. $AM = d(a,x) = \vert x-a\vert$
\end{DefT}

 

\section{Appartenance d'un point à un segment}

\begin{Th}
$A$ et $B$ deux points d'abscisse respective $a$ et $b$ sur la droite graduée. 

Alors le milieu $I$ du segment $[AB]$ a pour abscisse $m=\frac{a+b}{2}$
\end{Th}

\begin{ThT}{Appartenance d'un point à un segment}
Soit $[AB]$ un segment, $I$ le milieu de $[AB]$ d'abscisse $m$ et $r = IA = IB$.\\
Le segment $[AB]$ est l'ensemble des points $M$ d'abscisse $x$ de la droite graduée tels que $\vert x- m \vert \leq r$.
\end{ThT}
 
\end{pageCours}


\begin{pageAD}

\Sf{fdfbd}

\begin{ExoCad}{Représenter. Chercher.}{1234}{2}{0}{0}{0}{0}

\paragraph{Parie A}

\begin{enumerate}
\item Tracer une droite graduée d'unité 3 cm.
\item Placer les points $M$ tel que $\vert OM \vert = 1$. On notera $M_1$ et $M_2$.
\item Placer les points $A$ tel que $\vert OA \vert = \frac{2}{3}$. On notera $A_1$ et $A_2$.
\end{enumerate}

Remarque : Dans ce type de question, on ne spécifie pas les 2 points que l'élève doit trouver ! C'est à l'élève de savoir ce qu'il doit faire. Ce type d'exercice se pose alors comme :

\paragraph{Parie B}

\begin{enumerate}
\item Tracer une droite graduée d'unité 2 cm.
\item Placer le point $A$ d'abscisse $a$ tel que $\vert a \vert = 2$.
\item Placer le point $B$ d'abscisse $b$ tel que $\vert b \vert = \frac{5}{2}$.
\item Placer le point $M$ d'abscisse $x$ tel que $OM = \frac{3}{4}$.
\end{enumerate}
\end{ExoCad}

\begin{ExoCad}{Représenter. Chercher.}{1234}{2}{0}{0}{0}{0}
 
Soit $A$ et $B$ d'abscisse respective $4$ et $-2$.

\begin{enumerate}
\item Le point $I$ est le milieu de $[AB]$. Quelle est l'abscisse du point $I$ ? 
\item Soit $M$ le point d'abscisse $x$ de la droite $(AB)$. Calculer $IM$. 
\item Compléter le tableau suivant.

\begin{tabular}{|c|p{1.5cm}|c|p{1.5cm}|c|p{1.5cm}|}
\hline 
Abscisses de $M$ & $IM$ & Abscisses de $M$ & $IM$ & Abscisses de $M$ & $IM$ \\ 
\hline 
$1$ & & $-6$ &  & $-1$ & \\ 
\hline 
$-2$ & & $3$ &  & $1$ & \\ 
\hline 
$5$ & & $0$ &  & $4$ & \\ 
\hline 
\end{tabular}

\item  A quelle condition le point $M$ appartient-il au segment $[AB]$ ?

\end{enumerate}

On pourra se rendre à la page Se rendre à la page : \url{https://www.geogebra.org/m/jsvqnzbq} pour visualiser la situation.
\end{ExoCad}


\begin{ExoCad}{Représenter.}{1234}{2}{0}{0}{0}{0}

Soit $x$ un réel donné. Traduire chaque valeur absolue comme distance entre deux réels.
\begin{enumerate}
\item $\vert x - 5 \vert$
\item $\vert 3 - x \vert$
\item $\vert x + 1 \vert$
\item $\vert -2 - x \vert$
\item $\vert 4 - 2x \vert$
\end{enumerate}  
 
\end{ExoCad}


\begin{ExoCad}{Représenter. Raisonner.}{1234}{2}{0}{0}{0}{0}

Représenter et résoudre les équations suivantes :
\begin{enumerate}
\item $d((x;4)=3$
\item $\vert x + 3 \vert = 1$
\item $\vert \frac{1}{2} - x \vert = \frac{3}{2}$
\item $\vert x - \frac{1}{3} \vert = \frac{5}{6}$
\item $\vert 2x - 1 \vert = 4$
\end{enumerate}  
 
\end{ExoCad}

 


\begin{ExoCad}{Représenter. Chercher.}{1234}{2}{0}{0}{0}{0}
 
Soit $A$ et $B$ d'abscisse respective $4$ et $-2$.

\begin{enumerate}
\item Le point $I$ est le milieu de $[AB]$. Quelle est l'abscisse du point $I$ ? 
\item Soit $M$ le point d'abscisse $x$ de la droite $(AB)$. Calculer $IM$. 
\item Compléter le tableau suivant.

\begin{tabular}{|c|p{1.5cm}|c|p{1.5cm}|c|p{1.5cm}|}
\hline 
Abscisses de $M$ & $IM$ & Abscisses de $M$ & $IM$ & Abscisses de $M$ & $IM$ \\ 
\hline 
$1$ & & $-6$ &  & $-1$ & \\ 
\hline 
$-2$ & & $3$ &  & $1$ & \\ 
\hline 
$5$ & & $0$ &  & $4$ & \\ 
\hline 
\end{tabular}

\item  A quelle condition le point $M$ appartient-il au segment $[AB]$ ?

\end{enumerate}

On pourra se rendre à la page Se rendre à la page : \url{https://www.geogebra.org/m/jsvqnzbq} pour visualiser la situation. 
\end{ExoCad}
 

\begin{ExoCad}{Chercher. Communiquer.}{1234}{2}{0}{0}{0}{0}

 
\begin{enumerate}
\item Soit $A$ et $B$ d'abscisse respective $3$ et $-1$. Déterminer le rayon de l'intervalle $[AB]$.
\item Représenter le segment $[AB]$ par un intervalle puis par une inégalité.

\item Représenter sur la droite graduée le segment $[AB]$.
 
\end{enumerate}
  
\end{ExoCad}


 
 
\begin{ExoCad}{Représenter. Chercher.}{1234}{2}{0}{0}{0}{0}

On donne les segments $[AB]$ et $[EF]$ représentés ci-dessous.

\definecolor{ttqqqq}{rgb}{0.2,0.,0.}
\definecolor{qqzzff}{rgb}{0.,0.6,1.}
\definecolor{qqzzcc}{rgb}{0.,0.6,0.8}
\begin{tikzpicture}[line cap=round,line join=round,>=triangle 45,x=1.0cm,y=1.0cm]
\begin{axis}[
x=1.0cm,y=1.0cm,
axis lines=middle,
xmin=-2.200000000000002,
xmax=8.480000000000008,
ymin=-0.8000000000000048,
ymax=0.7799999999999957,
xtick={-2.0,-1.0,...,8.0},
ytick={-0.0,1.0,...,0.0},]
\clip(-2.2,-0.8) rectangle (8.48,0.78);
\draw [line width=2.pt,color=qqzzff] (-2.,0.)-- (4.,0.);
\draw [line width=2.pt] (5.,0.)-- (8.,0.);
\begin{scriptsize}
\draw [color=qqzzcc] (-2.,0.)-- ++(-2.5pt,0 pt) -- ++(5.0pt,0 pt) ++(-2.5pt,-2.5pt) -- ++(0 pt,5.0pt);
\draw[color=qqzzcc] (-1.86,0.37) node {$A$};
\draw [color=qqzzcc] (4.,0.)-- ++(-2.5pt,0 pt) -- ++(5.0pt,0 pt) ++(-2.5pt,-2.5pt) -- ++(0 pt,5.0pt);
\draw[color=qqzzcc] (3.96,0.37) node {$B$};
\draw [color=ttqqqq] (5.,0.)-- ++(-2.5pt,0 pt) -- ++(5.0pt,0 pt) ++(-2.5pt,-2.5pt) -- ++(0 pt,5.0pt);
\draw[color=ttqqqq] (5.14,0.37) node {$E$};
\draw [color=ttqqqq] (8.,0.)-- ++(-2.5pt,0 pt) -- ++(5.0pt,0 pt) ++(-2.5pt,-2.5pt) -- ++(0 pt,5.0pt);
\draw[color=ttqqqq] (8.14,0.37) node {$F$};
\end{scriptsize}
\end{axis}
\end{tikzpicture}


\begin{enumerate}
\item 
	\begin{enumerate}
		\item Déterminer le rayon de l'intervalle $[AB]$.
		\item Représenter $[AB]$ par une inégalité.
	\end{enumerate}
\item 
	\begin{enumerate}
		\item Déterminer le rayon de l'intervalle $[EF]$.
		\item Représenter $[EF]$ par une inégalité.
	\end{enumerate}
\end{enumerate}
 
\end{ExoCad}

\begin{ExoCad}{Représenter. Chercher.}{1234}{2}{0}{0}{0}{0}

Soit $x$ un réel.   


\begin{enumerate}
	\item Déterminer puis représenter l'ensemble des points $M$ d'abscisse $x$ tel que $\vert x- 3 \vert \leq 3$.
	\item Déterminer puis représenter l'ensemble des points $M$ d'abscisse $x$ tel que $\vert x+4 \vert \leq 1$.
	\item Déterminer puis représenter l'ensemble des points $M$ d'abscisse $x$ tel que $\vert x+\frac{2}{3} \vert \leq 4$.
\end{enumerate}
 
\end{ExoCad}

\begin{ExoCad}{Représenter. Chercher.}{1234}{2}{0}{0}{0}{0}

Soit $x$ un réel.   


\begin{enumerate}
	\item Déterminer puis représenter l'ensemble des points $M$ d'abscisse $x$ tel que $\vert x - \frac{4}{5} \vert \leq \frac{1}{2}$.
    \item Déterminer puis représenter l'ensemble des points $M$ d'abscisse $x$ tel que $\vert x- \pi \vert \leq 1$.
	\item Déterminer puis représenter l'ensemble des points $M$ d'abscisse $x$ tel que $\vert x - \sqrt{2}  \vert \leq  1$.
	\item Écrire à l'aide d'une double inégalité puis représenter  l'ensemble tel  que $\vert x + \frac{2}{3} \vert \leq 5$.
	\item Écrire à l'aide d'une double inégalité puis représenter  l'ensemble tel  que $\vert x+ \pi \vert \leq 10^{-1}$.	
\end{enumerate}
 
\end{ExoCad}

\begin{ExoCad}{Représenter. Chercher.}{1234}{2}{0}{0}{0}{0}

\begin{enumerate}
\item On s'intéresse à $\frac{3}{7}$.
\begin{enumerate}
	\item 1 est-elle une valeur approchée de $\frac{3}{7}$ à $ 10^{-1}$ près ?
	\item Déterminer une valeur approchée $a$ à $ 10^{-1}$ près de $\frac{3}{7}$.	
\end{enumerate}

\item
On s'intéresse à $\sqrt{10}$.
\begin{enumerate}
	\item Déterminer un encadrement de $\sqrt{10}$ à $ 10^{-2}$ près .
	\item Déterminer à la calculatrice  une valeur approchée de  $\sqrt{10}$.	
\end{enumerate}
\end{enumerate}
  
\end{ExoCad}


\end{pageAD}

%%%%%%%%%%%%%%%%%%%%%%%%%%%%%%%%%%%%%%%%%%%%%%%%%%%%%%%%%%%%%%%%%%%%%%%%%%%%%%%%%%%%%%%%%%%%%%%%%%%%%%%%%%%%%%%%%%%%%%%%%%%
%%%%%%%%%%%%%%%%%%%%%%%%%%%%%%%%%%%%%%%%%%%%%%%%%%%%%%%%%%%%%%%%%%%%%%%%%%%%%%%%%%%%%%%%%%%%%%%%%%%%%%%%%%%%%%%%%%%%%%%%%%%
%%%%%%%%%%%%%%%              pageParcoursu                         %%%%%%%%%%%%%%%%%%%%%%%%%%%%%%%%%%%%%%%%%%%%%%%%%%%%%%%%
%%%%%%%%%%%%%%%%%%%%%%%%%%%%%%%%%%%%%%%%%%%%%%%%%%%%%%%%%%%%%%%%%%%%%%%%%%%%%%%%%%%%%%%%%%%%%%%%%%%%%%%%%%%%%%%%%%%%%%%%%%%
%%%%%%%%%%%%%%%%%%%%%%%%%%%%%%%%%%%%%%%%%%%%%%%%%%%%%%%%%%%%%%%%%%%%%%%%%%%%%%%%%%%%%%%%%%%%%%%%%%%%%%%%%%%%%%%%%%%%%%%%%%%
\begin{pageParcoursu}

\begin{ExoCu}{Représenter. Chercher.}{1234}{2}{0}{0}{0}{0}
 
\end{ExoCu}

\begin{ExoCu}{Représenter. Chercher.}{1234}{2}{0}{0}{0}{0}
 
\end{ExoCu}

\begin{ExoCu}{Représenter. Chercher.}{1234}{2}{0}{0}{0}{0}
 
\end{ExoCu}


\end{pageParcoursu}
%%%%%%%%%%%%%%%%%%%%%%%%%%%%%%%%%%%%%%%%%%%%%%%%%%%%%%%%%%%%%%%%%%%%%%%%%%%%%%%%%%%%%%%%%%%%%%%%%%%%%%%%%%%%%%%%%%%%%%%%%%%
%%%%%%%%%%%%%%%%%%%%%%%%%%%%%%%%%%%%%%%%%%%%%%%%%%%%%%%%%%%%%%%%%%%%%%%%%%%%%%%%%%%%%%%%%%%%%%%%%%%%%%%%%%%%%%%%%%%%%%%%%%%
%%%%%%%%%%%%%%%              pageParcoursd                    %%%%%%%%%%%%%%%%%%%%%%%%%%%%%%%%%%%%%%%%%%%%%%%%%%%%%%%%%%%%%
%%%%%%%%%%%%%%%%%%%%%%%%%%%%%%%%%%%%%%%%%%%%%%%%%%%%%%%%%%%%%%%%%%%%%%%%%%%%%%%%%%%%%%%%%%%%%%%%%%%%%%%%%%%%%%%%%%%%%%%%%%%
%%%%%%%%%%%%%%%%%%%%%%%%%%%%%%%%%%%%%%%%%%%%%%%%%%%%%%%%%%%%%%%%%%%%%%%%%%%%%%%%%%%%%%%%%%%%%%%%%%%%%%%%%%%%%%%%%%%%%%%%%%%

\begin{pageParcoursd}
 

\begin{ExoCd}{Chercher. Communiquer.}{1234}{2}{0}{0}{0}{0}

 
\begin{enumerate}
\item Soit $A$ et $B$ d'abscisse respective $3$ et $-1$. Déterminer le rayon de l'intervalle $[AB]$.
\item Représenter le segment $[AB]$ par un intervalle puis par une inégalité.

\item Représenter sur la droite graduée le segment $[AB]$.
 
\end{enumerate}
  
\end{ExoCd}


 
 
\begin{ExoCd}{Représenter. Chercher.}{1234}{2}{0}{0}{0}{0}

On donne les segments $[AB]$ et $[EF]$ représentés ci-dessous.

\definecolor{ttqqqq}{rgb}{0.2,0.,0.}
\definecolor{qqzzff}{rgb}{0.,0.6,1.}
\definecolor{qqzzcc}{rgb}{0.,0.6,0.8}
\begin{tikzpicture}[line cap=round,line join=round,>=triangle 45,x=1.0cm,y=1.0cm]
\begin{axis}[
x=1.0cm,y=1.0cm,
axis lines=middle,
xmin=-2.200000000000002,
xmax=8.480000000000008,
ymin=-0.8000000000000048,
ymax=0.7799999999999957,
xtick={-2.0,-1.0,...,8.0},
ytick={-0.0,1.0,...,0.0},]
\clip(-2.2,-0.8) rectangle (8.48,0.78);
\draw [line width=2.pt,color=qqzzff] (-2.,0.)-- (4.,0.);
\draw [line width=2.pt] (5.,0.)-- (8.,0.);
\begin{scriptsize}
\draw [color=qqzzcc] (-2.,0.)-- ++(-2.5pt,0 pt) -- ++(5.0pt,0 pt) ++(-2.5pt,-2.5pt) -- ++(0 pt,5.0pt);
\draw[color=qqzzcc] (-1.86,0.37) node {$A$};
\draw [color=qqzzcc] (4.,0.)-- ++(-2.5pt,0 pt) -- ++(5.0pt,0 pt) ++(-2.5pt,-2.5pt) -- ++(0 pt,5.0pt);
\draw[color=qqzzcc] (3.96,0.37) node {$B$};
\draw [color=ttqqqq] (5.,0.)-- ++(-2.5pt,0 pt) -- ++(5.0pt,0 pt) ++(-2.5pt,-2.5pt) -- ++(0 pt,5.0pt);
\draw[color=ttqqqq] (5.14,0.37) node {$E$};
\draw [color=ttqqqq] (8.,0.)-- ++(-2.5pt,0 pt) -- ++(5.0pt,0 pt) ++(-2.5pt,-2.5pt) -- ++(0 pt,5.0pt);
\draw[color=ttqqqq] (8.14,0.37) node {$F$};
\end{scriptsize}
\end{axis}
\end{tikzpicture}


\begin{enumerate}
\item 
	\begin{enumerate}
		\item Déterminer le rayon de l'intervalle $[AB]$.
		\item Représenter $[AB]$ par une inégalité.
	\end{enumerate}
\item 
	\begin{enumerate}
		\item Déterminer le rayon de l'intervalle $[EF]$.
		\item Représenter $[EF]$ par une inégalité.
	\end{enumerate}
\end{enumerate}
 
\end{ExoCd}

\begin{ExoCd}{Représenter. Chercher.}{1234}{2}{0}{0}{0}{0}

Soit $x$ un réel.   


\begin{enumerate}
	\item Déterminer puis représenter l'ensemble des points $M$ d'abscisse $x$ tel que $\vert x- 3 \vert \leq 3$.
	\item Déterminer puis représenter l'ensemble des points $M$ d'abscisse $x$ tel que $\vert x+4 \vert \leq 1$.
	\item Déterminer puis représenter l'ensemble des points $M$ d'abscisse $x$ tel que $\vert x+\frac{2}{3} \vert \leq 4$.
\end{enumerate}
 
\end{ExoCd}

\begin{ExoCd}{Représenter. Chercher.}{1234}{2}{0}{0}{0}{0}

Soit $x$ un réel.   


\begin{enumerate}
	\item Déterminer puis représenter l'ensemble des points $M$ d'abscisse $x$ tel que $\vert x - \frac{4}{5} \vert \leq \frac{1}{2}$.
    \item Déterminer puis représenter l'ensemble des points $M$ d'abscisse $x$ tel que $\vert x- \pi \vert \leq 1$.
	\item Déterminer puis représenter l'ensemble des points $M$ d'abscisse $x$ tel que $\vert x - \sqrt{2}  \vert \leq  1$.
	\item Écrire à l'aide d'une double inégalité puis représenter  l'ensemble tel  que $\vert x + \frac{2}{3} \vert \leq 5$.
	\item Écrire à l'aide d'une double inégalité puis représenter  l'ensemble tel  que $\vert x+ \pi \vert \leq 10^{-1}$.	
\end{enumerate}
 
\end{ExoCd}

\begin{ExoCd}{Représenter. Chercher.}{1234}{2}{0}{0}{0}{0}

\begin{enumerate}
\item On s'intéresse à $\frac{3}{7}$.
\begin{enumerate}
	\item 1 est-elle une valeur approchée de $\frac{3}{7}$ à $ 10^{-1}$ près ?
	\item Déterminer une valeur approchée $a$ à $ 10^{-1}$ près de $\frac{3}{7}$.	
\end{enumerate}

\item
On s'intéresse à $\sqrt{10}$.
\begin{enumerate}
	\item Déterminer un encadrement de $\sqrt{10}$ à $ 10^{-2}$ près .
	\item Déterminer à la calculatrice  une valeur approchée de  $\sqrt{10}$.	
\end{enumerate}
\end{enumerate}
  
\end{ExoCd}


\end{pageParcoursd}

%%%%%%%%%%%%%%%%%%%%%%%%%%%%%%%%%%%%%%%%%%%%%%%%%%%%%%%%%%%%%%%%%%%%%%%%%%%%%%%%%%%%%%%%%%%%%%%%%%%%%%%%%%%%%%%%%%%%%%%%%%%
%%%%%%%%%%%%%%%%%%%%%%%%%%%%%%%%%%%%%%%%%%%%%%%%%%%%%%%%%%%%%%%%%%%%%%%%%%%%%%%%%%%%%%%%%%%%%%%%%%%%%%%%%%%%%%%%%%%%%%%%%%%
%%%%%%%%%%%%%%%            pageParcourst                      %%%%%%%%%%%%%%%%%%%%%%%%%%%%%%%%%%%%%%%%%%%%%%%%%%%%%%%%%%%%%
%%%%%%%%%%%%%%%%%%%%%%%%%%%%%%%%%%%%%%%%%%%%%%%%%%%%%%%%%%%%%%%%%%%%%%%%%%%%%%%%%%%%%%%%%%%%%%%%%%%%%%%%%%%%%%%%%%%%%%%%%%%
%%%%%%%%%%%%%%%%%%%%%%%%%%%%%%%%%%%%%%%%%%%%%%%%%%%%%%%%%%%%%%%%%%%%%%%%%%%%%%%%%%%%%%%%%%%%%%%%%%%%%%%%%%%%%%%%%%%%%%%%%%%

\begin{pageParcourst}

\begin{ExoCt}{Représenter. Chercher.}{1234}{2}{0}{0}{0}{0}
 
\end{ExoCt}

\begin{ExoCt}{Représenter. Chercher.}{1234}{2}{0}{0}{0}{0}
 
\end{ExoCt}

\begin{ExoCt}{Représenter. Chercher.}{1234}{2}{0}{0}{0}{0}
 
\end{ExoCt}


\end{pageParcourst}
%%%%%%%%%%%%%%%%%%%%%%%%%%%%%%%%%%%%%%%%%%%%%%%%%%%%%%%%%%%%%%%%%%%%%%%%%%%%%%%%%%%%%%%%%%%%%%%%%%%%%%%%%%%%%%%%%%%%%%%%%%%
%%%%%%%%%%%%%%%%%%%%%%%%%%%%%%%%%%%%%%%%%%%%%%%%%%%%%%%%%%%%%%%%%%%%%%%%%%%%%%%%%%%%%%%%%%%%%%%%%%%%%%%%%%%%%%%%%%%%%%%%%%%
%%%%%%%%%%%%%%%              pageAuto                         %%%%%%%%%%%%%%%%%%%%%%%%%%%%%%%%%%%%%%%%%%%%%%%%%%%%%%%%%%%%%
%%%%%%%%%%%%%%%%%%%%%%%%%%%%%%%%%%%%%%%%%%%%%%%%%%%%%%%%%%%%%%%%%%%%%%%%%%%%%%%%%%%%%%%%%%%%%%%%%%%%%%%%%%%%%%%%%%%%%%%%%%%
%%%%%%%%%%%%%%%%%%%%%%%%%%%%%%%%%%%%%%%%%%%%%%%%%%%%%%%%%%%%%%%%%%%%%%%%%%%%%%%%%%%%%%%%%%%%%%%%%%%%%%%%%%%%%%%%%%%%%%%%%%%
\begin{pageAuto}

\begin{ExoAuto}{Représenter. Chercher.}{1234}{2}{0}{0}{0}{0}
 
\end{ExoAuto}

\begin{ExoAuto}{Représenter. Chercher.}{1234}{2}{0}{0}{0}{0}
 
\end{ExoAuto}

\begin{ExoAuto}{Représenter. Chercher.}{1234}{2}{0}{0}{0}{0}
 
\end{ExoAuto}


\end{pageAuto}