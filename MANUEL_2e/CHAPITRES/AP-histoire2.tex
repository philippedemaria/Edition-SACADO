
\begin{His}

\vspace{0.4cm}

{\color{brown}Qu'est-ce exactement que "l'art numérique" ?}

\vspace{0.4cm}

Il s'agit tout simplement d'un art généré à partir de l'informatique et de l'imagerie électronique. Il est donc né dans les années soixante, avec des travaux créatifs exploratoires, et des logiciels de développement qui généraient des formes graphiques et des sons -assez rudimentaires au début, bien sûr.

Par exemple,  en 1968, l'exposition "Cybernetic Serenpidity" a fait date en exposant des travaux réalisés à l'ordinateur, principalement par des scientifiques, et utilisant les mathématiques pour produire des dessins.

Dans les années 70 commence la démocratisation des ordinateurs, c'est donc l'apparition des palettes graphiques et des écrans couleurs, qui étendent le champ de la création. Des artistes tels David Em utilisent la palette graphique pour des créations entre géométrie et peinture.

Cette période était donc en quelque sorte la "préhistoire" de l'art numérique ?

Oui, c'est un peu ça, mais dès les années 80, c'est l'essor de la scène des "demo-makers" qui mixent musique, et représentations générées par des algorithmes. En parallèle se développe aussi l'imagerie 3D, avec des oeuvres emblématiques telles que Sexy Robot, 1984 (Robert Abel), ou Mysterious Galaxy, 1983 (Yoichiro Kawaguchi).

Une décennie plus tard, des artistes tels Maurice Benayoun et Jeffrey Show explorent la réalité virtuelle pour raconter des histoires et produire des oeuvres immersives.  Par ailleurs se developpent les oeuvres multimedia sur CD Rom, qui délinérarisent les contenus videos, par exemple de Chris Marker.

\vspace{0.4cm}

{\color{brown}Et aujourd'hui ?}

\vspace{0.4cm}
Vers 2010, le grand public a pu enfin avoir accès à l'art numérique -sans savoir forcément que c'était de "l'art numérique- grâce à au "video mapping": ce sont ces projections lumineuses et multicolores sur les façades de cathédrales ou de monuments, qui sont précises au millimètre près, ce qui permet aux visuels d'épouser parfaitement les formes et reliefs des immeubles, comme c'est le cas avec la Fête des lumières de Lyon  ou le festival Chartres en lumières . Ce sont des artistes comme Matt Pyke qui sont en pointe sur ce mouvement.

L'art numérique est immatériel  etest constitué de "zéros" et de "uns", il est donc difficile de le "vendre", de le promouvoir, de l'expliquer… Comment s'en sort-il ?

Oui, c'est justement pour cette raison que nous avons fondé Bright avec Martin Zack-Mekkaou, qui est directeur technique: pour faire connaître cet art, notamment aux entreprises, qui disposent des finances pour le mettre en lumière. Cette dimension pédagogique est essentielle car cet art est en outre protéiforme : prenez le "data-art", par exemple, qui consiste à générer des oeuvres crées à partir d'un espace ou d'un lieu, comme les data concernant un musée.

Ah… Là, je crois, il faudrait un exemple…

\end{His}