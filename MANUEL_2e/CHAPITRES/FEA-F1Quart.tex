\begin{titre}[Calculs numériques]

\Titre{Les puissances}{2}
\end{titre}


\begin{CpsCol}
\begin{description}
\item[$\square$] Manipuler les puissances
\end{description}
\end{CpsCol}

\begin{minipage}{0.5\linewidth}
\begin{ThT}{Les puissances\index{Puissances}}
Pour tous nombres $a$ et $b$, et $n$ un entier, 
\begin{description}
\item $a^n \times a^m = a^{n+m}$
\item $\frac{a^n}{a^m}=a^{n-m}$, $a \neq 0$
\item $a^n \times b^n = (ab)^n$
\item $\frac{a^n}{b^n}=\left( \frac{a}{b} \right)^n$, $b \neq 0$

\end{description}
\end{ThT}
\end{minipage}
\begin{minipage}{0.5\linewidth}
\begin{DefT}{Écriture scientifique\index{Écriture scientifique}}
 
Un nombre décimal positif est écrit en notation scientifique lorsqu'il est sous la forme $a \times 10^p$ où :
\begin{description}
\item[•] $a$ est un décimal tel que sa partie entière est comprise entre 1 et 9.
\item[•] $p$ est un entier relatif. 
\end{description}
\textbf{Exemples} : en notation scientifique
\begin{description}
\item 2019 s'écrit $2,019\times10^{3}$; 250 s'écrit $2,5\times10^{2}$
\item 0,0035 s'écrit $3,5\times10^{-3}$; 0,2 s'écrit $2\times10^{-1}$ 
\end{description}
\end{DefT}
\end{minipage}




\begin{minipage}{0.5\linewidth}


\EPCN{Calculer }

Effectuer les opérations sans calculatrice.

\begin{enumerate}
\item $A = 1 + 3^2$
\item $B = 2 \times 5^3$
\item $C = ( 2 \times 5)^3$
\item $D = 2^{-1} \times 5^{-2}$
\end{enumerate}
\end{minipage}
\begin{minipage}{0.5\linewidth}

\EPCN{Calculer }

\begin{enumerate}
\item Factoriser par $2^3$ l'expression de $A$ :  $A = 2^5-2^3$
\item Factoriser par $3^n$ l'expression de $B$ :  $B = 3^{5n} + 3^{n+1}$
\item Factoriser par $a^n$ l'expression de $C$ :  $C = a^{n+1}-a^n$
\item Factoriser par $2^n$ l'expression de $D$ :  $D = 8^n + 2^{n+1}$
\end{enumerate}
\end{minipage}

\EPCN{Calculer }

Une molécule d'hydrogène pèse 1,008 unité de masse atomique. Une unité de masse atomique représente $ \np{1,660538922} \times 10^{-27}$ kg. Dans un litre d'hydrogène, il y a $5,38 \times 10^{22}$ molécules.

Quelle est la masse d'un litre d'hydrogène ?


\EPCP{1}{FEA-95}{Calculer.}


\EPCN{Modeliser }

La vitesse de la lumière dans le vide est de $3\times 10^8 \text{m.s}^{-1}$. Quelle distance la lumière parcourt-elle en une année de 365 jours ? Donner une écriture scientifique du résultat.






