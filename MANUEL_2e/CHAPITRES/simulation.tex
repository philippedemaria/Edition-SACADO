\documentclass[20pt]{article}

\input{../../../latex_preambule_style/preambule}
\input{../../../latex_preambule_style/styleCourslycee}
\input{../../../latex_preambule_style/styleExercices}
%\input{../../latex_preambule_style/styleCahier}
\input{../../../latex_preambule_style/bas_de_page_seconde}
\input{../../../latex_preambule_style/algobox}



%%%%%%%%%%%%%%%  Affichage ou impression  %%%%%%%%%%%%%%%%%%
\newcommand{\impress}[2]{
\ifthenelse{\equal{#1}{1}}  %   1 imprime / affiche  -----    0 n'affiche pas
{%condition vraie
#2
}% fin condition vraie
{%condition fausse
}% fin condition fausse
} % fin de la procédure
%%%%%%%%%%%%%%%  Affichage ou impression  %%%%%%%%%%%%%%%%%%



%%%%%%%%%%%%%%%  Indentation  %%%%%%%%%%%%%%%%%%
\parindent=0pt
%%%%%%%%%%%%%%%%%%%%%%%%%%%%%%%%%%%%%%%%%%%%%%%%



\begin{document}


\textbf{Expérience :} \\
On lance 3 dés cubiques équilibrés et on calcule la somme des faces obtenues. \\
Sur 100 lancers, combien  de fois obtient-on une somme supérieure ou égale  à 14 ?
\vspace{0.2cm}
\hrule
\vspace{1cm}

\textbf{Expérience :} \\
On lance 3 dés cubiques équilibrés et on calcule la somme des faces obtenues. \\
Sur 100 lancers, combien  de fois obtient-on une somme supérieure ou égale  à 14 ?
\vspace{0.2cm}
\hrule
\vspace{1cm}
\textbf{Expérience :} \\
On lance 3 dés cubiques équilibrés et on calcule la somme des faces obtenues. \\
Sur 100 lancers, combien  de fois obtient-on une somme supérieure ou égale  à 14 ?
\vspace{0.2cm}
\hrule
\vspace{1cm}
\textbf{Expérience :} \\
On lance 3 dés cubiques équilibrés et on calcule la somme des faces obtenues. \\
Sur 100 lancers, combien  de fois obtient-on une somme supérieure ou égale  à 14 ?
\vspace{0.2cm}
\hrule
\vspace{1cm}
\textbf{Expérience :} \\
On lance 3 dés cubiques équilibrés et on calcule la somme des faces obtenues. \\
Sur 100 lancers, combien  de fois obtient-on une somme supérieure ou égale  à 14 ?
\vspace{0.2cm}
\hrule
\vspace{1cm}
\textbf{Expérience :} \\
On lance 3 dés cubiques équilibrés et on calcule la somme des faces obtenues. \\
Sur 100 lancers, combien  de fois obtient-on une somme supérieure ou égale  à 14 ?
\vspace{0.2cm}
\hrule
\vspace{1cm}
\textbf{Expérience :} \\
On lance 3 dés cubiques équilibrés et on calcule la somme des faces obtenues. \\
Sur 100 lancers, combien  de fois obtient-on une somme supérieure ou égale  à 14 ?
\vspace{0.2cm}
\hrule
\vspace{1cm}
\textbf{Expérience :} \\
On lance 3 dés cubiques équilibrés et on calcule la somme des faces obtenues. \\
Sur 100 lancers, combien  de fois obtient-on une somme supérieure ou égale  à 14 ?
\vspace{0.2cm}
\hrule
\vspace{1cm}
\textbf{Expérience :} \\
On lance 3 dés cubiques équilibrés et on calcule la somme des faces obtenues. \\
Sur 100 lancers, combien  de fois obtient-on une somme supérieure ou égale à 14 ?
\vspace{0.2cm}
\hrule
\vspace{1cm}
\textbf{Expérience :} \\
On lance 3 dés cubiques équilibrés et on calcule la somme des faces obtenues. \\
Sur 100 lancers, combien  de fois obtient-on une somme supérieure ou égale à 14 ?
\vspace{0.2cm}
\hrule
\vspace{1cm}

\newpage


\textbf{Aide 1 : Algorithme}

\textbf{Que fait ce pseudo-code ?}

Choisir un nombre aléatoire entre 1 et 6\\
Choisir un nombre aléatoire entre 1 et 6\\
Choisir un nombre aléatoire entre 1 et 6\\
Sommer les 3 nombres\\
compteur = 0\\
Si la somme est supérieure à égale à 14\\
\hspace{1cm} Ajouter 1 à compteur \\
Afficher  compteur  \\
\vspace{0.2cm}
\hrule
\vspace{0.2cm}

\textbf{Aide 2 : Python}

\textbf{Les commandes}

\begin{description}
\item Pour obtenir des nombres aléatoires, il est nécessaire d'importer la bibliothèque random avec : \textit{import random}

\item  \textit{Choisir un nombre aléatoire entre 1 et 6} se traduit par \textit{x=random.randint(1,6)}
 
\item  \textit{un test si la somme est supérieure ou égale à 14 } se traduit par \textit{if  som > 13 \textbf{:}}   \\
 
\item Pour itérer 100 fois une action, on utilise la boucle \textit{for i in range(100) :} 
\end{description}
\hrule
\vspace{0.2cm}

\textbf{Aide 3 : Python}

 

Tester ce code puis le modifier pour obtenir l'action demandée.

\begin{description}
\item  import random
\item  x = random.randint(1,6)
\item  compteur = 0
\item  if x > 5 :     
\item  \hspace{1cm}  compteur +=1 
\item  print(compteur)
\end{description}
\hrule
\vspace{0.2cm}

\textbf{Aide 1 : Tableur}

\begin{enumerate}

\item Lors d'un lancer d'un dé, quelle est la probabilité d'obtenir une face paire ?

\item Mise en œuvre

\begin{list}{}{}
		\item Dans la cellule $A1$ écrire : Face 2
		\item Dans la cellule $A2$ écrire : Face 4
		\item Dans la cellule $A3$ écrire : Face 6
		\item Dans la cellule $A4$ écrire : Fréquence
		\item Dans la cellule $B5$ écrire : ech.1
		\item Dans la cellule $B6$ écrire : = ALEA.ENTRE.BORNES(1;6). 
		\item Glisser copier la cellule $B6$ jusqu'à la ligne 105 pour obtenir 100 lancers.
		\item Dans la cellule $B1$ écrire : =NB.SI($B6:B105$;2)
		\item Dans la cellule $B2$ écrire : =NB.SI($B6:B105$;4)
		\item Dans la cellule $B3$ écrire : =NB.SI($B6:B105$;6)
		\item Dans la cellule $B4$ faire calculer la somme des cellules $B1+B2+B3$ divisée par 100. 
		\item Donner la signification du résultat de la cellule $B4$.
	\end{list}

La fréquence observée est-elle égale à la probabilité de la question 1 ?De plus, on note une fluctuation des résultats.
\item Faire 50 simulations de ces 100 lancers. On pourra copier alors 50 fois la colonne B jusqu'à la colonne AY. On obtient alors 50 fréquences.
\item Utiliser l'assistant graphique pour illustrer la situation par un nuage de points.
\item Dans quel intervalle se situent au moins 95\% de ces fréquences ?
\end{enumerate}




\end{document}
