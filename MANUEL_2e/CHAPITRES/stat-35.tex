
Étienne a obtenu les notes 5, 7, 6, 10 et 9. On appelle $N_{initales}$ la série de ses notes à l'état initial.


\begin{enumerate}
\item Calculer la moyenne $m_{initales}$ de Étienne.
\item Son enseignant décide d'ajouter 20\% à toutes les notes et de les arrondir au dixième près. Calculer chacune des nouvelles notes. On appelle $N_{trans}$ la série de ses notes après ce changement.
\item Comme Étienne est un élève très sérieux, son enseignant lui rajoute encore un point à chacune de ses notes. Calculer chacune des nouvelles notes. On appelle $N_{finales}$ la série de ses notes après ce changement.
\item Calculer alors la moyenne de Étienne, arrondie à l'unité près. 
\item Déterminer un algorithme $A_1$ qui permet de calculer une note de $N_{finales}$, connaissant une note de $N_{initales}$.
\item Déterminer un algorithme $A_2$ qui permet de calculer la moyenne de $N_{finales}$, connaissant $N_{initales}$.
\item Appliquer l'algorithme $A_1$ à la moyenne $m_{initales}$. Qu'obtient t-on ?
\end{enumerate} 