\begin{titre}[Programmation en Python]

\Titre{Variables et condition}{2}
\end{titre}


\begin{CpsCol}
\textbf{Python}
\begin{description}
\item[$\square$] Utiliser les variables
\item[$\square$] Utiliser les instructions conditionnelles
\end{description}
\end{CpsCol}


\begin{PC}

Dans un pays où le système monétaire n’est constitué que de pièces de 3 et de 5, il s’agit d’aider les habitants en créant un programme  qui donne le nombre de pièces nécessaires à tout achat d’un montant entier supérieur ou égal à 8.

\hfill{{\footnotesize Source : d’après PISA, items libérés}}
\end{PC}

\mini{
\begin{Mt}
Pour afficher un calcul ou une action effectué par l'ordinateur, on utilise le mot clé {\color{violet}print}.
\end{Mt}
}{
\begin{Cod}
{\color{violet}print}({\color{vert}"Coucou"})
\end{Cod}
}

\begin{DefT}{Variable}
Une variable est un tiroir dans lequel on place une valeur qui servira lors de l'exécution du programme.
\end{DefT}

\begin{Mt}
Pour affecter une valeur à une variable var, on utilise le mot réservé {\color{violet}input}.

var={\color{violet}input}({\color{vert}"Entrer une variable"})
\end{Mt}

\begin{Rq}
Si la variable est un entier, on écrit avant le {\color{violet}input}, un {\color{violet}int}.

Si la variable est un réel, on écrit avant le {\color{violet}input}, un {\color{violet}float}.
\end{Rq}


\begin{minipage}[]{0.49\linewidth}
\begin{Ex}
n={\color{violet}int}({\color{violet}input}({\color{vert}"Entrer un nombre "}))

m={\color{violet}int}({\color{violet}input}({\color{vert}"Entrer un second nombre"}))

s=n+m

{\color{violet}print}({\color{vert}"La somme est égale à "},s)
\end{Ex}

\begin{Cod}
\texttt{n={\color{violet}int}({\color{violet}input}({\color{vert}"Entrer un entier "}))}

\texttt{m={\color{violet}int}({\color{violet}input}({\color{vert}"Entrer un second entier "}))}

\texttt{s=n+m}

\texttt{{\color{violet}print}({\color{vert}"La somme est égale à "},s)}

\end{Cod}
\end{minipage} 
\hfill
\begin{minipage}[]{0.49\linewidth}
\subsection*{Les opérateurs les plus courants}


\begin{tabular}{|c|c|c|c|}
\hline 
Opération& Signe & Exemple & Retour \\
\hline 
Addition & $+$ & $2+3$ & 5\\ 
\hline 
Soustraction & $-$ & $2-3$ & $-1$ \\ 
\hline 
Multiplication & * & 2*3&6 \\ 
\hline 
Division & / & $8/4$ & 2.0\\ 
\hline 
Puissance & ** & 2**3 & 8 \\ 
\hline 
Reste & \% & $13\%5$ & 3 \\ 
\hline 
Quotient \textbf{entier}  & $//$ & $12//5$ & 2 \\ 
\hline 
Comparaison   & $==$ & $x==5$ & {\color{orange}True}/{\color{orange}False} \\ 
\hline 
Affectation   & $=$ & $n=5$ &  \\ 
\hline 
\end{tabular} 
\end{minipage} 

\begin{DefT}{Condition}
Une \textbf{condition} \index{condition} est une instruction qui ouvre le choix parmi deux actions suivant le résultat d'un test \index{test}.

Sa structure est : \textbf{Si} test vérifié \textbf{alors} Action 1 \textbf{sinon} Action 2.
\end{DefT}

\begin{Ex}
\textbf{Si} il fait beau \textbf{alors} je vais à la plage \textbf{sinon} je vais au cinéma.
\begin{description}
\item[Test :] il fait beau
\item[Action 1 :] je vais à la plage
\item[Action 2 :] je vais au cinéma
\end{description}
\end{Ex}


\begin{Syn}
\begin{minipage}[t]{0.49\linewidth}
L'écriture algorithmique d'un test est :
\begin{verbatim}
Si {test vrai}
Alors
	 action 1
Sinon
	 action 2    	
\end{verbatim}
\end{minipage}
\hfill\vrule\hfill
\begin{minipage}[t]{0.49\linewidth}
La programmation en Python d'un test est :
\begin{lstlisting}
if condition :
   action 1 
else :
   action 2
\end{lstlisting}
\end{minipage}
\end{Syn}


\begin{minipage}[t]{0.49\linewidth}
\begin{Ex}
L'écriture algorithmique d'une condition est :
\begin{algobox}
\Si {$x < 3$}
\DebutSi
\Ligne $y \longleftarrow x+3$ 
\FinSi
\Sinon
\DebutSinon
\Ligne $y \longleftarrow x^2$ 
\FinSinon
\end{algobox}
\end{Ex}
\end{minipage}
\hfill
\begin{minipage}[t]{0.49\linewidth}
\begin{Cod}
\begin{lstlisting}
x = int(input("Donner un nombre entier"))
if x < 3 :
	y=x+3 
else :
	y=x**2 
print(y)
\end{lstlisting}
\end{Cod}
\end{minipage}


\begin{Nt}
Le test d'une égalité en Python s'emploie avec l'opérateur == (double signe égal). Cette écriture sert à différencier le test de l'affection.
\end{Nt}


\begin{Rq}[. Plusieurs instructions si-sinon imbriquées]

\color{orange} if \color{black}  Condition :

\hspace{0.4cm}    instruction 1
    
\color{orange} elif \color{black} :

\hspace{0.4cm}     instruction 2

$\cdots$

\color{orange} elif \color{black} :

\hspace{0.4cm}     instruction $n-2$

\color{orange} elif \color{black} :

\hspace{0.4cm}     instruction $n-1$

\color{orange} else \color{black} :

\hspace{0.4cm}     instruction $n$

\end{Rq}
