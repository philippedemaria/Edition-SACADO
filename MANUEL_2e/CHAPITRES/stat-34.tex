
Paul a obtenu les notes 6, 10 et 12 lors de ses trois premiers contrôles de maths. 

 Il crée un petit programme Python pour connaitre sa moyenne avec une note supplémentaire. Voici le début de son programme où \texttt{m} est la moyenne voulue, \texttt{Snote} est la somme de ses notes et \texttt{compteur} la variable compteur.
 
\begin{lstlisting} 

def MoyenneVoulue(Snote,c):
	note = .........
	if note > 20 :
		return "impossible"
	else :
		return note

somme = 0
compteur = 0
ok = True
while ok :
	note = float(input("quel est la note ?"))
	somme = somme + note
	compteur +=1
	suite = input("une autre note ? O/N ")
	if suite == "N" :
		ok = False
		
print(MoyenneVoulue(somme,compteur))		
\end{lstlisting} 


\begin{enumerate}
\item Quel est le rôle de compteur.
\item \textbf{Compléter} la ligne 2 de la fonction \texttt{MoyenneVoulue}.
\item Écrire un programme en Python qui renvoie le même résultat avec une fonction supplémentaire.
\end{enumerate} 