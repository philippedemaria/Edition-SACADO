
On souhaite étudier les fonctions de la forme $x \mapsto a(x-\alpha)^2 + \beta$ et établir quelques propriétés.

\begin{enumerate}
\item Coordonnées du sommet
\begin{enumerate}
\item Tracer avec Géogébra les courbes des fonctions définies par :
\begin{enumerate}
\item $f(x)=(x-1)^2+2$
\item $g(x)=(x+3)^2-1$
\item $h(x)=2(2+x)^2$
\end{enumerate}
\item Que peut-on conjecturer sur les coordonnées du sommet de chaque courbe ?
\item Soit $k$ la fonction définie par $k (x)=2(x-6)^2+4$. La conjecture énoncée semble-t-elle encore valide ?
\end{enumerate}

\item Sens de variations
\begin{enumerate}
\item Tracer le tableau de variations des fonctions $f$, $g$ et $h$.
\item Tracer les courbes des fonctions définies par :
\begin{enumerate}
\item $a(x)=-(x-2)^2+2$
\item $b(x)=2(4-x)^2$
\item $c(x)=-2(7-5x)^2-5$
\end{enumerate}
\item Que peut-on conjecturer sur les variations d'une fonction du second degré ? (on pourra utiliser d'autres
exemples de fonctions pour affiner la conjecture)
\item Tracer alors le tableau de variations des fonctions $a$ et $b$.
\item Dans quel cas la fonction étudiée admet-elle un maximum ? Un minimum ? Quelle est alors sa valeur ?
\end{enumerate}
\end{enumerate}
