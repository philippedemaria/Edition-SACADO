
$ABC$ est un triangle isocèle en $A$. Le point $A’$ est le milieu de $[BC]$. La perpendiculaire à $(AC)$ passant par $A’$ coupe $(AC)$ en $H$. Le point $I$ est le milieu de $[A’H]$ et $K$ celui de $[HC]$.

Le but de l'exercice est de démontrer que $(BH)$ et $(AI)$ sont perpendiculaires.

\begin{enumerate}
\item Que peut-on dire des droites $(KI)$ et $(AA’)$ ?
\item Que représente le point $I$ dans le triangle $AA’K$ ?
\item Déduisez-en que $(BH)$ et $(AI)$ sont perpendiculaires.
\end{enumerate}