\begin{titre}[Géométrie vectorielle et analytique]

\Titre{Opérations de vecteur}{4}
\end{titre}

\begin{CpsCol}
\textbf{Calculer avec les vecteurs}
\begin{description}
\item[$\square$] Calculer les coordonnées d’une somme de vecteurs, d’un produit d’un vecteur par un
nombre réel.
\item[$\square$] Multiplier un vecteur par un réel
\end{description}
\end{CpsCol}




\begin{DefT}{Somme de deux vecteurs}\index{Vecteurs!Somme}
\begin{minipage}{0.48\linewidth}
Soit $\overrightarrow{u}$ et $\overrightarrow{v}$ deux vecteurs quelconques.

On appelle somme de vecteurs $\overrightarrow{u}$ et $\overrightarrow{v}$, notée $\overrightarrow{u}+\overrightarrow{v}$, le vecteurs $\overrightarrow{w}$ qui résulte de la translation suivant $\overrightarrow{u}$ puis de la translation suivant $\overrightarrow{v}$.
\end{minipage}
\hfill
\begin{minipage}{0.48\linewidth}
\definecolor{qqqqff}{rgb}{0.,0.,1.}
\definecolor{ffqqqq}{rgb}{1.,0.,0.}
\begin{tikzpicture}[line cap=round,line join=round,>=triangle 45,x=1.0cm,y=1.0cm]
\clip(-2.08,-1.32) rectangle (4.12,2.3);
\draw [->,color=ffqqqq] (-2.,-1.) -- (1.,2.);
\draw [->,color=qqqqff] (1.,2.) -- (4.,0.);
\draw [->] (-2.,-1.) -- (4.,0.);
\draw [color=ffqqqq](-0.82,1.26) node[anchor=north west] {$\vec{u}$};
\draw [color=qqqqff](2.64,1.76) node[anchor=north west] {$\vec{v}$};
\draw (0.96,-0.36) node[anchor=north west] {$\vec{w}$};
\end{tikzpicture}
\end{minipage}
\end{DefT}


\begin{Nt}
$\overrightarrow{w}=\overrightarrow{u}+\overrightarrow{v}$
\end{Nt}





\begin{DefT}{Relation de Chasles}\index{Relation de Chasles}
Soit $A$, $B$ et $C$trois points quelconques.

$\overrightarrow{AC}=\overrightarrow{AB}+\overrightarrow{BC}$
\end{DefT}

\mini{
\AD{1}{GVA-22}
}{
\EPC{0}{GVA-50}{Raisonner. Calculer}
}

\begin{Th}
Soit $A$, $B$ et $C$trois points quelconques.\\
Le quadrilatère $ABDC$ est un parallélogramme si et seulement si $\overrightarrow{AD}=\overrightarrow{AB}+\overrightarrow{AC}$.
\end{Th}

\ROC

\begin{Th}

Soit $\overrightarrow{u}$ de coordonnées $(x,y)$ et $\overrightarrow{v}$ de coordonnées $(x',y')$.

Le vecteur  $\overrightarrow{u}+\overrightarrow{v}$ a pour coordonnées $(x+x',y+y')$.
\end{Th}


\mini{
\EPC{1}{GVA-31}{Raisonner. Calculer}

\EPC{1}{GVA-47}{Raisonner. Calculer}
}{
\EPC{1}{GVA-49}{Calculer}
}


\begin{DefT}{Différence de deux vecteurs}\index{Vecteurs!Différence}
\begin{minipage}{0.48\linewidth}
Soit $\overrightarrow{u}$ et $\overrightarrow{v}$ deux vecteurs quelconques.

On appelle \textbf{différence de vecteurs} $\overrightarrow{u}$ et $\overrightarrow{v}$, notée $\overrightarrow{u}-\overrightarrow{v}$, le vecteurs $\overrightarrow{w}$ qui résulte de la translation suivant $\overrightarrow{u}$ puis de la translation suivant $-\overrightarrow{v}$.

\end{minipage}
\hfill
\begin{minipage}{0.48\linewidth}
\definecolor{qqqqff}{rgb}{0.,0.,1.}
\definecolor{qqwuqq}{rgb}{0.,0.39215686274509803,0.}
\definecolor{ffqqqq}{rgb}{1.,0.,0.}
\begin{tikzpicture}[line cap=round,line join=round,>=triangle 45,x=1.0cm,y=1.0cm]
\clip(-2.3,-2.3) rectangle (6.24,2.2);
\draw [->,color=ffqqqq] (-2.,-1.) -- (1.,2.);
\draw [color=ffqqqq](0.42,0.94) node[anchor=north west] {$\vec{u}$};
\draw [->,color=qqwuqq] (6.,1.) -- (4.,2.);
\draw [color=qqwuqq](5.06,2.18) node[anchor=north west] {$\vec{v}$};
\draw [->,color=ffqqqq] (-1.,-2.) -- (2.,1.);
\draw [->,color=qqwuqq] (2.,1.) -- (4.,0.);
\draw [color=qqwuqq](2.64,1.26) node[anchor=north west] {$-\vec{v}$};
\draw [->,color=qqqqff] (-1.,-2.) -- (4.,0.);
\draw [color=qqqqff](1.52,-0.78) node[anchor=north west] {$\vec{u}-\vec{v}$};
\end{tikzpicture}
\end{minipage}
\end{DefT}

\begin{Th}

Soit $\overrightarrow{u}$ de coordonnées $(x,y)$ et $\overrightarrow{v}$ de coordonnées $(x',y')$.

Le vecteur  $\overrightarrow{u}-\overrightarrow{v}$ a pour coordonnées $(x-x',y-y')$.
\end{Th}

\mini{
\AD{1}{GVA-38}
}{
\Exo{1}{GVA-52}

\PO{1}{GVA-41}
}

\begin{DefT}{Produit d'un vecteur par un réel}\index{Vecteurs!Produit d'un vecteur par un réel}
\begin{minipage}{0.48\linewidth}
Soit $\overrightarrow{u}$ un vecteur quelconque et $k$ un réel non nul.

On appelle \textbf{produit du vecteur} $\overrightarrow{u}$ par le réel $k$, le vecteur $k\overrightarrow{u}$ :
\begin{description}
\item[•] de même direction que $\overrightarrow{u}$
\item[•] 
\begin{description}
\item[•] de même sens que $\overrightarrow{u}$ si $k$ est positif
\item[•] de sens opposé à $\overrightarrow{u}$ si $k$ est négatif
\end{description}
\item[•] de norme
\begin{description}
\item[•] $k \times \Vert \overrightarrow{u} \Vert$ si $k$ est positif
\item[•] $-k \times \Vert \overrightarrow{u} \Vert$ si $k$ est négatif
\end{description}

\end{description}

\end{minipage}
\hfill
\begin{minipage}{0.48\linewidth}

\definecolor{ffxfqq}{rgb}{1.,0.4980392156862745,0.}
\definecolor{qqqqff}{rgb}{0.,0.,1.}
\definecolor{ffqqqq}{rgb}{1.,0.,0.}
\begin{tikzpicture}[line cap=round,line join=round,>=triangle 45,x=1.0cm,y=1.0cm]
\clip(-2.1,-1.26) rectangle (4.52,2.16);
\draw [->,color=ffqqqq] (-2.,-1.) -- (1.,2.);
\draw [color=ffqqqq](0.08,1.24) node[anchor=north west] {$\vec{u}$};
\draw [->,color=qqqqff] (1.,0.) -- (2.,1.);
\draw [->,color=ffxfqq] (4.,1.) -- (2.,-1.);
\draw [color=qqqqff](0.34,0.24) node[anchor=north west] {$k>0$};
\draw [color=ffxfqq](2.64,-0.38) node[anchor=north west] {$k<0$};
\draw (2.46,1.7) node[anchor=north west] {$k\vec{u}$};
\end{tikzpicture}

\end{minipage}
\end{DefT}


\begin{Rq}
$k=0$ ou $\vec{u}=\vec{0}$ si et seulement $k\vec{u}=\vec{0}$
\end{Rq}

\mini{
\AD{1}{GVA-23}

\Exo{1}{GVA-48}
}{
\Exo{1}{GVA-24}
}

\begin{Th}
Soit $\overrightarrow{AB}$ de coordonnées $(x,y)$ et $k$ un réel.

Le vecteur  $k\overrightarrow{AB}$ a pour coordonnées $(kx,ky)$.
\end{Th}


\mini{
\AD{1}{GVA-25}
}{
\AD{1}{GVA-38}
}

