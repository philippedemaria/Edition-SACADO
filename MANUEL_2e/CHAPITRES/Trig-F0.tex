\begin{titre}[Trigonométrie]

\Titre{Le cercle trigonométrique}{4}
\end{titre}


\begin{CpsCol}
\textbf{Enroulement autour du cercle}
\begin{description}
\item[$\square$] Connaitre le cercle trigonométrique
\item[$\square$] Faire le lien entre radian et degré
\end{description}
\end{CpsCol}


\Rec{1}{Trig-4}




\begin{DefT}{Cercle trigonométrique}\index{Cercle trigonométrique}
On appelle cercle trigonométrique tout cercle
\begin{description}
\item[•] de rayon 1,
\item[•] orienté dans le sens opposé aux aiguilles d'une montre,
\end{description} 
Un tel cercle est dit orienté dans le sens direct.\index{Sens direct}
\end{DefT}




\begin{DefT}{Enroulement de la droite}\index{Enroulement de la droite}
On obtient les correspondances suivantes.

\begin{tabular}{|c|c|c|c|c|c|c|c|}
\hline 
Abscisse du point $N$ & $-\frac{\pi}{2}$ & $0$ & $\frac{\pi}{2}$ & $\pi$ & $\frac{\pi}{6}$ & $\frac{\pi}{4}$ & $\frac{\pi}{3}$ \\ 
\hline 
Mesure de l'angle $\widehat{AOM}$ & $-90$ & 0 & 90 & 180 & 30 & 45 & 60 \\ 
\hline 
\end{tabular} 
\end{DefT}




\mini{
\Exo{1}{Trig-0}

\Exo{1}{Trig-1}
}{
\Exo{1}{Trig-2}

\Exo{1}{Trig-3}
}

\mini{
\Exo{1}{Trig-5}

\PO{1}{Trig-8}
}{
\Exo{1}{Trig-6}
}



\begin{DefT}{Mesure principale}\index{Mesure principale}
On appelle mesure principale de l'angle $\left(\overrightarrow{OA};\overrightarrow{OM} \right)$ la mesure de cet angle compris entre $]-\pi;\pi]$
\end{DefT}

\Exo{1}{Trig-7}
