\chapter{Intervalles}
{https://sacado.xyz/qcm/parcours_show_course/0/117129}
{


 \begin{CpsCol}
\textbf{Les savoir-faire du parcours}
 \begin{itemize}
 \item \textbf{Utiliser des nombres pour calculer et résoudre des problèmes}
	\item[$\square$] Représenter un intervalle de la droite numérique. 
	\item[$\square$]  Déterminer si un nombre réel appartient à un intervalle donné.
 \end{itemize}
 \end{CpsCol}

}




%%%%%%%%%%%%%%%%%%%%%%%%%%%%%%%%%%%%%%%%%%%%%%%%%%%%%%%%%%%%%%%%%%%
%%%%  Niveau 1
%%%%%%%%%%%%%%%%%%%%%%%%%%%%%%%%%%%%%%%%%%%%%%%%%%%%%%%%%%%%%%%%%%%
\begin{pageExercices} 


 %%%%%%%%%%%%%%%%%%%%%%%%%%%
\begin{ExoCuN}{Représenter.}{2}{0}{0}{0}{0}

Démontrer que $0,\underline{1}$ est un nombre rationnel à préciser.
 
\end{ExoCuN}
 
%%%%%%%%%%%%%%%%%%%%%%%%%%%
\begin{ExoCuN}{Représenter.}{2}{0}{0}{0}{0}
Je suis un nombre à trois chiffres non nuls. Je suis divisible par 94. Changez l'ordre de mes chiffres et je deviens divisible par 49.
Qui suis-je ? 
\end{ExoCuN}


\begin{ExoCdN}{Chercher.communiquer.}{2}{0}{0}{0}{0}
 
Dans chaque cas, trouver, lorsque cela est possible, le nombre $x$ qui remplit les critères suivants :
\begin{enumerate}
\item $x \in \Q$ et $x \not\in \Z$
\item $x \in \R$ et $x \not\in \N$. 
\end{enumerate}
\end{ExoCdN}

\begin{ExoCdN}{Représenter. Raisonner.}{2}{0}{0}{0}{0}

Démontrer que $\sqrt{2}$ est un irrationnel ou encore que $\sqrt{2} \in \R\setminus\Q$

\end{ExoCdN}

%%%%%%%%%%%%%%%%%%%%%%%%%%%%%%%%%%%%%%%%%%%%%%%%%%%%%%%%%%%%%%%%%%%
\begin{ExoCtN}{Représenter.}{2}{1}{0}{0}{0}

\begin{enumerate}
\item Démontrer que $0,\underline{12}$ est un nombre rationnel à préciser.
\item Démontrer que $0,\underline{485}$ est un nombre rationnel à préciser.
\end{enumerate}
\end{ExoCtN}

%%%%%%%%%%%%%%%%%%%%%%%%%%%%%%%%%%%%%%%%%%%%%%%%%%%%%%%%%%%%%%%%%%%
\begin{ExoCtN}{Raisonner.}{1}{0}{0}{0}{0}

\begin{enumerate}
\item Démontrer que tout entier $n$ multiple de $9$ est aussi un multiple de $3$.
\item Démontrer que si $m$ est un multiple de $6$ alors $m$ est aussi un multiple de $3$.
\end{enumerate}
\end{ExoCtN}


\begin{ExoCtN}{Représenter.}{1}{0}{0}{0}{0}

Dans un pays où le système monétaire n’est constitué que de pièces de 3 et de 5, il s’agit d’aider les habitants en créant un programme  qui donne le nombre de pièces nécessaires à tout achat d’un montant entier supérieur ou égal à 8.

\hfill{{\footnotesize Source : d’après PISA, items libérés}}

\end{ExoCtN}


\end{pageExercices}

  
%%%%%%%%%%%%%%%%%%%%%%%%%%%%%%%%%%%%%%%%%%%%%%%%%%%%%%%%%%%%%%%%%%%
%%%%  Niveau 2
%%%%%%%%%%%%%%%%%%%%%%%%%%%%%%%%%%%%%%%%%%%%%%%%%%%%%%%%%%%%%%%%%%%



%\begin{pageParcoursd} 
% 
%%%%%%%%%%%%%%%%%%%%%%%%%%%%%%%%%%%%%%%%%%%%%%%%%%%%%%%%%%%%%%%%%%%%
%
%
% 
%%%%%%%%%%%%%%%%%%%%%%%%%%%%%%%%%%%%%%%%%%%%%%%%%%%%%%%%%%%%%%%%%%%%
%
%
%
%%%%%%%%%%%%%%%%%%%%%%%%%%%%%%%%%%%%%%%%%%%%%%%%%%%%%%%%%%%%%%%%%%%%
%
%
% %%%%%%%%%%%%%%%%%%%%%%%%%%%%%%%%%%%%%%%%%%%%%%%%%%%%%%%%%%%%%%%%%%%
%\begin{ExoCd}{Représenter. Raisonner.}{1234}{2}{0}{0}{0}{0}
%
%
%\end{ExoCd}
% 
%%%%%%%%%%%%%%%%%%%%%%%%%%%%%%%%%%%%%%%%%%%%%%%%%%%%%%%%%%%%%%%%%%%%
%\begin{ExoCd}{Représenter. Raisonner.}{1234}{2}{0}{0}{0}{0}
%
%
%\end{ExoCd}
% 
%\end{pageParcoursd}
%
%%%%%%%%%%%%%%%%%%%%%%%%%%%%%%%%%%%%%%%%%%%%%%%%%%%%%%%%%%%%%%%%%%%%
%%%%%  Niveau 3
%%%%%%%%%%%%%%%%%%%%%%%%%%%%%%%%%%%%%%%%%%%%%%%%%%%%%%%%%%%%%%%%%%%%
%\begin{pageParcourst}
%
%
%
%
%%%%%%%%%%%%%%%%%%%%%%%%%%%%%%%%%%%%%%%%%%%%%%%%%%%%%%%%%%%%%%%%%%%%
%\begin{ExoCt}{Raisonner.}{1234}{2}{0}{0}{0}{0}
% 
%\end{ExoCt}
%
%%%%%%%%%%%%%%%%%%%%%%%%%%%%%%%%%%%%%%%%%%%%%%%%%%%%%%%%%%%%%%%%%%%%
%\begin{ExoCt}{Représenter.}{1234}{2}{0}{0}{0}{0}
%
% 
%
%\end{ExoCt}
%
%%%%%%%%%%%%%%%%%%%%%%%%%%%%%%%%%%%%%%%%%%%%%%%%%%%%%%%%%%%%%%%%%%%%
%\begin{ExoCt}{Représenter.}{1234}{2}{0}{0}{0}{0}
%
% 
%
%\end{ExoCt} 
% 
%\end{pageParcourst}
%
%%%%%%%%%%%%%%%%%%%%%%%%%%%%%%%%%%%%%%%%%%%%%%%%%%%%%%%%%%%%%%%%%%%%
%%%%%  Brouillon
%%%%%%%%%%%%%%%%%%%%%%%%%%%%%%%%%%%%%%%%%%%%%%%%%%%%%%%%%%%%%%%%%%%%


\begin{pageBrouillon} 
 
\ligne{32}



\end{pageBrouillon}

%%%%%%%%%%%%%%%%%%%%%%%%%%%%%%%%%%%%%%%%%%%%%%%%%%%%%%%%%%%%%%%%%%%
%%%%  Auto
%%%%%%%%%%%%%%%%%%%%%%%%%%%%%%%%%%%%%%%%%%%%%%%%%%%%%%%%%%%%%%%%%%%


%%%%%%%%%%%%%%%%%%%%%%%%%%%%%%%%%%%%%%%%%%%%%%%%%%%%%%%%%%%%%%%%%%%
\begin{pageAuto} 


\begin{ExoAuto}{Raisonner.}{1234}{2}{0}{0}{0}{0}

 
%%%%%%%%%%%%%%%%%%%%%%%%%%%%%%%%%%%%%%%%%%%%%%%%%%%%%%%%%%%%%%%%%%%
\end{ExoAuto}

\begin{ExoAuto}{Raisonner.}{1234}{2}{0}{0}{0}{0}
  

\end{ExoAuto}

%%%%%%%%%%%%%%%%%%%%%%%%%%%%%%%%%%%%%%%%%%%%%%%%%%%%%%%%%%%%%%%%%%%
\begin{ExoAuto}{Raisonner.}{1234}{2}{0}{0}{0}{0}

 
 

\end{ExoAuto}

 
%%%%%%%%%%%%%%%%%%%%%%%%%%%%%%%%%%%%%%%%%%%%%%%%%%%%%%%%%%%%%%%%%%%
\begin{ExoAuto}{Raisonner.}{1234}{2}{0}{0}{0}{0}

 
 

\end{ExoAuto}


\end{pageAuto}
