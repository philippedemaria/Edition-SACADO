\chapter{Éléments de géométrie}
{https://sacado.xyz/qcm/parcours_show_course/0/117129}
{


 \begin{CpsCol}
\textbf{Les savoir-faire du parcours}
 \begin{itemize}
 \item \textbf{Utiliser des nombres pour calculer et résoudre des problèmes}
	\item[$\square$] Représenter un intervalle de la droite numérique. 
	\item[$\square$]  Déterminer si un nombre réel appartient à un intervalle donné.
 \end{itemize}
 \end{CpsCol}

}

\begin{pageCours}

 

\begin{DefT}{Intervalle fermé. Intervalle ouvert}\index{Intervalle}
Un \textbf{intervalle fermé} de $\R$ est un sous-ensemble borné de $\R$, c'est à dire un ensemble de nombres compris entre deux valeurs réelles.
 
Un \textbf{intervalle ouvert} de $\R$ est un sous-ensemble de $\R$ dont les bornes ne sont pas incluses dans l'ensemble, c'est à dire un ensemble de nombres compris entre deux valeurs réelles non comprises.
\end{DefT}

\begin{Ex}
\begin{enumerate}
\item On a représenté sur la droite des nombres réels tous les nombres réels $x$ tels que $-1 \leq x \leq 3$.

\begin{center}
\definecolor{ffxfqq}{rgb}{1.,0.4980392156862745,0.}
\begin{tikzpicture}[line cap=round,line join=round,>=triangle 45,x=1.0cm,y=1.0cm]
\draw[->,color=black] (-4.390839866186475,0.) -- (7.64974334956303,0.);
\foreach \x in {-4.,-3.,-2.,-1.,1.,2.,3.,4.,5.,6.,7.}
\draw[shift={(\x,0)},color=black] (0pt,2pt) -- (0pt,-2pt) node[below] {\footnotesize $\x$};
\draw[color=black] (0pt,-10pt) node[right] {\footnotesize $0$};
\clip(-4.390839866186475,-0.5880295569511441) rectangle (7.64974334956303,0.5863275715079787);
\draw [line width=2.4pt,color=ffxfqq] (-1.,0.)-- (3.,0.);
\draw [color=ffxfqq](-1.16,0.25) node[anchor=north west] {[};
\draw [color=ffxfqq](2.85,0.25) node[anchor=north west] {]};
\end{tikzpicture}
 \end{center} 
 
Cet intervalle est noté $[-1;3]$. Il est fermé.
 \item On a représenté sur la droite des nombres réels tous les nombres réels $x$ tels que $-1 < x < 3$.

\begin{center}
\definecolor{ffxfqq}{rgb}{1.,0.4980392156862745,0.}
\begin{tikzpicture}[line cap=round,line join=round,>=triangle 45,x=1.0cm,y=1.0cm]
\draw[->,color=black] (-4.390839866186475,0.) -- (7.64974334956303,0.);
\foreach \x in {-4.,-3.,-2.,-1.,1.,2.,3.,4.,5.,6.,7.}
\draw[shift={(\x,0)},color=black] (0pt,2pt) -- (0pt,-2pt) node[below] {\footnotesize $\x$};
\draw[color=black] (0pt,-10pt) node[right] {\footnotesize $0$};
\clip(-4.390839866186475,-0.5295569511441) rectangle (7.64974334956303,0.563275715079787);
\draw [line width=2.4pt,color=ffxfqq] (-1.,0.)-- (3.,0.);
\draw [color=ffxfqq](-1.16,0.25) node[anchor=north west] {]};
\draw [color=ffxfqq](2.85,0.25) node[anchor=north west] {[};
\end{tikzpicture}
 \end{center}
Cet intervalle ouvert est noté $]-1;3[$. Il est ouvert.

 \item On a représenté sur la droite des nombres réels tous les nombres réels $x$ tels que $x \geq -1$.


\begin{center}
\definecolor{ffxfqq}{rgb}{1.,0.4980392156862745,0.}
\begin{tikzpicture}[line cap=round,line join=round,>=triangle 45,x=1.0cm,y=1.0cm]
\draw[->,color=black] (-4.390839866186475,0.) -- (7.64974334956303,0.);
\foreach \x in {-4.,-3.,-2.,-1.,1.,2.,3.,4.,5.,6.,7.}
\draw[shift={(\x,0)},color=black] (0pt,2pt) -- (0pt,-2pt) node[below] {\footnotesize $\x$};
\draw[color=black] (0pt,-10pt) node[right] {\footnotesize $0$};
\clip(-4.390839866186475,-0.5880295569511441) rectangle (7.64974334956303,0.53275715079787);
\draw [line width=2.4pt,color=ffxfqq] (-1.,0.)-- (8.,0.);
\draw [color=ffxfqq](-1.16,0.25) node[anchor=north west] {[};
\end{tikzpicture}
 \end{center} 
Cet ensemble est noté $[-1 ; +\infty[$, cet intervalle est semi-ouvert.
\end{enumerate}
\end{Ex}

\begin{Rqs}
\begin{enumerate}
\item  $+ \infty$ se lit "plus l’infini". L'ensemble des nombres réels $\R$ est l’intervalle $]-\infty ; +\infty[ = \R$.
\item Un intervalle est une partie de $\R$ "sans trou", en "un seul morceau".
\item $+\infty$ et $-\infty$ ne sont pas des nombres. Ce ne sont que des notations (ce qui explique qu'ils soient toujours exclus).
\item Les intervalles correspondants aux quatre premières lignes du tableau sont dits bornés.
\item  Plus généralement, les différents types d'intervalles sont donnés dans le tableau ci-dessous (où $a$ et $b$ représentent deux réels, avec $a < b$).
\end{enumerate}
\end{Rqs}

 


\begin{DefT}{Intersection}\index{Ensemble!Intersection}
L'\textbf{intersection} de deux ensembles $A$ et $B$ est l'ensemble $A \cap B$ qui contient tous les éléments communs aux deux ensembles.
\end{DefT}

\begin{Rq}\index{Ensembles disjoints}
Deux ensembles sont disjoints lorsque $A \cap B = \oslash$. $\oslash$ est l'ensemble vide.\index{Ensemble!vide}
\end{Rq}


\begin{DefT}{Réunion}\index{Ensemble!Réunion}
La réunion de deux ensembles est l'ensemble $A \cup B$ qui contient tous les éléments des deux ensembles pris une seule fois.
\end{DefT}

\end{pageCours} 

\begin{pageAD} 
 

\Sf{Opérer avec les ensembles}
 
  
\begin{ExoCad}{Modéliser.}{1234}{0}{0}{0}{0}{0}

Recopier et compléter le tableau.

%\begin{tabular}{|c|c|c|}
%\hline 
%Intervalle & Inégalité & Représentation  \vplus \\ 
%\hline 
%$x\in \left[ -6 ; \frac{2}{7}\right]$ & $-6  \leq x \leq  \frac{2}{7} $  &  \vplus \\ 
%\hline 
% & $-3 \leq x <5$ &  \vplus  \\ 
%\hline 
%$x\in \left[ 5 ; 8 \right[ $  &  &  \vplus  \\ 
%\hline 
% &  & \definecolor{ffdxqq}{rgb}{1.,0.8431372549019608,0.}
%\definecolor{ffxfqq}{rgb}{1.,0.4980392156862745,0.}
%\begin{tikzpicture}[line cap=round,line join=round,>=triangle 45,x=1.0cm,y=1.0cm]
%\draw[->,color=black] (-5.174092090680384,0.) -- (2.566282833730012,0.);
%\foreach \x in {-5.,-4.,-3.,-2.,-1.,1.,2.}
%\draw[shift={(\x,0)},color=black] (0pt,2pt) -- (0pt,-2pt) node[below] {\footnotesize $\x$};
%\draw[color=black] (0pt,-10pt) node[right] {\footnotesize $0$};
%\clip(-5.174092090680384,-0.4115875953650586) rectangle (2.566282833730012,0.4791698364123281);
%\draw [line width=2.4pt,color=ffxfqq] (-4.,0.)-- (1.,0.);
%\draw [color=ffdxqq](-4.122250584071695,0.25) node[anchor=north west] {]};
%\draw [color=ffdxqq](0.9 ,0.25) node[anchor=north west] {]};
%\end{tikzpicture}  \vplus \\ 
%\hline 
%\end{tabular} 
 
\end{ExoCad}


\begin{ExoCad}{Modéliser.}{1234}{0}{0}{0}{0}{0}


Déterminer les intersections des ensembles $A$ et $B$ suivants.   $A \cap B = $ se lit $A$ inter $B$.
\begin{enumerate}
\item $A = \lbrace 1;2;3;4;5;6\rbrace$ et $B = \lbrace 1;2 \rbrace$. $A \cap B = $\point{1}
\item $A = \lbrace -4;-2;-1;0;1;2;\rbrace$ et $B = \lbrace -2;1;2 \rbrace$.  $A \cap B = $\point{1}
\item $A = \N$ et $B = \R$. $A \cap B = $ \point{1}
\item $A = [-4;3[$ et $B =[-2;7]$. $A \cap B = $ \point{1}
\item $A = [-2;1]$ et $B =[2;3]$. $A \cap B = $ \point{1}
\item $A = \N$ et $B =]-\infty;5]$. $A \cap B = $ \point{1}
\end{enumerate} 
 
\end{ExoCad}

\begin{ExoCad}{Modéliser.}{1234}{0}{0}{0}{0}{0}


On propose dans chaque cas deux ensembles. Lequel est inclus dans l'autre ?  

\begin{enumerate}
\item $[-1,1;3]$ et $]-2,9;6]$
\item $[0,7;0,8]$ et $[0,5;+\infty[$
\item $]1;2[$ et $[1;2]$
\item $\Q$ et $\Z$
\end{enumerate} 
 
\end{ExoCad}


\begin{ExoCad}{Modéliser.}{1234}{0}{0}{0}{0}{0}


Déterminer l'ensemble des valeurs de $x$ dans chaque cas.
\begin{enumerate}
\item On jette un dé à 6 face et on regarde la face obtenue. Soit $x$ le numéro de la face. 
\item Le segment $[AB]$ mesure 8 cm. Soit I le milieu de $[AB]$ et $M$ un point de $[AI]$. $AM = x$. 
\item $x < -4$ et $x \geq 10$
\item $x \leq 6$ et $x \leq 3$
\item $x \leq 6$ ou $x \geq 3$
\end{enumerate} 
 
\end{ExoCad}

\begin{ExoCad}{Modéliser.}{1234}{0}{0}{0}{0}{0}


Déterminer les réunions des ensembles suivants. On écrira : $A \cup B = $ où $A$ et $B$ sont les ensembles ci-dessous.
\begin{enumerate}
\item $\Z$ et $\R$
\item $\left\lbrace 1;2;8;6  \right\rbrace $ et $\left\lbrace 0;2;4;8  \right\rbrace $
\item $[-2;1]$ et $[2;3]$
\item $[0;+\infty[$ et $]-\infty;5]$
\end{enumerate} 
 
\end{ExoCad}

\begin{ExoCad}{Représenter.}{1234}{0}{0}{0}{0}{0}


Représenter graphiquement dans le plan muni d'un repère orthonormal 
 
\begin{enumerate}
\item l'ensemble des points $M(x;y)$ tes que  $1 < x < 4$ et $-2 \leq y <4$.
\item l'ensemble des points $M(x;y)$ tes que  $1 \leq  2x+1 \leq  5$ et $-2 \leq 3y + 4  \leq  13$.
\end{enumerate} 
 
\end{ExoCad}

\begin{ExoCad}{Modéliser.}{1234}{0}{0}{0}{0}{0}


L'INSEE estime qu'un couple avec deux enfants appartient à la classe moyenne quand les revenus du foyer sont situés dans l'intervalle [3253;5609].

M.Twicks gagne 2731 euros et madame Twicks gagne 2952 euros et ils ont deux enfants. La famille appartient-elle à la classe moyenne ? 
 
\end{ExoCad}

\begin{ExoCad}{Représenter. Raisonner. Communiquer.}{1234}{0}{0}{0}{0}{0}


Soit $x$ un réel.Écrire sous forme d'intervalle ou de réunion d'intervalles le plus grand ensemble auquel appartient $x$.


\begin{enumerate}
	\item $x \geq 1$ ou $x<3$.  
	\item $x \geq 1$ et $x<3$.  
\end{enumerate}
 
 
\end{ExoCad}

\begin{ExoCad}{Représenter.}{1234}{0}{0}{0}{0}{0}


On donne le programme en Python ci dessous. 
 
\begin{lstlisting}
def is_in(x,a,b):
    if x > a and x < b :
    	test = "{} is in  ]{};{}[".format(x,a,b) 
    else :
        test = "{} is not in ]{};{}[".format(x,a,b) 
    return test    
x=int(input("Entrer un nombre  :")) 
a=int(input("Entrer la borne inf :"))
b=int(input("Entrer la borne sup :"))    
print(is_in(x,a,b))
\end{lstlisting}
 


\begin{enumerate}
\item Ouvrir le logiciel PyScripter et taper ce code. Que fait ce programme ? Vous pouvez aussi ouvrir en ligne l'éditeur Python : \url{https://www.tutorialspoint.com/execute_python_online.php}
\item Modifier ce programme pour qu'il teste si un nombre $x$ appartient à l'intervalle $[a;b]$.
\end{enumerate} 
 
\end{ExoCad}

\begin{ExoCad}{Représenter. Raisonner. Communiquer.}{1234}{0}{0}{0}{0}{0}


Soit $x$ un réel.Écrire sous forme d'intervalle ou de réunion d'intervalles le plus grand ensemble auquel appartient $x$.


\begin{enumerate}
	\item $x \geq 1$ ou $x<3$.  
	\item $x \geq 1$ et $x<3$.  
\end{enumerate}
 
 
\end{ExoCad}

\begin{ExoCad}{Représenter. Raisonner. Communiquer.}{1234}{0}{0}{0}{0}{0}


Soit $x$ un réel. Écrire sous forme d'intervalle ou de réunion d'intervalles le plus grand ensemble auquel appartient $x$.

\begin{enumerate}
	\item $7x -4 \geq 3$ ou $1-x>0$  
	\item $x \leq 2$ ou $-4x \leq 20$ 
	\item $x < 3$ et $x > -6$ 
	\item $2x+1 \leq 3$ et $3x-1 \geq 0$ 
	\item $3(2-a)<3$ et $a-1 \geq 2$ 
\end{enumerate}
 
 
 
\end{ExoCad}

\begin{ExoCad}{Représenter. Raisonner. Communiquer.}{1234}{0}{0}{0}{0}{0}


Déterminer l'ensemble des valeurs de $x$ dans chaque cas.
\begin{enumerate}
\item On jette un dé à 6 face et on regarde la face obtenue. Soit $x$ le numéro de la face. 
\item $[-1,1;3]$ et $[2,9;6]$
\item $x > -4$ et $x \leq 10$
\item $x \leq -3$ et $x \leq 5$
\item $x \leq 5$ ou $x \geq 2$
\end{enumerate} 
 
\end{ExoCad}


\begin{ExoCad}{Représenter. Raisonner. Communiquer.}{1234}{0}{0}{0}{0}{0}


On propose dans chaque cas deux ensembles. Lequel est inclus dans l'autre ? Écrire ensuite une phrase :" $x$ appartient à .... donc $x$ appartient à ....."

\begin{enumerate}
\item $\left[ -\frac{11}{10};\frac{29}{10}\right]$ et $\left[-\frac{3}{2};3 \right]$
\item $\left[ \frac{1}{2}; +\infty \right[$ et $[0,7;0,8]$.
\item $[1;2]$ et $]1;2[$. 
\end{enumerate} 
 
\end{ExoCad}


\begin{ExoCad}{Représenter. Raisonner. Communiquer.}{1234}{0}{0}{0}{0}{0}


Déterminer les intersections des ensembles suivants. On écrira : $A \cap B = $ où $A$ et $B$ sont les ensembles ci-dessous.
 

\textit{{\small On pourra représenter chaque intervalle sur une droite graduée.}}



\begin{minipage}{0.48\linewidth}

\begin{enumerate}
\item $\Z$ et $\Q$
\item $[-5;2[$ et $[0;7]$
\item $[-1;4]$ et $[-3;-1]$
\item $\N$ et $]-\infty;5]$
\item $[-5;0[$ et $[0;3]$
\end{enumerate}

\end{minipage}
\hfill
\begin{minipage}{0.48\linewidth}
 
\begin{enumerate}
\item

\begin{tikzpicture}[line cap=round,line join=round,>=triangle 45,x=1.0cm,y=1.0cm]
\draw [->,line width=1.pt,domain=0.34:6.36] plot(\x,{(-14.-0.*\x)/7.});
\end{tikzpicture}
\item

\begin{tikzpicture}[line cap=round,line join=round,>=triangle 45,x=1.0cm,y=1.0cm]
\draw [->,line width=1.pt,domain=0.34:6.36] plot(\x,{(-14.-0.*\x)/7.});
\end{tikzpicture}
\item

\begin{tikzpicture}[line cap=round,line join=round,>=triangle 45,x=1.0cm,y=1.0cm]
\draw [->,line width=1.pt,domain=0.34:6.36] plot(\x,{(-14.-0.*\x)/7.});
\end{tikzpicture}
\item

\begin{tikzpicture}[line cap=round,line join=round,>=triangle 45,x=1.0cm,y=1.0cm]
\draw [->,line width=1.pt,domain=0.34:6.36] plot(\x,{(-14.-0.*\x)/7.});
\end{tikzpicture}
\item

\begin{tikzpicture}[line cap=round,line join=round,>=triangle 45,x=1.0cm,y=1.0cm]
\draw [->,line width=1.pt,domain=0.34:6.36] plot(\x,{(-14.-0.*\x)/7.});
\end{tikzpicture}
\end{enumerate}

\end{minipage} 
 
\end{ExoCad}

\begin{ExoCad}{Représenter. Raisonner. Communiquer.}{1234}{0}{0}{0}{0}{0}


Déterminer, dans chaque cas, la réunion des ensembles suivants. On écrira : $A \cup B = $ où $A$ et $B$ sont les ensembles ci-dessous.
\begin{enumerate}
\item $A=\left\lbrace 1;3;5;7  \right\rbrace $ et $B=\left\lbrace 0;2;4;5;7;8  \right\rbrace $\point{1}
\item $A=[-3;4]$ et $B=[2;6]$\point{1}
\item $A=[0;+\infty[$ et $B=]-\infty;5]$\point{1}
\end{enumerate}
On pourra représenter les intervalles sur une droite graduée tracée à main levée. 
 
\end{ExoCad}

\begin{ExoCad}{Représenter. Communiquer..}{1234}{0}{0}{0}{0}{0}


\begin{enumerate}
\item On considère le nombre $\frac{19}{11}$.

\begin{enumerate}
\item Donner le développement décimal de $\frac{19}{11}$ avec 8 chiffres significatifs. $\frac{19}{11}$ semble-t-il décimal ?
\item On dit que $\frac{19}{11}$ a une écriture périodique.
Préciser sa période (série de chiffres qui se répète à l'infini dans le développement décimal).
\end{enumerate}
\item On considère le nombre $x=0,13131313....$ dont le développement décimal a pour période 13.
\begin{enumerate}
\item Démontrer que $100x = 13 + x$. 
\item  En déduire une écriture fractionnaire de $x$. Quelle est la nature du nombre $x$ ?
\end{enumerate}
\item Démontrer que $x=3,412412412...$ est un nombre rationnel. 
\item Estimer le résultat avec la calculatrice.
\end{enumerate}

  
 
\end{ExoCad}


 
\end{pageAD}


%%%%%%%%%%%%%%%%%%%%%%%%%%%%%%%%%%%%%%%%%%%%%%%%%%%%%%%%%%%%%%%%%%%
%%%%  Niveau 1
%%%%%%%%%%%%%%%%%%%%%%%%%%%%%%%%%%%%%%%%%%%%%%%%%%%%%%%%%%%%%%%%%%%
\begin{pageExercices} 


 %%%%%%%%%%%%%%%%%%%%%%%%%%%
\begin{ExoCuN}{Représenter.}{2}{0}{0}{0}{0}

Démontrer que $0,\underline{1}$ est un nombre rationnel à préciser.
 
\end{ExoCuN}
 
%%%%%%%%%%%%%%%%%%%%%%%%%%%
\begin{ExoCuN}{Représenter.}{2}{0}{0}{0}{0}
Je suis un nombre à trois chiffres non nuls. Je suis divisible par 94. Changez l'ordre de mes chiffres et je deviens divisible par 49.
Qui suis-je ? 
\end{ExoCuN}


\begin{ExoCdN}{Chercher.communiquer.}{2}{0}{0}{0}{0}
 
Dans chaque cas, trouver, lorsque cela est possible, le nombre $x$ qui remplit les critères suivants :
\begin{enumerate}
\item $x \in \Q$ et $x \not\in \Z$
\item $x \in \R$ et $x \not\in \N$. 
\end{enumerate}
\end{ExoCdN}

\begin{ExoCdN}{Représenter. Raisonner.}{2}{0}{0}{0}{0}

Démontrer que $\sqrt{2}$ est un irrationnel ou encore que $\sqrt{2} \in \R\setminus\Q$

\end{ExoCdN}

%%%%%%%%%%%%%%%%%%%%%%%%%%%%%%%%%%%%%%%%%%%%%%%%%%%%%%%%%%%%%%%%%%%
\begin{ExoCtN}{Représenter.}{2}{1}{0}{0}{0}

\begin{enumerate}
\item Démontrer que $0,\underline{12}$ est un nombre rationnel à préciser.
\item Démontrer que $0,\underline{485}$ est un nombre rationnel à préciser.
\end{enumerate}
\end{ExoCtN}

%%%%%%%%%%%%%%%%%%%%%%%%%%%%%%%%%%%%%%%%%%%%%%%%%%%%%%%%%%%%%%%%%%%
\begin{ExoCtN}{Raisonner.}{1}{0}{0}{0}{0}

\begin{enumerate}
\item Démontrer que tout entier $n$ multiple de $9$ est aussi un multiple de $3$.
\item Démontrer que si $m$ est un multiple de $6$ alors $m$ est aussi un multiple de $3$.
\end{enumerate}
\end{ExoCtN}


\begin{ExoCtN}{Représenter.}{1}{0}{0}{0}{0}

Dans un pays où le système monétaire n’est constitué que de pièces de 3 et de 5, il s’agit d’aider les habitants en créant un programme  qui donne le nombre de pièces nécessaires à tout achat d’un montant entier supérieur ou égal à 8.

\hfill{{\footnotesize Source : d’après PISA, items libérés}}

\end{ExoCtN}


\end{pageExercices}

  
%%%%%%%%%%%%%%%%%%%%%%%%%%%%%%%%%%%%%%%%%%%%%%%%%%%%%%%%%%%%%%%%%%%
%%%%  Niveau 2
%%%%%%%%%%%%%%%%%%%%%%%%%%%%%%%%%%%%%%%%%%%%%%%%%%%%%%%%%%%%%%%%%%%



%\begin{pageParcoursd} 
% 
%%%%%%%%%%%%%%%%%%%%%%%%%%%%%%%%%%%%%%%%%%%%%%%%%%%%%%%%%%%%%%%%%%%%
%
%
% 
%%%%%%%%%%%%%%%%%%%%%%%%%%%%%%%%%%%%%%%%%%%%%%%%%%%%%%%%%%%%%%%%%%%%
%
%
%
%%%%%%%%%%%%%%%%%%%%%%%%%%%%%%%%%%%%%%%%%%%%%%%%%%%%%%%%%%%%%%%%%%%%
%
%
% %%%%%%%%%%%%%%%%%%%%%%%%%%%%%%%%%%%%%%%%%%%%%%%%%%%%%%%%%%%%%%%%%%%
%\begin{ExoCd}{Représenter. Raisonner.}{1234}{2}{0}{0}{0}{0}
%
%
%\end{ExoCd}
% 
%%%%%%%%%%%%%%%%%%%%%%%%%%%%%%%%%%%%%%%%%%%%%%%%%%%%%%%%%%%%%%%%%%%%
%\begin{ExoCd}{Représenter. Raisonner.}{1234}{2}{0}{0}{0}{0}
%
%
%\end{ExoCd}
% 
%\end{pageParcoursd}
%
%%%%%%%%%%%%%%%%%%%%%%%%%%%%%%%%%%%%%%%%%%%%%%%%%%%%%%%%%%%%%%%%%%%%
%%%%%  Niveau 3
%%%%%%%%%%%%%%%%%%%%%%%%%%%%%%%%%%%%%%%%%%%%%%%%%%%%%%%%%%%%%%%%%%%%
%\begin{pageParcourst}
%
%
%
%
%%%%%%%%%%%%%%%%%%%%%%%%%%%%%%%%%%%%%%%%%%%%%%%%%%%%%%%%%%%%%%%%%%%%
%\begin{ExoCt}{Raisonner.}{1234}{2}{0}{0}{0}{0}
% 
%\end{ExoCt}
%
%%%%%%%%%%%%%%%%%%%%%%%%%%%%%%%%%%%%%%%%%%%%%%%%%%%%%%%%%%%%%%%%%%%%
%\begin{ExoCt}{Représenter.}{1234}{2}{0}{0}{0}{0}
%
% 
%
%\end{ExoCt}
%
%%%%%%%%%%%%%%%%%%%%%%%%%%%%%%%%%%%%%%%%%%%%%%%%%%%%%%%%%%%%%%%%%%%%
%\begin{ExoCt}{Représenter.}{1234}{2}{0}{0}{0}{0}
%
% 
%
%\end{ExoCt} 
% 
%\end{pageParcourst}
%
%%%%%%%%%%%%%%%%%%%%%%%%%%%%%%%%%%%%%%%%%%%%%%%%%%%%%%%%%%%%%%%%%%%%
%%%%%  Brouillon
%%%%%%%%%%%%%%%%%%%%%%%%%%%%%%%%%%%%%%%%%%%%%%%%%%%%%%%%%%%%%%%%%%%%


\begin{pageBrouillon} 
 
\ligne{32}



\end{pageBrouillon}

%%%%%%%%%%%%%%%%%%%%%%%%%%%%%%%%%%%%%%%%%%%%%%%%%%%%%%%%%%%%%%%%%%%
%%%%  Auto
%%%%%%%%%%%%%%%%%%%%%%%%%%%%%%%%%%%%%%%%%%%%%%%%%%%%%%%%%%%%%%%%%%%


%%%%%%%%%%%%%%%%%%%%%%%%%%%%%%%%%%%%%%%%%%%%%%%%%%%%%%%%%%%%%%%%%%%
\begin{pageAuto} 


\begin{ExoAuto}{Raisonner.}{1234}{2}{0}{0}{0}{0}

 
%%%%%%%%%%%%%%%%%%%%%%%%%%%%%%%%%%%%%%%%%%%%%%%%%%%%%%%%%%%%%%%%%%%
\end{ExoAuto}

\begin{ExoAuto}{Raisonner.}{1234}{2}{0}{0}{0}{0}
  

\end{ExoAuto}

%%%%%%%%%%%%%%%%%%%%%%%%%%%%%%%%%%%%%%%%%%%%%%%%%%%%%%%%%%%%%%%%%%%
\begin{ExoAuto}{Raisonner.}{1234}{2}{0}{0}{0}{0}

 
 

\end{ExoAuto}

 
%%%%%%%%%%%%%%%%%%%%%%%%%%%%%%%%%%%%%%%%%%%%%%%%%%%%%%%%%%%%%%%%%%%
\begin{ExoAuto}{Raisonner.}{1234}{2}{0}{0}{0}{0}

 
 

\end{ExoAuto}


\end{pageAuto}
