\chapter{Arithmétique}
{https://sacado.xyz/qcm/parcours_show_course/0/117129}
{


 \begin{CpsCol}
\textbf{Les savoir-faire du parcours}
 \begin{itemize}
 \item \textbf{Utiliser des nombres pour calculer et résoudre des problèmes}
 \item Connaitre les bases de l'arithmétique
 \item Simplifier une fraction pour la rendre irréductible
 \end{itemize}
 \end{CpsCol}

\begin{His}

  
L'arithmétique est une branche des mathématiques qui correspond à la science des nombres. De nombreux nombres entiers ont des propriétés particulières. Ces propriétés font l'objet de la théorie des nombres. Parmi ces nombres particuliers, les nombres premiers sont sans doute les plus importants.

On connait aussi les nombres pairs et les nombres impairs. 

\end{His}

 

\begin{ExoDec}{Chercher.}{1234}{1}{0}{0}{0}

Pour fêter les 25 ans de sa boutique, un chocolatier souhaite offrir aux premiers clients de la journée une boîte contenant des truffes au chocolat.
Il a confectionné 300 truffes : 125 truffes parfumées au café et 175 truffes enrobées de noix de coco. Combien y aura-t-il de truffes de chaque sorte dans chaque boîte ?
 
\end{ExoDec}



}

\begin{pageCours}


\section{Les entiers naturels et entiers relatifs}

\begin{DefT}{Ensemble de nombres}

\begin{enumerate}
\item On appelle \textbf{entiers naturels} les nombres : 0 ; 1 ; 2 ; 3 . . . Leur ensemble est noté $\N$.\index{Ensemble de nombres! Entiers naturels $\N$}. Parfois, on dit abusivement que les nombres entiers sont des nombres sans partie décimale.
On a donc : $\N =  \lbrace 0 ; 1 ; 2 ; 3 \cdots \rbrace $
\item  On appelle \textbf{entiers relatifs} les nombres entiers naturels et leur symétrique par rapport à 0. Leur ensemble est noté $\Z$.\index{Ensemble de nombres! Entiers $\Z$}. L'utilisation de la lettre $\Z$ vient de l'allemand Zahlen (Chiffre).
On a donc : $\Z = \lbrace \cdots -3 ; -2 ; -1 ; 0 ; 1 ; 2 ; 3  \cdots  \rbrace$
\end{enumerate}
\end{DefT}

 
\section{Multiples et diviseurs}

\begin{minipage}{0.5\linewidth}

\begin{DefT}{Multiple}\index{Multiple!Nombres}

Soit $a$ un nombre entier. Le nombre $m$ est dit \textbf{multiple} de $a$ s'il existe un entier $k \in \Z$ tel que $m=ka$.
\end{DefT}
\end{minipage}
\begin{minipage}{0.5\linewidth}
\begin{Ex} 

$5 \times 7 = 35$ où $7 \in \Z$ donc $35$ est un multiple de $5$ et aussi,  $35$ est un multiple de $7$.
\end{Ex}
\end{minipage}


\begin{minipage}{0.5\linewidth}
\begin{DefT}{Diviseur}\index{Diviseur!Nombres}

Soit $n$ un nombre entier. Le nombre $q$ est dit \textbf{diviseur} de $n$ s'il existe un entier $k \in \Z$ tel que $n=kq$.
\end{DefT}
\end{minipage}
\begin{minipage}{0.5\linewidth}
\begin{Ex} 

$48 = 6 \times 8$ où $6 \in \Z$  donc $8$ est un diviseur de $48$ et aussi,  $6$ est diviseur de $48$.
\end{Ex}
\end{minipage}

\begin{minipage}{0.5\linewidth}
\begin{DefT}{Nombre premier}\index{Nombre premier}

Un \textbf{nombre premier} est un nombre entier supérieur à 2 avec exactement 2 diviseurs : 1 et lui-même. 

\end{DefT}
\end{minipage}
\begin{minipage}{0.5\linewidth}
\begin{Ex} 

$19$ est un nombre premier. Il n'est divisible que par 1 et lui-même. 
\end{Ex}
\end{minipage}


\begin{minipage}{0.5\linewidth}
\begin{DefT}{Nombre pair}\index{Nombre pair}

Un \textbf{nombre pair} est un nombre entier divisible par 2. \\  Soit $n$ un nombre pair, $n=2\times k$ avec $k\in\Z$.

\end{DefT}
\end{minipage}
\begin{minipage}{0.5\linewidth}
\begin{Ex} 
$46 = 2 \times 23$ et $23\in\Z$ donc $46$ est un nombre pair.

$15= 2 \times 7,5$. Comme $7,5 \not\in\Z$, $15$ n'est pas pair.
\end{Ex}
\end{minipage}


\begin{minipage}{0.5\linewidth}
\begin{DefT}{Nombres premiers entre eux}\index{Premiers entre eux}
\begin{enumerate}[leftmargin=*]
\item Tout nombre se décompose de façon unique comme produit de facteurs premiers.
\item Deux nombres entiers $a$ et $b$ sont premiers entre eux lorsque leurs seuls diviseurs communs sont 1 ou $-1$.
\end{enumerate}
\end{DefT}
\end{minipage}
\begin{minipage}{0.5\linewidth}
\begin{Att}
 
$8=2^3$ et $15=3 \times 5$.
Donc $8$ et $15$ n'ont pas de diviseurs communs. \textbf{$8$ et $15$ sont donc premiers entre eux}. Pourtant $8$ et $15$ ne sont pas premiers.
  
\end{Att}
\end{minipage}




\section{Logique}

\begin{minipage}{0.5\linewidth}
\begin{DefT}{Proposition universelle}\index{Proposition universelle}

Une proposition universelle est une proposition qui inclut tous les cas existants de l'univers. 

\end{DefT}
\end{minipage}
\begin{minipage}{0.5\linewidth}
\begin{Ex} 

Quelque soit trois nombres $a, b, c$, $a(b+c)=ab+ac$. La distributivité est une proposition universelle.

\end{Ex}
\end{minipage}

\begin{minipage}{0.5\linewidth}
\begin{DefT}{Contre-exemple}\index{Contre-exemple}

Un \textbf{contre-exemple} est un cas particulier qui vient contredire une propriété universelle. 

\end{DefT}

\end{minipage}
\begin{minipage}{0.5\linewidth}

\begin{Ex} 

Considérons une proposition universelle : Tous les nombres sont pairs. Pour démontrer que cette proposition est fausse, il suffit de démontrer qu'un seul nombre n'est pas pair. La contradiction vient sur le mot \textbf{tous}.

$3=2\times1,5$ et $1,5 \not\in\Z$ donc $3$ n'est pas pair. \textbf{Donc tous les nombres ne sont pas pairs.}

\end{Ex}
\end{minipage}




\end{pageCours} 
\begin{pageAD} 
 

\Sf{Utiliser les multiples et les diviseurs}

\begin{ExoCad}{Calculer.}{1234}{0}{0}{0}{0}{0}
\begin{enumerate}
\item Déterminer les 10 premiers multiples de 4 : \point{1}
\item Déterminer les 10 premiers multiples de 6 : \point{1}
\item Déterminer les multiples communs de 4 et de 6 : \point{1}
\end{enumerate}
\end{ExoCad}


\begin{ExoCad}{Calculer.}{1234}{0}{0}{0}{0}{0}

$51$ est-il un nombre premier ? Justifier. 

\point{2}

\end{ExoCad}


\begin{ExoCad}{Calculer.}{1234}{0}{0}{0}{0}{0}

Décomposer 24 en produit de facteurs premiers.

\point{3}

\end{ExoCad}



\begin{ExoCad}{Calculer.}{1234}{0}{0}{0}{0}{0}

Simplifie la fraction $\dfrac{735}{840}$ 

\point{3}

\end{ExoCad}



\begin{ExoCad}{Calculer.}{1234}{0}{0}{0}{0}{0}

Soit $a$ un entier. La somme de deux multiples de $a$ est un multiple de $a$. \point{3}

\end{ExoCad}


\begin{ExoCad}{Calculer.}{1234}{0}{0}{0}{0}{0}
Vrai ou faux : Quel que soit l'entier $n$, $2n-1$ est un nombre premier. Justifier.\point{2}
\end{ExoCad}


 
\begin{ExoCad}{Raisonner.}{1234}{0}{0}{0}{0}{0}

Les affirmations sont-elles vraies ou fausses ?
\begin{enumerate}[leftmargin=*]
\item La différence de deux nombres entiers naturels est un entier naturel. \point{1}
\item Le quotient de deux nombres décimal est un nombre décimal. \point{1}
\item Le quotient de deux nombres premiers distincts peut être un entier relatif. \point{1}
\item Le quotient de deux nombres premiers distincts peut être un nombre décimal. \point{1}
\end{enumerate} 
 
 \end{ExoCad}
 
\end{pageAD}


%%%%%%%%%%%%%%%%%%%%%%%%%%%%%%%%%%%%%%%%%%%%%%%%%%%%%%%%%%%%%%%%%%%
%%%%  Niveau 1
%%%%%%%%%%%%%%%%%%%%%%%%%%%%%%%%%%%%%%%%%%%%%%%%%%%%%%%%%%%%%%%%%%%
\begin{pageParcoursu} 



 
%%%%%%%%%%%%%%%%%%%%%%%%%%%
\begin{ExoCu}{Représenter.}{1234}{2}{0}{0}{0}{0}

\begin{enumerate}
\item Décomposer $186$ et $155$ en produit de facteurs premiers. \point{3}
\item Déterminer le PGCD de $186$ et $155$. \point{2}
\item Un chocolatier a fabriqué $186$ pralines et $155$ chocolats.
Les colis sont constitués ainsi :
\begin{description}
\item Le nombre de pralines est le même dans chaque colis.
\item Le nombre de chocolats est le même dans chaque colis.
\item Tous les chocolats et toutes les pralines sont utilisés.
\end{description}
\begin{enumerate}
\item Quel nombre maximal de colis pourra-t-il réaliser ?  \point{3}
\item Combien y aura-t-il de chocolats et de pralines dans chaque colis.   \point{3}
\end{enumerate}
\end{enumerate}
  
\end{ExoCu}


\begin{ExoCuN}{Raisonner.}{1}{0}{0}{0}{0}
Simplifier la fraction $\dfrac{2310}{2730}$ pour la rendre irréductible. \point{2}
\end{ExoCuN}


\begin{ExoCuN}{Raisonner.}{1}{0}{0}{0}{0}
Choisir 2 nombres entiers consécutifs. Vérifier que leur produit est pair. \point{2}
\end{ExoCuN}

\begin{ExoCuN}{Chercher.}{1}{0}{0}{0}{0}
Proposer 2 nombres premiers entre eux. \point{2}
\end{ExoCuN}

\begin{ExoCuN}{Chercher. Raisonner.}{1}{0}{0}{0}{0}
Les produits de deux nombres premiers est-il un nombre premier ? Justifier. \point{2}
\end{ExoCuN}


\begin{ExoCuN}{Chercher.}{2}{0}{0}{0}{0} 
Je suis un nombre à trois chiffres non nuls. Je suis divisible par 94. Changez l'ordre de mes chiffres et je deviens divisible par 49.
Qui suis-je ?   \point{5}
\end{ExoCuN}


\end{pageParcoursu} 
 
%%%%%%%%%%%%%%%%%%%%%%%%%%%%%%%%%%%%%%%%%%%%%%%%%%%%%%%%%%%%%%%%%%%
%%%%  Niveau 2
%%%%%%%%%%%%%%%%%%%%%%%%%%%%%%%%%%%%%%%%%%%%%%%%%%%%%%%%%%%%%%%%%%%
\begin{pageParcoursd} 



\begin{ExoCdN}{Représenter.}{2}{0}{0}{0}{0}
On veut démontrer que la proposition $\mathcal P$ suivante : "La somme de deux nombres impairs est un nombre pair".

\begin{enumerate}
\item Calculer $a=5+7$. La proposition $\mathcal P$ semble-t-elle vraie ?\point{2}
\item Soit $k$ et $q$ deux nombres relatifs. Le nombre impair $n$ s'écrit $n=2k+1$. Exprimer un autre nombre impair $m$ en fonction de $q$.\point{3}
\item Calculer $n+m$ en fonction de $k$ et $q$. \point{3}
\item En déduire que la somme $n+m$ est un nombre pair. \point{3}
\end{enumerate}

\end{ExoCdN}

 
\begin{ExoCdN}{Représenter.}{1}{1}{0}{0}{0}

Démontrer que si $p$ est impair alors $p^2$ est impair.

 \point{6}
 
\end{ExoCdN}

 %%%%%%%%%%%%%%%%%%%%%%%%%%%%%%%%%%%%%%%%%%%%%%%%%%%%%%%%%%%%%%%%%%%
\begin{ExoCdN}{Raisonner.}{1}{0}{0}{0}{0}

Démontrer que tout nombre entier $n$ multiple de $9$ est un multiple de $3$.  \point{6}

\end{ExoCdN}

\begin{ExoCdN}{Raisonner.}{1}{0}{0}{0}{0}
Montrer que la somme de trois entiers consécutifs est toujours un multiple de 3. \point{6}
\end{ExoCdN}


\end{pageParcoursd}
 
%
%%%%%%%%%%%%%%%%%%%%%%%%%%%%%%%%%%%%%%%%%%%%%%%%%%%%%%%%%%%%%%%%%%%%
%%%%%  Niveau 3
%%%%%%%%%%%%%%%%%%%%%%%%%%%%%%%%%%%%%%%%%%%%%%%%%%%%%%%%%%%%%%%%%%%%
\begin{pageParcourst}



 
\begin{ExoCtN}{Raisonner.}{1}{1}{0}{0}{0}

Démontrer que si $p$ est pair alors $p^2$ est pair.

 \point{5}
 
\end{ExoCtN}


\begin{ExoCtN}{Raisonner.}{1}{0}{0}{0}{0}
On donne le programme en Python ci dessous. 
 
\begin{lstlisting}[language=Python] 
def is_divisible(x,y):
    if x%y == 0 :
    	test = "{} est divisible par {}".format(x,y) 
    else :	
        test = "{} n'est pas divisible par {}".format(x,y)
    return test    
    
n=int(input("Entrer un nombre n :"))
    
print(is_divisible(n,4))
\end{lstlisting}
 
\begin{enumerate}
\item Que fait ce programme ? on tester le programme écrit en Python avec l'éditeur : \url{https://sacado.xyz/tool/show/18}  \point{3}
\item Modifier directement le programme pour qu'il teste si un nombre $a$ divise $n$.   
\end{enumerate}
 
\end{ExoCtN}



%%%%%%%%%%%%%%%%%%%%%%%%%%%%%%%%%%%%%%%%%%%%%%%%%%%%%%%%%%%%%%%%%%%
\begin{ExoCtN}{Raisonner.}{2}{1}{0}{0}{0}
 
 
Dans un pays où le système monétaire n'est constitué que de pièces de 3 et de 5, il s'agit d'aider les habitants en créant un algorithme  qui donne le nombre de pièces nécessaires à tout achat d'un montant entier supérieur ou égal à 8. 

Pour tester l'algorithme, on peut utiliser l'éditeur Python : \url{https://sacado.xyz/tool/show/18}

\hfill{{\footnotesize Source : d’après PISA, items libérés}}
 
\end{ExoCtN}

 %%%%%%%%%%%%%%%%%%%%%%%%%%%%%%%%%%%%%%%%%%%%%%%%%%%%%%%%%%%%%%%%%%%
\begin{ExoCtN}{Raisonner.}{2}{1}{0}{0}{0}
 
 \begin{minipage}{0.5\linewidth} 
 
\textbf{Le crible d'Eratosthène}

L'algorithme procède par élimination : il s'agit de supprimer d'une table d'entiers tous les multiples d'un entier $n$ (autres que lui-même).

En supprimant tous ces multiples, à la fin il ne restera que les entiers qui ne sont multiples d'aucun entier à part 1 et eux-mêmes, et qui sont donc les nombres premiers.

On commence par rayer les multiples de 2, puis les multiples de 3 restants, puis les multiples de 5 restants, et ainsi de suite en rayant à chaque fois tous les multiples du plus petit entier restant.

\begin{enumerate}

\item Faire fonctionner le crible sur la table ci-contre.
\item Quel est le résultat de ce crible ? \point{3}

\item Écrire un code en Python du crible d'Eratosthène : \url{https://sacado.xyz/tool/show/18}

\end{enumerate}

\end{minipage}
\begin{minipage}{0.5\linewidth}

 


\begin{tabular}{|c|c|c|c|c|c|c|c|c|c|}
 \hline 
 &  & 2 & 3 & 4 & 5 & 6 & 7 & 8 & 9 \\ 
 \hline 
 10&11 & 12 & 13 & 14 & 15 & 16 & 17 & 18 & 19 \\
 \hline 
 20&21 & 22 & 23 & 24 & 25 & 26 & 27 & 28 & 29 \\
 \hline 
 30&31 & 32 & 33 & 34 & 35 & 36 & 37 & 38 & 39 \\
 \hline 
 40&41 & 42 & 43 & 44 & 45 & 46 & 47 & 48 & 49 \\
 \hline 
 50&51 & 52 & 53 & 54 & 55 & 56 & 57 & 58 & 59 \\
 \hline 
 60&61 & 62 & 63 & 64 & 65 & 66 & 67 & 68 & 69 \\
 \hline 
 70&71 & 72 & 73 & 74 & 75 & 76 & 77 & 78 & 79 \\
 \hline 
 80&81 & 82 & 83 & 84 & 85 & 86 & 87 & 88 & 89 \\
 \hline 
 90&91 & 92 & 93 & 94 & 95 & 96 & 97 & 98 & 99 \\
 \hline 
 \end{tabular}  
 
 \end{minipage}


 
\end{ExoCtN}
 
\end{pageParcourst}
%
%%%%%%%%%%%%%%%%%%%%%%%%%%%%%%%%%%%%%%%%%%%%%%%%%%%%%%%%%%%%%%%%%%%%
%%%%%  Brouillon
%%%%%%%%%%%%%%%%%%%%%%%%%%%%%%%%%%%%%%%%%%%%%%%%%%%%%%%%%%%%%%%%%%%%


%%%%%%%%%%%%%%%%%%%%%%%%%%%%%%%%%%%%%%%%%%%%%%%%%%%%%%%%%%%%%%%%%%%
%%%%  Auto
%%%%%%%%%%%%%%%%%%%%%%%%%%%%%%%%%%%%%%%%%%%%%%%%%%%%%%%%%%%%%%%%%%%


%%%%%%%%%%%%%%%%%%%%%%%%%%%%%%%%%%%%%%%%%%%%%%%%%%%%%%%%%%%%%%%%%%%
\begin{pageAuto} 

 
%%%%%%%%%%%%%%%%%%%%%%%%%%%%%%%%%%%%%%%%%%%%%%%%%%%%%%%%%%%%%%%%%%%
\begin{ExoAutoN}{Raisonner.}{1}{0}{0}{0}{0}
Simplifier le nombre $a=\dfrac{60}{126}$ pour la rendre irréductible. \point{4}
\end{ExoAutoN}
%%%%%%%%%%%%%%%%%%%%%%%%%%%%%%%%%%%%%%%%%%%%%%%%%%%%%%%%%%%%%%%%%%%
%%%%%%%%%%%%%%%%%%%%%%%%%%%%%%%%%%%%%%%%%%%%%%%%%%%%%%%%%%%%%%%%%%%
\begin{ExoAutoN}{Raisonner.}{1}{0}{0}{0}{0}
Simplifier le nombre $b=\dfrac{12a+4}{8}$. \point{2}
\end{ExoAutoN}
%%%%%%%%%%%%%%%%%%%%%%%%%%%%%%%%%%%%%%%%%%%%%%%%%%%%%%%%%%%%%%%%%%%
\begin{ExoAutoN}{Raisonner.}{2}{0}{0}{0}{0}
Le produit de deux nombres impairs est-il impair ? \point{4}
\end{ExoAutoN}


%%%%%%%%%%%%%%%%%%%%%%%%%%%%%%%%%%%%%%%%%%%%%%%%%%%%%%%%%%%%%%%%%%%
\begin{ExoAutoN}{Raisonner.}{2}{0}{0}{0}{0}
Soit $n$ un entier.

Démontrer que la différence de deux multiples de $n$ est un multiple de $n$. \point{6}
\end{ExoAutoN}

%%%%%%%%%%%%%%%%%%%%%%%%%%%%%%%%%%%%%%%%%%%%%%%%%%%%%%%%%%%%%%%%%%%
\begin{ExoAutoN}{Raisonner.}{2}{0}{0}{0}{0}
Pour déterminer le PGCD de deux nombres $a$ et $b$, $a>b$, on effectue la division euclidienne de $a$ par $b$. On appelle $r_0$ le reste. \\
Puis on divise $b$ par $r_0$ et on appelle $r_1$ le reste. \\
On divise alors $r_0$ par $r_1$ et on appelle $r_2$ le reste.\\ 
On divise alors $r_1$ par $r_2$ et on appelle $r_3$ le reste. Et ainsi de suite. \\ 
Le PGCD de $a$ et de $b$ est alors le dernier reste non nul.\\ 
On appelle ce procédé, la méthode par divisions successives.

\begin{enumerate}[leftmargin=*]

\item Déterminer à l'aide de ce procédé le PGCD de $2 622$ et de $2 530$. \point{3}
\item 
\begin{enumerate}[leftmargin=*]
\item Un tapissier achète $2 622$ clous tête plate et $2 530$ clous tête ronde pour la fabrication de fauteuils identiques. Après la fabrication, il ne lui reste plus aucun clous. Quel est le plus grand nombre de fauteuil que le tapissier peut réaliser ? 
 \point{3}
\item Dans ce cas, quelle sera le nombre de chaque type de clou par fauteuil ? \point{2}
\end{enumerate}
\end{enumerate}
\end{ExoAutoN}

\end{pageAuto}
