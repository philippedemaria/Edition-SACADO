\chapter{Arithmétique}
{https://sacado.xyz/qcm/parcours_show_course/0/117129}
{


 \begin{CpsCol}
\textbf{Les savoir-faire du parcours}
 \begin{itemize}
 \item \textbf{Utiliser des nombres pour calculer et résoudre des problèmes}
 \item Connaitre les bases de l'arithmétique
 \item Simplifier une fraction pour la rendre irréductible
 \end{itemize}
 \end{CpsCol}

\begin{His}

  
L'arithmétique est une branche des mathématiques qui correspond à la science des nombres. De nombreux nombres entiers ont des propriétés particulières. Ces propriétés font l'objet de la théorie des nombres. Parmi ces nombres particuliers, les nombres premiers sont sans doute les plus importants.

On connait aussi les nombres pairs et les nombres impairs. 

\end{His}

 

\begin{ExoDec}{Chercher.}{1234}{1}{0}{0}{0}

Pour fêter les 25 ans de sa boutique, un chocolatier souhaite offrir aux premiers clients de la journée une boîte contenant des truffes au chocolat.
Il a confectionné 300 truffes : 125 truffes parfumées au café et 175 truffes enrobées de noix de coco, et toutes les boîtes devront être identiques : elles
devront contenir le même nombre de truffes café, et le même nombre
de truffes noix de coco.

Combien, au minimum,  y aura-t-il de truffes de chaque sorte dans chaque boîte ?
 
\end{ExoDec}



}

\begin{pageCours}


\section{Les nombres entiers naturels et nombres entiers relatifs}

\begin{DefT}{Entiers naturels et relatifs.}

\begin{enumerate}
\item On appelle \textbf{entiers naturels} les nombres : $0$ ; $1$ ; $2$ ; $3$ ; $\ldots$  Leur ensemble est noté $\N$.\index{Ensemble de nombres! Entiers naturels $\mathbb N$}, on a donc : $\mathbb N=  \lbrace 0 ; 1 ; 2 ; 3 \ldots \rbrace $
 
\item  On appelle \textbf{entiers relatifs} ou simplement \textbf{entiers} les nombres entiers naturels et leurs opposés. Leur ensemble est noté $\mathbb Z$ \index{Ensemble de nombres! Entiers $\mathbb Z$}(d'après le mot allemand Zahl qui signifie chiffre, nombre).
On a donc : $\mathbb Z = \lbrace \ldots -3 ; -2 ; -1 ; 0 ; 1 ; 2 ; 3  \cdots  \rbrace$
Parfois, on dit abusivement que les nombres entiers sont les nombres sans partie décimale.
\end{enumerate}
\end{DefT}

\begin{minipage}{0.6\linewidth}
\begin{Rqs} 
\begin{itemize}[leftmargin=*]
\item L'ensemble des nombres entiers $\Z$ contient l'ensemble des nombres naturels $\N$. 

On écrit $\mathbb N  \subset \mathbb Z$ et on dit que l'ensemble $\mathbb N$ est inclus dans $\mathbb Z$.

\item Si $x \in \mathbb N$ alors $x \in \mathbb Z$. Le symbole $\in$ signifie \textbf{appartient à}. 
\end{itemize}
\end{Rqs}
 
\end{minipage}
\begin{minipage}{0.4\linewidth} 
 
\definecolor{xfqqff}{rgb}{0.4980392156862745,0.,1.}
\definecolor{ffqqqq}{rgb}{1.,0.,0.}
\definecolor{ududff}{rgb}{0.30196078431372547,0.30196078431372547,1.}
\begin{tikzpicture}[line cap=round,line join=round,>=triangle 45,x=1.0cm,y=1.0cm]
\clip(-0.7,-1.22) rectangle (4.54,2.62);
\draw [rotate around={12.739646792482095:(2.15,0.66)},line width=2.pt,color=ffqqqq] (2.15,0.66) ellipse (1.5997803770827426cm and 1.08129424991489cm);
\draw [rotate around={18.434948822922024:(1.92,0.68)},line width=2.pt,color=xfqqff] (1.92,0.68) ellipse (2.576113244705657cm and 1.6710354423374485cm);
\draw [color=xfqqff](3.32,2.14) node[anchor=north west] {$\mathbb Z$};
\draw [color=ffqqqq](2.78,1.58) node[anchor=north west] {$\mathbb N$};
\draw (1.54,0.14) node[anchor=north west] {0};
\draw (1.96,0.94) node[anchor=north west] {1};
\draw (2.72,0.38) node[anchor=north west] {2};
\draw (1.06,-0.46) node[anchor=north west] {$-5$};
\draw (-0.18,1.36) node[anchor=north west] {$-2$};
\draw (0.02,0.02) node[anchor=north west] {$-1$};
\end{tikzpicture}
 
 
\end{minipage} 

\section{Multiples et diviseurs, nombre premier, nombre pair, nombre impair}

 

\begin{DefT}{Multiples et diviseurs}\index{Multiple!Nombres}\index{Diviseur!Nombres}

Soit $d$ un nombre entier. Le nombre $m$ est dit \textbf{multiple} de $d$ s'il existe un entier $q \in \Z$ tel que $m=qd$. Dans ce cas, on dit aussi que $d$ est un \textbf{diviseur} de $m$.
\end{DefT}


\begin{Exs} 
  $35=5 \times 7$ où $7 \in \Z$ donc $35$ est un multiple de $5$
  et $5$ est un diviseur de $35$. Comme $5 \in \Z$, on peut aussi
  dire que $35$ est un multiple de $7$ et que $7$ est un diviseur de
  $35$.
  1 est un diviseur de tous les entiers, et tous les entiers
  divisent 0.
\end{Exs}


 

\begin{DefT}{Nombre premier}\index{Nombre premier}

Un \textbf{nombre premier} est un nombre entier naturel qui a exactement 2 diviseurs positifs (qui sont alors 1 et lui-même) 

\end{DefT}


\begin{Ex} 

  $19$ est un nombre premier :  il n'est divisible que par 1 et lui-même.

  $1$ n'est pas premier, car il n'a qu'un seul diviseur.  
\end{Ex}

\begin{DefT}{Nombres pairs et impairs}\index{Nombres pairs et impairs}

  Un \textbf{nombre pair} est un nombre entier divisible par 2, autrement
  dit un nombre entier $n$ est pair lorsqu'il existe $k \in \Z$
  tel que $n=2k$.
 % . \\  Soit $n$ un nombre pair, $n=2\times k$ avec $k\in\Z$.

  Un entier $n$ est un \textbf{nombre impair} lorsqu'il existe $k \in \Z$
  tel que $n=2k+1$.
\end{DefT}


\begin{Exs} 
$46 = 2 \times 23$ et $23\in\Z$ donc $46$ est un nombre pair.

  $15= 2 \times 7,5$. Comme $7,5 \not\in\Z$, $15$ n'est pas pair. Par contre,
  $15=2 \times 7+1$ avec $7 \in \Z$, donc $15$ est impair.
\end{Exs}


\end{pageCours} 
\begin{pageAD} 
 

\Sf{Connaitre les ensembles $\mathbb N$ et $\mathbb Z$}

\begin{ExoCad}{Calculer.}{1234}{0}{0}{0}{0}{0}

Complète avec $\in$ ou $\not \in$.

\begin{enumerate}
\begin{minipage}{0.3\linewidth}
\item $2 \ldots \ldots \mathbb Z$
\item $-4 \ldots \ldots \mathbb N$
\item $7 \ldots \ldots \mathbb N$
\end{minipage}
\begin{minipage}{0.3\linewidth}
\item $42 \ldots \ldots \mathbb N$
\item $-123 \ldots \ldots \mathbb Z$
\item $-9 \ldots \ldots \mathbb N$
\end{minipage}
\begin{minipage}{0.3\linewidth}
\item $-22 \ldots \ldots \mathbb N$
\item $25 \ldots \ldots \mathbb Z$
\item $0 \ldots \ldots \mathbb Z$
\end{minipage}
\end{enumerate}


\end{ExoCad}

\Sf{Utiliser les multiples et les diviseurs}

\begin{ExoCad}{Calculer.}{1234}{0}{0}{0}{0}{0}
\begin{enumerate}
\item Déterminer les 10 premiers multiples positifs de 4 : \point{1}
\item Déterminer les 10 premiers multiples positifs de 6 : \point{1}
\item Déterminer les 3 premiers multiples communs de 4 et de 6 : \point{1}
\end{enumerate}
\end{ExoCad}

\begin{ExoCad}{Calculer.}{1234}{0}{0}{0}{0}{0}

$51$ est-il un nombre premier ? Justifier. 

\point{2}

\end{ExoCad}


\begin{ExoCad}{Raisonner.}{1234}{0}{0}{0}{0}{0}

Démontrer que $126$ est un nombre pair.

\point{3}

\end{ExoCad}



\begin{ExoCad}{Raisonner.}{1234}{0}{0}{0}{0}{0}

Démontrer que si $n$ est un nombre impair alors $n^2$ est impair.

\point{3}

\end{ExoCad}



\begin{ExoCad}{Calculer, raisonner.}{1234}{0}{0}{0}{0}{0}

Démontrer que la somme de deux multiples de $3$ est un multiple de $3$. \point{4}

Soit $a$ un entier. Démontrer que la somme de deux multiples de $a$ est un multiple de $a$. \point{4}

\end{ExoCad}




\end{pageAD}
\begin{pageCours}

\section{Nombres premiers entre eux}

  \begin{DefT}{Nombres premiers entre eux}\index{Premiers entre eux}

    Deux nombres entiers $a$ et $b$ sont \textbf{premiers entre eux}
    lorsque leur seul diviseur positif commun est $1$.
  \end{DefT}

\begin{Ex}
  Les diviseurs positifs de $8$ sont $1$ ; $2$ ; $4$ et $8$. Ceux de $15$ sont
  $1$ ; $3$ ; $5$ et $15$. Le seul diviseur commun est $1$ et \textbf{$8$ et $15$ sont donc premiers entre eux} (Pourtant, ils ne sont pas premiers)
\end{Ex}

\section{Décomposition en facteurs premiers}

  \begin{ThT}{Décomposition en facteurs premiers}
    Tout entier naturel supérieur ou égal à 2 se décompose de façon unique en produit de facteurs premiers.
  \end{ThT}


\begin{Exs}
Les décompositions en facteurs premiers de $8$, $15$ et $19$ sont respectivement  
$8=2^3$ ; $15=3 \times 5$ ; $19=19$.
  
\end{Exs}







\section{Logique}

 
\begin{DefT}{Proposition universelle}\index{Proposition universelle}

  Une \textbf{proposition universelle} est une proposition qui porte sur tous
  les éléments d'un ensemble. 

\end{DefT}
 
 
\begin{Rq} 
  Le théorème de décomposition en facteurs premiers est une proposition
  universelle (et elle est vraie).
  
%Quels que soient trois nombres $a$, $b$ et  $c$, $a(b+c)=ab+ac$. La distributivité est une proposition universelle.

\end{Rq}
 
\begin{DefT}{Contre-exemple}\index{Contre-exemple}

Un \textbf{contre-exemple} est un cas particulier qui vient contredire une proposition universelle. 

\end{DefT}
 

\begin{Ex} 

Considérons une proposition universelle : « tous les nombres sont pairs ». Pour démontrer que cette proposition est fausse, il suffit de démontrer qu'un seul nombre n'est pas pair. La contradiction vient sur le mot \textbf{tous}.

$3=2\times1,5$ et $1,5 \not\in\Z$ donc $3$ n'est pas pair. \textbf{Il existe
  des nombres non-pairs}, la proposition universelle initiale est fausse. 

\end{Ex}
 

\begin{LogT}{On souhaite répondre à la proposition : "Tout entier est-il un entier naturel ?"  }

Cette proposition est une proposition universelle qui se traduit par : "Tout nombre entier appartient-il à $\N$ ?". 

A priori, on ne sait pas quelle est la réponse. On peut utiliser alors des exemples pour tester la proposition.

Il suffit de trouver un seul nombre entier n'appartenant pas à $\N$ pour informer la proposition : Principe du contre exemple. 


$-2 \in \Z$ mais $-2 \not \in \N$. Donc $\Z \not \subset \N$.

\end{LogT}


\end{pageCours} 
\begin{pageAD} 



\Sf{Décomposer en produit de facteurs premiers}

\begin{ExoCad}{Calculer.}{1234}{0}{0}{0}{0}{0}

Les nombres $145$ et $126$ sont-ils premiers entre eux ? Justifier.

\point{3}

\end{ExoCad}





\begin{ExoCad}{Calculer.}{1234}{0}{0}{0}{0}{0}

Décomposer 24 en produit de facteurs premiers.

\point{3}

\end{ExoCad}

\begin{ExoCad}{Représenter. Calculer. }{1234}{1}{0}{0}{0}

\begin{minipage}{0.55\linewidth}

Pour déterminer tous les diviseurs d'un nombre, on utilise un arbre de diviseurs. C'est un représentation qui propose tous les calculs possibles avec les facteurs premiers du nombre.

On donne en exemple l'arbre de diviseurs de $36$.

Construire l'arbre des diviseurs de $20$.

\vspace{4cm}

\end{minipage}
\begin{minipage}{0.4\linewidth}

\definecolor{ududff}{rgb}{0.30196078431372547,0.30196078431372547,1.}
\begin{tikzpicture}[line cap=round,line join=round,>=triangle 45,x=1.0cm,y=1.0cm]
\clip(-3.6,-1.) rectangle (3.9,5.42);
\draw (-3.34,2.36) node[anchor=north west] {$36=2^2\times 3^2$};
\draw [line width=1.pt] (-1.,2.)-- (1.,4.);
\draw [line width=1.pt] (-1.,2.)-- (1.,2.);
\draw [line width=1.pt] (-1.,2.)-- (1.,0.);
\draw [line width=1.pt] (1.,2.)-- (2.04,2.78);
\draw [line width=1.pt] (1.,2.)-- (2.,2.);
\draw [line width=1.pt] (1.,2.)-- (1.98,1.24);
\draw [line width=1.pt] (1.,0.)-- (2.02,0.64);
\draw [line width=1.pt] (1.,0.)-- (2.,0.);
\draw [line width=1.pt] (1.,0.)-- (2.02,-0.74);
\draw [line width=1.pt] (1.,4.)-- (1.96,4.66);
\draw [line width=1.pt] (1.,4.)-- (2.,4.);
\draw [line width=1.pt] (1.,4.)-- (2.06,3.5);
\begin{scriptsize}
\draw[color=ududff] (2.9,2.91) node {$2^1 \times 3^0=2$};
\draw[color=ududff] (2.9,2.09) node {$2^1 \times 3^1=6$};
\draw[color=ududff] (2.9,1.31) node {$2^1 \times 3^2=18$};
\draw[color=ududff] (2.9,0.77) node {$2^2 \times 3^0=4$};
\draw[color=ududff] (2.9,0.07) node {$2^2 \times 3^1=12$};
\draw[color=ududff] (2.9,-0.63) node {$2^2 \times 3^2=36$};
\draw[color=ududff] (2.9,4.73) node {$2^0 \times 3^0=1$};
\draw[color=ududff] (2.9,4.09) node {$2^0 \times 3^1=3$}; 

\draw[color=black] (0.16,3.37) node {$2^0$};
\draw[color=black] (0.16,2.35) node {$2^1$};
\draw[color=black] (0.16,1.23) node {$2^2$};

\draw[color=black] (1.56,4.5) node {$3^0$};
\draw[color=black] (1.56,4.15) node {$3^1$};
\draw[color=black] (1.56,3.55) node {$3^2$};

\draw[color=black] (1.56,2.61) node {$3^0$};
\draw[color=black] (1.56,2.2) node {$3^1$};
\draw[color=black] (1.56,1.35) node {$3^2$};

\draw[color=black] (1.56,0.63) node {$3^0$};
\draw[color=black] (1.56,0.15) node {$3^1$};
\draw[color=black] (1.56,-0.63) node {$3^2$};

\end{scriptsize}
\end{tikzpicture}
\end{minipage}

\end{ExoCad} 

\begin{ExoCad}{Calculer.}{1234}{0}{0}{0}{0}{0}

Simplifier la fraction $\dfrac{735}{840}$ 

\point{3}

\end{ExoCad}



\begin{ExoCad}{Raisonner.}{1234}{0}{0}{0}{0}{0}
Vrai ou faux : quel que soit l'entier $n$, $2n-1$ est un nombre premier. Justifier.\point{2}
\end{ExoCad}


 
\begin{ExoCad}{Raisonner.}{1234}{0}{0}{0}{0}{0}

Les propositions suivantes sont-elles vraies ou fausses ?
\begin{enumerate}[leftmargin=*]
\item La différence de deux nombres entiers naturels est un entier naturel. \point{1}
\item Le quotient de deux nombres décimaux non nuls est un nombre décimal. \point{1} 
\item Il existe deux nombres premiers distincts dont le quotient est un
  entier relatif. \point{1}
\item Il existe deux nombres premiers distincts dont le quotient est un
  nombre décimal. \point{1}
\end{enumerate} 
 
 \end{ExoCad}
 
\end{pageAD}


%%%%%%%%%%%%%%%%%%%%%%%%%%%%%%%%%%%%%%%%%%%%%%%%%%%%%%%%%%%%%%%%%%%
%%%%  Niveau 1
%%%%%%%%%%%%%%%%%%%%%%%%%%%%%%%%%%%%%%%%%%%%%%%%%%%%%%%%%%%%%%%%%%%
\begin{pageParcoursu} 



 
%%%%%%%%%%%%%%%%%%%%%%%%%%%
\begin{ExoCu}{Représenter.}{1234}{2}{0}{0}{0}{0}

\begin{enumerate}
\item Décomposer $186$ et $155$ en produit de facteurs premiers. \point{3}
\item Déterminer le PGCD (plus grand diviseur commun)
  de $186$ et $155$. \point{2}
\item Un chocolatier a fabriqué $186$ pralines et $155$ chocolats
  qu'il répartit dans des colis. 
Les colis sont constitués ainsi :
\begin{description}
\item Le nombre de pralines est le même dans chaque colis.
\item Le nombre de chocolats est le même dans chaque colis.
\item Tous les chocolats et toutes les pralines sont utilisés.
\end{description}
\begin{enumerate}
\item Quel nombre maximal de colis pourra-t-il réaliser ?  \point{3}
\item Combien y aura-t-il de chocolats et de pralines dans chaque colis ?   \point{3}
\end{enumerate}
\end{enumerate}
  
\end{ExoCu}


\begin{ExoCuN}{Raisonner.}{1}{0}{0}{0}{0}
Simplifier la fraction $\dfrac{2310}{2730}$ pour la rendre irréductible. \point{2}
\end{ExoCuN}


\begin{ExoCuN}{Raisonner.}{1}{0}{0}{0}{0}
Démontrer que le produit de deux entiers consécutifs est pair. \point{2}
\end{ExoCuN}

\begin{ExoCuN}{Chercher.}{1}{0}{0}{0}{0}
Proposer deux nombres premiers entre eux. \point{2}
\end{ExoCuN}

\begin{ExoCuN}{Chercher. Raisonner.}{1}{0}{0}{0}{0}
Les produits de deux nombres premiers est-il un nombre premier ? Justifier. \point{2}
\end{ExoCuN}


\begin{ExoCuN}{Chercher.}{2}{0}{0}{0}{0} 
Je suis un nombre à trois chiffres non nuls. Je suis divisible par 94. Changez l'ordre de mes chiffres d'une certaine manière, et je deviens divisible par 49.
Qui suis-je ?   \point{5}
\end{ExoCuN}


\end{pageParcoursu} 
 
%%%%%%%%%%%%%%%%%%%%%%%%%%%%%%%%%%%%%%%%%%%%%%%%%%%%%%%%%%%%%%%%%%%
%%%%  Niveau 2
%%%%%%%%%%%%%%%%%%%%%%%%%%%%%%%%%%%%%%%%%%%%%%%%%%%%%%%%%%%%%%%%%%%
\begin{pageParcoursd} 



\begin{ExoCdN}{Représenter.}{2}{0}{0}{0}{0}
On veut démontrer que la proposition universelle $\mathcal P$ suivante : « La somme de deux nombres impairs est un nombre pair » est vraie.

\begin{enumerate}
\item Calculer $a=5+7$. Peut-on en déduire que la proposition $P$
  est vraie ? %La proposition $\mathcal P$ semble-t-elle vraie ?\point{2}
\item Soient $n$ et $m$ deux nombres impairs. Il existe donc deux entiers
  relatifs $k$ et $q$ tels que $n=2k+1$ et $m=2q+1$.
  %Exprimer un autre nombre impair $m$ en fonction de $q$.\point{3}
  %\item
  Calculer $n+m$ en fonction de $k$ et $q$. \point{3}
\item En déduire que la somme $n+m$ est un nombre pair. \point{3}
\end{enumerate}

\end{ExoCdN}

 
\begin{ExoCdN}{Représenter.}{1}{1}{0}{0}{0}

  
Démontrer que si $p$ est pair alors $p^2$ est pair.


 \point{6}
 
\end{ExoCdN}

 %%%%%%%%%%%%%%%%%%%%%%%%%%%%%%%%%%%%%%%%%%%%%%%%%%%%%%%%%%%%%%%%%%%
\begin{ExoCdN}{Raisonner.}{1}{0}{0}{0}{0}

Démontrer que tout nombre entier $n$ multiple de $9$ est un multiple de $3$.  \point{6}

\end{ExoCdN}

\begin{ExoCdN}{Raisonner.}{1}{0}{0}{0}{0}
Montrer que la somme de trois entiers consécutifs est toujours un multiple de 3. \point{6}
\end{ExoCdN}


\begin{ExoCdN}{Raisonner.}{2}{0}{0}{0}{0}
Le produit de deux nombres impairs est-il impair ? \point{3}
\end{ExoCdN}





\end{pageParcoursd}
 
%
%%%%%%%%%%%%%%%%%%%%%%%%%%%%%%%%%%%%%%%%%%%%%%%%%%%%%%%%%%%%%%%%%%%%
%%%%%  Niveau 3
%%%%%%%%%%%%%%%%%%%%%%%%%%%%%%%%%%%%%%%%%%%%%%%%%%%%%%%%%%%%%%%%%%%%
\begin{pageParcourst}



 
\begin{ExoCtN}{Raisonner.}{1}{1}{0}{0}{0}

  Démontrer que si $n$ est impair alors $n^2$ est impair.

 \point{5}
 
\end{ExoCtN}


\begin{ExoCtN}{Raisonner.}{1}{0}{0}{0}{0}
On donne le programme en Python ci dessous. 
 
\begin{lstlisting}[language=Python] 
def is_divisible(x,y):
    if x%y == 0 :
    	test = "{} est divisible par {}".format(x,y) 
    else :	
        test = "{} n'est pas divisible par {}".format(x,y)
    return test    
    
n=int(input("Entrer un nombre n :"))
    
print(is_divisible(n,4))
\end{lstlisting}
 
\begin{enumerate}
\item Que fait ce programme ? On pourra tester le programme avec l'éditeur : \url{https://sacado.xyz/tool/show/18}  \point{3}
\item Modifier le programme pour qu'il teste si un nombre $a$ divise $n$.   
\end{enumerate}
 
\end{ExoCtN}



%%%%%%%%%%%%%%%%%%%%%%%%%%%%%%%%%%%%%%%%%%%%%%%%%%%%%%%%%%%%%%%%%%%
\begin{ExoCtN}{Raisonner.}{2}{1}{0}{0}{0}
 
 
Dans un pays où le système fiduciaire (les pièces et les billets) n'est constitué que de pièces de 3 et de 5, il s'agit d'aider les habitants en créant un algorithme  qui donne le nombre minimal de pièces nécessaires à tout achat d'un montant entier supérieur ou égal à 8. 

Pour tester l'algorithme, on peut utiliser l'éditeur Python : \url{https://sacado.xyz/tool/show/18}

\hfill{{\footnotesize Source : d’après PISA, items libérés}}
 
\end{ExoCtN}

 %%%%%%%%%%%%%%%%%%%%%%%%%%%%%%%%%%%%%%%%%%%%%%%%%%%%%%%%%%%%%%%%%%%
\begin{ExoCtN}{Raisonner.}{2}{1}{0}{0}{0}
 
 \begin{minipage}{0.5\linewidth} 
 
\textbf{Le crible d'Eratosthène}

L'algorithme procède par élimination : il s'agit de rayer d'une table d'entiers tous les multiples d'un entier $n$ (autres que lui-même), et d'entourer
tous les autres. 

En supprimant tous ces multiples, à la fin il ne restera que les entiers qui ne sont multiples d'aucun entier à part 1 et eux-mêmes, et qui sont donc les nombres premiers.

On commence par entourer deux, puis on barre tous
les multiples de 2 à partir de 4. On entoure alors le premier
nombre non rayé ni entouré, qui est 3, et on raye puis les multiples de 3
sauf 3. Puis on entoure le premier nombre non rayé ni entouré, qui est
5, et on raye tous les multiples de 5 sauf 5... On répète l'opération
jusqu'à ce qu'il n'y ait plus d'entier à rayer. 

\begin{enumerate}

\item Exécuter le crible sur la table ci-contre.
\item Quel est le résultat de ce crible ? \point{3}

\item Écrire un code en Python du crible d'Eratosthène : \url{https://sacado.xyz/tool/show/18}

\end{enumerate}

\end{minipage}
\begin{minipage}{0.5\linewidth}

 


\begin{tabular}{|c|c|c|c|c|c|c|c|c|c|}
 \hline 
 &  & 2 & 3 & 4 & 5 & 6 & 7 & 8 & 9 \\ 
 \hline 
 10&11 & 12 & 13 & 14 & 15 & 16 & 17 & 18 & 19 \\
 \hline 
 20&21 & 22 & 23 & 24 & 25 & 26 & 27 & 28 & 29 \\
 \hline 
 30&31 & 32 & 33 & 34 & 35 & 36 & 37 & 38 & 39 \\
 \hline 
 40&41 & 42 & 43 & 44 & 45 & 46 & 47 & 48 & 49 \\
 \hline 
 50&51 & 52 & 53 & 54 & 55 & 56 & 57 & 58 & 59 \\
 \hline 
 60&61 & 62 & 63 & 64 & 65 & 66 & 67 & 68 & 69 \\
 \hline 
 70&71 & 72 & 73 & 74 & 75 & 76 & 77 & 78 & 79 \\
 \hline 
 80&81 & 82 & 83 & 84 & 85 & 86 & 87 & 88 & 89 \\
 \hline 
 90&91 & 92 & 93 & 94 & 95 & 96 & 97 & 98 & 99 \\
 \hline 
 \end{tabular}  
 
 \end{minipage}


 
\end{ExoCtN}
 
\end{pageParcourst}
%
%%%%%%%%%%%%%%%%%%%%%%%%%%%%%%%%%%%%%%%%%%%%%%%%%%%%%%%%%%%%%%%%%%%%
%%%%%  Brouillon
%%%%%%%%%%%%%%%%%%%%%%%%%%%%%%%%%%%%%%%%%%%%%%%%%%%%%%%%%%%%%%%%%%%%


%%%%%%%%%%%%%%%%%%%%%%%%%%%%%%%%%%%%%%%%%%%%%%%%%%%%%%%%%%%%%%%%%%%
%%%%  Auto
%%%%%%%%%%%%%%%%%%%%%%%%%%%%%%%%%%%%%%%%%%%%%%%%%%%%%%%%%%%%%%%%%%%


%%%%%%%%%%%%%%%%%%%%%%%%%%%%%%%%%%%%%%%%%%%%%%%%%%%%%%%%%%%%%%%%%%%
\begin{pageAuto} 

 
%%%%%%%%%%%%%%%%%%%%%%%%%%%%%%%%%%%%%%%%%%%%%%%%%%%%%%%%%%%%%%%%%%%
\begin{ExoAutoN}{Raisonner.}{1}{0}{0}{0}{0}
Simplifier le nombre $a=\dfrac{60}{126}$ pour la rendre irréductible. \point{4}
\end{ExoAutoN}
%%%%%%%%%%%%%%%%%%%%%%%%%%%%%%%%%%%%%%%%%%%%%%%%%%%%%%%%%%%%%%%%%%%
%%%%%%%%%%%%%%%%%%%%%%%%%%%%%%%%%%%%%%%%%%%%%%%%%%%%%%%%%%%%%%%%%%%
\begin{ExoAutoN}{Raisonner.}{1}{0}{0}{0}{0}
Simplifier le nombre $b=\dfrac{12a+4}{8}$. \point{2}
\end{ExoAutoN}
%%%%%%%%%%%%%%%%%%%%%%%%%%%%%%%%%%%%%%%%%%%%%%%%%%%%%%%%%%%%%%%%%%%


%%%%%%%%%%%%%%%%%%%%%%%%%%%%%%%%%%%%%%%%%%%%%%%%%%%%%%%%%%%%%%%%%%%
\begin{ExoAutoN}{Raisonner.}{2}{0}{0}{0}{0}
Soit $n$ un entier.

Démontrer que la différence de deux multiples de $n$ est un multiple de $n$. \point{6}
\end{ExoAutoN}

%%%%%%%%%%%%%%%%%%%%%%%%%%%%%%%%%%%%%%%%%%%%%%%%%%%%%%%%%%%%%%%%%%%
\begin{ExoAutoN}{Raisonner.}{2}{0}{0}{0}{0}
  Pour déterminer le PGCD de deux entiers naturel $a$ et $b$ (avec $b \neq 0$),
  on effectue la division euclidienne de $a$ par $b$. On appelle $r_0$ le reste. \\
Puis on divise $b$ par $r_0$ et on appelle $r_1$ le reste. \\
On divise alors $r_0$ par $r_1$ et on appelle $r_2$ le reste.\\ 
On divise alors $r_1$ par $r_2$ et on appelle $r_3$ le reste. Et ainsi de suite,
jusqu'à obtenir un reste nul. 
Le PGCD de $a$ et de $b$ est alors le dernier reste non nul (ou $b$
si le premier reste est déjà nul).
On appelle ce procédé « la méthode par divisions successives » ou
« l'algorithme d'Euclide ». 

\begin{enumerate}[leftmargin=*]

\item Déterminer à l'aide de ce procédé le PGCD de $912$ et de $\np{1104}$. \point{3}
\item 
%\begin{enumerate}[leftmargin=*]
% \item Un tapissier achète $2 622$ clous tête plate et $2 530$ clous tête ronde pour la fabrication de fauteuils identiques : ch Après la fabrication, il ne lui reste plus aucun clous. Quel est le plus grand nombre de fauteuil que le tapissier peut réaliser ? 
% \point{3}
  %\item Dans ce cas, quelle sera le nombre de chaque type de clou par fauteuil ? \point{2}
  Un carreleur doit carreler une pièce rectangulaire de $912$cm par
  $\np{1104}$cm
  en utilisant des carreaux carrés. Le travail sera grandement facilité
  si :
  \begin{itemize}
  \item il n'a pas de découpe à faire : il disposera un nombre entier
    de carreaux sur la longueur et sur la largeur de la pièce ;
  \item il utilise le moins possible de carreaux, donc les carreaux
    sont les plus grands possibles.
    
  \end{itemize}
  \begin{enumerate}
  \item Quel est le côté $c$ des carreaux qui répond à ces deux contraintes ?\point{3}
  \item Combien de carreaux seront disposés en longueur ? en largeur ?
    Combien de carreaux seront utilisés au total ?\point{4}
  \end{enumerate}
\end{enumerate}
\end{ExoAutoN}

\end{pageAuto}
