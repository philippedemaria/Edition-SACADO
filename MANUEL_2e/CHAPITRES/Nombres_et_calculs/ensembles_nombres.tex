\chapter{Ensembles de nombres}
{https://sacado.xyz/qcm/parcours_show_course/0/117129}
{


 \begin{CpsCol}
\textbf{Les savoir-faire du parcours}
 \begin{itemize}
 \item \textbf{Connaitre le vocabulaire ensembliste et logique}
 \item Connaitre les ensembles de nombres
 \item Représenter les ensembles de nombres sur la droite graduée
 \item Identifier les intervalles de $\R$
 \item Lire et écrire des propositions contenant les connecteurs « et », « ou » ;

 \end{itemize}
 \end{CpsCol}


\begin{His}

\textbf{Georg Cantor} est un mathématicien allemand, né le 3 mars 1845 à Saint-Pétersbourg et mort le $6$ janvier $1918$ à Halle.\\ 
Il est connu pour être le créateur de la théorie des ensembles.\\ 
Il prouva que les nombres réels sont "plus nombreux" que les entiers naturels. En fait, le théorème de Cantor implique\\ 
l'existence d'une "infinité d'infinis". Il définit les nombres cardinaux, les nombres ordinaux et leur arithmétique. \\ 
Une partie du travail de Cantor consista à dire que l'infini a plusieurs taille. Et qu'il y a des infini plus grand que d'autres. \\ 
Cantor rejoint ainsi le monde de la philosophie, comme \textbf{Pascal} 3 siècle plus tôt. 

\end{His}

 

\begin{ExoDec}{Chercher.}{1234}{2}{0}{0}{0}

L'\textbf{E}urope est composée de 27 pays dont la \textbf{F}rance, l'\textbf{I}talie et la \textbf{G}rèce. Les capitales respectives sont \textbf{P}aris, \textbf{R}ome et \textbf{A}thènes. 
\begin{enumerate}
\item Un habitant de Rome est-il en Italie ? 
\item Un habitant de Grèce est-il une habitant de Rome ? 
\item Un habitant d'Italie est-il en Europe ?
\item Si je ne suis pas en Europe, puis-je être en France ?
\item Construire un diagramme de Venn qui illustre cette situation.
\end{enumerate}

{\small \textit{Pour faciliter l'écriture, on pourra utilisera l'initiale des noms respectifs : $E$, $F$, $I$, $G$, $P$, $R$ et $A$.}}
\end{ExoDec}


\begin{ExoDec}{Chercher.}{1234}{2}{0}{0}{0}
 
L'INSEE estime qu'un couple avec deux enfants appartient à la classe moyenne quand les revenus du foyer sont situés dans l'intervalle $[3253;5609]$. Monsieur Twicks gagne $2731$ euros et madame Twicks gagne $2952$ euros et ont deux enfants Mary et Paul.

 La famille appartient-elle à la classe moyenne ?

%Le couple gagne donc $s_{global}=2731+2952 = 5683$ euros. On compare cette valeur à l'intervalle de la classe moyenne $[3253;5609]$.
%
%$5683>5609 $ donc $5683 \not \in [3253;5609]$. Le couple n'appartient donc pas à la classe moyenne.
 
 \end{ExoDec}



}

\begin{pageCours}

\section{Ensembles de nombres}

\begin{DefT}{Ensemble des réels}
L'ensemble de tous les nombres connus en Seconde est appelé ensemble des nombres réels, noté $\R$.
\end{DefT}

\begin{DefT}{Ensemble de nombres}\index{Ensemble de nombres!Réels $\R$}
Les autres ensembles de nombres, inclus dans $\R$.
\begin{enumerate}
\item On appelle \textbf{entiers naturels} les nombres : 0 ; 1 ; 2 ; 3 . . . Leur ensemble est noté $\N$.\index{Ensemble de nombres! Entiers naturels $\N$}
On a donc : $\N =  \lbrace 0 ; 1 ; 2 ; 3 \cdots \rbrace $
\item  On appelle \textbf{entiers relatifs} les nombres entiers naturels et leurs symétriques par rapport à 0. Leur ensemble est noté $\Z$.\index{Ensemble de nombres! Entiers $\Z$}. L'utilisation de la lettre $\Z$ vient de l'allemand Zahlen (Chiffre).
On a donc : $\Z = \lbrace \cdots -3 ; -2 ; -1 ; 0 ; 1 ; 2 ; 3  \cdots  \rbrace$
\item  On appelle \textbf{nombres rationnels} les nombres de la forme $\frac{a}{b}$, $a$ et $b$ entiers et $b$ non nuls.  Leur ensemble est noté $\Q$. \index{Ensemble de nombres! Rationnels $\Q$}
On a donc : $\Q = \lbrace \cdots \frac{5}{3} ; -\frac{5}{7} ; -\frac{13}{22} \cdots   \rbrace$. Les réels non rationnels sont dit \textbf{irrationnels} : $\sqrt{2}$
\item Par construction, $\N$ est inclus dans $\Z$  est inclus dans $\D$  est inclus dans $\Q$  est inclus dans $\R$. Ces ensembles sont dits "emboités".
\item  On peut représenter l'ensemble des réels sur une droite graduée.
\begin{center}
\begin{tikzpicture}[line cap=round,line join=round,>=triangle 45,x=1.0cm,y=1.0cm]
\draw[->,color=black] (-4.36,0.) -- (10.66,0.);
\foreach \x in {-4.,-3.,-2.,-1.,1.,2.,3.,4.,5.,6.,7.,8.,9.,10.}
\draw[shift={(\x,0)},color=black] (0pt,2pt) -- (0pt,-2pt) node[below] {\footnotesize $\x$};
\draw[color=black] (0pt,-10pt) node[right] {\footnotesize $0$};
\clip(-4.36,-0.5) rectangle (10.66,0.5);
\end{tikzpicture}
 \end{center} 
\end{enumerate}
\end{DefT}


%\begin{Nt}
%
%Soit $M$ un point d'abscisse $a$, sur la droite graduée.\\
%On note $\vert a\vert$ la distance de $M$ à $O$.
%
%Ainsi, on peut établir que $\vert -a \vert = \vert a \vert$. En effet, deux nombres opposés sont à la même distance de l'origine.
%\end{Nt}

 

\begin{DefT}{Appartenance}\index{Ensemble!Appartenance}
Pour symboliser l'appartenance d'un nombre à un ensemble, on utilise le symbole $\in$ : 
$-2 \in \Z \quad \quad \frac{5}{3} \in \Q $ 
 
 $ \not\in $ signifie n'appartient pas : $-5 \not \in \N \quad \quad \pi \not\in \Q$
\end{DefT}


\begin{Nt}

\begin{description}[leftmargin=*]
\item  On appelle $\R^+$ l'ensemble des réels positifs, $\R^-$ l'ensemble des réels négatifs.
\item  On utilise une étoile pour enlever $0$ d'un ensemble. $\R^*$ l'ensemble des réels non nuls.
\end{description}

\end{Nt}

\begin{DefT}{Nombres décimaux}
Les \textbf{nombres décimaux} sont des nombres rationnels dont le dénominateur est une puissance de 2, de 5 ou de 10 ou un produit de puissances de ces nombres. Les nombres décimaux peuvent donc s'écrire sous la forme $\dfrac{a}{10^n}$, $a \in \Z$ et $n \in \N$.
\end{DefT}

\begin{Ex}
\begin{description}[leftmargin=*]
\item  $\dfrac{3}{25}$ est un nombre décimal car $25 = 5^2$, 25 est donc une puissance de $5$.
\item  $\dfrac{13}{20}$ est un nombre décimal car $20 = 2^2 \times 5$, $20$ est donc un produit d'une puissance de $2$ et de $5$.
\end{description}
\end{Ex}

\begin{Att}

$\dfrac{7}{15}$ n'est pas un nombre décimal. $15=3 \times 5$ n'est pas une puissance de $5$ mais un multiple de $5$ !

\end{Att}






%%\PESP{https://fr.wikipedia.org/wiki/D\%C3\%A9veloppement\_d\%C3\%A9cimal\_p\%C3\%A9riodique}
%%
%%\AV{https://www.youtube.com/watch?v=N_cDA6tF-40}{Conjecture de Cantor}

\end{pageCours} 

\begin{pageAD} 
 

\Sf{Connaitre les ensembles de nombres}
 
  
\begin{ExoCad}{Représenter.}{1234}{2}{0}{0}{0}{0}
 
 \begin{minipage}{0.48\linewidth}
Placer, dans le plus petit ensemble, les nombres suivants :
$$ -4 ~~ ;~~  \sqrt{36}~~  ; ~~ \frac{5}{7} ~~  ;~~  -\sqrt{28}~~ ;~~  \frac{72}{6} ~~  ; ~~ 2,3 ~~  ; ~~ 0 ;~~ -1,785 $$
\end{minipage}
\hfill
\begin{minipage}{0.48\linewidth}

\definecolor{qqqqff}{rgb}{0.,0.,1.}
\definecolor{qqwuqq}{rgb}{0.,0.39215686274509803,0.}
\definecolor{xfqqff}{rgb}{0.4980392156862745,0.,1.}
\definecolor{ffqqqq}{rgb}{1.,0.,0.}
\begin{tikzpicture}[line cap=round,line join=round,>=triangle 45,x=0.6782617187500004cm,y=0.6782617187500004cm]
\clip(-6.0955652959673365,-0.7955347642346219) rectangle (5.699291731705649,4.717946580732306);
\draw [rotate around={0.:(-2.,2.)},color=ffqqqq] (-2.,2.) ellipse (0.8410791202168588cm and 0.4973682009769633cm);
\draw [rotate around={-0.20389871046679844:(-1.6955307407221418,2.00727956851319)},color=xfqqff] (-1.6955307407221418,2.00727956851319) ellipse (1.661680538260386cm and 0.7868931785368128cm);
\draw [rotate around={-0.7690246825780399:(-0.8815627427186662,1.9688848516262334)},color=qqwuqq] (-0.8815627427186662,1.9688848516262334) ellipse (2.664297761397899cm and 1.2951216094501943cm);
\draw [rotate around={-0.002358769768743201:(-0.14948545367760763,1.9998003125678991)},color=qqqqff] (-0.14948545367760763,1.9998003125678991) ellipse (3.765828366589848cm and 1.832457538495793cm);
\draw [color=ffqqqq](-1.242473081456056,2.3296214542697094) node[anchor=north west] {$\mathbb{N}$};
\draw [color=xfqqff](0.26259982051263214,2.3142635675149267) node[anchor=north west] {$\mathbb{Z}$};
\draw [color=qqwuqq](2.6123564939943598,2.1989056807601443) node[anchor=north west] {$\mathbb{Q}$};
\draw [color=qqqqff](4.946755280721304,2.3142635675149267) node[anchor=north west] {$\mathbb{R}$};
\end{tikzpicture}
\end{minipage}

\end{ExoCad}


  
\begin{ExoCad}{Raisonner. Communiquer.}{1234}{0}{0}{0}{0}{0}

Complète par  $\in$ ou $\notin$.

\begin{enumerate}
\begin{minipage}{0.32\linewidth}
\item $\pi \ldots \ldots \R$
\item $-5 \ldots \ldots \Z$
\item $\dfrac{1}{2} \ldots \ldots \D$
\end{minipage}
\hfill
\begin{minipage}{0.32\linewidth}
\item $-9 \ldots \ldots \N$
\item $\dfrac{1}{2}  \ldots \ldots \R$
\item $\dfrac{7}{3}  \ldots \ldots \Q$
\end{minipage}
\hfill
\begin{minipage}{0.32\linewidth}
\item $\sqrt{4} \ldots \ldots \N$
\item $-1,5 \ldots \ldots \Z$
\item $\dfrac{12}{4}  \ldots \ldots \N$
\end{minipage}
\end{enumerate}
 
\end{ExoCad}
 
 
 
\begin{ExoCad}{Représenter.}{1234}{0}{0}{0}{0}{0}

On pose $a=4\times \left(\frac{3}{5}\right)^3 \times \frac{20}{3}$. Montrer que $\sqrt{a}$ est un nombre décimal. \point{5}
 
 \end{ExoCad}



 
\begin{DemoE}

Démontrer que $\dfrac{1}{3}$ n'est pas un nombre décimal.

\point{5}

 
\end{DemoE}

 
 

 


\begin{DemoE}

Démontrer que $\sqrt{2}$ est un irrationnel. Une autre formulation est : démontrer que $\sqrt{2} \in \R\setminus\Q$.

\point{8}

\end{DemoE}
   


 
\end{pageAD}


\begin{pageCours}

\section{Intervalles de $\R$ }

\begin{DefT}{Intervalle fermé. Intervalle ouvert}\index{Intervalle}
Un \textbf{intervalle fermé} de $\R$ est un sous-ensemble borné de $\R$, c'est à dire un ensemble de nombres compris entre deux valeurs réelles.
 
Un \textbf{intervalle ouvert} de $\R$ est un sous-ensemble de $\R$ dont les bornes ne sont pas incluses dans l'ensemble, c'est à dire un ensemble de nombres compris entre deux valeurs réelles non comprises.
\end{DefT}

\begin{Rep}
\begin{enumerate}
\item On a représenté sur la droite des nombres réels tous les nombres réels $x$ tels que $-1 \leq x \leq 3$.

\begin{center}
\definecolor{ffxfqq}{rgb}{1.,0.4980392156862745,0.}
\begin{tikzpicture}[line cap=round,line join=round,>=triangle 45,x=1.0cm,y=1.0cm]
\draw[->,color=black] (-4.390839866186475,0.) -- (7.64974334956303,0.);
\foreach \x in {-4.,-3.,-2.,-1.,1.,2.,3.,4.,5.,6.,7.}
\draw[shift={(\x,0)},color=black] (0pt,2pt) -- (0pt,-2pt) node[below] {\footnotesize $\x$};
\draw[color=black] (0pt,-10pt) node[right] {\footnotesize $0$};
\clip(-4.390839866186475,-0.5880295569511441) rectangle (7.64974334956303,0.5863275715079787);
\draw [line width=2.4pt,color=ffxfqq] (-1.,0.)-- (3.,0.);
\draw [color=ffxfqq](-1.16,0.35) node[anchor=north west] {\Large{[}};
\draw [color=ffxfqq](2.85,0.35) node[anchor=north west] {\Large{]}};
\end{tikzpicture}
 \end{center} 
 
Cet intervalle est noté $[-1;3]$ et on écrit alors $x \in [-1;3]$. Cet intervalle est dit \textbf{fermé}.
 \item On a représenté sur la droite des nombres réels tous les nombres réels $x$ tels que $-1 < x < 3$.

\begin{center}
\definecolor{ffxfqq}{rgb}{1.,0.4980392156862745,0.}
\begin{tikzpicture}[line cap=round,line join=round,>=triangle 45,x=1.0cm,y=1.0cm]
\draw[->,color=black] (-4.390839866186475,0.) -- (7.64974334956303,0.);
\foreach \x in {-4.,-3.,-2.,-1.,1.,2.,3.,4.,5.,6.,7.}
\draw[shift={(\x,0)},color=black] (0pt,2pt) -- (0pt,-2pt) node[below] {\footnotesize $\x$};
\draw[color=black] (0pt,-10pt) node[right] {\footnotesize $0$};
\clip(-4.390839866186475,-0.5295569511441) rectangle (7.64974334956303,0.563275715079787);
\draw [line width=2.4pt,color=ffxfqq] (-1.,0.)-- (3.,0.);
\draw [color=ffxfqq](-1.2,0.35) node[anchor=north west] {\Large{]}};
\draw [color=ffxfqq](2.85,0.35) node[anchor=north west] {\Large{[}};
\end{tikzpicture}
 \end{center}
Cet intervalle ouvert est noté $]-1;3[$ et on écrit alors $x \in ]-1;3[$. Cet intervalle est dit \textbf{ouvert}.

 \item On a représenté sur la droite des nombres réels tous les nombres réels $x$ tels que $x \geq -1$.


\begin{center}
\definecolor{ffxfqq}{rgb}{1.,0.4980392156862745,0.}
\begin{tikzpicture}[line cap=round,line join=round,>=triangle 45,x=1.0cm,y=1.0cm]
\draw[->,color=black] (-4.390839866186475,0.) -- (7.64974334956303,0.);
\foreach \x in {-4.,-3.,-2.,-1.,0,1.,2.,3.,4.,5.,6.,7.}
\draw[shift={(\x,0)},color=black] (0pt,2pt) -- (0pt,-2pt) node[below] {\footnotesize $\x$};
\clip(-4.390839866186475,-0.5880295569511441) rectangle (7.64974334956303,0.53275715079787);
\draw [line width=2.4pt,color=ffxfqq] (-1.,0.)-- (8.,0.);
\draw [color=ffxfqq](-1.2,0.35) node[anchor=north west] {\Large{[}};
\end{tikzpicture}
 \end{center} 
Cet ensemble est noté $[-1 ; +\infty[$ et on écrit alors $x \in [-1 ; +\infty[$. Cet intervalle est semi-ouvert à droite.


 \item On a représenté sur la droite des nombres réels tous les nombres réels $x$ tels que $x \geq -1$.


\begin{center}
\definecolor{ffxfqq}{rgb}{1.,0.4980392156862745,0.}
\begin{tikzpicture}[line cap=round,line join=round,>=triangle 45,x=1.0cm,y=1.0cm]
\draw[->,color=black] (-4.390839866186475,0.) -- (7.64974334956303,0.);
\foreach \x in {-4.,-3.,-2.,-1.,0,1.,2.,3.,4.,5.,6.,7.}
\draw[shift={(\x,0)},color=black] (0pt,2pt) -- (0pt,-2pt) node[below] {\footnotesize $\x$};
\clip(-4.390839866186475,-0.5880295569511441) rectangle (7.64974334956303,0.53275715079787);
\draw [line width=2.4pt,color=ffxfqq] (-5.,0.)-- (2.,0.);
\draw [color=ffxfqq](1.8,0.35) node[anchor=north west] {\Large{]}};
\end{tikzpicture}
 \end{center} 
Cet ensemble est noté $]-\infty;2]$ et on écrit alors $x \in ]-\infty;2]$. Cet intervalle est semi-ouvert à gauche.



\end{enumerate}
\end{Rep}

\begin{Rqs}
\begin{enumerate}
\item  $+ \infty$ se lit "plus l’infini". L'ensemble des nombres réels $\R$ est l'intervalle $]-\infty ; +\infty[ = \R$.
\item Un intervalle est une partie de $\R$ "sans trou", en "un seul morceau".
\item $+\infty$ et $-\infty$ ne sont pas des nombres. Ce ne sont que des notations (ce qui explique qu'ils soient toujours exclus).
\item Les intervalles correspondants aux quatre premières lignes du tableau sont dits bornés.
\item  Plus généralement, les différents types d'intervalles sont donnés dans le tableau ci-dessous (où $a$ et $b$ représentent deux réels, avec $a < b$).
\end{enumerate}
\end{Rqs}
 

\end{pageCours} 

\begin{pageAD} 
 

\Sf{Connaitre et représenter les intervalles}
 

  
\begin{ExoCad}{Représenter.}{1234}{0}{0}{0}{0}{0}

Soit $x$ un nombre. Écrire sous forme d'intervalle les inégalités suivantes :


\begin{enumerate}[leftmargin=*]
\begin{minipage}{0.49\linewidth}
	\item $-2 \leq x \leq 3$  \point{1}
	\item $-\pi \leq x <  \pi$  \point{1}
	\item Soient $a$ et $b$ deux réels, $a \leq x \leq  b$  \point{1}
\end{minipage}
\hfill
\begin{minipage}{0.49\linewidth}
	\item $ x \leq -2$  \point{1}
	\item $0 \geq x $  \point{1}
	\item Soient $a$ un réel, $x <  a$  \point{1}
\end{minipage}
\end{enumerate} 
\end{ExoCad}

  
\begin{ExoCad}{Représenter.}{1234}{0}{0}{0}{0}{0}

Soit $x$ un nombre. Écrire sous forme d'inégalités  les  intervalles suivants :

\begin{enumerate}[leftmargin=*]
\begin{minipage}{0.3\linewidth}
	\item $x \in [-3;2]$  \point{1}
	\item $x \in ]-\infty;3]$  \point{1}
\end{minipage}
\hfill
\begin{minipage}{0.3\linewidth}
	\item $x \in ]0;1]$  \point{1}
	\item $x \in ]1;+\infty[$  \point{1}
\end{minipage}
\hfill
\begin{minipage}{0.3\linewidth}
	\item $x \in ]-5;8[$  \point{1}
	\item $x \in [a;b]$  \point{1}
\end{minipage}
\end{enumerate} 
\end{ExoCad}



\begin{ExoCad}{Représenter.}{1234}{0}{0}{0}{0}{0}
\begin{enumerate}[leftmargin=*]
	\item Écrire l'ensemble des réels strictement supérieurs à $-2$ et inférieurs à $4$  \point{1}
	\item Écrire l'ensemble des réels supérieurs à $2$ et inférieurs à $6$  \point{1}
	\item Écrire l'ensemble des réels inférieurs à $-1$  \point{1}
\end{enumerate}
\end{ExoCad}



\begin{ExoCad}{Représenter.}{1234}{0}{0}{0}{0}{0}

Recopier et compléter le tableau.

\begin{tabular}{|c|c|c|}
\hline 
Intervalle & Inégalité & Représentation  sur la droite graduée  \\ 
\hline 
$x\in \left[ -6 ; \dfrac{2}{7}\right]$ & $-6  \leq x \leq  \dfrac{2}{7} $  &   \\ 
\hline 
 & $-3 \leq x <5$ &    \\ 
\hline 
$x\in \left[ 5 ; 8 \right[ $  &  &     \\ 
\hline 
 &  & 
 \definecolor{ffdxqq}{rgb}{1.,0.8431372549019608,0.}
\definecolor{ffxfqq}{rgb}{1.,0.4980392156862745,0.}
\begin{tikzpicture}[line cap=round,line join=round,>=triangle 45,x=1.0cm,y=1.0cm]
\draw[->,color=black] (-5.174092090680384,0.) -- (2.566282833730012,0.);
\foreach \x in {-5.,-4.,-3.,-2.,-1.,1.,2.}
\draw[shift={(\x,0)},color=black] (0pt,2pt) -- (0pt,-2pt) node[below] {\footnotesize $\x$};
\draw[color=black] (0pt,-10pt) node[right] {\footnotesize $0$};
\clip(-5.174092090680384,-0.4115875953650586) rectangle (2.566282833730012,0.4791698364123281);
\draw [line width=2.4pt,color=ffxfqq] (-4.,0.)-- (1.,0.);
\draw [color=ffdxqq](-4.2,0.35) node[anchor=north west] {\Large{]}};
\draw [color=ffdxqq](0.88 ,0.35) node[anchor=north west] {\Large{]}};
\end{tikzpicture} 
 \\ 
\hline 
\end{tabular} 
 
\end{ExoCad}



\begin{ExoCad}{Représenter.}{1234}{0}{0}{0}{0}{0}

Représenter sur une droite graduée les intervalles donnés. Soit $x$ un réel,
\begin{enumerate}[leftmargin=*]
	\item $-10< x < 3$  \point{1}
	\item $ x \in [0;\pi]$  \point{1}
	\item $ x \geq \dfrac{1}{3}$  \point{1}
	\item l'ensemble des réels strictement supérieurs à $-1$ et inférieurs à $3$  \point{1}
\end{enumerate}
\end{ExoCad}


 
 
 
\begin{ExoCad}{Raisonner. Communiquer.}{1234}{0}{0}{0}{0}{0}

Complète par  $\in$ ou $\notin$.

\begin{enumerate}
\begin{minipage}{0.32\linewidth}
\item $4 \ldots \ldots [4;5]$ \vspace{0.1cm}
\item $1,5 \ldots \ldots [1;3]$\vspace{0.1cm}
\item $-5,9 \ldots \ldots ]-\infty;-6]$
\end{minipage}
\hfill
\begin{minipage}{0.32\linewidth}
\item $3 \ldots \ldots [0;3[$\vspace{0.1cm}
\item $\dfrac{2}{3} \ldots \ldots [2;3]$\vspace{0.1cm}
\item $\dfrac{1}{2} \ldots \ldots ]-1;1[$
\end{minipage}
\hfill
\begin{minipage}{0.32\linewidth}
\item $\dfrac{5}{3} \ldots \ldots \left[ \dfrac{5}{6}; \dfrac{11}{6}  \right]$ 
\item $\sqrt{5} \ldots \ldots \left[ 1;3 \right]$\vspace{0.1cm}
\item $\sqrt{24} \ldots \ldots \left[0; 4\right]$ 
\end{minipage}

\end{enumerate}
 
\end{ExoCad}
 
 
 
\end{pageAD}
 
 \begin{pageCours}
 
 
\section{Opérations d'ensembles de nombres et logique mathématique}
 
\begin{minipage}[t]{0.69\linewidth}

\begin{DefT}{Inclusion}\index{Ensemble!Inclusion}

\begin{minipage}{0.58\linewidth}
Un ensemble $A$ \textbf{est inclus dans} un ensemble $B$ lorsque \textbf{tous} les éléments de $A$ sont contenus dans $B$. On note $A \subset B$. \point{3}
\end{minipage}
\hfill
\begin{minipage}{0.38\linewidth}

\definecolor{ffttww}{rgb}{1.,0.2,0.4}
\definecolor{xdxdff}{rgb}{0.49019607843137253,0.49019607843137253,1.}
\begin{tikzpicture}[line cap=round,line join=round,>=triangle 45,x=1.0cm,y=1.0cm]
\clip(1.32,0.48) rectangle (6.42,3.82);
\draw [rotate around={-1.123302714075422:(3.77,2.23)},line width=2.pt,color=xdxdff,fill=xdxdff,fill opacity=0.30000001192092896] (3.77,2.23) ellipse (2.1702715668569548cm and 1.5389212695611632cm);
\draw [rotate around={-61.97549946792974:(3.77,2.32)},line width=2.pt,color=ffttww,fill=ffttww,fill opacity=0.3499999940395355] (3.77,2.32) ellipse (1.1237882510756807cm and 0.8772685069325904cm);
\draw (4.76,3.48) node[anchor=north west] {$B$};
\draw (3.6,3.24) node[anchor=north west] {$A$};
\end{tikzpicture}
\end{minipage}
\end{DefT}
\end{minipage}
\begin{minipage}[t]{0.28\linewidth}
\begin{Rq}
\begin{description}[leftmargin=*]
\item Un ensemble \textbf{est inclus dans} un ensemble. $A$ et $B$ deux ensembles : $A \subset B$.
\item Un élément \textbf{appartient à} un ensemble. $x$ un élément et $A$ un ensemble : $x \in B$.
\end{description}
\end{Rq}
\end{minipage}


\begin{minipage}[t]{0.55\linewidth}
\begin{LogT}{Le contre exemple}

$\Z$ est-il inclus dans $\N$ ? Autrement dit, tous les éléments de $\Z$ sont -ils contenus dans $\N$ ?\\ On cherche  un seul élément de $\Z$ qui n'appartient pas à $\N$ : $-2 \in \Z$ mais $-2 \not\in \N$ donc $\Z \not\subset \N$. 

$-2$ est appelé un \textbf{contre exemple}\index{Contre exemple!Logique}. 
 

\end{LogT}
\end{minipage}
\begin{minipage}[t]{0.43\linewidth}
 
\begin{LogT}{La contra-posée} 
Soit $A$ et $B$ deux ensembles tels que $A \subset B$. 

Ces deux propositions disent la même idée :

Si $x \in A$ alors $x \in B \Longleftrightarrow$ Si $x \not\in B$ alors $x \not\in A$.

$\sqrt{3} \not\in \Q \Rightarrow \sqrt{3} \not\in \Z$
\end{LogT}

\end{minipage}

\begin{DefT}{Complémentaire}\index{Ensemble!Complémentaire}

\begin{minipage}{0.68\linewidth}
Soit $\Omega$ un ensemble contenant un ensemble $A$. On appelle \textbf{complémentaire} de $A$ dans $\Omega$, \textbf{tous} les éléments de $\Omega$ qui n'appartiennent pas à $A$. \point{2}
Si $\Omega=\lbrace{0;1;2;3;4;5;6\rbrace}$ et $A=\lbrace{3;4;6\rbrace}$ alors $\Omega\setminus A=\lbrace{0;1;2;5\rbrace}$
\end{minipage}
\begin{minipage}{0.28\linewidth}

\definecolor{ffffff}{rgb}{1.,1.,1.}
\definecolor{xfqqff}{rgb}{0.4980392156862745,0.,1.}
\definecolor{ffttww}{rgb}{1.,0.2,0.4}
\definecolor{xdxdff}{rgb}{0.49019607843137253,0.49019607843137253,1.}
\definecolor{ududff}{rgb}{0.30196078431372547,0.30196078431372547,1.}
\begin{tikzpicture}[line cap=round,line join=round,>=triangle 45,x=1.0cm,y=1.0cm]
\clip(1.32,0.5) rectangle (5.9,3.96);
\draw [rotate around={0.916654256385289:(3.49,2.28)},line width=2.pt,color=xdxdff,fill=xdxdff,fill opacity=0.5] (3.49,2.28) ellipse (2.0420114277198222cm and 1.6145930357022946cm);
\draw [rotate around={-61.97549946792974:(2.99,2.34)},line width=2.pt,color=ffttww,fill=ffttww,fill opacity=1.0] (2.99,2.34) ellipse (1.1237882510756807cm and 0.8772685069325904cm);
\draw [color=xfqqff](5.36,3.78) node[anchor=north west] {$B$};
\draw [color=ffffff](2.32,3.3) node[anchor=north west] {$A$};
\draw (3.8,3.48) node[anchor=north west] {$B\setminus A$};
\end{tikzpicture}

\end{minipage}
\end{DefT}
 


 
\begin{minipage}[t]{0.68\linewidth}
\begin{DefT}{Intersection}\index{Ensemble!Intersection}

\begin{minipage}{0.58\linewidth}
L'\textbf{intersection} de deux ensembles $A$ et $B$ est l'ensemble $A \cap B$ qui contient \textbf{tous} les éléments communs aux deux ensembles. 

On lit $A$ "inter" $B$. 

L'ensemble $A \cap B$ est l'ensemble qui réunit les éléments de $A$ \textbf{et} de $B$ pris une seule fois. \point{2}
Si $A=\lbrace{0;1;2;3;4;5;6\rbrace}$ et $B=\lbrace{3;4;7;8\rbrace}$ alors $A \cap B=\lbrace{3;4 \rbrace}$
\end{minipage}
\hfill
\begin{minipage}{0.38\linewidth}

\definecolor{ffttww}{rgb}{1.,0.2,0.4}
\definecolor{xdxdff}{rgb}{0.49019607843137253,0.49019607843137253,1.}
\begin{tikzpicture}[line cap=round,line join=round,>=triangle 45,x=1.0cm,y=1.0cm]
\clip(-0.06,0.86) rectangle (4.72,4.1);
\draw [rotate around={-2.7263109939062526:(3.08,2.22)},line width=2.pt,color=xdxdff,fill=xdxdff,fill opacity=0.30000001192092896] (3.08,2.22) ellipse (1.44092369450198cm and 1.1700688412983384cm);
\draw [rotate around={-61.97549946792966:(1.29,2.28)},line width=2.pt,color=ffttww,fill=ffttww,fill opacity=0.3499999940395355] (1.29,2.28) ellipse (1.123788251075678cm and 0.8772685069325883cm);
\draw (4.16,3.52) node[anchor=north west] {$B$};
\draw (0.28,3.98) node[anchor=north west] {$A$};
\draw [->,line width=1.pt] (1.84,3.38) -- (1.88,2.16);
\draw (1.24,3.98) node[anchor=north west] {$A\cap B$};
\end{tikzpicture}

\end{minipage}

\end{DefT}

\end{minipage}
\begin{minipage}[t]{0.28\linewidth}


\begin{Rq}\index{Ensembles disjoints}
Deux ensembles sont \textbf{disjoints} lorsque $A \cap B = \oslash$. $\oslash$ est l'ensemble vide.\index{Ensemble!vide}
\vspace{0.2cm}

Si $C=\lbrace{0;1;2;3\rbrace}$ 

et $D=\lbrace{4;5;6\rbrace}$ 

alors $A \cap B= \oslash $

\definecolor{xdxdff}{rgb}{0.49019607843137253,0.49019607843137253,1.}
\definecolor{ffttww}{rgb}{1.,0.2,0.4}
\begin{tikzpicture}[line cap=round,line join=round,>=triangle 45,x=1.0cm,y=1.0cm]
\clip(-5.398001213875861,3.010943854292776) rectangle (-0.9359834713983067,4.7475129216353915);
\draw [rotate around={9.950626687951598:(-4.276467024550424,3.8551093731398804)},line width=2.pt,color=ffttww,fill=ffttww,fill opacity=0.15000000596046448] (-4.276467024550424,3.8551093731398804) ellipse (0.8703197956159994cm and 0.5200054996195858cm);
\draw [rotate around={-0.9240453527727059:(-2.0695771681358472,3.7706928212551696)},line width=2.pt,color=xdxdff,fill=xdxdff,fill opacity=0.25] (-2.0695771681358472,3.7706928212551696) ellipse (0.9314625917474598cm and 0.5553715874828972cm);
\draw (-4.867382887743395,4.10984671409655) node[anchor=north west] {$1$};
\draw (-4.360883576435132,3.892775580678723) node[anchor=north west] {$2$};
\draw (-4.553835695028756,4.35103686233858) node[anchor=north west] {$3$};
\draw (-4.095574413368899,4.182203758569159) node[anchor=north west] {$0$};
\draw (-2.624314509092516,4.158084743744956) node[anchor=north west] {$5$};
\draw (-1.7801489902454115,4.158084743744956) node[anchor=north west] {$6$};
\draw (-2.3590053460262834,3.823942476909302) node[anchor=north west] {$4$};
\end{tikzpicture}

\end{Rq}
\end{minipage}


\begin{DefT}{Réunion}\index{Ensemble!Réunion}

\begin{minipage}{0.68\linewidth}
La \textbf{réunion} de deux ensembles est l'ensemble $A \cup B$ qui contient \textbf{tous} les éléments de $A$ \textbf{ou} de $B$ pris une seule fois. 

On lit $A$ "union" $B$. \point{2}

Si $A=\lbrace{0;1;2;3;4;5;6\rbrace}$ et $B=\lbrace{3;4;7;8\rbrace}$ alors $A \cup B=\lbrace{0;1;2;3;4;5;6;7;8 \rbrace}$
\end{minipage}
\hfill
\begin{minipage}{0.28\linewidth}

\definecolor{ffqqqq}{rgb}{1.,0.,0.}
\definecolor{xfqqff}{rgb}{0.4980392156862745,0.,1.}
\definecolor{ffttww}{rgb}{1.,0.2,0.4}
\definecolor{xdxdff}{rgb}{0.49019607843137253,0.49019607843137253,1.}
\begin{tikzpicture}[line cap=round,line join=round,>=triangle 45,x=1.0cm,y=1.0cm]
\clip(0.06,0.8) rectangle (4.82,4.12);
\draw [rotate around={-2.7263109939062526:(3.08,2.22)},line width=2.pt,color=xdxdff,fill=xdxdff,fill opacity=0.5] (3.08,2.22) ellipse (1.44092369450198cm and 1.1700688412983384cm);
\draw [rotate around={-61.97549946792966:(1.29,2.28)},line width=2.pt,color=ffttww,fill=ffttww,fill opacity=0.550000011920929] (1.29,2.28) ellipse (1.123788251075678cm and 0.8772685069325883cm);
\draw [color=xfqqff](4.16,3.52) node[anchor=north west] {$B$};
\draw [color=ffqqqq](0.28,3.98) node[anchor=north west] {$A$};
\draw (1.36,3.96) node[anchor=north west] {$A\cup B$};
\draw [rotate around={-2.7263109939062526:(3.08,2.22)},line width=1.pt,fill=black,pattern=north east lines,pattern color=black] (3.08,2.22) ellipse (1.44092369450198cm and 1.1700688412983384cm);
\draw [rotate around={-61.97549946792966:(1.29,2.28)},line width=1.pt,fill=black,pattern=north east lines,pattern color=black] (1.29,2.28) ellipse (1.123788251075678cm and 0.8772685069325883cm);
\end{tikzpicture}


\end{minipage}

\end{DefT}






\end{pageCours} 
\begin{pageAD} 
 





\Sf{Opérer avec les ensembles}
 
  
\begin{ExoCad}{Communiquer.}{1234}{2}{0}{0}{0}{0}

Compléter par $\subset$  ou $\not\subset$. Justifier l'utilisation de  $\not\subset$.

\begin{enumerate}
\begin{minipage}{0.49\linewidth}
\item $\Z \ldots \ldots \R$ \point{1}
\item $\Q \ldots \ldots \Z$ \point{1}
\item $\Z \ldots \ldots \N$ \point{1}
\end{minipage}
\hfill
\begin{minipage}{0.49\linewidth}
\item $\Z \ldots \ldots \D$ \point{1}
\item $\D \ldots \ldots \Q$ \point{1}
\item $\Q \ldots \ldots \N$ \point{1}
\end{minipage}
\end{enumerate}
 
\end{ExoCad}

\begin{ExoCad}{Représenter.}{1234}{0}{0}{0}{0}{0}


On propose dans chaque cas deux ensembles. Lequel est inclus dans l'autre ?  

\begin{enumerate}
\item $[-1,1;3]$ et $]-2,9;6]$
\item $[0,7;0,8]$ et $[0,5;+\infty[$
\item $]1;2[$ et $[1;2]$
\item $\Q$ et $\Z$
\end{enumerate} 
 
\end{ExoCad}




\begin{ExoCad}{Raisonner.}{1234}{0}{0}{0}{0}{0}

Déterminer les complémentaires de l'ensemble $A$ par rapport $\Omega$. 
\begin{enumerate}
\item $\Omega = \lbrace 1;2;3;4;5;6\rbrace$ et $A = \lbrace 1;2 \rbrace$. $\Omega \setminus A = $\point{1}
\item $\Omega = \lbrace -4;-2;-1;0;1;2;\rbrace$ et $A = \lbrace -2;1;2 \rbrace$.  $\Omega \setminus A = $\point{1}
\item $\Omega = \R$ et $A = \Q$. $\Omega \setminus A = $\point{1}
\item $\Omega = \R$ et $A = \R^-$. $\Omega \setminus A = $\point{1}
\end{enumerate} 
 \end{ExoCad}


\begin{ExoCad}{Représenter.}{1234}{0}{0}{0}{0}{0}


Déterminer les intersections des ensembles $A$ et $B$ suivants.   $A \cap B = $ se lit $A$ inter $B$.
\begin{enumerate}
\item $A = \lbrace 1;2;3;4;5;6\rbrace$ et $B = \lbrace 1;2 \rbrace$. $A \cap B = $\point{1}
\item $A = \lbrace -4;-2;-1;0;1;2;\rbrace$ et $B = \lbrace -2;1;2 \rbrace$.  $A \cap B = $\point{1}
\item $A = \N$ et $B = \R$. $A \cap B = $ \point{1}
\item $A = [-4;3[$ et $B =[-2;7]$. $A \cap B = $ \point{1}
\item $A = [-2;1]$ et $B =[2;3]$. $A \cap B = $ \point{1}
\item $A = \N$ et $B =]-\infty;5]$. $A \cap B = $ \point{1}
\end{enumerate} 
 
\end{ExoCad}









\begin{ExoCad}{Représenter.}{e/1234}{0}{0}{0}{0}{0}


Déterminer les réunions des ensembles suivants. On écrira : $A \cup B = $ où $A$ et $B$ sont les ensembles ci-dessous.
\begin{enumerate}
\item $\Z$ et $\R$
\item $\left\lbrace 1;2;8;6  \right\rbrace $ et $\left\lbrace 0;2;4;8  \right\rbrace $
\item $[-2;1]$ et $[2;3]$
\item $[0;+\infty[$ et $]-\infty;5]$
\end{enumerate} 
 
\end{ExoCad}



 
\end{pageAD}
 
  
%%%%%%%%%%%%%%%%%%%%%%%%%%%%%%%%%%%%%%%%%%%%%%%%%%%%%%%%%%%%%%%%%%%
%%%%  Niveau 1
%%%%%%%%%%%%%%%%%%%%%%%%%%%%%%%%%%%%%%%%%%%%%%%%%%%%%%%%%%%%%%%%%%%
\begin{pageParcoursu}

\begin{ExoCu}{Communiquer.}{1234}{2}{0}{0}{0}{0}

Johan visite Vienne, la capitale de l'Autriche. L' Autriche est un pays Européen.
\begin{enumerate}[leftmargin=*]
\item Arrivé en Autriche, Johan est-il à Vienne ? \point{1}
\item A Vienne, Johan est-il en Autriche ?\point{1}
\item Peut-on dire que Johan est en Europe ?\point{1}
\item Lettres $A$, $E$ et $V$ sont les initiales respectives de Autriche, Europe et Vienne. Compléter avec les lettres $A$, $E$ et $V$ : $\ldots\ldots \subset \ldots\ldots \subset \ldots\ldots $
\end{enumerate}


\end{ExoCu}


\begin{ExoCu}{Communiquer.}{1234}{2}{0}{0}{0}{0}

Recopie et complète par $\subset$, $\in$, $\not\subset$ ou $\notin$.

\begin{enumerate}
\begin{minipage}{0.32\linewidth}
\item $\N \ldots \ldots \R$
\item $-5 \ldots \ldots \Z$
\item $\Z \ldots \ldots \N$
\end{minipage}
\hfill
\begin{minipage}{0.32\linewidth}
\item $\left\lbrace 0;1;2 \right\rbrace \ldots \ldots \N$
\item $]-\infty;1] \ldots \ldots \R$
\item $]0;+\infty[ \ldots \ldots \Z$
\end{minipage}
\hfill
\begin{minipage}{0.32\linewidth}
\item $\sqrt{3} \ldots \ldots \N$
\item $-1,5 \ldots \ldots \Z$
\item $\Q \ldots \ldots \Z$
\end{minipage}
\end{enumerate}
 
\end{ExoCu}



\begin{ExoCu}{Communiquer.}{1234}{2}{0}{0}{0}{0}

Déterminer les intersections des ensembles suivants.
\begin{enumerate}
\item $A = \Z$ et $B = \Q$. $A \cap B = $ \point{1}
\item $A = ]-2;3[$ et $B =[-1;5]$. $A \cap B = $ \point{1}
\item $A = [-4;3]$ et $B =[2;6]$. $A \cap B = $ \point{1}
\item $A = ]-6;1[$ et $B =[0;5]$. $A \cap B = $ \point{1}
\end{enumerate}
\end{ExoCu}


\begin{ExoCu}{Chercher.communiquer.}{2}{0}{0}{0}{0}
 
Dans chaque cas, trouver, lorsque cela est possible, un nombre $x$ qui remplit les critères suivants :
\begin{enumerate}
\item $x \in \Q$ et $x \not\in \Z$ \point{1}
\item $x \in \R$ et $x \not\in \N$ \point{1}
\end{enumerate}
\end{ExoCu}

 





\begin{ExoCu}{Représenter. Raisonner. Communiquer.}{1234}{2}{0}{0}{0}{0}


Déterminer, dans chaque cas, la réunion des ensembles suivants. On écrira : $A \cup B = $ où $A$ et $B$ sont les ensembles ci-dessous.
\begin{enumerate}
\item $A=\left\lbrace 1;3;5;7  \right\rbrace $ et $B=\left\lbrace 0;2;4;5;7;8  \right\rbrace $\point{1}
\item $A=[-3;4]$ et $B=[2;6]$\point{1}
\item $A=[0;+\infty[$ et $B=]-\infty;5]$\point{1}
\end{enumerate}
On pourra représenter les intervalles sur une droite graduée tracée à main levée.
 \end{ExoCu}



\begin{ExoCu}{Modéliser.}{1234}{0}{0}{0}{0}{0}


Déterminer l'ensemble des valeurs de $x$ dans chaque cas.
\begin{enumerate}
\item On jette un dé à 6 face et on regarde la face obtenue. Soit $x$ le numéro de la face. 
\item Le segment $[AB]$ mesure 8 cm. Soit I le milieu de $[AB]$ et $M$ un point de $[AI]$. $AM = x$. 
\item $x < -4$ et $x \geq 10$
\item $x \leq 6$ et $x \leq 3$
\item $x \leq 6$ ou $x \geq 3$
\end{enumerate} 
 
\end{ExoCu}
 


\end{pageParcoursu}
  
  
%%%%%%%%%%%%%%%%%%%%%%%%%%%%%%%%%%%%%%%%%%%%%%%%%%%%%%%%%%%%%%%%%%%
%%%%  Niveau 2
%%%%%%%%%%%%%%%%%%%%%%%%%%%%%%%%%%%%%%%%%%%%%%%%%%%%%%%%%%%%%%%%%%%
\begin{pageParcoursd} 

 

\begin{ExoCd}{Représenter. Raisonner.}{1234}{0}{0}{0}{0}{0}

Déterminer l'intervalle des valeurs de $x$ dans chaque cas.

\begin{enumerate}
\begin{minipage}{0.5\linewidth}
%
Déterminer l'ensemble des valeurs de $x$ dans chaque cas.
\begin{enumerate}
\item On jette un dé à 6 face et on regarde la face obtenue. Soit $x$ le numéro de la face. 
\item $[-1,1;3]$ et $[2,9;6]$
\item $x > -4$ et $x \leq 10$
\item $x \leq -3$ et $x \leq 5$
\item $x \leq 5$ ou $x \geq 2$
\end{enumerate} 

\item $[-1,1;3]$ et $[2,9;6]$
\item $x > -4$ et $x \leq 10$
\item $x \leq -3$ et $x \leq 5$


\end{minipage}
\begin{minipage}{0.5\linewidth}
%
Soit $x$ un réel.Écrire sous forme d'intervalle ou de réunion d'intervalles le plus grand ensemble auquel appartient $x$.


\begin{enumerate}
	\item $x \geq 1$ ou $x<3$.  
	\item $x \geq 1$ et $x<3$.  
\end{enumerate}
 
\item $x \geq 1$ ou $x<3$.  
\item $x \geq 1$ et $x<3$.  
\item $x \leq 5$ ou $x \geq 2$
\end{minipage}
\end{enumerate}

\end{ExoCd}


\begin{ExoCd}{Représenter.}{1234}{0}{0}{0}{0}{0}

Dans chaque cas, trouver, lorsque cela est possible, un nombre $x$ qui remplit les conditions suivantes :
 
\begin{enumerate}[leftmargin=*]
\item $x \not\in D$ et $x \in \R$  \point{1}
\item $x \in \Q$ et $x \not\in \Z$  \point{1}
\item $x \not\in \N$ et $x \in \Z$  \point{1}
\item $x \in \D$ ou $x \in \Q$  \point{1}
\item $x \not\in \N$ ou $x \in \Z$  \point{1}
\end{enumerate}
 
 \end{ExoCd}


 

\begin{ExoCd}{Représenter. Raisonner. Communiquer.}{1234}{0}{0}{0}{0}{0}


On propose dans chaque cas deux ensembles. Lequel est inclus dans l'autre ? Écrire ensuite une phrase :" $x$ appartient à .... donc $x$ appartient à ....."

\begin{enumerate}
\item $\left[ -\frac{11}{10};\frac{29}{10}\right]$ et $\left[-\frac{3}{2};3 \right]$
\item $\left[ \frac{1}{2}; +\infty \right[$ et $[0,7;0,8]$.
\item $[1;2]$ et $]1;2[$. 
\end{enumerate} 
 
\end{ExoCd}


\begin{ExoCd}{Représenter. Raisonner. Communiquer.}{1234}{0}{0}{0}{0}{0}


Déterminer les intersections des ensembles suivants. On écrira : $A \cap B = $ où $A$ et $B$ sont les ensembles ci-dessous.
 

\textit{{\small On pourra représenter chaque intervalle sur une droite graduée.}}



\begin{minipage}{0.48\linewidth}

\begin{enumerate}
\item $\Z$ et $\Q$
\item $[-5;2[$ et $[0;7]$
\item $[-1;4]$ et $[-3;-1]$
\item $\N$ et $]-\infty;5]$
\item $[-5;0[$ et $[0;3]$
\end{enumerate}

\end{minipage}
\hfill
\begin{minipage}{0.48\linewidth}
 
\begin{enumerate}
\item

\begin{tikzpicture}[line cap=round,line join=round,>=triangle 45,x=1.0cm,y=1.0cm]
\draw [->,line width=1.pt,domain=0.34:6.36] plot(\x,{(-14.-0.*\x)/7.});
\end{tikzpicture}
\item

\begin{tikzpicture}[line cap=round,line join=round,>=triangle 45,x=1.0cm,y=1.0cm]
\draw [->,line width=1.pt,domain=0.34:6.36] plot(\x,{(-14.-0.*\x)/7.});
\end{tikzpicture}
\item

\begin{tikzpicture}[line cap=round,line join=round,>=triangle 45,x=1.0cm,y=1.0cm]
\draw [->,line width=1.pt,domain=0.34:6.36] plot(\x,{(-14.-0.*\x)/7.});
\end{tikzpicture}
\item

\begin{tikzpicture}[line cap=round,line join=round,>=triangle 45,x=1.0cm,y=1.0cm]
\draw [->,line width=1.pt,domain=0.34:6.36] plot(\x,{(-14.-0.*\x)/7.});
\end{tikzpicture}
\item

\begin{tikzpicture}[line cap=round,line join=round,>=triangle 45,x=1.0cm,y=1.0cm]
\draw [->,line width=1.pt,domain=0.34:6.36] plot(\x,{(-14.-0.*\x)/7.});
\end{tikzpicture}
\end{enumerate}

\end{minipage} 
 
\end{ExoCd}







\begin{ExoCd}{Chercher.}{1234}{0}{0}{0}{0}{0}


On donne le programme en Python ci dessous. 
 
\begin{lstlisting}
def is_in(x,a,b):
    if x > a and x < b :
    	test = "{} is in  ]{};{}[".format(x,a,b) 
    else :
        test = "{} is not in ]{};{}[".format(x,a,b) 
    return test    
x=int(input("Entrer un nombre  :")) 
a=int(input("Entrer la borne inf :"))
b=int(input("Entrer la borne sup :"))    
print(is_in(x,a,b))
\end{lstlisting}
 


\begin{enumerate}
\item Ouvrir le logiciel PyScripter et taper ce code. Que fait ce programme ? Vous pouvez aussi ouvrir en ligne l'éditeur Python : \url{https://www.tutorialspoint.com/execute_python_online.php}
\item Modifier ce programme pour qu'il teste si un nombre $x$ appartient à l'intervalle $[a;b]$.
\end{enumerate} 
 
\end{ExoCd}




\end{pageParcoursd}
 
%%%%%%%%%%%%%%%%%%%%%%%%%%%%%%%%%%%%%%%%%%%%%%%%%%%%%%%%%%%%%%%%%%%%
%%%%%  Niveau 3
%%%%%%%%%%%%%%%%%%%%%%%%%%%%%%%%%%%%%%%%%%%%%%%%%%%%%%%%%%%%%%%%%%%%
\begin{pageParcourst}


\begin{ExoCtN}{Représenter.}{1}{1}{0}{0}{0}

Démontrer que si $p^2$ est impair alors $p$ est impair.

 
\end{ExoCtN}

\begin{DefT}{Nombre décimal périodique}
Le nombre $a_0,\underline{a_1a_2a_3}$ est un nombre décimal périodique de période $a_1a_2a_3$. Les chiffres $a_1$, $a_2$, $a_3$ se répètent indéfiniment.
\end{DefT}

\begin{ExoCtN}{Représenter. Calculer.}{1234}{0}{0}{0}{0} 
 
Démontrer que $0,\underline{9}=1$.   \point{6}
 
\end{ExoCtN}

 
\begin{ExoCtN}{Représenter. Calculer.}{1234}{1}{0}{0}{0}

\begin{enumerate}
\item On considère le nombre $\frac{19}{11}$.

\begin{enumerate}
\item Donner le développement décimal de $\frac{19}{11}$ avec 8 chiffres significatifs. $\frac{19}{11}$ semble-t-il décimal ?
\item On dit que $\frac{19}{11}$ a une écriture périodique.
Préciser sa période (série de chiffres qui se répète à l'infini dans le développement décimal).
\end{enumerate}
\item On considère le nombre $x=0,13131313....$ dont le développement décimal a pour période 13.
\begin{enumerate}
\item Démontrer que $100x = 13 + x$. 
\item  En déduire une écriture fractionnaire de $x$. Quelle est la nature du nombre $x$ ?
\end{enumerate}
\end{enumerate} 
 
\end{ExoCtN}

\begin{ExoCtN}{Représenter. Calculer.}{2}{1}{0}{0} 

\begin{enumerate}
\item Démontrer que $0,\underline{12}$ est un nombre rationnel à préciser.
\item Démontrer que $x=3,\underline{412}$ est un nombre rationnel. 
\end{enumerate}
\end{ExoCtN} 
 

%%%%%%%%%%%%%%%%%%%%%%%%%%%%%%%%%%%%%%%%%%%%%%%%%%%%%%%%%%%%%%%%%%%
\begin{ExoCt}{Raisonner.}{1234}{2}{0}{0}{0}{0}
 
Représenter graphiquement dans le plan muni d'un repère orthonormal 
 
\begin{enumerate}
\item l'ensemble des points $M(x;y)$ tes que  $$1 < x < 3 \text{ et} 0 \leq y <4$$
\item l'ensemble des points $M(x;y)$ tes que  $$1 \leq  x \leq  5 \text{ et } -2 \leq y   \leq  1$$
\end{enumerate}
 
\end{ExoCt}


%%%%%%%%%%%%%%%%%%%%%%%%%%%%%%%%%%%%%%%%%%%%%%%%%%%%%%%%%%%%%%%%%%%
\begin{ExoCt}{Représenter.}{1234}{2}{0}{0}{0}{0}
Représenter graphiquement, dans le plan muni d'un repère orthonormal, l'ensemble des points $M(x;y)$ tes que  $$1 \leq  2x+1 \leq  5 \text{ et }  -2 \leq 3y + 4  \leq  13$$
\end{ExoCt} 
 

\end{pageParcourst}
 
\begin{pageAuto}

\begin{ExoAutoN}{Raisonner.}{2}{0}{0}{0}{0}

Compléter avec $\in$, $\not\in$, $\subset$ ou $\not\subset$.

 \begin{tabular}{ccc}

$\frac{2}{10}..........\Z$ & $-\sqrt{25}..........\Z$ & $\frac{\sqrt{3}}{4}..........\Q$ \\ 

$\pi..........\R$  & $-\frac{5}{3}..........\Q$  &  $\sqrt{11}..........\R$ \\ 

\end{tabular} 

\end{ExoAutoN}
%%%%%%%%%%%%%%%%%%%%%%%%%%%%%%%%%%%%%%%%%%%%%%%%%%%%%%%%%%%%%%%%%%%
\begin{ExoAutoN}{Raisonner.}{2}{0}{0}{0}{0}
 

Déterminer, dans chaque cas, l'intersection puis la réunion des ensembles suivants. 

\begin{enumerate}
\item $A=\left\lbrace 1;3;5;7  \right\rbrace $ et $B=\left\lbrace 0;2;4;5;7;8  \right\rbrace $\point{2}
\item $A=[-3;4]$ et $B=[2;6]$\point{2}
\item $A=[0;+\infty[$ et $B=]-\infty;5]$\point{2}
\end{enumerate}
On pourra représenter les intervalles sur une droite graduée tracée à main levée.
 
 

\end{ExoAutoN}
%%%%%%%%%%%%%%%%%%%%%%%%%%%%%%%%%%%%%%%%%%%%%%%%%%%%%%%%%%%%%%%%%%%
\begin{ExoAutoN}{Représenter. Raisonner.}{2}{0}{0}{0}{0}

 Recopier et compléter le tableau.

\begin{tabular}{|c|c|c|}
\hline 
Intervalle & Inégalité & Représentation   \\ 
\hline 
$x\in \left[ -2 ; 6\right]$ & $-2  \leq x \leq  6 $  &  \\ 
\hline 
 & $1 \leq x <3$ &     \\ 
\hline 
$x\in \left[ -6 ; 6 \right[ $  &  &  \\ 
\hline 
 &  & \definecolor{ffdxqq}{rgb}{1.,0.8431372549019608,0.}
\definecolor{ffxfqq}{rgb}{1.,0.4980392156862745,0.}
\begin{tikzpicture}[line cap=round,line join=round,>=triangle 45,x=1.0cm,y=1.0cm]
\draw[->,color=black] (-5.174092090680384,0.) -- (2.566282833730012,0.);
\foreach \x in {-5.,-4.,-3.,-2.,0,-1.,1.,2.}
\draw[shift={(\x,0)},color=black] (0pt,2pt) -- (0pt,-2pt) node[below] {\footnotesize $\x$};
\clip(-5.174092090680384,-0.4115875953650586) rectangle (2.566282833730012,0.4791698364123281);
\draw [line width=2.4pt,color=ffxfqq] (-4.,0.)-- (1.,0.);
\draw [color=ffxfqq](0.8,0.35) node[anchor=north west] {\Large{]}};
\draw [color=ffxfqq](-4.2,0.35) node[anchor=north west] {\Large{]}};
\end{tikzpicture}    \\ 
\hline 
\end{tabular} 
 

\end{ExoAutoN}


\begin{ExoAutoN}{Raisonner.}{2}{0}{0}{0}{0}

Démontrer que $0,\underline{485}$ est un nombre rationnel à préciser.

\end{ExoAutoN}


\end{pageAuto}
