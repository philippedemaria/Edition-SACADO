
Dans un village, il y a deux boulangerie. On considère les événements : 

\begin{description}
\item[A] la boulangerie A est ouverte ;
\item[B] la boulangerie B est ouverte.
\end{description}

On sait que $p(A)=0,6$ et $p(B)=0,8$.

De plus, il y a toujours au moins un des deux boulangeries ouvertes. Exprimer chacun des événements suivants en fonction des événements A et B et déterminer leur probabilité.
\begin{enumerate}
\item D : "Au moins une des deux boulangeries est ouverte". 
\item E : "Aucune des boulangeries n'est ouverte".
\item F : "Les deux boulangeries sont ouvertes".
\end{enumerate}