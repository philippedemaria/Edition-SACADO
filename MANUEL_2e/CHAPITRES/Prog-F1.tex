\begin{titre}[Programmation]

\Titre{Moyenne pondérée et tableau}{2}
\end{titre}


\begin{CpsCol}
\textbf{Python}
\begin{description}
\item[$\square$] Utiliser des tableaux
\end{description}
\end{CpsCol}


\begin{PC}
\input{Prog-10}
\end{PC}

\begin{DefT}{Tableau}
Un \index{Tableau}tableau est un ensemble ordonné d'éléments portant le même nom de variable. Les éléments sont repérés par rapport à leur position dans le tableau par un \index{tableau!Indice}indice.

L'indice d'un tableau commence toujours à 0.
\end{DefT}


\begin{Ex}
Soit tab un tableau de fruits tel que tab=['fraise','cerise','Myrtille','goyave','coco']. tab[0]='fraise', tab[4]='coco'.
\end{Ex}


\begin{minipage}[t]{0.49\linewidth}
\begin{Syn}
\begin{description}[leftmargin=*]
\item[tab=['1','a']] déclaration du tableau par extension.
\item[tab=[]] Le tableau peut être déclaré vide.  
\item[tab.append(a)] Ajout de l'élément a à la fin du tableau. On peut aussi utiliser Tab=Tab+[a]
\item[tab.sort()] Tri le tableau selon l'ordre croissant ou alphabétique.
\item[tab.reverse()] Changement d'ordre des éléments
\end{description}
\end{Syn}
\end{minipage}
\hfill
\begin{minipage}[t]{0.49\linewidth}
\begin{Cod}
\begin{description}[leftmargin=*]
\item \texttt{tab=[]}
\item \texttt{{\color{orange}for} i {\color{orange}in}  {\color{violet}range}(10):}
\item \hspace{0,4cm}\texttt{tab.append(i)}
\item \hspace{0,4cm}\texttt{{\color{violet}print}(tab)}
\item \texttt{tab.sort()}
\item \texttt{tab.reverse()}
\end{description}

\end{Cod}
\end{minipage}



\mini{
\AD{1}{AP-50}
}{
\AD{1}{AP-51}
}

\begin{DefT}{Chaine de caractère}
Une \index{Chaine de caractère}chaine de caractère est une liste ordonnée de caractères. Chaque caractère est repéré par un indice.

L'indice d'une chaine commence toujours à 0.
\end{DefT}


\begin{DefT}{Longueur de chaine, de tableau} \index{Chaine de caractère!Longueur}\index{Tableau!Longueur}
Pour déterminer la longueur d'une chaine ou d'un tableau, on utilise mot anglais "length", qui est diminué en Python par {\color{violet}len}
\end{DefT}

\begin{DefT}{Concaténation}
On appelle \index{Concaténation}concaténation l'opération consistant à mettre bout à bout des éléments de deux tableaux ou deux chaines.
\end{DefT}

\begin{Ex}
\begin{minipage}[t]{0.49\linewidth}
\textbf{Avec une chaine}

scr1 ="Bon" et scr2 ="jour"\\
scr = scr1 + scr2\\
scr ="Bonjour"
\end{minipage}
\hfill\vrule\hfill
\begin{minipage}[t]{0.49\linewidth}
\textbf{Avec un tableau}

tab1=['1','a'] et tab2=['b']\\
tab = tab1 + tab2\\
tab = ['1','a','b']
\end{minipage}
\end{Ex}

\begin{Mt}[ pour lire les termes d'une chaine par rapport à leur indice]
\begin{description}
\item[•] Pour lire la chaine \texttt{str} du $i^{ième}$ terme jusqu'au $p^{ième}$ non inclus, on écrit \texttt{str[i:p]}
\item[•] Pour lire la chaine \texttt{str} jusqu'au $i^{ième}$ terme non inclus, on écrit \texttt{str[:i]}
\item[•] Pour lire la chaine \texttt{str} du $i^{ième}$ terme jusqu'au dernier, on écrit \texttt{str[i:]}
\end{description}
\end{Mt}


\begin{minipage}{0.48\linewidth}
\begin{Ex}
Le mot \textbf{rase} est une chaine de caractère contenant 4 caractères. 

On appelle \texttt{mot} la chaine de caractère "rase" donc mot[0]=r, mot[1]=a, mot[2]=s et mot[3]=e. On veut remplacer la lettre s par la lettre m.
\end{Ex}
\end{minipage}
\hfill
\begin{minipage}{0.48\linewidth}
\begin{Cod}
\begin{description}[leftmargin=*]
\item \texttt{mot={\color{vert}"rase"}}
\item \texttt{{\color{orange}for} i {\color{orange}in}  {\color{violet}range}({\color{violet}len}(mot)):}
\item \hspace{0,5cm}\texttt{{\color{orange}if} mot[i] == "s" : }
\item \hspace{1cm}\texttt{mot = mot[:i] + {\color{vert}'m'} + mot[i+1:]}
\item \hspace{0,5cm}\texttt{{\color{violet}print}(mot)}
\end{description}
\end{Cod}
\end{minipage}

\mini{
\AD{1}{AP-57}
}{
\AD{1}{Prog-11}
}

\AD{1}{AP-64}








