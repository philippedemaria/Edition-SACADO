
Le coucou est un oiseau qui fait couver ses œufs par des oiseaux d'autres espèces de tailles très
différentes. Une étude a été faite sur des œufs déposés dans des nids de petite taille (nids de roitelets)
ou de grande taille (nids de fauvettes).
Le tableau suivant donne en mm le diamètre des œufs.

\begin{tabular}{|c|c|}
\hline 
nids de roitelets & 19,8 - 22,1 - 21,5 - 20,9 - 22 - 22,3 - 21 - 20,3 - 20,9 - 22 - 20,8 - 21,2 - 21 \\ 
\hline 
nids de fauvettes & 22 - 23,9 - 20,9 - 23,8 - 25 - 24 - 23,8 - 21,7 - 22,8 - 23,1 - 23,5 - 23 - 23,1 \\ 
\hline 
\end{tabular} 


\begin{enumerate}
\item Donner pour chacune des deux séries la moyenne, la médiane et l'étendue.
\item  Regrouper les valeurs des deux séries en classes, en précisant les fréquences pour chaque classe.

Prendre $[19 ; 20[$, $[20 ; 21[$, $[21 ; 22[$, $[22 ; 23[$ pour la première série ; $[20 ; 21[$, $[21 ; 22[$,
$[22 ; 23[$, $[23 ; 24[$, $[24 ; 25]$ pour la deuxième.
\item  Représenter sur un même graphique les histogrammes donnant la distribution des fréquences en
utilisant deux couleurs différentes.
\item  Au vu de ces résultats, quelle hypothèse peut formuler le biologiste concernant l'existence d'un lien
entre la taille des nids et celle des œufs déposés ?\end{enumerate}