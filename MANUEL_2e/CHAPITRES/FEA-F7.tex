\begin{titre}[Calcul littéral]

\Titre{Système d'équations}{2}
\end{titre}

\begin{CpsCol}
\begin{description}
\item[$\square$] Résoudre un système d'équations
\end{description}
\end{CpsCol}




\paragraphe{Pre-requis}


\begin{ThT}{Condition de colinéarité}
Deux vecteurs $\vec u$ et $\vec v$ sont colinéaires si et seulement si dét$\left( \vec u,\vec v \right)$=0.
\end{ThT}


\begin{ThT}{Coordonnées de vecteur directeur}
Soit $d$ la droite d'équation cartésienne $ax+by+c =0$. Un vecteur directeur a pour coordonnées $(-b;a)$.
\end{ThT}


\begin{ThT}{Coordonnées de vecteur directeur}
Soit $d$ la droite d'équation réduite $y = mx+p$. Un vecteur directeur a pour coordonnées $(1;m)$.
\end{ThT}



\begin{ThT}{Appartenance d'un point à une droite}
Un point M de coordonnées $(x_M;y_M)$ appartient à la droite d'équation $ax+by+c=0$ si et seulement si $ax_M+by_M+c=0$.
\end{ThT}


\begin{ThT}{Appartenance d'un point à une courbe représentative}
Soit $f$ une fonction définie sur un intervalle I.

Un point M de coordonnées $(x_M;y_M)$ appartient à une courbe d'équation $y=f(x)$ si et seulement si $y_M=f(x_M)$.
\end{ThT}



\paragraphe{Exercices}

\begin{minipage}{0.48\linewidth}
\EPCN{Calculer.}

Soit $d$ la droite d'équation cartésienne $2x+3y-1=0$\\ et $d'$ la droite d'équation cartésienne $3x-1y-1=0$.
 
\begin{enumerate}
\item Déterminer un vecteur directeur de $d$.
\item Déterminer un vecteur directeur de $d'$.
\item Les droites $d$ et $d'$ sont-elles parallèles.
\end{enumerate}

\end{minipage}
\hfill
\begin{minipage}{0.48\linewidth}


\EPCN{Calculer.}

Soit $d$ la droite d'équation cartésienne $-x+2y-4=0$\\  et $d'$ la droite d'équation cartésienne $2x+4y-1=0$.
 
\begin{enumerate}
\item Déterminer un vecteur directeur de $d$.
\item Déterminer un vecteur directeur de $d'$.
\item Les droites $d$ et $d'$ sont-elles parallèles.
\end{enumerate}

\end{minipage}
 


\begin{minipage}{0.48\linewidth}
\EPCN{Calculer.}

Soit $d$ la droite d'équation cartésienne $6x-2y-6=0$\\  et $d'$ la droite d'équation cartésienne $3x-1y=3$.
 
\begin{enumerate}
\item Déterminer un vecteur directeur de $d$.
\item Déterminer un vecteur directeur de $d'$.
\item Les droites $d$ et $d'$ sont-elles parallèles.
\end{enumerate}

\end{minipage}
\hfill
\begin{minipage}{0.48\linewidth}


\EPCN{Calculer.}

Soit $d$ la droite d'équation cartésienne $y=\frac{2}{3}x+2$ \\ et $d'$ la droite d'équation cartésienne $y=2x+6$.
 
\begin{enumerate}
\item Déterminer un vecteur directeur de $d$.
\item Déterminer un vecteur directeur de $d'$.
\item Les droites $d$ et $d'$ sont-elles parallèles.
\end{enumerate}

\end{minipage}

 \begin{minipage}{0.48\linewidth}

\EPCN{Calculer }

Résoudre le système d'équation suivant.

$\left\lbrace  \begin{tabular}{lcc}
$y$ & = & $2x-3$ \\ 
$y$ & = & $-x+6$ \\ 
\end{tabular} \right. $

\end{minipage}
\hfill
\begin{minipage}{0.48\linewidth}

\EPCN{Calculer }
 
Résoudre le système d'équation suivant.

$\left\lbrace  \begin{tabular}{lcc}
$y$ & = & $x-1$ \\ 
$y$ & = & $x+1$ \\ 
\end{tabular} \right. $

\end{minipage}

\begin{minipage}{0.48\linewidth}

\EPCN{Calculer }

Résoudre le système d'équation suivant.

$\left\lbrace  \begin{tabular}{lcc}
$x+y$ & = & 4 \\ 
$2x+3y$ & = & 7 \\ 
\end{tabular} \right. $

\end{minipage}
\hfill
\begin{minipage}{0.48\linewidth}

\EPCN{Calculer }
 
Résoudre le système d'équation suivant.

$\left\lbrace  \begin{tabular}{lcc}
$4x-y$ & = & 3 \\ 
$x+5y$ & = & -2 \\ 
\end{tabular} \right. $

\end{minipage}

\EPCN{Modéliser. Calculer. }

Dans une ferme, il y a des lapins et des poules. On compte 120 têtes et 298 pattes. Combien la ferme compte-t-elle de poules et de lapins ?



\EPCNM{Modéliser. Calculer.  }

Dans le panier de M.Marchais il y a 5 kg de pommes et 2kg de carottes. Dans le panier de Madame Simson, il y a 3kg de pommes et 7kg de carottes. Madame Marchais a payé 18,5 euros et madame Simson a payé 28,5 euros. Quel est la prix d'un kg de pommes et d'un kilogramme de carottes ?


\EPCN{Représenter. Calculer. }

Résoudre le système d'équation suivant.

$\left\lbrace  \begin{tabular}{lcc}
$x^2+y^2$ & = & 25 \\ 
$2x^2-y^2$ & = & 23 \\ 
\end{tabular} \right. $

