\begin{titre}[Probabilités]

\Titre{Tests et boucles en Python}{1}
\end{titre}


\begin{CpsCol}
\begin{description}
\item[$\square$] Connaitre la syntaxe de test
\item[$\square$] Connaitre la syntaxe de boucle
\end{description}
\end{CpsCol}

\subsection*{Instruction conditionnelle}

\color{orange} if \color{black}  Condition :

\hspace{0.4cm}    instruction 1
    
\color{orange} else \color{black} :

\hspace{0.4cm}     instruction 2

\subsection*{La boucle for}

La boucle ci dessous génère 5 nombres aléatoires naturels de 1 à 6

\color{orange} import \color{black} random

\color{orange} for \color{black} i \color{orange} in \color{purple}range\color{black}(5):

\hspace{0.4cm}     x = random.randint(1,6)
 
\hspace{0.4cm}   \color{purple} print \color{black}(x)

\subsection*{La boucle While}
\color{orange} import \color{black} random

i=1

\color{orange}  while \color{black} i <= 5:

\hspace{0.4cm}    x = random.randint(1,6)
    
\hspace{0.4cm}   \color{purple} print \color{black}   (x)
   
\hspace{0.4cm} \color{black}     i=i+1