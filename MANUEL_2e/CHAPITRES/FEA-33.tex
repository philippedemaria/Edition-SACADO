Le carré ABCD a un côté de longueur 8 cm.

M est un point du segment [AB].On dessine dans le carré ABCD :
\begin{enumerate}
\item Un carré de côté [AM]
\item Un triangle isocèle de base [MB] et dont la hauteur amême mesure que le côté [AM] du carré.
\end{enumerate}

Trois dessins sont proposés pour trois positions différentes du point M.

\begin{center}
\definecolor{ffxfqq}{rgb}{1.,0.4980392156862745,0.}
\definecolor{uuuuuu}{rgb}{0.26666666666666666,0.26666666666666666,0.26666666666666666}
\definecolor{qqqqff}{rgb}{0.,0.,1.}
\begin{tikzpicture}[line cap=round,line join=round,>=triangle 45,x=1.0cm,y=1.0cm]
\clip(-1.68,-2.26) rectangle (13.56,2.84);
\draw (-1.,-2.)-- (3.,-2.);
\draw (3.,-2.)-- (3.,2.);
\draw (3.,2.)-- (-1.,2.);
\draw (-1.,2.)-- (-1.,-2.);
\draw (4.,-2.)-- (8.,-2.);
\draw (8.,-2.)-- (8.,2.);
\draw (8.,2.)-- (4.,2.);
\draw (4.,2.)-- (4.,-2.);
\draw (9.,-2.)-- (13.,-2.);
\draw (13.,-2.)-- (13.,2.);
\draw (13.,2.)-- (9.,2.);
\draw (9.,2.)-- (9.,-2.);
\draw (-1.,-2.)-- (0.,-2.);
\draw (0.,-2.)-- (0.,-1.);
\draw (0.,-1.)-- (-1.,-1.);
\draw (-1.,-1.)-- (-1.,-2.);
\draw [color=ffxfqq] (0.,-2.)-- (3.,-2.);
\draw [color=ffxfqq] (3.,-2.)-- (1.54,-1.);
\draw [color=ffxfqq] (1.54,-1.)-- (0.,-2.);
\draw (4.,-2.)-- (6.,-2.);
\draw (6.,-2.)-- (6.,0.);
\draw (6.,0.)-- (4.,0.);
\draw (4.,0.)-- (4.,-2.);
\draw (9.,-2.)-- (12.,-2.);
\draw (12.,-2.)-- (12.,1.);
\draw (12.,1.)-- (9.,1.);
\draw (9.,1.)-- (9.,-2.);
\draw [color=ffxfqq] (6.,-2.)-- (7.,0.);
\draw [color=ffxfqq] (7.,0.)-- (8.,-2.);
\draw [color=ffxfqq] (8.,-2.)-- (6.,-2.);
\draw [color=ffxfqq] (12.,-2.)-- (12.5,0.98);
\draw [color=ffxfqq] (12.5,0.98)-- (13.,-2.);
\draw [color=ffxfqq] (13.,-2.)-- (12.,-2.);
\begin{scriptsize}

\draw [color=qqqqff] (-1.,-2.)-- ++(-1.5pt,0 pt) -- ++(3.0pt,0 pt) ++(-1.5pt,-1.5pt) -- ++(0 pt,3.0pt);
\draw[color=qqqqff] (-1.26,-1.72) node {$A$};
\draw [color=qqqqff] (3.,-2.)-- ++(-1.5pt,0 pt) -- ++(3.0pt,0 pt) ++(-1.5pt,-1.5pt) -- ++(0 pt,3.0pt);
\draw[color=qqqqff] (3.14,-1.72) node {$B$};
\draw [color=uuuuuu] (3.,2.)-- ++(-1.5pt,0 pt) -- ++(3.0pt,0 pt) ++(-1.5pt,-1.5pt) -- ++(0 pt,3.0pt);
\draw[color=uuuuuu] (3.14,2.28) node {$C$};
\draw [color=uuuuuu] (-1.,2.)-- ++(-1.5pt,0 pt) -- ++(3.0pt,0 pt) ++(-1.5pt,-1.5pt) -- ++(0 pt,3.0pt);
\draw[color=uuuuuu] (-1.26,2.28) node {$D$};
\draw [color=qqqqff] (4.,-2.)-- ++(-1.5pt,0 pt) -- ++(3.0pt,0 pt) ++(-1.5pt,-1.5pt) -- ++(0 pt,3.0pt);
\draw[color=qqqqff] (4.14,-1.72) node {$A$};
\draw [color=qqqqff] (8.,-2.)-- ++(-1.5pt,0 pt) -- ++(3.0pt,0 pt) ++(-1.5pt,-1.5pt) -- ++(0 pt,3.0pt);
\draw[color=qqqqff] (8.14,-1.72) node {$B$};
\draw [color=uuuuuu] (4.,2.)-- ++(-1.5pt,0 pt) -- ++(3.0pt,0 pt) ++(-1.5pt,-1.5pt) -- ++(0 pt,3.0pt);
\draw[color=uuuuuu] (4.14,2.28) node {$D$};
\draw [color=qqqqff] (9.,-2.)-- ++(-1.5pt,0 pt) -- ++(3.0pt,0 pt) ++(-1.5pt,-1.5pt) -- ++(0 pt,3.0pt);
\draw[color=qqqqff] (9.14,-1.72) node {$A$};
\draw [color=qqqqff] (13.,-2.)-- ++(-1.5pt,0 pt) -- ++(3.0pt,0 pt) ++(-1.5pt,-1.5pt) -- ++(0 pt,3.0pt);
\draw[color=qqqqff] (13.14,-1.72) node {$B$};
\draw [color=uuuuuu] (13.,2.)-- ++(-1.5pt,0 pt) -- ++(3.0pt,0 pt) ++(-1.5pt,-1.5pt) -- ++(0 pt,3.0pt);
\draw[color=uuuuuu] (13.14,2.28) node {$C$};
\draw [color=uuuuuu] (9.,2.)-- ++(-1.5pt,0 pt) -- ++(3.0pt,0 pt) ++(-1.5pt,-1.5pt) -- ++(0 pt,3.0pt);
\draw[color=uuuuuu] (9.14,2.28) node {$D$};
\draw [color=uuuuuu] (8.,2.)-- ++(-1.5pt,0 pt) -- ++(3.0pt,0 pt) ++(-1.5pt,-1.5pt) -- ++(0 pt,3.0pt);
\draw[color=uuuuuu] (8.14,2.28) node {$C$};
\end{scriptsize}
\end{tikzpicture}
\end{center}

Dans quelle situation a-t-on l'aire du triangle la plus grande ?