\begin{titre}[Géométrie vectorielle et analytique]

\Titre{Notions de vecteur}{4}
\end{titre}

\begin{CpsCol}
\textbf{Connaitre la notion de vecteur}
\begin{description}
\item[$\square$] Représenter géométriquement des vecteurs.
\item[$\square$] Représenter un vecteur dont on connaît les coordonnées. Lire les coordonnées d’un
vecteur.
\end{description}
\end{CpsCol}



\begin{DefT}{Vecteur}\index{Vecteurs}
On dit que le vecteur $\overrightarrow{AB}$ est le représentant du déplacement de A vers B. Le vecteur $\overrightarrow{AB}$ possède donc une \textbf{direction}, un \textbf{sens}, une longueur appelée \textbf{norme}. 
Ces 3 données caractérisent un vecteur.

La translation de vecteur $\overrightarrow{AB}$ est le déplacement de A vers B.
\end{DefT}


 
 
\begin{ThT}{Vecteurs égaux} \index{Vecteurs!Égaux} 
\begin{minipage}{0.48\linewidth}
Deux vecteurs $\overrightarrow{AB}$ et $\overrightarrow{CD}$ sont égaux lorsque ils ont la même direction (leurs supports sont parallèles), le même sens, la même norme $(AB=CD)$.

On écrit $\overrightarrow{AB}=\overrightarrow{CD}$. 

Le quadrilatère $ABDC$ est donc un parallélogramme.
\end{minipage}
\hfill
\begin{minipage}{0.48\linewidth}
\definecolor{ffqqqq}{rgb}{1.,0.,0.}
\definecolor{qqqqff}{rgb}{0.,0.,1.}
\begin{tikzpicture}[line cap=round,line join=round,>=triangle 45,x=1.0cm,y=1.0cm]
\clip(-2.46,-0.22) rectangle (4.58,2.42);
\draw [->,color=ffqqqq] (-2.,0.) -- (1.,2.);
\draw [color=ffqqqq](-0.82,1.46) node[anchor=north west] {$\vec{u}$};
\draw [->,color=ffqqqq] (1.,0.) -- (4.,2.);
\draw [color=ffqqqq](2.06,1.38) node[anchor=north west] {$\vec{v}$};
\begin{scriptsize}
\draw [fill=qqqqff] (-2.,0.) circle (0.5pt);
\draw[color=qqqqff] (-2.24,0.) node {$A$};
\draw [fill=qqqqff] (1.,2.) circle (0.5pt);
\draw[color=qqqqff] (1.14,2.2) node {$B$};
\draw [fill=qqqqff] (1.,0.) circle (0.5pt);
\draw[color=qqqqff] (0.7,-0) node {$C$};
\draw [fill=qqqqff] (4.,2.) circle (0.5pt);
\draw[color=qqqqff] (4.14,2.2) node {$D$};
\end{scriptsize}
\end{tikzpicture}
\end{minipage}
\end{ThT}


\begin{DefT}{Vecteur nul}\index{Vecteurs!Nul}
On dit que le vecteur $\overrightarrow{AB}$ est nul lorsque $A$ et $B$ sont confondus. On note $\overrightarrow{AB}=\overrightarrow{0}$
\end{DefT}




\mini{

\EPC{1}{GVA-32}{Représenter. Raisonner.}

\EPC{1}{GVA-33}{Représenter. Raisonner.}

\EPC{1}{GVA-35}{Représenter. Raisonner.}
}{
\EPC{1}{GVA-34}{Représenter. Calculer.}

\begin{DefT}{Vecteurs opposés}\index{Vecteurs!Opposés}
Deux vecteurs sont dits opposés lorsqu'ils ont la même direction, la même norme et sont de sens opposés. $\overrightarrow{AB}$ et  $\overrightarrow{BA}$ sont opposés. On écrit : $\overrightarrow{AB} = -\overrightarrow{BA}$.
\end{DefT}
}


\EPC{1}{GVA-40}{Représenter}

\begin{DefT}{Base et repère du plan}\index{Base du plan}\index{Repère du plan}
Deux vecteurs non colinéaires forment une \textbf{base} du plan. On le note $\left(\vec{u};\vec{v}\right)$.  \\ 
Deux vecteurs non colinéaires et un point du plan forment un \textbf{repère} du plan. On la note \Oij. Un repère permet de repérer des points dans le plan. Les coordonnées des points sont dépendantes de l'origine $O$.
\end{DefT}




\begin{ThT}{Coordonnées de vecteurs}\index{Vecteurs!Coordonnées}

Deux vecteurs égaux ont des coordonnées égales.
$\overrightarrow{AB}(x,y)$ et $\overrightarrow{CD}(x',y')$.

$\overrightarrow{AB}=\overrightarrow{CD} \Longleftrightarrow x=x'$ et $y=y'$.

Dans un repère on donne deux points $A(x_A; y_A)$ et $B(x_B;y_B)$. Les coordonnées du vecteur $\overrightarrow{AB}$ sont $(x_B – x_A; y_B – y_A)$.
\end{ThT}


\mini{
\EPC{1}{GVA-37}{Représenter}

\EPC{0}{GVA-46}{Représenter. Calculer}
}{
\EPC{1}{GVA-44}{Représenter}

\EPC{0}{GVA-36}{Représenter. Calculer}
}
