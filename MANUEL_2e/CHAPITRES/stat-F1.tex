\begin{titre}[Statistiques descriptives]

\Titre{Moyenne, écart type}{4}
\end{titre}


\begin{CpsCol}
\textbf{Utiliser des nombres pour calculer et résoudre des problèmes}
\begin{description}
\item[$\square$] Calculer les caractéristiques moyenne et écart type d'une série définie par effectifs ou par fréquences
\end{description}
\end{CpsCol}

\EPC{1}{stat-15}{Calculer}


\begin{DefT}{Moyenne Pondérée}
La \textbf{moyenne} de la série statistique contenant $p$ valeurs donnée par le tableau :

\begin{tabular}{|c|c|c|c|c|c|}
\hline 
Valeur du caractère & $x_1$ & $x_2$ & $\cdots$ & $x_{p-1}$ & $x_p$ \\ 
\hline 
effectif & $n_1$ & $n_2$ & $\cdots$ & $n_{p-1}$ & $n_p$ \\ 
\hline 
\end{tabular} 

est le nombre noté $\overline{x}$, défini par  $$\overline{x}= \frac{n_1x_1+n_2x_2+\cdots+n_px_p}{n_1+n_2+\cdots+n_p}$$

Lorsque la série est donnée par classes, on peut calculer la moyenne en remplaçant chaque classe par son
milieu.

 La \textbf{moyenne} d'une série statistique est un indicateur de tendance centrale.
\end{DefT}



\begin{Th}
On peut calculer la moyenne $\overline{x}$ à partir de la distribution des fréquences :  $\overline{x} = f_1 x_1+ f_2 x_2+\cdots+f_p x_p$
\end{Th}


\begin{minipage}{0.48\linewidth}
\Fl{1}{stat-19} 

\EPC{1}{stat-18}{Représenter. Calculer}

\begin{Rq}
Lorsque la série est donnée par classes, la valeur du caractère est le milieu de la classe. On considère donc que la classe est \textbf{équirépartie}.
\end{Rq}
\end{minipage}
\hfill
\begin{minipage}{0.48\linewidth}
\EPC{0}{stat-17}{Représenter. Calculer}



\end{minipage}

 
 
 
 
 
 

\begin{Th}
Soient $a$ et $b$ deux réels. Si une série de valeurs $(x_i)_{1 \leq i \leq p}$ a pour moyenne $\overline{x}$ alors la série $(ax_i + b)_{1 \leq i \leq p}$ a pour moyenne $a\overline{x}+b$ .
\end{Th} 
 
 
\EPC{1}{stat-35}{Chercher. Modéliser.}
 
 
 
\begin{DefT}{Écart type}\index{Écart type}
L'écart type d'une série statistique est un indicateur de dispersion, c'est à dire qu'il mesure la dispersion des valeurs autour de la moyenne.
\end{DefT}
 
 

\begin{DefT}{Formule de l'écart type}\index{Écart type!Formule}
L'écart type de la série statistique contenant $p$ valeurs donnée par le tableau :

\begin{tabular}{|c|c|c|c|c|c|}
\hline 
Valeur du caractère & $x_1$ & $x_2$ & $\cdots$ & $x_{p-1}$ & $x_p$ \\ 
\hline 
effectif & $n_1$ & $n_2$ & $\cdots$ & $n_{p-1}$ & $n_p$ \\ 
\hline 
\end{tabular} 

est le nombre \textbf{positif} noté $\sigma$ ou $\sigma_x$, défini par  $$\sigma = \sqrt{\frac{n_1(x_1-\overline{x})^2 +n_2(x_2-\overline{x})^2 +\cdots+n_p(x_p-\overline{x})^2 }{n_1+n_2+\cdots+n_p}}$$
\end{DefT}

\begin{Rq}
Plus l'écart type est grand, plus les valeurs sont dispersées autour de la moyenne.
\end{Rq}

\begin{minipage}{0.48\linewidth}
\EPC{1}{stat-21}{Chercher. Calculer.}

\EPC{1}{stat-8}{Raisonner. Calculer}
\end{minipage}
\hfill
\begin{minipage}{0.48\linewidth}
\EPC{1}{stat-9}{Modéliser. Calculer}

\EPC{0}{stat-12}{Modéliser. Calculer}

\EPC{0}{stat-23}{Chercher. Modéliser. Calculer} 
\end{minipage}


\mini{
\EPC{1}{stat-32}{Représenter.}

\EPC{1}{stat-33}{Représenter.}
}{
\EPCP{1}{stat-34}{Représenter.}
}
