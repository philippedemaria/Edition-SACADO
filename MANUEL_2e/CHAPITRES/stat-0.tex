
Le tableau suivant indique la population (en millions d’habitants) et la densité de population (en $hab/km^2$) des pays du Proche-Orient.

\begin{tabular}{|c|c|c|}
\hline 
Pays & Population (Millions) & Densité ($hab/km^2$) \\ 
\hline 
Arabie Saoudite & 20,9 & 9,7 \\ 
\hline 
Barhein & 0,7 & 700 \\ 
\hline 
E.A.U & 2,8 & 33,3 \\ 
\hline 
Égypte & 66,9 & 66,8 \\ 
\hline 
Iran & 66,2 & 40,1 \\ 
\hline 
Irak & 22,5 & 51,8 \\ 
\hline 
Israël & 6,1 & 290,4 \\ 
\hline 
Jordanie & 4,7 & 47,9 \\ 
\hline 
Koweït & 2,1 & 116,6 \\ 
\hline 
Liban & 4,1 & 410 \\ 
\hline 
Oman & 2,5 & 11,7 \\ 
\hline 
Qatar & 0,5 & 45,5 \\ 
\hline 
Syrie & 16 & 86,4 \\ 
\hline 
Yemen & 16,4 & 31 \\ 
\hline 
\end{tabular} 

\begin{enumerate}
\item On considère la série statistique des populations.
\begin{enumerate}
\item  Calculer la moyenne, la médiane et l’étendue de cette série.
\item Quel(s) est (sont) le(s) pays dont la population est la plus voisine de la moyenne ? de la médiane ?
\end{enumerate}
\item  On considère la série statistique des densités.
Répondre, pour cette série, à la question 1.
\item  Pour la série des populations, le Bahreïn et le Qatar d’une part, l’Egypte et l’Iran d’autre part, ont des valeurs exceptionnelles. Calculer la moyenne, l’étendue et la médiane de la série des populations, élaguée de ces quatre valeurs ;
\item \begin{enumerate}
\item  Pour la série des densités, calculer la moyenne élaguée de la densité du Bahreïn, ainsi que la médiane de cette nouvelle série.
\item  Calculer la différence entre cette moyenne élaguée et la moyenne initiale, puis entre cette médiane et la médiane initiale.
\item Quels commentaires vous inspirent les résultats de la question b).
\end{enumerate}
\item 
\begin{enumerate}
\item  Déterminer la population totale du Proche-Orient.
\item  Calculer la superficie de chaque pays. Quelle relation lie la population, la superficie, la densité ?
Déterminer alors la superficie totale des pays du Proche-Orient.
\item  En déduire la densité des pays du Proche-Orient.
\end{enumerate}
\end{enumerate}