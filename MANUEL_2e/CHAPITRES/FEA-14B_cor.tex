
On propose dans chaque cas deux ensembles. Lequel est inclus dans l'autre ? Écrire ensuite une phrase :" $x$ appartient à .... donc $x$ appartient à ....."

\begin{enumerate}
\item $\left[ -\frac{11}{10};\frac{29}{10}\right] \subset \left[-\frac{3}{2};3 \right]$. " $x$ appartient à $\left[ -\frac{11}{10};\frac{29}{10}\right]$ donc $x$ appartient à $\left[-\frac{3}{2};3 \right]$ 
\item $[0,7;0,8] \subset \left[ \frac{1}{2}; +\infty \right[$.  $x$ appartient à $[0,7;0,8]$ donc $x$ appartient à $\left[ \frac{1}{2}; +\infty \right[$"
\item $]1;2[ \subset [1;2]$.  $x$ appartient à $]1;2[$ donc $x$ appartient à $[1;2]$"
\end{enumerate}

\textbf{Attention}, il n'est pas possible d'écrire les phrases dans l'autre sens.