\begin{titre}[Géométrie vectorielle et analytique]

\Titre{Caractérisation d'une droite}{4}
\end{titre}


\begin{CpsCol}
\textbf{Résoudre des problèmes de géométrie dans le plan}
\begin{description}
\item[$\square$] Déterminer si deux droites sont parallèles ou sécantes.
\item[$\square$] Résoudre un système de deux équations linéaires à deux inconnues, déterminer le point d'intersection de deux droites sécantes.
\end{description}
\end{CpsCol}

%\Rec{1}{ED-9}
\Rec{1}{ED-00}

\paragraphe{Équation cartésienne}

\begin{DefT}{La droite du plan}
Soit $A$ et $B$ deux points du plan. On appelle la droite $(AB)$ l'ensemble des points $M(x;y)$ du plan  tels que $A$, $B$ et $M$ sont alignés.
\end{DefT}



\begin{DefT}{Vecteur directeur}
On appelle vecteur directeur  d'une droite $d$ tout représentant du vecteur $\overrightarrow{AB}$ où $A$ et $B$ sont deux points distincts de la droite $d$.
\end{DefT}

\EPC{1}{ED-14}{Calculer.}



\begin{ThT}{Caractériser analytiquement une droite}
Dans un repère, les coordonnées de l'ensemble des points $M(x;y)$ d'une droite vérifient une relation $ax+by+c =0$ où $a$, $b$ et $c$ sont des nombres réels.
\end{ThT}


\EPC{1}{ED-24}{Calculer.}

\begin{DefT}{Équation cartésienne}
$a$, $b$ et $c$ sont des nombres réels. La relation $ax+by+c =0$ s'appelle une équation cartésienne  de la droite $d$.
\end{DefT}



\begin{ThT}{Coordonnées de vecteur directeur}
Soit $d$ la droite d'équation cartésienne $ax+by+c =0$. Un vecteur directeur a pour coordonnées $(-b;a)$.
\end{ThT}


 \EPC{1}{ED-16}{Représenter.}




\begin{Rq} 
Un vecteur directeur de la droite $d$ d'équation cartésienne $ax+by+c =0$ est $\overrightarrow{u}\left(1;-\frac{b}{a}\right)$.
\end{Rq}



\paragraphe{Coefficient directeur et équation réduite}

\begin{ThT}{Coefficient directeur}\index{Coefficient directeur}
Soit $d$ la droite d'équation cartésienne $ax+by+c =0$, $b\neq 0$. Le coefficient directeur de la droite $d$ est $m=-\frac{b}{a}$.
\end{ThT}


\begin{ThT}{Coefficient directeur à l'aide de deux points} \index{Coefficient directeur à l'aide de deux points}
Dans un repère, $A(x_A;y_A)$ et $B(x_B;y_B)$ sont deux points tels que $x_A \neq x_B$.

Le coefficient directeur $m$ de la droite (AB) est $ m = \frac{y_B-y_A}{x_B-x_A}$.

Pour  $x_A = x_B$, une équation de la droite $(AB)$ est $x=x_A$ ou $x=x_B$.
\end{ThT}


\begin{ThT}{Équation réduite d'une droite}
L'équation de la forme $y=mx+p$ s'appelle l'équation réduite de la droite $d$.
\end{ThT}




\EPC{1}{ED-13bis}{Représenter.}

\EPC{1}{ED-14bis}{Représenter.}




%
%\begin{ThT}{Caractériser analytiquement une droite}
%Dans un repère, l'ensemble des points du plan $M(x;y)$ tels $y=mx+p$ ou $x=k$, $k \in \mathbb{R}$, est une droite\index{Droites}.
%\end{ThT}
%
%
%\begin{ThT}{Caractériser analytiquement une droite}
%Dans un repère, toute droite $d$ a une équation de la forme $y=mx+p$ ou $x=k$, $k \in \mathbb{R}$.
%\end{ThT}
%
%
%\begin{Rq}
%Soit $A(x_A,y_A)$ un point du plan dans un repère \Oij.
%
%Toute droite d'équation $x=x_A$ passe par $A$ et est parallèle à l'axe des ordonnées.
%
%Toute droite parallèle à l'axe des ordonnées passant par $A$ a pour équation $x=x_A$.
%\end{Rq}
%
%\begin{Rq}
%Soit $p \in \R$
%
%Toute droite parallèle à l'axe des abscisse a pour équation $y=p$.
%
%Toute droite d'équation $y=p$ est parallèle à l'axe des abscisses.
%\end{Rq}
%
%
%\mini{
%\EPC{1}{ED-10}{Représenter}
%
%\EPC{0}{ED-11}{Représenter}
%}{
%\EPC{1}{GVA-53}{Représenter}
%
%\EPC{0}{ED-12}{Calculer}
%}
%
%
%
%
%

 


\mini{
%\EPC{1}{ED-0}{Représenter.}

\EPC{0}{ED-2}{Représenter.}

\EPC{1}{ED-14}{Représenter.}

 
\EPC{0}{ED-3} {Représenter.}
}{
 \EPC{1}{ED-1}{Représenter.}

 \EPC{1}{ED-16}{Représenter.}
 
  \EPC{1}{ED-19}{Représenter.}
}

 \EPC{0}{ED-20}{Représenter. Calculer.}

 \EPC{0}{ED-21}{Représenter.}
 
\begin{Approfondissement}
 
  
\textbf{Le théorème de Pappus.} Sur la figure suivante, on a : $OA=AB=\frac{2}{3}BC ~~  \text{et} ~~ OA'=A'B'=\frac{1}{2}B'C'$


\definecolor{ffqqqq}{rgb}{1.,0.,0.}
\definecolor{qqwuqq}{rgb}{0.,0.39215686274509803,0.}
\definecolor{uuuuuu}{rgb}{0.26666666666666666,0.26666666666666666,0.26666666666666666}
\definecolor{xdxdff}{rgb}{0.49019607843137253,0.49019607843137253,1.}
\definecolor{qqqqff}{rgb}{0.,0.,1.}
\begin{tikzpicture}[line cap=round,line join=round,>=triangle 45,x=1.0cm,y=1.0cm]
\clip(-4.86,-0.88) rectangle (4.48,5.08);
\draw [domain=-4.86:4.48] plot(\x,{(--6.4-2.34*\x)/7.88});
\draw [domain=-4.86:4.48] plot(\x,{(--23.12--2.32*\x)/6.92});
\draw [color=qqqqff] (-1.7907078289181593,1.3439411573183366)-- (-0.54,3.16);
\draw [color=qqwuqq] (-1.7907078289181593,1.3439411573183366)-- (2.92,4.32);
\draw [color=qqqqff] (0.41858434216368146,0.6878823146366733)-- (-2.27,2.58);
\draw [color=ffqqqq] (0.41858434216368146,0.6878823146366733)-- (2.92,4.32);
\draw [color=ffqqqq] (3.88,-0.34)-- (-0.54,3.16);
\draw [color=qqwuqq] (3.88,-0.34)-- (-2.27,2.58);
\begin{scriptsize}
\draw [color=qqqqff] (-4.,2.)-- ++(-2.5pt,0 pt) -- ++(5.0pt,0 pt) ++(-2.5pt,-2.5pt) -- ++(0 pt,5.0pt);
\draw[color=qqqqff] (-3.86,2.37) node {$O$};
\draw [color=qqqqff] (3.88,-0.34)-- ++(-2.5pt,0 pt) -- ++(5.0pt,0 pt) ++(-2.5pt,-2.5pt) -- ++(0 pt,5.0pt);
\draw[color=qqqqff] (4.02,0.03) node {$C$};
\draw[color=black] (-6.42,2.57) node {$f$};
\draw [color=qqqqff] (2.92,4.32)-- ++(-2.5pt,0 pt) -- ++(5.0pt,0 pt) ++(-2.5pt,-2.5pt) -- ++(0 pt,5.0pt);
\draw[color=qqqqff] (3.12,4.69) node {$C'$};
\draw[color=black] (-6.42,1.07) node {$g$};
\draw [color=xdxdff] (-1.7907078289181593,1.3439411573183366)-- ++(-2.5pt,0 pt) -- ++(5.0pt,0 pt) ++(-2.5pt,-2.5pt) -- ++(0 pt,5.0pt);
\draw[color=xdxdff] (-1.66,1.71) node {$A$};
\draw [color=qqqqff] (0.41858434216368146,0.6878823146366733)-- ++(-2.5pt,0 pt) -- ++(5.0pt,0 pt) ++(-2.5pt,-2.5pt) -- ++(0 pt,5.0pt);
\draw[color=qqqqff] (0.56,1.05) node {$B$};
\draw [color=uuuuuu] (-0.54,3.16)-- ++(-2.5pt,0 pt) -- ++(5.0pt,0 pt) ++(-2.5pt,-2.5pt) -- ++(0 pt,5.0pt);
\draw[color=uuuuuu] (-0.34,3.53) node {$B'$};
\draw [color=uuuuuu] (-2.27,2.58)-- ++(-2.5pt,0 pt) -- ++(5.0pt,0 pt) ++(-2.5pt,-2.5pt) -- ++(0 pt,5.0pt);
\draw[color=uuuuuu] (-2.06,2.95) node {$A'$};
\draw [color=uuuuuu] (-1.3738052192787729,1.9492941048788914)-- ++(-1.5pt,0 pt) -- ++(3.0pt,0 pt) ++(-1.5pt,-1.5pt) -- ++(0 pt,3.0pt);
\draw[color=uuuuuu] (-1.78,2.07) node {$M$};
\draw [color=uuuuuu] (-0.8793327805531729,1.919715726701669)-- ++(-1.5pt,0 pt) -- ++(3.0pt,0 pt) ++(-1.5pt,-1.5pt) -- ++(0 pt,3.0pt);
\draw[color=uuuuuu] (-0.74,2.21) node {$N$};
\draw [color=uuuuuu] (1.1820198414004142,1.7964096278503507)-- ++(-1.5pt,0 pt) -- ++(3.0pt,0 pt) ++(-1.5pt,-1.5pt) -- ++(0 pt,3.0pt);
\draw[color=uuuuuu] (1.46,1.95) node {$P$};
\end{scriptsize}
\end{tikzpicture}

$M$ est le point d'intersection des droites $(AB')$ et $(A'B)$. \\
$N$ est le point d'intersection des droites $(AC')$ et $(A'C)$.\\
$P$ est le point d'intersection des droites $(BC')$ et $(B'C)$.\\

On se propose de démontrer que les points $M$, $N$ et $P$ sont alignés.

\begin{enumerate}
\item Dans le repère (O,OA,OA'), exprimer les coordonnées des points $A, B, C, A', B', C'$.
\item \begin{enumerate}
\item Déterminer une équation des droites $(AB')$ et $(A'B)$.
\item En déduire les coordonnées de $M$.
\end{enumerate}
\item Calculer de même les coordonnées des points $N$ et $P$.
\item Conclure.
\end{enumerate}
  
\end{Approfondissement}