\begin{titre}[Informations chiffrées]

\Titre{Taux d'évolution}{4}
\end{titre}


\begin{CpsCol}
\begin{description}
\item[$\square$] Exploiter la relation entre deux valeurs successives et leur taux d’évolution.
\end{description}
\end{CpsCol}

\EPC{1}{IC-41}{Modéliser. Calculer.}

\begin{DefT}{Variation absolue, relative}\index{Variation absolue}\index{Variation relative, Taux d'évolution}
On considère une quantité qui varie et on appelle $V_i$ sa valeur initiale et $V_f$ sa valeur finale. 
La \textbf{variation absolue} $\Delta V$ est donnée par : $\Delta V = V_f - V_i$.

La \textbf{variation relative} ou \textbf{taux d'évolution $t$} est donnée par : $t = \frac{V_f - V_i}{V_i}$.
\end{DefT}

\begin{Ex}
Paul place 110 euros sur son compte. Il se rend compte 15 jours plus tard qu'il possède 121 euros sur son compte.

La variation absolue est $\Delta V = 121 - 110 = +11$ euros.

La variation relative est $t = \frac{121 - 110}{110}=\frac{11}{110}= +10\%$.
\end{Ex}


%%%%%%% Exercices



\begin{DefT}{Taux d'évolution}\index{Taux d'évolution}

 Soit $t$ le \textbf{taux d'évolution} qui permet à une quantité de passer de $V_i$ à $V_f$.
 On a alors : $V_f = (1+t) \times V_i$.
\end{DefT}

 
\begin{DefT}{Coefficient multiplicateur}
 
Le réel $1+t$ est appelé \textbf{coefficient multiplicateur} associé au taux $t$. On peut le noter $CM$. Avec cette notation, on a alors  $V_f = CM \times V_i$ où $CM =1+t$. 
\end{DefT}

\EPC{1}{IC-2}{Calculer.}


\begin{minipage}{0.58\linewidth}
\begin{Th} 

Soit $CM$ le coefficient multiplicateur et $t$ le taux d'évolution, $t =CM-1$. 
\end{Th}
\end{minipage}
\hfill
\begin{minipage}{0.38\linewidth}

\begin{Rq} 
Pour $V_i \neq 0$, $CM=\frac{V_f}{V_i}$ . 
\end{Rq}
\end{minipage}


\begin{Th} 
\begin{description}
\item[•] Dans le cas d'une baisse, le taux d'évolution est négatif et le $CM$ est compris entre 0 et 1.
\item[•] Dans le cas d'une hausse, le taux d'évolution est positif et le $CM$ est supérieur à 1.
\end{description}

 
\end{Th}

%Exercices




\mini{
\EPC{1}{IC-27}{Calculer.}  

\EPC{1}{IC-28}{Raisonner.}

\EPC{1}{IC-29}{Modéliser. Calculer.}

\EPC{0}{IC-24}{Modéliser. Calculer.} 

}{
\EPC{1}{IC-25}{Modéliser. Calculer.} 

\EPC{0}{IC-26}{Raisonner. Calculer.} 

\EPC{1}{IC-30}{Raisonner. Calculer.}


\EPC{1}{IC-31}{Modéliser. Calculer.} 
}


 

\EPC{0}{IC-1}{Raisonner. Modéliser. Calculer.}