
Soit $f$ la fonction Racine carrée.

\begin{enumerate}
\item \textbf{Le produit}
\begin{enumerate}
\item Calculer $f(4)$, $f(9)$ et $f(36)$.
\item Comparer $\sqrt{4} \times \sqrt{9}$ et $\sqrt{4 \times 9}$.
\item Calculer $f(16)$, $f(25)$ et $f(400)$
\item Comparer $\sqrt{16} \times \sqrt{25}$ et $\sqrt{16 \times 25}$.
\item Quelle conjecture peut-on formuler concernant $\sqrt{a} \times \sqrt{b}$ et $\sqrt{a \times b}$, $a$ et $b$ deux réels positifs. 
\item Essayer de démontrer ce résultat. Attention à la logique et la rigueur de rédaction.
\end{enumerate}
\item \textbf{La somme}
\begin{enumerate}
\item Calculer $f(4)$, $f(9)$ et $f(13)$.
\item Comparer $\sqrt{4} + \sqrt{9}$ et $\sqrt{13}$.
\item Quelle conclusion peut-on formuler concernant $\sqrt{a} + \sqrt{b}$ et $\sqrt{a+b}$, $a$ et $b$ deux réels positifs. 
\end{enumerate}
\item \textbf{Application :}Écrire sous la forme $a\sqrt{b}$, $a$ et $b$ deux réels, $b \geq 0$ où $b$ est le plus petit possible.
\begin{enumerate}
\item $\sqrt{50}=\sqrt{\cdots \times \cdots}=\cdots \times\sqrt{ \cdots}$
\item $\sqrt{27}=\sqrt{\cdots \times \cdots}=\cdots \times\sqrt{ \cdots}$
\item $\sqrt{32}=\sqrt{\cdots \times \cdots}=\cdots \times\sqrt{ \cdots}$
\end{enumerate}
\end{enumerate}