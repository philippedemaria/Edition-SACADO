\begin{titre}[Géométrie vectorielle et analytique]

\Titre{Colinéarité de vecteurs}{4}
\end{titre}

\begin{CpsCol}
\textbf{Démontrer avec les vecteurs}
\begin{description}
\item[$\square$] Caractériser alignement et parallélisme par la colinéarité de vecteurs.
\end{description}
\end{CpsCol}



\begin{DefT}{Colinéarité de 2 vecteurs}\index{Vecteurs!Colinéarité}
Deux vecteurs $\overrightarrow{u}$ et $\overrightarrow{v}$ sont dits colinéaires lorsqu'ils ont la même direction, c'est à ire lorsqu'il existe un réel $k$ tel que $\overrightarrow{u}=k\overrightarrow{v}$.
\end{DefT}



\begin{Rq}
Le vecteur nul est colinéaire à tout autre vecteur du plan.
\end{Rq}


\begin{Th}
\begin{enumerate}
\item $A$, $B$, $C$ et $D$ sont 4 points distincts du plan, les droites $(AB)$ et $(CD)$ sont parallèles lorsque les vecteurs $\overrightarrow{AB}$ et $\overrightarrow{CD}$ sont colinéaires.
\item $A$, $B$ et $C$ sont 3 points distincts du plan  sont alignés lorsque les vecteurs $\overrightarrow{AB}$ et $\overrightarrow{AC}$ sont colinéaires.
\end{enumerate}
\end{Th}

\begin{ThT}{Condition de colinéarité}\index{Vecteurs!Condition de colinéarité}
Soit $\overrightarrow{u}$ de coordonnées $(x,y)$ et $\overrightarrow{v}$ de coordonnées $(x',y')$.

Les vecteurs  $\overrightarrow{u}$ et $\overrightarrow{v}$sont colinéaires si et seulement si $xy'-x'y=0$.

Le nombre réel $xy'-x'y$ est appelé \textbf{déterminant} de $\overrightarrow{u}$ et de $\overrightarrow{v}$. On note : $\text{dét}\left(\overrightarrow{u},\overrightarrow{v}\right)= xy'-x'y$.
\end{ThT}

\begin{Nt}
$\text{dét}\left(\overrightarrow{u},\overrightarrow{v}\right)=$ \begin{tabular}{|cc|}
$x$ & $x'$ \\  
$y$ & $y'$ \\ 
\end{tabular} $=xy'-x'y$
\end{Nt}

\Rec{1}{GVA-39}

\mini{
\EPC{1}{GVA-26}{Représenter. Calculer}

\EPC{0}{GVA-28}{Représenter. Calculer}

\EPC{1}{GVA-29}{Représenter. Calculer}

\EPC{0}{GVA-42}{Représenter. Calculer}
}{
\EPC{1}{GVA-27}{Représenter. Calculer}

\EPC{0}{GVA-30}{Représenter. Calculer}

\EPC{1}{GVA-43}{Représenter. Calculer}

\EPC{0}{GVA-51}{Représenter. Calculer}
}

\EPCP{1}{GVA-58}{Modéliser. Calculer}