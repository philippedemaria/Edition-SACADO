
La courbe de $f$ est représentée ci dessous.

\definecolor{ffqqqq}{rgb}{1.,0.,0.}
\definecolor{cqcqcq}{rgb}{0.7529411764705882,0.7529411764705882,0.7529411764705882}
\begin{tikzpicture}[line cap=round,line join=round,>=triangle 45,x=0.8849557522123893cm,y=0.8849557522123893cm]
\draw [color=cqcqcq,, xstep=0.8849557522123893cm,ystep=0.8849557522123893cm] (-3.44,-0.86) grid (5.6,3.24);
\draw[->,color=black] (-3.44,0.) -- (5.6,0.);
\foreach \x in {-3.,-2.,-1.,1.,2.,3.,4.,5.}
\draw[shift={(\x,0)},color=black] (0pt,2pt) -- (0pt,-2pt) node[below] {\footnotesize $\x$};
\draw[->,color=black] (0.,-0.86) -- (0.,3.24);
\foreach \y in {,1.,2.,3.}
\draw[shift={(0,\y)},color=black] (2pt,0pt) -- (-2pt,0pt) node[left] {\footnotesize $\y$};
\draw[color=black] (0pt,-10pt) node[right] {\footnotesize $0$};
\clip(-3.44,-0.86) rectangle (5.6,3.24);
\draw [line width=1.6pt,color=ffqqqq] (-2.86,2.08)-- (-2.78,2.1)-- (-2.7,2.1)-- (-2.64,2.14)-- (-2.58,2.16)-- (-2.5,2.2)-- (-2.44,2.26)-- (-3,2.)--(-2.38,2.3)-- (-2.32,2.36)-- (-2.26,2.4)-- (-2.2,2.42)-- (-2.14,2.44)-- (-2.08,2.48)-- (-2.02,2.52)-- (-1.96,2.54)-- (-1.88,2.56)-- (-1.82,2.6)-- (-1.74,2.62)-- (-1.68,2.64)-- (-1.62,2.68)-- (-1.56,2.7)-- (-1.5,2.74)-- (-1.44,2.76)-- (-1.38,2.78)-- (-1.3,2.8)-- (-1.22,2.82)-- (-1.16,2.84)-- (-1.08,2.84)-- (-0.98,2.84)-- (-0.9,2.84)-- (-0.82,2.84)-- (-0.74,2.82)-- (-0.66,2.8)-- (-0.6,2.78)-- (-0.52,2.74)-- (-0.46,2.72)-- (-0.4,2.7)-- (-0.34,2.66)-- (-0.28,2.62)-- (-0.22,2.58)-- (-0.16,2.54)-- (-0.08,2.48)-- (0.,2.42)-- (0.04,2.36)-- (0.1,2.32)-- (0.16,2.26)-- (0.22,2.22)-- (0.26,2.16)-- (0.32,2.12)-- (0.38,2.08)-- (0.44,2.02)-- (0.48,1.96)-- (0.54,1.9)-- (0.58,1.82)-- (0.64,1.76)-- (0.72,1.68)-- (0.8,1.62)-- (0.88,1.54)-- (0.94,1.5)-- (0.98,1.44)-- (1.04,1.42)-- (1.1,1.36)-- (1.16,1.34)-- (1.22,1.28)-- (1.3,1.24)-- (1.36,1.22)-- (1.42,1.2)-- (1.48,1.16)-- (1.56,1.16)-- (1.64,1.16)-- (1.72,1.16)-- (1.78,1.2)-- (1.84,1.22)-- (1.9,1.24)-- (1.96,1.28)-- (2.02,1.3)-- (2.08,1.34)-- (2.1,1.4)-- (2.16,1.42)-- (2.22,1.46)-- (2.28,1.5)-- (2.34,1.54)-- (2.4,1.58)-- (2.48,1.62)-- (2.54,1.66)-- (2.62,1.66)-- (2.68,1.68)-- (2.76,1.68)-- (2.84,1.68)-- (2.92,1.68)-- (3.,1.68)-- (3.08,1.68)-- (3.14,1.66)-- (3.24,1.66)-- (3.28,1.6)-- (3.34,1.54)-- (3.4,1.52)-- (3.48,1.48)-- (3.54,1.42)-- (3.6,1.38)-- (3.66,1.34)-- (3.72,1.28)-- (3.78,1.26)-- (3.82,1.2)-- (3.88,1.18)-- (3.96,1.12)-- (4.04,1.06)-- (4.08,1.)-- (4.14,0.98)-- (4.22,0.92)-- (4.28,0.86)-- (4.34,0.8)-- (4.4,0.74)-- (4.46,0.7)-- (4.5,0.64)-- (4.56,0.6)-- (4.62,0.54)-- (4.68,0.48)-- (4.74,0.4)-- (4.8,0.36)-- (4.86,0.3)-- (4.92,0.24)-- (4.98,0.16)-- (5.,0.1);
\end{tikzpicture}


\begin{enumerate}
\item Dresser le tableau de variations de $f$.
\item Comparer $f(-1,5)$ et $f(0,5)$.
\item Comparer $f(3,6)$ et $f(4)$.
\item Comparer $f(-2,5)$ et $f(-2,9)$.  
\item Quel est le maximum de $f$ sur $[-3;5]$ ?
\end{enumerate}