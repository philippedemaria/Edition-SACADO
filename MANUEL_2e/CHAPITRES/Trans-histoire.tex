
\begin{His}

{\color{brown}{\Large Histoire de l'art. M.C. \textsc{Escher}}}

\vspace{0.4cm}

\begin{wrapfigure}[11]{l}{5cm}
\vspace{-7mm}
\includegraphics[scale=0.4]{image_chapitres/escher2.jpg} 
\end{wrapfigure}
« \textit{Quiconque  veut  représenter  quelque  chose  d'irréel  doit  se  conformer à  certaines  règles.  Ces  règles  sont,  à  peu de choses près,  celles que doit respecter  tout  narrateur  de contes de  fées  : créer le contraste, provoquer l'étonnement. (...). Seuls  ceux  qui  sont  prêts  à  aller  au-delà  des  apparences  peuvent  jouer  et  comprendre  un  tel  jeu  - ceux  qui acceptent de se servir de leur intelligence, comme ils le font pour résoudre une énigme. Ce n'est donc pas affaire de sens, mais de cerveau.  Nul besoin  d'être profond, il suffit  d'avoir le  sens de l'humour et de  savoir  se moquer de soi, du moins dans le cas de celui qui fait les représentations.}» 

\vspace{0.4cm}
\begin{wrapfigure}[4]{r}{3.6cm}
\vspace{-7mm}
\includegraphics[scale=0.5]{image_chapitres/escher.jpg} 
%\unnumberedcaption{M.C. \textsc{Escher}}
\end{wrapfigure}

Enfin, un peu avant sa mort, \textsc{Escher} a écrit : « \textit{Un de mes plus grands plaisirs est la fréquentation et l'amitié des mathématiciens, qui a résulté de mon travail. Ils m'ont souvent donné des idées nouvelles et parfois même je leur ai rendu la pareille. Que ces hommes et femmes si savants sont joueurs} ! » 



\textbf{Maurits  Cornelis  \textsc{Escher}} (1898 - 1972)  est  un  artiste  néerlandais,  connu  pour  ses  gravures  sur  bois  et lithographies. Après le lycée, période durant laquelle il a déjà été initié à la gravure, il commence des cours d’architecture à l’Ecole d’Architecture  et  d’Arts  décoratifs de Haarlem où rapidement  l’un  de  ses professeurs, J. de Mesquita (graveur lui-même) lui conseille d’abandonner l’architecture au profit des cours de dessin et d’arts graphiques. 

%\textbf{1ère période :} 
%
% Après sa formation (1919-1922), \textsc{Escher} se fixe  à Rome de 1923  à 1935 et entreprend chaque année des voyages essentiellement en Italie durant lesquels il dessine tout ce qui l'intéresse et choisit les meilleurs dessins pour en faire des estampes. En juillet 1935 il quitte l’Italie du fait de la montée du fascisme. 
%
%\textbf{2ème période :  }
%
%En  1936,  il  fait  un  voyage  en  Espagne pour  découvrir  les  motifs  décoratifs  de  l'un  des monuments  majeurs  de l'architecture  islamique  méditerranéenne :  l’Alhambra  à  Grenade. Durant  ce  voyage,  il  copie  les  plus  belles  mosaïques  de l’Alhambra  et  de  La  Mezquita (à Cordoue) et observe les perspectives des colonnades.  Après un court séjour en Suisse et en Belgique, il se fixe aux Pays Bas. Il  se  détourne  de  plus  en  plus  de  la  représentation  de  la  réalité  pour  se  consacrer  à  ses constructions originales d’images. La  cristallographie  qu’il  découvre  grâce à  son  demi-frère  géologue  et  sa  rencontre  avec  le mathématicien Roger Penrose furent décisives dans ses rapports aux arts graphiques.
% 
%\vspace{0.2cm}
%{\color{brown}{\large Son œuvre}}
%
%\vspace{0.2cm}
%
%Escher a produit plus de 150 dessins en couleurs, dans lesquels s'imbriquaient des créatures qui rampaient, nageaient ou planaient, emplissant tout le plan.  On distingue essentiellement deux périodes dans son œuvre.
%
%\textbf{1ère période : 1922 – 1937 } 
%
%La  représentation  de  la  réalité  (paysages,  architecture  de  villes  et  villages)  occupe  une  place prépondérante  pendant  toute  cette  période  et  Escher  apporte  déjà  une  attention  particulière  à la structure de l’espace où il utilise simultanément plusieurs points  de  vues  souvent  opposés (haut/bas ; près/lointain ...). Il  travaille  la  technique  de  gravure :  hachures/  contrastes  entre  noir  et  blanc /  utilisation  de plusieurs bois. 
%
%Il  dira  qu’avant  37,  il  a  tendance  à  accorder  la  plus  grande  importance  à  ses expériences graphiques ; il considère la plupart de ses œuvres comme des exercices pratiques. 
%
%\textbf{2ème période : à partir de 1937  }
%
%Escher s’éloigne de la réalité.Les mosaïques observées en Espagne ont fasciné Escher. Ces copies et observations (répétitions régulières  de  figures  géométriques  qui  permettent  de  paver  le  plan,  perspectives,  arabesques, colonnades...) lui servent dans ses œuvres de pavages et de constructions  impossibles » . 
%C’est à  partir  des  mosaïques  et  de  ses  lectures  sur  la  cristallographie, qu’\textsc{Escher}  entreprend  la construction de ses propres formes de remplissage du plan. Des cristallographes comparent ses travaux au phénomène de cristallisation. Après   44,   Escher   se  consacre   davantage   à   ses   constructions   spatiales :   relativité,   cage d’escalier, mouvement perpétuel, belvédère ; montée descente.

\PESP{http://pedagogie2.ac-reunion.fr/col-j.solesse/HDA\%20FicheESCHERmathemathiques.pdf}

\end{His}

