\begin{titreTice}[Problème du premier degré]

\Titre{Déterminer une fonction affine}{1}
\end{titreTice}


\begin{CpsCol}
\begin{description}
\item[$\square$] Résoudre un système, éventuellement avec sa calculatrice
\end{description}
\end{CpsCol}


\Rec{1}{FPD-18}

\begin{Mt}
Toute équation affine est de la forme $f(x)=ax+b, a\neq0$. ICi, $x$ et $f(x)$ sont connus, on cherche $a$ et $b$.

On se retrouve donc à résoudre un système de deux inconnues à 2 équations. Les deux équations déduites sont : 
\begin{enumerate}
\item $f(4)=2 \Longleftrightarrow 4a+b=2$
\item $f(3)=-1 \Longleftrightarrow 3a+b=-1$
\end{enumerate}
On utilise alors la fonction système de la calculatrice.
\end{Mt}

\subsection*{A la main}

$\left\lbrace \begin{tabular}{c}
$f(4)=2$ \\ 
$f(3)=-1$ \\
\end{tabular} \right. \Longleftrightarrow \left\lbrace \begin{tabular}{c}
$4a+b=2$ \\ 
$3a+b=-1$ \\
\end{tabular} \right.$

En opérant une soustraction des 2 lignes, on obtient 

$\Longleftrightarrow \left\lbrace \begin{tabular}{c}
$4a+b=2$ \\ 
$a=3$ \\
\end{tabular} \right.$

Par substitution,

$\Longleftrightarrow \left\lbrace \begin{tabular}{c}
$4\times 3 +b=2$ \\ 
a=3 \\
\end{tabular} \right. \Longleftrightarrow \left\lbrace \begin{tabular}{c}
$b=2-12$ \\ 
$a=3$ \\
\end{tabular} \right. \Longleftrightarrow \left\lbrace \begin{tabular}{c}
$b=-10$ \\ 
$a=3$ \\
\end{tabular} \right.$

Donc \fbox{$f(x)=3x-10$}



\subsection*{Avec Casio}

\begin{enumerate}
\item Aller dans le menu EQUA
\item Sélectionner F1 : Simultaneous
\item Écrire le nombre d'incinnues : ici 2.
\item On remplace les "0" par les nombres connus. La première colonnes est complétée par les coefficients de $a$, la deuxième par les coefficient de $b$ et la troisième par les constantes.

\begin{tabular}{ccc}
4 & 1 & 2 \\ 
3 & 1 & $-1$ \\ 
\end{tabular} 

\item Appuyer sur \touche{EXE}
\end{enumerate}

\newpage

\subsection*{Avec TI}

\begin{enumerate}
\item Cliquer sur \touche{Shift} \touche{MATRIX} et sélectionner la matrice \texttt{A}
\item Sélectionner \texttt{EDIT} avec les touches de direction

\item Remplacer $ 1 \times 1$ par $2 \times 2$. (2 lignes $\times$ 2 colonnes).
\item Écrire les coefficients de $a$ et de $b$. Voici la matrice \texttt{A}.

\begin{tabular}{cc}
 4 & 1 \\ 
3 & 1 \\ 
\end{tabular} 

\item Cliquer sur \touche{Shift} \touche{MATRIX} et sélectionner la \texttt{B} avec les touches de direction
\item Sélectionner \texttt{EDIT} avec les touches de direction

\item Remplacer $ 1 \times 1$ par $2 \times 1$. (2 lignes $\times$ 1 colonne).
\item Écrire les coefficients de $a$ et de $b$. Voici la matrice \texttt{B}.

\begin{tabular}{c}
2  \\ 
$-1$ \\ 
\end{tabular} 

\item \textbf{A ce stade là, les 2 matrices \texttt{A} et \texttt{B} sont renseignées.}

\item Appuyer sur  \touche{Shift} \touche{QUIT}

\item Cliquer sur \touche{Shift} \touche{MATRIX} \touche{EXE} \touche{$x^{-1}$} \touche{$\times$} \touche{Shift} \touche{MATRIX} \texttt{B} , sur l'écran, il doit s'afficher : [A]*[B]$^{-1}$.

\item Appuyer sur \touche{EXE}

\end{enumerate}




