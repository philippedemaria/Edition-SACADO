\ \\
\vspace{0.5cm}
\ \\

\begin{flushright}
{\huge {\color{bleu3} {\sffamily Codification des exercices}}}

{\color{bleu3}\rule{0.6\linewidth}{0.4pt}}
\end{flushright}

\vspace{2cm}

\begin{description}
\item \tikz\node[rounded corners=0pt,draw,fill=bleu2]{\color{white}\textbf{ $n$}}; \quad  {\color{bleu2}\textbf{ Application directe}}

Les applications directes portent uniquement sur les capacités et compétences annoncés en entête de la notion. Ils permettent de comprendre et de mettre en œuvre les acquisitions de base.


\item \tikz\node[rounded corners=0pt,draw,fill=bleu2]{\color{white}\textbf{ $n$}}; \quad  {\color{bleu2}\textbf{ Exercice d'application}}

Les exercices d'applications portent principalement sur les capacités et compétences annoncées en entête de la notion mais mettent en jeu des savoir faire et des notions antérieures. Ils renforcent leur compréhension.

\item \tikz\node[rounded corners=0pt,draw,fill=bleu2]{\color{white}\textbf{$n$}}; \quad  {\color{bleu2}\textbf{ Exercice de découverte}}

Les activités de découverte permettent la découverte d'une notion encore inconnue et sont axées sur des situations problèmes, situations où a priori les outils de résolution ne sont pas encore déterminés, ni connus.

\item \tikz\node[rounded corners=0pt,draw,fill=bleu3]{\color{white}\textbf{ $n$}}; \quad  {\color{bleu3}\textbf{ Situation de recherche}}

Les activités de recherche amènent une notion encore inconnue et sont axées sur des situation problèmes, situation où à priori les outils de résolution ne sont pas encore déterminés, ni connus. Ces activités ont pour objectif la construction de savoir, savoir faire.

\item \tikz\node[rounded corners=0pt,draw,fill=bleu3]{\color{white}\textbf{ $n$}}; \quad  {\color{bleu3}\textbf{ Défi}}

Les défis ou problèmes ouverts sont des activités où le mode de résolution et opératoire n'est pas défini. C'est à l'élève de prendre toutes les initiatives pour prendre le chemin de la solution. La résolution n'est pas nécessairement attendue. La recherche est primordiale. La prise d'initiative est valorisée.

\item \tikz\node[rounded corners=0pt,draw,fill=black]{\color{white}\textbf{ $n$}}; \quad  {\color{black}\textbf{ Approfondissement}}

Ces activités sont plus difficiles et demandent la synthèse de nombreuses compétences.

\item \tikz\node[rounded corners=0pt,draw,fill=orange]{\color{white}\textbf{ $n$}}; \quad  {\color{orange}\textbf{  Activité Scratch}}

Activités dont le support est le logiciel Scratch

\item \tikz\node[rounded corners=0pt,draw,fill=yellow]{\color{blue}\textbf{ $n$}}; \quad  {\color{blue}\textbf{  Activité Python}}

Activités dont le support est le logiciel Python


\item \tikz\node[rounded corners=0pt,draw,fill=red]{\color{white}\textbf{ $n$}}; \quad  {\color{red}\textbf{ Compte rendu}}

Outre les compétences purement disciplinaires, ces activités mettent en pratique les compétences suivantes :
\begin{description}[leftmargin=*]
\item[•] Faire le lien entre le langage naturel et le langage algébrique. Distinguer des spécificités du langage mathématique par rapport à la langue française.
\item[•] Expliquer à l'oral ou à l'écrit (sa démarche, son raisonnement, un calcul, un protocole de construction géométrique, un algorithme), comprendre les explications d'un autre et argumenter dans l'échange.
\end{description}

\item \tikz\node[rounded corners=0pt,draw,fill=vert]{\color{white}\textbf{ $n$}}; \quad  {\color{vert}\textbf{ Parcours}}

Ces activités sont en lien avec un des 3 parcours éducatifs du cycle :
\begin{description}
\item[•] Enseignement moral et civique.
\item[•] Avenir
\item[•] Éducation artistique et culturelle
\end{description}

\item \tikz\node[rounded corners=0pt,draw,fill=violet]{\color{white}\textbf{ $n$}}; \quad {\color{violet}\textbf{ En ligne}}

En ligne propose un lien extérieur vers un support numérique :Géogébra, Youtube, Python, Scratch, liens externes.


\item \tikz\node[rounded corners=2pt,draw,fill=red]{\color{white}\textbf{ $n$}}; \quad {\color{red}\textbf{ Question Flash}}

Ces questions sont à faire rapidement, certaines à l'oral ou en activité mentale. Elles peuvent amener des automatismes de résolution.

\end{description}



