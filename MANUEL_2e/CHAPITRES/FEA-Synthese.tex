 \begin{titreTice}[Synthèse]

\Titre{Nombres et calculs}{2}
\end{titreTice}


\EPCN{Calculer.}

Soient $x$ et $y$ deux réels tels que $x=\frac{y-3}{8-2y}$.

\begin{enumerate}
\item Pour quelles valeurs de $y$ cette égalité est-elle définie ?
\item Exprimer $y$ en fonction de $x$. Préciser les valeurs de $x$ pour lesquelles cette égalité $y=f(x)$ est définie.
\end{enumerate}


\EPCN{Représenter. Calculer.}

Dans un repère du plan, $d$ et $d'$ sont deux droites d'équations respectives $2x-3y+5=0$ et $5y+4x-1=0$. Déterminer les coordonnées du point d'intersection de $d$ et de $d'$. 


\EPCN{Représenter. Calculer.}

$x$ et $y$ sont deux réels non nuls. Comparer les nombres $A = \frac{-2xy}{x-y}$ et $B=\frac{x-y}{2}$.


\EPCN{Représenter. Calculer.}

\begin{enumerate}
\item Donner l'ensemble des diviseurs positifs de 36.
\item En déduire les solutions de $x(x+5)=36$.
\end{enumerate}

\EPCN{Représenter. Calculer.}

Soit $N$ un entier naturel, impair et non premier. On suppose que $N=a^2-b^2$ où $a$ et $b$ sont deux entiers naturels.

\begin{enumerate}
\item Montrer que $a$ et $b$ n'ont pas la même parité.
\item Montrer que $N$ peut s'écrire comme produit de deux entiers naturels $p$ et $q$.
\item Quelle est la parité de $p$ et $q$ ? 
\end{enumerate}


\EPCN{Représenter. Calculer.}

On considère un entier naturel $n$ non nul. 

\begin{enumerate}
\item On suppose $S_n = 1+2+3+...+(n-2)+(n-1)+n$ 
\begin{enumerate}
\item En remarquant que $S_n = n + (n-1)+ (n-2)+ ...+3+2+1$, déterminer une expression de $2S_n$ en fonction de $n$.
\item En déduire une expression de $S_n$ en fonction de $n$.
\item Calculer $1+2+3+...+1031$.
\end{enumerate}
 \item En déduire des questions précédentes que $n(n+1)$ est un nombre pair pour tout entier entier naturel $n$ non nul.
\end{enumerate}


