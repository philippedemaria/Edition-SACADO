
Dans un paquet pris au hasard, on souhaite déterminer la probabilité de tirer successivement un
bonbon bleu puis un bonbon rouge.

\begin{enumerate}
\item Modélisation du paquet de M\&M's.
\begin{enumerate}
\item  Lancer le programme ci dessous qui modélise un paquet de 10 bonbons M\&M's. 

\begin{center}
\begin{lstlisting}
import random
couleur=["jaune","bleu","rouge","orange","vert","marron"]
proba_couleur=[0]*6
for i in range (10) :
    alea = random.randint(0,5)
    proba_couleur[alea]=proba_couleur[alea]+1
    
for j in range (6) :
    print("On a alors p(",couleur[j],") = ",proba_couleur[j]/n)
\end{lstlisting}
\end{center}

Quel est le rôle de l'instruction \texttt{for j in range (6)} ?

\item Quel est le rôle de la variable couleur ? Quel est le rôle de la variable proba\_couleur ?
\end{enumerate}
\item Compléter alors le tableau suivant

\begin{tabular}{|c|c|c|c|c|c|c|c|}
\hline 
Couleur& Jaune& Bleu& Rouge &Orange& Vert& Marron&Somme\\ 
\hline 
Fréquence&&&&&&&\\ 
\hline 
Effectif&&&&&&&\\ 
\hline 
\end{tabular} 

\item On assimile les probabilités aux fréquences observées.
On désigne par
\begin{description}
\item[B] "l'événement le bonbon est bleu".
\item[R] "l'événement le bonbon est rouge".
\item[A] "l'événement le bonbon est orange, vert, jaune ou marron".
\end{description}
\item A l'aide des résultats du tableau, construire une arbre de probabilité.
\item Répondre à la question en préambule pour le paquet généré.
\end{enumerate}