\begin{titre}[Probabilités]

\Titre{Probabilité, loi de probabilité}{4}
\end{titre}


\begin{CpsCol}
\begin{description}
\item[$\square$] Utiliser des modèles théoriques de référence (dé, pièce équilibrée, tirage au sort avec
équiprobabilité dans une population) en comprenant que les probabilités sont définies
a priori.
\item[$\square$] Construire un modèle à partir de fréquences observées, en distinguant nettement
modèle et réalité.
\item[$\square$] Calculer des probabilités dans des cas simples : expérience aléatoire à deux ou trois
épreuves.
\end{description}
\end{CpsCol}

\Rec{1}{Prob-20}

\begin{DefT}{Épreuve ou expérience aléatoire}\index{Épreuve}\index{Expérience aléatoire}
On appelle \textbf{épreuve} ou \textbf{expérience aléatoire}, une situation soumise au hasard.

On appelle \textbf{issue} le résultat d'une épreuve, notée $\omega$. 

L'ensemble des issues est l'\textbf{univers}, noté $\Omega$.\index{Univers} \index{Issue}
\end{DefT}

\begin{Ex}
On lance un dé cubique dont les faces sont numérotées de 1 à 6 et on observe le numéro de la face obtenue. Cette expérience a 6 issues possibles dont une issue est le 6. L'univers est $\Omega = \lbrace 1;2;3;4;5;6 \rbrace$.
\end{Ex}



\mini{
\EPC{1}{Prob-6}{Modéliser. Représenter.}
}{
\EPC{0}{Prob-5}{Modéliser. Représenter.}
}


\begin{DefT}{Loi de probabilités}\index{Loi de probabilités}
Définir la loi de probabilité pour une expérience aléatoire dont l'univers est $\Omega = \lbrace x_1; x_2; ... ; x_n \rbrace$ consiste à déterminer pour chacune des issues $x_i$ un nombre $p_i$, $i$ variant de 1 à $n$, appelée \textbf{probabilité}, tel que $p_1+p_2+...p_n=1$
\end{DefT}

\begin{Ex}
On lance un dé cubique équilibré dont les faces sont numérotées de 1 à 6 et on observe le numéro de la face obtenue.   Donc $p_1+p_2+p_3+p_4+p_5+p_6=1$. Comme le dé est équilibré, $p_i=\frac{1}{6}$.
\end{Ex}


\begin{Pp}
En répétant un grand nombre de fois une expérience aléatoire, la fréquence de chaque issue se stabilise autour d'une valeur. Cette valeur est la probabilité de l'issue.
\end{Pp}



\mini{
\EPC{1}{Prob-31}{Communiquer.}
 
\EPC{1}{Prob-26}{Communiquer.}

\EPC{1}{Prob-28}{Modéliser.}

\EPC{1}{Prob-27}{Communiquer.}
}{
\EPC{1}{Prob-32}{Modéliser. }

\EPC{1}{Prob-8}{Chercher.}

\EPC{0}{Prob-18}{Chercher.}

\EPC{0}{Prob-29}{Modéliser.}
}





\EPCP{1}{Prob-25}{Modéliser. Représenter.}

%\EPCP{1}{Prob-0}{Modéliser. Représenter.}

 