
\begin{minipage}{0.48\linewidth}
Un verre à pied peut être assimilé à un cone de révolution
dont la base est un disque de diamètre AB = 4cm et la
hauteur CH = 6cm.

On verse une boisson dans ce verre. La figure ci-contre
schématise cette situation On admettra que les droites (AB)
et (PQ) sont parallèles.

Étudier les interprétations de la l’expression : "Le verre
est à moitié plein".

On demande de rédiger la démarche comme un article à
publier.
\end{minipage}
\hfill
\begin{minipage}{0.48\linewidth}

\definecolor{ffdxqq}{rgb}{1.,0.8431372549019608,0.}
\definecolor{sqsqsq}{rgb}{0.12549019607843137,0.12549019607843137,0.12549019607843137}
\definecolor{zzttqq}{rgb}{0.6,0.2,0.}
\definecolor{bfffqq}{rgb}{0.7490196078431373,1.,0.}
\begin{tikzpicture}[line cap=round,line join=round,>=triangle 45,x=1.0cm,y=1.0cm]
\clip(-5.404460392002123,0.5559592701862407) rectangle (-0.25956832914997396,6.3612404634940365);
\fill[color=zzttqq,fill=zzttqq,fill opacity=0.1] (-4.491281276919434,4.982562553838868) -- (-1.502956014330115,4.99408797133977) -- (-3.,2.) -- cycle;
\draw [line width=2.4pt,color=bfffqq] (-5.,6.)-- (-3.,2.);
\draw [line width=2.4pt,color=bfffqq] (-3.,2.)-- (-1.,6.);
\draw [line width=2.4pt,color=bfffqq] (-5.,6.)-- (-1.,6.);
\draw [line width=2.4pt,color=bfffqq] (-5.,1.)-- (-1.,1.);
\draw [line width=2.4pt,color=bfffqq] (-3.,2.)-- (-3.,1.);
\draw [color=sqsqsq] (-1.,5.)-- (-1.,2.);
\draw [color=black](-0.7970943655673626,3.7350418284262243) node[anchor=north west] {$x$};
\begin{scriptsize}
\draw [fill=sqsqsq,shift={(-1.,5.)}] (0,0) ++(0 pt,3.75pt) -- ++(3.2475952641916446pt,-5.625pt)--++(-6.495190528383289pt,0 pt) -- ++(3.2475952641916446pt,5.625pt);
\draw [fill=sqsqsq,shift={(-1.,2.)},rotate=180] (0,0) ++(0 pt,3.75pt) -- ++(3.2475952641916446pt,-5.625pt)--++(-6.495190528383289pt,0 pt) -- ++(3.2475952641916446pt,5.625pt);
\end{scriptsize}
\end{tikzpicture}

\end{minipage}