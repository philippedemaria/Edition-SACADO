
Sur la représentation graphique de $g$ telle que $g(x)=\frac{1}{x}$, on a placé les points $A$ et $B$ d'abscisses respectives $2$ et $\frac{1}{2}$.

\begin{center}
\definecolor{ffqqqq}{rgb}{1.,0.,0.}
\definecolor{qqwuqq}{rgb}{0.,0.39215686274509803,0.}
\definecolor{cqcqcq}{rgb}{0.7529411764705882,0.7529411764705882,0.7529411764705882}
\begin{tikzpicture}[line cap=round,line join=round,>=triangle 45,x=1.0cm,y=1.0cm]
\draw [color=cqcqcq,, xstep=0.5cm,ystep=0.5cm] (-0.863846865692171,-0.41690888822012473) grid (3.1454552355878582,2.599705433175666);
\draw[->,color=black] (-0.863846865692171,0.) -- (3.1454552355878582,0.);
\foreach \x in {-0.5,0.5,1.,1.5,2.,2.5,3.}
\draw[shift={(\x,0)},color=black] (0pt,2pt) -- (0pt,-2pt) node[below] {\footnotesize $\x$};
\draw[->,color=black] (0.,-0.41690888822012473) -- (0.,2.599705433175666);
\foreach \y in {,0.5,1.,1.5,2.,2.5}
\draw[shift={(0,\y)},color=black] (2pt,0pt) -- (-2pt,0pt) node[left] {\footnotesize $\y$};
\draw[color=black] (0pt,-10pt) node[right] {\footnotesize $0$};
\clip(-0.863846865692171,-0.41690888822012473) rectangle (3.1454552355878582,2.599705433175666);
\draw[line width=1.2pt,color=qqwuqq,smooth,samples=100,domain=-0.863846865692171:3.1454552355878582] plot(\x,{1.0/(\x)});
\draw [color=ffqqqq,domain=-0.863846865692171:3.1454552355878582] plot(\x,{(--3.75-1.5*\x)/1.5});
\begin{scriptsize}
\draw[color=qqwuqq] (-5.14300583917374,-0.25306721581204666) node {$f$};
\draw [color=ffqqqq] (2.,0.5)-- ++(-2.5pt,0 pt) -- ++(5.0pt,0 pt) ++(-2.5pt,-2.5pt) -- ++(0 pt,5.0pt);
\draw[color=ffqqqq] (2.066027746781696,0.6721563460218062) node {$A$};
\draw [color=ffqqqq] (0.5,2.)-- ++(-2.5pt,0 pt) -- ++(5.0pt,0 pt) ++(-2.5pt,-2.5pt) -- ++(0 pt,5.0pt);
\draw[color=ffqqqq] (0.5721772042374548,2.175644634001817) node {$B$};
\end{scriptsize}
\end{tikzpicture}
\end{center}

\begin{enumerate}
\item Déterminer la fonction $f$ affine représentée par la droite $(AB)$.
\item La droite $(AB)$ coupent les axes en $M$ et $N$. Montrer que les segments $[AB]$ et $[MN]$ ont même milieu.
\item Soit $P$ et $Q$ deux points quelconques non confondus de l'hyperbole. La droite $(AB)$ coupent les axes en $M$ et $N$. Démontrer que les segments $[PQ]$ et $[MN]$ ont même milieu.
\end{enumerate}