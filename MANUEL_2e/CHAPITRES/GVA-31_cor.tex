
$ABC$ est un triangle. $A’$ est le symétrique de $A$ par rapport à $C$ et $B’$ est tel que  $\overrightarrow{BA}=\overrightarrow{A'B'}$. Montrer que  $\overrightarrow{BC}=\overrightarrow{CB'}$ . 

\definecolor{zzttqq}{rgb}{0.6,0.2,0.}
\definecolor{ududff}{rgb}{0.30196078431372547,0.30196078431372547,1.}
\begin{tikzpicture}[line cap=round,line join=round,>=triangle 45,x=1.0cm,y=1.0cm]
\clip(-1.48,-0.16) rectangle (6.2,5.08);
\draw [line width=2.pt,color=zzttqq] (-0.84,1.26)-- (3.34,0.52);
\draw [line width=2.pt,color=zzttqq] (3.34,0.52)-- (2.28,2.36);
\draw [line width=2.pt,color=zzttqq] (2.28,2.36)-- (-0.84,1.26);
\draw [->,line width=2.pt] (3.34,0.52) -- (-0.84,1.26);
\draw [->,line width=2.pt] (5.4,3.46) -- (1.22,4.2);
\begin{scriptsize}
\draw [fill=ududff] (-0.84,1.26) circle (2.5pt);
\draw[color=ududff] (-1.14,1.63) node {$A$};
\draw [fill=ududff] (3.34,0.52) circle (2.5pt);
\draw[color=ududff] (3.48,0.89) node {$B$};
\draw [fill=ududff] (2.28,2.36) circle (2.5pt);
\draw[color=ududff] (2.42,2.73) node {$C$};
\draw [fill=ududff] (5.4,3.46) circle (2.5pt);
\draw[color=ududff] (5.6,3.83) node {$A'$};
\draw [fill=ududff] (1.22,4.2) circle (2.5pt);
\draw[color=ududff] (1.42,4.57) node {$B'$};
\end{scriptsize}
\end{tikzpicture}

$A’$ est le symétrique de $A$ par rapport à $C$ donc $\overrightarrow{AC}=\overrightarrow{CA'}$

De plus, d'après l'énoncé, $\overrightarrow{BA}=\overrightarrow{A'B'}$.

D'après la relation de Chasles, $\overrightarrow{BC} = \overrightarrow{BA}+\overrightarrow{AC} =\overrightarrow{A'B'}+\overrightarrow{CA'} = \overrightarrow{CA'}+\overrightarrow{A'B'}=\overrightarrow{CB'}$

Remarque : $C$ est le milieu de $[BB']$

