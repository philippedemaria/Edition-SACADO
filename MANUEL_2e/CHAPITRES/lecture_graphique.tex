\documentclass[10pt]{article}

\input{../../../latex_preambule_style/preambule}
\input{../../../latex_preambule_style/styleCourslycee}
\input{../../../latex_preambule_style/styleExercices}
%\input{../../latex_preambule_style/styleCahier}
\input{../../../latex_preambule_style/bas_de_page_seconde}
\input{../../../latex_preambule_style/algobox}



%%%%%%%%%%%%%%%  Affichage ou impression  %%%%%%%%%%%%%%%%%%
\newcommand{\impress}[2]{
\ifthenelse{\equal{#1}{1}}  %   1 imprime / affiche  -----    0 n'affiche pas
{%condition vraie
#2
}% fin condition vraie
{%condition fausse
}% fin condition fausse
} % fin de la procédure
%%%%%%%%%%%%%%%  Affichage ou impression  %%%%%%%%%%%%%%%%%%



%%%%%%%%%%%%%%%  Indentation  %%%%%%%%%%%%%%%%%%
\parindent=0pt
%%%%%%%%%%%%%%%%%%%%%%%%%%%%%%%%%%%%%%%%%%%%%%%%
\begin{document} 


En utilisant le lien : \url{https://www.geogebra.org/m/h8djagst}, répondre aux questions suivantes en cochant, décochant les bonnes cases  et renseigner la bonne valeur. 

Il est possible de zoomer ou déplacer l'origine du repère.
 
\subsection*{Lectures graphiques} 

\begin{enumerate}
\item Équations
\begin{enumerate}
\item Conjecturer la solution de l'équation $x^3=10$. \ligne{1}
\item Conjecturer la solution de l'équation $\frac{1}{x}=-5$. \ligne{1} 
\item Conjecturer la solution de l'équation $x^3=-30$. \ligne{1}  
\item Conjecturer la solution de l'équation $\sqrt{x}=12$. \ligne{1}  
\item Conjecturer la solution de l'équation $\sqrt{x}=25$. \ligne{1} 
\item Conjecturer la solution de l'équation $x^2=6$. \ligne{1}
\item Conjecturer la solution de l'équation $\frac{1}{x}=2$. \ligne{1} 
\end{enumerate}
\item Inéquations
\begin{enumerate}
\item Conjecturer la solution de l'inéquation $x^3>10$. \ligne{1} 
\item Conjecturer la solution de l'inéquation $\frac{1}{x} \geq -5$. \ligne{1} 
\item Conjecturer la solution de l'inéquation $x^3 \leq -30$. \ligne{1} 
\item Conjecturer la solution de l'inéquation $\sqrt{x}>12$. \ligne{1}  
\item Conjecturer la solution de l'inéquation $\sqrt{x}\leq 5$. \ligne{1} 
\item Conjecturer la solution de l'inéquation $x^2<6$. \ligne{1} 
\item Conjecturer la solution de l'inéquation $\frac{1}{x}<2$.  \ligne{1} 
\end{enumerate} 
\end{enumerate}
 
\subsection*{Calculs algébriques} 

\begin{enumerate}
\item Soit $x$ et $a$ deux réels, démontrer que $x^3-a=(x-a)(x^2+ax+a^2)$
\item Résoudre les inéquations sur l'ensemble de définition à expliciter. 
\begin{enumerate}
\item $x^3>8$. On démontrer que $x^3-8=(x-2)(x^2+2x+4)$
\item $\frac{1}{x} \geq -5$. Il faudra penser à un tableau de signe ou une disjonction de cas.
\item $x^3 \leq -30$.  
\item $\sqrt{x}\leq 5$.
\item $\frac{1}{x}<2$. Il faudra penser à un tableau de signe ou une disjonction de cas.
\end{enumerate} 
\end{enumerate}




\end{document}
