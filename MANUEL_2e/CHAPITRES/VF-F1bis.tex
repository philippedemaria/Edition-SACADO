\begin{seance}[Étude qualitatives de fonctions]

\Titre{Variations de fonctions}{5}
\end{seance}


\begin{CpsCol}
\textbf{Variations de fonctions}
\begin{description}
\item[$\square$] Déterminer tous les nombres dont l'image est supérieure (inférieure) à une image donnée
\item[$\square$] Comparer deux images sur un intervalle donné
\end{description}
\end{CpsCol}


\begin{multicols}{2}
\Rec{1}{VF-4}

\Rec{1}{VF-14}
\end{multicols}

\begin{DefT}{Fonction croissante, décroissante sur $I$}\index{Fonction croissante sur} \index{Fonction décroissante sur}
On dit que
\begin{description}[leftmargin=*]
\item[•] une fonction $f$ est \textbf{croissante sur $I$}, lorsque pour tout nombre $a$ et $b$ de $I$ tels que $a \leq b$, $f(a) \leq f(b)$.
\item[•] une fonction $f$ est \textbf{strictement croissante sur $I$}, lorsque pour tout nombre $a$ et $b$ de $I$ tels que $a \leq b$, $f(a) < f(b)$.
\item[•] une fonction $f$ est décroissante sur $I$, lorsque pour tout nombre $a$ et $b$ de $I$ tels que $a \leq b$, $f(a) \geq f(b)$.
\item[•] une fonction $f$ est décroissante sur $I$, lorsque pour tout nombre $a$ et $b$ de $I$ tels que $a \leq b$, $f(a) > f(b)$.
\end{description} 
\end{DefT}


\mini{
\AD{1}{VF-12}
}{
\AD{1}{VF-13}
}

\begin{DefT}{Maximum, minimum sur $I$}\index{Minimum} \index{Maximum}
\begin{description}[leftmargin=*]
\item[•] On dit que le réel $M$ est le \textbf{maximum} de $f$ sur $I$ lorsque pour tout réel $x$ de $I$, $f(x) \leq M$.
\item[•] On dit que le réel $m$ est le \textbf{minimum} de $f$ sur $I$ lorsque pour tout réel $x$ de $I$, $f(x) \geq m$.
\end{description} 
\end{DefT}


\mini{
\Exo{1}{VF-7}

\Exo{1}{VF-10}

\Exo{1}{VF-11}
}{
\Exo{1}{VF-15}

\Exo{1}{VF-8}

\Exo{1}{VF-9}

\Exo{1}{VF-16}
}




