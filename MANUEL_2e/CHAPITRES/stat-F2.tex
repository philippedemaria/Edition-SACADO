\begin{titre}[Statistiques descriptives]

\Titre{Histogramme}{3}
\end{titre}


\begin{CpsCol}
\textbf{Utiliser des nombres pour calculer et résoudre des problèmes}
\begin{description}
\item[$\square$] Construire un histogramme
\end{description}
\end{CpsCol}

\begin{DefT}{Histogramme}\index{Histogramme}
Un \textbf{histogramme} est un graphique composée de rectangles dont l'aire est proportionnelle à l'effectif de la classe.
\begin{description}[leftmargin=*]
\item[Cas 1. Chaque classe à la même amplitude] La hauteur de chaque rectangle est égale à son effectif ou sa fréquence.
\item[Cas 2. Les classes ont des amplitudes différentes] Pour chaque classe, on représente un rectangle dont la largeur est l'amplitude $A$ de sa classe et la hauteur égale à $\frac{E}{A}$ où $E$ est l'effectif de la classe ou sa fréquence.
\end{description}
\end{DefT}



\subsection*{Classe de largeur constante}

Construire un histogramme avec 1cm pour 10.


\begin{tabular}{|c|c|c|c|c|c|c|c|c|c|}
\hline 
Taille des roses en cm&[0;10[& [10;20[& [20;30[& [30;40[ &[40;50[ &[50;60[&[60;70[ &[70;80[ &[80;90[\\
\hline 
Effectif &10& 7& 29 &25& 15 &12& 5 &6 &5\\
\hline 
\end{tabular} 



\subsection*{Classe de largeur non constante}

\begin{enumerate}
\item Recopier et compléter le tableau.
\item Construire un histogramme avec 1cm pour 10.
\end{enumerate}


\begin{tabular}{|c|c|c|c|c|c|}
\hline 
Taille des roses en cm & [0 ; 20[& [20 ; 50[& [50 ; 60[& [60 ; 85[& [85 ; 100[\\
\hline 
Effectif &20& 30 &30&21 &6\\
\hline 
Amplitude &  &   & &  &\\
\hline 
Hauteur &&  &&  &\\
\hline 
\end{tabular} 


\EPC{1}{stat-3}{Représenter}