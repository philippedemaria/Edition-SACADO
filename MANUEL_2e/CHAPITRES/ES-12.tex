
Voici un extrait d'article, publié dans le journal "Le Monde" par le statisticien Michel Lejeune, après le premier tour de l'élection présidentielle de 2002.
« Pour les rares scientifiques qui savent comment sont produites les estimations, il était clair que l'écart des intentions de vote entre les candidats Le Pen et Jospin rendait tout à fait plausible le scénario qui s'est réalisé. En effet, certains des derniers sondages indiquaient 18\% pour Jospin et 14\% pour Le Pen. Si l'on se réfère à un sondage qui serait effectué dans des conditions idéales [...], on obtient sur de tels pourcentages une incertitude de plus ou moins 3\% étant donné la taille de l'échantillon [...]. »

Expliquer la phrase "l'écart des intentions de vote entre les candidats Le Pen et Jospin rendait tout à fait plausible le scénario qui s'est réalisé".

