\begin{titreTice}[Fonctions de référence]

\Titre{Algorithme de dichotomie}{0}
\end{titreTice}


\begin{CpsCol}
\textbf{Variations de fonctions}
\begin{description}
\item[$\square$] Connaitre un algorithme de dichotomie
\end{description}
\end{CpsCol}

\begin{Ety}\index{Algorithme de dichotomie}
\textbf{Dichotomie} : Du grec ancien dikhotomia : « division en deux parties »
\end{Ety}


On souhaite résoudre l'équation $x^2=x+1$ avec $x>0$. On considère l'algorithme suivant.


\begin{algobox}
\Variables
\Ligne a EST\_DU\_TYPE NOMBRE
\Ligne b EST\_DU\_TYPE NOMBRE
\Ligne m EST\_DU\_TYPE NOMBRE
\DebutAlgo
\Ligne a PREND\_LA\_VALEUR 1
\Ligne b PREND\_LA\_VALEUR 2
\Tantque{(b-a>0.01)}
\DebutTantQue
\Ligne m PREND\_LA\_VALEUR (a+b)/2
\Si{(m*m<m+1)}
\DebutSi
\Ligne a PREND\_LA\_VALEUR m
\FinSi
\Sinon
\DebutSinon
\Ligne b PREND\_LA\_VALEUR m
\FinSinon
\FinTantQue
\Ligne AFFICHER a
\Ligne AFFICHER b
\FinAlgo

\end{algobox}

On décide maintenant de créer ce tableau à l'aide d'un tableur.

\begin{enumerate}
\item  Compléter les 2 premières lignes à la mains. 

\item  Créer le tableau ci-dessous dans un tableur.

\begin{tabular}{|c|c|c|c|c|c|c|c|}
\hline 
\rowcolor{gray} & A & B & C & D & E & F & G\\ 
\hline 
\cellcolor{gray} 1 & $a$ & $b$ & $b-a$ & $m=\frac{a+b}{2}$ & $m^2$ & $m+1$ & Condition $m^2<m+1$\\ 
\hline 
\cellcolor{gray}2 & 0 & 10 &  &  &  & &  \\ 
\hline 
\cellcolor{gray}3 &  &  &  &  &  & &  \\
\hline 
\cellcolor{gray}4 &  &  &  &  &  & &  \\
\hline 
\cellcolor{gray}5 &  &  &  &  &  & &  \\
\hline 
\end{tabular} 


\item  Que faut-il écrire dans la cellule $C2$ ? Justifier.
\item  Compléter de même les cellule $D2$, $E2$ et $F2$.
\item  Quelles valeurs faut-il écrire dans les cellule $A3$ et $B3$ ? Justifier.

Dans la cellule $A3$ et la cellule $B3$, on doit inscrire une formule contenant un $SI$.

\textit{Dans un module d'aide de Open office, on a relevé :
La formule $=SI(A2=’3’ ;A3 ;A4)$ signifie : si la valeur de la cellule $A2$ est égale à 3 alors
choisir la cellule $A3$ sinon choisir la cellule $A4$.}

En transformant cette formule donnée, écrire la formule adéquate dans la cellule $A3$ puis une
formule dans la cellule $B3$.

\item  Compléter le tableau en tirant les cellule vers le bas jusqu'à ce que $b – a < 0,01$.
\item  Retrouver la valeur solution.
\item  Programmer cet algorithme en Python.
\end{enumerate}

