\begin{titre}[Fonctions et expressions algébriques]

\Titre{Calculs dans $\R$}{4}
\end{titre}


\begin{CpsCol}
\textbf{Utiliser des nombres pour calculer et résoudre des problèmes}
\begin{description}
\item[$\square$] Manipuler les racines carrées
\end{description}
\end{CpsCol}


\begin{ThT}{Les racines carrées\index{Racines carrées}}
Pour tous nombres $a$ et $b$ positifs, 
\begin{description}
\item $\sqrt{ab}=\sqrt{a}\sqrt{b}$
\item $\sqrt{\frac{a}{b}}=\frac{\sqrt{a}}{\sqrt{b}}$, $b >0$
\end{description}
\end{ThT}


\Exo

\begin{enumerate}
\item Écrire sous la forme $a\sqrt{b}$ les nombres suivants :

$A = \sqrt{8}$ , $B=\sqrt{27}$ , $C=\sqrt{20}$  , $D=\sqrt{18}$  , $D=\sqrt{12}$  , $D=\sqrt{72}$

\item

Simplifie les écritures :

$A=\frac{\sqrt{20}}{4}$ , $B = \frac{2-\sqrt{8}}{4}$ , $C=\frac{3-\sqrt{27}}{3}$ , $D=\sqrt{18}\sqrt{2}$  , $D=\sqrt{12}+\sqrt{45}$  , $E=\sqrt{18} -\sqrt{72} +\sqrt{32}$

\end{enumerate}


\Exo

\begin{enumerate}
\item
\begin{enumerate}
\item Tracer un carré $ABCD$ de coté $a$.  
\item Calculer la longueur de la diagonale $AC^2=2a$.
\item En déduire que $AC=a\sqrt{2}$
\end{enumerate}
\item
Applications : 
\begin{enumerate}
\item Quelle est la longueur de la diagonale d'un carré de coté 5 ?
\item Quelle est la longueur de l'hypoténuse d'un triangle isocèle rectangle dont un coté mesure 3cm ?
\end{enumerate}
\end{enumerate}



\Exo
 
\begin{enumerate}
\item Tracer un triangle équilatéral $ABC$ de coté $a$. $I$ est le milieu de $[AB]$. 
\item Quelle est la longueur $AI$ ?
\item Démontrer que $IC^2=\frac{3a^2}{4}$.
\item Démontrer que la hauteur d'un triangle équilatéral est égale à $\frac{a\sqrt3}{2}$.
\item Quelle est la longueur de la hauteur d'un triangle équilatéral de coté 6 ?
\end{enumerate}









