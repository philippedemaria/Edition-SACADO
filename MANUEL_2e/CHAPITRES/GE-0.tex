
ABCDEFGH est un cube de coté 6 cm représenté en perspective cavalière.


On coupe le cube par le plan (IJK). I est le milieu de [BC], GJ = 1,5 cm et DK = 1,5 cm.

\begin{center}
\definecolor{ffqqqq}{rgb}{1.,0.,0.}
\definecolor{qqqqcc}{rgb}{0.,0.,0.8}
\definecolor{qqqqff}{rgb}{0.,0.,1.}
\begin{tikzpicture}[line cap=round,line join=round,>=triangle 45,x=1.0cm,y=1.0cm]
\clip(-0.84,-0.24) rectangle (8.56,6.38);
\draw [color=qqqqcc] (0.,0.)-- (5.,0.);
\draw [color=qqqqcc] (5.,0.)-- (8.,1.);
\draw [dash pattern=on 3pt off 3pt,color=qqqqcc] (8.,1.)-- (3.,1.);
\draw [dash pattern=on 3pt off 3pt,color=qqqqcc] (3.,1.)-- (0.,0.);
\draw [color=qqqqcc] (0.,5.)-- (5.,5.);
\draw [color=qqqqcc] (5.,5.)-- (8.,6.);
\draw [color=qqqqcc] (8.,6.)-- (3.,6.);
\draw [color=qqqqcc] (3.,6.)-- (0.,5.);
\draw [color=qqqqff] (0.,5.)-- (0.,0.);
\draw [color=qqqqff] (5.,5.)-- (5.,0.);
\draw [color=qqqqff] (8.,6.)-- (8.,1.);
\draw [dash pattern=on 3pt off 3pt,color=qqqqff] (3.,6.)-- (3.,1.);
\draw [dash pattern=on 3pt off 3pt,color=ffqqqq] (6.65,0.55)-- (8.,4.76);
\draw [dash pattern=on 3pt off 3pt,color=ffqqqq] (8.,4.76)-- (4.42,1.);
\draw [dash pattern=on 3pt off 3pt,color=ffqqqq] (4.42,1.)-- (6.65,0.55);
\begin{scriptsize}
\draw [color=qqqqff] (0.,0.)-- ++(-1.0pt,0 pt) -- ++(2.0pt,0 pt) ++(-1.0pt,-1.0pt) -- ++(0 pt,2.0pt);
\draw[color=qqqqff] (0.14,0.24) node {$A$};
\draw [color=qqqqff] (5.,0.)-- ++(-1.0pt,0 pt) -- ++(2.0pt,0 pt) ++(-1.0pt,-1.0pt) -- ++(0 pt,2.0pt);
\draw[color=qqqqff] (5.14,0.24) node {$B$};
\draw [color=qqqqff] (8.,1.)-- ++(-1.0pt,0 pt) -- ++(2.0pt,0 pt) ++(-1.0pt,-1.0pt) -- ++(0 pt,2.0pt);
\draw[color=qqqqff] (8.14,1.24) node {$C$};
\draw [color=qqqqff] (3.,1.)-- ++(-1.0pt,0 pt) -- ++(2.0pt,0 pt) ++(-1.0pt,-1.0pt) -- ++(0 pt,2.0pt);
\draw[color=qqqqff] (3.14,1.24) node {$D$};
\draw [color=qqqqff] (0.,5.)-- ++(-1.0pt,0 pt) -- ++(2.0pt,0 pt) ++(-1.0pt,-1.0pt) -- ++(0 pt,2.0pt);
\draw[color=qqqqff] (-0.2,5.3) node {$E$};
\draw [color=qqqqff] (5.,5.)-- ++(-1.0pt,0 pt) -- ++(2.0pt,0 pt) ++(-1.0pt,-1.0pt) -- ++(0 pt,2.0pt);
\draw[color=qqqqff] (5.14,5.24) node {$F$};
\draw [color=qqqqff] (8.,6.)-- ++(-1.0pt,0 pt) -- ++(2.0pt,0 pt) ++(-1.0pt,-1.0pt) -- ++(0 pt,2.0pt);
\draw[color=qqqqff] (8.14,6.24) node {$G$};
\draw [color=qqqqff] (3.,6.)-- ++(-1.0pt,0 pt) -- ++(2.0pt,0 pt) ++(-1.0pt,-1.0pt) -- ++(0 pt,2.0pt);
\draw[color=qqqqff] (3.14,6.24) node {$H$};
\draw [color=ffqqqq] (6.65,0.55)-- ++(-1.0pt,0 pt) -- ++(2.0pt,0 pt) ++(-1.0pt,-1.0pt) -- ++(0 pt,2.0pt);
\draw[color=ffqqqq] (6.9,0.28) node {$I$};
\draw [color=ffqqqq] (8.,4.76)-- ++(-1.0pt,0 pt) -- ++(2.0pt,0 pt) ++(-1.0pt,-1.0pt) -- ++(0 pt,2.0pt);
\draw[color=ffqqqq] (8.14,5.) node {$J$};
\draw [color=ffqqqq] (4.42,1.)-- ++(-1.0pt,0 pt) -- ++(2.0pt,0 pt) ++(-1.0pt,-1.0pt) -- ++(0 pt,2.0pt);
\draw[color=ffqqqq] (4.22,1.32) node {$K$};
\end{scriptsize}
\end{tikzpicture}
\end{center}


\begin{enumerate}
\item Construire la pyramide KICJ. On pensera à placer les languettes pour que le solide soit fermé. La face de section doit être en papier. 
\item Construire le cube tronqué. La face de section doit être en papier. 
\item Rassembler les 2 solides. Le cube ABCDEFGH doit être reconstruit.
\end{enumerate}