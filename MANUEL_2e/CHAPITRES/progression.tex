
\chapter*{Progression}

% \usepackage{array} is required
\begin{tabular}{|c|>{\centering\arraybackslash}p{3cm}|p{4cm}|p{3cm}|>{\centering\arraybackslash}p{3cm}|c|}
\hline 
 & Thème & Contenus & Liens interdisciplinaires & Utilisation numérique & Durée \\ 
\hline 
1 & Fonctions et expressions algébriques & Ensemble de nombre, généralités sur les fonctions, expressions et equations simples & Physique (Courbe de Wien)/SVT(Codon) & Geogebra, Python & 2 \\ 
\hline 
2 & Statistiques & Pourcentage, évolution, Moyenne, médiane, les cumulés  & Histoire/géographie &  Tableur / Geogebra / Python & 2 \\ 
\hline 
3 & Variations de fonctions & Tableau, courbe, variation, optimisation, comparaison & Physique / SVT & Geogebra, Python & 2 \\ 
\hline 
4 & Repérage plan/espace & coordo milieu, distance, repérage & Français(construction de masque) Physique / SVT / Géographie & Geogebra & 2 \\ 
\hline 
5 & Fonctions affines & Problème de degré 1, signe de $ax+b$, variations & SVT & Geogebra / Python & 2 \\ 
\hline 
6 & Équations de droite & Tracé, lien avec la fonction affine, coefficient directeur & • & Tableur /  Geogébra & 2 \\ 
\hline 
7 & Fonctions de référence & Carré et inverse & Physique (Courbe de Wien + Loi de newton) & Geogebra / Python & 2 \\ 
\hline 
8 & Inéquations et ordre & Résolution et graphique & • & Geogébra & 2 \\ 
\hline 
9 & Géométrie dans l'espace & Incidence, forme et solide & Français (Versailles) / Physique (Rayonnement de corps noir) & Geogebra & 2 \\ 
\hline 
10 & Probabilités & Equiprobabilité, Propriété fondamentale & • & Python / Tableur  & 2 \\ 
\hline 
11 & Trigonométrie & Radian, cercle trigo, mesure d'angle & Geogébra & Physique(Optique) & 2 \\ 
\hline 
12 & Fonctions associées & seconde degré et homographique & Physique(Loi de Newton) & Geogébra & 2 \\ 
\hline 
13 & Inéquations et ordre & Résolution & • & Geogébra / Tableur  & 2 \\ 
\hline 
14 & Échantillonnage & Fluctuation et estimation & Géographie & Tableur / Python & 2 \\ 
\hline 
15 & Vecteurs & Construction, Chasles  & Mécanique & Geogébra & 2 \\ 
\hline 
\end{tabular} 