
Paul étudie une série statistique. Il calcule la moyenne et obtient $\overline{x}=6$ .
Or le professeur donne la valeur de la moyenne qu'il fallait trouver $\overline{x}=7,5$.
Paul s'aperçoit qu'il a oublié une valeur en n'utilisant que 7 valeurs.
\begin{enumerate}
\item \textit{Précise sans calcul si la valeur oubliée est inférieure ou supérieure à 6.}

La moyenne calculée par Paul est inférieure à la moyenne réelle, donc il a oublié une valeur supérieure 6.

\item Calculer la valeur manquante.

Paul a oublié une valeur et a calculé avec 7 valeurs. Il a donc 8 valeurs.


Soit $\overline{x}$ la moyenne.

$\overline{x}= \frac{M}{N}$ où M est la somme des valeurs et N le nombre de valeurs.


$6= \frac{M}{7}$ donc $M=42$

Comme Paul a oublié une valeur, il aurait du faire le calcul suivant :

$\overline{x}= \frac{M+ \text{valeur manquante}}{8}$. Or la moyenne réelle est 7,5 donc

$7,5= \frac{42+ \text{valeur manquante}}{8}$

$7,5 \times 8= 42+ \text{valeur manquante}$

$7,5 \times 8-42 = \text{valeur manquante}$

$60-42 = \text{valeur manquante}$

$18 = \text{valeur manquante}$
\end{enumerate}