
Si $p$ désigne la pression d'un gaz, $v$ son volume, $t$ sa température en degré Celsius, on a :$$ \frac{pv}{1+\frac{t}{273}}=C$$ $C$ est une constate.
\begin{enumerate}
\item En supposant que la masse gazeuse occupe à 0\deg à un volume de 10 $cm^3$, sous la pression de $1 kg/cm^2$, déterminer $C$.
\item Construire les courbes représentatives de la pression $p=y$ en fonction du volume $v=x$ de cette masse gazeuse, aux températures $t=50$\deg, $t=100$\deg. On fera varier $v$ de 0,1 à 20 $cm^3$.
\end{enumerate}