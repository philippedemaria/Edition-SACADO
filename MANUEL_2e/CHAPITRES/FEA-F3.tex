\begin{titre}[Fonctions et expressions algébriques]

\Titre{Expressions et problèmes}{4}
\end{titre}


\begin{CpsCol}
\textbf{Utiliser des nombres pour calculer et résoudre des problèmes}
\begin{description}
\item[$\square$] Associer à un problème une expression algébrique
\item[$\square$] Identifier une forme adéquate
\item[$\square$] Traduire le lien entre deux quantités
\end{description}
\end{CpsCol}



\Exo{1}{FEA-31}

\Exo{1}{FEA-33}

\mini{
\Exo{1}{FEA-34}
}{
\Exo{1}{FEA-35}
}
\mini{
\Exo{1}{FEA-36}

\Exo{1}{FEA-38}

\Exo{1}{FEA-41}

\Exo{1}{FEA-43}
}{
\Exo{1}{FEA-37}

\PO{1}{FEA-42}

\PO{1}{FEA-44}

\CR{1}{FEA-32}
}


%% \Exo{1}{FEA-69}

\EPCNA{Calculer. Raisonner.}

Démontrer chacune des égalités suivantes :
\begin{description}
\item[•] Pour $x \neq -1$ et $x \neq 0$, $\frac{1}{x}-\frac{1}{x+1}=\frac{1}{x(x+1)}$
\item[•] Pour $x \geq 0$ et $x \neq -1$, $\frac{1}{\sqrt{x}-1}-\frac{1}{\sqrt{x}+1}=\frac{3}{x-1}$
\end{description}


\EPCNA{Calculer. Raisonner.}

Soit $f(x) =\frac{1}{x-1} -\frac{1}{x+1}$, avec $x$ un nombre réel différent de $-1$ et $1$.

\begin{enumerate}
\item Déterminer le domaine de définition de la fonction $f$.
\item Démontrer que $f(x)=\frac{2}{x^2-1}$
\item Choisir la forme de $f(x)$ la plus adaptée pour calculer $f(3)$.
\item En déduire l'image de 3 par la fonction $f$.
\end{enumerate}
