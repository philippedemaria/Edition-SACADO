
La capacité vitale est le volume d'air maximal pouvant être mobilisé en une seule inspiration. Sur un échantillon de 17 personnes, on a mesuré la capacité vitale (en litres). Voici la liste des résultats :

$$4,15 - 4,48 - 5,24 - 4,8 - 4,95 - 4,05 - 4,3 - 4,7 - 5,51$$
$$ 4,58 - 4,12 - 5,7 - 4,85 - 5,05 - 4,65 - 4,7 - 4,28$$

\begin{enumerate}
\item Déterminer l'étendue et la moyenne de cette série. Arrondir la moyenne au centilitre près.
\item  En expliquant la méthode utilisée, déterminer la médiane de cette série.
\item  On décide de regrouper les valeurs de la série par classes. Recopier et compléter le tableau suivant :

\begin{tabular}{|c|c|c|c|c|}
\hline 
Capacité vitale   & $[4 ; 4,5[$ & $[4,5 ; 5[$ & $[5 ; 5,5[$ & $[5,5 ; 6[$ \\ 
\hline 
Effectifs &  &  &  &  \\ 
\hline 
ECC &  &  &  &  \\ 
\hline 
\end{tabular} 

\item  A l'aide de cette répartition par classes, déterminer la moyenne des valeurs.
On admet que dans chaque classe, la répartition est uniforme.
\item  Tracer alors la courbe des effectifs cumulés.
\item En déduire graphiquement la médiane de ces valeurs. Confirmer cette valeur par le calcul.
\end{enumerate}