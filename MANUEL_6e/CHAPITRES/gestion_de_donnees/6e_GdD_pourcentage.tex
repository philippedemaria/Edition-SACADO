\chapter{Pourcentage}
{https://sacado.xyz/qcm/parcours_show_course/0/117120}
{ 

 \begin{CpsCol}
 \textbf{Les savoir-faire du parcours} 
 \begin{itemize}
 \item Donner un ordre de grandeur d'un pourcentage.
 \item Savoir calculer un pourcentage.
 \end{itemize}
 \end{CpsCol}
}
%
%
%\begin{pageHistoire} 
% 
%En 1585, dans son ouvrage \textbf{La Disme}, Simon Stevin (1548 - 1620) ingénieur et mathématicien flamand, propose une écriture des nombres qui permet de simplifier les calculs (quelquefois très lourds en écriture fractionnaire).\\
%
%Il est considéré comme un précurseur de l'écriture décimale.
% 
%
%\end{pageHistoire} 

  
%%%%%%%%%%%%%%%%%%%%%%%%%%%%%%%%%%%%%%%%%%%%%%%%%%%%%%%%%%%%%%%%%%%%%%%%%%%%%%%%%%%%
%%%%%%%%%%        Cours             %%%%%%%%%%%%%%%%%%%%%%%%%%%%%%%%%%%%%%%%%%%%%%%%
%%%%%%%%%%%%%%%%%%%%%%%%%%%%%%%%%%%%%%%%%%%%%%%%%%%%%%%%%%%%%%%%%%%%%%%%%%%%%%%%%%%%
\begin{pageCours}
\section{Pourcentages}

\begin{DefT}{pourcentage}
Lorsqu'on partage une quantité en \textbf{$100$ parties égales}, on exprime une proportion de cette quantité en \textbf{pourcentage}\index{Pourcentage}.
\end{DefT}

\begin{Ex}
Dire que "$71\%$ des élèves aiment les mathématiques" signifie que "le nombre d'élèves qui aiment les mathématiques" est \textbf{proportionnel} au nombre total d'élèves \textbf{et} que pour $100$ élèves $71$ aiment les mathématiques.

\begin{center}
\begin{tabular}{|c|c|c|}\hline
Nombre d'élèves qui aiment les maths  & $71$ & $\dfrac{71 \times 275}{100}$ \\\hline
Nombre d'élèves & $100$ & $275$ \\\hline
\end{tabular}
\end{center}
\end{Ex}

\begin{Pp}
Soit $t$ un nombre. Prendre $t\%$ d'une quantité, c'est multiplier cette quantité par $\dfrac{t}{100}$.
\end{Pp}

\vspace{0.2cm}

\Sf{Quelques pourcentages à connaître}

\begin{center}
\begin{tabular}{|c|c|c|c|c|c|c|}\hline
Pourcentage & $10\%$ & $25\%$ & $50\%$ & $75\%$ & $100\%$ & $200\%$  \\\hline
revient à prendre ... & le dixième & le quart & la moitié & les trois-quarts & le tout & le double \\\hline
ou multiplier par ... & $0,1$ & $0,25$ & $0,5$ & $0,75$ & $1$ & $2$ \\\hline
\end{tabular}
\end{center}


\begin{ExCor}

Dans un collège de  $352$ élèves, $75\%$ des élèves adorent les mathématiques et $211$ aiment l'anglais.
 
 \begin{enumerate}[leftmargin=*]
 \item Combien d'élèves aiment les maths ? $\dfrac{75}{100} \times 352 = 264$. $264$ élèves aiment les mathématiques.
 \item Combien d'élèves aiment l'anglais ? $\dfrac{211}{352} \approx \dfrac{60}{100} $. $60\%$ des élèves aiment l'anglais.
 \end{enumerate}
 
\end{ExCor}
 



\end{pageCours}

%%%%%%%%%%%%%%%%%%%%%%%%%%%%%%%%%%%%%%%%%%%%%%%%%%%%%%%%%%%%%%%%%%%%%%%%%%%%%%%%%%%%
%%%%%%%%%%   Application directe    %%%%%%%%%%%%%%%%%%%%%%%%%%%%%%%%%%%%%%%%%%%%%%%%
%%%%%%%%%%%%%%%%%%%%%%%%%%%%%%%%%%%%%%%%%%%%%%%%%%%%%%%%%%%%%%%%%%%%%%%%%%%%%%%%%%%%
\begin{pageAD} 

\Sf{Écrire un nombre avec une fraction décimale}

\ExoCad{Calculer.}

Complète les égalités :
\begin{enumerate}[leftmargin=*]
\item $\dfrac{3}{5} = \dfrac{\ldots\ldots}{100}  \quad;\quad \dfrac{8}{10} = \dfrac{\ldots\ldots}{100}  \quad;\quad \dfrac{7}{25} = \dfrac{\ldots\ldots}{100}  \quad;\quad \dfrac{54}{75} = \dfrac{\ldots\ldots}{100} $ \vspace{0.4cm}


\item $12 \% = \dfrac{\ldots\ldots}{100}  \quad;\quad 20 \% = \dfrac{\ldots\ldots}{100}  \quad;\quad 36 \% = \dfrac{\ldots\ldots}{100} \quad;\quad 84\% = \dfrac{\ldots\ldots}{100}$\vspace{0.4cm}
\item $0,45 = \dfrac{\ldots\ldots}{100}  \quad;\quad 0,05 = \dfrac{\ldots\ldots}{100}  \quad;\quad 0,37 = \dfrac{\ldots\ldots}{100} \quad;\quad 0,6 = \dfrac{\ldots\ldots}{100}$ 
\end{enumerate}

\vspace{0.3cm}

\Sf{Donner un ordre de grandeur d'un pourcentage}

\ExoCad{Calculer.}

Donne un ordre de grandeur de chaque pourcentage :
\begin{enumerate}[leftmargin=*]
\item  $48\%$ de $60,45$\euro = \point{1} 
\item  $22\%$ de $125,99$\euro  = \point{1} 
\item  $73\%$ de $25,30$\euro  = \point{1} 
\end{enumerate}

\vspace{0.3cm}

\Sf{Calculer un pourcentage}

\ExoCad{Calculer.}

Dans une classe de 30 élèves, 30 \% d'entre eux portent des lunettes. Combien d'élèves portent des lunettes ?
 
\point{3}

\ExoCad{Calculer.}

Dans une tablette de chocolat de $220$g contient 70\% de cacao pur. Quelle est la masse de cacao pur dans une tablette de chocolat ?
 
\point{3}
 

\end{pageAD} 
 
%%%%%%%%%%%%%%%%%%%%%%%%%%%%%%%%%%%%%%%%%%%%%%%%%%%%%%%%%%%%%%%%%%%%%%%%%%%%%%%%%%%%
%%%%%%%%%%        Parcours 1    %%%%%%%%%%%%%%%%%%%%%%%%%%%%%%%%%%%%%%%%%%%%%%%%%%%%
%%%%%%%%%%%%%%%%%%%%%%%%%%%%%%%%%%%%%%%%%%%%%%%%%%%%%%%%%%%%%%%%%%%%%%%%%%%%%%%%%%%%
\begin{pageParcoursu} 

 
\ExoCu{Représenter.}

Complète le tableau suivant.

 \begin{tabular}{|c|c|c|c|}
  \hline 
  Ecriture décimale & Fraction décimale & Pourcentage \\ 
  \hline 
  $\ldots\ldots$  & $\dfrac{30}{100}$  &  $\ldots\ldots$  \\ 
  \hline 
  $\ldots\ldots$  & $\dfrac{\ldots\ldots}{100}$    & $50\%$  \\ 
  \hline 
 $0,09$  & $\dfrac{\ldots\ldots}{100}$   & $\ldots\ldots$   \\ 
  \hline 
$\ldots\ldots$    & $\dfrac{\ldots\ldots}{100}$   &  $5\%$   \\ 
  \hline 
$\ldots\ldots$    & $\dfrac{30}{40}$   &  $\ldots\ldots$   \\ 
  \hline 
 $0,74$   &  $\dfrac{\ldots\ldots}{100}$  &  $\ldots\ldots$   \\ 
  \hline
  \end{tabular}  


\ExoCu{Représenter. }

Entourer la bonne réponse.
\begin{itemize}[leftmargin=*]
\item Les 40\% de 200 représentent	$\quad \quad 80	\quad\quad 160\quad\quad	120$ \vspace{0.1cm}
\item Les 10\% de 150 représentent	$\quad\quad 30	\quad\quad 15	\quad\quad40$\vspace{0.1cm}
\item Les 100\% de 75 représentent	$\quad\quad 7,5 \quad\quad 750	\quad\quad 75$\vspace{0.1cm}
\end{itemize}


\ExoCu{Modéliser. Calculer.}

Un collège comporte $250$ élèves. $30\%$ des élèves sont demi-pensionnaires. Calcule le nombre d'élèves demi-pensionnaires. \point{3}
 

 
\ExoCu{Représenter. Calculer.}
 
 A l'assemblée parlementaire de la France, il y $577$ députés dont $215$ femmes.
 
\begin{itemize}[leftmargin=*]
\item Quel est le pourcentage de femmes députées en France ?\point{3}
\item Est-il vrai que le pourcentage d'hommes députés en France est d'environ $63\%$ ?\point{3}
\end{itemize}

\end{pageParcoursu}
%%%%%%%%%%%%%%%%%%%%%%%%%%%%%%%%%%%%%%%%%%%%%%%%%%%%%%%%%%%%%%%%%%%%%%%%%%%%%%%%%%%%
%%%%%%%%%%        Parcours 2    %%%%%%%%%%%%%%%%%%%%%%%%%%%%%%%%%%%%%%%%%%%%%%%%%%%%
%%%%%%%%%%%%%%%%%%%%%%%%%%%%%%%%%%%%%%%%%%%%%%%%%%%%%%%%%%%%%%%%%%%%%%%%%%%%%%%%%%%%
\begin{pageParcoursd} 



\ExoCd{Calculer.}

Un collège comporte $775$ élèves. $24\%$ des élèves sont externes. Calcule le nombre d'élèves externes. \point{3}
 


 

\ExoCd{Calculer.}

Une citerne peut contenir $5000$ L d'eau. Elle est remplie à $60\%$.

\begin{enumerate}[leftmargin=*]
\item Combien de litres d'eau contient la citerne ? \point{3}
\item Combien de litres d'eau peut-on rajouter pour qu'elle soit pleine ? \point{3}
\end{enumerate}


\ExoCd{Chercher. Représenter.}

Pour son anniversaire, Alicia a trouvé une recette de gâteau. Pour 4 personnes il faut : $4$ oeufs, $20\,g$ de sucre, $100\,g$ de chocolat, $150\,g$ de farine.

Quel est le pourcentage de sucre dans cette recette ? \point{3}

\point{4}

\ExoCd{Modéliser. Calculer.}

Lors des soldes, un magasin affiche une réduction de 35\%. 
\begin{enumerate}[leftmargin=*]
\item Quelle est la réduction sur une jupe affichée à 57 \euro ?\point{3}
\item Quel est alors le prix de cette jupe après la réduction ?\point{3}
\end{enumerate}


\end{pageParcoursd}
%%%%%%%%%%%%%%%%%%%%%%%%%%%%%%%%%%%%%%%%%%%%%%%%%%%%%%%%%%%%%%%%%%%%%%%%%%%%%%%%%%%%
%%%%%%%%%%        Parcours 3    %%%%%%%%%%%%%%%%%%%%%%%%%%%%%%%%%%%%%%%%%%%%%%%%%%%%
%%%%%%%%%%%%%%%%%%%%%%%%%%%%%%%%%%%%%%%%%%%%%%%%%%%%%%%%%%%%%%%%%%%%%%%%%%%%%%%%%%%%
\begin{pageParcourst}

\ExoCt{Modéliser. Calculer.}

Sur un paquet de gâteaux de $200$g, on lit "contient $20\%$ de sucre".  Si je mange $30\%$ des gâteaux du paquet, quelle masse de sucre que ai-je absorbé ?\point{5}
 


\ExoCt{Modéliser. Calculer.}

Au mois de janvier les prix ont augmenté de $5\%$ puis en février les prix ont encore augmenté de $4\%$. On souhaite connaitre l'augmentation sur ces deux mois.
 
\begin{enumerate}[leftmargin=*]
\item Une veste coute 50 \euro au début du mois de janvier. Quel est son prix à la fin du mois de janvier ?  \point{3}
\item Quel est son prix à la fin du mois de février ?  \point{3}
\item L'augmentation est-elle de $9\%$ sur ces deux mois ?  \point{3}
\end{enumerate}

\ExoCt{Modéliser. Calculer.}

Dans l'atmosphère, l'air est composé d'environ $78\%$ de diazote, $21\%$ de dioxygène et le reste est un mélange de gaz rares. Les dimensions de la classe sont : $7m \times 6m  \times 2,50m$

Calcule le volume occupé par chaque gaz dans la classe.

\point{6}


\end{pageParcourst}


\begin{pageBrouillon} 
 
\ligne{30}

\end{pageBrouillon}


%%%%%%%%%%%%%%%%%%%%%%%%%%%%%%%%%%%%%%%%%%%%%%%%%%%%%%%%%%%%%%%%%%%%%%%%%%%%%%%%%%%%
%%%%%%%%%%   Auto-evaluation    %%%%%%%%%%%%%%%%%%%%%%%%%%%%%%%%%%%%%%%%%%%%%%%%%%%%
%%%%%%%%%%%%%%%%%%%%%%%%%%%%%%%%%%%%%%%%%%%%%%%%%%%%%%%%%%%%%%%%%%%%%%%%%%%%%%%%%%%%
\begin{pageAuto} 

\ExoAuto

Lors de l'élection municipale, sur 3569 votants, Monsieur Pi obtient $26\%$ des voix, Madame Tetha 942 voix et Madame Mu le reste des voix.

\begin{enumerate}[leftmargin=*]
\item Calcule le nombre de voix obtenue par Monsieur Pi. \point{3}
\item Calcule le nombre de voix obtenue par Madame Mu.  \point{3}
\item Calcule le pourcentage des voix obtenue par Madame Tetha.  \point{3}
\item Lequel de ces trois candidats est élu ?  \point{3}
\end{enumerate}


\ExoAuto

On dit qu'un nouveau-né gagne $25\%$ du poids de sa naissance le premier mois de sa vie. Némo pèse $4,250$ kg le jour de sa naissance le $1$ avril.

Quel sera le poids de Némo le $1$ mai de la même année ?\point{3}
  
\ExoAuto

Lors des soldes, un magasin affiche une réduction de 22\%. 
\begin{enumerate}[leftmargin=*]
\item Quelle est la réduction sur un pantalon affichée à 49 \euro ?\point{3}
\item Quel est alors le prix de ce pantalon après la réduction ?\point{3}
\end{enumerate}



 
\end{pageAuto}
%%%%%%%%%%%%%%%%%%%%%%%%%%%%%%%%%%%%%%%%%%%%%%%%%%%%%%%%%%%%%%%%%%%%%%%%%%%%%%%%%%%%
% Parcours 3
%%%%%%%%%%%%%%%%%%%%%%%%%%%%%%%%%%%%%%%%%%%%%%%%%%%%%%%%%%%%%%%%%%%%%%%%%%%%%%%%%%%%

 
 
