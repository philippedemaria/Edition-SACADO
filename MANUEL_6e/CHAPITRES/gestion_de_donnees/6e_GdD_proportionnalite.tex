\chapter{Proportionnalité}
{https://sacado.xyz/qcm/parcours_show_course/0/117120}
{ 

 \begin{CpsCol}
 \textbf{Les savoir-faire du parcours} 
 \begin{itemize}
 \item Reconnaitre une situation de proportionnalité
 \item Reconnaitre un tableau de proportionnalité
 \item Compléter un tableau de proportionnalité
 \item Savoir utiliser la proportionnalité.
 \end{itemize}
 \end{CpsCol}
}
%
%
%\begin{pageHistoire} 
% 
%En 1585, dans son ouvrage \textbf{La Disme}, Simon Stevin (1548 - 1620) ingénieur et mathématicien flamand, propose une écriture des nombres qui permet de simplifier les calculs (quelquefois très lourds en écriture fractionnaire).\\
%
%Il est considéré comme un précurseur de l'écriture décimale.
% 
%
%\end{pageHistoire} 



%%%%%%%%%%%%%%%%%%%%%%%%%%%%%%%%%%%%%%%%%%%%%%%%%%%%%%%%%%%%%%%%%%%%%%%%%%%%%%%%%%%%
%%%%%%%%%%        Cours             %%%%%%%%%%%%%%%%%%%%%%%%%%%%%%%%%%%%%%%%%%%%%%%%
%%%%%%%%%%%%%%%%%%%%%%%%%%%%%%%%%%%%%%%%%%%%%%%%%%%%%%%%%%%%%%%%%%%%%%%%%%%%%%%%%%%%
\begin{pageCours} 

\section{Grandeurs proportionnelles}



\begin{DefT}{Grandeurs proportionnelles} 
Deux grandeurs sont \textbf{proportionnelles}\index{Grandeurs proportionnelles|Proportionnalité} si les valeurs de l'une s'obtiennent en \textbf{multipliant} les valeurs de l'autre par un \textbf{même nombre}.
\end{DefT}


\begin{Ex}

Une feuille de papier pèse $5$g. Deux de ces feuilles de papier pèsent $10$g. Trois de ces feuilles de papier pèsent $15$g.  
Mille de ces feuilles de papier pèsent $\np{5000}$g.  

\end{Ex}



\begin{Att}

Beaucoup de situations de la vie ne sont pas proportionnelles : 
\begin{itemize}[leftmargin=*]
\item La montée de l'océan pendant la marée n'est pas une situation de proportionnalité.
\item La taille des êtres vivants en général.
\end{itemize}
\end{Att}

\section{Tableau de proportionnalité}

\begin{DefT}{Tableau de proportionnalité}\index{Tableau de|Proportionnalité}

Dans un tableau de nombres liant deux grandeurs, on reconnait une \textbf{situation de proportionnalité}\index{Situation de|Proportionnalité} lorsque les nombres de la deuxième ligne s'obtiennent en \textbf{multipliant} ceux de la première par un \textbf{même nombre}.

Ce nombre est appelé \textbf{coefficient de proportionnalité}\index{coefficient de|Proportionnalité}.
\end{DefT}

\begin{ExT}{Prix des avocats}

\begin{minipage}{0.5\linewidth}
\begin{center}
    \begin{tabular}{|c|c|c|c|}\hline 
        Nombre de avocats & 6 & 10 & 15  \\
        \hline Prix en \euro & 8,4 & 14 & 21 \\\hline 
    \end{tabular}
\end{center}
\end{minipage}
\begin{minipage}{0.5\linewidth}
    
\[\frac{8,4}{6}=1,4\hspace{.5cm}\frac{14}{10}=1,4\hspace{.5cm}\frac{21}{15}=1,4\]

Le coefficient de proportionnalité est 1,4. Cela signifie que \textbf{1 avocat} coûte \textbf{1,40 \euro}.
\end{minipage}
\end{ExT}



\end{pageCours} 
%%%%%%%%%%%%%%%%%%%%%%%%%%%%%%%%%%%%%%%%%%%%%%%%%%%%%%%%%%%%%%%%%%%%%%%%%%%%%%%%%%%%
%%%%%%%%%%   Application directe    %%%%%%%%%%%%%%%%%%%%%%%%%%%%%%%%%%%%%%%%%%%%%%%%
%%%%%%%%%%%%%%%%%%%%%%%%%%%%%%%%%%%%%%%%%%%%%%%%%%%%%%%%%%%%%%%%%%%%%%%%%%%%%%%%%%%%
\begin{pageAD} 

\Sf{Reconnaître une situation de proportionnalité}

\ExoCad{Chercher.}
 
Parmi les situations suivantes, nomme les grandeurs étudiées et celles proportionnelles.
\begin{itemize}[leftmargin=*]
 \item Evan mesurait $1,20\,m$ l'année dernière, aujourd'hui sa taille est de $1,31\,m$. \point{2}
 \item Louna achète $8\,m$ de corde à $3,40\,$\euro le mètre. \point{2}
 \item Dans sa vitrine, un primeur a écrit : un melon - $1,65\,$\euro ; deux melons - $3,10\,$\euro. \point{2}
\end{itemize}
 




\ExoCad{Chercher. Communiquer.}

Un pâtissier regarde une recette de gâteau.
\begin{center}
    \begin{tabular}{|c|c|c|c|}        
    	\hline 
        Nombre de convives & $4$ & $6$ & $12$ \\ 
    	\hline 
         masse (g) & $150$ & $225$ & $450$ \\\hline 
    \end{tabular}
\end{center}

\begin{enumerate}[leftmargin=*]
\item Quelles sont les grandeurs étudiées ? \point{2}
\item Ces grandeurs sont-elles proportionnelles ? \point{3}
\end{enumerate}
 




\ExoCad{Représenter.}

Un loueur de vélo propose les tarifs suivants :


\begin{center}
    \begin{tabular}{|c|c|c|c|}\hline 
        Durée de location (h) & $1$ & $2$ & $4$ \\
        \hline Prix (\euro) & $12$ & $24$ & $40$ \\\hline  
    \end{tabular}
\end{center}

\begin{enumerate}[leftmargin=*]
\item Quelles sont les grandeurs étudiées ? \point{2}
\item Ces grandeurs sont-elles proportionnelles ? \point{3}
\end{enumerate}
 

 

\end{pageAD}  
%%%%%%%%%%%%%%%%%%%%%%%%%%%%%%%%%%%%%%%%%%%%%%%%%%%%%%%%%%%%%%%%%%%%%%%%%%%%%%%%%%%%
%%%%%%%%%%        Cours             %%%%%%%%%%%%%%%%%%%%%%%%%%%%%%%%%%%%%%%%%%%%%%%%
%%%%%%%%%%%%%%%%%%%%%%%%%%%%%%%%%%%%%%%%%%%%%%%%%%%%%%%%%%%%%%%%%%%%%%%%%%%%%%%%%%%%
\begin{pageCours} 

\section{Compléter un tableau de proportionnalité}

\begin{Def}
Dans un \textbf{tableau de proportionnalité}, lorsqu'on connaît trois nombres non nuls (dont deux se correspondent), on peut calculer le \textbf{quatrième nombre manquant}.

Ce nombre manquant est appelé une \textbf{quatrième proportionnelle}.
\end{Def}

\begin{Pp}
Dans une situation de proportionnalité, on peut :
\begin{itemize}[leftmargin=*]
\item Multiplier une colonne par un nombre pour passer à une autre colonne.
\item Ajouter deux colonnes entre elles pour en obtenir une troisième.
\end{itemize}
\end{Pp} 

\begin{Ex}
Le débit d'un robinet est régulier, c'est-à-dire que le nombre de litres qui s'écoulent est proportionnel à la durée d'écoulement. En 5 min, il s'écoule $8 \,L$ d'eau. En combien de temps s'écoule-t-il $20 \,L$ ? $28\,L$ ?\\

Pour passer de $8\,L$ à $20\,L$ je multiplie par $2,5$ : $8\,(L)\times2,5=20\,(L)$ donc $5\,(min)\times2,5=12,5\,(min)$. Il faudra 12 min 30 s pour que s'écoule $20\,L$.\\


Je sais que $8\,(L)+20\,(L)=28\,(L)$ donc $5\,(min)+12,5\,(min)=17,5\,(min)$. Il faudra 17 min et 30 s pour que s'écoule $28\,L$.\\
\begin{center}
    \begin{tabular}{|c|c|c|c|}\hline 
        Quantité d'eau (L) & $8$ & $20$ & $28$  \\\hline 
        Durée (min) & $5$ & $12,5$ & $17,5$ \\\hline 
    \end{tabular}
\end{center}
\end{Ex}



\end{pageCours} 
%%%%%%%%%%%%%%%%%%%%%%%%%%%%%%%%%%%%%%%%%%%%%%%%%%%%%%%%%%%%%%%%%%%%%%%%%%%%%%%%%%%%
%%%%%%%%%%   Application directe    %%%%%%%%%%%%%%%%%%%%%%%%%%%%%%%%%%%%%%%%%%%%%%%%
%%%%%%%%%%%%%%%%%%%%%%%%%%%%%%%%%%%%%%%%%%%%%%%%%%%%%%%%%%%%%%%%%%%%%%%%%%%%%%%%%%%%
\begin{pageAD} 

\Sf{Compléter un tableau de proportionnalité}

\ExoCad{Chercher. Calculer. }


 \begin{enumerate}
 \item Mathilde souhaite préparer un cocktail et pour cela, elle a besoin de jus d'oranges.
 Avec 2 oranges, elle obtient $40$ cL de jus d'oranges. Compléter alors le tableau en supposant que le volume de jus est proportionnel au nombre d'oranges.
 \begin{center}
     \begin{tabular}{|c|c|c|c|c|}\hline
         Nombre d'oranges & $2$ & $6$ & $7$ &  \\
         \hline Volume de jus (cL) & $40$ &  & & $180$ \\\hline
     \end{tabular}
 \end{center}
 
 \point{3}
 
 \item Un cycliste a parcouru $50$ km en 3 heures. En supposant qu’il roule toujours à la
 même vitesse, compléter le tableau :
 \begin{center}
     \begin{tabular}{|c|c|c|c|c|c|c|c|}\hline
         Distance (km) & $75$ & $100$ & $150$ &  $\ldots\ldots$  & $110$ & $30$ &  $\ldots\ldots$ \\\hline
          Temps (min) &  $\ldots\ldots$ &  $\ldots\ldots$  &  $\ldots\ldots$ & $270$ &  $\ldots\ldots$ &  $\ldots\ldots$& $172$ \\\hline
     \end{tabular}
 \end{center} 
 
\point{3}

 \item Pour faire de la glace, on a besoin de $250$ g de sucre et de $8$ œufs par litre de lait. , compléter le tableau :
 \begin{center}
     \begin{tabular}{|c|c|c|c|c|c|c|c|}\hline
         Lait (L) & $0,5$ & $1$ & $2$ &  $\ldots\ldots$ & $5$ & $10$ &  $\ldots\ldots$ \\\hline
          Quantité d'œufs &  $\ldots\ldots$ & $\ldots\ldots$   & $\ldots\ldots$ & $24$ & $\ldots\ldots$ & $\ldots\ldots$ &  $96$ \\\hline
     \end{tabular}
 \end{center} 
 
 \point{3}
 \end{enumerate}


 

\end{pageAD} 
 
%%%%%%%%%%%%%%%%%%%%%%%%%%%%%%%%%%%%%%%%%%%%%%%%%%%%%%%%%%%%%%%%%%%%%%%%%%%%%%%%%%%%
%%%%%%%%%%        Parcours 1    %%%%%%%%%%%%%%%%%%%%%%%%%%%%%%%%%%%%%%%%%%%%%%%%%%%%
%%%%%%%%%%%%%%%%%%%%%%%%%%%%%%%%%%%%%%%%%%%%%%%%%%%%%%%%%%%%%%%%%%%%%%%%%%%%%%%%%%%%
\begin{pageParcoursu} 

\ExoCu{Chercher.}
 
Parmi les situations suivantes, sont-elles proportionnelles ? Explique ton raisonnement.
\begin{itemize}[leftmargin=*]
 \item Une fournisseur d'accès téléphonique propose trois formules : $10$ min. à $2,5$ \euro ; $20$  min. à $5$\euro ; $60$  min. à $12$\euro. \point{2}
 \item Un robinet a une fuite, il perd $0,6\,L$ par heure. \point{2}
\end{itemize}


\ExoCu{Chercher.}

Candice a téléchargé deux fichiers et elle a écrit les détails dans le tableau. La durée de téléchargement est-elle proportionnelle à la taille du fichier ? Explique ton raisonnement.

\begin{center}
\begin{tabular}{|c|c|c|}
\hline 
Durée & 5 minutes & 20 minutes \\ 
\hline 
Taille & 4Go & 10Go  \\ 
\hline 
\end{tabular} 
\end{center}
\point{2}
 
\ExoCu{Calculer.}

Chez le primeur, les noix de coco sont en promotion. On peut lire :
\begin{center}
\begin{tabular}{|c|c|c|c|}
\hline 
nombre de noix de coco& 1 & 2 & 4 \\ 
\hline 
Prix (\euro) & 1,5 & 3  & 6  \\ 
\hline 
\end{tabular} 
\end{center}
Le prix des  noix de coco est-il proportionnel au nombre de  noix de coco achetées ? Explique ton raisonnement.
 
\point{4}



\ExoCu{Calculer.}

Complète les tableaux suivants pour qu'ils représentent à des situations de proportionnalité.

\begin{minipage}{0.3\linewidth}
\begin{tabular}{|c|c|c|c|}
\hline 
$1$ & $2$ & $3$ & $\ldots$  \\ 
\hline 
$\ldots$ & $8$ & $\ldots$ & $24$ \\ 
\hline 
\end{tabular} 
\end{minipage} 
\begin{minipage}{0.33\linewidth}
\begin{tabular}{|c|c|c|c|}
\hline 
$1$ & $3,2$ & $9$ & $\ldots\ldots$  \\ 
\hline 
$\ldots$ & $4,8$ & $\ldots\ldots$ & $19,5$ \\ 
\hline 
\end{tabular} 
\end{minipage} 
\begin{minipage}{0.33\linewidth}
\begin{tabular}{|c|c|c|c|}
\hline 
$12$ & $23$ & $\ldots\ldots$ & $\ldots\ldots$  \\ 
\hline 
$1,2$ & $\ldots\ldots$ & $3,9$ & $5,45$ \\ 
\hline 
\end{tabular} 
\end{minipage} 

\end{pageParcoursu}
%%%%%%%%%%%%%%%%%%%%%%%%%%%%%%%%%%%%%%%%%%%%%%%%%%%%%%%%%%%%%%%%%%%%%%%%%%%%%%%%%%%%
%%%%%%%%%%        Parcours 2    %%%%%%%%%%%%%%%%%%%%%%%%%%%%%%%%%%%%%%%%%%%%%%%%%%%%
%%%%%%%%%%%%%%%%%%%%%%%%%%%%%%%%%%%%%%%%%%%%%%%%%%%%%%%%%%%%%%%%%%%%%%%%%%%%%%%%%%%%
\begin{pageParcoursd} 
 
\ExoCd{Calculer.}

Popeye achète $23$ mètres de corde à $4,5$ \euro le mètre pour amarrer son bateau. 
\vspace{0.2cm}
Combien paye-t-il la corde ? \point{2}

\ExoCd{Chercher. Calculer.}
 

Le prix des bouteilles d'eau est proportionnel au nombre de bouteilles achetées. Le pack de $6$ bouteilles coûte $1,80$ \euro. Complète le tableau.
\begin{center}
\begin{tabular}{|c|c|c|c|c|}
\hline 
nombre de bouteilles & $1$ & $3$ & $6$ & $10$ \\ 
\hline 
Prix (\euro) & $\ldots\ldots$ & $\ldots\ldots$ & $\ldots\ldots$ & $\ldots\ldots$   \\ 
\hline 
\end{tabular} 
\end{center}


 

 
\ExoCd{Représenter. Calculer. Communiquer.}
 
Tom et Zoë jouent à leur jeu favori en $8$ parties. Chaque partie gagnée rapporte $4,5$ points. 

\begin{enumerate}[leftmargin=*]
\item Tom a gagné $3$ parties. Combien de points a-t-il ?\point{2}
\item Combien de points Zoë a-t-elle obtenu ? \point{3}
\end{enumerate}
 



  
\ExoCd{Calculer.} 
  
Dans une pâtisserie, $6$ gâteaux coûtent $3,59$\euro. Sachant que ces gâteaux coûtent tous le même prix,

\begin{enumerate}[leftmargin=*]
\item combien coûtent $7$ de ces gâteaux ?  \point{3}
\item combien coûtent $9$ de ces gâteaux ? \point{3}
\item Combien de gâteaux puis-je acheter avec $33$\euro ?  \point{3}
\end{enumerate}
 
\end{pageParcoursd}
%%%%%%%%%%%%%%%%%%%%%%%%%%%%%%%%%%%%%%%%%%%%%%%%%%%%%%%%%%%%%%%%%%%%%%%%%%%%%%%%%%%%
%%%%%%%%%%        Parcours 3    %%%%%%%%%%%%%%%%%%%%%%%%%%%%%%%%%%%%%%%%%%%%%%%%%%%%
%%%%%%%%%%%%%%%%%%%%%%%%%%%%%%%%%%%%%%%%%%%%%%%%%%%%%%%%%%%%%%%%%%%%%%%%%%%%%%%%%%%%
\begin{pageParcourst}



\ExoCt{Calculer.}

Sur une carte à l'échelle $\dfrac{1}{\np{100000}}$, deux villes sont distantes de $5,5$ km. Quelle est la distance réelle entre ces deux villes ?

\point{3}



\ExoCt{Calculer.}
 
  12 maçons construisent coûtent 12 maisons en 12 jours. Combien de maisons sont-elles construites par 24 maçons en 24 jours ?
\point{5}


\ExoCt{Calculer.}
 
Une personne distribue l'argent de poche à ses trois enfants, Marie, Mathis($5$ ans) et Maël 
proportionnellement à leur âge. Elle donne $2,5$ \euro à Mathis, $4$ \euro  à Marie et le reste à Maël. 
Sachant que la somme des âges des enfants est $23$ ans, 
\begin{enumerate}[leftmargin=*]
\item Quels sont les âges de Marie et de Maël ? \point{4}
\item Quel est l'argent de poche de Maël ? \point{2}
\item Quelle est la somme totale distribuée ?\point{3}
\end{enumerate}

\ExoCt{Calculer.}
 
Trouve les nombres $a$, $b$ et $c$ pour que les quadruplets ($a$ ; $b$ ; $10$ ; $a+b+10$) et ($50$ ; $75$ ; $250$ ; $c$) soient proportionnels.
\point{4}



\end{pageParcourst}
%%%%%%%%%%%%%%%%%%%%%%%%%%%%%%%%%%%%%%%%%%%%%%%%%%%%%%%%%%%%%%%%%%%%%%%%%%%%%%%%%%%%
%%%%%%%%%%   Auto-evaluation    %%%%%%%%%%%%%%%%%%%%%%%%%%%%%%%%%%%%%%%%%%%%%%%%%%%%
%%%%%%%%%%%%%%%%%%%%%%%%%%%%%%%%%%%%%%%%%%%%%%%%%%%%%%%%%%%%%%%%%%%%%%%%%%%%%%%%%%%%
\begin{pageAuto}
 
\ExoAuto

Antoine a pris deux fois l'autobus. Il a payé $3$ \euro pour $10$ minutes lors de son premier trajet et $5$ \euro lors de son second trajet de $12$ minutes. 
\begin{enumerate}[leftmargin=*]
\item Quelles sont les grandeurs étudiées ? \point{2}
\item Le prix d'un trajet est-il proportionnel au temps du trajet ? Explique ton raisonnement.
\point{3}
\end{enumerate}

\ExoAuto

A la boulangerie, on peut lire :
\begin{center}
\begin{tabular}{|c|c|c|c|}
\hline 
nombre de baguettes & 1 & 3 & 10 \\ 
\hline 
Prix  & 0,99\euro & 2,97\euro & 9,70\euro \\ 
\hline 
\end{tabular} 
\end{center}
Le prix des baguettes est-il proportionnel au nombre de baguettes achetées ? Explique ton raisonnement.

\point{3}
 

\ExoAuto

 

En septembre, Valentine fait les vendanges. Son salaire est proportionnel au nombre d'heures travaillées. Le premier jour pour 6 heures de travail, elle a gagné 57 euros.

 Complète le tableau.

\begin{center}
\begin{tabular}{|c|c|c|c|c|}
\hline 
nombre d'heures & 6 & 7 & 13 & $\ldots\ldots$ \\ 
\hline 
salaire (\euro) & $\ldots\ldots$ & $\ldots\ldots$   & $\ldots\ldots$  & $95$   \\ 
\hline 
\end{tabular} 
\end{center}
 
 \point{2}
  

\end{pageAuto}
%%%%%%%%%%%%%%%%%%%%%%%%%%%%%%%%%%%%%%%%%%%%%%%%%%%%%%%%%%%%%%%%%%%%%%%%%%%%%%%%%%%%
% Parcours 3
%%%%%%%%%%%%%%%%%%%%%%%%%%%%%%%%%%%%%%%%%%%%%%%%%%%%%%%%%%%%%%%%%%%%%%%%%%%%%%%%%%%%
\begin{pageBrouillon} 
 
\ligne{30}

\end{pageBrouillon}


