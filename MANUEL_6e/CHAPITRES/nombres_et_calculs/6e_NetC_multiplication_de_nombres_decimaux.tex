 
\chapter{Multiplication de nombres décimaux}
{https://sacado.xyz/qcm/parcours_show_course/0/117121}
{
 \section{Les savoir-faire du parcours}

\begin{CpsCol}
  \textbf{Les savoir-faire du parcours}
 \begin{itemize}
 \item Savoir utiliser les tables de multiplication.
 \item Savoir utiliser le vocabulaire de la multiplication.
 \item Savoir calculer mentalement certains produits.
 \item Savoir calculer astucieusement certains produits.
 \item Savoir poser la multiplication de deux nombres entiers.
 \item Savoir multiplier un nombre par 10 ; 100 ou 1 000.
 \item Savoir multiplier un nombre par 0,1 ; 0,01 ou 0,001.
 \item Savoir multiplier deux nombres décimaux.
 \item Savoir poser la multiplication de deux nombres décimaux.
 \item Savoir déterminer l'ordre de grandeur d'un résultat.
 \item Savoir résoudre un problème numérique.
 \end{itemize}
 \end{CpsCol}

 


}
\begin{pageCours} 

\section{Les tables de multiplication}

\begin{center}
\begin{tabular}{c||c|c|c|c|c|c|c|c|c|c}
\textbf{$\times$} & \textbf{1} & \textbf{2} & \textbf{3} & \textbf{4} & \textbf{5} & \textbf{6} & \textbf{7} & \textbf{8} & \textbf{9} & \textbf{10} \\\hline\hline
\textbf{1} & 1 & 2 & 3 & 4 & 5 & 6 & 7 & 8 & 9 & 10 \\\hline
\textbf{2} & 2 & 4 & 6 & 8 & 10 & 12 & 14 & 16 & 18 & 20 \\\hline
\textbf{3} & 3 & 6 & 9 & 12 & 15 & 18 & 21 & 24 & 27 & 30 \\\hline
\textbf{4} & 4 & 8 & 12 & 16 & 20 & 24 & 28 & 32 & 36 & 40 \\\hline
\textbf{5} & 5 & 10 & 15 & 20 & 25 & 30 & 35 & 40 & 45 & 50 \\\hline
\textbf{6} & 6 & 12 & 18 & 24 & 30 & 36 & 42 & 48 & 54 & 60 \\\hline
\textbf{7} & 7 & 14 & 21 & 28 & 35 & 42 & 49 & 56 & 63 & 70 \\\hline
\textbf{8} & 8 & 16 & 24 & 32 & 40 & 48 & 56 & 64 & 72 & 80 \\\hline
\textbf{9} & 9 & 18 & 27 & 36 & 45 & 54 & 63 & 72 & 81 & 90 \\\hline
\textbf{10} & 10 & 20 & 30 & 40 & 50 & 60 & 70 & 80 & 90 & 100 \\
\end{tabular}
\end{center}

\section{La multiplication}

\begin{Voc}
Lorsqu'on multiplie deux nombres :
\begin{itemize}
\item On appelle les nombres que l’on multiplie les facteurs de la multiplication.
\item On appelle le résultat d’une multiplication le produit des facteurs.
\end{itemize}
\end{Voc}

\begin{Ex}
Dans l'opération : $84\times35=2940$, les nombres $84$ et $35$ sont les \textcolor{red}{facteurs}, le nombre $2940$ est le \textcolor{red}{produit}.
\end{Ex}

\begin{Pp}
Dans une multiplication on peut :
\begin{itemize}
    \item Changer les facteurs de place : $4\times3=3\times4=12$
    \item Regrouper les facteurs : $4\times3\times25=4\times25\times3=100\times3=300$
\end{itemize}
On dit que la multiplication est une opération \textbf{commutative}.
\end{Pp}

\section{Multiplier des nombres entiers}

\subsection{Multiplier un nombre par 10, 100 ou 1 000}

\begin{Mt}
Si on multiplie un nombre pas \textbf{10 ; 100 ou 1000}, le chiffre des \textbf{unités} de ce nombre prend une valeur \textbf{10 ; 100 ou 1000} fois plus grande et devient le chiffre des \textbf{dizaines ; centaines ou milliers}.
\end{Mt}

\begin{Ex}
\[\textcolor{red}{2},34\times1000=\textcolor{red}{2}\,340\]
Le chiffre des \textbf{unités} est \textcolor{red}{$2$}, multiplié par $1000$, il devient le chiffre des \textbf{milliers}.
\end{Ex} 

\subsection{Multiplier un nombre par 5, 50 ou 500}

\begin{Mt}
5 est la moitié de 10 ; 50 est la moitié de 100 et 500 est la moitié de 1000, donc pour multiplier un nombre décimal par 5 ; 50 ou 500, on le multiplie par 10 ; 100 ou 1000 puis on calcule la moitié du résultat.\\
\end{Mt}

\begin{Ex}
 \[34,6\times50\rightarrow34,6\times100=3460
 \rightarrow3460\div2=1730\]
\end{Ex}

\subsection{Poser la multiplication de deux nombres entiers}

Lorsque le calcul mental ne permet pas de trouver facilement un produit, on peut poser la multiplication :

\begin{Ex}
\[\opmul[decimalsepsymbol={,},
voperator=bottom,
displayshiftintermediary=all]{8447}{368}\]

Ainsi : \opmul[style=text]{8447}{368}
\end{Ex}

\section{Multiplier des nombres décimaux}

\subsection{Multiplier un nombre par 0,1 , 0,01 ou 0,001}

\begin{Mt}
Si on multiplie un nombre pas \textbf{0,1 ; 0,01 ou 0,001}, le chiffre des \textbf{unités} de ce nombre prend une valeur \textbf{10 ; 100 ou 1000} fois plus petite et devient le chiffre des \textbf{dixièmes ; centièmes ou millièmes}.
\end{Mt}

\begin{Ex}
\[7\textcolor{red}{3},22\times0,1=7,\textcolor{red}{3}22\]
Le chiffre des \textbf{unités} est $\textcolor{red}{3}$, multiplié par $0,1$ il devient le chiffre des \textbf{dixièmes}.
\end{Ex} 

\subsection{Multiplier un nombre par 0,5}

\begin{Mt}
Multiplier un nombre décimal par \textbf{0,5} revient à calculer sa \textbf{moitié}.

On peut soit diviser le nombre par 2 , Soit multiplier le nombre par 5 puis diviser le résultat par 10.
\end{Mt}

\begin{Ex}
Calcule du produit : $A=307\times0,5$
\begin{itemize}
\item Multiplier un nombre par 0,5 revient à le diviser par 2 :
\[307\times0,5=307\div 2=153,5\]
\item Multiplier un nombre par 0,5 revient à le multiplier par 5 et diviser le résultat par 10 :
\[307\times5=1535\hspace{.2cm}et\hspace{.2cm}1535\div10=153,5\]
\end{itemize}
\end{Ex}

\subsection{Multiplier des nombres décimaux}

\begin{Mt}
\begin{itemize}
\item Puisque l'on sait maintenant multiplier un nombre par 0,1 ; 0,01 ou 0,001 nous pouvons utiliser ça pour transformer un nombre décimal en produit d'un nombre entier et de 0,1 ; 0,01 ou 0,001.

\begin{Ex}
$84,41=8441$ centièmes donc :
\[84,41=8441\times0,01\]
\end{Ex}
\item On peut alors calculer des produits simples de nombres décimaux en commençant par convertir en produits de nombre entiers et de 0,1 ; 0,01 ou 0,001.
\begin{Ex}
Calcule du produit : $A=0,6\times0,05$\\
On sait que $0,6=6\times0,1$ et que $0,05=5\times0,01$, donc :
\[0,6\times0,05=6\times0,1\times5\times0,01\]
donc :
\[0,6\times0,05=6\times5\times0,1\times0,01\]
donc :
\[0,6\times0,05=30\times0,001=0,03\]
\end{Ex}
\end{itemize}
\end{Mt}

\begin{Ex}
Cette méthode est très pratique pour calculer des produits de nombres décimaux en utilisant le produit des nombres entiers.

En effet, si on sait que  $416\times372=154752$ on peut déduire le résultat de $4,16\times3,72$ en faisant :
\[4,16\times3,72=416\times
0,01\times372\times0,01\]
donc : \[4,16\times3,72=416\times372\times0,0001\]
donc :
\[4,16\times3,72=154752\times0,0001=15,4752\]
\end{Ex}

\subsection{Poser la multiplication de deux nombres décimaux}

\begin{Mt}
Lorsque le calcul mental ne permet pas de trouver facilement un produit, on peut poser la multiplication :

Pour effectuer la multiplication de deux nombres décimaux : On effectue d'abord la multiplication sans tenir compte des virgules puis on place la virgule dans le produit en comptabilisant le nombre de chiffres à droite de la virgule de chacun des facteurs.
\end{Mt}

\begin{Ex}
Multiplication posée de $35,66\times2,68$. On comptabilise 1 chiffre après la virgule pour le premier facteur et 2 chiffres après la virgule pour le second facteur. Le résultat aura alors 3 chiffres après la virgule.
\[\opmul[decimalsepsymbol={,},
voperator=bottom,
displayshiftintermediary=all]{35,66}{2,68}\]
\end{Ex} 

\section{Ordre de grandeur d'un produit}

\begin{Def}
Un ordre de grandeur d'un nombre est une valeur approchée simple de ce nombre.
\end{Def}

\begin{Ex}
On considère le nombre $a=2931,22$.

Une valeur approchée de $a$ est $3000$, on dit que $3000$ est un ordre de grandeur de $a$ et on note $2931,22\approx3000$.
\end{Ex}

\begin{Rq}
\begin{itemize}
\item  Calculer un ordre de grandeur permet de vérifier la cohérence du résultat d'un calcul.
\item Un ordre de grandeur n'est pas unique.
\end{itemize}
\end{Rq}

\begin{Reg}
Pour obtenir un \textbf{ordre de grandeur} d'un produit, on multiplie un ordre de grandeur de chacun des facteurs.
\end{Reg} 

\begin{Ex}
Un ordre de grandeur de $1,95$ est $2$, un ordre de grandeur de $4,2$ est $4$ ; donc un ordre de grandeur de $1,95\times4,2$ est $2\times4=8$.
\end{Ex} 



\end{pageCours}



\begin{pageParcoursu} 

 


\begin{ExoCu}{Modéliser. Calculer.}{1234}{2}{0}{0}{0}{0}

Intercale un nombre décimal entre $3,451$ et $3,452$.

$\ldots \ldots$



\end{ExoCu}


\begin{ExoCu}{Modéliser. Calculer.}{1234}{2}{0}{0}{0}{0}


Un vase pouvant contenir 3L contient déjà 2L d'eau. Si on ajoute à nouveau 50cL d'eau, l'eau débordera-t-elle ?

\point{3}
 \end{ExoCu}
 
\end{pageParcoursu}



\begin{pageParcoursd} 

 


\begin{ExoCd}{Modéliser. Calculer.}{1234}{2}{0}{0}{0}{0}

Le CDI du collège achète 57 revues à 1,90 euro l'unité. Quel est le prix total de la facture ?

\textit{Le calcul malin propose $57 \times 19 = 1083$}  

\point{3}


\end{ExoCd}


\begin{ExoCd}{Modéliser. Calculer.}{1234}{2}{0}{0}{0}{0}

Le prix d'un kilogramme de fraise coute 6,10 euros. Elisa achète 1,5kg de fraises. Quel est le prix que va payer Elisa au maraicher ?

\point{3}

 
\end{ExoCd}


\begin{ExoCd}{Modéliser. Calculer.}{1234}{2}{0}{0}{0}{0}

La maman d'Assia lui donne 10 euros pour acheter 1,8kg de myrtilles. Le prix d'un kilogramme de myrtilles est de 5,2 euros. La somme dont dispose Assia est-elle suffisante  ?

\point{3}
 
\end{ExoCd}
\begin{ExoCd}{Chercher. Calculer.}{1234}{2}{0}{0}{0}{0}

Un professeur de tennis achète sur Internet 15 raquettes de tennis à 6,80 euros et 24 cerceaux. Il paie au total 176,40 euros. Quel est le prix d'un cerceau  ?

\point{3}
 
\end{ExoCd} 
 

\end{pageParcoursd} 