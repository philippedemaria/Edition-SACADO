%-------------------------------
%	CONTENTS
%-------------------------------
\chapter{Additions et soustractions de nombres décimaux}
{https://sacado.xyz/qcm/parcours_show_course/0/117119}
{

\begin{CpsCol}
 \textbf{Les savoir-faire du parcours}
 \begin{itemize}
 \item Savoir calculer des compléments.
 \item Savoir calculer astucieusement une somme.
 \item Savoir poser une addition avec des nombres entiers.
 \item Savoir poser une addition avec des nombres décimaux.
 \item Savoir poser une soustraction avec des nombres entiers.
 \item Savoir poser une soustraction avec des nombres décimaux.
 \item Savoir compléter une addition à trou.
 \item Savoir déterminer l'ordre de grandeur d'une somme ou d'une différence.
 \item Savoir résoudre un problème numérique.
 \end{itemize}
\end{CpsCol}

}

\begin{pageCours} 

\section{Calculer un complément}

\begin{minipage}{0.6\linewidth}
\begin{Def}
Un \textbf{complément} est un nombre à \textbf{ajouter} pour atteindre un nombre donné.
\end{Def}

\begin{Ex}
Le complément à $314$ pour atteindre $1000$ est $686$
\[314+686=1000\]
\begin{description}
\item[]  $314 + 686 = 1000$
\item[] ou  $314 = 1000 - 686$
\end{description}
\end{Ex}

\end{minipage}
\begin{minipage}{0.4\linewidth}

\definecolor{wqwqwq}{rgb}{0.3764705882352941,0.3764705882352941,0.3764705882352941}
\definecolor{wwqqcc}{rgb}{0.4,0.,0.8}
\definecolor{ffwwqq}{rgb}{1.,0.4,0.}
\begin{tikzpicture}[line cap=round,line join=round,>=triangle 45,x=0.620970746495717cm,y=0.620970746495717cm]
\clip(-1.2872701767628254,1.716286654474951) rectangle (8.37502073445868,5.32161908403523);
\fill[line width=1.pt,color=ffwwqq,fill=ffwwqq,fill opacity=0.4] (-1.,5.) -- (2.,5.) -- (2.,4.) -- (-1.,4.) -- cycle;
\fill[line width=1.pt,color=wwqqcc,fill=wwqqcc,fill opacity=0.2] (2.,5.) -- (8.,5.) -- (8.,4.) -- (2.,4.) -- cycle;
\fill[line width=1.pt,dash pattern=on 1pt off 1pt,color=wqwqwq ,fill opacity=0.2] (-1.,3.) -- (8.,3.) -- (8.,2.) -- (-1.,2.) -- cycle;
\fill[line width=1.pt,color=ffwwqq,fill=ffwwqq,fill opacity=0.4] (-1.,3.) -- (2.,3.) -- (2.,2.) -- (-1.,2.) -- cycle;
\fill[line width=1.pt,color=wwqqcc,fill=wwqqcc,fill opacity=0.2] (2.252373270069218,2.837760040669787) -- (8.252373270069228,2.837760040669787) -- (8.252373270069228,1.837760040669787) -- (2.252373270069218,1.837760040669787) -- cycle;
\draw [line width=1.pt,color=ffwwqq] (-1.,5.)-- (2.,5.);
\draw [line width=1.pt,color=ffwwqq] (2.,5.)-- (2.,4.);
\draw [line width=1.pt,color=ffwwqq] (2.,4.)-- (-1.,4.);
\draw [line width=1.pt,color=ffwwqq] (-1.,4.)-- (-1.,5.);
\draw [line width=1.pt,color=wwqqcc] (2.,5.)-- (8.,5.);
\draw [line width=1.pt,color=wwqqcc] (8.,5.)-- (8.,4.);
\draw [line width=1.pt,color=wwqqcc] (8.,4.)-- (2.,4.);
\draw [line width=1.pt,color=wwqqcc] (2.,4.)-- (2.,5.);
\draw (0.3171027543914916,4.907005854635798) node[anchor=north west] {$314$};
\draw (4.67955499415941,4.907005854635798) node[anchor=north west] {$686$};
\draw [line width=2.pt] (-0.9988435823980044,3.609086179994098)-- (7.996460829354854,3.5910595178462965);
\draw (2.8228087929358745,3.789352801472112) node[anchor=north west] {$1000$};
\draw [line width=1.pt,dash pattern=on 1pt off 1pt,color=wqwqwq] (-1.,3.)-- (8.,3.);
\draw [line width=1.pt,dash pattern=on 1pt off 1pt,color=wqwqwq] (8.,3.)-- (8.,2.);
\draw [line width=1.pt,dash pattern=on 1pt off 1pt,color=wqwqwq] (8.,2.)-- (-1.,2.);
\draw [line width=1.pt,dash pattern=on 1pt off 1pt,color=wqwqwq] (-1.,2.)-- (-1.,3.);
\draw [line width=1.pt,color=ffwwqq] (-1.,3.)-- (2.,3.);
\draw [line width=1.pt,color=ffwwqq] (2.,3.)-- (2.,2.);
\draw [line width=1.pt,color=ffwwqq] (2.,2.)-- (-1.,2.);
\draw [line width=1.pt,color=ffwwqq] (-1.,2.)-- (-1.,3.);
\draw [line width=1.pt,color=wwqqcc] (2.252373270069218,2.837760040669787)-- (8.252373270069228,2.837760040669787);
\draw [line width=1.pt,color=wwqqcc] (8.252373270069228,2.837760040669787)-- (8.252373270069228,1.837760040669787);
\draw [line width=1.pt,color=wwqqcc] (8.252373270069228,1.837760040669787)-- (2.252373270069218,1.837760040669787);
\draw [line width=1.pt,color=wwqqcc] (2.252373270069218,1.837760040669787)-- (2.252373270069218,2.837760040669787);
\draw (0.3171027543914916,4.907005854635798) node[anchor=north west] {$314$};
\draw (0.20894278150468373,2.9601263426732474) node[anchor=north west] {$314$};
\draw (4.67955499415941,4.888979192487997) node[anchor=north west] {$686$};
\draw (4.607448345568205,2.8159130454908365) node[anchor=north west] {$686$};
\end{tikzpicture}


\end{minipage}

\section{L'addition}

\subsection{Vocabulaire et propriétés}

\begin{Def}
Lorsqu'on ajoute deux nombres :
\begin{itemize}
\item On appelle les \textbf{nombres} que l'on ajoute les \textbf{termes} de l'addition.
\item On appelle le \textbf{résultat} d'une addition la \textbf{somme des termes}.
\end{itemize}
\end{Def}

\begin{Prop}
Dans une addition on peut regrouper les termes ou changer les termes de place. On dit que l'addition est une opération \textbf{commutative}.
\end{Prop}

\begin{Mt}
Calculer astucieusement une somme :
\[A=127+73+314\]
On regroupe les termes $127$ et $73$ car leur somme vaut $200$ ainsi :
\[A=127+73+314=200+314=514\]
\end{Mt}

\subsection{Poser l'addition de deux nombres entiers}

\begin{Mt}
Pour effectuer une addition avec des nombres entiers, il faut :
\begin{itemize}
    \item Aligner les chiffres des unités et disposer les chiffres de même rang les uns sous les autres
    \item Commencer les calculs par la droite \textbf{sans oublier les retenues}.
\end{itemize}
\bigskip
\[\opadd[carrystyle=\scriptsize\red,
decimalsepsymbol={,},
voperator=bottom]{3192}{345}\]
\end{Mt} 


\end{pageCours}

\begin{pageAD}

\Sf{Connaitre les complément de nombres}

\Sf{Additionner deux nombres décimaux}

\end{pageAD}

\begin{pageCours}

\subsection{Poser l'addition de deux nombres décimaux}

\begin{Mt}
Pour effectuer une addition avec des nombres décimaux, il faut :
\begin{itemize}
    \item Aligner les virgules et disposer les chiffres de même rang les uns sous les autres
    \item Commencer les calculs par la droite \textbf{sans oublier les retenues}.
\end{itemize}
\bigskip
\[\opadd[carrystyle=\scriptsize\red,
decimalsepsymbol={,},
voperator=bottom]{45.05}{78.4}\]
\end{Mt} 

% \begin{ExOApp}[]
% Effectuer les calculs suivants :
% \[A=45,06+12,2
% \hspace{1cm}
% B=3,455+23,73\]
% \end{ExOApp} 

\section{La soustraction}

\subsection{Vocabulaire et propriétés}

\begin{Def}
Lorsqu'on soustrait deux nombres :
\begin{itemize}
\item On appelle les \textbf{nombres} que l'on ajoute les \textbf{termes} de la soustraction.
\item On appelle le \textbf{résultat} d'une addition la \textbf{différence des termes}.
\end{itemize}
\end{Def}

\begin{Prop}
Dans une soustraction on ne peut pas changer les termes de place. On dit que l'addition \textbf{n'est pas} une opération \textbf{commutative}.
\end{Prop}

\subsection{Poser la soustraction de deux nombres entiers}

\begin{Mt}
Pour effectuer une soustraction avec des nombres entiers on observe les mêmes règles que pour les additions.
\[\opsub[carrystyle=\scriptsize\red,
carrysub,
lastcarry,
columnwidth=2.5ex,
offsetcarry=-0.4,
decimalsepoffset=-3pt,
deletezero=false,
decimalsepsymbol={,},
voperator=bottom]{8644}{2785}\]
\end{Mt}

\subsection{Poser la soustraction de deux nombres décimaux}

\begin{Mt}
Pour effectuer une soustraction avec des nombres décimaux on observe les mêmes règles que pour les additions.
\[\opsub[carrystyle=\scriptsize\red,
carrysub,
lastcarry,
columnwidth=2.5ex,
offsetcarry=-0.4,
decimalsepoffset=-3pt,
deletezero=false,
decimalsepsymbol={,},
voperator=bottom]{60,77}{21,21}\]
\end{Mt}

% \begin{ExOApp}[]
% Effectuer les calculs suivants :
% \[A=52,61-23,73
% \hspace{1cm}
% B=9,034-1,078\]
% \end{ExOApp} 

\end{pageCours}

\begin{pageAD}

\Sf{Connaitre un ordre de grandeur}

\Sf{Additionner deux nombres décimaux}


\end{pageAD}

\begin{pageCours}


\section{Ordre de grandeur d'une somme ou d'une différence}

\begin{DefT}{ordre de grandeur}
Un \textbf{ordre de grandeur}\index{Ordre de grandeur} d'un nombre est une valeur approchée simple de ce nombre/
\end{DefT}

\begin{Ex}
On considère le nombre $a=41,82$. 

Une valeur approchée de $a$ est $40$. On dit que $40$ est un ordre de grandeur de $a$. On note $40\approx41,82$.
\end{Ex}

\begin{Rqs}
\begin{itemize}
\item Calculer un ordre de grandeur permet de vérifier la cohérence du résultat d'un calcul.
\item Un ordre de grandeur n'est pas unique.
\end{itemize}
\end{Rqs}

\begin{Mt}
Utiliser un ordre de grandeur pour retrouver une somme ou une différence :

Lily a posé l'opération : $2619+1496$ mais ne se souvient plus de quel résultat elle a obtenu parmi les 4 suivants :
\[411\;-\;6115\;-\;4115\;-\;41158\]

Pour retrouver le résultaton peut estimer l'ordre de grandeur plutôt que de refaire le calcul. $2619\approx2600$ et $1496\approx1500$ donc un ordre de grandeur de $2619+1496$ est $2600+1500=4100$. Ainsi, $2619+1496=4115$.
\end{Mt}

\section{Résoudre un problème}

\begin{Mt}
\begin{itemize}
\item \textbf{Ordre de grandeur} : Lorsque l'on veut résoudre un problème, il peut être utile de vérifier la cohérence de son résultat en utilisant un ordre de grandeur.
\item Sens des opérations : Une addition est utilisée lorsque l'on veut ajouter des quantités. Une soustraction est utilisée lorsque l'on veut retirer une quantité d'une autre quantité.
\end{itemize}
\end{Mt}

% \begin{ExOApp}[]
% Manon achète $3$ baguettes de pain à $1,50$\euro chacune, une brioche à $5,50$\euro et un gâteau à $19,90$\euro. Manon a $40$\euro. Combien de croissants à $1,50$\euro pièce pourra-t-elle encore s'acheter ?
% \end{ExOApp}


\end{pageCours} 
\begin{pageAD} 
 

\Sf{Connaitre le vocabulaire des opérations}
 
  
\begin{ExoCad}{Communiquer.}{1234}{2}{0}{0}{0}{0}

 
\end{ExoCad}

\Sf{Connaitre les règles de priorités}

\begin{ExoCad}{Calculer.}{1234}{0}{0}{0}{0}{0}

 
 
\end{ExoCad}

\begin{ExoCad}{Calculer.}{1234}{0}{0}{0}{0}{0}

 
\end{ExoCad}


\Sf{Utiliser la distributivité}

\begin{ExoCad}{Représenter. Calculer.}{1234}{0}{0}{0}{0}{0}

 
 
\end{ExoCad}

\begin{ExoCad}{Calculer.}{1234}{0}{0}{0}{0}{0}

 
\end{ExoCad}

 
\end{pageAD}


%%%%%%%%%%%%%%%%%%%%%%%%%%%%%%%%%%%%%%%%%%%%%%%%%%%%%%%%%%%%%%%%%%%
%%%%  Niveau 1
%%%%%%%%%%%%%%%%%%%%%%%%%%%%%%%%%%%%%%%%%%%%%%%%%%%%%%%%%%%%%%%%%%%
\begin{pageParcoursu} 

 %%%%%%%%%%%%%%%%%%%%%%%%%%%
\begin{ExoCu}{Représenter.}{1234}{2}{0}{0}{0}{0}


\end{ExoCu}
%%%%%%%%%%%%%%%%%%%%%%%%%%%
\begin{ExoCu}{Représenter.}{1234}{2}{0}{0}{0}{0}


\end{ExoCu}
%%%%%%%%%%%%%%%%%%%%%%%%%%%
\begin{ExoCu}{Représenter.}{1234}{2}{0}{0}{0}{0}

\end{ExoCu}


%%%%%%%%%%%%%%%%%%%%%%%%%%%
\begin{ExoCu}{Raisonner.}{1234}{2}{0}{0}{0}{0}

\end{ExoCu}

%%%%%%%%%%%%%%%%%%%%%%%%%%%
\begin{ExoCu}{Représenter.}{1234}{2}{0}{0}{0}{0}


\end{ExoCu}


\end{pageParcoursu}

  
%%%%%%%%%%%%%%%%%%%%%%%%%%%%%%%%%%%%%%%%%%%%%%%%%%%%%%%%%%%%%%%%%%%
%%%%  Niveau 2
%%%%%%%%%%%%%%%%%%%%%%%%%%%%%%%%%%%%%%%%%%%%%%%%%%%%%%%%%%%%%%%%%%%



\begin{pageParcoursd} 
 
%%%%%%%%%%%%%%%%%%%%%%%%%%%%%%%%%%%%%%%%%%%%%%%%%%%%%%%%%%%%%%%%%%%
\begin{ExoCd}{Représenter.}{1234}{2}{0}{0}{0}{0}


 
\end{ExoCd}

 
%%%%%%%%%%%%%%%%%%%%%%%%%%%%%%%%%%%%%%%%%%%%%%%%%%%%%%%%%%%%%%%%%%%
\begin{ExoCd}{Chercher.communiquer.}{1234}{2}{0}{0}{0}{0}



\end{ExoCd}


%%%%%%%%%%%%%%%%%%%%%%%%%%%%%%%%%%%%%%%%%%%%%%%%%%%%%%%%%%%%%%%%%%%
\begin{ExoCd}{Représenter. Raisonner.}{1234}{2}{0}{0}{0}{0}


\end{ExoCd}

 %%%%%%%%%%%%%%%%%%%%%%%%%%%%%%%%%%%%%%%%%%%%%%%%%%%%%%%%%%%%%%%%%%%
\begin{ExoCd}{Représenter. Raisonner.}{1234}{2}{0}{0}{0}{0}


\end{ExoCd}
 
%%%%%%%%%%%%%%%%%%%%%%%%%%%%%%%%%%%%%%%%%%%%%%%%%%%%%%%%%%%%%%%%%%%
\begin{ExoCd}{Représenter. Raisonner.}{1234}{2}{0}{0}{0}{0}


\end{ExoCd}
 
\end{pageParcoursd}

%%%%%%%%%%%%%%%%%%%%%%%%%%%%%%%%%%%%%%%%%%%%%%%%%%%%%%%%%%%%%%%%%%%
%%%%  Niveau 3
%%%%%%%%%%%%%%%%%%%%%%%%%%%%%%%%%%%%%%%%%%%%%%%%%%%%%%%%%%%%%%%%%%%
\begin{pageParcourst}

%%%%%%%%%%%%%%%%%%%%%%%%%%%%%%%%%%%%%%%%%%%%%%%%%%%%%%%%%%%%%%%%%%%
\begin{ExoCt}{Représenter.}{1234}{2}{0}{0}{0}{0}

 

\end{ExoCt}

%%%%%%%%%%%%%%%%%%%%%%%%%%%%%%%%%%%%%%%%%%%%%%%%%%%%%%%%%%%%%%%%%%%
\begin{ExoCt}{Représenter. Raisonner.}{1234}{2}{0}{0}{0}{0}
 
 


\end{ExoCt}


%%%%%%%%%%%%%%%%%%%%%%%%%%%%%%%%%%%%%%%%%%%%%%%%%%%%%%%%%%%%%%%%%%%
\begin{ExoCt}{Raisonner.}{1234}{2}{0}{0}{0}{0}
 
\end{ExoCt}

%%%%%%%%%%%%%%%%%%%%%%%%%%%%%%%%%%%%%%%%%%%%%%%%%%%%%%%%%%%%%%%%%%%
\begin{ExoCt}{Représenter.}{1234}{2}{0}{0}{0}{0}

 

\end{ExoCt}

%%%%%%%%%%%%%%%%%%%%%%%%%%%%%%%%%%%%%%%%%%%%%%%%%%%%%%%%%%%%%%%%%%%
\begin{ExoCt}{Représenter.}{1234}{2}{0}{0}{0}{0}

 

\end{ExoCt} 
 
\end{pageParcourst}

%%%%%%%%%%%%%%%%%%%%%%%%%%%%%%%%%%%%%%%%%%%%%%%%%%%%%%%%%%%%%%%%%%%
%%%%  Brouillon
%%%%%%%%%%%%%%%%%%%%%%%%%%%%%%%%%%%%%%%%%%%%%%%%%%%%%%%%%%%%%%%%%%%


\begin{pageBrouillon} 
 
\ligne{32}



\end{pageBrouillon}

%%%%%%%%%%%%%%%%%%%%%%%%%%%%%%%%%%%%%%%%%%%%%%%%%%%%%%%%%%%%%%%%%%%
%%%%  Auto
%%%%%%%%%%%%%%%%%%%%%%%%%%%%%%%%%%%%%%%%%%%%%%%%%%%%%%%%%%%%%%%%%%%


%%%%%%%%%%%%%%%%%%%%%%%%%%%%%%%%%%%%%%%%%%%%%%%%%%%%%%%%%%%%%%%%%%%
\begin{pageAuto} 


\begin{ExoAuto}{Raisonner.}{1234}{2}{0}{0}{0}{0}

 
%%%%%%%%%%%%%%%%%%%%%%%%%%%%%%%%%%%%%%%%%%%%%%%%%%%%%%%%%%%%%%%%%%%
\end{ExoAuto}

\begin{ExoAuto}{Raisonner.}{1234}{2}{0}{0}{0}{0}
  

\end{ExoAuto}

%%%%%%%%%%%%%%%%%%%%%%%%%%%%%%%%%%%%%%%%%%%%%%%%%%%%%%%%%%%%%%%%%%%
\begin{ExoAuto}{Raisonner.}{1234}{2}{0}{0}{0}{0}

 
 

\end{ExoAuto}

 
%%%%%%%%%%%%%%%%%%%%%%%%%%%%%%%%%%%%%%%%%%%%%%%%%%%%%%%%%%%%%%%%%%%
\begin{ExoAuto}{Raisonner.}{1234}{2}{0}{0}{0}{0}

 
 

\end{ExoAuto}


\end{pageAuto}
