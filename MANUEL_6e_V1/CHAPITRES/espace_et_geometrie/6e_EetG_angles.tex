\chapter{Angles}
{https://sacado.xyz/qcm/parcours_show_course/0/117130}
{
\begin{CpsCol}
 \textbf{Les savoir-faire du parcours}
\begin{itemize}
\item Savoir nommer un angle.
\item Estimer si un angle est droit, aigu ou obtus.   
\item Savoir reconnaitre un angle aigu ou obtus.   
\item Utiliser un rapporteur pour mesure un angle.
\item Savoir construire un angle de mesure donnée.
\end{itemize}
\end{CpsCol}
}




\begin{pageCours}

\section{Définition et notation d'un secteur angulaire}

\begin{minipage}{.58\linewidth}

\begin{DefT}{Secteur angulaire}
\begin{itemize}[leftmargin=*]
\item Un \textbf{secteur angulaire}\index{secteur angulaire!Angle} est limité par deux demi-droites de même origine.
\item Le point d'intersection des deux demi-droites est appelé le \textbf{sommet} de l'angle\index{sommet!Angle}.
\item Les deux demi-droites sont appelées les \textbf{côtés}\index{cotes@côté!Angle} du secteur angulaire.
\end{itemize}
Un secteur angulaire est communément appelé un \textbf{angle}\index{Angle}.
\end{DefT}

\begin{Nt}
L'angle représenté se note $\widehat{BAC}$ ou $\widehat{CAB}$.
\end{Nt}
\end{minipage}
\begin{minipage}{.38\linewidth}


\begin{center}
\begin{tikzpicture}[line cap=round,line join=round,>=triangle 45,x=.8cm,y=.8cm]
\clip(-.5,-3) rectangle (5,4);
\draw [shift={(0.,0.)},line width=1.4pt,color=xdxdff,fill=xdxdff,fill opacity=0.10000000149011612] (0,0) -- (-18.434948822921967:1.0340709101622498) arc (-18.434948822921967:63.434948822922024:1.0340709101622498) -- cycle;
\draw [line width=1.pt,domain=0.0:4.993389700427471] plot(\x,{(-0.--2.*\x)/1.});
\draw [line width=1.pt,domain=0.0:4.993389700427471] plot(\x,{(-0.-1.*\x)/3.});
\draw[color=qqqqff] (-0.42809635713746785,0.25731172259058155) node {$A$};
\draw [color=qqqqff] (1.,2.)-- ++(-2.0pt,-2.0pt) -- ++(4.0pt,4.0pt) ++(-4.0pt,0) -- ++(4.0pt,-4.0pt);
\draw[color=qqqqff] (0.9605131507946963,2.547040166521276) node {$B$};
\draw [color=qqqqff] (3.,-1.)-- ++(-2.0pt,0 pt) -- ++(4.0pt,0 pt) ++(-2.0pt,-2.0pt) -- ++(0 pt,4.0pt);
\draw[color=qqqqff] (3.265014036299139,-0.7176694212766819) node {$C$};
\end{tikzpicture}
\end{center}

\end{minipage}

\section{Comparer des angles}

 

\begin{DefT}{Caractérisation des angles}
\begin{minipage}[c]{.3\linewidth}
\begin{center}
\begin{tikzpicture}[line cap=round,line join=round,>=triangle 45,x=.6cm,y=.6cm]
\clip(-2,-.5) rectangle (5,4);
\draw [shift={(0.,0.)},line width=1.4pt,color=xdxdff,fill=xdxdff,fill opacity=0.10000000149011612] (0,0) -- (0.:1.0340709101622498) arc (0.:48.48908304625411:1.0340709101622498) -- cycle;
\draw [line width=1.pt,domain=0.0:9.425122172551399] plot(\x,{(-0.--1.7862594254733355*\x)/1.5809556968920464});
\draw [line width=1.pt,domain=0.0:9.425122172551399] plot(\x,{(-0.-0.*\x)/3.});
\draw[color=qqqqff] (-0.42809635713746785,0.25731172259058155) node {$A$};
\draw [color=qqqqff] (1.5809556968920464,1.7862594254733355)-- ++(-2.0pt,-2.0pt) -- ++(4.0pt,4.0pt) ++(-4.0pt,0) -- ++(4.0pt,-4.0pt);
\draw[color=qqqqff] (1.536638372170807,2.3402259844888262) node {$B$};
\draw [color=qqqqff] (3.,0.)-- ++(-2.0pt,0 pt) -- ++(4.0pt,0 pt) ++(-2.0pt,-2.0pt) -- ++(0 pt,4.0pt);
\draw[color=qqqqff] (3.265014036299139,0.28685660573807437) node {$C$};
\end{tikzpicture}

\vspace{.2cm}
Un angle \textbf{aigu}\index{aigu!Angle} est un angle plus petit qu'un angle droit.
\end{center}
\end{minipage}
\hfill % espace horizontal
\begin{minipage}[c]{.3\linewidth}
\begin{center}
\begin{tikzpicture}[line cap=round,line join=round,>=triangle 45,x=.6cm,y=.6cm]
\clip(-2,-.5) rectangle (5,4);
\draw[line width=1.4pt,color=xdxdff,fill=xdxdff,fill opacity=0.10000000149011612] (0.731198552803472,0.) -- (0.7311985528034722,0.7311985528034719) -- (0.,0.731198552803472) -- (0.,0.) -- cycle; 
\draw [line width=1.pt] (0.,0.) -- (0.,6.01117771556481);
\draw [line width=1.pt,domain=0.0:9.425122172551399] plot(\x,{(-0.-0.*\x)/3.});
\draw[color=qqqqff] (-0.42809635713746785,0.25731172259058155) node {$A$};
\draw [color=qqqqff] (0.,2.)-- ++(-2.0pt,0 pt) -- ++(4.0pt,0 pt);% ++(-2.0pt,-2.0pt) -- ++(0 pt,4.0pt);
\draw[color=qqqqff] (-.5,2.547040166521276) node {$B$};
\draw [color=qqqqff] (3.,0.)-- ++(-2.0pt,0 pt) -- ++(4.0pt,0 pt) ++(-2.0pt,-2.0pt) -- ++(0 pt,4.0pt);
\draw[color=qqqqff] (3.265014036299139,0.28685660573807437) node {$C$};
\end{tikzpicture}

\vspace{.2cm}
Un angle \textbf{droit}\index{droit!Angle}
\end{center}
\end{minipage}
\hfill % espace horizontal
\begin{minipage}[c]{.3\linewidth}
\begin{center}
\begin{tikzpicture}[line cap=round,line join=round,>=triangle 45,x=.6cm,y=.6cm]
\clip(-2,-.5) rectangle (5,3);
\draw [shift={(0.,0.)},line width=1.pt,color=xdxdff,fill=xdxdff,fill opacity=0.10000000149011612] (0,0) -- (0.:1.0340709101622498) arc (0.:122.60780771993535:1.0340709101622498) -- cycle;
\draw [line width=1.pt,domain=-8.360897482239299:0.0] plot(\x,{(-0.--2.5396539457344027*\x)/-1.6246641246109286});
\draw [line width=1.pt,domain=0.0:8.425122172551399] plot(\x,{(-0.-0.*\x)/3.});
\draw[color=qqqqff] (-0.42809635713746785,0.25731172259058155) node {$A$};
\draw [color=qqqqff] (-1.6246641246109286,2.5396539457344027)-- ++(-2.0pt,-2.0pt) -- ++(4.0pt,4.0pt) ++(-4.0pt,0) -- ++(4.0pt,-4.0pt);
\draw[color=qqqqff] (-1.368981449332168,2.693620504749893) node {$B$};
\draw [color=qqqqff] (3.,0.)-- ++(-2.0pt,0 pt) -- ++(4.0pt,0 pt) ++(-2.0pt,-2.0pt) -- ++(0 pt,4.0pt);
\draw[color=qqqqff] (3.265014036299139,0.28685660573807437) node {$C$};
\end{tikzpicture}

\vspace{.2cm}
Un angle \textbf{obtus}\index{obtus!Angle} est un angle plus grand qu'un angle droit.
\end{center}
\end{minipage}
\end{DefT}
\end{pageCours}


\begin{pageAD} 

\Sf{Savoir nommer un angle}

\ExoCad{Chercher. Communiquer.}

\begin{minipage}{0.5\linewidth}
On propose ci-contre une figure.
\begin{enumerate}
\item Colorie en bleu les angles $\widehat{ABC}$ et  $\widehat{GAB}$
\item Colorie en rouge les angles $\widehat{BCD}$ et  $\widehat{FED}$
\item Colorie en vert les angles $\widehat{EFG}$ et  $\widehat{EDC}$
\end{enumerate}

\end{minipage}
\begin{minipage}{0.5\linewidth}

\definecolor{zzttqq}{rgb}{0.6,0.2,0.}
\definecolor{uququq}{rgb}{0.25098039215686274,0.25098039215686274,0.25098039215686274}
\definecolor{sqsqsq}{rgb}{0.12549019607843137,0.12549019607843137,0.12549019607843137}
\begin{tikzpicture}[line cap=round,line join=round,>=triangle 45,x=1.0cm,y=1.0cm]
\clip(-1.76,0.02) rectangle (3.94,5.1);
\draw [shift={(-1.36,1.52)},line width=1.pt,color=uququq] (0,0) -- (-24.85158585971217:0.6) arc (-24.85158585971217:88.49256424122504:0.6) -- cycle;
\draw [shift={(-1.3,3.8)},line width=1.pt,color=uququq] (0,0) -- (-91.50743575877497:0.6) arc (-91.50743575877497:24.35300917502964:0.6) -- cycle;
\draw [shift={(0.6,4.66)},line width=1.pt,color=uququq] (0,0) -- (-155.64699082497037:0.6) arc (-155.64699082497037:-14.43206293613592:0.6) -- cycle;
\draw [shift={(1.26,2.7)},line width=1.pt,color=uququq] (0,0) -- (31.45206101199951:0.6) arc (31.45206101199951:329.84545208208385:0.6) -- cycle;
\draw [shift={(0.54,0.64)},line width=1.pt,color=uququq] (0,0) -- (16.587338556927413:0.6) arc (16.587338556927413:155.14841414028785:0.6) -- cycle;
\draw [shift={(3.32,3.96)},line width=1.pt,color=uququq] (0,0) -- (165.56793706386406:0.6) arc (165.56793706386406:211.45206101199952:0.6) -- cycle;
\draw [shift={(3.36,1.48)},line width=1.pt,color=uququq] (0,0) -- (149.84545208208382:0.6) arc (149.84545208208382:196.58733855692742:0.6) -- cycle;
\draw [line width=1.pt,color=uququq] (-1.3,3.8)-- (0.6,4.66);
\draw [line width=1.pt,color=uququq] (0.6,4.66)-- (3.32,3.96);
\draw [line width=1.pt,color=uququq] (3.32,3.96)-- (1.26,2.7);
\draw [line width=1.pt,color=uququq] (1.26,2.7)-- (3.36,1.48);
\draw [line width=1.pt,color=uququq] (3.36,1.48)-- (0.54,0.64);
\draw [line width=1.pt,color=uququq] (0.54,0.64)-- (-1.36,1.52);
\draw [line width=1.pt,color=uququq] (-1.36,1.52)-- (-1.3,3.8);
\begin{scriptsize}
\draw [color=sqsqsq] (-1.3,3.8)-- ++(-2.5pt,0 pt) -- ++(5.0pt,0 pt) ++(-2.5pt,-2.5pt) -- ++(0 pt,5.0pt);
\draw[color=sqsqsq] (-1.6,4.11) node {$A$};
\draw [color=sqsqsq] (0.6,4.66)-- ++(-2.5pt,0 pt) -- ++(5.0pt,0 pt) ++(-2.5pt,-2.5pt) -- ++(0 pt,5.0pt);
\draw[color=sqsqsq] (0.74,4.83) node {$B$};
\draw [color=sqsqsq] (3.32,3.96)-- ++(-2.5pt,0 pt) -- ++(5.0pt,0 pt) ++(-2.5pt,-2.5pt) -- ++(0 pt,5.0pt);
\draw[color=sqsqsq] (3.5,4.33) node {$C$};
\draw [color=sqsqsq] (1.26,2.7)-- ++(-2.5pt,0 pt) -- ++(5.0pt,0 pt) ++(-2.5pt,-2.5pt) -- ++(0 pt,5.0pt);
\draw[color=sqsqsq] (1.6,2.75) node {$D$};
\draw [color=sqsqsq] (3.36,1.48)-- ++(-2.5pt,0 pt) -- ++(5.0pt,0 pt) ++(-2.5pt,-2.5pt) -- ++(0 pt,5.0pt);
\draw[color=sqsqsq] (3.5,1.85) node {$E$};
\draw [color=sqsqsq] (0.54,0.64)-- ++(-2.5pt,0 pt) -- ++(5.0pt,0 pt) ++(-2.5pt,-2.5pt) -- ++(0 pt,5.0pt);
\draw[color=sqsqsq] (0.48,0.19) node {$F$};
\draw [color=uququq] (-1.36,1.52)-- ++(-2.5pt,0 pt) -- ++(5.0pt,0 pt) ++(-2.5pt,-2.5pt) -- ++(0 pt,5.0pt);
\draw[color=uququq] (-1.62,1.79) node {$G$};
\end{scriptsize}
\end{tikzpicture}

\end{minipage}


\ExoCad{Chercher. Communiquer.}

\begin{minipage}{0.6\linewidth}
On propose ci-contre une figure.
\begin{enumerate}
\item L'angle colorié en bleu se nomme $\ldots\ldots\ldots$.\vspace{0.3cm}
\item L'angle colorié en vert se nomme $\ldots\ldots\ldots$.\vspace{0.3cm}
\item L'angle colorié en bleu se nomme $\ldots\ldots\ldots$.\vspace{0.3cm}
\end{enumerate}

\end{minipage}
\begin{minipage}{0.4\linewidth}

\definecolor{qqwwzz}{rgb}{0.,0.4,0.6}
\definecolor{ffqqqq}{rgb}{1.,0.,0.}
\definecolor{uququq}{rgb}{0.25098039215686274,0.25098039215686274,0.25098039215686274}
\definecolor{qqzzqq}{rgb}{0.,0.6,0.}
\definecolor{sqsqsq}{rgb}{0.12549019607843137,0.12549019607843137,0.12549019607843137}
\begin{tikzpicture}[line cap=round,line join=round,>=triangle 45,x=1.0cm,y=1.0cm]
\clip(-1.04,0.74) rectangle (5.18,4.54);
\draw [shift={(4.08,1.3)},line width=1.pt,color=qqzzqq,fill=qqzzqq,fill opacity=0.30000001192092896] (0,0) -- (105.94539590092288:0.6) arc (105.94539590092288:180.8069294551024:0.6) -- cycle;
\draw [shift={(3.32,3.96)},line width=1.pt,color=uququq,fill=uququq,fill opacity=0.10000000149011612] (0,0) -- (178.06506769044725:0.6) arc (178.06506769044725:285.9453959009229:0.6) -- cycle;
\draw [shift={(0.36,4.06)},line width=1.pt,color=ffqqqq,fill=ffqqqq,fill opacity=0.44999998807907104] (0,0) -- (-100.84030545433058:0.6) arc (-100.84030545433058:-1.9349323095527693:0.6) -- cycle;
\draw [shift={(-0.18,1.24)},line width=1.pt,color=qqwwzz,fill=qqwwzz,fill opacity=0.44999998807907104] (0,0) -- (0.8069294551023751:0.6) arc (0.8069294551023751:79.15969454566942:0.6) -- cycle;
\draw [line width=1.pt] (0.36,4.06)-- (-0.18,1.24);
\draw [line width=1.pt] (-0.18,1.24)-- (4.08,1.3);
\draw [line width=1.pt] (4.08,1.3)-- (3.32,3.96);
\draw [line width=1.pt] (3.32,3.96)-- (0.36,4.06);
\begin{scriptsize}
\draw [color=sqsqsq] (0.36,4.06)-- ++(-2.5pt,0 pt) -- ++(5.0pt,0 pt) ++(-2.5pt,-2.5pt) -- ++(0 pt,5.0pt);
\draw[color=sqsqsq] (0.14,4.33) node {$B$};
\draw [color=sqsqsq] (3.32,3.96)-- ++(-2.5pt,0 pt) -- ++(5.0pt,0 pt) ++(-2.5pt,-2.5pt) -- ++(0 pt,5.0pt);
\draw[color=sqsqsq] (3.46,4.33) node {$C$};
\draw [color=sqsqsq] (4.08,1.3)-- ++(-2.5pt,0 pt) -- ++(5.0pt,0 pt) ++(-2.5pt,-2.5pt) -- ++(0 pt,5.0pt);
\draw[color=sqsqsq] (4.42,1.35) node {$D$};
\draw [color=sqsqsq] (-0.18,1.24)-- ++(-2.5pt,0 pt) -- ++(5.0pt,0 pt) ++(-2.5pt,-2.5pt) -- ++(0 pt,5.0pt);
\draw[color=sqsqsq] (-0.64,1.05) node {$E$};
\end{scriptsize}
\end{tikzpicture} 

\end{minipage}


\vspace{0.4cm}

\Sf{Estimer si un angle est droit, aigu ou obtus}

\vspace{0.4cm}
 
 
 \ExoCad{Chercher. Communiquer.}

\begin{minipage}{0.6\linewidth}
On propose ci-contre un polygone.
\begin{enumerate}
\item Cite 2 angles aigus et colorie-les en bleu: $\ldots\ldots \ldots$, $\ldots\ldots \ldots$\vspace{0.3cm} 
\item Cite 3 angles obtus et colorie-les en rouge : $\ldots\ldots \ldots$, $\ldots\ldots \ldots  $ , $\ldots \ldots \ldots$\vspace{0.3cm}
\item Cite un angle droit : $\ldots \ldots \ldots$
\end{enumerate}

\end{minipage}
\begin{minipage}{0.4\linewidth}

\definecolor{sqsqsq}{rgb}{0.12549019607843137,0.12549019607843137,0.12549019607843137}
\begin{tikzpicture}[line cap=round,line join=round,>=triangle 45,x=0.7832898172323756cm,y=0.7832898172323756cm]
\clip(-0.7,0.76) rectangle (6.96,5.08);
\draw [line width=1.pt] (0.14,4.44)-- (0.14,1.44);
\draw [line width=1.pt] (3.14,4.44)-- (0.14,4.44);
\draw [line width=1.pt] (0.14,1.44)-- (2.58,2.9);
\draw [line width=1.pt] (2.58,2.9)-- (6.18,1.48);
\draw [line width=1.pt] (6.18,1.48)-- (5.14,3.44);
\draw [line width=1.pt] (5.14,3.44)-- (3.14,4.44);
\begin{scriptsize}
\draw [color=sqsqsq] (0.14,4.44)-- ++(-2.5pt,0 pt) -- ++(5.0pt,0 pt) ++(-2.5pt,-2.5pt) -- ++(0 pt,5.0pt);
\draw[color=sqsqsq] (-0.08,4.71) node {$B$};
\draw [color=sqsqsq] (3.14,4.44)-- ++(-2.5pt,0 pt) -- ++(5.0pt,0 pt) ++(-2.5pt,-2.5pt) -- ++(0 pt,5.0pt);
\draw[color=sqsqsq] (3.32,4.75) node {$C$};
\draw [color=sqsqsq] (0.14,1.44)-- ++(-2.5pt,0 pt) -- ++(5.0pt,0 pt) ++(-2.5pt,-2.5pt) -- ++(0 pt,5.0pt);
\draw[color=sqsqsq] (-0.28,1.33) node {$E$};
\draw [color=sqsqsq] (2.58,2.9)-- ++(-2.5pt,0 pt) -- ++(5.0pt,0 pt) ++(-2.5pt,-2.5pt) -- ++(0 pt,5.0pt);
\draw[color=sqsqsq] (2.72,3.27) node {$A$};
\draw [color=sqsqsq] (6.18,1.48)-- ++(-2.5pt,0 pt) -- ++(5.0pt,0 pt) ++(-2.5pt,-2.5pt) -- ++(0 pt,5.0pt);
\draw[color=sqsqsq] (6.26,1.17) node {$D$};
\draw [color=sqsqsq] (5.14,3.44)-- ++(-2.5pt,0 pt) -- ++(5.0pt,0 pt) ++(-2.5pt,-2.5pt) -- ++(0 pt,5.0pt);
\draw[color=sqsqsq] (5.28,3.81) node {$F$};
\end{scriptsize}
\end{tikzpicture}
\end{minipage}
 
 
\ExoCad{Représenter. Communiquer} 

\begin{minipage}{0.5\linewidth}
Construis un angle aigu $\widehat{ABC}$
\end{minipage}
\begin{minipage}{0.5\linewidth}
Construis un angle droit $\widehat{RUE}$
\end{minipage}

 
\end{pageAD}

\begin{pageCours}
\section{Unité de mesures d'angles : Le degré}

\begin{DefT}{Degré}
L'unité de mesure des angles est le \textbf{degré}\index{Degré}, on le note °.
\end{DefT}

 

\begin{DefT}{Mesure des angles}\index{Mesure|Angle}
\begin{minipage}{0.5\linewidth}
 Un angle aigu  mesure moins de $90$°.\\
 Un angle droit  mesure $90$°.

\end{minipage}
\begin{minipage}{0.5\linewidth}
Un angle obtus mesure entre $90$° et $180$°.\\
Un angle nul mesure $0$°.
\end{minipage}
\end{DefT}
 

\section{Mesurer des angles}

\begin{Mt}
\mini{.70\linewidth}{
\begin{enumerate}
    \item On place le centre du rapporteur sur le sommet de l'angle.
    \item Le « 0° » du rapporteur repose sur une extrémité de l'angle : la demi-droite [AC)
    \item Les flèches du rapporteur recouvrent l'angle.
    \item La mesure de l'angle se lit sur l'autre extrémité de l'angle : la demi-droite [AB)
\end{enumerate}}
{.26\linewidth}{
\begin{center}
\begin{tikzpicture}[line cap=round,line join=round,>=triangle 45,x=.8cm,y=.8cm]
\clip(-.5,-.5) rectangle (4.,4);
\draw [shift={(0.,0.)},line width=1pt,color=xdxdff,fill=xdxdff,fill opacity=0.10000000149011612] (0,0) -- (0.:1.0340709101622498) arc (0.:70.03348028439399:1.0340709101622498) -- cycle;
\draw [line width=1.pt,domain=0.0:9.425122172551399] plot(\x,{(-0.--1.8305767501945749*\x)/0.6650643193197678});
\draw [line width=1.pt,domain=0.0:9.425122172551399] plot(\x,{(-0.-0.*\x)/3.});
\draw [color=qqqqff] (0.,0.)-- ++(-2.0pt,-2.0pt) -- ++(4.0pt,4.0pt) ++(-4.0pt,0) -- ++(4.0pt,-4.0pt);
\draw[color=qqqqff] (-0.42809635713746785,0.25731172259058155) node {$A$};
\draw [color=qqqqff] (0.6650643193197678,1.8305767501945749)-- ++(-2.0pt,0 pt) -- ++(4.0pt,0 pt) ++(-2.0pt,-2.0pt) -- ++(0 pt,4.0pt);
\draw[color=qqqqff] (0.6207469945985286,2.3845433092100654) node {$B$};
\draw [color=qqqqff] (3.,0.)-- ++(-2.0pt,0 pt) -- ++(4.0pt,0 pt) ++(-2.0pt,-2.0pt) -- ++(0 pt,4.0pt);
\draw[color=qqqqff] (3.265014036299139,0.28685660573807437) node {$C$};
\draw[color=xdxdff] (1.3411895406958786,0.6118503203604956) node {$70\textrm{\degre}$};
\end{tikzpicture}
\end{center}}
\end{Mt}


\section{Construire des angles}

\begin{Mt}
Pour construire un angle $\widehat{ABC}$ de 68°
\begin{enumerate}
    \item On commence par tracer une demi-droite [BA)
    \item On place le rapporteur en alignant le « 0° » du rapporteur avec la demi-droite [BA).
    \item On marque l'angle à 68° sur la feuille.
    \item On trace la demi-droite [BC) partant de B et passant par la marque faite à 68°.
\end{enumerate}
 
\end{Mt}

\end{pageCours}
 
 
\begin{pageAD} 
 

\Sf{Utiliser un rapporteur pour mesurer un angle en degrés}


\ExoCad{Représenter.} 

Mesure chaque angle à l'aide du rapporteur.

\begin{minipage}{0.33\linewidth}

\definecolor{uququq}{rgb}{0.25098039215686274,0.25098039215686274,0.25098039215686274}
\definecolor{sqsqsq}{rgb}{0.12549019607843137,0.12549019607843137,0.12549019607843137}
\begin{tikzpicture}[line cap=round,line join=round,>=triangle 45,x=0.8955223880597011cm,y=0.8955223880597011cm]
\clip(-0.6,1.06) rectangle (6.1,4.52);
\draw [shift={(2.56,1.9)},line width=1.pt,color=uququq,fill=uququq,fill opacity=0.10000000149011612] (0,0) -- (7.926926682689598:0.6) arc (7.926926682689598:146.30993247402023:0.6) -- cycle;
\draw [line width=1.pt,domain=2.56:6.100000000000003] plot(\x,{(--4.8776--0.44*\x)/3.16});
\draw [line width=1.pt,domain=-0.5999999999999998:2.56] plot(\x,{(-9.9544--1.84*\x)/-2.76});
\begin{scriptsize}
\draw [color=sqsqsq] (-0.2,3.74)-- ++(-2.5pt,0 pt) -- ++(5.0pt,0 pt) ++(-2.5pt,-2.5pt) -- ++(0 pt,5.0pt);
\draw[color=sqsqsq] (-0.46,3.57) node {$R$};
\draw [color=sqsqsq] (2.56,1.9)-- ++(-2.5pt,0 pt) -- ++(5.0pt,0 pt) ++(-2.5pt,-2.5pt) -- ++(0 pt,5.0pt);
\draw[color=sqsqsq] (2.28,1.65) node {$O$};
\draw [color=sqsqsq] (5.72,2.34)-- ++(-2.5pt,0 pt) -- ++(5.0pt,0 pt) ++(-2.5pt,-2.5pt) -- ++(0 pt,5.0pt);
\draw[color=sqsqsq] (5.76,2.05) node {$I$};
\end{scriptsize}
\end{tikzpicture}

$\widehat{ROI}= \ldots\ldots$
\end{minipage}
\begin{minipage}{0.33\linewidth}

\definecolor{uququq}{rgb}{0.25098039215686274,0.25098039215686274,0.25098039215686274}
\definecolor{sqsqsq}{rgb}{0.12549019607843137,0.12549019607843137,0.12549019607843137}
\begin{tikzpicture}[line cap=round,line join=round,>=triangle 45,x=1.0cm,y=1.0cm]
\clip(-1.38,0.96) rectangle (4.5,4.68);
\draw [shift={(1.2,1.76)},line width=1.pt,color=uququq,fill=uququq,fill opacity=0.10000000149011612] (0,0) -- (46.208592432026855:0.6) arc (46.208592432026855:125.26301172959727:0.6) -- cycle;
\draw [line width=1.pt,domain=1.2:4.500000000000003] plot(\x,{(--1.1792--2.42*\x)/2.32});
\draw [line width=1.pt,domain=-1.3800000000000001:1.2] plot(\x,{(-4.84--1.98*\x)/-1.4});
\begin{scriptsize}
\draw [color=sqsqsq] (-0.2,3.74)-- ++(-2.5pt,0 pt) -- ++(5.0pt,0 pt) ++(-2.5pt,-2.5pt) -- ++(0 pt,5.0pt);
\draw[color=sqsqsq] (-0.46,3.57) node {$T$};
\draw [color=sqsqsq] (1.2,1.76)-- ++(-2.5pt,0 pt) -- ++(5.0pt,0 pt) ++(-2.5pt,-2.5pt) -- ++(0 pt,5.0pt);
\draw[color=sqsqsq] (0.92,1.51) node {$O$};
\draw [color=sqsqsq] (3.52,4.18)-- ++(-2.5pt,0 pt) -- ++(5.0pt,0 pt) ++(-2.5pt,-2.5pt) -- ++(0 pt,5.0pt);
\draw[color=sqsqsq] (3.56,3.89) node {$N$};
\end{scriptsize}
\end{tikzpicture}

$\widehat{TON}= \ldots\ldots$
\end{minipage}
\begin{minipage}{0.33\linewidth}

\definecolor{uququq}{rgb}{0.25098039215686274,0.25098039215686274,0.25098039215686274}
\definecolor{sqsqsq}{rgb}{0.12549019607843137,0.12549019607843137,0.12549019607843137}
\begin{tikzpicture}[line cap=round,line join=round,>=triangle 45,x=1.0cm,y=1.0cm]
\clip(-1.94,1.58) rectangle (3.78,4.96);
\draw [shift={(1.92,2.14)},line width=1.pt,color=uququq,fill=uququq,fill opacity=0.10000000149011612] (0,0) -- (54.92624550665169:0.6) arc (54.92624550665169:172.38145351243108:0.6) -- cycle;
\draw [line width=1.pt,domain=1.92:3.7800000000000025] plot(\x,{(-0.7848--1.88*\x)/1.32});
\draw [line width=1.pt,domain=-1.9400000000000004:1.92] plot(\x,{(-7.526--0.42*\x)/-3.14});
\begin{scriptsize}
\draw [color=sqsqsq] (-1.22,2.56)-- ++(-2.5pt,0 pt) -- ++(5.0pt,0 pt) ++(-2.5pt,-2.5pt) -- ++(0 pt,5.0pt);
\draw[color=sqsqsq] (-1.48,2.39) node {$C$};
\draw [color=sqsqsq] (1.92,2.14)-- ++(-2.5pt,0 pt) -- ++(5.0pt,0 pt) ++(-2.5pt,-2.5pt) -- ++(0 pt,5.0pt);
\draw[color=sqsqsq] (1.64,1.89) node {$R$};
\draw [color=sqsqsq] (3.24,4.02)-- ++(-2.5pt,0 pt) -- ++(5.0pt,0 pt) ++(-2.5pt,-2.5pt) -- ++(0 pt,5.0pt);
\draw[color=sqsqsq] (3.28,3.73) node {$U$};
\end{scriptsize}
\end{tikzpicture}

$\widehat{CRU}= \ldots\ldots$
\end{minipage}


\ExoCad{Représenter. Calculer.} 

\begin{minipage}{0.5\linewidth}

Mesure les angles du triangle ci-contre :  
\begin{itemize}
\item $\widehat{TOI} = \ldots\ldots$
\item $\widehat{OIT} = \ldots\ldots$
\item $\widehat{ITO} = \ldots\ldots$
\end{itemize}
\end{minipage}
\begin{minipage}{0.5\linewidth}

\definecolor{uququq}{rgb}{0.25098039215686274,0.25098039215686274,0.25098039215686274}
\begin{tikzpicture}[line cap=round,line join=round,>=triangle 45,x=0.8cm,y=0.8cm]
\clip(0.64,0.28) rectangle (7.4,5.74);
\draw [line width=1.pt,color=uququq] (1.,5.)-- (7.,3.);
\draw [line width=1.pt,color=uququq] (7.,3.)-- (1.64,0.66);
\draw [line width=1.pt,color=uququq] (1.64,0.66)-- (1.,5.);
\begin{scriptsize}
\draw [color=uququq] (1.,5.)-- ++(-2.5pt,0 pt) -- ++(5.0pt,0 pt) ++(-2.5pt,-2.5pt) -- ++(0 pt,5.0pt);
\draw[color=uququq] (1.14,5.37) node {$A$};
\draw [color=uququq] (7.,3.)-- ++(-2.5pt,0 pt) -- ++(5.0pt,0 pt) ++(-2.5pt,-2.5pt) -- ++(0 pt,5.0pt);
\draw[color=uququq] (7.14,3.37) node {$B$};
\draw [color=uququq] (1.64,0.66)-- ++(-2.5pt,0 pt) -- ++(5.0pt,0 pt) ++(-2.5pt,-2.5pt) -- ++(0 pt,5.0pt);
\draw[color=uququq] (1.92,0.61) node {$C$};
\end{scriptsize}
\end{tikzpicture}
\end{minipage}






  
 

\Sf{Construire à l'aide du rapporteur, un angle de mesure donnée en degrés}
 
\ExoCad{Représenter.} 

A l'aide du rapporteur, construis un angle de mesure donnée.

\begin{minipage}{0.32\linewidth}
\begin{center}
$\widehat{SUR}= 36$°
\end{center}
\vspace{5cm}
 

\end{minipage}
\vrule
\begin{minipage}{0.32\linewidth}

\begin{center}
$\widehat{CAR}= 120$°
\end{center}
  \vspace{5cm}
\end{minipage}  
\vrule
\begin{minipage}{0.32\linewidth}
 
 
\begin{center}
$\widehat{SIR}= 90$°
\end{center}\vspace{5cm}
\end{minipage}
 
\end{pageAD}


%%%%%%%%%%%%%%%%%%%%%%%%%%%%%%%%%%%%%%%%%%%%%%%%%%%%%%%%%%%%%%%%%%%
%%%%  Niveau 1
%%%%%%%%%%%%%%%%%%%%%%%%%%%%%%%%%%%%%%%%%%%%%%%%%%%%%%%%%%%%%%%%%%%
\begin{pageParcoursu} 

\ExoCu{Chercher. Communiquer.}

On considère le rectangle $ABCD$ de centre O. Nomme les 3 angles codés sur la figure et donne sa mesure. 

\begin{minipage}{0.5\linewidth}

 
\definecolor{ffqqqq}{rgb}{1.,0.,0.}
\definecolor{qqwwtt}{rgb}{0.,0.4,0.2}
\definecolor{qqqqff}{rgb}{0.,0.,1.}
\definecolor{uuuuuu}{rgb}{0.26666666666666666,0.26666666666666666,0.26666666666666666}
\definecolor{sqsqsq}{rgb}{0.12549019607843137,0.12549019607843137,0.12549019607843137}
\begin{tikzpicture}[line cap=round,line join=round,>=triangle 45,x=1.0cm,y=1.0cm]
\clip(-1.64,1.34) rectangle (4.14,5.66);
\draw [shift={(1.2447058823529418,3.478823529411765)},line width=1.pt,color=qqqqff] (0,0) -- (-40.11391319135916:0.6) arc (-40.11391319135916:12.041426255506181:0.6) -- cycle;
\draw [shift={(3.489411764705882,3.9576470588235293)},line width=1.pt,color=qqwwtt] (0,0) -- (165.96375653207352:0.9) arc (165.96375653207352:192.04142625550617:0.9) -- cycle;
\draw [shift={(-1.,3.)},line width=1.pt,color=ffqqqq] (0,0) -- (12.041426255506192:0.6) arc (12.041426255506192:75.96375653207352:0.6) -- cycle;
\draw [line width=1.pt] (-1.,3.)-- (3.,2.);
\draw [line width=1.pt] (-0.5105882352941176,4.95764705882353)-- (-1.,3.);
\draw [line width=1.pt] (-0.5105882352941176,4.95764705882353)-- (3.489411764705882,3.9576470588235293);
\draw [line width=1.pt] (3.489411764705882,3.9576470588235293)-- (3.,2.);
\draw [line width=1.pt] (-0.5105882352941176,4.95764705882353)-- (3.,2.);
\draw [line width=1.pt] (-1.,3.)-- (3.489411764705882,3.9576470588235293);
\draw [shift={(1.2447058823529418,3.478823529411765)},line width=1.pt,color=qqqqff] (-40.11391319135916:0.6) arc (-40.11391319135916:12.041426255506181:0.6);
\draw[line width=1.pt,color=qqqqff] (1.7634967179895455,3.3983184434887335) -- (1.9117226710285753,3.3753169903678675);
\draw[line width=1.pt,color=qqqqff] (1.7403472483490192,3.305720564926626) -- (1.8819590672050404,3.256262575073729);
\draw [shift={(3.489411764705882,3.9576470588235293)},line width=1.pt,color=qqwwtt] (165.96375653207352:0.9) arc (165.96375653207352:192.04142625550617:0.9);
\draw [shift={(3.489411764705882,3.9576470588235293)},line width=1.pt,color=qqwwtt] (165.96375653207352:0.77) arc (165.96375653207352:192.04142625550617:0.77);
\draw [shift={(-1.,3.)},line width=1.pt,color=ffqqqq] (12.041426255506192:0.6) arc (12.041426255506192:75.96375653207352:0.6);
\draw[line width=1.pt,color=ffqqqq] (-0.6223630997921233,3.364712724759346) -- (-0.5144668425898729,3.468916360404873);
\begin{scriptsize}
\draw [color=sqsqsq] (-1.,3.)-- ++(-2.5pt,0 pt) -- ++(5.0pt,0 pt) ++(-2.5pt,-2.5pt) -- ++(0 pt,5.0pt);
\draw[color=sqsqsq] (-1.26,2.83) node {$C$};
\draw [color=sqsqsq] (3.,2.)-- ++(-2.5pt,0 pt) -- ++(5.0pt,0 pt) ++(-2.5pt,-2.5pt) -- ++(0 pt,5.0pt);
\draw[color=sqsqsq] (2.72,1.75) node {$D$};
\draw [color=black] (-0.5105882352941176,4.95764705882353)-- ++(-2.5pt,0 pt) -- ++(5.0pt,0 pt) ++(-2.5pt,-2.5pt) -- ++(0 pt,5.0pt);
\draw[color=black] (-0.38,5.33) node {$B$};
\draw [color=black] (3.489411764705882,3.9576470588235293)-- ++(-2.5pt,0 pt) -- ++(5.0pt,0 pt) ++(-2.5pt,-2.5pt) -- ++(0 pt,5.0pt);
\draw[color=black] (3.62,4.33) node {$A$};
\draw [fill=uuuuuu] (1.2447058823529418,3.478823529411765) circle (2.0pt);
\draw[color=uuuuuu] (1.38,3.81) node {$O$};
\end{scriptsize}
\end{tikzpicture}

\end{minipage}
\begin{minipage}{0.5\linewidth}
\begin{itemize}
\item $ \ldots\ldots\ldots = \ldots\ldots\ldots$ \vspace{0.3cm}
\item $ \ldots\ldots\ldots = \ldots\ldots\ldots$ \vspace{0.3cm} 
\item $ \ldots\ldots\ldots = \ldots\ldots\ldots$ 
\end{itemize}


\end{minipage}
 
\ExoCu{Représenter.}

\begin{minipage}{0.5\linewidth}

\begin{itemize}
\item Colorie en bleu les angles aigus.
\item Colorie en rouge les angles droits.
\item Colorie en vert les angles obtus.
\end{itemize}

\end{minipage}
\begin{minipage}{0.5\linewidth}
\definecolor{uququq}{rgb}{0.25098039215686274,0.25098039215686274,0.25098039215686274}
\definecolor{sqsqsq}{rgb}{0.12549019607843137,0.12549019607843137,0.12549019607843137}
\begin{tikzpicture}[line cap=round,line join=round,>=triangle 45,x=0.8cm,y=0.8cm]
\clip(-2.52,-0.2) rectangle (5.62,5.44);
\draw [shift={(-2.,4.)},line width=1.pt,color=uququq] (0,0) -- (-63.43494882292201:0.6) arc (-63.43494882292201:45.:0.6) -- cycle;
\draw [shift={(1.,3.)},line width=1.pt,color=uququq] (0,0) -- (46.97493401088198:0.6) arc (46.97493401088198:135.:0.6) -- cycle;
\draw [shift={(4.,4.)},line width=1.pt,color=uququq] (0,0) -- (162.6459753637387:0.6) arc (162.6459753637387:315.:0.6) -- cycle;
\draw[line width=1.pt,color=uququq] (4.7,3.3) -- (4.4,3.) -- (4.7,2.7) -- (5.,3.) -- cycle; 
\draw [shift={(4.,2.)},line width=1.pt,color=uququq] (0,0) -- (45.:0.6) arc (45.:180.:0.6) -- cycle;
\draw[line width=1.pt,color=uququq] (0.18973665961010275,0.3794733192202054) -- (-0.18973665961010266,0.5692099788303082) -- (-0.3794733192202054,0.18973665961010278) -- (0.,0.) -- cycle; 
\draw [shift={(-1.,2.)},line width=1.pt,color=uququq] (0,0) -- (116.56505117707799:0.6) arc (116.56505117707799:225.:0.6) -- cycle;
\draw[line width=1.pt,color=uququq] (-1.3,4.7) -- (-1.,4.4) -- (-0.7,4.7) -- (-1.,5.) -- cycle; 
\draw [shift={(2.4,4.5)},line width=1.pt,color=uququq] (0,0) -- (-133.02506598911805:0.6) arc (-133.02506598911805:-17.354024636261336:0.6) -- cycle;
\draw [shift={(1.,2.)},line width=1.pt,color=uququq] (0,0) -- (-116.56505117707799:0.6) arc (-116.56505117707799:0.:0.6) -- cycle;
\draw [shift={(-2.,1.)},line width=1.pt,color=uququq] (0,0) -- (-26.56505117707799:0.6) arc (-26.56505117707799:45.:0.6) -- cycle;
\draw [line width=1.pt,color=sqsqsq] (-2.,4.)-- (-1.,2.);
\draw [line width=1.pt,color=sqsqsq] (-1.,2.)-- (-2.,1.);
\draw [line width=1.pt,color=sqsqsq] (-2.,1.)-- (0.,0.);
\draw [line width=1.pt,color=sqsqsq] (0.,0.)-- (1.,2.);
\draw [line width=1.pt,color=sqsqsq] (1.,2.)-- (4.,2.);
\draw [line width=1.pt,color=sqsqsq] (4.,2.)-- (5.,3.);
\draw [line width=1.pt,color=sqsqsq] (5.,3.)-- (4.,4.);
\draw [line width=1.pt,color=sqsqsq] (4.,4.)-- (2.4,4.5);
\draw [line width=1.pt,color=sqsqsq] (2.4,4.5)-- (1.,3.);
\draw [line width=1.pt,color=sqsqsq] (1.,3.)-- (-1.,5.);
\draw [line width=1.pt,color=sqsqsq] (-1.,5.)-- (-2.,4.);
\begin{scriptsize}
\draw [color=sqsqsq] (-2.,4.)-- ++(-1.0pt,0 pt) -- ++(2.0pt,0 pt) ++(-1.0pt,-1.0pt) -- ++(0 pt,2.0pt);
\draw[color=sqsqsq] (-2.24,4.27) node {$A$};
\draw [color=sqsqsq] (-1.,2.)-- ++(-1.0pt,0 pt) -- ++(2.0pt,0 pt) ++(-1.0pt,-1.0pt) -- ++(0 pt,2.0pt);
\draw[color=sqsqsq] (-0.86,2.25) node {$B$};
\draw [color=sqsqsq] (-2.,1.)-- ++(-1.0pt,0 pt) -- ++(2.0pt,0 pt) ++(-1.0pt,-1.0pt) -- ++(0 pt,2.0pt);
\draw[color=sqsqsq] (-2.3,1.15) node {$C$};
\draw [color=sqsqsq] (0.,0.)-- ++(-1.0pt,0 pt) -- ++(2.0pt,0 pt) ++(-1.0pt,-1.0pt) -- ++(0 pt,2.0pt);
\draw[color=sqsqsq] (0.28,0.01) node {$D$};
\draw [color=sqsqsq] (1.,2.)-- ++(-1.0pt,0 pt) -- ++(2.0pt,0 pt) ++(-1.0pt,-1.0pt) -- ++(0 pt,2.0pt);
\draw[color=sqsqsq] (1.14,2.25) node {$E$};
\draw [color=sqsqsq] (4.,2.)-- ++(-1.0pt,0 pt) -- ++(2.0pt,0 pt) ++(-1.0pt,-1.0pt) -- ++(0 pt,2.0pt);
\draw[color=sqsqsq] (4.14,1.89) node {$F$};
\draw [color=sqsqsq] (5.,3.)-- ++(-1.0pt,0 pt) -- ++(2.0pt,0 pt) ++(-1.0pt,-1.0pt) -- ++(0 pt,2.0pt);
\draw[color=sqsqsq] (5.14,3.25) node {$G$};
\draw [color=sqsqsq] (4.,4.)-- ++(-1.0pt,0 pt) -- ++(2.0pt,0 pt) ++(-1.0pt,-1.0pt) -- ++(0 pt,2.0pt);
\draw[color=sqsqsq] (4.14,4.25) node {$H$};
\draw [color=sqsqsq] (2.4,4.5)-- ++(-1.0pt,0 pt) -- ++(2.0pt,0 pt) ++(-1.0pt,-1.0pt) -- ++(0 pt,2.0pt);
\draw[color=sqsqsq] (2.54,4.75) node {$I$};
\draw [color=sqsqsq] (1.,3.)-- ++(-1.0pt,0 pt) -- ++(2.0pt,0 pt) ++(-1.0pt,-1.0pt) -- ++(0 pt,2.0pt);
\draw[color=sqsqsq] (0.98,2.85) node {$J$};
\draw [color=sqsqsq] (-1.,5.)-- ++(-1.0pt,0 pt) -- ++(2.0pt,0 pt) ++(-1.0pt,-1.0pt) -- ++(0 pt,2.0pt);
\draw[color=sqsqsq] (-0.86,5.25) node {$K$};
\end{scriptsize}
\end{tikzpicture}
\end{minipage}

\ExoCu{Représenter.}

Mesure à l'aide de ton rapporteur chaque angle.


\begin{minipage}{0.32\linewidth}
\definecolor{uququq}{rgb}{0.25098039215686274,0.25098039215686274,0.25098039215686274}
\begin{tikzpicture}[line cap=round,line join=round,>=triangle 45,x=0.5319148936170213cm,y=1.0cm]
\clip(-1.64,0.04) rectangle (5.88,3.54);
\draw [line width=1.pt,color=uququq] (5.34,1.48)-- (1.64,0.66);
\draw [line width=1.pt,color=uququq] (1.64,0.66)-- (-1.3,2.72);
\begin{scriptsize}
\draw [color=uququq] (-1.3,2.72)-- ++(-2.5pt,0 pt) -- ++(5.0pt,0 pt) ++(-2.5pt,-2.5pt) -- ++(0 pt,5.0pt);
\draw[color=uququq] (-1.16,3.09) node {$R$};
\draw [color=uququq] (5.34,1.48)-- ++(-2.5pt,0 pt) -- ++(5.0pt,0 pt) ++(-2.5pt,-2.5pt) -- ++(0 pt,5.0pt);
\draw[color=uququq] (5.48,1.85) node {$S$};
\draw [color=uququq] (1.64,0.66)-- ++(-2.5pt,0 pt) -- ++(5.0pt,0 pt) ++(-2.5pt,-2.5pt) -- ++(0 pt,5.0pt);
\draw[color=uququq] (1.82,0.47) node {$T$};
\end{scriptsize}
\end{tikzpicture}

$\widehat{RST} =\ldots\ldots$

\end{minipage}
\vrule
\begin{minipage}{0.32\linewidth}
\definecolor{uququq}{rgb}{0.25098039215686274,0.25098039215686274,0.25098039215686274}
\begin{tikzpicture}[line cap=round,line join=round,>=triangle 45,x=1.0cm,y=1.0cm]
\clip(0.36,2.8) rectangle (4.28,5.46);
\draw [line width=1.pt,color=uququq] (3.96,3.1)-- (2.42,4.94);
\draw [line width=1.pt,color=uququq] (2.42,4.94)-- (0.7,3.12);
\begin{scriptsize}
\draw [color=uququq] (0.7,3.12)-- ++(-2.5pt,0 pt) -- ++(5.0pt,0 pt) ++(-2.5pt,-2.5pt) -- ++(0 pt,5.0pt);
\draw[color=uququq] (0.56,3.45) node {$E$};
\draw [color=uququq] (3.96,3.1)-- ++(-2.5pt,0 pt) -- ++(5.0pt,0 pt) ++(-2.5pt,-2.5pt) -- ++(0 pt,5.0pt);
\draw[color=uququq] (4.1,3.47) node {$F$};
\draw [color=uququq] (2.42,4.94)-- ++(-2.5pt,0 pt) -- ++(5.0pt,0 pt) ++(-2.5pt,-2.5pt) -- ++(0 pt,5.0pt);
\draw[color=uququq] (2.8,5.21) node {$D$};
\end{scriptsize}
\end{tikzpicture}

$\widehat{EDF} =\ldots\ldots$
\end{minipage}\vrule
\begin{minipage}{0.32\linewidth}

\definecolor{uququq}{rgb}{0.25098039215686274,0.25098039215686274,0.25098039215686274}
\begin{tikzpicture}[line cap=round,line join=round,>=triangle 45,x=0.5917159763313609cm,y=0.5917159763313609cm]
\clip(0.64,0.28) rectangle (7.4,5.74);
\draw [line width=1.pt,color=uququq] (1.,5.)-- (7.,3.);
\draw [line width=1.pt,color=uququq] (7.,3.)-- (1.64,0.66);
\draw [line width=1.pt,color=uququq] (1.64,0.66)-- (1.,5.);
\begin{scriptsize}
\draw [color=uququq] (1.,5.)-- ++(-2.5pt,0 pt) -- ++(5.0pt,0 pt) ++(-2.5pt,-2.5pt) -- ++(0 pt,5.0pt);
\draw[color=uququq] (1.14,5.37) node {$A$};
\draw [color=uququq] (7.,3.)-- ++(-2.5pt,0 pt) -- ++(5.0pt,0 pt) ++(-2.5pt,-2.5pt) -- ++(0 pt,5.0pt);
\draw[color=uququq] (7.14,3.37) node {$B$};
\draw [color=uququq] (1.64,0.66)-- ++(-2.5pt,0 pt) -- ++(5.0pt,0 pt) ++(-2.5pt,-2.5pt) -- ++(0 pt,5.0pt);
\draw[color=uququq] (1.92,0.61) node {$C$};
\end{scriptsize}
\end{tikzpicture}

$\widehat{ABC} =\ldots\ldots$

\end{minipage}


\ExoCu{Représenter.}

Complète les phrases suivantes :
\begin{itemize}
\item La mesure d'un angle aigu est comprise entre $\ldots \ldots \ldots$ et  $\ldots \ldots \ldots$
\item La mesure d'un angle plat est égale à  $\ldots \ldots \ldots$
\item La mesure d'un angle obtus est comprise entre $\ldots \ldots \ldots$ et  $\ldots \ldots \ldots$
\end{itemize}

 
\end{pageParcoursu}


%%%%%%%%%%%%%%%%%%%%%%%%%%%%%%%%%%%%%%%%%%%%%%%%%%%%%%%%%%%%%%%%%%%
%%%%  Niveau 2
%%%%%%%%%%%%%%%%%%%%%%%%%%%%%%%%%%%%%%%%%%%%%%%%%%%%%%%%%%%%%%%%%%%
\begin{pageParcoursd} 


\ExoCd{Représenter. Calculer.} 

\begin{minipage}{0.5\linewidth}

Mesure les angles du triangle ci-contre :  
\begin{itemize}
\item $\widehat{RAT} = \ldots\ldots$
\item $\widehat{ART} = \ldots\ldots$
\item $\widehat{TAR} = \ldots\ldots$
\end{itemize}
Calcule $\widehat{TAR}+\widehat{ART}+\widehat{RTA}$ \vspace{0.4cm}

$ = \ldots\ldots\ldots+\ldots\ldots\ldots+\ldots\ldots\ldots=\ldots\ldots$
\end{minipage}
\begin{minipage}{0.5\linewidth}

\definecolor{uququq}{rgb}{0.25098039215686274,0.25098039215686274,0.25098039215686274}
\begin{tikzpicture}[line cap=round,line join=round,>=triangle 45,x=0.8cm,y=0.8cm]
\clip(0.64,0.28) rectangle (7.4,5.74);
\draw [line width=1.pt,color=uququq] (1.,5.)-- (7.,3.);
\draw [line width=1.pt,color=uququq] (7.,3.)-- (1.64,0.66);
\draw [line width=1.pt,color=uququq] (1.64,0.66)-- (1.,5.);
\begin{scriptsize}
\draw [color=uququq] (1.,5.)-- ++(-2.5pt,0 pt) -- ++(5.0pt,0 pt) ++(-2.5pt,-2.5pt) -- ++(0 pt,5.0pt);
\draw[color=uququq] (1.14,5.37) node {$A$};
\draw [color=uququq] (7.,3.)-- ++(-2.5pt,0 pt) -- ++(5.0pt,0 pt) ++(-2.5pt,-2.5pt) -- ++(0 pt,5.0pt);
\draw[color=uququq] (7.14,3.37) node {$B$};
\draw [color=uququq] (1.64,0.66)-- ++(-2.5pt,0 pt) -- ++(5.0pt,0 pt) ++(-2.5pt,-2.5pt) -- ++(0 pt,5.0pt);
\draw[color=uququq] (1.92,0.61) node {$C$};
\end{scriptsize}
\end{tikzpicture}
\end{minipage}

\begin{PpT}{Somme des angles d'un triangle}
La somme des angles d'un triangle mesure $180$°.
\end{PpT}
 
\ExoCd{Modéliser. Calculer.}

Construis le triangle $TOI$ tel que $OI=5$ cm, $\widehat{IOT}=45$° et $OT=4$ cm.

\vspace{4cm}
 
 
\ExoCd{Modéliser. Calculer.}

\begin{minipage}{0.5\linewidth}

\begin{enumerate}
\item Construis le triangle $RUE$ tel que
\begin{itemize}
\item $RU=6$ cm 
\item $RE=3$ cm
\item $EU=8$ cm
\end{itemize}
\item Code et mesure chacun des angles.
\end{enumerate}
\end{minipage}
\begin{minipage}{0.5\linewidth}
\vspace{3cm}
\end{minipage}


\vspace{5cm} 
 
 
 
 
\end{pageParcoursd}

%%%%%%%%%%%%%%%%%%%%%%%%%%%%%%%%%%%%%%%%%%%%%%%%%%%%%%%%%%%%%%%%%%%
%%%%  Niveau 3
%%%%%%%%%%%%%%%%%%%%%%%%%%%%%%%%%%%%%%%%%%%%%%%%%%%%%%%%%%%%%%%%%%%
\begin{pageParcourst}


\ExoCt{Modéliser. Calculer.}

Construis le triangle $ANC$ tel que $AN=6$ cm, $\widehat{NAC}=40$° et  $\widehat{CNA}=42$° 

\vspace{4cm}


\ExoCt{Représenter. Calculer.}

\begin{itemize}
\item Construis le triangle $ABC$ tel que $\widehat{BAC}=40$° et  $\widehat{CBA}=42$° 
\item Construis le triangle $DEF$ tel que $\widehat{NAC}=40$° et  $\widehat{CNA}=42$°

\definecolor{sqsqsq}{rgb}{0.12549019607843137,0.12549019607843137,0.12549019607843137}
\definecolor{uququq}{rgb}{0.25098039215686274,0.25098039215686274,0.25098039215686274}
\begin{tikzpicture}[line cap=round,line join=round,>=triangle 45,x=1.0cm,y=1.0cm]
\clip(-3.56,1.) rectangle (8.9,5.48);
\draw [line width=1.pt] (-3.,1.62)-- (0.5,1.48);
\draw [line width=1.pt] (2.66,1.78)-- (7.9,1.4);
\begin{scriptsize}
\draw [color=uququq] (-3.,1.62)-- ++(-2.5pt,0 pt) -- ++(5.0pt,0 pt) ++(-2.5pt,-2.5pt) -- ++(0 pt,5.0pt);
\draw[color=uququq] (-2.86,1.99) node {$A$};
\draw [color=uququq] (0.5,1.48)-- ++(-2.5pt,0 pt) -- ++(5.0pt,0 pt) ++(-2.5pt,-2.5pt) -- ++(0 pt,5.0pt);
\draw[color=uququq] (0.64,1.85) node {$B$};
\draw [color=sqsqsq] (2.66,1.78)-- ++(-2.5pt,0 pt) -- ++(5.0pt,0 pt) ++(-2.5pt,-2.5pt) -- ++(0 pt,5.0pt);
\draw[color=sqsqsq] (2.8,2.15) node {$E$};
\draw [color=sqsqsq] (7.9,1.4)-- ++(-2.5pt,0 pt) -- ++(5.0pt,0 pt) ++(-2.5pt,-2.5pt) -- ++(0 pt,5.0pt);
\draw[color=sqsqsq] (8.04,1.77) node {$D$};
\end{scriptsize}
\end{tikzpicture}

\item Les cotés des triangles sont-ils proportionnels ? Explique ton raisonnement par des calculs.\point{3}
\end{itemize}


\ExoCt{Représenter. Calculer.}

Reproduis cette figure en vraie grandeur.

\definecolor{uququq}{rgb}{0.25098039215686274,0.25098039215686274,0.25098039215686274}
\begin{tikzpicture}[line cap=round,line join=round,>=triangle 45,x=0.5cm,y=0.5cm]
\clip(-2.68,0.66) rectangle (6.86,5.3);
\draw [shift={(-2.,2.)},line width=1.pt,color=uququq,fill=uququq,fill opacity=0.10000000149011612] (0,0) -- (-7.34795600009932:0.6) arc (-7.34795600009932:52.38060420837576:0.6) -- cycle;
\draw [shift={(0.21090566587084397,4.868909974960695)},line width=1.pt,color=uququq,fill=uququq,fill opacity=0.10000000149011612] (0,0) -- (-127.61939579162427:0.6) arc (-127.61939579162427:-67.07651620857442:0.6) -- cycle;
\draw [shift={(1.6216727420951496,1.5329716415303092)},line width=1.pt,color=uququq,fill=uququq,fill opacity=0.10000000149011612] (0,0) -- (112.92348379142561:0.6) arc (112.92348379142561:172.65204399990068:0.6) -- cycle;
\draw[line width=1.pt,color=uququq,fill=uququq,fill opacity=0.10000000149011612] (3.9386748547101296,3.2723051368385816) -- (4.193381530098294,2.933005108084223) -- (4.532681558852652,3.1877117834723876) -- (4.277974883464488,3.5270118122267458) -- cycle; 
\draw [line width=1.pt] (-2.,2.)-- (6.22,0.94);
\draw [line width=1.pt] (0.21090566587084397,4.868909974960695)-- (-2.,2.);
\draw [line width=1.pt] (0.21090566587084397,4.868909974960695)-- (1.6216727420951496,1.5329716415303092);
\draw [line width=1.pt] (4.277974883464488,3.5270118122267458)-- (1.6216727420951496,1.5329716415303092);
\draw [line width=1.pt] (4.277974883464488,3.5270118122267458)-- (6.22,0.94);
\draw [line width=1.pt,dash pattern=on 2pt off 2pt] (0.21090566587084397,4.868909974960695)-- (4.277974883464488,3.5270118122267458);
\draw [line width=1.pt] (-2.,2.)-- (1.6216727420951496,1.5329716415303092);
\draw (-2.38,4.08) node[anchor=north west] {$6~cm$};
\draw (2.08,4.98) node[anchor=north west] {$7~cm$};
\draw [shift={(-2.,2.)},line width=1.pt,color=uququq] (-7.34795600009932:0.6) arc (-7.34795600009932:52.38060420837576:0.6);
\draw[line width=1.pt,color=uququq] (-1.515020505915956,2.201046985349159) -- (-1.376454936177658,2.2584889811632043);
\draw [shift={(0.21090566587084397,4.868909974960695)},line width=1.pt,color=uququq] (-127.61939579162427:0.6) arc (-127.61939579162427:-67.07651620857442:0.6);
\draw[line width=1.pt,color=uququq] (0.1437609117136424,4.34822140970391) -- (0.12457669624015637,4.199453248201971);
\draw [shift={(1.6216727420951496,1.5329716415303092)},line width=1.pt,color=uququq] (112.92348379142561:0.6) arc (112.92348379142561:172.65204399990068:0.6);
\draw[line width=1.pt,color=uququq] (1.2035623318140847,1.8504754758608568) -- (1.0841022145909234,1.9411908570981564);
\begin{scriptsize}
\draw [color=black] (-2.,2.)-- ++(-2.5pt,0 pt) -- ++(5.0pt,0 pt) ++(-2.5pt,-2.5pt) -- ++(0 pt,5.0pt);
\draw[color=black] (-2.4,2.05) node {$A$};
\draw [color=black] (6.22,0.94)-- ++(-2.5pt,0 pt) -- ++(5.0pt,0 pt) ++(-2.5pt,-2.5pt) -- ++(0 pt,5.0pt);
\draw[color=black] (6.36,1.31) node {$B$};
\draw [color=black] (1.6216727420951496,1.5329716415303092)-- ++(-2.5pt,0 pt) -- ++(5.0pt,0 pt) ++(-2.5pt,-2.5pt) -- ++(0 pt,5.0pt);
\draw[color=black] (1.56,1.29) node {$C$};
\draw [color=black] (0.21090566587084397,4.868909974960695)-- ++(-2.5pt,0 pt) -- ++(5.0pt,0 pt) ++(-2.5pt,-2.5pt) -- ++(0 pt,5.0pt);
\draw[color=black] (0.36,5.23) node {$E$};
\draw [color=black] (4.277974883464488,3.5270118122267458)-- ++(-2.5pt,0 pt) -- ++(5.0pt,0 pt) ++(-2.5pt,-2.5pt) -- ++(0 pt,5.0pt);
\draw[color=black] (4.42,3.89) node {$G$};
\end{scriptsize}
\end{tikzpicture}

 
\end{pageParcourst}




\begin{pageBrouillon} 
 
\ligne{32}



\end{pageBrouillon}



\begin{pageAuto} 

\ExoAuto

Mesure chaque angle dont le rapporteur est déjà correctement positionné.

A faire avec TikZ

\begin{minipage}{0.32\linewidth}

\vspace{5cm}

\end{minipage}
\vrule
\begin{minipage}{0.32\linewidth}

\vspace{5cm}

\end{minipage}
\vrule
\begin{minipage}{0.32\linewidth}

\vspace{5cm}

\end{minipage}


\ExoAuto


Mesure chaque angle avec ton rapporteur

\begin{minipage}{0.32\linewidth}

\definecolor{xdxdff}{rgb}{0.49019607843137253,0.49019607843137253,1.}
\definecolor{ududff}{rgb}{0.30196078431372547,0.30196078431372547,1.}
\definecolor{ffqqqq}{rgb}{1.,0.,0.}
\definecolor{wqwqwq}{rgb}{0.3764705882352941,0.3764705882352941,0.3764705882352941}
\definecolor{yqyqyq}{rgb}{0.5019607843137255,0.5019607843137255,0.5019607843137255}
\begin{tikzpicture}[line cap=round,line join=round,>=triangle 45,x=1.5cm,y=1.5cm]
\draw [line width=1.pt] (3.242472322182025,-0.7637583746856168)-- (4.,-2.);
\draw [line width=1.pt] (4.,-2.)-- (4.363419836540274,-0.6648512327614576);
\begin{scriptsize}
\draw [color=ududff] (3.242472322182025,-0.7637583746856168)-- ++(-2.5pt,0 pt) -- ++(5.0pt,0 pt) ++(-2.5pt,-2.5pt) -- ++(0 pt,5.0pt);
\draw[color=ududff] (3.3001681501269347,-0.6112765308858714) node {$I$};
\draw [color=ududff] (4.,-2.)-- ++(-2.5pt,0 pt) -- ++(5.0pt,0 pt) ++(-2.5pt,-2.5pt) -- ++(0 pt,5.0pt);
\draw[color=ududff] (4.099667480220686,-1.9877342559970865) node {$L$};
\draw [color=ududff] (4.363419836540274,-0.6648512327614576)-- ++(-2.5pt,0 pt) -- ++(5.0pt,0 pt) ++(-2.5pt,-2.5pt) -- ++(0 pt,5.0pt);
\draw[color=ududff] (4.421115664485184,-0.5123693889617122) node {$M$};
\end{scriptsize}
\end{tikzpicture}

$\widehat{ILM}=\ldots\ldots\ldots$

\end{minipage}
\vrule
\begin{minipage}{0.32\linewidth}

\definecolor{xdxdff}{rgb}{0.49019607843137253,0.49019607843137253,1.}
\definecolor{ududff}{rgb}{0.30196078431372547,0.30196078431372547,1.}
\definecolor{ffqqqq}{rgb}{1.,0.,0.}
\definecolor{wqwqwq}{rgb}{0.3764705882352941,0.3764705882352941,0.3764705882352941}
\definecolor{yqyqyq}{rgb}{0.5019607843137255,0.5019607843137255,0.5019607843137255}
\begin{tikzpicture}[line cap=round,line join=round,>=triangle 45,x=1.5cm,y=1.5cm]
\draw [line width=1.pt] (5.657454834733252,-0.895634563917829)-- (6.89379400498132,-1.4725928918087574);
\draw [line width=1.pt] (6.89379400498132,-1.4725928918087574)-- (7.957045691394659,-0.9286036112258821);
\begin{scriptsize}
\draw [color=ududff] (5.657454834733252,-0.895634563917829)-- ++(-2.5pt,0 pt) -- ++(5.0pt,0 pt) ++(-2.5pt,-2.5pt) -- ++(0 pt,5.0pt);
\draw[color=ududff] (5.715150662678162,-0.7431527201180836) node {$N$};
\draw [color=ududff] (6.89379400498132,-1.4725928918087574)-- ++(-2.5pt,0 pt) -- ++(5.0pt,0 pt) ++(-2.5pt,-2.5pt) -- ++(0 pt,5.0pt);
\draw[color=ududff] (6.869067221576358,-1.6085902119544764) node {$P$};
\draw [color=ududff] (7.957045691394659,-0.9286036112258821)-- ++(-2.5pt,0 pt) -- ++(5.0pt,0 pt) ++(-2.5pt,-2.5pt) -- ++(0 pt,5.0pt);
\draw[color=ududff] (8.014741519339568,-0.7761217674261367) node {$Q$};
\end{scriptsize}
\end{tikzpicture}

$\widehat{NPQ}=\ldots\ldots\ldots$

\end{minipage}
\vrule
\begin{minipage}{0.32\linewidth}

\definecolor{xdxdff}{rgb}{0.49019607843137253,0.49019607843137253,1.}
\definecolor{ududff}{rgb}{0.30196078431372547,0.30196078431372547,1.}
\definecolor{ffqqqq}{rgb}{1.,0.,0.}
\definecolor{wqwqwq}{rgb}{0.3764705882352941,0.3764705882352941,0.3764705882352941}
\definecolor{yqyqyq}{rgb}{0.5019607843137255,0.5019607843137255,0.5019607843137255}
\begin{tikzpicture}[line cap=round,line join=round,>=triangle 45,x=1.5cm,y=1.5cm]
\draw [line width=1.pt] (9.03678190007797,-0.8791500402638025)-- (9.753858618821852,-1.7775565794082482);
\draw [line width=1.pt] (10.82631119658979,-0.9215623919587671)-- (9.753858618821852,-1.7775565794082482);
\begin{scriptsize}
\draw [color=ududff] (9.03678190007797,-0.8791500402638025)-- ++(-2.5pt,0 pt) -- ++(5.0pt,0 pt) ++(-2.5pt,-2.5pt) -- ++(0 pt,5.0pt);
\draw[color=ududff] (8.970843810998074,-0.7266681964640571) node {$R$};
\draw [color=ududff] (9.753858618821852,-1.7775565794082482)-- ++(-2.5pt,0 pt) -- ++(5.0pt,0 pt) ++(-2.5pt,-2.5pt) -- ++(0 pt,5.0pt);
\draw[color=ududff] (9.819796707901748,-1.8805848522459139) node {$S$};
\draw [color=xdxdff] (10.82631119658979,-0.9215623919587671)-- ++(-2.5pt,0 pt) -- ++(5.0pt,0 pt) ++(-2.5pt,-2.5pt) -- ++(0 pt,5.0pt);
\draw[color=xdxdff] (10.883048394315086,-0.7678795055991234) node {$T$};
\end{scriptsize}
\end{tikzpicture}

$\widehat{RST}=\ldots\ldots\ldots$

\end{minipage}



\ExoAuto

Construis un angle $\widehat{AOB}$ de mesure $70$° et un angle $\widehat{COD}$ de mesure $150$°

\begin{tabularx}{\linewidth}{|c|}
\hline 

\vspace{5cm} \\ 
\hline 
\end{tabularx} 

 



\end{pageAuto}



