\textbf{Utiliser des nombres décimaux ayant au plus quatre décimales.}

\begin{enumerate}{0.5\linewidth}

\ExoCun{Représenter.}

Écrire les nombres suivants sous forme décimale :
\begin{enumerate}
\item quatre-cent-douze unités et six-dixièmes $= \cdots\cdots\cdots\cdots\cdots\cdots\cdots\cdots $
\item $\frac{6}{10}= \cdots\cdots\cdots\cdots\cdots\cdots\cdots\cdots$
\item $\frac{162}{100}= \cdots\cdots\cdots\cdots\cdots\cdots\cdots\cdots$
\item $\frac{5129}{100}= \cdots\cdots\cdots\cdots\cdots\cdots\cdots\cdots$
\end{enumerate}
 

\ExoCun{Représenter.}

\begin{enumerate}
\item Priscille a presque réussi à décomposer le nombre décimal  $324,67$. Elle a écrit : $324,67 = \cdots\cdots\cdots\  \times 100 + 2\times 10 + \cdots\cdots\cdots\ + 6\times \dfrac{1}{10} +  7 \times \dfrac{1}{100} $. Peux-tu l'aider ?

$324,67 = \cdots\cdots\cdots  \times 100 + \cdots\cdots\cdots \times 10 + \cdots\cdots\cdots  + \cdots\cdots\cdots \times \dfrac{1}{10} + \cdots\cdots\cdots \times \dfrac{1}{100} $

\item Mathilde a décomposé le nombre décimal  $A = 5\times 100 + 2\times 10 + 4 + 3\times \dfrac{1}{10} +  6\times \dfrac{1}{100} $. Peux-tu le retrouver ?
$A = \cdots\cdots\cdots\cdots\cdots\cdots\cdots\cdots $

\item Nour a décomposé le nombre décimal  $A = 6\times 1000 + 2\times 100 + 5\times 10 + 8 + 7\times \dfrac{1}{10} +   \dfrac{1}{100} $. Peux-tu le retrouver ?
$A = \cdots\cdots\cdots\cdots\cdots\cdots\cdots\cdots $
\end{enumerate}


\ExoCun{Représenter.}

Léon a décomposé un nombre décimal  $A = 5\times 100 + 2\times 10 + 4 + 3\times 0,1 +  6\times 0,01 $. Peux-tu le retrouver ?
$N = \cdots\cdots\cdots\cdots\cdots\cdots\cdots\cdots $



\end{enumerate}
\begin{enumerate}{0.5\linewidth}

\ExoCdeux{Représenter.}

Écrire les nombres suivants sous forme décimale :
\begin{enumerate}
\item quatre-mille-deux-cent-sept unités et six-dixièmes $= \cdots\cdots\cdots\cdots\cdots\cdots\cdots\cdots $
\item $\frac{8619}{1000}= \cdots\cdots\cdots\cdots\cdots\cdots\cdots\cdots$
\item $\frac{62}{100}= \cdots\cdots\cdots\cdots\cdots\cdots\cdots\cdots$
\item $\frac{652}{10}= \cdots\cdots\cdots\cdots\cdots\cdots\cdots\cdots$
\end{enumerate}
 

\ExoCdeux{Représenter.}

\begin{enumerate}
\item Mathilde a décomposé le nombre décimal  $A = 5\times 100 + 2\times 10 + 4 + 3\times \dfrac{1}{10} +  6\times \dfrac{1}{100} $. Peux-tu le retrouver ?
$A = \cdots\cdots\cdots\cdots\cdots\cdots\cdots\cdots $

\item Nour a décomposé le nombre décimal  $A = 6\times 1000 + 2\times \dfrac{1}{10} +  7\times 100 + 5\times 10 + 6\times 1000 + 8 +   \dfrac{1}{100} $. Peux-tu le retrouver ?
$A = \cdots\cdots\cdots\cdots\cdots\cdots\cdots\cdots $
\end{enumerate}


\ExoCdeux{Représenter.}

\begin{enumerate}
\item Léon a décomposé un nombre décimal  $A = 5\times 100 + 2\times 10 + 4 + 3\times 0,1 +  6\times 0,01 $. Peux-tu le retrouver ?
$N = \cdots\cdots\cdots\cdots\cdots\cdots\cdots\cdots $

\item Aziz a décomposé un nombre décimal  $A = 5\times 100 + 2\times 10 + 4 + 3\times 0,1 +  6\times 0,01 $. Peux-tu le retrouver ?
$N = \cdots\cdots\cdots\cdots\cdots\cdots\cdots\cdots $

\end{enumerate}

\ExoCdeux{Représenter.}

\begin{enumerate}
\item Léon a décomposé un nombre décimal  $A = 5\times 100 + 2\times 10 + 4 + 3\times 0,1 +  6\times 0,01 $. Peux-tu le retrouver ?
$N = \cdots\cdots\cdots\cdots\cdots\cdots\cdots\cdots $

\item Aziz a décomposé un nombre décimal  $A = 5\times 100 + 2\times 10 + 4 + 3\times 0,1 +  6\times 0,01 $. Peux-tu le retrouver ?
$N = \cdots\cdots\cdots\cdots\cdots\cdots\cdots\cdots $

\end{enumerate}